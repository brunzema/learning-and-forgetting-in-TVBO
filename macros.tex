
\newcommand*\submitdate{6. Dezember 2021}


%% ABBREVIATIONS & MACROS

\newcommand{\swname}[1]{\texttt{#1}}


%% GENERAL MATH STUFF

\newcommand*{\diff}{\mathrm{d}}
\newcommand*{\pdiff}[0]{\partial}
\newcommand*{\dd}[2][]{\frac{\pdiff #1}{\pdiff #2}}
\newcommand{\argmin}{\arg\min} 
\newcommand{\argmax}{\arg\max} 
\newcommand{\MatBold}[1]{\mathrm{\mathbf{#1}}}
\newcommand{\R}{\mathrm{I\!R}}
\DeclareMathOperator{\EX}{\mathbb{E}}
\renewcommand{\algorithmicrequire}{\textbf{Initialize:}} 
\renewcommand{\algorithmicensure}{\textbf{Output:}} 

%% CHAPTER MATERIALS AND METHODS

\newcommand*{\vel}{v}
\newcommand*{\loc}{x}
\newcommand*{\acc}{a}
\newcommand*{\mass}{m}
\newcommand*{\force}{f}

% convenience commands that make writing faster and also allow easily changing formatting later during the writing process
\newcommand*{\matlab}{\swname{Matlab}}
\newcommand*{\python}{\swname{Python}}
\newcommand*{\pgfplots}{\swname{pgfplots}}
\newcommand*{\gnuplot}{\swname{gnuplot}}

%% ALGORITHM STUFF

\newcommand{\algrule}[1][.2pt]{\par\vskip.5\baselineskip\hrule height #1\par\vskip.5\baselineskip}
\algnewcommand{\LineComment}[1]{\State \(\#\) \textit{#1}}


%% CHAPTER CONCEPT

\newcommand*{\Bfun}{B}
\newcommand*{\Afun}{A}


%% COLORS
\definecolor{bordeaux100}{rgb}{0.631372549019608,0.0627450980392157,0.2078431372549}
\definecolor{blau100}{rgb}{0,0.329411764705882,0.623529411764706}
\definecolor{blau25}{rgb}{0.7803921568627451, 0.8666666666666667, 0.9490196078431372}
\definecolor{orange100}{rgb}{0.9647058823529412, 0.6588235294117647, 0}



%%%%% Emacs-related stuff
%%% Local Variables: 
%%% mode: latex
%%% TeX-master: "main"
%%% End: 
