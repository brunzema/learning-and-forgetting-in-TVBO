\section{Bayesian Optimization}
\label{sec:bo}

Optimizing a black-box function $f\colon \mathcal{X} \mapsto \R$ as 
\begin{equation}
    \mathbf{x}^* = \argmin_{\mathbf{x} \in \mathcal{X}} f(\mathbf{x})
    \label{eq:bo_setup}
\end{equation}
is complex, especially if only noisy observations of the form $y = f(\mathbf{x}) + w$ with ${w \sim \mathcal{N}(0, \sigma_n^2)}$ are available to the optimization algorithm. This is also referred to as bandit feedback. If the function evaluations are cheap, gradients of $f$ can be approximated, and stochastic optimization methods can be applied to find local optima. 
However, if the function evaluations are expensive, such an approximation of the gradient is not practical. Furthermore, in some applications, it is desirable to find the global optimum. For this setting, \gls{bo} has been developed as a global optimization method in the case of expensive function evaluation and has been applied, e.g., for optimizing the hyperparameter in deep learning and other machine learning algorithms. 

\begin{figure}[t]
    \centering
    %% Creator: Matplotlib, PGF backend
%%
%% To include the figure in your LaTeX document, write
%%   \input{<filename>.pgf}
%%
%% Make sure the required packages are loaded in your preamble
%%   \usepackage{pgf}
%%
%% Figures using additional raster images can only be included by \input if
%% they are in the same directory as the main LaTeX file. For loading figures
%% from other directories you can use the `import` package
%%   \usepackage{import}
%%
%% and then include the figures with
%%   \import{<path to file>}{<filename>.pgf}
%%
%% Matplotlib used the following preamble
%%   \usepackage{fontspec}
%%
\begingroup%
\makeatletter%
\begin{pgfpicture}%
\pgfpathrectangle{\pgfpointorigin}{\pgfqpoint{5.507126in}{3.403591in}}%
\pgfusepath{use as bounding box, clip}%
\begin{pgfscope}%
\pgfsetbuttcap%
\pgfsetmiterjoin%
\definecolor{currentfill}{rgb}{1.000000,1.000000,1.000000}%
\pgfsetfillcolor{currentfill}%
\pgfsetlinewidth{0.000000pt}%
\definecolor{currentstroke}{rgb}{1.000000,1.000000,1.000000}%
\pgfsetstrokecolor{currentstroke}%
\pgfsetdash{}{0pt}%
\pgfpathmoveto{\pgfqpoint{0.000000in}{0.000000in}}%
\pgfpathlineto{\pgfqpoint{5.507126in}{0.000000in}}%
\pgfpathlineto{\pgfqpoint{5.507126in}{3.403591in}}%
\pgfpathlineto{\pgfqpoint{0.000000in}{3.403591in}}%
\pgfpathclose%
\pgfusepath{fill}%
\end{pgfscope}%
\begin{pgfscope}%
\pgfsetbuttcap%
\pgfsetmiterjoin%
\definecolor{currentfill}{rgb}{1.000000,1.000000,1.000000}%
\pgfsetfillcolor{currentfill}%
\pgfsetlinewidth{0.000000pt}%
\definecolor{currentstroke}{rgb}{0.000000,0.000000,0.000000}%
\pgfsetstrokecolor{currentstroke}%
\pgfsetstrokeopacity{0.000000}%
\pgfsetdash{}{0pt}%
\pgfpathmoveto{\pgfqpoint{0.220285in}{2.412707in}}%
\pgfpathlineto{\pgfqpoint{2.698492in}{2.412707in}}%
\pgfpathlineto{\pgfqpoint{2.698492in}{3.335519in}}%
\pgfpathlineto{\pgfqpoint{0.220285in}{3.335519in}}%
\pgfpathclose%
\pgfusepath{fill}%
\end{pgfscope}%
\begin{pgfscope}%
\pgfpathrectangle{\pgfqpoint{0.220285in}{2.412707in}}{\pgfqpoint{2.478207in}{0.922812in}}%
\pgfusepath{clip}%
\pgfsetbuttcap%
\pgfsetroundjoin%
\definecolor{currentfill}{rgb}{0.556863,0.729412,0.898039}%
\pgfsetfillcolor{currentfill}%
\pgfsetfillopacity{0.700000}%
\pgfsetlinewidth{0.000000pt}%
\definecolor{currentstroke}{rgb}{0.556863,0.729412,0.898039}%
\pgfsetstrokecolor{currentstroke}%
\pgfsetstrokeopacity{0.700000}%
\pgfsetdash{}{0pt}%
\pgfsys@defobject{currentmarker}{\pgfqpoint{0.220285in}{2.502617in}}{\pgfqpoint{2.698492in}{3.126836in}}{%
\pgfpathmoveto{\pgfqpoint{0.220285in}{3.126835in}}%
\pgfpathlineto{\pgfqpoint{0.220285in}{2.621390in}}%
\pgfpathlineto{\pgfqpoint{0.245317in}{2.621389in}}%
\pgfpathlineto{\pgfqpoint{0.270350in}{2.621386in}}%
\pgfpathlineto{\pgfqpoint{0.295382in}{2.621379in}}%
\pgfpathlineto{\pgfqpoint{0.320415in}{2.621363in}}%
\pgfpathlineto{\pgfqpoint{0.345447in}{2.621331in}}%
\pgfpathlineto{\pgfqpoint{0.370479in}{2.621265in}}%
\pgfpathlineto{\pgfqpoint{0.395512in}{2.621136in}}%
\pgfpathlineto{\pgfqpoint{0.420544in}{2.620895in}}%
\pgfpathlineto{\pgfqpoint{0.445577in}{2.620470in}}%
\pgfpathlineto{\pgfqpoint{0.470609in}{2.619758in}}%
\pgfpathlineto{\pgfqpoint{0.495641in}{2.618634in}}%
\pgfpathlineto{\pgfqpoint{0.520674in}{2.616982in}}%
\pgfpathlineto{\pgfqpoint{0.545706in}{2.614754in}}%
\pgfpathlineto{\pgfqpoint{0.570739in}{2.612073in}}%
\pgfpathlineto{\pgfqpoint{0.595771in}{2.609361in}}%
\pgfpathlineto{\pgfqpoint{0.620803in}{2.607453in}}%
\pgfpathlineto{\pgfqpoint{0.645836in}{2.607639in}}%
\pgfpathlineto{\pgfqpoint{0.670868in}{2.611559in}}%
\pgfpathlineto{\pgfqpoint{0.695901in}{2.620942in}}%
\pgfpathlineto{\pgfqpoint{0.720933in}{2.637198in}}%
\pgfpathlineto{\pgfqpoint{0.745965in}{2.660949in}}%
\pgfpathlineto{\pgfqpoint{0.770998in}{2.691582in}}%
\pgfpathlineto{\pgfqpoint{0.796030in}{2.726584in}}%
\pgfpathlineto{\pgfqpoint{0.821062in}{2.745101in}}%
\pgfpathlineto{\pgfqpoint{0.846095in}{2.694715in}}%
\pgfpathlineto{\pgfqpoint{0.871127in}{2.643106in}}%
\pgfpathlineto{\pgfqpoint{0.896160in}{2.599579in}}%
\pgfpathlineto{\pgfqpoint{0.921192in}{2.567150in}}%
\pgfpathlineto{\pgfqpoint{0.946224in}{2.547949in}}%
\pgfpathlineto{\pgfqpoint{0.971257in}{2.543349in}}%
\pgfpathlineto{\pgfqpoint{0.996289in}{2.553869in}}%
\pgfpathlineto{\pgfqpoint{1.021322in}{2.578937in}}%
\pgfpathlineto{\pgfqpoint{1.046354in}{2.616244in}}%
\pgfpathlineto{\pgfqpoint{1.071386in}{2.642598in}}%
\pgfpathlineto{\pgfqpoint{1.096419in}{2.606699in}}%
\pgfpathlineto{\pgfqpoint{1.121451in}{2.572595in}}%
\pgfpathlineto{\pgfqpoint{1.146484in}{2.546626in}}%
\pgfpathlineto{\pgfqpoint{1.171516in}{2.528646in}}%
\pgfpathlineto{\pgfqpoint{1.196548in}{2.517081in}}%
\pgfpathlineto{\pgfqpoint{1.221581in}{2.509863in}}%
\pgfpathlineto{\pgfqpoint{1.246613in}{2.505281in}}%
\pgfpathlineto{\pgfqpoint{1.271646in}{2.502648in}}%
\pgfpathlineto{\pgfqpoint{1.296678in}{2.502617in}}%
\pgfpathlineto{\pgfqpoint{1.321710in}{2.507040in}}%
\pgfpathlineto{\pgfqpoint{1.346743in}{2.518425in}}%
\pgfpathlineto{\pgfqpoint{1.371775in}{2.539139in}}%
\pgfpathlineto{\pgfqpoint{1.396807in}{2.570529in}}%
\pgfpathlineto{\pgfqpoint{1.421840in}{2.610775in}}%
\pgfpathlineto{\pgfqpoint{1.446872in}{2.607268in}}%
\pgfpathlineto{\pgfqpoint{1.471905in}{2.571060in}}%
\pgfpathlineto{\pgfqpoint{1.496937in}{2.545408in}}%
\pgfpathlineto{\pgfqpoint{1.521969in}{2.531860in}}%
\pgfpathlineto{\pgfqpoint{1.547002in}{2.529249in}}%
\pgfpathlineto{\pgfqpoint{1.572034in}{2.535158in}}%
\pgfpathlineto{\pgfqpoint{1.597067in}{2.546594in}}%
\pgfpathlineto{\pgfqpoint{1.622099in}{2.560595in}}%
\pgfpathlineto{\pgfqpoint{1.647131in}{2.574705in}}%
\pgfpathlineto{\pgfqpoint{1.672164in}{2.587257in}}%
\pgfpathlineto{\pgfqpoint{1.697196in}{2.597449in}}%
\pgfpathlineto{\pgfqpoint{1.722228in}{2.605281in}}%
\pgfpathlineto{\pgfqpoint{1.747261in}{2.611426in}}%
\pgfpathlineto{\pgfqpoint{1.772293in}{2.617103in}}%
\pgfpathlineto{\pgfqpoint{1.797326in}{2.623948in}}%
\pgfpathlineto{\pgfqpoint{1.822358in}{2.633864in}}%
\pgfpathlineto{\pgfqpoint{1.847390in}{2.648784in}}%
\pgfpathlineto{\pgfqpoint{1.872423in}{2.670357in}}%
\pgfpathlineto{\pgfqpoint{1.897455in}{2.699594in}}%
\pgfpathlineto{\pgfqpoint{1.922488in}{2.736535in}}%
\pgfpathlineto{\pgfqpoint{1.947520in}{2.779943in}}%
\pgfpathlineto{\pgfqpoint{1.972552in}{2.825137in}}%
\pgfpathlineto{\pgfqpoint{1.997585in}{2.811893in}}%
\pgfpathlineto{\pgfqpoint{2.022617in}{2.766047in}}%
\pgfpathlineto{\pgfqpoint{2.047650in}{2.724474in}}%
\pgfpathlineto{\pgfqpoint{2.072682in}{2.689986in}}%
\pgfpathlineto{\pgfqpoint{2.097714in}{2.663448in}}%
\pgfpathlineto{\pgfqpoint{2.122747in}{2.644599in}}%
\pgfpathlineto{\pgfqpoint{2.147779in}{2.632376in}}%
\pgfpathlineto{\pgfqpoint{2.172812in}{2.625272in}}%
\pgfpathlineto{\pgfqpoint{2.197844in}{2.621708in}}%
\pgfpathlineto{\pgfqpoint{2.222876in}{2.620305in}}%
\pgfpathlineto{\pgfqpoint{2.247909in}{2.620036in}}%
\pgfpathlineto{\pgfqpoint{2.272941in}{2.620246in}}%
\pgfpathlineto{\pgfqpoint{2.297973in}{2.620576in}}%
\pgfpathlineto{\pgfqpoint{2.323006in}{2.620870in}}%
\pgfpathlineto{\pgfqpoint{2.348038in}{2.621083in}}%
\pgfpathlineto{\pgfqpoint{2.373071in}{2.621220in}}%
\pgfpathlineto{\pgfqpoint{2.398103in}{2.621301in}}%
\pgfpathlineto{\pgfqpoint{2.423135in}{2.621346in}}%
\pgfpathlineto{\pgfqpoint{2.448168in}{2.621369in}}%
\pgfpathlineto{\pgfqpoint{2.473200in}{2.621381in}}%
\pgfpathlineto{\pgfqpoint{2.498233in}{2.621386in}}%
\pgfpathlineto{\pgfqpoint{2.523265in}{2.621389in}}%
\pgfpathlineto{\pgfqpoint{2.548297in}{2.621390in}}%
\pgfpathlineto{\pgfqpoint{2.573330in}{2.621390in}}%
\pgfpathlineto{\pgfqpoint{2.598362in}{2.621390in}}%
\pgfpathlineto{\pgfqpoint{2.623395in}{2.621390in}}%
\pgfpathlineto{\pgfqpoint{2.648427in}{2.621391in}}%
\pgfpathlineto{\pgfqpoint{2.673459in}{2.621391in}}%
\pgfpathlineto{\pgfqpoint{2.698492in}{2.621391in}}%
\pgfpathlineto{\pgfqpoint{2.698492in}{3.126836in}}%
\pgfpathlineto{\pgfqpoint{2.698492in}{3.126836in}}%
\pgfpathlineto{\pgfqpoint{2.673459in}{3.126836in}}%
\pgfpathlineto{\pgfqpoint{2.648427in}{3.126836in}}%
\pgfpathlineto{\pgfqpoint{2.623395in}{3.126836in}}%
\pgfpathlineto{\pgfqpoint{2.598362in}{3.126836in}}%
\pgfpathlineto{\pgfqpoint{2.573330in}{3.126835in}}%
\pgfpathlineto{\pgfqpoint{2.548297in}{3.126835in}}%
\pgfpathlineto{\pgfqpoint{2.523265in}{3.126834in}}%
\pgfpathlineto{\pgfqpoint{2.498233in}{3.126831in}}%
\pgfpathlineto{\pgfqpoint{2.473200in}{3.126826in}}%
\pgfpathlineto{\pgfqpoint{2.448168in}{3.126814in}}%
\pgfpathlineto{\pgfqpoint{2.423135in}{3.126790in}}%
\pgfpathlineto{\pgfqpoint{2.398103in}{3.126743in}}%
\pgfpathlineto{\pgfqpoint{2.373071in}{3.126654in}}%
\pgfpathlineto{\pgfqpoint{2.348038in}{3.126489in}}%
\pgfpathlineto{\pgfqpoint{2.323006in}{3.126190in}}%
\pgfpathlineto{\pgfqpoint{2.297973in}{3.125656in}}%
\pgfpathlineto{\pgfqpoint{2.272941in}{3.124704in}}%
\pgfpathlineto{\pgfqpoint{2.247909in}{3.123025in}}%
\pgfpathlineto{\pgfqpoint{2.222876in}{3.120106in}}%
\pgfpathlineto{\pgfqpoint{2.197844in}{3.115162in}}%
\pgfpathlineto{\pgfqpoint{2.172812in}{3.107113in}}%
\pgfpathlineto{\pgfqpoint{2.147779in}{3.094638in}}%
\pgfpathlineto{\pgfqpoint{2.122747in}{3.076358in}}%
\pgfpathlineto{\pgfqpoint{2.097714in}{3.051125in}}%
\pgfpathlineto{\pgfqpoint{2.072682in}{3.018368in}}%
\pgfpathlineto{\pgfqpoint{2.047650in}{2.978402in}}%
\pgfpathlineto{\pgfqpoint{2.022617in}{2.932657in}}%
\pgfpathlineto{\pgfqpoint{1.997585in}{2.884414in}}%
\pgfpathlineto{\pgfqpoint{1.972552in}{2.870821in}}%
\pgfpathlineto{\pgfqpoint{1.947520in}{2.917738in}}%
\pgfpathlineto{\pgfqpoint{1.922488in}{2.964702in}}%
\pgfpathlineto{\pgfqpoint{1.897455in}{3.006550in}}%
\pgfpathlineto{\pgfqpoint{1.872423in}{3.041388in}}%
\pgfpathlineto{\pgfqpoint{1.847390in}{3.068482in}}%
\pgfpathlineto{\pgfqpoint{1.822358in}{3.087997in}}%
\pgfpathlineto{\pgfqpoint{1.797326in}{3.100685in}}%
\pgfpathlineto{\pgfqpoint{1.772293in}{3.107507in}}%
\pgfpathlineto{\pgfqpoint{1.747261in}{3.109283in}}%
\pgfpathlineto{\pgfqpoint{1.722228in}{3.106442in}}%
\pgfpathlineto{\pgfqpoint{1.697196in}{3.098912in}}%
\pgfpathlineto{\pgfqpoint{1.672164in}{3.086133in}}%
\pgfpathlineto{\pgfqpoint{1.647131in}{3.067182in}}%
\pgfpathlineto{\pgfqpoint{1.622099in}{3.040975in}}%
\pgfpathlineto{\pgfqpoint{1.597067in}{3.006570in}}%
\pgfpathlineto{\pgfqpoint{1.572034in}{2.963511in}}%
\pgfpathlineto{\pgfqpoint{1.547002in}{2.912189in}}%
\pgfpathlineto{\pgfqpoint{1.521969in}{2.854095in}}%
\pgfpathlineto{\pgfqpoint{1.496937in}{2.791886in}}%
\pgfpathlineto{\pgfqpoint{1.471905in}{2.729265in}}%
\pgfpathlineto{\pgfqpoint{1.446872in}{2.671537in}}%
\pgfpathlineto{\pgfqpoint{1.421840in}{2.663804in}}%
\pgfpathlineto{\pgfqpoint{1.396807in}{2.716563in}}%
\pgfpathlineto{\pgfqpoint{1.371775in}{2.774196in}}%
\pgfpathlineto{\pgfqpoint{1.346743in}{2.829862in}}%
\pgfpathlineto{\pgfqpoint{1.321710in}{2.878705in}}%
\pgfpathlineto{\pgfqpoint{1.296678in}{2.916759in}}%
\pgfpathlineto{\pgfqpoint{1.271646in}{2.941129in}}%
\pgfpathlineto{\pgfqpoint{1.246613in}{2.950074in}}%
\pgfpathlineto{\pgfqpoint{1.221581in}{2.942998in}}%
\pgfpathlineto{\pgfqpoint{1.196548in}{2.920433in}}%
\pgfpathlineto{\pgfqpoint{1.171516in}{2.884047in}}%
\pgfpathlineto{\pgfqpoint{1.146484in}{2.836646in}}%
\pgfpathlineto{\pgfqpoint{1.121451in}{2.782099in}}%
\pgfpathlineto{\pgfqpoint{1.096419in}{2.725282in}}%
\pgfpathlineto{\pgfqpoint{1.071386in}{2.676257in}}%
\pgfpathlineto{\pgfqpoint{1.046354in}{2.700912in}}%
\pgfpathlineto{\pgfqpoint{1.021322in}{2.747755in}}%
\pgfpathlineto{\pgfqpoint{0.996289in}{2.791660in}}%
\pgfpathlineto{\pgfqpoint{0.971257in}{2.827286in}}%
\pgfpathlineto{\pgfqpoint{0.946224in}{2.850759in}}%
\pgfpathlineto{\pgfqpoint{0.921192in}{2.859786in}}%
\pgfpathlineto{\pgfqpoint{0.896160in}{2.853947in}}%
\pgfpathlineto{\pgfqpoint{0.871127in}{2.834762in}}%
\pgfpathlineto{\pgfqpoint{0.846095in}{2.805644in}}%
\pgfpathlineto{\pgfqpoint{0.821062in}{2.776874in}}%
\pgfpathlineto{\pgfqpoint{0.796030in}{2.817198in}}%
\pgfpathlineto{\pgfqpoint{0.770998in}{2.874915in}}%
\pgfpathlineto{\pgfqpoint{0.745965in}{2.929312in}}%
\pgfpathlineto{\pgfqpoint{0.720933in}{2.977431in}}%
\pgfpathlineto{\pgfqpoint{0.695901in}{3.017777in}}%
\pgfpathlineto{\pgfqpoint{0.670868in}{3.049925in}}%
\pgfpathlineto{\pgfqpoint{0.645836in}{3.074325in}}%
\pgfpathlineto{\pgfqpoint{0.620803in}{3.092030in}}%
\pgfpathlineto{\pgfqpoint{0.595771in}{3.104372in}}%
\pgfpathlineto{\pgfqpoint{0.570739in}{3.112687in}}%
\pgfpathlineto{\pgfqpoint{0.545706in}{3.118132in}}%
\pgfpathlineto{\pgfqpoint{0.520674in}{3.121611in}}%
\pgfpathlineto{\pgfqpoint{0.495641in}{3.123782in}}%
\pgfpathlineto{\pgfqpoint{0.470609in}{3.125104in}}%
\pgfpathlineto{\pgfqpoint{0.445577in}{3.125885in}}%
\pgfpathlineto{\pgfqpoint{0.420544in}{3.126332in}}%
\pgfpathlineto{\pgfqpoint{0.395512in}{3.126578in}}%
\pgfpathlineto{\pgfqpoint{0.370479in}{3.126709in}}%
\pgfpathlineto{\pgfqpoint{0.345447in}{3.126776in}}%
\pgfpathlineto{\pgfqpoint{0.320415in}{3.126809in}}%
\pgfpathlineto{\pgfqpoint{0.295382in}{3.126824in}}%
\pgfpathlineto{\pgfqpoint{0.270350in}{3.126831in}}%
\pgfpathlineto{\pgfqpoint{0.245317in}{3.126834in}}%
\pgfpathlineto{\pgfqpoint{0.220285in}{3.126835in}}%
\pgfpathclose%
\pgfusepath{fill}%
}%
\begin{pgfscope}%
\pgfsys@transformshift{0.000000in}{0.000000in}%
\pgfsys@useobject{currentmarker}{}%
\end{pgfscope}%
\end{pgfscope}%
\begin{pgfscope}%
\definecolor{textcolor}{rgb}{0.000000,0.000000,0.000000}%
\pgfsetstrokecolor{textcolor}%
\pgfsetfillcolor{textcolor}%
\pgftext[x=0.164729in,y=2.874113in,,bottom,rotate=90.000000]{\color{textcolor}\rmfamily\fontsize{10.000000}{12.000000}\selectfont \(\displaystyle f(x)\)}%
\end{pgfscope}%
\begin{pgfscope}%
\pgfpathrectangle{\pgfqpoint{0.220285in}{2.412707in}}{\pgfqpoint{2.478207in}{0.922812in}}%
\pgfusepath{clip}%
\pgfsetbuttcap%
\pgfsetroundjoin%
\pgfsetlinewidth{0.752812pt}%
\definecolor{currentstroke}{rgb}{0.000000,0.000000,0.000000}%
\pgfsetstrokecolor{currentstroke}%
\pgfsetdash{{2.775000pt}{1.200000pt}}{0.000000pt}%
\pgfpathmoveto{\pgfqpoint{0.633319in}{2.412707in}}%
\pgfpathlineto{\pgfqpoint{0.633319in}{3.335519in}}%
\pgfusepath{stroke}%
\end{pgfscope}%
\begin{pgfscope}%
\pgfpathrectangle{\pgfqpoint{0.220285in}{2.412707in}}{\pgfqpoint{2.478207in}{0.922812in}}%
\pgfusepath{clip}%
\pgfsetbuttcap%
\pgfsetroundjoin%
\pgfsetlinewidth{0.752812pt}%
\definecolor{currentstroke}{rgb}{0.000000,0.000000,0.000000}%
\pgfsetstrokecolor{currentstroke}%
\pgfsetdash{{2.775000pt}{1.200000pt}}{0.000000pt}%
\pgfpathmoveto{\pgfqpoint{2.285457in}{2.412707in}}%
\pgfpathlineto{\pgfqpoint{2.285457in}{3.335519in}}%
\pgfusepath{stroke}%
\end{pgfscope}%
\begin{pgfscope}%
\pgfpathrectangle{\pgfqpoint{0.220285in}{2.412707in}}{\pgfqpoint{2.478207in}{0.922812in}}%
\pgfusepath{clip}%
\pgfsetbuttcap%
\pgfsetroundjoin%
\pgfsetlinewidth{1.505625pt}%
\definecolor{currentstroke}{rgb}{0.631373,0.062745,0.207843}%
\pgfsetstrokecolor{currentstroke}%
\pgfsetdash{{1.500000pt}{2.475000pt}}{0.000000pt}%
\pgfpathmoveto{\pgfqpoint{1.353220in}{2.412707in}}%
\pgfpathlineto{\pgfqpoint{1.353220in}{3.335519in}}%
\pgfusepath{stroke}%
\end{pgfscope}%
\begin{pgfscope}%
\pgfpathrectangle{\pgfqpoint{0.220285in}{2.412707in}}{\pgfqpoint{2.478207in}{0.922812in}}%
\pgfusepath{clip}%
\pgfsetbuttcap%
\pgfsetroundjoin%
\definecolor{currentfill}{rgb}{0.000000,0.000000,0.000000}%
\pgfsetfillcolor{currentfill}%
\pgfsetlinewidth{1.003750pt}%
\definecolor{currentstroke}{rgb}{0.000000,0.000000,0.000000}%
\pgfsetstrokecolor{currentstroke}%
\pgfsetdash{}{0pt}%
\pgfsys@defobject{currentmarker}{\pgfqpoint{-0.020833in}{-0.020833in}}{\pgfqpoint{0.020833in}{0.020833in}}{%
\pgfpathmoveto{\pgfqpoint{0.000000in}{-0.020833in}}%
\pgfpathcurveto{\pgfqpoint{0.005525in}{-0.020833in}}{\pgfqpoint{0.010825in}{-0.018638in}}{\pgfqpoint{0.014731in}{-0.014731in}}%
\pgfpathcurveto{\pgfqpoint{0.018638in}{-0.010825in}}{\pgfqpoint{0.020833in}{-0.005525in}}{\pgfqpoint{0.020833in}{0.000000in}}%
\pgfpathcurveto{\pgfqpoint{0.020833in}{0.005525in}}{\pgfqpoint{0.018638in}{0.010825in}}{\pgfqpoint{0.014731in}{0.014731in}}%
\pgfpathcurveto{\pgfqpoint{0.010825in}{0.018638in}}{\pgfqpoint{0.005525in}{0.020833in}}{\pgfqpoint{0.000000in}{0.020833in}}%
\pgfpathcurveto{\pgfqpoint{-0.005525in}{0.020833in}}{\pgfqpoint{-0.010825in}{0.018638in}}{\pgfqpoint{-0.014731in}{0.014731in}}%
\pgfpathcurveto{\pgfqpoint{-0.018638in}{0.010825in}}{\pgfqpoint{-0.020833in}{0.005525in}}{\pgfqpoint{-0.020833in}{0.000000in}}%
\pgfpathcurveto{\pgfqpoint{-0.020833in}{-0.005525in}}{\pgfqpoint{-0.018638in}{-0.010825in}}{\pgfqpoint{-0.014731in}{-0.014731in}}%
\pgfpathcurveto{\pgfqpoint{-0.010825in}{-0.018638in}}{\pgfqpoint{-0.005525in}{-0.020833in}}{\pgfqpoint{0.000000in}{-0.020833in}}%
\pgfpathclose%
\pgfusepath{stroke,fill}%
}%
\begin{pgfscope}%
\pgfsys@transformshift{0.817804in}{2.762108in}%
\pgfsys@useobject{currentmarker}{}%
\end{pgfscope}%
\begin{pgfscope}%
\pgfsys@transformshift{1.981205in}{2.847832in}%
\pgfsys@useobject{currentmarker}{}%
\end{pgfscope}%
\begin{pgfscope}%
\pgfsys@transformshift{1.067108in}{2.658198in}%
\pgfsys@useobject{currentmarker}{}%
\end{pgfscope}%
\begin{pgfscope}%
\pgfsys@transformshift{1.432752in}{2.636355in}%
\pgfsys@useobject{currentmarker}{}%
\end{pgfscope}%
\end{pgfscope}%
\begin{pgfscope}%
\pgfpathrectangle{\pgfqpoint{0.220285in}{2.412707in}}{\pgfqpoint{2.478207in}{0.922812in}}%
\pgfusepath{clip}%
\pgfsetrectcap%
\pgfsetroundjoin%
\pgfsetlinewidth{0.752812pt}%
\definecolor{currentstroke}{rgb}{0.000000,0.329412,0.623529}%
\pgfsetstrokecolor{currentstroke}%
\pgfsetdash{}{0pt}%
\pgfpathmoveto{\pgfqpoint{0.220285in}{2.874112in}}%
\pgfpathlineto{\pgfqpoint{0.245317in}{2.874111in}}%
\pgfpathlineto{\pgfqpoint{0.270350in}{2.874108in}}%
\pgfpathlineto{\pgfqpoint{0.295382in}{2.874101in}}%
\pgfpathlineto{\pgfqpoint{0.320415in}{2.874086in}}%
\pgfpathlineto{\pgfqpoint{0.345447in}{2.874054in}}%
\pgfpathlineto{\pgfqpoint{0.370479in}{2.873987in}}%
\pgfpathlineto{\pgfqpoint{0.395512in}{2.873857in}}%
\pgfpathlineto{\pgfqpoint{0.420544in}{2.873614in}}%
\pgfpathlineto{\pgfqpoint{0.445577in}{2.873178in}}%
\pgfpathlineto{\pgfqpoint{0.470609in}{2.872431in}}%
\pgfpathlineto{\pgfqpoint{0.495641in}{2.871208in}}%
\pgfpathlineto{\pgfqpoint{0.520674in}{2.869297in}}%
\pgfpathlineto{\pgfqpoint{0.545706in}{2.866443in}}%
\pgfpathlineto{\pgfqpoint{0.570739in}{2.862380in}}%
\pgfpathlineto{\pgfqpoint{0.595771in}{2.856867in}}%
\pgfpathlineto{\pgfqpoint{0.620803in}{2.849742in}}%
\pgfpathlineto{\pgfqpoint{0.645836in}{2.840982in}}%
\pgfpathlineto{\pgfqpoint{0.670868in}{2.830742in}}%
\pgfpathlineto{\pgfqpoint{0.695901in}{2.819360in}}%
\pgfpathlineto{\pgfqpoint{0.720933in}{2.807314in}}%
\pgfpathlineto{\pgfqpoint{0.745965in}{2.795131in}}%
\pgfpathlineto{\pgfqpoint{0.770998in}{2.783248in}}%
\pgfpathlineto{\pgfqpoint{0.796030in}{2.771891in}}%
\pgfpathlineto{\pgfqpoint{0.821062in}{2.760988in}}%
\pgfpathlineto{\pgfqpoint{0.846095in}{2.750179in}}%
\pgfpathlineto{\pgfqpoint{0.871127in}{2.738934in}}%
\pgfpathlineto{\pgfqpoint{0.896160in}{2.726763in}}%
\pgfpathlineto{\pgfqpoint{0.921192in}{2.713468in}}%
\pgfpathlineto{\pgfqpoint{0.946224in}{2.699354in}}%
\pgfpathlineto{\pgfqpoint{0.971257in}{2.685318in}}%
\pgfpathlineto{\pgfqpoint{0.996289in}{2.672764in}}%
\pgfpathlineto{\pgfqpoint{1.021322in}{2.663346in}}%
\pgfpathlineto{\pgfqpoint{1.046354in}{2.658578in}}%
\pgfpathlineto{\pgfqpoint{1.071386in}{2.659427in}}%
\pgfpathlineto{\pgfqpoint{1.096419in}{2.665990in}}%
\pgfpathlineto{\pgfqpoint{1.121451in}{2.677347in}}%
\pgfpathlineto{\pgfqpoint{1.146484in}{2.691636in}}%
\pgfpathlineto{\pgfqpoint{1.171516in}{2.706347in}}%
\pgfpathlineto{\pgfqpoint{1.196548in}{2.718757in}}%
\pgfpathlineto{\pgfqpoint{1.221581in}{2.726430in}}%
\pgfpathlineto{\pgfqpoint{1.246613in}{2.727677in}}%
\pgfpathlineto{\pgfqpoint{1.271646in}{2.721888in}}%
\pgfpathlineto{\pgfqpoint{1.296678in}{2.709688in}}%
\pgfpathlineto{\pgfqpoint{1.321710in}{2.692873in}}%
\pgfpathlineto{\pgfqpoint{1.346743in}{2.674143in}}%
\pgfpathlineto{\pgfqpoint{1.371775in}{2.656667in}}%
\pgfpathlineto{\pgfqpoint{1.396807in}{2.643546in}}%
\pgfpathlineto{\pgfqpoint{1.421840in}{2.637290in}}%
\pgfpathlineto{\pgfqpoint{1.446872in}{2.639402in}}%
\pgfpathlineto{\pgfqpoint{1.471905in}{2.650162in}}%
\pgfpathlineto{\pgfqpoint{1.496937in}{2.668647in}}%
\pgfpathlineto{\pgfqpoint{1.521969in}{2.692977in}}%
\pgfpathlineto{\pgfqpoint{1.547002in}{2.720719in}}%
\pgfpathlineto{\pgfqpoint{1.572034in}{2.749335in}}%
\pgfpathlineto{\pgfqpoint{1.597067in}{2.776582in}}%
\pgfpathlineto{\pgfqpoint{1.622099in}{2.800785in}}%
\pgfpathlineto{\pgfqpoint{1.647131in}{2.820943in}}%
\pgfpathlineto{\pgfqpoint{1.672164in}{2.836695in}}%
\pgfpathlineto{\pgfqpoint{1.697196in}{2.848180in}}%
\pgfpathlineto{\pgfqpoint{1.722228in}{2.855861in}}%
\pgfpathlineto{\pgfqpoint{1.747261in}{2.860354in}}%
\pgfpathlineto{\pgfqpoint{1.772293in}{2.862305in}}%
\pgfpathlineto{\pgfqpoint{1.797326in}{2.862316in}}%
\pgfpathlineto{\pgfqpoint{1.822358in}{2.860930in}}%
\pgfpathlineto{\pgfqpoint{1.847390in}{2.858633in}}%
\pgfpathlineto{\pgfqpoint{1.872423in}{2.855873in}}%
\pgfpathlineto{\pgfqpoint{1.897455in}{2.853072in}}%
\pgfpathlineto{\pgfqpoint{1.922488in}{2.850618in}}%
\pgfpathlineto{\pgfqpoint{1.947520in}{2.848841in}}%
\pgfpathlineto{\pgfqpoint{1.972552in}{2.847979in}}%
\pgfpathlineto{\pgfqpoint{1.997585in}{2.848153in}}%
\pgfpathlineto{\pgfqpoint{2.022617in}{2.849352in}}%
\pgfpathlineto{\pgfqpoint{2.047650in}{2.851438in}}%
\pgfpathlineto{\pgfqpoint{2.072682in}{2.854177in}}%
\pgfpathlineto{\pgfqpoint{2.097714in}{2.857287in}}%
\pgfpathlineto{\pgfqpoint{2.122747in}{2.860479in}}%
\pgfpathlineto{\pgfqpoint{2.147779in}{2.863507in}}%
\pgfpathlineto{\pgfqpoint{2.172812in}{2.866193in}}%
\pgfpathlineto{\pgfqpoint{2.197844in}{2.868435in}}%
\pgfpathlineto{\pgfqpoint{2.222876in}{2.870205in}}%
\pgfpathlineto{\pgfqpoint{2.247909in}{2.871531in}}%
\pgfpathlineto{\pgfqpoint{2.272941in}{2.872475in}}%
\pgfpathlineto{\pgfqpoint{2.297973in}{2.873116in}}%
\pgfpathlineto{\pgfqpoint{2.323006in}{2.873530in}}%
\pgfpathlineto{\pgfqpoint{2.348038in}{2.873786in}}%
\pgfpathlineto{\pgfqpoint{2.373071in}{2.873937in}}%
\pgfpathlineto{\pgfqpoint{2.398103in}{2.874022in}}%
\pgfpathlineto{\pgfqpoint{2.423135in}{2.874068in}}%
\pgfpathlineto{\pgfqpoint{2.448168in}{2.874092in}}%
\pgfpathlineto{\pgfqpoint{2.473200in}{2.874103in}}%
\pgfpathlineto{\pgfqpoint{2.498233in}{2.874109in}}%
\pgfpathlineto{\pgfqpoint{2.523265in}{2.874111in}}%
\pgfpathlineto{\pgfqpoint{2.548297in}{2.874112in}}%
\pgfpathlineto{\pgfqpoint{2.573330in}{2.874113in}}%
\pgfpathlineto{\pgfqpoint{2.598362in}{2.874113in}}%
\pgfpathlineto{\pgfqpoint{2.623395in}{2.874113in}}%
\pgfpathlineto{\pgfqpoint{2.648427in}{2.874113in}}%
\pgfpathlineto{\pgfqpoint{2.673459in}{2.874113in}}%
\pgfpathlineto{\pgfqpoint{2.698492in}{2.874113in}}%
\pgfusepath{stroke}%
\end{pgfscope}%
\begin{pgfscope}%
\pgfpathrectangle{\pgfqpoint{0.220285in}{2.412707in}}{\pgfqpoint{2.478207in}{0.922812in}}%
\pgfusepath{clip}%
\pgfsetrectcap%
\pgfsetroundjoin%
\pgfsetlinewidth{0.752812pt}%
\definecolor{currentstroke}{rgb}{0.000000,0.000000,0.000000}%
\pgfsetstrokecolor{currentstroke}%
\pgfsetdash{}{0pt}%
\pgfpathmoveto{\pgfqpoint{0.403210in}{3.345519in}}%
\pgfpathlineto{\pgfqpoint{0.420544in}{3.276526in}}%
\pgfpathlineto{\pgfqpoint{0.445577in}{3.209803in}}%
\pgfpathlineto{\pgfqpoint{0.470609in}{3.172696in}}%
\pgfpathlineto{\pgfqpoint{0.495641in}{3.159889in}}%
\pgfpathlineto{\pgfqpoint{0.520674in}{3.164603in}}%
\pgfpathlineto{\pgfqpoint{0.545706in}{3.179272in}}%
\pgfpathlineto{\pgfqpoint{0.570739in}{3.196267in}}%
\pgfpathlineto{\pgfqpoint{0.595771in}{3.208582in}}%
\pgfpathlineto{\pgfqpoint{0.620803in}{3.210447in}}%
\pgfpathlineto{\pgfqpoint{0.645836in}{3.197783in}}%
\pgfpathlineto{\pgfqpoint{0.670868in}{3.168475in}}%
\pgfpathlineto{\pgfqpoint{0.695901in}{3.122450in}}%
\pgfpathlineto{\pgfqpoint{0.720933in}{3.061549in}}%
\pgfpathlineto{\pgfqpoint{0.745965in}{2.989222in}}%
\pgfpathlineto{\pgfqpoint{0.770998in}{2.910089in}}%
\pgfpathlineto{\pgfqpoint{0.796030in}{2.829404in}}%
\pgfpathlineto{\pgfqpoint{0.821062in}{2.752489in}}%
\pgfpathlineto{\pgfqpoint{0.846095in}{2.684195in}}%
\pgfpathlineto{\pgfqpoint{0.871127in}{2.628439in}}%
\pgfpathlineto{\pgfqpoint{0.896160in}{2.587863in}}%
\pgfpathlineto{\pgfqpoint{0.921192in}{2.563637in}}%
\pgfpathlineto{\pgfqpoint{0.946224in}{2.555427in}}%
\pgfpathlineto{\pgfqpoint{0.971257in}{2.561518in}}%
\pgfpathlineto{\pgfqpoint{0.996289in}{2.579069in}}%
\pgfpathlineto{\pgfqpoint{1.021322in}{2.604471in}}%
\pgfpathlineto{\pgfqpoint{1.046354in}{2.633759in}}%
\pgfpathlineto{\pgfqpoint{1.071386in}{2.663046in}}%
\pgfpathlineto{\pgfqpoint{1.096419in}{2.688907in}}%
\pgfpathlineto{\pgfqpoint{1.121451in}{2.708699in}}%
\pgfpathlineto{\pgfqpoint{1.146484in}{2.720773in}}%
\pgfpathlineto{\pgfqpoint{1.171516in}{2.724560in}}%
\pgfpathlineto{\pgfqpoint{1.196548in}{2.720539in}}%
\pgfpathlineto{\pgfqpoint{1.221581in}{2.710079in}}%
\pgfpathlineto{\pgfqpoint{1.246613in}{2.695197in}}%
\pgfpathlineto{\pgfqpoint{1.271646in}{2.678250in}}%
\pgfpathlineto{\pgfqpoint{1.296678in}{2.661615in}}%
\pgfpathlineto{\pgfqpoint{1.321710in}{2.647375in}}%
\pgfpathlineto{\pgfqpoint{1.346743in}{2.637067in}}%
\pgfpathlineto{\pgfqpoint{1.371775in}{2.631508in}}%
\pgfpathlineto{\pgfqpoint{1.396807in}{2.630725in}}%
\pgfpathlineto{\pgfqpoint{1.421840in}{2.633990in}}%
\pgfpathlineto{\pgfqpoint{1.446872in}{2.639955in}}%
\pgfpathlineto{\pgfqpoint{1.471905in}{2.646865in}}%
\pgfpathlineto{\pgfqpoint{1.496937in}{2.652833in}}%
\pgfpathlineto{\pgfqpoint{1.521969in}{2.656120in}}%
\pgfpathlineto{\pgfqpoint{1.547002in}{2.655406in}}%
\pgfpathlineto{\pgfqpoint{1.572034in}{2.650008in}}%
\pgfpathlineto{\pgfqpoint{1.597067in}{2.640014in}}%
\pgfpathlineto{\pgfqpoint{1.622099in}{2.626316in}}%
\pgfpathlineto{\pgfqpoint{1.647131in}{2.610540in}}%
\pgfpathlineto{\pgfqpoint{1.672164in}{2.594873in}}%
\pgfpathlineto{\pgfqpoint{1.697196in}{2.581811in}}%
\pgfpathlineto{\pgfqpoint{1.722228in}{2.573847in}}%
\pgfpathlineto{\pgfqpoint{1.747261in}{2.573147in}}%
\pgfpathlineto{\pgfqpoint{1.772293in}{2.581245in}}%
\pgfpathlineto{\pgfqpoint{1.797326in}{2.598797in}}%
\pgfpathlineto{\pgfqpoint{1.822358in}{2.625431in}}%
\pgfpathlineto{\pgfqpoint{1.847390in}{2.659705in}}%
\pgfpathlineto{\pgfqpoint{1.872423in}{2.699201in}}%
\pgfpathlineto{\pgfqpoint{1.897455in}{2.740734in}}%
\pgfpathlineto{\pgfqpoint{1.922488in}{2.780669in}}%
\pgfpathlineto{\pgfqpoint{1.947520in}{2.815311in}}%
\pgfpathlineto{\pgfqpoint{1.972552in}{2.841335in}}%
\pgfpathlineto{\pgfqpoint{1.997585in}{2.856198in}}%
\pgfpathlineto{\pgfqpoint{2.022617in}{2.858500in}}%
\pgfpathlineto{\pgfqpoint{2.047650in}{2.848235in}}%
\pgfpathlineto{\pgfqpoint{2.072682in}{2.826916in}}%
\pgfpathlineto{\pgfqpoint{2.097714in}{2.797545in}}%
\pgfpathlineto{\pgfqpoint{2.122747in}{2.764411in}}%
\pgfpathlineto{\pgfqpoint{2.147779in}{2.732754in}}%
\pgfpathlineto{\pgfqpoint{2.172812in}{2.708300in}}%
\pgfpathlineto{\pgfqpoint{2.197844in}{2.696721in}}%
\pgfpathlineto{\pgfqpoint{2.222876in}{2.703070in}}%
\pgfpathlineto{\pgfqpoint{2.247909in}{2.731245in}}%
\pgfpathlineto{\pgfqpoint{2.272941in}{2.783551in}}%
\pgfpathlineto{\pgfqpoint{2.297973in}{2.860392in}}%
\pgfpathlineto{\pgfqpoint{2.323006in}{2.960148in}}%
\pgfpathlineto{\pgfqpoint{2.348038in}{3.079245in}}%
\pgfpathlineto{\pgfqpoint{2.373071in}{3.212437in}}%
\pgfpathlineto{\pgfqpoint{2.396727in}{3.345519in}}%
\pgfusepath{stroke}%
\end{pgfscope}%
\begin{pgfscope}%
\pgfpathrectangle{\pgfqpoint{0.220285in}{2.412707in}}{\pgfqpoint{2.478207in}{0.922812in}}%
\pgfusepath{clip}%
\pgfsetbuttcap%
\pgfsetroundjoin%
\definecolor{currentfill}{rgb}{0.631373,0.062745,0.207843}%
\pgfsetfillcolor{currentfill}%
\pgfsetlinewidth{1.003750pt}%
\definecolor{currentstroke}{rgb}{0.631373,0.062745,0.207843}%
\pgfsetstrokecolor{currentstroke}%
\pgfsetdash{}{0pt}%
\pgfsys@defobject{currentmarker}{\pgfqpoint{-0.027778in}{-0.027778in}}{\pgfqpoint{0.027778in}{0.027778in}}{%
\pgfpathmoveto{\pgfqpoint{0.000000in}{-0.027778in}}%
\pgfpathcurveto{\pgfqpoint{0.007367in}{-0.027778in}}{\pgfqpoint{0.014433in}{-0.024851in}}{\pgfqpoint{0.019642in}{-0.019642in}}%
\pgfpathcurveto{\pgfqpoint{0.024851in}{-0.014433in}}{\pgfqpoint{0.027778in}{-0.007367in}}{\pgfqpoint{0.027778in}{0.000000in}}%
\pgfpathcurveto{\pgfqpoint{0.027778in}{0.007367in}}{\pgfqpoint{0.024851in}{0.014433in}}{\pgfqpoint{0.019642in}{0.019642in}}%
\pgfpathcurveto{\pgfqpoint{0.014433in}{0.024851in}}{\pgfqpoint{0.007367in}{0.027778in}}{\pgfqpoint{0.000000in}{0.027778in}}%
\pgfpathcurveto{\pgfqpoint{-0.007367in}{0.027778in}}{\pgfqpoint{-0.014433in}{0.024851in}}{\pgfqpoint{-0.019642in}{0.019642in}}%
\pgfpathcurveto{\pgfqpoint{-0.024851in}{0.014433in}}{\pgfqpoint{-0.027778in}{0.007367in}}{\pgfqpoint{-0.027778in}{0.000000in}}%
\pgfpathcurveto{\pgfqpoint{-0.027778in}{-0.007367in}}{\pgfqpoint{-0.024851in}{-0.014433in}}{\pgfqpoint{-0.019642in}{-0.019642in}}%
\pgfpathcurveto{\pgfqpoint{-0.014433in}{-0.024851in}}{\pgfqpoint{-0.007367in}{-0.027778in}}{\pgfqpoint{0.000000in}{-0.027778in}}%
\pgfpathclose%
\pgfusepath{stroke,fill}%
}%
\begin{pgfscope}%
\pgfsys@transformshift{1.353220in}{2.635158in}%
\pgfsys@useobject{currentmarker}{}%
\end{pgfscope}%
\end{pgfscope}%
\begin{pgfscope}%
\pgfsetrectcap%
\pgfsetmiterjoin%
\pgfsetlinewidth{0.752812pt}%
\definecolor{currentstroke}{rgb}{0.000000,0.000000,0.000000}%
\pgfsetstrokecolor{currentstroke}%
\pgfsetdash{}{0pt}%
\pgfpathmoveto{\pgfqpoint{0.220285in}{2.412707in}}%
\pgfpathlineto{\pgfqpoint{0.220285in}{3.335519in}}%
\pgfusepath{stroke}%
\end{pgfscope}%
\begin{pgfscope}%
\pgfsetrectcap%
\pgfsetmiterjoin%
\pgfsetlinewidth{0.752812pt}%
\definecolor{currentstroke}{rgb}{0.000000,0.000000,0.000000}%
\pgfsetstrokecolor{currentstroke}%
\pgfsetdash{}{0pt}%
\pgfpathmoveto{\pgfqpoint{2.698492in}{2.412707in}}%
\pgfpathlineto{\pgfqpoint{2.698492in}{3.335519in}}%
\pgfusepath{stroke}%
\end{pgfscope}%
\begin{pgfscope}%
\pgfsetrectcap%
\pgfsetmiterjoin%
\pgfsetlinewidth{0.752812pt}%
\definecolor{currentstroke}{rgb}{0.000000,0.000000,0.000000}%
\pgfsetstrokecolor{currentstroke}%
\pgfsetdash{}{0pt}%
\pgfpathmoveto{\pgfqpoint{0.220285in}{2.412707in}}%
\pgfpathlineto{\pgfqpoint{2.698492in}{2.412707in}}%
\pgfusepath{stroke}%
\end{pgfscope}%
\begin{pgfscope}%
\pgfsetrectcap%
\pgfsetmiterjoin%
\pgfsetlinewidth{0.752812pt}%
\definecolor{currentstroke}{rgb}{0.000000,0.000000,0.000000}%
\pgfsetstrokecolor{currentstroke}%
\pgfsetdash{}{0pt}%
\pgfpathmoveto{\pgfqpoint{0.220285in}{3.335519in}}%
\pgfpathlineto{\pgfqpoint{2.698492in}{3.335519in}}%
\pgfusepath{stroke}%
\end{pgfscope}%
\begin{pgfscope}%
\definecolor{textcolor}{rgb}{0.000000,0.000000,0.000000}%
\pgfsetstrokecolor{textcolor}%
\pgfsetfillcolor{textcolor}%
\pgftext[x=1.459388in,y=3.197097in,,base]{\color{textcolor}\rmfamily\fontsize{10.000000}{12.000000}\selectfont (1)}%
\end{pgfscope}%
\begin{pgfscope}%
\pgfsetbuttcap%
\pgfsetmiterjoin%
\definecolor{currentfill}{rgb}{1.000000,1.000000,1.000000}%
\pgfsetfillcolor{currentfill}%
\pgfsetlinewidth{0.000000pt}%
\definecolor{currentstroke}{rgb}{0.000000,0.000000,0.000000}%
\pgfsetstrokecolor{currentstroke}%
\pgfsetstrokeopacity{0.000000}%
\pgfsetdash{}{0pt}%
\pgfpathmoveto{\pgfqpoint{0.220285in}{1.940047in}}%
\pgfpathlineto{\pgfqpoint{2.698492in}{1.940047in}}%
\pgfpathlineto{\pgfqpoint{2.698492in}{2.390199in}}%
\pgfpathlineto{\pgfqpoint{0.220285in}{2.390199in}}%
\pgfpathclose%
\pgfusepath{fill}%
\end{pgfscope}%
\begin{pgfscope}%
\pgfsetbuttcap%
\pgfsetroundjoin%
\definecolor{currentfill}{rgb}{0.000000,0.000000,0.000000}%
\pgfsetfillcolor{currentfill}%
\pgfsetlinewidth{0.803000pt}%
\definecolor{currentstroke}{rgb}{0.000000,0.000000,0.000000}%
\pgfsetstrokecolor{currentstroke}%
\pgfsetdash{}{0pt}%
\pgfsys@defobject{currentmarker}{\pgfqpoint{0.000000in}{-0.048611in}}{\pgfqpoint{0.000000in}{0.000000in}}{%
\pgfpathmoveto{\pgfqpoint{0.000000in}{0.000000in}}%
\pgfpathlineto{\pgfqpoint{0.000000in}{-0.048611in}}%
\pgfusepath{stroke,fill}%
}%
\begin{pgfscope}%
\pgfsys@transformshift{0.220285in}{1.940047in}%
\pgfsys@useobject{currentmarker}{}%
\end{pgfscope}%
\end{pgfscope}%
\begin{pgfscope}%
\pgfsetbuttcap%
\pgfsetroundjoin%
\definecolor{currentfill}{rgb}{0.000000,0.000000,0.000000}%
\pgfsetfillcolor{currentfill}%
\pgfsetlinewidth{0.803000pt}%
\definecolor{currentstroke}{rgb}{0.000000,0.000000,0.000000}%
\pgfsetstrokecolor{currentstroke}%
\pgfsetdash{}{0pt}%
\pgfsys@defobject{currentmarker}{\pgfqpoint{0.000000in}{-0.048611in}}{\pgfqpoint{0.000000in}{0.000000in}}{%
\pgfpathmoveto{\pgfqpoint{0.000000in}{0.000000in}}%
\pgfpathlineto{\pgfqpoint{0.000000in}{-0.048611in}}%
\pgfusepath{stroke,fill}%
}%
\begin{pgfscope}%
\pgfsys@transformshift{0.633319in}{1.940047in}%
\pgfsys@useobject{currentmarker}{}%
\end{pgfscope}%
\end{pgfscope}%
\begin{pgfscope}%
\pgfsetbuttcap%
\pgfsetroundjoin%
\definecolor{currentfill}{rgb}{0.000000,0.000000,0.000000}%
\pgfsetfillcolor{currentfill}%
\pgfsetlinewidth{0.803000pt}%
\definecolor{currentstroke}{rgb}{0.000000,0.000000,0.000000}%
\pgfsetstrokecolor{currentstroke}%
\pgfsetdash{}{0pt}%
\pgfsys@defobject{currentmarker}{\pgfqpoint{0.000000in}{-0.048611in}}{\pgfqpoint{0.000000in}{0.000000in}}{%
\pgfpathmoveto{\pgfqpoint{0.000000in}{0.000000in}}%
\pgfpathlineto{\pgfqpoint{0.000000in}{-0.048611in}}%
\pgfusepath{stroke,fill}%
}%
\begin{pgfscope}%
\pgfsys@transformshift{1.046354in}{1.940047in}%
\pgfsys@useobject{currentmarker}{}%
\end{pgfscope}%
\end{pgfscope}%
\begin{pgfscope}%
\pgfsetbuttcap%
\pgfsetroundjoin%
\definecolor{currentfill}{rgb}{0.000000,0.000000,0.000000}%
\pgfsetfillcolor{currentfill}%
\pgfsetlinewidth{0.803000pt}%
\definecolor{currentstroke}{rgb}{0.000000,0.000000,0.000000}%
\pgfsetstrokecolor{currentstroke}%
\pgfsetdash{}{0pt}%
\pgfsys@defobject{currentmarker}{\pgfqpoint{0.000000in}{-0.048611in}}{\pgfqpoint{0.000000in}{0.000000in}}{%
\pgfpathmoveto{\pgfqpoint{0.000000in}{0.000000in}}%
\pgfpathlineto{\pgfqpoint{0.000000in}{-0.048611in}}%
\pgfusepath{stroke,fill}%
}%
\begin{pgfscope}%
\pgfsys@transformshift{1.459388in}{1.940047in}%
\pgfsys@useobject{currentmarker}{}%
\end{pgfscope}%
\end{pgfscope}%
\begin{pgfscope}%
\pgfsetbuttcap%
\pgfsetroundjoin%
\definecolor{currentfill}{rgb}{0.000000,0.000000,0.000000}%
\pgfsetfillcolor{currentfill}%
\pgfsetlinewidth{0.803000pt}%
\definecolor{currentstroke}{rgb}{0.000000,0.000000,0.000000}%
\pgfsetstrokecolor{currentstroke}%
\pgfsetdash{}{0pt}%
\pgfsys@defobject{currentmarker}{\pgfqpoint{0.000000in}{-0.048611in}}{\pgfqpoint{0.000000in}{0.000000in}}{%
\pgfpathmoveto{\pgfqpoint{0.000000in}{0.000000in}}%
\pgfpathlineto{\pgfqpoint{0.000000in}{-0.048611in}}%
\pgfusepath{stroke,fill}%
}%
\begin{pgfscope}%
\pgfsys@transformshift{1.872423in}{1.940047in}%
\pgfsys@useobject{currentmarker}{}%
\end{pgfscope}%
\end{pgfscope}%
\begin{pgfscope}%
\pgfsetbuttcap%
\pgfsetroundjoin%
\definecolor{currentfill}{rgb}{0.000000,0.000000,0.000000}%
\pgfsetfillcolor{currentfill}%
\pgfsetlinewidth{0.803000pt}%
\definecolor{currentstroke}{rgb}{0.000000,0.000000,0.000000}%
\pgfsetstrokecolor{currentstroke}%
\pgfsetdash{}{0pt}%
\pgfsys@defobject{currentmarker}{\pgfqpoint{0.000000in}{-0.048611in}}{\pgfqpoint{0.000000in}{0.000000in}}{%
\pgfpathmoveto{\pgfqpoint{0.000000in}{0.000000in}}%
\pgfpathlineto{\pgfqpoint{0.000000in}{-0.048611in}}%
\pgfusepath{stroke,fill}%
}%
\begin{pgfscope}%
\pgfsys@transformshift{2.285457in}{1.940047in}%
\pgfsys@useobject{currentmarker}{}%
\end{pgfscope}%
\end{pgfscope}%
\begin{pgfscope}%
\pgfsetbuttcap%
\pgfsetroundjoin%
\definecolor{currentfill}{rgb}{0.000000,0.000000,0.000000}%
\pgfsetfillcolor{currentfill}%
\pgfsetlinewidth{0.803000pt}%
\definecolor{currentstroke}{rgb}{0.000000,0.000000,0.000000}%
\pgfsetstrokecolor{currentstroke}%
\pgfsetdash{}{0pt}%
\pgfsys@defobject{currentmarker}{\pgfqpoint{0.000000in}{-0.048611in}}{\pgfqpoint{0.000000in}{0.000000in}}{%
\pgfpathmoveto{\pgfqpoint{0.000000in}{0.000000in}}%
\pgfpathlineto{\pgfqpoint{0.000000in}{-0.048611in}}%
\pgfusepath{stroke,fill}%
}%
\begin{pgfscope}%
\pgfsys@transformshift{2.698492in}{1.940047in}%
\pgfsys@useobject{currentmarker}{}%
\end{pgfscope}%
\end{pgfscope}%
\begin{pgfscope}%
\definecolor{textcolor}{rgb}{0.000000,0.000000,0.000000}%
\pgfsetstrokecolor{textcolor}%
\pgfsetfillcolor{textcolor}%
\pgftext[x=0.164729in,y=2.165123in,,bottom,rotate=90.000000]{\color{textcolor}\rmfamily\fontsize{10.000000}{12.000000}\selectfont \(\displaystyle \alpha(x|\mathcal{D})\)}%
\end{pgfscope}%
\begin{pgfscope}%
\pgfpathrectangle{\pgfqpoint{0.220285in}{1.940047in}}{\pgfqpoint{2.478207in}{0.450152in}}%
\pgfusepath{clip}%
\pgfsetbuttcap%
\pgfsetroundjoin%
\pgfsetlinewidth{0.752812pt}%
\definecolor{currentstroke}{rgb}{0.000000,0.000000,0.000000}%
\pgfsetstrokecolor{currentstroke}%
\pgfsetdash{{2.775000pt}{1.200000pt}}{0.000000pt}%
\pgfpathmoveto{\pgfqpoint{0.633319in}{1.940047in}}%
\pgfpathlineto{\pgfqpoint{0.633319in}{2.390199in}}%
\pgfusepath{stroke}%
\end{pgfscope}%
\begin{pgfscope}%
\pgfpathrectangle{\pgfqpoint{0.220285in}{1.940047in}}{\pgfqpoint{2.478207in}{0.450152in}}%
\pgfusepath{clip}%
\pgfsetbuttcap%
\pgfsetroundjoin%
\pgfsetlinewidth{0.752812pt}%
\definecolor{currentstroke}{rgb}{0.000000,0.000000,0.000000}%
\pgfsetstrokecolor{currentstroke}%
\pgfsetdash{{2.775000pt}{1.200000pt}}{0.000000pt}%
\pgfpathmoveto{\pgfqpoint{2.285457in}{1.940047in}}%
\pgfpathlineto{\pgfqpoint{2.285457in}{2.390199in}}%
\pgfusepath{stroke}%
\end{pgfscope}%
\begin{pgfscope}%
\pgfpathrectangle{\pgfqpoint{0.220285in}{1.940047in}}{\pgfqpoint{2.478207in}{0.450152in}}%
\pgfusepath{clip}%
\pgfsetbuttcap%
\pgfsetroundjoin%
\pgfsetlinewidth{1.505625pt}%
\definecolor{currentstroke}{rgb}{0.631373,0.062745,0.207843}%
\pgfsetstrokecolor{currentstroke}%
\pgfsetdash{{1.500000pt}{2.475000pt}}{0.000000pt}%
\pgfpathmoveto{\pgfqpoint{1.353220in}{1.940047in}}%
\pgfpathlineto{\pgfqpoint{1.353220in}{2.390199in}}%
\pgfusepath{stroke}%
\end{pgfscope}%
\begin{pgfscope}%
\pgfpathrectangle{\pgfqpoint{0.220285in}{1.940047in}}{\pgfqpoint{2.478207in}{0.450152in}}%
\pgfusepath{clip}%
\pgfsetrectcap%
\pgfsetroundjoin%
\pgfsetlinewidth{0.752812pt}%
\definecolor{currentstroke}{rgb}{0.964706,0.658824,0.000000}%
\pgfsetstrokecolor{currentstroke}%
\pgfsetdash{}{0pt}%
\pgfpathmoveto{\pgfqpoint{0.220285in}{2.225826in}}%
\pgfpathlineto{\pgfqpoint{0.245317in}{2.225824in}}%
\pgfpathlineto{\pgfqpoint{0.270350in}{2.225820in}}%
\pgfpathlineto{\pgfqpoint{0.295382in}{2.225812in}}%
\pgfpathlineto{\pgfqpoint{0.320415in}{2.225792in}}%
\pgfpathlineto{\pgfqpoint{0.345447in}{2.225749in}}%
\pgfpathlineto{\pgfqpoint{0.370479in}{2.225663in}}%
\pgfpathlineto{\pgfqpoint{0.395512in}{2.225495in}}%
\pgfpathlineto{\pgfqpoint{0.420544in}{2.225180in}}%
\pgfpathlineto{\pgfqpoint{0.445577in}{2.224620in}}%
\pgfpathlineto{\pgfqpoint{0.470609in}{2.223671in}}%
\pgfpathlineto{\pgfqpoint{0.495641in}{2.222145in}}%
\pgfpathlineto{\pgfqpoint{0.520674in}{2.219827in}}%
\pgfpathlineto{\pgfqpoint{0.545706in}{2.216522in}}%
\pgfpathlineto{\pgfqpoint{0.570739in}{2.212135in}}%
\pgfpathlineto{\pgfqpoint{0.595771in}{2.206785in}}%
\pgfpathlineto{\pgfqpoint{0.620803in}{2.200911in}}%
\pgfpathlineto{\pgfqpoint{0.645836in}{2.195334in}}%
\pgfpathlineto{\pgfqpoint{0.670868in}{2.191223in}}%
\pgfpathlineto{\pgfqpoint{0.695901in}{2.189923in}}%
\pgfpathlineto{\pgfqpoint{0.720933in}{2.192662in}}%
\pgfpathlineto{\pgfqpoint{0.745965in}{2.200186in}}%
\pgfpathlineto{\pgfqpoint{0.770998in}{2.212380in}}%
\pgfpathlineto{\pgfqpoint{0.796030in}{2.227759in}}%
\pgfpathlineto{\pgfqpoint{0.821062in}{2.232712in}}%
\pgfpathlineto{\pgfqpoint{0.846095in}{2.192910in}}%
\pgfpathlineto{\pgfqpoint{0.871127in}{2.152030in}}%
\pgfpathlineto{\pgfqpoint{0.896160in}{2.115803in}}%
\pgfpathlineto{\pgfqpoint{0.921192in}{2.086064in}}%
\pgfpathlineto{\pgfqpoint{0.946224in}{2.064396in}}%
\pgfpathlineto{\pgfqpoint{0.971257in}{2.052275in}}%
\pgfpathlineto{\pgfqpoint{0.996289in}{2.050952in}}%
\pgfpathlineto{\pgfqpoint{1.021322in}{2.061131in}}%
\pgfpathlineto{\pgfqpoint{1.046354in}{2.082294in}}%
\pgfpathlineto{\pgfqpoint{1.071386in}{2.099987in}}%
\pgfpathlineto{\pgfqpoint{1.096419in}{2.080907in}}%
\pgfpathlineto{\pgfqpoint{1.121451in}{2.066112in}}%
\pgfpathlineto{\pgfqpoint{1.146484in}{2.058515in}}%
\pgfpathlineto{\pgfqpoint{1.171516in}{2.056389in}}%
\pgfpathlineto{\pgfqpoint{1.196548in}{2.056938in}}%
\pgfpathlineto{\pgfqpoint{1.221581in}{2.057235in}}%
\pgfpathlineto{\pgfqpoint{1.246613in}{2.055065in}}%
\pgfpathlineto{\pgfqpoint{1.271646in}{2.049588in}}%
\pgfpathlineto{\pgfqpoint{1.296678in}{2.041632in}}%
\pgfpathlineto{\pgfqpoint{1.321710in}{2.033573in}}%
\pgfpathlineto{\pgfqpoint{1.346743in}{2.028796in}}%
\pgfpathlineto{\pgfqpoint{1.371775in}{2.030901in}}%
\pgfpathlineto{\pgfqpoint{1.396807in}{2.042783in}}%
\pgfpathlineto{\pgfqpoint{1.421840in}{2.064891in}}%
\pgfpathlineto{\pgfqpoint{1.446872in}{2.063984in}}%
\pgfpathlineto{\pgfqpoint{1.471905in}{2.047433in}}%
\pgfpathlineto{\pgfqpoint{1.496937in}{2.042771in}}%
\pgfpathlineto{\pgfqpoint{1.521969in}{2.049784in}}%
\pgfpathlineto{\pgfqpoint{1.547002in}{2.066129in}}%
\pgfpathlineto{\pgfqpoint{1.572034in}{2.088584in}}%
\pgfpathlineto{\pgfqpoint{1.597067in}{2.113744in}}%
\pgfpathlineto{\pgfqpoint{1.622099in}{2.138592in}}%
\pgfpathlineto{\pgfqpoint{1.647131in}{2.160880in}}%
\pgfpathlineto{\pgfqpoint{1.672164in}{2.179289in}}%
\pgfpathlineto{\pgfqpoint{1.697196in}{2.193388in}}%
\pgfpathlineto{\pgfqpoint{1.722228in}{2.203478in}}%
\pgfpathlineto{\pgfqpoint{1.747261in}{2.210397in}}%
\pgfpathlineto{\pgfqpoint{1.772293in}{2.215358in}}%
\pgfpathlineto{\pgfqpoint{1.797326in}{2.219818in}}%
\pgfpathlineto{\pgfqpoint{1.822358in}{2.225365in}}%
\pgfpathlineto{\pgfqpoint{1.847390in}{2.233575in}}%
\pgfpathlineto{\pgfqpoint{1.872423in}{2.245811in}}%
\pgfpathlineto{\pgfqpoint{1.897455in}{2.263006in}}%
\pgfpathlineto{\pgfqpoint{1.922488in}{2.285436in}}%
\pgfpathlineto{\pgfqpoint{1.947520in}{2.312513in}}%
\pgfpathlineto{\pgfqpoint{1.972552in}{2.341347in}}%
\pgfpathlineto{\pgfqpoint{1.997585in}{2.332846in}}%
\pgfpathlineto{\pgfqpoint{2.022617in}{2.303808in}}%
\pgfpathlineto{\pgfqpoint{2.047650in}{2.278125in}}%
\pgfpathlineto{\pgfqpoint{2.072682in}{2.257476in}}%
\pgfpathlineto{\pgfqpoint{2.097714in}{2.242237in}}%
\pgfpathlineto{\pgfqpoint{2.122747in}{2.232054in}}%
\pgfpathlineto{\pgfqpoint{2.147779in}{2.226074in}}%
\pgfpathlineto{\pgfqpoint{2.172812in}{2.223200in}}%
\pgfpathlineto{\pgfqpoint{2.197844in}{2.222340in}}%
\pgfpathlineto{\pgfqpoint{2.222876in}{2.222579in}}%
\pgfpathlineto{\pgfqpoint{2.247909in}{2.223267in}}%
\pgfpathlineto{\pgfqpoint{2.272941in}{2.224017in}}%
\pgfpathlineto{\pgfqpoint{2.297973in}{2.224648in}}%
\pgfpathlineto{\pgfqpoint{2.323006in}{2.225109in}}%
\pgfpathlineto{\pgfqpoint{2.348038in}{2.225414in}}%
\pgfpathlineto{\pgfqpoint{2.373071in}{2.225601in}}%
\pgfpathlineto{\pgfqpoint{2.398103in}{2.225709in}}%
\pgfpathlineto{\pgfqpoint{2.423135in}{2.225768in}}%
\pgfpathlineto{\pgfqpoint{2.448168in}{2.225799in}}%
\pgfpathlineto{\pgfqpoint{2.473200in}{2.225814in}}%
\pgfpathlineto{\pgfqpoint{2.498233in}{2.225821in}}%
\pgfpathlineto{\pgfqpoint{2.523265in}{2.225824in}}%
\pgfpathlineto{\pgfqpoint{2.548297in}{2.225826in}}%
\pgfpathlineto{\pgfqpoint{2.573330in}{2.225826in}}%
\pgfpathlineto{\pgfqpoint{2.598362in}{2.225827in}}%
\pgfpathlineto{\pgfqpoint{2.623395in}{2.225827in}}%
\pgfpathlineto{\pgfqpoint{2.648427in}{2.225827in}}%
\pgfpathlineto{\pgfqpoint{2.673459in}{2.225827in}}%
\pgfpathlineto{\pgfqpoint{2.698492in}{2.225827in}}%
\pgfusepath{stroke}%
\end{pgfscope}%
\begin{pgfscope}%
\pgfsetrectcap%
\pgfsetmiterjoin%
\pgfsetlinewidth{0.752812pt}%
\definecolor{currentstroke}{rgb}{0.000000,0.000000,0.000000}%
\pgfsetstrokecolor{currentstroke}%
\pgfsetdash{}{0pt}%
\pgfpathmoveto{\pgfqpoint{0.220285in}{1.940047in}}%
\pgfpathlineto{\pgfqpoint{0.220285in}{2.390199in}}%
\pgfusepath{stroke}%
\end{pgfscope}%
\begin{pgfscope}%
\pgfsetrectcap%
\pgfsetmiterjoin%
\pgfsetlinewidth{0.752812pt}%
\definecolor{currentstroke}{rgb}{0.000000,0.000000,0.000000}%
\pgfsetstrokecolor{currentstroke}%
\pgfsetdash{}{0pt}%
\pgfpathmoveto{\pgfqpoint{2.698492in}{1.940047in}}%
\pgfpathlineto{\pgfqpoint{2.698492in}{2.390199in}}%
\pgfusepath{stroke}%
\end{pgfscope}%
\begin{pgfscope}%
\pgfsetrectcap%
\pgfsetmiterjoin%
\pgfsetlinewidth{0.752812pt}%
\definecolor{currentstroke}{rgb}{0.000000,0.000000,0.000000}%
\pgfsetstrokecolor{currentstroke}%
\pgfsetdash{}{0pt}%
\pgfpathmoveto{\pgfqpoint{0.220285in}{1.940047in}}%
\pgfpathlineto{\pgfqpoint{2.698492in}{1.940047in}}%
\pgfusepath{stroke}%
\end{pgfscope}%
\begin{pgfscope}%
\pgfsetrectcap%
\pgfsetmiterjoin%
\pgfsetlinewidth{0.752812pt}%
\definecolor{currentstroke}{rgb}{0.000000,0.000000,0.000000}%
\pgfsetstrokecolor{currentstroke}%
\pgfsetdash{}{0pt}%
\pgfpathmoveto{\pgfqpoint{0.220285in}{2.390199in}}%
\pgfpathlineto{\pgfqpoint{2.698492in}{2.390199in}}%
\pgfusepath{stroke}%
\end{pgfscope}%
\begin{pgfscope}%
\pgfsetbuttcap%
\pgfsetmiterjoin%
\definecolor{currentfill}{rgb}{1.000000,1.000000,1.000000}%
\pgfsetfillcolor{currentfill}%
\pgfsetlinewidth{0.000000pt}%
\definecolor{currentstroke}{rgb}{0.000000,0.000000,0.000000}%
\pgfsetstrokecolor{currentstroke}%
\pgfsetstrokeopacity{0.000000}%
\pgfsetdash{}{0pt}%
\pgfpathmoveto{\pgfqpoint{2.918777in}{2.412707in}}%
\pgfpathlineto{\pgfqpoint{5.396984in}{2.412707in}}%
\pgfpathlineto{\pgfqpoint{5.396984in}{3.335519in}}%
\pgfpathlineto{\pgfqpoint{2.918777in}{3.335519in}}%
\pgfpathclose%
\pgfusepath{fill}%
\end{pgfscope}%
\begin{pgfscope}%
\pgfpathrectangle{\pgfqpoint{2.918777in}{2.412707in}}{\pgfqpoint{2.478207in}{0.922812in}}%
\pgfusepath{clip}%
\pgfsetbuttcap%
\pgfsetroundjoin%
\definecolor{currentfill}{rgb}{0.556863,0.729412,0.898039}%
\pgfsetfillcolor{currentfill}%
\pgfsetfillopacity{0.700000}%
\pgfsetlinewidth{0.000000pt}%
\definecolor{currentstroke}{rgb}{0.556863,0.729412,0.898039}%
\pgfsetstrokecolor{currentstroke}%
\pgfsetstrokeopacity{0.700000}%
\pgfsetdash{}{0pt}%
\pgfsys@defobject{currentmarker}{\pgfqpoint{2.918777in}{2.530265in}}{\pgfqpoint{5.396984in}{3.126836in}}{%
\pgfpathmoveto{\pgfqpoint{2.918777in}{3.126835in}}%
\pgfpathlineto{\pgfqpoint{2.918777in}{2.621390in}}%
\pgfpathlineto{\pgfqpoint{2.943809in}{2.621389in}}%
\pgfpathlineto{\pgfqpoint{2.968842in}{2.621386in}}%
\pgfpathlineto{\pgfqpoint{2.993874in}{2.621379in}}%
\pgfpathlineto{\pgfqpoint{3.018906in}{2.621363in}}%
\pgfpathlineto{\pgfqpoint{3.043939in}{2.621330in}}%
\pgfpathlineto{\pgfqpoint{3.068971in}{2.621264in}}%
\pgfpathlineto{\pgfqpoint{3.094004in}{2.621133in}}%
\pgfpathlineto{\pgfqpoint{3.119036in}{2.620891in}}%
\pgfpathlineto{\pgfqpoint{3.144068in}{2.620462in}}%
\pgfpathlineto{\pgfqpoint{3.169101in}{2.619743in}}%
\pgfpathlineto{\pgfqpoint{3.194133in}{2.618608in}}%
\pgfpathlineto{\pgfqpoint{3.219166in}{2.616938in}}%
\pgfpathlineto{\pgfqpoint{3.244198in}{2.614684in}}%
\pgfpathlineto{\pgfqpoint{3.269230in}{2.611967in}}%
\pgfpathlineto{\pgfqpoint{3.294263in}{2.609208in}}%
\pgfpathlineto{\pgfqpoint{3.319295in}{2.607242in}}%
\pgfpathlineto{\pgfqpoint{3.344327in}{2.607360in}}%
\pgfpathlineto{\pgfqpoint{3.369360in}{2.611211in}}%
\pgfpathlineto{\pgfqpoint{3.394392in}{2.620533in}}%
\pgfpathlineto{\pgfqpoint{3.419425in}{2.636752in}}%
\pgfpathlineto{\pgfqpoint{3.444457in}{2.660508in}}%
\pgfpathlineto{\pgfqpoint{3.469489in}{2.691210in}}%
\pgfpathlineto{\pgfqpoint{3.494522in}{2.726367in}}%
\pgfpathlineto{\pgfqpoint{3.519554in}{2.745148in}}%
\pgfpathlineto{\pgfqpoint{3.544587in}{2.695189in}}%
\pgfpathlineto{\pgfqpoint{3.569619in}{2.644158in}}%
\pgfpathlineto{\pgfqpoint{3.594651in}{2.601344in}}%
\pgfpathlineto{\pgfqpoint{3.619684in}{2.569720in}}%
\pgfpathlineto{\pgfqpoint{3.644716in}{2.551326in}}%
\pgfpathlineto{\pgfqpoint{3.669749in}{2.547392in}}%
\pgfpathlineto{\pgfqpoint{3.694781in}{2.558213in}}%
\pgfpathlineto{\pgfqpoint{3.719813in}{2.582895in}}%
\pgfpathlineto{\pgfqpoint{3.744846in}{2.618672in}}%
\pgfpathlineto{\pgfqpoint{3.769878in}{2.642244in}}%
\pgfpathlineto{\pgfqpoint{3.794911in}{2.605226in}}%
\pgfpathlineto{\pgfqpoint{3.819943in}{2.570751in}}%
\pgfpathlineto{\pgfqpoint{3.844975in}{2.546235in}}%
\pgfpathlineto{\pgfqpoint{3.870008in}{2.532843in}}%
\pgfpathlineto{\pgfqpoint{3.895040in}{2.530265in}}%
\pgfpathlineto{\pgfqpoint{3.920072in}{2.537194in}}%
\pgfpathlineto{\pgfqpoint{3.945105in}{2.551579in}}%
\pgfpathlineto{\pgfqpoint{3.970137in}{2.570797in}}%
\pgfpathlineto{\pgfqpoint{3.995170in}{2.591755in}}%
\pgfpathlineto{\pgfqpoint{4.020202in}{2.610822in}}%
\pgfpathlineto{\pgfqpoint{4.045234in}{2.621588in}}%
\pgfpathlineto{\pgfqpoint{4.070267in}{2.612087in}}%
\pgfpathlineto{\pgfqpoint{4.095299in}{2.606321in}}%
\pgfpathlineto{\pgfqpoint{4.120332in}{2.616792in}}%
\pgfpathlineto{\pgfqpoint{4.145364in}{2.622907in}}%
\pgfpathlineto{\pgfqpoint{4.170396in}{2.613003in}}%
\pgfpathlineto{\pgfqpoint{4.195429in}{2.600398in}}%
\pgfpathlineto{\pgfqpoint{4.220461in}{2.589520in}}%
\pgfpathlineto{\pgfqpoint{4.245494in}{2.582444in}}%
\pgfpathlineto{\pgfqpoint{4.270526in}{2.579916in}}%
\pgfpathlineto{\pgfqpoint{4.295558in}{2.581578in}}%
\pgfpathlineto{\pgfqpoint{4.320591in}{2.586301in}}%
\pgfpathlineto{\pgfqpoint{4.345623in}{2.592628in}}%
\pgfpathlineto{\pgfqpoint{4.370655in}{2.599210in}}%
\pgfpathlineto{\pgfqpoint{4.395688in}{2.605124in}}%
\pgfpathlineto{\pgfqpoint{4.420720in}{2.610048in}}%
\pgfpathlineto{\pgfqpoint{4.445753in}{2.614296in}}%
\pgfpathlineto{\pgfqpoint{4.470785in}{2.618776in}}%
\pgfpathlineto{\pgfqpoint{4.495817in}{2.624892in}}%
\pgfpathlineto{\pgfqpoint{4.520850in}{2.634377in}}%
\pgfpathlineto{\pgfqpoint{4.545882in}{2.649051in}}%
\pgfpathlineto{\pgfqpoint{4.570915in}{2.670490in}}%
\pgfpathlineto{\pgfqpoint{4.595947in}{2.699656in}}%
\pgfpathlineto{\pgfqpoint{4.620979in}{2.736561in}}%
\pgfpathlineto{\pgfqpoint{4.646012in}{2.779952in}}%
\pgfpathlineto{\pgfqpoint{4.671044in}{2.825138in}}%
\pgfpathlineto{\pgfqpoint{4.696077in}{2.811891in}}%
\pgfpathlineto{\pgfqpoint{4.721109in}{2.766044in}}%
\pgfpathlineto{\pgfqpoint{4.746141in}{2.724470in}}%
\pgfpathlineto{\pgfqpoint{4.771174in}{2.689983in}}%
\pgfpathlineto{\pgfqpoint{4.796206in}{2.663445in}}%
\pgfpathlineto{\pgfqpoint{4.821238in}{2.644597in}}%
\pgfpathlineto{\pgfqpoint{4.846271in}{2.632374in}}%
\pgfpathlineto{\pgfqpoint{4.871303in}{2.625271in}}%
\pgfpathlineto{\pgfqpoint{4.896336in}{2.621707in}}%
\pgfpathlineto{\pgfqpoint{4.921368in}{2.620304in}}%
\pgfpathlineto{\pgfqpoint{4.946400in}{2.620036in}}%
\pgfpathlineto{\pgfqpoint{4.971433in}{2.620246in}}%
\pgfpathlineto{\pgfqpoint{4.996465in}{2.620576in}}%
\pgfpathlineto{\pgfqpoint{5.021498in}{2.620870in}}%
\pgfpathlineto{\pgfqpoint{5.046530in}{2.621083in}}%
\pgfpathlineto{\pgfqpoint{5.071562in}{2.621220in}}%
\pgfpathlineto{\pgfqpoint{5.096595in}{2.621301in}}%
\pgfpathlineto{\pgfqpoint{5.121627in}{2.621346in}}%
\pgfpathlineto{\pgfqpoint{5.146660in}{2.621369in}}%
\pgfpathlineto{\pgfqpoint{5.171692in}{2.621381in}}%
\pgfpathlineto{\pgfqpoint{5.196724in}{2.621386in}}%
\pgfpathlineto{\pgfqpoint{5.221757in}{2.621389in}}%
\pgfpathlineto{\pgfqpoint{5.246789in}{2.621390in}}%
\pgfpathlineto{\pgfqpoint{5.271822in}{2.621390in}}%
\pgfpathlineto{\pgfqpoint{5.296854in}{2.621390in}}%
\pgfpathlineto{\pgfqpoint{5.321886in}{2.621390in}}%
\pgfpathlineto{\pgfqpoint{5.346919in}{2.621391in}}%
\pgfpathlineto{\pgfqpoint{5.371951in}{2.621391in}}%
\pgfpathlineto{\pgfqpoint{5.396984in}{2.621391in}}%
\pgfpathlineto{\pgfqpoint{5.396984in}{3.126836in}}%
\pgfpathlineto{\pgfqpoint{5.396984in}{3.126836in}}%
\pgfpathlineto{\pgfqpoint{5.371951in}{3.126836in}}%
\pgfpathlineto{\pgfqpoint{5.346919in}{3.126836in}}%
\pgfpathlineto{\pgfqpoint{5.321886in}{3.126836in}}%
\pgfpathlineto{\pgfqpoint{5.296854in}{3.126836in}}%
\pgfpathlineto{\pgfqpoint{5.271822in}{3.126835in}}%
\pgfpathlineto{\pgfqpoint{5.246789in}{3.126835in}}%
\pgfpathlineto{\pgfqpoint{5.221757in}{3.126834in}}%
\pgfpathlineto{\pgfqpoint{5.196724in}{3.126831in}}%
\pgfpathlineto{\pgfqpoint{5.171692in}{3.126826in}}%
\pgfpathlineto{\pgfqpoint{5.146660in}{3.126814in}}%
\pgfpathlineto{\pgfqpoint{5.121627in}{3.126790in}}%
\pgfpathlineto{\pgfqpoint{5.096595in}{3.126743in}}%
\pgfpathlineto{\pgfqpoint{5.071562in}{3.126654in}}%
\pgfpathlineto{\pgfqpoint{5.046530in}{3.126489in}}%
\pgfpathlineto{\pgfqpoint{5.021498in}{3.126190in}}%
\pgfpathlineto{\pgfqpoint{4.996465in}{3.125655in}}%
\pgfpathlineto{\pgfqpoint{4.971433in}{3.124704in}}%
\pgfpathlineto{\pgfqpoint{4.946400in}{3.123025in}}%
\pgfpathlineto{\pgfqpoint{4.921368in}{3.120105in}}%
\pgfpathlineto{\pgfqpoint{4.896336in}{3.115161in}}%
\pgfpathlineto{\pgfqpoint{4.871303in}{3.107112in}}%
\pgfpathlineto{\pgfqpoint{4.846271in}{3.094637in}}%
\pgfpathlineto{\pgfqpoint{4.821238in}{3.076356in}}%
\pgfpathlineto{\pgfqpoint{4.796206in}{3.051123in}}%
\pgfpathlineto{\pgfqpoint{4.771174in}{3.018365in}}%
\pgfpathlineto{\pgfqpoint{4.746141in}{2.978399in}}%
\pgfpathlineto{\pgfqpoint{4.721109in}{2.932654in}}%
\pgfpathlineto{\pgfqpoint{4.696077in}{2.884412in}}%
\pgfpathlineto{\pgfqpoint{4.671044in}{2.870823in}}%
\pgfpathlineto{\pgfqpoint{4.646012in}{2.917748in}}%
\pgfpathlineto{\pgfqpoint{4.620979in}{2.964728in}}%
\pgfpathlineto{\pgfqpoint{4.595947in}{3.006612in}}%
\pgfpathlineto{\pgfqpoint{4.570915in}{3.041519in}}%
\pgfpathlineto{\pgfqpoint{4.545882in}{3.068743in}}%
\pgfpathlineto{\pgfqpoint{4.520850in}{3.088489in}}%
\pgfpathlineto{\pgfqpoint{4.495817in}{3.101564in}}%
\pgfpathlineto{\pgfqpoint{4.470785in}{3.108992in}}%
\pgfpathlineto{\pgfqpoint{4.445753in}{3.111643in}}%
\pgfpathlineto{\pgfqpoint{4.420720in}{3.109943in}}%
\pgfpathlineto{\pgfqpoint{4.395688in}{3.103706in}}%
\pgfpathlineto{\pgfqpoint{4.370655in}{3.092092in}}%
\pgfpathlineto{\pgfqpoint{4.345623in}{3.073717in}}%
\pgfpathlineto{\pgfqpoint{4.320591in}{3.046934in}}%
\pgfpathlineto{\pgfqpoint{4.295558in}{3.010299in}}%
\pgfpathlineto{\pgfqpoint{4.270526in}{2.963187in}}%
\pgfpathlineto{\pgfqpoint{4.245494in}{2.906420in}}%
\pgfpathlineto{\pgfqpoint{4.220461in}{2.842730in}}%
\pgfpathlineto{\pgfqpoint{4.195429in}{2.776868in}}%
\pgfpathlineto{\pgfqpoint{4.170396in}{2.715346in}}%
\pgfpathlineto{\pgfqpoint{4.145364in}{2.666802in}}%
\pgfpathlineto{\pgfqpoint{4.120332in}{2.647793in}}%
\pgfpathlineto{\pgfqpoint{4.095299in}{2.648279in}}%
\pgfpathlineto{\pgfqpoint{4.070267in}{2.647260in}}%
\pgfpathlineto{\pgfqpoint{4.045234in}{2.654767in}}%
\pgfpathlineto{\pgfqpoint{4.020202in}{2.690602in}}%
\pgfpathlineto{\pgfqpoint{3.995170in}{2.737581in}}%
\pgfpathlineto{\pgfqpoint{3.970137in}{2.783973in}}%
\pgfpathlineto{\pgfqpoint{3.945105in}{2.821687in}}%
\pgfpathlineto{\pgfqpoint{3.920072in}{2.844843in}}%
\pgfpathlineto{\pgfqpoint{3.895040in}{2.850186in}}%
\pgfpathlineto{\pgfqpoint{3.870008in}{2.837237in}}%
\pgfpathlineto{\pgfqpoint{3.844975in}{2.808062in}}%
\pgfpathlineto{\pgfqpoint{3.819943in}{2.766758in}}%
\pgfpathlineto{\pgfqpoint{3.794911in}{2.718953in}}%
\pgfpathlineto{\pgfqpoint{3.769878in}{2.675647in}}%
\pgfpathlineto{\pgfqpoint{3.744846in}{2.701986in}}%
\pgfpathlineto{\pgfqpoint{3.719813in}{2.749754in}}%
\pgfpathlineto{\pgfqpoint{3.694781in}{2.794168in}}%
\pgfpathlineto{\pgfqpoint{3.669749in}{2.829886in}}%
\pgfpathlineto{\pgfqpoint{3.644716in}{2.853125in}}%
\pgfpathlineto{\pgfqpoint{3.619684in}{2.861714in}}%
\pgfpathlineto{\pgfqpoint{3.594651in}{2.855347in}}%
\pgfpathlineto{\pgfqpoint{3.569619in}{2.835635in}}%
\pgfpathlineto{\pgfqpoint{3.544587in}{2.806053in}}%
\pgfpathlineto{\pgfqpoint{3.519554in}{2.776918in}}%
\pgfpathlineto{\pgfqpoint{3.494522in}{2.816961in}}%
\pgfpathlineto{\pgfqpoint{3.469489in}{2.874513in}}%
\pgfpathlineto{\pgfqpoint{3.444457in}{2.928842in}}%
\pgfpathlineto{\pgfqpoint{3.419425in}{2.976961in}}%
\pgfpathlineto{\pgfqpoint{3.394392in}{3.017352in}}%
\pgfpathlineto{\pgfqpoint{3.369360in}{3.049566in}}%
\pgfpathlineto{\pgfqpoint{3.344327in}{3.074040in}}%
\pgfpathlineto{\pgfqpoint{3.319295in}{3.091815in}}%
\pgfpathlineto{\pgfqpoint{3.294263in}{3.104218in}}%
\pgfpathlineto{\pgfqpoint{3.269230in}{3.112581in}}%
\pgfpathlineto{\pgfqpoint{3.244198in}{3.118062in}}%
\pgfpathlineto{\pgfqpoint{3.219166in}{3.121567in}}%
\pgfpathlineto{\pgfqpoint{3.194133in}{3.123756in}}%
\pgfpathlineto{\pgfqpoint{3.169101in}{3.125088in}}%
\pgfpathlineto{\pgfqpoint{3.144068in}{3.125876in}}%
\pgfpathlineto{\pgfqpoint{3.119036in}{3.126327in}}%
\pgfpathlineto{\pgfqpoint{3.094004in}{3.126576in}}%
\pgfpathlineto{\pgfqpoint{3.068971in}{3.126708in}}%
\pgfpathlineto{\pgfqpoint{3.043939in}{3.126776in}}%
\pgfpathlineto{\pgfqpoint{3.018906in}{3.126808in}}%
\pgfpathlineto{\pgfqpoint{2.993874in}{3.126824in}}%
\pgfpathlineto{\pgfqpoint{2.968842in}{3.126831in}}%
\pgfpathlineto{\pgfqpoint{2.943809in}{3.126834in}}%
\pgfpathlineto{\pgfqpoint{2.918777in}{3.126835in}}%
\pgfpathclose%
\pgfusepath{fill}%
}%
\begin{pgfscope}%
\pgfsys@transformshift{0.000000in}{0.000000in}%
\pgfsys@useobject{currentmarker}{}%
\end{pgfscope}%
\end{pgfscope}%
\begin{pgfscope}%
\pgfpathrectangle{\pgfqpoint{2.918777in}{2.412707in}}{\pgfqpoint{2.478207in}{0.922812in}}%
\pgfusepath{clip}%
\pgfsetbuttcap%
\pgfsetroundjoin%
\pgfsetlinewidth{0.752812pt}%
\definecolor{currentstroke}{rgb}{0.000000,0.000000,0.000000}%
\pgfsetstrokecolor{currentstroke}%
\pgfsetdash{{2.775000pt}{1.200000pt}}{0.000000pt}%
\pgfpathmoveto{\pgfqpoint{3.331811in}{2.412707in}}%
\pgfpathlineto{\pgfqpoint{3.331811in}{3.335519in}}%
\pgfusepath{stroke}%
\end{pgfscope}%
\begin{pgfscope}%
\pgfpathrectangle{\pgfqpoint{2.918777in}{2.412707in}}{\pgfqpoint{2.478207in}{0.922812in}}%
\pgfusepath{clip}%
\pgfsetbuttcap%
\pgfsetroundjoin%
\pgfsetlinewidth{0.752812pt}%
\definecolor{currentstroke}{rgb}{0.000000,0.000000,0.000000}%
\pgfsetstrokecolor{currentstroke}%
\pgfsetdash{{2.775000pt}{1.200000pt}}{0.000000pt}%
\pgfpathmoveto{\pgfqpoint{4.983949in}{2.412707in}}%
\pgfpathlineto{\pgfqpoint{4.983949in}{3.335519in}}%
\pgfusepath{stroke}%
\end{pgfscope}%
\begin{pgfscope}%
\pgfpathrectangle{\pgfqpoint{2.918777in}{2.412707in}}{\pgfqpoint{2.478207in}{0.922812in}}%
\pgfusepath{clip}%
\pgfsetbuttcap%
\pgfsetroundjoin%
\pgfsetlinewidth{1.505625pt}%
\definecolor{currentstroke}{rgb}{0.631373,0.062745,0.207843}%
\pgfsetstrokecolor{currentstroke}%
\pgfsetdash{{1.500000pt}{2.475000pt}}{0.000000pt}%
\pgfpathmoveto{\pgfqpoint{3.873031in}{2.412707in}}%
\pgfpathlineto{\pgfqpoint{3.873031in}{3.335519in}}%
\pgfusepath{stroke}%
\end{pgfscope}%
\begin{pgfscope}%
\pgfpathrectangle{\pgfqpoint{2.918777in}{2.412707in}}{\pgfqpoint{2.478207in}{0.922812in}}%
\pgfusepath{clip}%
\pgfsetbuttcap%
\pgfsetroundjoin%
\definecolor{currentfill}{rgb}{0.000000,0.000000,0.000000}%
\pgfsetfillcolor{currentfill}%
\pgfsetlinewidth{1.003750pt}%
\definecolor{currentstroke}{rgb}{0.000000,0.000000,0.000000}%
\pgfsetstrokecolor{currentstroke}%
\pgfsetdash{}{0pt}%
\pgfsys@defobject{currentmarker}{\pgfqpoint{-0.020833in}{-0.020833in}}{\pgfqpoint{0.020833in}{0.020833in}}{%
\pgfpathmoveto{\pgfqpoint{0.000000in}{-0.020833in}}%
\pgfpathcurveto{\pgfqpoint{0.005525in}{-0.020833in}}{\pgfqpoint{0.010825in}{-0.018638in}}{\pgfqpoint{0.014731in}{-0.014731in}}%
\pgfpathcurveto{\pgfqpoint{0.018638in}{-0.010825in}}{\pgfqpoint{0.020833in}{-0.005525in}}{\pgfqpoint{0.020833in}{0.000000in}}%
\pgfpathcurveto{\pgfqpoint{0.020833in}{0.005525in}}{\pgfqpoint{0.018638in}{0.010825in}}{\pgfqpoint{0.014731in}{0.014731in}}%
\pgfpathcurveto{\pgfqpoint{0.010825in}{0.018638in}}{\pgfqpoint{0.005525in}{0.020833in}}{\pgfqpoint{0.000000in}{0.020833in}}%
\pgfpathcurveto{\pgfqpoint{-0.005525in}{0.020833in}}{\pgfqpoint{-0.010825in}{0.018638in}}{\pgfqpoint{-0.014731in}{0.014731in}}%
\pgfpathcurveto{\pgfqpoint{-0.018638in}{0.010825in}}{\pgfqpoint{-0.020833in}{0.005525in}}{\pgfqpoint{-0.020833in}{0.000000in}}%
\pgfpathcurveto{\pgfqpoint{-0.020833in}{-0.005525in}}{\pgfqpoint{-0.018638in}{-0.010825in}}{\pgfqpoint{-0.014731in}{-0.014731in}}%
\pgfpathcurveto{\pgfqpoint{-0.010825in}{-0.018638in}}{\pgfqpoint{-0.005525in}{-0.020833in}}{\pgfqpoint{0.000000in}{-0.020833in}}%
\pgfpathclose%
\pgfusepath{stroke,fill}%
}%
\begin{pgfscope}%
\pgfsys@transformshift{3.516296in}{2.762108in}%
\pgfsys@useobject{currentmarker}{}%
\end{pgfscope}%
\begin{pgfscope}%
\pgfsys@transformshift{4.679697in}{2.847832in}%
\pgfsys@useobject{currentmarker}{}%
\end{pgfscope}%
\begin{pgfscope}%
\pgfsys@transformshift{3.765600in}{2.658198in}%
\pgfsys@useobject{currentmarker}{}%
\end{pgfscope}%
\begin{pgfscope}%
\pgfsys@transformshift{4.131244in}{2.636355in}%
\pgfsys@useobject{currentmarker}{}%
\end{pgfscope}%
\begin{pgfscope}%
\pgfsys@transformshift{4.051712in}{2.635158in}%
\pgfsys@useobject{currentmarker}{}%
\end{pgfscope}%
\end{pgfscope}%
\begin{pgfscope}%
\pgfpathrectangle{\pgfqpoint{2.918777in}{2.412707in}}{\pgfqpoint{2.478207in}{0.922812in}}%
\pgfusepath{clip}%
\pgfsetrectcap%
\pgfsetroundjoin%
\pgfsetlinewidth{0.752812pt}%
\definecolor{currentstroke}{rgb}{0.000000,0.329412,0.623529}%
\pgfsetstrokecolor{currentstroke}%
\pgfsetdash{}{0pt}%
\pgfpathmoveto{\pgfqpoint{2.918777in}{2.874112in}}%
\pgfpathlineto{\pgfqpoint{2.943809in}{2.874111in}}%
\pgfpathlineto{\pgfqpoint{2.968842in}{2.874108in}}%
\pgfpathlineto{\pgfqpoint{2.993874in}{2.874101in}}%
\pgfpathlineto{\pgfqpoint{3.018906in}{2.874086in}}%
\pgfpathlineto{\pgfqpoint{3.043939in}{2.874053in}}%
\pgfpathlineto{\pgfqpoint{3.068971in}{2.873986in}}%
\pgfpathlineto{\pgfqpoint{3.094004in}{2.873855in}}%
\pgfpathlineto{\pgfqpoint{3.119036in}{2.873609in}}%
\pgfpathlineto{\pgfqpoint{3.144068in}{2.873169in}}%
\pgfpathlineto{\pgfqpoint{3.169101in}{2.872415in}}%
\pgfpathlineto{\pgfqpoint{3.194133in}{2.871182in}}%
\pgfpathlineto{\pgfqpoint{3.219166in}{2.869253in}}%
\pgfpathlineto{\pgfqpoint{3.244198in}{2.866373in}}%
\pgfpathlineto{\pgfqpoint{3.269230in}{2.862274in}}%
\pgfpathlineto{\pgfqpoint{3.294263in}{2.856713in}}%
\pgfpathlineto{\pgfqpoint{3.319295in}{2.849529in}}%
\pgfpathlineto{\pgfqpoint{3.344327in}{2.840700in}}%
\pgfpathlineto{\pgfqpoint{3.369360in}{2.830389in}}%
\pgfpathlineto{\pgfqpoint{3.394392in}{2.818943in}}%
\pgfpathlineto{\pgfqpoint{3.419425in}{2.806856in}}%
\pgfpathlineto{\pgfqpoint{3.444457in}{2.794675in}}%
\pgfpathlineto{\pgfqpoint{3.469489in}{2.782861in}}%
\pgfpathlineto{\pgfqpoint{3.494522in}{2.771664in}}%
\pgfpathlineto{\pgfqpoint{3.519554in}{2.761033in}}%
\pgfpathlineto{\pgfqpoint{3.544587in}{2.750621in}}%
\pgfpathlineto{\pgfqpoint{3.569619in}{2.739897in}}%
\pgfpathlineto{\pgfqpoint{3.594651in}{2.728346in}}%
\pgfpathlineto{\pgfqpoint{3.619684in}{2.715717in}}%
\pgfpathlineto{\pgfqpoint{3.644716in}{2.702226in}}%
\pgfpathlineto{\pgfqpoint{3.669749in}{2.688639in}}%
\pgfpathlineto{\pgfqpoint{3.694781in}{2.676190in}}%
\pgfpathlineto{\pgfqpoint{3.719813in}{2.666325in}}%
\pgfpathlineto{\pgfqpoint{3.744846in}{2.660329in}}%
\pgfpathlineto{\pgfqpoint{3.769878in}{2.658946in}}%
\pgfpathlineto{\pgfqpoint{3.794911in}{2.662090in}}%
\pgfpathlineto{\pgfqpoint{3.819943in}{2.668754in}}%
\pgfpathlineto{\pgfqpoint{3.844975in}{2.677148in}}%
\pgfpathlineto{\pgfqpoint{3.870008in}{2.685040in}}%
\pgfpathlineto{\pgfqpoint{3.895040in}{2.690226in}}%
\pgfpathlineto{\pgfqpoint{3.920072in}{2.691019in}}%
\pgfpathlineto{\pgfqpoint{3.945105in}{2.686633in}}%
\pgfpathlineto{\pgfqpoint{3.970137in}{2.677385in}}%
\pgfpathlineto{\pgfqpoint{3.995170in}{2.664668in}}%
\pgfpathlineto{\pgfqpoint{4.020202in}{2.650712in}}%
\pgfpathlineto{\pgfqpoint{4.045234in}{2.638177in}}%
\pgfpathlineto{\pgfqpoint{4.070267in}{2.629674in}}%
\pgfpathlineto{\pgfqpoint{4.095299in}{2.627300in}}%
\pgfpathlineto{\pgfqpoint{4.120332in}{2.632293in}}%
\pgfpathlineto{\pgfqpoint{4.145364in}{2.644854in}}%
\pgfpathlineto{\pgfqpoint{4.170396in}{2.664174in}}%
\pgfpathlineto{\pgfqpoint{4.195429in}{2.688633in}}%
\pgfpathlineto{\pgfqpoint{4.220461in}{2.716125in}}%
\pgfpathlineto{\pgfqpoint{4.245494in}{2.744432in}}%
\pgfpathlineto{\pgfqpoint{4.270526in}{2.771551in}}%
\pgfpathlineto{\pgfqpoint{4.295558in}{2.795939in}}%
\pgfpathlineto{\pgfqpoint{4.320591in}{2.816617in}}%
\pgfpathlineto{\pgfqpoint{4.345623in}{2.833173in}}%
\pgfpathlineto{\pgfqpoint{4.370655in}{2.845651in}}%
\pgfpathlineto{\pgfqpoint{4.395688in}{2.854415in}}%
\pgfpathlineto{\pgfqpoint{4.420720in}{2.859996in}}%
\pgfpathlineto{\pgfqpoint{4.445753in}{2.862970in}}%
\pgfpathlineto{\pgfqpoint{4.470785in}{2.863884in}}%
\pgfpathlineto{\pgfqpoint{4.495817in}{2.863228in}}%
\pgfpathlineto{\pgfqpoint{4.520850in}{2.861433in}}%
\pgfpathlineto{\pgfqpoint{4.545882in}{2.858897in}}%
\pgfpathlineto{\pgfqpoint{4.570915in}{2.856005in}}%
\pgfpathlineto{\pgfqpoint{4.595947in}{2.853134in}}%
\pgfpathlineto{\pgfqpoint{4.620979in}{2.850645in}}%
\pgfpathlineto{\pgfqpoint{4.646012in}{2.848850in}}%
\pgfpathlineto{\pgfqpoint{4.671044in}{2.847980in}}%
\pgfpathlineto{\pgfqpoint{4.696077in}{2.848151in}}%
\pgfpathlineto{\pgfqpoint{4.721109in}{2.849349in}}%
\pgfpathlineto{\pgfqpoint{4.746141in}{2.851434in}}%
\pgfpathlineto{\pgfqpoint{4.771174in}{2.854174in}}%
\pgfpathlineto{\pgfqpoint{4.796206in}{2.857284in}}%
\pgfpathlineto{\pgfqpoint{4.821238in}{2.860477in}}%
\pgfpathlineto{\pgfqpoint{4.846271in}{2.863505in}}%
\pgfpathlineto{\pgfqpoint{4.871303in}{2.866191in}}%
\pgfpathlineto{\pgfqpoint{4.896336in}{2.868434in}}%
\pgfpathlineto{\pgfqpoint{4.921368in}{2.870204in}}%
\pgfpathlineto{\pgfqpoint{4.946400in}{2.871531in}}%
\pgfpathlineto{\pgfqpoint{4.971433in}{2.872475in}}%
\pgfpathlineto{\pgfqpoint{4.996465in}{2.873116in}}%
\pgfpathlineto{\pgfqpoint{5.021498in}{2.873530in}}%
\pgfpathlineto{\pgfqpoint{5.046530in}{2.873786in}}%
\pgfpathlineto{\pgfqpoint{5.071562in}{2.873937in}}%
\pgfpathlineto{\pgfqpoint{5.096595in}{2.874022in}}%
\pgfpathlineto{\pgfqpoint{5.121627in}{2.874068in}}%
\pgfpathlineto{\pgfqpoint{5.146660in}{2.874091in}}%
\pgfpathlineto{\pgfqpoint{5.171692in}{2.874103in}}%
\pgfpathlineto{\pgfqpoint{5.196724in}{2.874109in}}%
\pgfpathlineto{\pgfqpoint{5.221757in}{2.874111in}}%
\pgfpathlineto{\pgfqpoint{5.246789in}{2.874112in}}%
\pgfpathlineto{\pgfqpoint{5.271822in}{2.874113in}}%
\pgfpathlineto{\pgfqpoint{5.296854in}{2.874113in}}%
\pgfpathlineto{\pgfqpoint{5.321886in}{2.874113in}}%
\pgfpathlineto{\pgfqpoint{5.346919in}{2.874113in}}%
\pgfpathlineto{\pgfqpoint{5.371951in}{2.874113in}}%
\pgfpathlineto{\pgfqpoint{5.396984in}{2.874113in}}%
\pgfusepath{stroke}%
\end{pgfscope}%
\begin{pgfscope}%
\pgfpathrectangle{\pgfqpoint{2.918777in}{2.412707in}}{\pgfqpoint{2.478207in}{0.922812in}}%
\pgfusepath{clip}%
\pgfsetrectcap%
\pgfsetroundjoin%
\pgfsetlinewidth{0.752812pt}%
\definecolor{currentstroke}{rgb}{0.000000,0.000000,0.000000}%
\pgfsetstrokecolor{currentstroke}%
\pgfsetdash{}{0pt}%
\pgfpathmoveto{\pgfqpoint{3.101702in}{3.345519in}}%
\pgfpathlineto{\pgfqpoint{3.119036in}{3.276526in}}%
\pgfpathlineto{\pgfqpoint{3.144068in}{3.209803in}}%
\pgfpathlineto{\pgfqpoint{3.169101in}{3.172696in}}%
\pgfpathlineto{\pgfqpoint{3.194133in}{3.159889in}}%
\pgfpathlineto{\pgfqpoint{3.219166in}{3.164603in}}%
\pgfpathlineto{\pgfqpoint{3.244198in}{3.179272in}}%
\pgfpathlineto{\pgfqpoint{3.269230in}{3.196267in}}%
\pgfpathlineto{\pgfqpoint{3.294263in}{3.208582in}}%
\pgfpathlineto{\pgfqpoint{3.319295in}{3.210447in}}%
\pgfpathlineto{\pgfqpoint{3.344327in}{3.197783in}}%
\pgfpathlineto{\pgfqpoint{3.369360in}{3.168475in}}%
\pgfpathlineto{\pgfqpoint{3.394392in}{3.122450in}}%
\pgfpathlineto{\pgfqpoint{3.419425in}{3.061549in}}%
\pgfpathlineto{\pgfqpoint{3.444457in}{2.989222in}}%
\pgfpathlineto{\pgfqpoint{3.469489in}{2.910089in}}%
\pgfpathlineto{\pgfqpoint{3.494522in}{2.829404in}}%
\pgfpathlineto{\pgfqpoint{3.519554in}{2.752489in}}%
\pgfpathlineto{\pgfqpoint{3.544587in}{2.684195in}}%
\pgfpathlineto{\pgfqpoint{3.569619in}{2.628439in}}%
\pgfpathlineto{\pgfqpoint{3.594651in}{2.587863in}}%
\pgfpathlineto{\pgfqpoint{3.619684in}{2.563637in}}%
\pgfpathlineto{\pgfqpoint{3.644716in}{2.555427in}}%
\pgfpathlineto{\pgfqpoint{3.669749in}{2.561518in}}%
\pgfpathlineto{\pgfqpoint{3.694781in}{2.579069in}}%
\pgfpathlineto{\pgfqpoint{3.719813in}{2.604471in}}%
\pgfpathlineto{\pgfqpoint{3.744846in}{2.633759in}}%
\pgfpathlineto{\pgfqpoint{3.769878in}{2.663046in}}%
\pgfpathlineto{\pgfqpoint{3.794911in}{2.688907in}}%
\pgfpathlineto{\pgfqpoint{3.819943in}{2.708699in}}%
\pgfpathlineto{\pgfqpoint{3.844975in}{2.720773in}}%
\pgfpathlineto{\pgfqpoint{3.870008in}{2.724560in}}%
\pgfpathlineto{\pgfqpoint{3.895040in}{2.720539in}}%
\pgfpathlineto{\pgfqpoint{3.920072in}{2.710079in}}%
\pgfpathlineto{\pgfqpoint{3.945105in}{2.695197in}}%
\pgfpathlineto{\pgfqpoint{3.970137in}{2.678250in}}%
\pgfpathlineto{\pgfqpoint{3.995170in}{2.661615in}}%
\pgfpathlineto{\pgfqpoint{4.020202in}{2.647375in}}%
\pgfpathlineto{\pgfqpoint{4.045234in}{2.637067in}}%
\pgfpathlineto{\pgfqpoint{4.070267in}{2.631508in}}%
\pgfpathlineto{\pgfqpoint{4.095299in}{2.630725in}}%
\pgfpathlineto{\pgfqpoint{4.120332in}{2.633990in}}%
\pgfpathlineto{\pgfqpoint{4.145364in}{2.639955in}}%
\pgfpathlineto{\pgfqpoint{4.170396in}{2.646865in}}%
\pgfpathlineto{\pgfqpoint{4.195429in}{2.652833in}}%
\pgfpathlineto{\pgfqpoint{4.220461in}{2.656120in}}%
\pgfpathlineto{\pgfqpoint{4.245494in}{2.655406in}}%
\pgfpathlineto{\pgfqpoint{4.270526in}{2.650008in}}%
\pgfpathlineto{\pgfqpoint{4.295558in}{2.640014in}}%
\pgfpathlineto{\pgfqpoint{4.320591in}{2.626316in}}%
\pgfpathlineto{\pgfqpoint{4.345623in}{2.610540in}}%
\pgfpathlineto{\pgfqpoint{4.370655in}{2.594873in}}%
\pgfpathlineto{\pgfqpoint{4.395688in}{2.581811in}}%
\pgfpathlineto{\pgfqpoint{4.420720in}{2.573847in}}%
\pgfpathlineto{\pgfqpoint{4.445753in}{2.573147in}}%
\pgfpathlineto{\pgfqpoint{4.470785in}{2.581245in}}%
\pgfpathlineto{\pgfqpoint{4.495817in}{2.598797in}}%
\pgfpathlineto{\pgfqpoint{4.520850in}{2.625431in}}%
\pgfpathlineto{\pgfqpoint{4.545882in}{2.659705in}}%
\pgfpathlineto{\pgfqpoint{4.570915in}{2.699201in}}%
\pgfpathlineto{\pgfqpoint{4.595947in}{2.740734in}}%
\pgfpathlineto{\pgfqpoint{4.620979in}{2.780669in}}%
\pgfpathlineto{\pgfqpoint{4.646012in}{2.815311in}}%
\pgfpathlineto{\pgfqpoint{4.671044in}{2.841335in}}%
\pgfpathlineto{\pgfqpoint{4.696077in}{2.856198in}}%
\pgfpathlineto{\pgfqpoint{4.721109in}{2.858500in}}%
\pgfpathlineto{\pgfqpoint{4.746141in}{2.848235in}}%
\pgfpathlineto{\pgfqpoint{4.771174in}{2.826916in}}%
\pgfpathlineto{\pgfqpoint{4.796206in}{2.797545in}}%
\pgfpathlineto{\pgfqpoint{4.821238in}{2.764411in}}%
\pgfpathlineto{\pgfqpoint{4.846271in}{2.732754in}}%
\pgfpathlineto{\pgfqpoint{4.871303in}{2.708300in}}%
\pgfpathlineto{\pgfqpoint{4.896336in}{2.696721in}}%
\pgfpathlineto{\pgfqpoint{4.921368in}{2.703070in}}%
\pgfpathlineto{\pgfqpoint{4.946400in}{2.731245in}}%
\pgfpathlineto{\pgfqpoint{4.971433in}{2.783551in}}%
\pgfpathlineto{\pgfqpoint{4.996465in}{2.860392in}}%
\pgfpathlineto{\pgfqpoint{5.021498in}{2.960148in}}%
\pgfpathlineto{\pgfqpoint{5.046530in}{3.079245in}}%
\pgfpathlineto{\pgfqpoint{5.071562in}{3.212437in}}%
\pgfpathlineto{\pgfqpoint{5.095219in}{3.345519in}}%
\pgfusepath{stroke}%
\end{pgfscope}%
\begin{pgfscope}%
\pgfpathrectangle{\pgfqpoint{2.918777in}{2.412707in}}{\pgfqpoint{2.478207in}{0.922812in}}%
\pgfusepath{clip}%
\pgfsetbuttcap%
\pgfsetroundjoin%
\definecolor{currentfill}{rgb}{0.631373,0.062745,0.207843}%
\pgfsetfillcolor{currentfill}%
\pgfsetlinewidth{1.003750pt}%
\definecolor{currentstroke}{rgb}{0.631373,0.062745,0.207843}%
\pgfsetstrokecolor{currentstroke}%
\pgfsetdash{}{0pt}%
\pgfsys@defobject{currentmarker}{\pgfqpoint{-0.027778in}{-0.027778in}}{\pgfqpoint{0.027778in}{0.027778in}}{%
\pgfpathmoveto{\pgfqpoint{0.000000in}{-0.027778in}}%
\pgfpathcurveto{\pgfqpoint{0.007367in}{-0.027778in}}{\pgfqpoint{0.014433in}{-0.024851in}}{\pgfqpoint{0.019642in}{-0.019642in}}%
\pgfpathcurveto{\pgfqpoint{0.024851in}{-0.014433in}}{\pgfqpoint{0.027778in}{-0.007367in}}{\pgfqpoint{0.027778in}{0.000000in}}%
\pgfpathcurveto{\pgfqpoint{0.027778in}{0.007367in}}{\pgfqpoint{0.024851in}{0.014433in}}{\pgfqpoint{0.019642in}{0.019642in}}%
\pgfpathcurveto{\pgfqpoint{0.014433in}{0.024851in}}{\pgfqpoint{0.007367in}{0.027778in}}{\pgfqpoint{0.000000in}{0.027778in}}%
\pgfpathcurveto{\pgfqpoint{-0.007367in}{0.027778in}}{\pgfqpoint{-0.014433in}{0.024851in}}{\pgfqpoint{-0.019642in}{0.019642in}}%
\pgfpathcurveto{\pgfqpoint{-0.024851in}{0.014433in}}{\pgfqpoint{-0.027778in}{0.007367in}}{\pgfqpoint{-0.027778in}{0.000000in}}%
\pgfpathcurveto{\pgfqpoint{-0.027778in}{-0.007367in}}{\pgfqpoint{-0.024851in}{-0.014433in}}{\pgfqpoint{-0.019642in}{-0.019642in}}%
\pgfpathcurveto{\pgfqpoint{-0.014433in}{-0.024851in}}{\pgfqpoint{-0.007367in}{-0.027778in}}{\pgfqpoint{0.000000in}{-0.027778in}}%
\pgfpathclose%
\pgfusepath{stroke,fill}%
}%
\begin{pgfscope}%
\pgfsys@transformshift{3.873031in}{2.724470in}%
\pgfsys@useobject{currentmarker}{}%
\end{pgfscope}%
\end{pgfscope}%
\begin{pgfscope}%
\pgfsetrectcap%
\pgfsetmiterjoin%
\pgfsetlinewidth{0.752812pt}%
\definecolor{currentstroke}{rgb}{0.000000,0.000000,0.000000}%
\pgfsetstrokecolor{currentstroke}%
\pgfsetdash{}{0pt}%
\pgfpathmoveto{\pgfqpoint{2.918777in}{2.412707in}}%
\pgfpathlineto{\pgfqpoint{2.918777in}{3.335519in}}%
\pgfusepath{stroke}%
\end{pgfscope}%
\begin{pgfscope}%
\pgfsetrectcap%
\pgfsetmiterjoin%
\pgfsetlinewidth{0.752812pt}%
\definecolor{currentstroke}{rgb}{0.000000,0.000000,0.000000}%
\pgfsetstrokecolor{currentstroke}%
\pgfsetdash{}{0pt}%
\pgfpathmoveto{\pgfqpoint{5.396984in}{2.412707in}}%
\pgfpathlineto{\pgfqpoint{5.396984in}{3.335519in}}%
\pgfusepath{stroke}%
\end{pgfscope}%
\begin{pgfscope}%
\pgfsetrectcap%
\pgfsetmiterjoin%
\pgfsetlinewidth{0.752812pt}%
\definecolor{currentstroke}{rgb}{0.000000,0.000000,0.000000}%
\pgfsetstrokecolor{currentstroke}%
\pgfsetdash{}{0pt}%
\pgfpathmoveto{\pgfqpoint{2.918777in}{2.412707in}}%
\pgfpathlineto{\pgfqpoint{5.396984in}{2.412707in}}%
\pgfusepath{stroke}%
\end{pgfscope}%
\begin{pgfscope}%
\pgfsetrectcap%
\pgfsetmiterjoin%
\pgfsetlinewidth{0.752812pt}%
\definecolor{currentstroke}{rgb}{0.000000,0.000000,0.000000}%
\pgfsetstrokecolor{currentstroke}%
\pgfsetdash{}{0pt}%
\pgfpathmoveto{\pgfqpoint{2.918777in}{3.335519in}}%
\pgfpathlineto{\pgfqpoint{5.396984in}{3.335519in}}%
\pgfusepath{stroke}%
\end{pgfscope}%
\begin{pgfscope}%
\definecolor{textcolor}{rgb}{0.000000,0.000000,0.000000}%
\pgfsetstrokecolor{textcolor}%
\pgfsetfillcolor{textcolor}%
\pgftext[x=4.157880in,y=3.197097in,,base]{\color{textcolor}\rmfamily\fontsize{10.000000}{12.000000}\selectfont (2)}%
\end{pgfscope}%
\begin{pgfscope}%
\pgfsetbuttcap%
\pgfsetmiterjoin%
\definecolor{currentfill}{rgb}{1.000000,1.000000,1.000000}%
\pgfsetfillcolor{currentfill}%
\pgfsetlinewidth{0.000000pt}%
\definecolor{currentstroke}{rgb}{0.000000,0.000000,0.000000}%
\pgfsetstrokecolor{currentstroke}%
\pgfsetstrokeopacity{0.000000}%
\pgfsetdash{}{0pt}%
\pgfpathmoveto{\pgfqpoint{2.918777in}{1.940047in}}%
\pgfpathlineto{\pgfqpoint{5.396984in}{1.940047in}}%
\pgfpathlineto{\pgfqpoint{5.396984in}{2.390199in}}%
\pgfpathlineto{\pgfqpoint{2.918777in}{2.390199in}}%
\pgfpathclose%
\pgfusepath{fill}%
\end{pgfscope}%
\begin{pgfscope}%
\pgfsetbuttcap%
\pgfsetroundjoin%
\definecolor{currentfill}{rgb}{0.000000,0.000000,0.000000}%
\pgfsetfillcolor{currentfill}%
\pgfsetlinewidth{0.803000pt}%
\definecolor{currentstroke}{rgb}{0.000000,0.000000,0.000000}%
\pgfsetstrokecolor{currentstroke}%
\pgfsetdash{}{0pt}%
\pgfsys@defobject{currentmarker}{\pgfqpoint{0.000000in}{-0.048611in}}{\pgfqpoint{0.000000in}{0.000000in}}{%
\pgfpathmoveto{\pgfqpoint{0.000000in}{0.000000in}}%
\pgfpathlineto{\pgfqpoint{0.000000in}{-0.048611in}}%
\pgfusepath{stroke,fill}%
}%
\begin{pgfscope}%
\pgfsys@transformshift{2.918777in}{1.940047in}%
\pgfsys@useobject{currentmarker}{}%
\end{pgfscope}%
\end{pgfscope}%
\begin{pgfscope}%
\pgfsetbuttcap%
\pgfsetroundjoin%
\definecolor{currentfill}{rgb}{0.000000,0.000000,0.000000}%
\pgfsetfillcolor{currentfill}%
\pgfsetlinewidth{0.803000pt}%
\definecolor{currentstroke}{rgb}{0.000000,0.000000,0.000000}%
\pgfsetstrokecolor{currentstroke}%
\pgfsetdash{}{0pt}%
\pgfsys@defobject{currentmarker}{\pgfqpoint{0.000000in}{-0.048611in}}{\pgfqpoint{0.000000in}{0.000000in}}{%
\pgfpathmoveto{\pgfqpoint{0.000000in}{0.000000in}}%
\pgfpathlineto{\pgfqpoint{0.000000in}{-0.048611in}}%
\pgfusepath{stroke,fill}%
}%
\begin{pgfscope}%
\pgfsys@transformshift{3.331811in}{1.940047in}%
\pgfsys@useobject{currentmarker}{}%
\end{pgfscope}%
\end{pgfscope}%
\begin{pgfscope}%
\pgfsetbuttcap%
\pgfsetroundjoin%
\definecolor{currentfill}{rgb}{0.000000,0.000000,0.000000}%
\pgfsetfillcolor{currentfill}%
\pgfsetlinewidth{0.803000pt}%
\definecolor{currentstroke}{rgb}{0.000000,0.000000,0.000000}%
\pgfsetstrokecolor{currentstroke}%
\pgfsetdash{}{0pt}%
\pgfsys@defobject{currentmarker}{\pgfqpoint{0.000000in}{-0.048611in}}{\pgfqpoint{0.000000in}{0.000000in}}{%
\pgfpathmoveto{\pgfqpoint{0.000000in}{0.000000in}}%
\pgfpathlineto{\pgfqpoint{0.000000in}{-0.048611in}}%
\pgfusepath{stroke,fill}%
}%
\begin{pgfscope}%
\pgfsys@transformshift{3.744846in}{1.940047in}%
\pgfsys@useobject{currentmarker}{}%
\end{pgfscope}%
\end{pgfscope}%
\begin{pgfscope}%
\pgfsetbuttcap%
\pgfsetroundjoin%
\definecolor{currentfill}{rgb}{0.000000,0.000000,0.000000}%
\pgfsetfillcolor{currentfill}%
\pgfsetlinewidth{0.803000pt}%
\definecolor{currentstroke}{rgb}{0.000000,0.000000,0.000000}%
\pgfsetstrokecolor{currentstroke}%
\pgfsetdash{}{0pt}%
\pgfsys@defobject{currentmarker}{\pgfqpoint{0.000000in}{-0.048611in}}{\pgfqpoint{0.000000in}{0.000000in}}{%
\pgfpathmoveto{\pgfqpoint{0.000000in}{0.000000in}}%
\pgfpathlineto{\pgfqpoint{0.000000in}{-0.048611in}}%
\pgfusepath{stroke,fill}%
}%
\begin{pgfscope}%
\pgfsys@transformshift{4.157880in}{1.940047in}%
\pgfsys@useobject{currentmarker}{}%
\end{pgfscope}%
\end{pgfscope}%
\begin{pgfscope}%
\pgfsetbuttcap%
\pgfsetroundjoin%
\definecolor{currentfill}{rgb}{0.000000,0.000000,0.000000}%
\pgfsetfillcolor{currentfill}%
\pgfsetlinewidth{0.803000pt}%
\definecolor{currentstroke}{rgb}{0.000000,0.000000,0.000000}%
\pgfsetstrokecolor{currentstroke}%
\pgfsetdash{}{0pt}%
\pgfsys@defobject{currentmarker}{\pgfqpoint{0.000000in}{-0.048611in}}{\pgfqpoint{0.000000in}{0.000000in}}{%
\pgfpathmoveto{\pgfqpoint{0.000000in}{0.000000in}}%
\pgfpathlineto{\pgfqpoint{0.000000in}{-0.048611in}}%
\pgfusepath{stroke,fill}%
}%
\begin{pgfscope}%
\pgfsys@transformshift{4.570915in}{1.940047in}%
\pgfsys@useobject{currentmarker}{}%
\end{pgfscope}%
\end{pgfscope}%
\begin{pgfscope}%
\pgfsetbuttcap%
\pgfsetroundjoin%
\definecolor{currentfill}{rgb}{0.000000,0.000000,0.000000}%
\pgfsetfillcolor{currentfill}%
\pgfsetlinewidth{0.803000pt}%
\definecolor{currentstroke}{rgb}{0.000000,0.000000,0.000000}%
\pgfsetstrokecolor{currentstroke}%
\pgfsetdash{}{0pt}%
\pgfsys@defobject{currentmarker}{\pgfqpoint{0.000000in}{-0.048611in}}{\pgfqpoint{0.000000in}{0.000000in}}{%
\pgfpathmoveto{\pgfqpoint{0.000000in}{0.000000in}}%
\pgfpathlineto{\pgfqpoint{0.000000in}{-0.048611in}}%
\pgfusepath{stroke,fill}%
}%
\begin{pgfscope}%
\pgfsys@transformshift{4.983949in}{1.940047in}%
\pgfsys@useobject{currentmarker}{}%
\end{pgfscope}%
\end{pgfscope}%
\begin{pgfscope}%
\pgfsetbuttcap%
\pgfsetroundjoin%
\definecolor{currentfill}{rgb}{0.000000,0.000000,0.000000}%
\pgfsetfillcolor{currentfill}%
\pgfsetlinewidth{0.803000pt}%
\definecolor{currentstroke}{rgb}{0.000000,0.000000,0.000000}%
\pgfsetstrokecolor{currentstroke}%
\pgfsetdash{}{0pt}%
\pgfsys@defobject{currentmarker}{\pgfqpoint{0.000000in}{-0.048611in}}{\pgfqpoint{0.000000in}{0.000000in}}{%
\pgfpathmoveto{\pgfqpoint{0.000000in}{0.000000in}}%
\pgfpathlineto{\pgfqpoint{0.000000in}{-0.048611in}}%
\pgfusepath{stroke,fill}%
}%
\begin{pgfscope}%
\pgfsys@transformshift{5.396984in}{1.940047in}%
\pgfsys@useobject{currentmarker}{}%
\end{pgfscope}%
\end{pgfscope}%
\begin{pgfscope}%
\pgfpathrectangle{\pgfqpoint{2.918777in}{1.940047in}}{\pgfqpoint{2.478207in}{0.450152in}}%
\pgfusepath{clip}%
\pgfsetbuttcap%
\pgfsetroundjoin%
\pgfsetlinewidth{0.752812pt}%
\definecolor{currentstroke}{rgb}{0.000000,0.000000,0.000000}%
\pgfsetstrokecolor{currentstroke}%
\pgfsetdash{{2.775000pt}{1.200000pt}}{0.000000pt}%
\pgfpathmoveto{\pgfqpoint{3.331811in}{1.940047in}}%
\pgfpathlineto{\pgfqpoint{3.331811in}{2.390199in}}%
\pgfusepath{stroke}%
\end{pgfscope}%
\begin{pgfscope}%
\pgfpathrectangle{\pgfqpoint{2.918777in}{1.940047in}}{\pgfqpoint{2.478207in}{0.450152in}}%
\pgfusepath{clip}%
\pgfsetbuttcap%
\pgfsetroundjoin%
\pgfsetlinewidth{0.752812pt}%
\definecolor{currentstroke}{rgb}{0.000000,0.000000,0.000000}%
\pgfsetstrokecolor{currentstroke}%
\pgfsetdash{{2.775000pt}{1.200000pt}}{0.000000pt}%
\pgfpathmoveto{\pgfqpoint{4.983949in}{1.940047in}}%
\pgfpathlineto{\pgfqpoint{4.983949in}{2.390199in}}%
\pgfusepath{stroke}%
\end{pgfscope}%
\begin{pgfscope}%
\pgfpathrectangle{\pgfqpoint{2.918777in}{1.940047in}}{\pgfqpoint{2.478207in}{0.450152in}}%
\pgfusepath{clip}%
\pgfsetbuttcap%
\pgfsetroundjoin%
\pgfsetlinewidth{1.505625pt}%
\definecolor{currentstroke}{rgb}{0.631373,0.062745,0.207843}%
\pgfsetstrokecolor{currentstroke}%
\pgfsetdash{{1.500000pt}{2.475000pt}}{0.000000pt}%
\pgfpathmoveto{\pgfqpoint{3.873031in}{1.940047in}}%
\pgfpathlineto{\pgfqpoint{3.873031in}{2.390199in}}%
\pgfusepath{stroke}%
\end{pgfscope}%
\begin{pgfscope}%
\pgfpathrectangle{\pgfqpoint{2.918777in}{1.940047in}}{\pgfqpoint{2.478207in}{0.450152in}}%
\pgfusepath{clip}%
\pgfsetrectcap%
\pgfsetroundjoin%
\pgfsetlinewidth{0.752812pt}%
\definecolor{currentstroke}{rgb}{0.964706,0.658824,0.000000}%
\pgfsetstrokecolor{currentstroke}%
\pgfsetdash{}{0pt}%
\pgfpathmoveto{\pgfqpoint{2.918777in}{2.225826in}}%
\pgfpathlineto{\pgfqpoint{2.943809in}{2.225824in}}%
\pgfpathlineto{\pgfqpoint{2.968842in}{2.225820in}}%
\pgfpathlineto{\pgfqpoint{2.993874in}{2.225811in}}%
\pgfpathlineto{\pgfqpoint{3.018906in}{2.225791in}}%
\pgfpathlineto{\pgfqpoint{3.043939in}{2.225749in}}%
\pgfpathlineto{\pgfqpoint{3.068971in}{2.225662in}}%
\pgfpathlineto{\pgfqpoint{3.094004in}{2.225492in}}%
\pgfpathlineto{\pgfqpoint{3.119036in}{2.225174in}}%
\pgfpathlineto{\pgfqpoint{3.144068in}{2.224609in}}%
\pgfpathlineto{\pgfqpoint{3.169101in}{2.223651in}}%
\pgfpathlineto{\pgfqpoint{3.194133in}{2.222110in}}%
\pgfpathlineto{\pgfqpoint{3.219166in}{2.219770in}}%
\pgfpathlineto{\pgfqpoint{3.244198in}{2.216431in}}%
\pgfpathlineto{\pgfqpoint{3.269230in}{2.211998in}}%
\pgfpathlineto{\pgfqpoint{3.294263in}{2.206586in}}%
\pgfpathlineto{\pgfqpoint{3.319295in}{2.200635in}}%
\pgfpathlineto{\pgfqpoint{3.344327in}{2.194970in}}%
\pgfpathlineto{\pgfqpoint{3.369360in}{2.190767in}}%
\pgfpathlineto{\pgfqpoint{3.394392in}{2.189386in}}%
\pgfpathlineto{\pgfqpoint{3.419425in}{2.192074in}}%
\pgfpathlineto{\pgfqpoint{3.444457in}{2.199602in}}%
\pgfpathlineto{\pgfqpoint{3.469489in}{2.211887in}}%
\pgfpathlineto{\pgfqpoint{3.494522in}{2.227471in}}%
\pgfpathlineto{\pgfqpoint{3.519554in}{2.232771in}}%
\pgfpathlineto{\pgfqpoint{3.544587in}{2.193506in}}%
\pgfpathlineto{\pgfqpoint{3.569619in}{2.153340in}}%
\pgfpathlineto{\pgfqpoint{3.594651in}{2.117980in}}%
\pgfpathlineto{\pgfqpoint{3.619684in}{2.089198in}}%
\pgfpathlineto{\pgfqpoint{3.644716in}{2.068460in}}%
\pgfpathlineto{\pgfqpoint{3.669749in}{2.057064in}}%
\pgfpathlineto{\pgfqpoint{3.694781in}{2.056005in}}%
\pgfpathlineto{\pgfqpoint{3.719813in}{2.065643in}}%
\pgfpathlineto{\pgfqpoint{3.744846in}{2.085012in}}%
\pgfpathlineto{\pgfqpoint{3.769878in}{2.099444in}}%
\pgfpathlineto{\pgfqpoint{3.794911in}{2.077412in}}%
\pgfpathlineto{\pgfqpoint{3.819943in}{2.059324in}}%
\pgfpathlineto{\pgfqpoint{3.844975in}{2.048838in}}%
\pgfpathlineto{\pgfqpoint{3.870008in}{2.045260in}}%
\pgfpathlineto{\pgfqpoint{3.895040in}{2.046957in}}%
\pgfpathlineto{\pgfqpoint{3.920072in}{2.051979in}}%
\pgfpathlineto{\pgfqpoint{3.945105in}{2.058483in}}%
\pgfpathlineto{\pgfqpoint{3.970137in}{2.064967in}}%
\pgfpathlineto{\pgfqpoint{3.995170in}{2.070327in}}%
\pgfpathlineto{\pgfqpoint{4.020202in}{2.073651in}}%
\pgfpathlineto{\pgfqpoint{4.045234in}{2.072501in}}%
\pgfpathlineto{\pgfqpoint{4.070267in}{2.060791in}}%
\pgfpathlineto{\pgfqpoint{4.095299in}{2.055496in}}%
\pgfpathlineto{\pgfqpoint{4.120332in}{2.065554in}}%
\pgfpathlineto{\pgfqpoint{4.145364in}{2.077702in}}%
\pgfpathlineto{\pgfqpoint{4.170396in}{2.083826in}}%
\pgfpathlineto{\pgfqpoint{4.195429in}{2.091535in}}%
\pgfpathlineto{\pgfqpoint{4.220461in}{2.102342in}}%
\pgfpathlineto{\pgfqpoint{4.245494in}{2.116150in}}%
\pgfpathlineto{\pgfqpoint{4.270526in}{2.132145in}}%
\pgfpathlineto{\pgfqpoint{4.295558in}{2.149088in}}%
\pgfpathlineto{\pgfqpoint{4.320591in}{2.165609in}}%
\pgfpathlineto{\pgfqpoint{4.345623in}{2.180492in}}%
\pgfpathlineto{\pgfqpoint{4.370655in}{2.192888in}}%
\pgfpathlineto{\pgfqpoint{4.395688in}{2.202436in}}%
\pgfpathlineto{\pgfqpoint{4.420720in}{2.209268in}}%
\pgfpathlineto{\pgfqpoint{4.445753in}{2.213965in}}%
\pgfpathlineto{\pgfqpoint{4.470785in}{2.217474in}}%
\pgfpathlineto{\pgfqpoint{4.495817in}{2.221024in}}%
\pgfpathlineto{\pgfqpoint{4.520850in}{2.226026in}}%
\pgfpathlineto{\pgfqpoint{4.545882in}{2.233921in}}%
\pgfpathlineto{\pgfqpoint{4.570915in}{2.245984in}}%
\pgfpathlineto{\pgfqpoint{4.595947in}{2.263087in}}%
\pgfpathlineto{\pgfqpoint{4.620979in}{2.285471in}}%
\pgfpathlineto{\pgfqpoint{4.646012in}{2.312525in}}%
\pgfpathlineto{\pgfqpoint{4.671044in}{2.341349in}}%
\pgfpathlineto{\pgfqpoint{4.696077in}{2.332844in}}%
\pgfpathlineto{\pgfqpoint{4.721109in}{2.303804in}}%
\pgfpathlineto{\pgfqpoint{4.746141in}{2.278120in}}%
\pgfpathlineto{\pgfqpoint{4.771174in}{2.257472in}}%
\pgfpathlineto{\pgfqpoint{4.796206in}{2.242234in}}%
\pgfpathlineto{\pgfqpoint{4.821238in}{2.232051in}}%
\pgfpathlineto{\pgfqpoint{4.846271in}{2.226071in}}%
\pgfpathlineto{\pgfqpoint{4.871303in}{2.223199in}}%
\pgfpathlineto{\pgfqpoint{4.896336in}{2.222339in}}%
\pgfpathlineto{\pgfqpoint{4.921368in}{2.222578in}}%
\pgfpathlineto{\pgfqpoint{4.946400in}{2.223266in}}%
\pgfpathlineto{\pgfqpoint{4.971433in}{2.224017in}}%
\pgfpathlineto{\pgfqpoint{4.996465in}{2.224648in}}%
\pgfpathlineto{\pgfqpoint{5.021498in}{2.225109in}}%
\pgfpathlineto{\pgfqpoint{5.046530in}{2.225414in}}%
\pgfpathlineto{\pgfqpoint{5.071562in}{2.225601in}}%
\pgfpathlineto{\pgfqpoint{5.096595in}{2.225709in}}%
\pgfpathlineto{\pgfqpoint{5.121627in}{2.225768in}}%
\pgfpathlineto{\pgfqpoint{5.146660in}{2.225799in}}%
\pgfpathlineto{\pgfqpoint{5.171692in}{2.225814in}}%
\pgfpathlineto{\pgfqpoint{5.196724in}{2.225821in}}%
\pgfpathlineto{\pgfqpoint{5.221757in}{2.225824in}}%
\pgfpathlineto{\pgfqpoint{5.246789in}{2.225826in}}%
\pgfpathlineto{\pgfqpoint{5.271822in}{2.225826in}}%
\pgfpathlineto{\pgfqpoint{5.296854in}{2.225827in}}%
\pgfpathlineto{\pgfqpoint{5.321886in}{2.225827in}}%
\pgfpathlineto{\pgfqpoint{5.346919in}{2.225827in}}%
\pgfpathlineto{\pgfqpoint{5.371951in}{2.225827in}}%
\pgfpathlineto{\pgfqpoint{5.396984in}{2.225827in}}%
\pgfusepath{stroke}%
\end{pgfscope}%
\begin{pgfscope}%
\pgfsetrectcap%
\pgfsetmiterjoin%
\pgfsetlinewidth{0.752812pt}%
\definecolor{currentstroke}{rgb}{0.000000,0.000000,0.000000}%
\pgfsetstrokecolor{currentstroke}%
\pgfsetdash{}{0pt}%
\pgfpathmoveto{\pgfqpoint{2.918777in}{1.940047in}}%
\pgfpathlineto{\pgfqpoint{2.918777in}{2.390199in}}%
\pgfusepath{stroke}%
\end{pgfscope}%
\begin{pgfscope}%
\pgfsetrectcap%
\pgfsetmiterjoin%
\pgfsetlinewidth{0.752812pt}%
\definecolor{currentstroke}{rgb}{0.000000,0.000000,0.000000}%
\pgfsetstrokecolor{currentstroke}%
\pgfsetdash{}{0pt}%
\pgfpathmoveto{\pgfqpoint{5.396984in}{1.940047in}}%
\pgfpathlineto{\pgfqpoint{5.396984in}{2.390199in}}%
\pgfusepath{stroke}%
\end{pgfscope}%
\begin{pgfscope}%
\pgfsetrectcap%
\pgfsetmiterjoin%
\pgfsetlinewidth{0.752812pt}%
\definecolor{currentstroke}{rgb}{0.000000,0.000000,0.000000}%
\pgfsetstrokecolor{currentstroke}%
\pgfsetdash{}{0pt}%
\pgfpathmoveto{\pgfqpoint{2.918777in}{1.940047in}}%
\pgfpathlineto{\pgfqpoint{5.396984in}{1.940047in}}%
\pgfusepath{stroke}%
\end{pgfscope}%
\begin{pgfscope}%
\pgfsetrectcap%
\pgfsetmiterjoin%
\pgfsetlinewidth{0.752812pt}%
\definecolor{currentstroke}{rgb}{0.000000,0.000000,0.000000}%
\pgfsetstrokecolor{currentstroke}%
\pgfsetdash{}{0pt}%
\pgfpathmoveto{\pgfqpoint{2.918777in}{2.390199in}}%
\pgfpathlineto{\pgfqpoint{5.396984in}{2.390199in}}%
\pgfusepath{stroke}%
\end{pgfscope}%
\begin{pgfscope}%
\pgfsetbuttcap%
\pgfsetmiterjoin%
\definecolor{currentfill}{rgb}{1.000000,1.000000,1.000000}%
\pgfsetfillcolor{currentfill}%
\pgfsetlinewidth{0.000000pt}%
\definecolor{currentstroke}{rgb}{0.000000,0.000000,0.000000}%
\pgfsetstrokecolor{currentstroke}%
\pgfsetstrokeopacity{0.000000}%
\pgfsetdash{}{0pt}%
\pgfpathmoveto{\pgfqpoint{0.220285in}{0.881091in}}%
\pgfpathlineto{\pgfqpoint{2.698492in}{0.881091in}}%
\pgfpathlineto{\pgfqpoint{2.698492in}{1.803903in}}%
\pgfpathlineto{\pgfqpoint{0.220285in}{1.803903in}}%
\pgfpathclose%
\pgfusepath{fill}%
\end{pgfscope}%
\begin{pgfscope}%
\pgfpathrectangle{\pgfqpoint{0.220285in}{0.881091in}}{\pgfqpoint{2.478207in}{0.922812in}}%
\pgfusepath{clip}%
\pgfsetbuttcap%
\pgfsetroundjoin%
\definecolor{currentfill}{rgb}{0.556863,0.729412,0.898039}%
\pgfsetfillcolor{currentfill}%
\pgfsetfillopacity{0.700000}%
\pgfsetlinewidth{0.000000pt}%
\definecolor{currentstroke}{rgb}{0.556863,0.729412,0.898039}%
\pgfsetstrokecolor{currentstroke}%
\pgfsetstrokeopacity{0.700000}%
\pgfsetdash{}{0pt}%
\pgfsys@defobject{currentmarker}{\pgfqpoint{0.220285in}{1.021281in}}{\pgfqpoint{2.698492in}{1.595220in}}{%
\pgfpathmoveto{\pgfqpoint{0.220285in}{1.595219in}}%
\pgfpathlineto{\pgfqpoint{0.220285in}{1.089774in}}%
\pgfpathlineto{\pgfqpoint{0.245317in}{1.089773in}}%
\pgfpathlineto{\pgfqpoint{0.270350in}{1.089770in}}%
\pgfpathlineto{\pgfqpoint{0.295382in}{1.089764in}}%
\pgfpathlineto{\pgfqpoint{0.320415in}{1.089750in}}%
\pgfpathlineto{\pgfqpoint{0.345447in}{1.089720in}}%
\pgfpathlineto{\pgfqpoint{0.370479in}{1.089658in}}%
\pgfpathlineto{\pgfqpoint{0.395512in}{1.089539in}}%
\pgfpathlineto{\pgfqpoint{0.420544in}{1.089317in}}%
\pgfpathlineto{\pgfqpoint{0.445577in}{1.088926in}}%
\pgfpathlineto{\pgfqpoint{0.470609in}{1.088270in}}%
\pgfpathlineto{\pgfqpoint{0.495641in}{1.087240in}}%
\pgfpathlineto{\pgfqpoint{0.520674in}{1.085734in}}%
\pgfpathlineto{\pgfqpoint{0.545706in}{1.083724in}}%
\pgfpathlineto{\pgfqpoint{0.570739in}{1.081356in}}%
\pgfpathlineto{\pgfqpoint{0.595771in}{1.079069in}}%
\pgfpathlineto{\pgfqpoint{0.620803in}{1.077708in}}%
\pgfpathlineto{\pgfqpoint{0.645836in}{1.078550in}}%
\pgfpathlineto{\pgfqpoint{0.670868in}{1.083192in}}%
\pgfpathlineto{\pgfqpoint{0.695901in}{1.093274in}}%
\pgfpathlineto{\pgfqpoint{0.720933in}{1.110053in}}%
\pgfpathlineto{\pgfqpoint{0.745965in}{1.133932in}}%
\pgfpathlineto{\pgfqpoint{0.770998in}{1.164001in}}%
\pgfpathlineto{\pgfqpoint{0.796030in}{1.197389in}}%
\pgfpathlineto{\pgfqpoint{0.821062in}{1.213224in}}%
\pgfpathlineto{\pgfqpoint{0.846095in}{1.161880in}}%
\pgfpathlineto{\pgfqpoint{0.871127in}{1.109900in}}%
\pgfpathlineto{\pgfqpoint{0.896160in}{1.067109in}}%
\pgfpathlineto{\pgfqpoint{0.921192in}{1.036904in}}%
\pgfpathlineto{\pgfqpoint{0.946224in}{1.021397in}}%
\pgfpathlineto{\pgfqpoint{0.971257in}{1.021281in}}%
\pgfpathlineto{\pgfqpoint{0.996289in}{1.035536in}}%
\pgfpathlineto{\pgfqpoint{1.021322in}{1.061169in}}%
\pgfpathlineto{\pgfqpoint{1.046354in}{1.092621in}}%
\pgfpathlineto{\pgfqpoint{1.071386in}{1.114632in}}%
\pgfpathlineto{\pgfqpoint{1.096419in}{1.118774in}}%
\pgfpathlineto{\pgfqpoint{1.121451in}{1.131945in}}%
\pgfpathlineto{\pgfqpoint{1.146484in}{1.155008in}}%
\pgfpathlineto{\pgfqpoint{1.171516in}{1.176713in}}%
\pgfpathlineto{\pgfqpoint{1.196548in}{1.174584in}}%
\pgfpathlineto{\pgfqpoint{1.221581in}{1.157239in}}%
\pgfpathlineto{\pgfqpoint{1.246613in}{1.136729in}}%
\pgfpathlineto{\pgfqpoint{1.271646in}{1.118531in}}%
\pgfpathlineto{\pgfqpoint{1.296678in}{1.105595in}}%
\pgfpathlineto{\pgfqpoint{1.321710in}{1.098146in}}%
\pgfpathlineto{\pgfqpoint{1.346743in}{1.092135in}}%
\pgfpathlineto{\pgfqpoint{1.371775in}{1.080309in}}%
\pgfpathlineto{\pgfqpoint{1.396807in}{1.074838in}}%
\pgfpathlineto{\pgfqpoint{1.421840in}{1.084737in}}%
\pgfpathlineto{\pgfqpoint{1.446872in}{1.093513in}}%
\pgfpathlineto{\pgfqpoint{1.471905in}{1.089085in}}%
\pgfpathlineto{\pgfqpoint{1.496937in}{1.080675in}}%
\pgfpathlineto{\pgfqpoint{1.521969in}{1.072068in}}%
\pgfpathlineto{\pgfqpoint{1.547002in}{1.065379in}}%
\pgfpathlineto{\pgfqpoint{1.572034in}{1.061771in}}%
\pgfpathlineto{\pgfqpoint{1.597067in}{1.061455in}}%
\pgfpathlineto{\pgfqpoint{1.622099in}{1.063844in}}%
\pgfpathlineto{\pgfqpoint{1.647131in}{1.067895in}}%
\pgfpathlineto{\pgfqpoint{1.672164in}{1.072505in}}%
\pgfpathlineto{\pgfqpoint{1.697196in}{1.076852in}}%
\pgfpathlineto{\pgfqpoint{1.722228in}{1.080612in}}%
\pgfpathlineto{\pgfqpoint{1.747261in}{1.084043in}}%
\pgfpathlineto{\pgfqpoint{1.772293in}{1.087978in}}%
\pgfpathlineto{\pgfqpoint{1.797326in}{1.093746in}}%
\pgfpathlineto{\pgfqpoint{1.822358in}{1.103020in}}%
\pgfpathlineto{\pgfqpoint{1.847390in}{1.117572in}}%
\pgfpathlineto{\pgfqpoint{1.872423in}{1.138943in}}%
\pgfpathlineto{\pgfqpoint{1.897455in}{1.168073in}}%
\pgfpathlineto{\pgfqpoint{1.922488in}{1.204959in}}%
\pgfpathlineto{\pgfqpoint{1.947520in}{1.248341in}}%
\pgfpathlineto{\pgfqpoint{1.972552in}{1.293523in}}%
\pgfpathlineto{\pgfqpoint{1.997585in}{1.280274in}}%
\pgfpathlineto{\pgfqpoint{2.022617in}{1.234427in}}%
\pgfpathlineto{\pgfqpoint{2.047650in}{1.192852in}}%
\pgfpathlineto{\pgfqpoint{2.072682in}{1.158366in}}%
\pgfpathlineto{\pgfqpoint{2.097714in}{1.131828in}}%
\pgfpathlineto{\pgfqpoint{2.122747in}{1.112980in}}%
\pgfpathlineto{\pgfqpoint{2.147779in}{1.100757in}}%
\pgfpathlineto{\pgfqpoint{2.172812in}{1.093655in}}%
\pgfpathlineto{\pgfqpoint{2.197844in}{1.090090in}}%
\pgfpathlineto{\pgfqpoint{2.222876in}{1.088688in}}%
\pgfpathlineto{\pgfqpoint{2.247909in}{1.088420in}}%
\pgfpathlineto{\pgfqpoint{2.272941in}{1.088630in}}%
\pgfpathlineto{\pgfqpoint{2.297973in}{1.088960in}}%
\pgfpathlineto{\pgfqpoint{2.323006in}{1.089254in}}%
\pgfpathlineto{\pgfqpoint{2.348038in}{1.089467in}}%
\pgfpathlineto{\pgfqpoint{2.373071in}{1.089604in}}%
\pgfpathlineto{\pgfqpoint{2.398103in}{1.089685in}}%
\pgfpathlineto{\pgfqpoint{2.423135in}{1.089730in}}%
\pgfpathlineto{\pgfqpoint{2.448168in}{1.089753in}}%
\pgfpathlineto{\pgfqpoint{2.473200in}{1.089765in}}%
\pgfpathlineto{\pgfqpoint{2.498233in}{1.089770in}}%
\pgfpathlineto{\pgfqpoint{2.523265in}{1.089773in}}%
\pgfpathlineto{\pgfqpoint{2.548297in}{1.089774in}}%
\pgfpathlineto{\pgfqpoint{2.573330in}{1.089774in}}%
\pgfpathlineto{\pgfqpoint{2.598362in}{1.089774in}}%
\pgfpathlineto{\pgfqpoint{2.623395in}{1.089774in}}%
\pgfpathlineto{\pgfqpoint{2.648427in}{1.089775in}}%
\pgfpathlineto{\pgfqpoint{2.673459in}{1.089775in}}%
\pgfpathlineto{\pgfqpoint{2.698492in}{1.089775in}}%
\pgfpathlineto{\pgfqpoint{2.698492in}{1.595220in}}%
\pgfpathlineto{\pgfqpoint{2.698492in}{1.595220in}}%
\pgfpathlineto{\pgfqpoint{2.673459in}{1.595220in}}%
\pgfpathlineto{\pgfqpoint{2.648427in}{1.595220in}}%
\pgfpathlineto{\pgfqpoint{2.623395in}{1.595220in}}%
\pgfpathlineto{\pgfqpoint{2.598362in}{1.595220in}}%
\pgfpathlineto{\pgfqpoint{2.573330in}{1.595219in}}%
\pgfpathlineto{\pgfqpoint{2.548297in}{1.595219in}}%
\pgfpathlineto{\pgfqpoint{2.523265in}{1.595218in}}%
\pgfpathlineto{\pgfqpoint{2.498233in}{1.595215in}}%
\pgfpathlineto{\pgfqpoint{2.473200in}{1.595210in}}%
\pgfpathlineto{\pgfqpoint{2.448168in}{1.595198in}}%
\pgfpathlineto{\pgfqpoint{2.423135in}{1.595174in}}%
\pgfpathlineto{\pgfqpoint{2.398103in}{1.595127in}}%
\pgfpathlineto{\pgfqpoint{2.373071in}{1.595038in}}%
\pgfpathlineto{\pgfqpoint{2.348038in}{1.594873in}}%
\pgfpathlineto{\pgfqpoint{2.323006in}{1.594574in}}%
\pgfpathlineto{\pgfqpoint{2.297973in}{1.594039in}}%
\pgfpathlineto{\pgfqpoint{2.272941in}{1.593088in}}%
\pgfpathlineto{\pgfqpoint{2.247909in}{1.591409in}}%
\pgfpathlineto{\pgfqpoint{2.222876in}{1.588489in}}%
\pgfpathlineto{\pgfqpoint{2.197844in}{1.583545in}}%
\pgfpathlineto{\pgfqpoint{2.172812in}{1.575495in}}%
\pgfpathlineto{\pgfqpoint{2.147779in}{1.563020in}}%
\pgfpathlineto{\pgfqpoint{2.122747in}{1.544739in}}%
\pgfpathlineto{\pgfqpoint{2.097714in}{1.519505in}}%
\pgfpathlineto{\pgfqpoint{2.072682in}{1.486748in}}%
\pgfpathlineto{\pgfqpoint{2.047650in}{1.446781in}}%
\pgfpathlineto{\pgfqpoint{2.022617in}{1.401036in}}%
\pgfpathlineto{\pgfqpoint{1.997585in}{1.352795in}}%
\pgfpathlineto{\pgfqpoint{1.972552in}{1.339207in}}%
\pgfpathlineto{\pgfqpoint{1.947520in}{1.386137in}}%
\pgfpathlineto{\pgfqpoint{1.922488in}{1.433126in}}%
\pgfpathlineto{\pgfqpoint{1.897455in}{1.475028in}}%
\pgfpathlineto{\pgfqpoint{1.872423in}{1.509972in}}%
\pgfpathlineto{\pgfqpoint{1.847390in}{1.537262in}}%
\pgfpathlineto{\pgfqpoint{1.822358in}{1.557128in}}%
\pgfpathlineto{\pgfqpoint{1.797326in}{1.570404in}}%
\pgfpathlineto{\pgfqpoint{1.772293in}{1.578153in}}%
\pgfpathlineto{\pgfqpoint{1.747261in}{1.581280in}}%
\pgfpathlineto{\pgfqpoint{1.722228in}{1.580239in}}%
\pgfpathlineto{\pgfqpoint{1.697196in}{1.574837in}}%
\pgfpathlineto{\pgfqpoint{1.672164in}{1.564179in}}%
\pgfpathlineto{\pgfqpoint{1.647131in}{1.546761in}}%
\pgfpathlineto{\pgfqpoint{1.622099in}{1.520757in}}%
\pgfpathlineto{\pgfqpoint{1.597067in}{1.484525in}}%
\pgfpathlineto{\pgfqpoint{1.572034in}{1.437278in}}%
\pgfpathlineto{\pgfqpoint{1.547002in}{1.379784in}}%
\pgfpathlineto{\pgfqpoint{1.521969in}{1.314888in}}%
\pgfpathlineto{\pgfqpoint{1.496937in}{1.247637in}}%
\pgfpathlineto{\pgfqpoint{1.471905in}{1.184971in}}%
\pgfpathlineto{\pgfqpoint{1.446872in}{1.135855in}}%
\pgfpathlineto{\pgfqpoint{1.421840in}{1.114232in}}%
\pgfpathlineto{\pgfqpoint{1.396807in}{1.111376in}}%
\pgfpathlineto{\pgfqpoint{1.371775in}{1.111989in}}%
\pgfpathlineto{\pgfqpoint{1.346743in}{1.123610in}}%
\pgfpathlineto{\pgfqpoint{1.321710in}{1.154225in}}%
\pgfpathlineto{\pgfqpoint{1.296678in}{1.190188in}}%
\pgfpathlineto{\pgfqpoint{1.271646in}{1.219868in}}%
\pgfpathlineto{\pgfqpoint{1.246613in}{1.236065in}}%
\pgfpathlineto{\pgfqpoint{1.221581in}{1.235882in}}%
\pgfpathlineto{\pgfqpoint{1.196548in}{1.221700in}}%
\pgfpathlineto{\pgfqpoint{1.171516in}{1.205842in}}%
\pgfpathlineto{\pgfqpoint{1.146484in}{1.200484in}}%
\pgfpathlineto{\pgfqpoint{1.121451in}{1.189173in}}%
\pgfpathlineto{\pgfqpoint{1.096419in}{1.167812in}}%
\pgfpathlineto{\pgfqpoint{1.071386in}{1.144002in}}%
\pgfpathlineto{\pgfqpoint{1.046354in}{1.149622in}}%
\pgfpathlineto{\pgfqpoint{1.021322in}{1.178639in}}%
\pgfpathlineto{\pgfqpoint{0.996289in}{1.215474in}}%
\pgfpathlineto{\pgfqpoint{0.971257in}{1.252054in}}%
\pgfpathlineto{\pgfqpoint{0.946224in}{1.281633in}}%
\pgfpathlineto{\pgfqpoint{0.921192in}{1.299281in}}%
\pgfpathlineto{\pgfqpoint{0.896160in}{1.302511in}}%
\pgfpathlineto{\pgfqpoint{0.871127in}{1.291477in}}%
\pgfpathlineto{\pgfqpoint{0.846095in}{1.268891in}}%
\pgfpathlineto{\pgfqpoint{0.821062in}{1.244848in}}%
\pgfpathlineto{\pgfqpoint{0.796030in}{1.286619in}}%
\pgfpathlineto{\pgfqpoint{0.770998in}{1.345279in}}%
\pgfpathlineto{\pgfqpoint{0.745965in}{1.400296in}}%
\pgfpathlineto{\pgfqpoint{0.720933in}{1.448663in}}%
\pgfpathlineto{\pgfqpoint{0.695901in}{1.488938in}}%
\pgfpathlineto{\pgfqpoint{0.670868in}{1.520790in}}%
\pgfpathlineto{\pgfqpoint{0.645836in}{1.544773in}}%
\pgfpathlineto{\pgfqpoint{0.620803in}{1.562028in}}%
\pgfpathlineto{\pgfqpoint{0.595771in}{1.573950in}}%
\pgfpathlineto{\pgfqpoint{0.570739in}{1.581909in}}%
\pgfpathlineto{\pgfqpoint{0.545706in}{1.587076in}}%
\pgfpathlineto{\pgfqpoint{0.520674in}{1.590352in}}%
\pgfpathlineto{\pgfqpoint{0.495641in}{1.592384in}}%
\pgfpathlineto{\pgfqpoint{0.470609in}{1.593614in}}%
\pgfpathlineto{\pgfqpoint{0.445577in}{1.594340in}}%
\pgfpathlineto{\pgfqpoint{0.420544in}{1.594754in}}%
\pgfpathlineto{\pgfqpoint{0.395512in}{1.594982in}}%
\pgfpathlineto{\pgfqpoint{0.370479in}{1.595103in}}%
\pgfpathlineto{\pgfqpoint{0.345447in}{1.595165in}}%
\pgfpathlineto{\pgfqpoint{0.320415in}{1.595195in}}%
\pgfpathlineto{\pgfqpoint{0.295382in}{1.595209in}}%
\pgfpathlineto{\pgfqpoint{0.270350in}{1.595215in}}%
\pgfpathlineto{\pgfqpoint{0.245317in}{1.595218in}}%
\pgfpathlineto{\pgfqpoint{0.220285in}{1.595219in}}%
\pgfpathclose%
\pgfusepath{fill}%
}%
\begin{pgfscope}%
\pgfsys@transformshift{0.000000in}{0.000000in}%
\pgfsys@useobject{currentmarker}{}%
\end{pgfscope}%
\end{pgfscope}%
\begin{pgfscope}%
\definecolor{textcolor}{rgb}{0.000000,0.000000,0.000000}%
\pgfsetstrokecolor{textcolor}%
\pgfsetfillcolor{textcolor}%
\pgftext[x=0.164729in,y=1.342497in,,bottom,rotate=90.000000]{\color{textcolor}\rmfamily\fontsize{10.000000}{12.000000}\selectfont \(\displaystyle f(x)\)}%
\end{pgfscope}%
\begin{pgfscope}%
\pgfpathrectangle{\pgfqpoint{0.220285in}{0.881091in}}{\pgfqpoint{2.478207in}{0.922812in}}%
\pgfusepath{clip}%
\pgfsetbuttcap%
\pgfsetroundjoin%
\pgfsetlinewidth{0.752812pt}%
\definecolor{currentstroke}{rgb}{0.000000,0.000000,0.000000}%
\pgfsetstrokecolor{currentstroke}%
\pgfsetdash{{2.775000pt}{1.200000pt}}{0.000000pt}%
\pgfpathmoveto{\pgfqpoint{0.633319in}{0.881091in}}%
\pgfpathlineto{\pgfqpoint{0.633319in}{1.803903in}}%
\pgfusepath{stroke}%
\end{pgfscope}%
\begin{pgfscope}%
\pgfpathrectangle{\pgfqpoint{0.220285in}{0.881091in}}{\pgfqpoint{2.478207in}{0.922812in}}%
\pgfusepath{clip}%
\pgfsetbuttcap%
\pgfsetroundjoin%
\pgfsetlinewidth{0.752812pt}%
\definecolor{currentstroke}{rgb}{0.000000,0.000000,0.000000}%
\pgfsetstrokecolor{currentstroke}%
\pgfsetdash{{2.775000pt}{1.200000pt}}{0.000000pt}%
\pgfpathmoveto{\pgfqpoint{2.285457in}{0.881091in}}%
\pgfpathlineto{\pgfqpoint{2.285457in}{1.803903in}}%
\pgfusepath{stroke}%
\end{pgfscope}%
\begin{pgfscope}%
\pgfpathrectangle{\pgfqpoint{0.220285in}{0.881091in}}{\pgfqpoint{2.478207in}{0.922812in}}%
\pgfusepath{clip}%
\pgfsetbuttcap%
\pgfsetroundjoin%
\pgfsetlinewidth{1.505625pt}%
\definecolor{currentstroke}{rgb}{0.631373,0.062745,0.207843}%
\pgfsetstrokecolor{currentstroke}%
\pgfsetdash{{1.500000pt}{2.475000pt}}{0.000000pt}%
\pgfpathmoveto{\pgfqpoint{0.979396in}{0.881091in}}%
\pgfpathlineto{\pgfqpoint{0.979396in}{1.803903in}}%
\pgfusepath{stroke}%
\end{pgfscope}%
\begin{pgfscope}%
\pgfpathrectangle{\pgfqpoint{0.220285in}{0.881091in}}{\pgfqpoint{2.478207in}{0.922812in}}%
\pgfusepath{clip}%
\pgfsetbuttcap%
\pgfsetroundjoin%
\definecolor{currentfill}{rgb}{0.000000,0.000000,0.000000}%
\pgfsetfillcolor{currentfill}%
\pgfsetlinewidth{1.003750pt}%
\definecolor{currentstroke}{rgb}{0.000000,0.000000,0.000000}%
\pgfsetstrokecolor{currentstroke}%
\pgfsetdash{}{0pt}%
\pgfsys@defobject{currentmarker}{\pgfqpoint{-0.020833in}{-0.020833in}}{\pgfqpoint{0.020833in}{0.020833in}}{%
\pgfpathmoveto{\pgfqpoint{0.000000in}{-0.020833in}}%
\pgfpathcurveto{\pgfqpoint{0.005525in}{-0.020833in}}{\pgfqpoint{0.010825in}{-0.018638in}}{\pgfqpoint{0.014731in}{-0.014731in}}%
\pgfpathcurveto{\pgfqpoint{0.018638in}{-0.010825in}}{\pgfqpoint{0.020833in}{-0.005525in}}{\pgfqpoint{0.020833in}{0.000000in}}%
\pgfpathcurveto{\pgfqpoint{0.020833in}{0.005525in}}{\pgfqpoint{0.018638in}{0.010825in}}{\pgfqpoint{0.014731in}{0.014731in}}%
\pgfpathcurveto{\pgfqpoint{0.010825in}{0.018638in}}{\pgfqpoint{0.005525in}{0.020833in}}{\pgfqpoint{0.000000in}{0.020833in}}%
\pgfpathcurveto{\pgfqpoint{-0.005525in}{0.020833in}}{\pgfqpoint{-0.010825in}{0.018638in}}{\pgfqpoint{-0.014731in}{0.014731in}}%
\pgfpathcurveto{\pgfqpoint{-0.018638in}{0.010825in}}{\pgfqpoint{-0.020833in}{0.005525in}}{\pgfqpoint{-0.020833in}{0.000000in}}%
\pgfpathcurveto{\pgfqpoint{-0.020833in}{-0.005525in}}{\pgfqpoint{-0.018638in}{-0.010825in}}{\pgfqpoint{-0.014731in}{-0.014731in}}%
\pgfpathcurveto{\pgfqpoint{-0.010825in}{-0.018638in}}{\pgfqpoint{-0.005525in}{-0.020833in}}{\pgfqpoint{0.000000in}{-0.020833in}}%
\pgfpathclose%
\pgfusepath{stroke,fill}%
}%
\begin{pgfscope}%
\pgfsys@transformshift{0.817804in}{1.230492in}%
\pgfsys@useobject{currentmarker}{}%
\end{pgfscope}%
\begin{pgfscope}%
\pgfsys@transformshift{1.981205in}{1.316216in}%
\pgfsys@useobject{currentmarker}{}%
\end{pgfscope}%
\begin{pgfscope}%
\pgfsys@transformshift{1.067108in}{1.126582in}%
\pgfsys@useobject{currentmarker}{}%
\end{pgfscope}%
\begin{pgfscope}%
\pgfsys@transformshift{1.432752in}{1.104739in}%
\pgfsys@useobject{currentmarker}{}%
\end{pgfscope}%
\begin{pgfscope}%
\pgfsys@transformshift{1.353220in}{1.103542in}%
\pgfsys@useobject{currentmarker}{}%
\end{pgfscope}%
\begin{pgfscope}%
\pgfsys@transformshift{1.174539in}{1.192854in}%
\pgfsys@useobject{currentmarker}{}%
\end{pgfscope}%
\end{pgfscope}%
\begin{pgfscope}%
\pgfpathrectangle{\pgfqpoint{0.220285in}{0.881091in}}{\pgfqpoint{2.478207in}{0.922812in}}%
\pgfusepath{clip}%
\pgfsetrectcap%
\pgfsetroundjoin%
\pgfsetlinewidth{0.752812pt}%
\definecolor{currentstroke}{rgb}{0.000000,0.329412,0.623529}%
\pgfsetstrokecolor{currentstroke}%
\pgfsetdash{}{0pt}%
\pgfpathmoveto{\pgfqpoint{0.220285in}{1.342496in}}%
\pgfpathlineto{\pgfqpoint{0.245317in}{1.342495in}}%
\pgfpathlineto{\pgfqpoint{0.270350in}{1.342493in}}%
\pgfpathlineto{\pgfqpoint{0.295382in}{1.342486in}}%
\pgfpathlineto{\pgfqpoint{0.320415in}{1.342472in}}%
\pgfpathlineto{\pgfqpoint{0.345447in}{1.342442in}}%
\pgfpathlineto{\pgfqpoint{0.370479in}{1.342381in}}%
\pgfpathlineto{\pgfqpoint{0.395512in}{1.342261in}}%
\pgfpathlineto{\pgfqpoint{0.420544in}{1.342036in}}%
\pgfpathlineto{\pgfqpoint{0.445577in}{1.341633in}}%
\pgfpathlineto{\pgfqpoint{0.470609in}{1.340942in}}%
\pgfpathlineto{\pgfqpoint{0.495641in}{1.339812in}}%
\pgfpathlineto{\pgfqpoint{0.520674in}{1.338043in}}%
\pgfpathlineto{\pgfqpoint{0.545706in}{1.335400in}}%
\pgfpathlineto{\pgfqpoint{0.570739in}{1.331633in}}%
\pgfpathlineto{\pgfqpoint{0.595771in}{1.326510in}}%
\pgfpathlineto{\pgfqpoint{0.620803in}{1.319868in}}%
\pgfpathlineto{\pgfqpoint{0.645836in}{1.311662in}}%
\pgfpathlineto{\pgfqpoint{0.670868in}{1.301991in}}%
\pgfpathlineto{\pgfqpoint{0.695901in}{1.291106in}}%
\pgfpathlineto{\pgfqpoint{0.720933in}{1.279358in}}%
\pgfpathlineto{\pgfqpoint{0.745965in}{1.267114in}}%
\pgfpathlineto{\pgfqpoint{0.770998in}{1.254640in}}%
\pgfpathlineto{\pgfqpoint{0.796030in}{1.242004in}}%
\pgfpathlineto{\pgfqpoint{0.821062in}{1.229036in}}%
\pgfpathlineto{\pgfqpoint{0.846095in}{1.215386in}}%
\pgfpathlineto{\pgfqpoint{0.871127in}{1.200689in}}%
\pgfpathlineto{\pgfqpoint{0.896160in}{1.184810in}}%
\pgfpathlineto{\pgfqpoint{0.921192in}{1.168093in}}%
\pgfpathlineto{\pgfqpoint{0.946224in}{1.151515in}}%
\pgfpathlineto{\pgfqpoint{0.971257in}{1.136668in}}%
\pgfpathlineto{\pgfqpoint{0.996289in}{1.125505in}}%
\pgfpathlineto{\pgfqpoint{1.021322in}{1.119904in}}%
\pgfpathlineto{\pgfqpoint{1.046354in}{1.121122in}}%
\pgfpathlineto{\pgfqpoint{1.071386in}{1.129317in}}%
\pgfpathlineto{\pgfqpoint{1.096419in}{1.143293in}}%
\pgfpathlineto{\pgfqpoint{1.121451in}{1.160559in}}%
\pgfpathlineto{\pgfqpoint{1.146484in}{1.177746in}}%
\pgfpathlineto{\pgfqpoint{1.171516in}{1.191278in}}%
\pgfpathlineto{\pgfqpoint{1.196548in}{1.198142in}}%
\pgfpathlineto{\pgfqpoint{1.221581in}{1.196561in}}%
\pgfpathlineto{\pgfqpoint{1.246613in}{1.186397in}}%
\pgfpathlineto{\pgfqpoint{1.271646in}{1.169200in}}%
\pgfpathlineto{\pgfqpoint{1.296678in}{1.147891in}}%
\pgfpathlineto{\pgfqpoint{1.321710in}{1.126185in}}%
\pgfpathlineto{\pgfqpoint{1.346743in}{1.107873in}}%
\pgfpathlineto{\pgfqpoint{1.371775in}{1.096149in}}%
\pgfpathlineto{\pgfqpoint{1.396807in}{1.093107in}}%
\pgfpathlineto{\pgfqpoint{1.421840in}{1.099484in}}%
\pgfpathlineto{\pgfqpoint{1.446872in}{1.114684in}}%
\pgfpathlineto{\pgfqpoint{1.471905in}{1.137028in}}%
\pgfpathlineto{\pgfqpoint{1.496937in}{1.164156in}}%
\pgfpathlineto{\pgfqpoint{1.521969in}{1.193478in}}%
\pgfpathlineto{\pgfqpoint{1.547002in}{1.222581in}}%
\pgfpathlineto{\pgfqpoint{1.572034in}{1.249524in}}%
\pgfpathlineto{\pgfqpoint{1.597067in}{1.272990in}}%
\pgfpathlineto{\pgfqpoint{1.622099in}{1.292301in}}%
\pgfpathlineto{\pgfqpoint{1.647131in}{1.307328in}}%
\pgfpathlineto{\pgfqpoint{1.672164in}{1.318342in}}%
\pgfpathlineto{\pgfqpoint{1.697196in}{1.325845in}}%
\pgfpathlineto{\pgfqpoint{1.722228in}{1.330425in}}%
\pgfpathlineto{\pgfqpoint{1.747261in}{1.332662in}}%
\pgfpathlineto{\pgfqpoint{1.772293in}{1.333065in}}%
\pgfpathlineto{\pgfqpoint{1.797326in}{1.332075in}}%
\pgfpathlineto{\pgfqpoint{1.822358in}{1.330074in}}%
\pgfpathlineto{\pgfqpoint{1.847390in}{1.327417in}}%
\pgfpathlineto{\pgfqpoint{1.872423in}{1.324457in}}%
\pgfpathlineto{\pgfqpoint{1.897455in}{1.321550in}}%
\pgfpathlineto{\pgfqpoint{1.922488in}{1.319043in}}%
\pgfpathlineto{\pgfqpoint{1.947520in}{1.317239in}}%
\pgfpathlineto{\pgfqpoint{1.972552in}{1.316365in}}%
\pgfpathlineto{\pgfqpoint{1.997585in}{1.316535in}}%
\pgfpathlineto{\pgfqpoint{2.022617in}{1.317731in}}%
\pgfpathlineto{\pgfqpoint{2.047650in}{1.319817in}}%
\pgfpathlineto{\pgfqpoint{2.072682in}{1.322557in}}%
\pgfpathlineto{\pgfqpoint{2.097714in}{1.325666in}}%
\pgfpathlineto{\pgfqpoint{2.122747in}{1.328859in}}%
\pgfpathlineto{\pgfqpoint{2.147779in}{1.331888in}}%
\pgfpathlineto{\pgfqpoint{2.172812in}{1.334575in}}%
\pgfpathlineto{\pgfqpoint{2.197844in}{1.336817in}}%
\pgfpathlineto{\pgfqpoint{2.222876in}{1.338588in}}%
\pgfpathlineto{\pgfqpoint{2.247909in}{1.339914in}}%
\pgfpathlineto{\pgfqpoint{2.272941in}{1.340859in}}%
\pgfpathlineto{\pgfqpoint{2.297973in}{1.341500in}}%
\pgfpathlineto{\pgfqpoint{2.323006in}{1.341914in}}%
\pgfpathlineto{\pgfqpoint{2.348038in}{1.342170in}}%
\pgfpathlineto{\pgfqpoint{2.373071in}{1.342321in}}%
\pgfpathlineto{\pgfqpoint{2.398103in}{1.342406in}}%
\pgfpathlineto{\pgfqpoint{2.423135in}{1.342452in}}%
\pgfpathlineto{\pgfqpoint{2.448168in}{1.342476in}}%
\pgfpathlineto{\pgfqpoint{2.473200in}{1.342487in}}%
\pgfpathlineto{\pgfqpoint{2.498233in}{1.342493in}}%
\pgfpathlineto{\pgfqpoint{2.523265in}{1.342495in}}%
\pgfpathlineto{\pgfqpoint{2.548297in}{1.342496in}}%
\pgfpathlineto{\pgfqpoint{2.573330in}{1.342497in}}%
\pgfpathlineto{\pgfqpoint{2.598362in}{1.342497in}}%
\pgfpathlineto{\pgfqpoint{2.623395in}{1.342497in}}%
\pgfpathlineto{\pgfqpoint{2.648427in}{1.342497in}}%
\pgfpathlineto{\pgfqpoint{2.673459in}{1.342497in}}%
\pgfpathlineto{\pgfqpoint{2.698492in}{1.342497in}}%
\pgfusepath{stroke}%
\end{pgfscope}%
\begin{pgfscope}%
\pgfpathrectangle{\pgfqpoint{0.220285in}{0.881091in}}{\pgfqpoint{2.478207in}{0.922812in}}%
\pgfusepath{clip}%
\pgfsetrectcap%
\pgfsetroundjoin%
\pgfsetlinewidth{0.752812pt}%
\definecolor{currentstroke}{rgb}{0.000000,0.000000,0.000000}%
\pgfsetstrokecolor{currentstroke}%
\pgfsetdash{}{0pt}%
\pgfpathmoveto{\pgfqpoint{0.403210in}{1.813903in}}%
\pgfpathlineto{\pgfqpoint{0.420544in}{1.744910in}}%
\pgfpathlineto{\pgfqpoint{0.445577in}{1.678187in}}%
\pgfpathlineto{\pgfqpoint{0.470609in}{1.641080in}}%
\pgfpathlineto{\pgfqpoint{0.495641in}{1.628273in}}%
\pgfpathlineto{\pgfqpoint{0.520674in}{1.632987in}}%
\pgfpathlineto{\pgfqpoint{0.545706in}{1.647656in}}%
\pgfpathlineto{\pgfqpoint{0.570739in}{1.664651in}}%
\pgfpathlineto{\pgfqpoint{0.595771in}{1.676966in}}%
\pgfpathlineto{\pgfqpoint{0.620803in}{1.678831in}}%
\pgfpathlineto{\pgfqpoint{0.645836in}{1.666167in}}%
\pgfpathlineto{\pgfqpoint{0.670868in}{1.636859in}}%
\pgfpathlineto{\pgfqpoint{0.695901in}{1.590834in}}%
\pgfpathlineto{\pgfqpoint{0.720933in}{1.529933in}}%
\pgfpathlineto{\pgfqpoint{0.745965in}{1.457606in}}%
\pgfpathlineto{\pgfqpoint{0.770998in}{1.378473in}}%
\pgfpathlineto{\pgfqpoint{0.796030in}{1.297788in}}%
\pgfpathlineto{\pgfqpoint{0.821062in}{1.220873in}}%
\pgfpathlineto{\pgfqpoint{0.846095in}{1.152579in}}%
\pgfpathlineto{\pgfqpoint{0.871127in}{1.096823in}}%
\pgfpathlineto{\pgfqpoint{0.896160in}{1.056247in}}%
\pgfpathlineto{\pgfqpoint{0.921192in}{1.032021in}}%
\pgfpathlineto{\pgfqpoint{0.946224in}{1.023811in}}%
\pgfpathlineto{\pgfqpoint{0.971257in}{1.029902in}}%
\pgfpathlineto{\pgfqpoint{0.996289in}{1.047453in}}%
\pgfpathlineto{\pgfqpoint{1.021322in}{1.072855in}}%
\pgfpathlineto{\pgfqpoint{1.046354in}{1.102143in}}%
\pgfpathlineto{\pgfqpoint{1.071386in}{1.131431in}}%
\pgfpathlineto{\pgfqpoint{1.096419in}{1.157291in}}%
\pgfpathlineto{\pgfqpoint{1.121451in}{1.177083in}}%
\pgfpathlineto{\pgfqpoint{1.146484in}{1.189157in}}%
\pgfpathlineto{\pgfqpoint{1.171516in}{1.192944in}}%
\pgfpathlineto{\pgfqpoint{1.196548in}{1.188923in}}%
\pgfpathlineto{\pgfqpoint{1.221581in}{1.178463in}}%
\pgfpathlineto{\pgfqpoint{1.246613in}{1.163581in}}%
\pgfpathlineto{\pgfqpoint{1.271646in}{1.146634in}}%
\pgfpathlineto{\pgfqpoint{1.296678in}{1.129999in}}%
\pgfpathlineto{\pgfqpoint{1.321710in}{1.115759in}}%
\pgfpathlineto{\pgfqpoint{1.346743in}{1.105451in}}%
\pgfpathlineto{\pgfqpoint{1.371775in}{1.099892in}}%
\pgfpathlineto{\pgfqpoint{1.396807in}{1.099109in}}%
\pgfpathlineto{\pgfqpoint{1.421840in}{1.102374in}}%
\pgfpathlineto{\pgfqpoint{1.446872in}{1.108339in}}%
\pgfpathlineto{\pgfqpoint{1.471905in}{1.115249in}}%
\pgfpathlineto{\pgfqpoint{1.496937in}{1.121217in}}%
\pgfpathlineto{\pgfqpoint{1.521969in}{1.124504in}}%
\pgfpathlineto{\pgfqpoint{1.547002in}{1.123790in}}%
\pgfpathlineto{\pgfqpoint{1.572034in}{1.118392in}}%
\pgfpathlineto{\pgfqpoint{1.597067in}{1.108398in}}%
\pgfpathlineto{\pgfqpoint{1.622099in}{1.094700in}}%
\pgfpathlineto{\pgfqpoint{1.647131in}{1.078924in}}%
\pgfpathlineto{\pgfqpoint{1.672164in}{1.063257in}}%
\pgfpathlineto{\pgfqpoint{1.697196in}{1.050195in}}%
\pgfpathlineto{\pgfqpoint{1.722228in}{1.042231in}}%
\pgfpathlineto{\pgfqpoint{1.747261in}{1.041531in}}%
\pgfpathlineto{\pgfqpoint{1.772293in}{1.049629in}}%
\pgfpathlineto{\pgfqpoint{1.797326in}{1.067181in}}%
\pgfpathlineto{\pgfqpoint{1.822358in}{1.093815in}}%
\pgfpathlineto{\pgfqpoint{1.847390in}{1.128089in}}%
\pgfpathlineto{\pgfqpoint{1.872423in}{1.167585in}}%
\pgfpathlineto{\pgfqpoint{1.897455in}{1.209118in}}%
\pgfpathlineto{\pgfqpoint{1.922488in}{1.249053in}}%
\pgfpathlineto{\pgfqpoint{1.947520in}{1.283695in}}%
\pgfpathlineto{\pgfqpoint{1.972552in}{1.309719in}}%
\pgfpathlineto{\pgfqpoint{1.997585in}{1.324582in}}%
\pgfpathlineto{\pgfqpoint{2.022617in}{1.326884in}}%
\pgfpathlineto{\pgfqpoint{2.047650in}{1.316619in}}%
\pgfpathlineto{\pgfqpoint{2.072682in}{1.295300in}}%
\pgfpathlineto{\pgfqpoint{2.097714in}{1.265929in}}%
\pgfpathlineto{\pgfqpoint{2.122747in}{1.232795in}}%
\pgfpathlineto{\pgfqpoint{2.147779in}{1.201138in}}%
\pgfpathlineto{\pgfqpoint{2.172812in}{1.176684in}}%
\pgfpathlineto{\pgfqpoint{2.197844in}{1.165105in}}%
\pgfpathlineto{\pgfqpoint{2.222876in}{1.171454in}}%
\pgfpathlineto{\pgfqpoint{2.247909in}{1.199629in}}%
\pgfpathlineto{\pgfqpoint{2.272941in}{1.251935in}}%
\pgfpathlineto{\pgfqpoint{2.297973in}{1.328776in}}%
\pgfpathlineto{\pgfqpoint{2.323006in}{1.428532in}}%
\pgfpathlineto{\pgfqpoint{2.348038in}{1.547629in}}%
\pgfpathlineto{\pgfqpoint{2.373071in}{1.680821in}}%
\pgfpathlineto{\pgfqpoint{2.396727in}{1.813903in}}%
\pgfusepath{stroke}%
\end{pgfscope}%
\begin{pgfscope}%
\pgfpathrectangle{\pgfqpoint{0.220285in}{0.881091in}}{\pgfqpoint{2.478207in}{0.922812in}}%
\pgfusepath{clip}%
\pgfsetbuttcap%
\pgfsetroundjoin%
\definecolor{currentfill}{rgb}{0.631373,0.062745,0.207843}%
\pgfsetfillcolor{currentfill}%
\pgfsetlinewidth{1.003750pt}%
\definecolor{currentstroke}{rgb}{0.631373,0.062745,0.207843}%
\pgfsetstrokecolor{currentstroke}%
\pgfsetdash{}{0pt}%
\pgfsys@defobject{currentmarker}{\pgfqpoint{-0.027778in}{-0.027778in}}{\pgfqpoint{0.027778in}{0.027778in}}{%
\pgfpathmoveto{\pgfqpoint{0.000000in}{-0.027778in}}%
\pgfpathcurveto{\pgfqpoint{0.007367in}{-0.027778in}}{\pgfqpoint{0.014433in}{-0.024851in}}{\pgfqpoint{0.019642in}{-0.019642in}}%
\pgfpathcurveto{\pgfqpoint{0.024851in}{-0.014433in}}{\pgfqpoint{0.027778in}{-0.007367in}}{\pgfqpoint{0.027778in}{0.000000in}}%
\pgfpathcurveto{\pgfqpoint{0.027778in}{0.007367in}}{\pgfqpoint{0.024851in}{0.014433in}}{\pgfqpoint{0.019642in}{0.019642in}}%
\pgfpathcurveto{\pgfqpoint{0.014433in}{0.024851in}}{\pgfqpoint{0.007367in}{0.027778in}}{\pgfqpoint{0.000000in}{0.027778in}}%
\pgfpathcurveto{\pgfqpoint{-0.007367in}{0.027778in}}{\pgfqpoint{-0.014433in}{0.024851in}}{\pgfqpoint{-0.019642in}{0.019642in}}%
\pgfpathcurveto{\pgfqpoint{-0.024851in}{0.014433in}}{\pgfqpoint{-0.027778in}{0.007367in}}{\pgfqpoint{-0.027778in}{0.000000in}}%
\pgfpathcurveto{\pgfqpoint{-0.027778in}{-0.007367in}}{\pgfqpoint{-0.024851in}{-0.014433in}}{\pgfqpoint{-0.019642in}{-0.019642in}}%
\pgfpathcurveto{\pgfqpoint{-0.014433in}{-0.024851in}}{\pgfqpoint{-0.007367in}{-0.027778in}}{\pgfqpoint{0.000000in}{-0.027778in}}%
\pgfpathclose%
\pgfusepath{stroke,fill}%
}%
\begin{pgfscope}%
\pgfsys@transformshift{0.979396in}{1.034515in}%
\pgfsys@useobject{currentmarker}{}%
\end{pgfscope}%
\end{pgfscope}%
\begin{pgfscope}%
\pgfsetrectcap%
\pgfsetmiterjoin%
\pgfsetlinewidth{0.752812pt}%
\definecolor{currentstroke}{rgb}{0.000000,0.000000,0.000000}%
\pgfsetstrokecolor{currentstroke}%
\pgfsetdash{}{0pt}%
\pgfpathmoveto{\pgfqpoint{0.220285in}{0.881091in}}%
\pgfpathlineto{\pgfqpoint{0.220285in}{1.803903in}}%
\pgfusepath{stroke}%
\end{pgfscope}%
\begin{pgfscope}%
\pgfsetrectcap%
\pgfsetmiterjoin%
\pgfsetlinewidth{0.752812pt}%
\definecolor{currentstroke}{rgb}{0.000000,0.000000,0.000000}%
\pgfsetstrokecolor{currentstroke}%
\pgfsetdash{}{0pt}%
\pgfpathmoveto{\pgfqpoint{2.698492in}{0.881091in}}%
\pgfpathlineto{\pgfqpoint{2.698492in}{1.803903in}}%
\pgfusepath{stroke}%
\end{pgfscope}%
\begin{pgfscope}%
\pgfsetrectcap%
\pgfsetmiterjoin%
\pgfsetlinewidth{0.752812pt}%
\definecolor{currentstroke}{rgb}{0.000000,0.000000,0.000000}%
\pgfsetstrokecolor{currentstroke}%
\pgfsetdash{}{0pt}%
\pgfpathmoveto{\pgfqpoint{0.220285in}{0.881091in}}%
\pgfpathlineto{\pgfqpoint{2.698492in}{0.881091in}}%
\pgfusepath{stroke}%
\end{pgfscope}%
\begin{pgfscope}%
\pgfsetrectcap%
\pgfsetmiterjoin%
\pgfsetlinewidth{0.752812pt}%
\definecolor{currentstroke}{rgb}{0.000000,0.000000,0.000000}%
\pgfsetstrokecolor{currentstroke}%
\pgfsetdash{}{0pt}%
\pgfpathmoveto{\pgfqpoint{0.220285in}{1.803903in}}%
\pgfpathlineto{\pgfqpoint{2.698492in}{1.803903in}}%
\pgfusepath{stroke}%
\end{pgfscope}%
\begin{pgfscope}%
\definecolor{textcolor}{rgb}{0.000000,0.000000,0.000000}%
\pgfsetstrokecolor{textcolor}%
\pgfsetfillcolor{textcolor}%
\pgftext[x=1.459388in,y=1.665481in,,base]{\color{textcolor}\rmfamily\fontsize{10.000000}{12.000000}\selectfont (3)}%
\end{pgfscope}%
\begin{pgfscope}%
\pgfsetbuttcap%
\pgfsetmiterjoin%
\definecolor{currentfill}{rgb}{1.000000,1.000000,1.000000}%
\pgfsetfillcolor{currentfill}%
\pgfsetlinewidth{0.000000pt}%
\definecolor{currentstroke}{rgb}{0.000000,0.000000,0.000000}%
\pgfsetstrokecolor{currentstroke}%
\pgfsetstrokeopacity{0.000000}%
\pgfsetdash{}{0pt}%
\pgfpathmoveto{\pgfqpoint{0.220285in}{0.408431in}}%
\pgfpathlineto{\pgfqpoint{2.698492in}{0.408431in}}%
\pgfpathlineto{\pgfqpoint{2.698492in}{0.858583in}}%
\pgfpathlineto{\pgfqpoint{0.220285in}{0.858583in}}%
\pgfpathclose%
\pgfusepath{fill}%
\end{pgfscope}%
\begin{pgfscope}%
\pgfsetbuttcap%
\pgfsetroundjoin%
\definecolor{currentfill}{rgb}{0.000000,0.000000,0.000000}%
\pgfsetfillcolor{currentfill}%
\pgfsetlinewidth{0.803000pt}%
\definecolor{currentstroke}{rgb}{0.000000,0.000000,0.000000}%
\pgfsetstrokecolor{currentstroke}%
\pgfsetdash{}{0pt}%
\pgfsys@defobject{currentmarker}{\pgfqpoint{0.000000in}{-0.048611in}}{\pgfqpoint{0.000000in}{0.000000in}}{%
\pgfpathmoveto{\pgfqpoint{0.000000in}{0.000000in}}%
\pgfpathlineto{\pgfqpoint{0.000000in}{-0.048611in}}%
\pgfusepath{stroke,fill}%
}%
\begin{pgfscope}%
\pgfsys@transformshift{0.220285in}{0.408431in}%
\pgfsys@useobject{currentmarker}{}%
\end{pgfscope}%
\end{pgfscope}%
\begin{pgfscope}%
\definecolor{textcolor}{rgb}{0.000000,0.000000,0.000000}%
\pgfsetstrokecolor{textcolor}%
\pgfsetfillcolor{textcolor}%
\pgftext[x=0.220285in,y=0.311209in,,top]{\color{textcolor}\rmfamily\fontsize{10.000000}{12.000000}\selectfont \(\displaystyle {\ensuremath{-}3}\)}%
\end{pgfscope}%
\begin{pgfscope}%
\pgfsetbuttcap%
\pgfsetroundjoin%
\definecolor{currentfill}{rgb}{0.000000,0.000000,0.000000}%
\pgfsetfillcolor{currentfill}%
\pgfsetlinewidth{0.803000pt}%
\definecolor{currentstroke}{rgb}{0.000000,0.000000,0.000000}%
\pgfsetstrokecolor{currentstroke}%
\pgfsetdash{}{0pt}%
\pgfsys@defobject{currentmarker}{\pgfqpoint{0.000000in}{-0.048611in}}{\pgfqpoint{0.000000in}{0.000000in}}{%
\pgfpathmoveto{\pgfqpoint{0.000000in}{0.000000in}}%
\pgfpathlineto{\pgfqpoint{0.000000in}{-0.048611in}}%
\pgfusepath{stroke,fill}%
}%
\begin{pgfscope}%
\pgfsys@transformshift{0.633319in}{0.408431in}%
\pgfsys@useobject{currentmarker}{}%
\end{pgfscope}%
\end{pgfscope}%
\begin{pgfscope}%
\definecolor{textcolor}{rgb}{0.000000,0.000000,0.000000}%
\pgfsetstrokecolor{textcolor}%
\pgfsetfillcolor{textcolor}%
\pgftext[x=0.633319in,y=0.311209in,,top]{\color{textcolor}\rmfamily\fontsize{10.000000}{12.000000}\selectfont \(\displaystyle {\ensuremath{-}2}\)}%
\end{pgfscope}%
\begin{pgfscope}%
\pgfsetbuttcap%
\pgfsetroundjoin%
\definecolor{currentfill}{rgb}{0.000000,0.000000,0.000000}%
\pgfsetfillcolor{currentfill}%
\pgfsetlinewidth{0.803000pt}%
\definecolor{currentstroke}{rgb}{0.000000,0.000000,0.000000}%
\pgfsetstrokecolor{currentstroke}%
\pgfsetdash{}{0pt}%
\pgfsys@defobject{currentmarker}{\pgfqpoint{0.000000in}{-0.048611in}}{\pgfqpoint{0.000000in}{0.000000in}}{%
\pgfpathmoveto{\pgfqpoint{0.000000in}{0.000000in}}%
\pgfpathlineto{\pgfqpoint{0.000000in}{-0.048611in}}%
\pgfusepath{stroke,fill}%
}%
\begin{pgfscope}%
\pgfsys@transformshift{1.046354in}{0.408431in}%
\pgfsys@useobject{currentmarker}{}%
\end{pgfscope}%
\end{pgfscope}%
\begin{pgfscope}%
\definecolor{textcolor}{rgb}{0.000000,0.000000,0.000000}%
\pgfsetstrokecolor{textcolor}%
\pgfsetfillcolor{textcolor}%
\pgftext[x=1.046354in,y=0.311209in,,top]{\color{textcolor}\rmfamily\fontsize{10.000000}{12.000000}\selectfont \(\displaystyle {\ensuremath{-}1}\)}%
\end{pgfscope}%
\begin{pgfscope}%
\pgfsetbuttcap%
\pgfsetroundjoin%
\definecolor{currentfill}{rgb}{0.000000,0.000000,0.000000}%
\pgfsetfillcolor{currentfill}%
\pgfsetlinewidth{0.803000pt}%
\definecolor{currentstroke}{rgb}{0.000000,0.000000,0.000000}%
\pgfsetstrokecolor{currentstroke}%
\pgfsetdash{}{0pt}%
\pgfsys@defobject{currentmarker}{\pgfqpoint{0.000000in}{-0.048611in}}{\pgfqpoint{0.000000in}{0.000000in}}{%
\pgfpathmoveto{\pgfqpoint{0.000000in}{0.000000in}}%
\pgfpathlineto{\pgfqpoint{0.000000in}{-0.048611in}}%
\pgfusepath{stroke,fill}%
}%
\begin{pgfscope}%
\pgfsys@transformshift{1.459388in}{0.408431in}%
\pgfsys@useobject{currentmarker}{}%
\end{pgfscope}%
\end{pgfscope}%
\begin{pgfscope}%
\definecolor{textcolor}{rgb}{0.000000,0.000000,0.000000}%
\pgfsetstrokecolor{textcolor}%
\pgfsetfillcolor{textcolor}%
\pgftext[x=1.459388in,y=0.311209in,,top]{\color{textcolor}\rmfamily\fontsize{10.000000}{12.000000}\selectfont \(\displaystyle {0}\)}%
\end{pgfscope}%
\begin{pgfscope}%
\pgfsetbuttcap%
\pgfsetroundjoin%
\definecolor{currentfill}{rgb}{0.000000,0.000000,0.000000}%
\pgfsetfillcolor{currentfill}%
\pgfsetlinewidth{0.803000pt}%
\definecolor{currentstroke}{rgb}{0.000000,0.000000,0.000000}%
\pgfsetstrokecolor{currentstroke}%
\pgfsetdash{}{0pt}%
\pgfsys@defobject{currentmarker}{\pgfqpoint{0.000000in}{-0.048611in}}{\pgfqpoint{0.000000in}{0.000000in}}{%
\pgfpathmoveto{\pgfqpoint{0.000000in}{0.000000in}}%
\pgfpathlineto{\pgfqpoint{0.000000in}{-0.048611in}}%
\pgfusepath{stroke,fill}%
}%
\begin{pgfscope}%
\pgfsys@transformshift{1.872423in}{0.408431in}%
\pgfsys@useobject{currentmarker}{}%
\end{pgfscope}%
\end{pgfscope}%
\begin{pgfscope}%
\definecolor{textcolor}{rgb}{0.000000,0.000000,0.000000}%
\pgfsetstrokecolor{textcolor}%
\pgfsetfillcolor{textcolor}%
\pgftext[x=1.872423in,y=0.311209in,,top]{\color{textcolor}\rmfamily\fontsize{10.000000}{12.000000}\selectfont \(\displaystyle {1}\)}%
\end{pgfscope}%
\begin{pgfscope}%
\pgfsetbuttcap%
\pgfsetroundjoin%
\definecolor{currentfill}{rgb}{0.000000,0.000000,0.000000}%
\pgfsetfillcolor{currentfill}%
\pgfsetlinewidth{0.803000pt}%
\definecolor{currentstroke}{rgb}{0.000000,0.000000,0.000000}%
\pgfsetstrokecolor{currentstroke}%
\pgfsetdash{}{0pt}%
\pgfsys@defobject{currentmarker}{\pgfqpoint{0.000000in}{-0.048611in}}{\pgfqpoint{0.000000in}{0.000000in}}{%
\pgfpathmoveto{\pgfqpoint{0.000000in}{0.000000in}}%
\pgfpathlineto{\pgfqpoint{0.000000in}{-0.048611in}}%
\pgfusepath{stroke,fill}%
}%
\begin{pgfscope}%
\pgfsys@transformshift{2.285457in}{0.408431in}%
\pgfsys@useobject{currentmarker}{}%
\end{pgfscope}%
\end{pgfscope}%
\begin{pgfscope}%
\definecolor{textcolor}{rgb}{0.000000,0.000000,0.000000}%
\pgfsetstrokecolor{textcolor}%
\pgfsetfillcolor{textcolor}%
\pgftext[x=2.285457in,y=0.311209in,,top]{\color{textcolor}\rmfamily\fontsize{10.000000}{12.000000}\selectfont \(\displaystyle {2}\)}%
\end{pgfscope}%
\begin{pgfscope}%
\pgfsetbuttcap%
\pgfsetroundjoin%
\definecolor{currentfill}{rgb}{0.000000,0.000000,0.000000}%
\pgfsetfillcolor{currentfill}%
\pgfsetlinewidth{0.803000pt}%
\definecolor{currentstroke}{rgb}{0.000000,0.000000,0.000000}%
\pgfsetstrokecolor{currentstroke}%
\pgfsetdash{}{0pt}%
\pgfsys@defobject{currentmarker}{\pgfqpoint{0.000000in}{-0.048611in}}{\pgfqpoint{0.000000in}{0.000000in}}{%
\pgfpathmoveto{\pgfqpoint{0.000000in}{0.000000in}}%
\pgfpathlineto{\pgfqpoint{0.000000in}{-0.048611in}}%
\pgfusepath{stroke,fill}%
}%
\begin{pgfscope}%
\pgfsys@transformshift{2.698492in}{0.408431in}%
\pgfsys@useobject{currentmarker}{}%
\end{pgfscope}%
\end{pgfscope}%
\begin{pgfscope}%
\definecolor{textcolor}{rgb}{0.000000,0.000000,0.000000}%
\pgfsetstrokecolor{textcolor}%
\pgfsetfillcolor{textcolor}%
\pgftext[x=2.698492in,y=0.311209in,,top]{\color{textcolor}\rmfamily\fontsize{10.000000}{12.000000}\selectfont \(\displaystyle {3}\)}%
\end{pgfscope}%
\begin{pgfscope}%
\definecolor{textcolor}{rgb}{0.000000,0.000000,0.000000}%
\pgfsetstrokecolor{textcolor}%
\pgfsetfillcolor{textcolor}%
\pgftext[x=1.459388in,y=0.132320in,,top]{\color{textcolor}\rmfamily\fontsize{10.000000}{12.000000}\selectfont \(\displaystyle x\)}%
\end{pgfscope}%
\begin{pgfscope}%
\definecolor{textcolor}{rgb}{0.000000,0.000000,0.000000}%
\pgfsetstrokecolor{textcolor}%
\pgfsetfillcolor{textcolor}%
\pgftext[x=0.164729in,y=0.633507in,,bottom,rotate=90.000000]{\color{textcolor}\rmfamily\fontsize{10.000000}{12.000000}\selectfont \(\displaystyle \alpha(x|\mathcal{D})\)}%
\end{pgfscope}%
\begin{pgfscope}%
\pgfpathrectangle{\pgfqpoint{0.220285in}{0.408431in}}{\pgfqpoint{2.478207in}{0.450152in}}%
\pgfusepath{clip}%
\pgfsetbuttcap%
\pgfsetroundjoin%
\pgfsetlinewidth{0.752812pt}%
\definecolor{currentstroke}{rgb}{0.000000,0.000000,0.000000}%
\pgfsetstrokecolor{currentstroke}%
\pgfsetdash{{2.775000pt}{1.200000pt}}{0.000000pt}%
\pgfpathmoveto{\pgfqpoint{0.633319in}{0.408431in}}%
\pgfpathlineto{\pgfqpoint{0.633319in}{0.858583in}}%
\pgfusepath{stroke}%
\end{pgfscope}%
\begin{pgfscope}%
\pgfpathrectangle{\pgfqpoint{0.220285in}{0.408431in}}{\pgfqpoint{2.478207in}{0.450152in}}%
\pgfusepath{clip}%
\pgfsetbuttcap%
\pgfsetroundjoin%
\pgfsetlinewidth{0.752812pt}%
\definecolor{currentstroke}{rgb}{0.000000,0.000000,0.000000}%
\pgfsetstrokecolor{currentstroke}%
\pgfsetdash{{2.775000pt}{1.200000pt}}{0.000000pt}%
\pgfpathmoveto{\pgfqpoint{2.285457in}{0.408431in}}%
\pgfpathlineto{\pgfqpoint{2.285457in}{0.858583in}}%
\pgfusepath{stroke}%
\end{pgfscope}%
\begin{pgfscope}%
\pgfpathrectangle{\pgfqpoint{0.220285in}{0.408431in}}{\pgfqpoint{2.478207in}{0.450152in}}%
\pgfusepath{clip}%
\pgfsetbuttcap%
\pgfsetroundjoin%
\pgfsetlinewidth{1.505625pt}%
\definecolor{currentstroke}{rgb}{0.631373,0.062745,0.207843}%
\pgfsetstrokecolor{currentstroke}%
\pgfsetdash{{1.500000pt}{2.475000pt}}{0.000000pt}%
\pgfpathmoveto{\pgfqpoint{0.979396in}{0.408431in}}%
\pgfpathlineto{\pgfqpoint{0.979396in}{0.858583in}}%
\pgfusepath{stroke}%
\end{pgfscope}%
\begin{pgfscope}%
\pgfpathrectangle{\pgfqpoint{0.220285in}{0.408431in}}{\pgfqpoint{2.478207in}{0.450152in}}%
\pgfusepath{clip}%
\pgfsetrectcap%
\pgfsetroundjoin%
\pgfsetlinewidth{0.752812pt}%
\definecolor{currentstroke}{rgb}{0.964706,0.658824,0.000000}%
\pgfsetstrokecolor{currentstroke}%
\pgfsetdash{}{0pt}%
\pgfpathmoveto{\pgfqpoint{0.220285in}{0.694210in}}%
\pgfpathlineto{\pgfqpoint{0.245317in}{0.694209in}}%
\pgfpathlineto{\pgfqpoint{0.270350in}{0.694205in}}%
\pgfpathlineto{\pgfqpoint{0.295382in}{0.694197in}}%
\pgfpathlineto{\pgfqpoint{0.320415in}{0.694178in}}%
\pgfpathlineto{\pgfqpoint{0.345447in}{0.694139in}}%
\pgfpathlineto{\pgfqpoint{0.370479in}{0.694060in}}%
\pgfpathlineto{\pgfqpoint{0.395512in}{0.693904in}}%
\pgfpathlineto{\pgfqpoint{0.420544in}{0.693613in}}%
\pgfpathlineto{\pgfqpoint{0.445577in}{0.693096in}}%
\pgfpathlineto{\pgfqpoint{0.470609in}{0.692221in}}%
\pgfpathlineto{\pgfqpoint{0.495641in}{0.690815in}}%
\pgfpathlineto{\pgfqpoint{0.520674in}{0.688685in}}%
\pgfpathlineto{\pgfqpoint{0.545706in}{0.685660in}}%
\pgfpathlineto{\pgfqpoint{0.570739in}{0.681669in}}%
\pgfpathlineto{\pgfqpoint{0.595771in}{0.676850in}}%
\pgfpathlineto{\pgfqpoint{0.620803in}{0.671645in}}%
\pgfpathlineto{\pgfqpoint{0.645836in}{0.666855in}}%
\pgfpathlineto{\pgfqpoint{0.670868in}{0.663584in}}%
\pgfpathlineto{\pgfqpoint{0.695901in}{0.663062in}}%
\pgfpathlineto{\pgfqpoint{0.720933in}{0.666334in}}%
\pgfpathlineto{\pgfqpoint{0.745965in}{0.673901in}}%
\pgfpathlineto{\pgfqpoint{0.770998in}{0.685345in}}%
\pgfpathlineto{\pgfqpoint{0.796030in}{0.698842in}}%
\pgfpathlineto{\pgfqpoint{0.821062in}{0.700707in}}%
\pgfpathlineto{\pgfqpoint{0.846095in}{0.658435in}}%
\pgfpathlineto{\pgfqpoint{0.871127in}{0.615067in}}%
\pgfpathlineto{\pgfqpoint{0.896160in}{0.576908in}}%
\pgfpathlineto{\pgfqpoint{0.921192in}{0.546390in}}%
\pgfpathlineto{\pgfqpoint{0.946224in}{0.525522in}}%
\pgfpathlineto{\pgfqpoint{0.971257in}{0.515790in}}%
\pgfpathlineto{\pgfqpoint{0.996289in}{0.517801in}}%
\pgfpathlineto{\pgfqpoint{1.021322in}{0.530830in}}%
\pgfpathlineto{\pgfqpoint{1.046354in}{0.552078in}}%
\pgfpathlineto{\pgfqpoint{1.071386in}{0.571725in}}%
\pgfpathlineto{\pgfqpoint{1.096419in}{0.583509in}}%
\pgfpathlineto{\pgfqpoint{1.121451in}{0.603305in}}%
\pgfpathlineto{\pgfqpoint{1.146484in}{0.629484in}}%
\pgfpathlineto{\pgfqpoint{1.171516in}{0.652402in}}%
\pgfpathlineto{\pgfqpoint{1.196548in}{0.655482in}}%
\pgfpathlineto{\pgfqpoint{1.221581in}{0.643173in}}%
\pgfpathlineto{\pgfqpoint{1.246613in}{0.623222in}}%
\pgfpathlineto{\pgfqpoint{1.271646in}{0.600200in}}%
\pgfpathlineto{\pgfqpoint{1.296678in}{0.577928in}}%
\pgfpathlineto{\pgfqpoint{1.321710in}{0.558965in}}%
\pgfpathlineto{\pgfqpoint{1.346743in}{0.543145in}}%
\pgfpathlineto{\pgfqpoint{1.371775in}{0.527828in}}%
\pgfpathlineto{\pgfqpoint{1.396807in}{0.522291in}}%
\pgfpathlineto{\pgfqpoint{1.421840in}{0.532877in}}%
\pgfpathlineto{\pgfqpoint{1.446872in}{0.548471in}}%
\pgfpathlineto{\pgfqpoint{1.471905in}{0.560124in}}%
\pgfpathlineto{\pgfqpoint{1.496937in}{0.572298in}}%
\pgfpathlineto{\pgfqpoint{1.521969in}{0.585772in}}%
\pgfpathlineto{\pgfqpoint{1.547002in}{0.600350in}}%
\pgfpathlineto{\pgfqpoint{1.572034in}{0.615527in}}%
\pgfpathlineto{\pgfqpoint{1.597067in}{0.630584in}}%
\pgfpathlineto{\pgfqpoint{1.622099in}{0.644697in}}%
\pgfpathlineto{\pgfqpoint{1.647131in}{0.657106in}}%
\pgfpathlineto{\pgfqpoint{1.672164in}{0.667268in}}%
\pgfpathlineto{\pgfqpoint{1.697196in}{0.674975in}}%
\pgfpathlineto{\pgfqpoint{1.722228in}{0.680400in}}%
\pgfpathlineto{\pgfqpoint{1.747261in}{0.684086in}}%
\pgfpathlineto{\pgfqpoint{1.772293in}{0.686908in}}%
\pgfpathlineto{\pgfqpoint{1.797326in}{0.690016in}}%
\pgfpathlineto{\pgfqpoint{1.822358in}{0.694746in}}%
\pgfpathlineto{\pgfqpoint{1.847390in}{0.702482in}}%
\pgfpathlineto{\pgfqpoint{1.872423in}{0.714457in}}%
\pgfpathlineto{\pgfqpoint{1.897455in}{0.731513in}}%
\pgfpathlineto{\pgfqpoint{1.922488in}{0.753873in}}%
\pgfpathlineto{\pgfqpoint{1.947520in}{0.780916in}}%
\pgfpathlineto{\pgfqpoint{1.972552in}{0.809734in}}%
\pgfpathlineto{\pgfqpoint{1.997585in}{0.801227in}}%
\pgfpathlineto{\pgfqpoint{2.022617in}{0.772186in}}%
\pgfpathlineto{\pgfqpoint{2.047650in}{0.746502in}}%
\pgfpathlineto{\pgfqpoint{2.072682in}{0.725853in}}%
\pgfpathlineto{\pgfqpoint{2.097714in}{0.710616in}}%
\pgfpathlineto{\pgfqpoint{2.122747in}{0.700434in}}%
\pgfpathlineto{\pgfqpoint{2.147779in}{0.694454in}}%
\pgfpathlineto{\pgfqpoint{2.172812in}{0.691582in}}%
\pgfpathlineto{\pgfqpoint{2.197844in}{0.690722in}}%
\pgfpathlineto{\pgfqpoint{2.222876in}{0.690961in}}%
\pgfpathlineto{\pgfqpoint{2.247909in}{0.691650in}}%
\pgfpathlineto{\pgfqpoint{2.272941in}{0.692401in}}%
\pgfpathlineto{\pgfqpoint{2.297973in}{0.693032in}}%
\pgfpathlineto{\pgfqpoint{2.323006in}{0.693493in}}%
\pgfpathlineto{\pgfqpoint{2.348038in}{0.693798in}}%
\pgfpathlineto{\pgfqpoint{2.373071in}{0.693985in}}%
\pgfpathlineto{\pgfqpoint{2.398103in}{0.694093in}}%
\pgfpathlineto{\pgfqpoint{2.423135in}{0.694152in}}%
\pgfpathlineto{\pgfqpoint{2.448168in}{0.694183in}}%
\pgfpathlineto{\pgfqpoint{2.473200in}{0.694198in}}%
\pgfpathlineto{\pgfqpoint{2.498233in}{0.694205in}}%
\pgfpathlineto{\pgfqpoint{2.523265in}{0.694209in}}%
\pgfpathlineto{\pgfqpoint{2.548297in}{0.694210in}}%
\pgfpathlineto{\pgfqpoint{2.573330in}{0.694211in}}%
\pgfpathlineto{\pgfqpoint{2.598362in}{0.694211in}}%
\pgfpathlineto{\pgfqpoint{2.623395in}{0.694211in}}%
\pgfpathlineto{\pgfqpoint{2.648427in}{0.694211in}}%
\pgfpathlineto{\pgfqpoint{2.673459in}{0.694211in}}%
\pgfpathlineto{\pgfqpoint{2.698492in}{0.694211in}}%
\pgfusepath{stroke}%
\end{pgfscope}%
\begin{pgfscope}%
\pgfsetrectcap%
\pgfsetmiterjoin%
\pgfsetlinewidth{0.752812pt}%
\definecolor{currentstroke}{rgb}{0.000000,0.000000,0.000000}%
\pgfsetstrokecolor{currentstroke}%
\pgfsetdash{}{0pt}%
\pgfpathmoveto{\pgfqpoint{0.220285in}{0.408431in}}%
\pgfpathlineto{\pgfqpoint{0.220285in}{0.858583in}}%
\pgfusepath{stroke}%
\end{pgfscope}%
\begin{pgfscope}%
\pgfsetrectcap%
\pgfsetmiterjoin%
\pgfsetlinewidth{0.752812pt}%
\definecolor{currentstroke}{rgb}{0.000000,0.000000,0.000000}%
\pgfsetstrokecolor{currentstroke}%
\pgfsetdash{}{0pt}%
\pgfpathmoveto{\pgfqpoint{2.698492in}{0.408431in}}%
\pgfpathlineto{\pgfqpoint{2.698492in}{0.858583in}}%
\pgfusepath{stroke}%
\end{pgfscope}%
\begin{pgfscope}%
\pgfsetrectcap%
\pgfsetmiterjoin%
\pgfsetlinewidth{0.752812pt}%
\definecolor{currentstroke}{rgb}{0.000000,0.000000,0.000000}%
\pgfsetstrokecolor{currentstroke}%
\pgfsetdash{}{0pt}%
\pgfpathmoveto{\pgfqpoint{0.220285in}{0.408431in}}%
\pgfpathlineto{\pgfqpoint{2.698492in}{0.408431in}}%
\pgfusepath{stroke}%
\end{pgfscope}%
\begin{pgfscope}%
\pgfsetrectcap%
\pgfsetmiterjoin%
\pgfsetlinewidth{0.752812pt}%
\definecolor{currentstroke}{rgb}{0.000000,0.000000,0.000000}%
\pgfsetstrokecolor{currentstroke}%
\pgfsetdash{}{0pt}%
\pgfpathmoveto{\pgfqpoint{0.220285in}{0.858583in}}%
\pgfpathlineto{\pgfqpoint{2.698492in}{0.858583in}}%
\pgfusepath{stroke}%
\end{pgfscope}%
\begin{pgfscope}%
\pgfsetbuttcap%
\pgfsetmiterjoin%
\definecolor{currentfill}{rgb}{1.000000,1.000000,1.000000}%
\pgfsetfillcolor{currentfill}%
\pgfsetlinewidth{0.000000pt}%
\definecolor{currentstroke}{rgb}{0.000000,0.000000,0.000000}%
\pgfsetstrokecolor{currentstroke}%
\pgfsetstrokeopacity{0.000000}%
\pgfsetdash{}{0pt}%
\pgfpathmoveto{\pgfqpoint{2.918777in}{0.881091in}}%
\pgfpathlineto{\pgfqpoint{5.396984in}{0.881091in}}%
\pgfpathlineto{\pgfqpoint{5.396984in}{1.803903in}}%
\pgfpathlineto{\pgfqpoint{2.918777in}{1.803903in}}%
\pgfpathclose%
\pgfusepath{fill}%
\end{pgfscope}%
\begin{pgfscope}%
\pgfpathrectangle{\pgfqpoint{2.918777in}{0.881091in}}{\pgfqpoint{2.478207in}{0.922812in}}%
\pgfusepath{clip}%
\pgfsetbuttcap%
\pgfsetroundjoin%
\definecolor{currentfill}{rgb}{0.556863,0.729412,0.898039}%
\pgfsetfillcolor{currentfill}%
\pgfsetfillopacity{0.700000}%
\pgfsetlinewidth{0.000000pt}%
\definecolor{currentstroke}{rgb}{0.556863,0.729412,0.898039}%
\pgfsetstrokecolor{currentstroke}%
\pgfsetstrokeopacity{0.700000}%
\pgfsetdash{}{0pt}%
\pgfsys@defobject{currentmarker}{\pgfqpoint{2.918777in}{1.010084in}}{\pgfqpoint{5.396984in}{1.599948in}}{%
\pgfpathmoveto{\pgfqpoint{2.918777in}{1.595220in}}%
\pgfpathlineto{\pgfqpoint{2.918777in}{1.089775in}}%
\pgfpathlineto{\pgfqpoint{2.943809in}{1.089776in}}%
\pgfpathlineto{\pgfqpoint{2.968842in}{1.089779in}}%
\pgfpathlineto{\pgfqpoint{2.993874in}{1.089785in}}%
\pgfpathlineto{\pgfqpoint{3.018906in}{1.089798in}}%
\pgfpathlineto{\pgfqpoint{3.043939in}{1.089826in}}%
\pgfpathlineto{\pgfqpoint{3.068971in}{1.089883in}}%
\pgfpathlineto{\pgfqpoint{3.094004in}{1.089993in}}%
\pgfpathlineto{\pgfqpoint{3.119036in}{1.090200in}}%
\pgfpathlineto{\pgfqpoint{3.144068in}{1.090570in}}%
\pgfpathlineto{\pgfqpoint{3.169101in}{1.091210in}}%
\pgfpathlineto{\pgfqpoint{3.194133in}{1.092283in}}%
\pgfpathlineto{\pgfqpoint{3.219166in}{1.094033in}}%
\pgfpathlineto{\pgfqpoint{3.244198in}{1.096824in}}%
\pgfpathlineto{\pgfqpoint{3.269230in}{1.101170in}}%
\pgfpathlineto{\pgfqpoint{3.294263in}{1.107748in}}%
\pgfpathlineto{\pgfqpoint{3.319295in}{1.117328in}}%
\pgfpathlineto{\pgfqpoint{3.344327in}{1.130585in}}%
\pgfpathlineto{\pgfqpoint{3.369360in}{1.147770in}}%
\pgfpathlineto{\pgfqpoint{3.394392in}{1.168316in}}%
\pgfpathlineto{\pgfqpoint{3.419425in}{1.190487in}}%
\pgfpathlineto{\pgfqpoint{3.444457in}{1.211256in}}%
\pgfpathlineto{\pgfqpoint{3.469489in}{1.226511in}}%
\pgfpathlineto{\pgfqpoint{3.494522in}{1.231176in}}%
\pgfpathlineto{\pgfqpoint{3.519554in}{1.209469in}}%
\pgfpathlineto{\pgfqpoint{3.544587in}{1.144476in}}%
\pgfpathlineto{\pgfqpoint{3.569619in}{1.084135in}}%
\pgfpathlineto{\pgfqpoint{3.594651in}{1.040138in}}%
\pgfpathlineto{\pgfqpoint{3.619684in}{1.015727in}}%
\pgfpathlineto{\pgfqpoint{3.644716in}{1.010084in}}%
\pgfpathlineto{\pgfqpoint{3.669749in}{1.017886in}}%
\pgfpathlineto{\pgfqpoint{3.694781in}{1.031399in}}%
\pgfpathlineto{\pgfqpoint{3.719813in}{1.053279in}}%
\pgfpathlineto{\pgfqpoint{3.744846in}{1.084399in}}%
\pgfpathlineto{\pgfqpoint{3.769878in}{1.116224in}}%
\pgfpathlineto{\pgfqpoint{3.794911in}{1.141146in}}%
\pgfpathlineto{\pgfqpoint{3.819943in}{1.159751in}}%
\pgfpathlineto{\pgfqpoint{3.844975in}{1.173637in}}%
\pgfpathlineto{\pgfqpoint{3.870008in}{1.179336in}}%
\pgfpathlineto{\pgfqpoint{3.895040in}{1.167610in}}%
\pgfpathlineto{\pgfqpoint{3.920072in}{1.144553in}}%
\pgfpathlineto{\pgfqpoint{3.945105in}{1.120975in}}%
\pgfpathlineto{\pgfqpoint{3.970137in}{1.103135in}}%
\pgfpathlineto{\pgfqpoint{3.995170in}{1.093604in}}%
\pgfpathlineto{\pgfqpoint{4.020202in}{1.091248in}}%
\pgfpathlineto{\pgfqpoint{4.045234in}{1.090435in}}%
\pgfpathlineto{\pgfqpoint{4.070267in}{1.083109in}}%
\pgfpathlineto{\pgfqpoint{4.095299in}{1.079023in}}%
\pgfpathlineto{\pgfqpoint{4.120332in}{1.086700in}}%
\pgfpathlineto{\pgfqpoint{4.145364in}{1.091454in}}%
\pgfpathlineto{\pgfqpoint{4.170396in}{1.082655in}}%
\pgfpathlineto{\pgfqpoint{4.195429in}{1.070089in}}%
\pgfpathlineto{\pgfqpoint{4.220461in}{1.058368in}}%
\pgfpathlineto{\pgfqpoint{4.245494in}{1.050061in}}%
\pgfpathlineto{\pgfqpoint{4.270526in}{1.046378in}}%
\pgfpathlineto{\pgfqpoint{4.295558in}{1.047258in}}%
\pgfpathlineto{\pgfqpoint{4.320591in}{1.051680in}}%
\pgfpathlineto{\pgfqpoint{4.345623in}{1.058135in}}%
\pgfpathlineto{\pgfqpoint{4.370655in}{1.065131in}}%
\pgfpathlineto{\pgfqpoint{4.395688in}{1.071586in}}%
\pgfpathlineto{\pgfqpoint{4.420720in}{1.077047in}}%
\pgfpathlineto{\pgfqpoint{4.445753in}{1.081750in}}%
\pgfpathlineto{\pgfqpoint{4.470785in}{1.086574in}}%
\pgfpathlineto{\pgfqpoint{4.495817in}{1.092927in}}%
\pgfpathlineto{\pgfqpoint{4.520850in}{1.102564in}}%
\pgfpathlineto{\pgfqpoint{4.545882in}{1.117330in}}%
\pgfpathlineto{\pgfqpoint{4.570915in}{1.138821in}}%
\pgfpathlineto{\pgfqpoint{4.595947in}{1.168015in}}%
\pgfpathlineto{\pgfqpoint{4.620979in}{1.204934in}}%
\pgfpathlineto{\pgfqpoint{4.646012in}{1.248332in}}%
\pgfpathlineto{\pgfqpoint{4.671044in}{1.293521in}}%
\pgfpathlineto{\pgfqpoint{4.696077in}{1.280276in}}%
\pgfpathlineto{\pgfqpoint{4.721109in}{1.234430in}}%
\pgfpathlineto{\pgfqpoint{4.746141in}{1.192856in}}%
\pgfpathlineto{\pgfqpoint{4.771174in}{1.158368in}}%
\pgfpathlineto{\pgfqpoint{4.796206in}{1.131830in}}%
\pgfpathlineto{\pgfqpoint{4.821238in}{1.112982in}}%
\pgfpathlineto{\pgfqpoint{4.846271in}{1.100759in}}%
\pgfpathlineto{\pgfqpoint{4.871303in}{1.093656in}}%
\pgfpathlineto{\pgfqpoint{4.896336in}{1.090091in}}%
\pgfpathlineto{\pgfqpoint{4.921368in}{1.088688in}}%
\pgfpathlineto{\pgfqpoint{4.946400in}{1.088420in}}%
\pgfpathlineto{\pgfqpoint{4.971433in}{1.088630in}}%
\pgfpathlineto{\pgfqpoint{4.996465in}{1.088960in}}%
\pgfpathlineto{\pgfqpoint{5.021498in}{1.089254in}}%
\pgfpathlineto{\pgfqpoint{5.046530in}{1.089467in}}%
\pgfpathlineto{\pgfqpoint{5.071562in}{1.089604in}}%
\pgfpathlineto{\pgfqpoint{5.096595in}{1.089685in}}%
\pgfpathlineto{\pgfqpoint{5.121627in}{1.089730in}}%
\pgfpathlineto{\pgfqpoint{5.146660in}{1.089753in}}%
\pgfpathlineto{\pgfqpoint{5.171692in}{1.089765in}}%
\pgfpathlineto{\pgfqpoint{5.196724in}{1.089770in}}%
\pgfpathlineto{\pgfqpoint{5.221757in}{1.089773in}}%
\pgfpathlineto{\pgfqpoint{5.246789in}{1.089774in}}%
\pgfpathlineto{\pgfqpoint{5.271822in}{1.089774in}}%
\pgfpathlineto{\pgfqpoint{5.296854in}{1.089774in}}%
\pgfpathlineto{\pgfqpoint{5.321886in}{1.089774in}}%
\pgfpathlineto{\pgfqpoint{5.346919in}{1.089775in}}%
\pgfpathlineto{\pgfqpoint{5.371951in}{1.089775in}}%
\pgfpathlineto{\pgfqpoint{5.396984in}{1.089775in}}%
\pgfpathlineto{\pgfqpoint{5.396984in}{1.595220in}}%
\pgfpathlineto{\pgfqpoint{5.396984in}{1.595220in}}%
\pgfpathlineto{\pgfqpoint{5.371951in}{1.595220in}}%
\pgfpathlineto{\pgfqpoint{5.346919in}{1.595220in}}%
\pgfpathlineto{\pgfqpoint{5.321886in}{1.595220in}}%
\pgfpathlineto{\pgfqpoint{5.296854in}{1.595220in}}%
\pgfpathlineto{\pgfqpoint{5.271822in}{1.595219in}}%
\pgfpathlineto{\pgfqpoint{5.246789in}{1.595219in}}%
\pgfpathlineto{\pgfqpoint{5.221757in}{1.595218in}}%
\pgfpathlineto{\pgfqpoint{5.196724in}{1.595215in}}%
\pgfpathlineto{\pgfqpoint{5.171692in}{1.595210in}}%
\pgfpathlineto{\pgfqpoint{5.146660in}{1.595198in}}%
\pgfpathlineto{\pgfqpoint{5.121627in}{1.595174in}}%
\pgfpathlineto{\pgfqpoint{5.096595in}{1.595127in}}%
\pgfpathlineto{\pgfqpoint{5.071562in}{1.595038in}}%
\pgfpathlineto{\pgfqpoint{5.046530in}{1.594873in}}%
\pgfpathlineto{\pgfqpoint{5.021498in}{1.594574in}}%
\pgfpathlineto{\pgfqpoint{4.996465in}{1.594039in}}%
\pgfpathlineto{\pgfqpoint{4.971433in}{1.593088in}}%
\pgfpathlineto{\pgfqpoint{4.946400in}{1.591409in}}%
\pgfpathlineto{\pgfqpoint{4.921368in}{1.588489in}}%
\pgfpathlineto{\pgfqpoint{4.896336in}{1.583545in}}%
\pgfpathlineto{\pgfqpoint{4.871303in}{1.575496in}}%
\pgfpathlineto{\pgfqpoint{4.846271in}{1.563021in}}%
\pgfpathlineto{\pgfqpoint{4.821238in}{1.544741in}}%
\pgfpathlineto{\pgfqpoint{4.796206in}{1.519508in}}%
\pgfpathlineto{\pgfqpoint{4.771174in}{1.486751in}}%
\pgfpathlineto{\pgfqpoint{4.746141in}{1.446784in}}%
\pgfpathlineto{\pgfqpoint{4.721109in}{1.401039in}}%
\pgfpathlineto{\pgfqpoint{4.696077in}{1.352797in}}%
\pgfpathlineto{\pgfqpoint{4.671044in}{1.339206in}}%
\pgfpathlineto{\pgfqpoint{4.646012in}{1.386128in}}%
\pgfpathlineto{\pgfqpoint{4.620979in}{1.433101in}}%
\pgfpathlineto{\pgfqpoint{4.595947in}{1.474971in}}%
\pgfpathlineto{\pgfqpoint{4.570915in}{1.509850in}}%
\pgfpathlineto{\pgfqpoint{4.545882in}{1.537021in}}%
\pgfpathlineto{\pgfqpoint{4.520850in}{1.556671in}}%
\pgfpathlineto{\pgfqpoint{4.495817in}{1.569582in}}%
\pgfpathlineto{\pgfqpoint{4.470785in}{1.576739in}}%
\pgfpathlineto{\pgfqpoint{4.445753in}{1.578960in}}%
\pgfpathlineto{\pgfqpoint{4.420720in}{1.576609in}}%
\pgfpathlineto{\pgfqpoint{4.395688in}{1.569429in}}%
\pgfpathlineto{\pgfqpoint{4.370655in}{1.556520in}}%
\pgfpathlineto{\pgfqpoint{4.345623in}{1.536481in}}%
\pgfpathlineto{\pgfqpoint{4.320591in}{1.507733in}}%
\pgfpathlineto{\pgfqpoint{4.295558in}{1.469037in}}%
\pgfpathlineto{\pgfqpoint{4.270526in}{1.420135in}}%
\pgfpathlineto{\pgfqpoint{4.245494in}{1.362346in}}%
\pgfpathlineto{\pgfqpoint{4.220461in}{1.298935in}}%
\pgfpathlineto{\pgfqpoint{4.195429in}{1.235047in}}%
\pgfpathlineto{\pgfqpoint{4.170396in}{1.177236in}}%
\pgfpathlineto{\pgfqpoint{4.145364in}{1.133512in}}%
\pgfpathlineto{\pgfqpoint{4.120332in}{1.115914in}}%
\pgfpathlineto{\pgfqpoint{4.095299in}{1.114614in}}%
\pgfpathlineto{\pgfqpoint{4.070267in}{1.114279in}}%
\pgfpathlineto{\pgfqpoint{4.045234in}{1.121644in}}%
\pgfpathlineto{\pgfqpoint{4.020202in}{1.143986in}}%
\pgfpathlineto{\pgfqpoint{3.995170in}{1.170505in}}%
\pgfpathlineto{\pgfqpoint{3.970137in}{1.192756in}}%
\pgfpathlineto{\pgfqpoint{3.945105in}{1.206674in}}%
\pgfpathlineto{\pgfqpoint{3.920072in}{1.211534in}}%
\pgfpathlineto{\pgfqpoint{3.895040in}{1.209770in}}%
\pgfpathlineto{\pgfqpoint{3.870008in}{1.207980in}}%
\pgfpathlineto{\pgfqpoint{3.844975in}{1.208304in}}%
\pgfpathlineto{\pgfqpoint{3.819943in}{1.198888in}}%
\pgfpathlineto{\pgfqpoint{3.794911in}{1.176230in}}%
\pgfpathlineto{\pgfqpoint{3.769878in}{1.145404in}}%
\pgfpathlineto{\pgfqpoint{3.744846in}{1.114153in}}%
\pgfpathlineto{\pgfqpoint{3.719813in}{1.084846in}}%
\pgfpathlineto{\pgfqpoint{3.694781in}{1.059957in}}%
\pgfpathlineto{\pgfqpoint{3.669749in}{1.050091in}}%
\pgfpathlineto{\pgfqpoint{3.644716in}{1.064191in}}%
\pgfpathlineto{\pgfqpoint{3.619684in}{1.095965in}}%
\pgfpathlineto{\pgfqpoint{3.594651in}{1.136448in}}%
\pgfpathlineto{\pgfqpoint{3.569619in}{1.177042in}}%
\pgfpathlineto{\pgfqpoint{3.544587in}{1.210889in}}%
\pgfpathlineto{\pgfqpoint{3.519554in}{1.239381in}}%
\pgfpathlineto{\pgfqpoint{3.494522in}{1.301820in}}%
\pgfpathlineto{\pgfqpoint{3.469489in}{1.375531in}}%
\pgfpathlineto{\pgfqpoint{3.444457in}{1.442217in}}%
\pgfpathlineto{\pgfqpoint{3.419425in}{1.497187in}}%
\pgfpathlineto{\pgfqpoint{3.394392in}{1.538775in}}%
\pgfpathlineto{\pgfqpoint{3.369360in}{1.567485in}}%
\pgfpathlineto{\pgfqpoint{3.344327in}{1.585291in}}%
\pgfpathlineto{\pgfqpoint{3.319295in}{1.594886in}}%
\pgfpathlineto{\pgfqpoint{3.294263in}{1.599003in}}%
\pgfpathlineto{\pgfqpoint{3.269230in}{1.599948in}}%
\pgfpathlineto{\pgfqpoint{3.244198in}{1.599381in}}%
\pgfpathlineto{\pgfqpoint{3.219166in}{1.598327in}}%
\pgfpathlineto{\pgfqpoint{3.194133in}{1.597305in}}%
\pgfpathlineto{\pgfqpoint{3.169101in}{1.596512in}}%
\pgfpathlineto{\pgfqpoint{3.144068in}{1.595971in}}%
\pgfpathlineto{\pgfqpoint{3.119036in}{1.595632in}}%
\pgfpathlineto{\pgfqpoint{3.094004in}{1.595435in}}%
\pgfpathlineto{\pgfqpoint{3.068971in}{1.595327in}}%
\pgfpathlineto{\pgfqpoint{3.043939in}{1.595271in}}%
\pgfpathlineto{\pgfqpoint{3.018906in}{1.595243in}}%
\pgfpathlineto{\pgfqpoint{2.993874in}{1.595230in}}%
\pgfpathlineto{\pgfqpoint{2.968842in}{1.595224in}}%
\pgfpathlineto{\pgfqpoint{2.943809in}{1.595221in}}%
\pgfpathlineto{\pgfqpoint{2.918777in}{1.595220in}}%
\pgfpathclose%
\pgfusepath{fill}%
}%
\begin{pgfscope}%
\pgfsys@transformshift{0.000000in}{0.000000in}%
\pgfsys@useobject{currentmarker}{}%
\end{pgfscope}%
\end{pgfscope}%
\begin{pgfscope}%
\pgfpathrectangle{\pgfqpoint{2.918777in}{0.881091in}}{\pgfqpoint{2.478207in}{0.922812in}}%
\pgfusepath{clip}%
\pgfsetbuttcap%
\pgfsetroundjoin%
\pgfsetlinewidth{0.752812pt}%
\definecolor{currentstroke}{rgb}{0.000000,0.000000,0.000000}%
\pgfsetstrokecolor{currentstroke}%
\pgfsetdash{{2.775000pt}{1.200000pt}}{0.000000pt}%
\pgfpathmoveto{\pgfqpoint{3.331811in}{0.881091in}}%
\pgfpathlineto{\pgfqpoint{3.331811in}{1.803903in}}%
\pgfusepath{stroke}%
\end{pgfscope}%
\begin{pgfscope}%
\pgfpathrectangle{\pgfqpoint{2.918777in}{0.881091in}}{\pgfqpoint{2.478207in}{0.922812in}}%
\pgfusepath{clip}%
\pgfsetbuttcap%
\pgfsetroundjoin%
\pgfsetlinewidth{0.752812pt}%
\definecolor{currentstroke}{rgb}{0.000000,0.000000,0.000000}%
\pgfsetstrokecolor{currentstroke}%
\pgfsetdash{{2.775000pt}{1.200000pt}}{0.000000pt}%
\pgfpathmoveto{\pgfqpoint{4.983949in}{0.881091in}}%
\pgfpathlineto{\pgfqpoint{4.983949in}{1.803903in}}%
\pgfusepath{stroke}%
\end{pgfscope}%
\begin{pgfscope}%
\pgfpathrectangle{\pgfqpoint{2.918777in}{0.881091in}}{\pgfqpoint{2.478207in}{0.922812in}}%
\pgfusepath{clip}%
\pgfsetbuttcap%
\pgfsetroundjoin%
\pgfsetlinewidth{1.505625pt}%
\definecolor{currentstroke}{rgb}{0.631373,0.062745,0.207843}%
\pgfsetstrokecolor{currentstroke}%
\pgfsetdash{{1.500000pt}{2.475000pt}}{0.000000pt}%
\pgfpathmoveto{\pgfqpoint{3.652297in}{0.881091in}}%
\pgfpathlineto{\pgfqpoint{3.652297in}{1.803903in}}%
\pgfusepath{stroke}%
\end{pgfscope}%
\begin{pgfscope}%
\pgfpathrectangle{\pgfqpoint{2.918777in}{0.881091in}}{\pgfqpoint{2.478207in}{0.922812in}}%
\pgfusepath{clip}%
\pgfsetbuttcap%
\pgfsetroundjoin%
\definecolor{currentfill}{rgb}{0.000000,0.000000,0.000000}%
\pgfsetfillcolor{currentfill}%
\pgfsetlinewidth{1.003750pt}%
\definecolor{currentstroke}{rgb}{0.000000,0.000000,0.000000}%
\pgfsetstrokecolor{currentstroke}%
\pgfsetdash{}{0pt}%
\pgfsys@defobject{currentmarker}{\pgfqpoint{-0.020833in}{-0.020833in}}{\pgfqpoint{0.020833in}{0.020833in}}{%
\pgfpathmoveto{\pgfqpoint{0.000000in}{-0.020833in}}%
\pgfpathcurveto{\pgfqpoint{0.005525in}{-0.020833in}}{\pgfqpoint{0.010825in}{-0.018638in}}{\pgfqpoint{0.014731in}{-0.014731in}}%
\pgfpathcurveto{\pgfqpoint{0.018638in}{-0.010825in}}{\pgfqpoint{0.020833in}{-0.005525in}}{\pgfqpoint{0.020833in}{0.000000in}}%
\pgfpathcurveto{\pgfqpoint{0.020833in}{0.005525in}}{\pgfqpoint{0.018638in}{0.010825in}}{\pgfqpoint{0.014731in}{0.014731in}}%
\pgfpathcurveto{\pgfqpoint{0.010825in}{0.018638in}}{\pgfqpoint{0.005525in}{0.020833in}}{\pgfqpoint{0.000000in}{0.020833in}}%
\pgfpathcurveto{\pgfqpoint{-0.005525in}{0.020833in}}{\pgfqpoint{-0.010825in}{0.018638in}}{\pgfqpoint{-0.014731in}{0.014731in}}%
\pgfpathcurveto{\pgfqpoint{-0.018638in}{0.010825in}}{\pgfqpoint{-0.020833in}{0.005525in}}{\pgfqpoint{-0.020833in}{0.000000in}}%
\pgfpathcurveto{\pgfqpoint{-0.020833in}{-0.005525in}}{\pgfqpoint{-0.018638in}{-0.010825in}}{\pgfqpoint{-0.014731in}{-0.014731in}}%
\pgfpathcurveto{\pgfqpoint{-0.010825in}{-0.018638in}}{\pgfqpoint{-0.005525in}{-0.020833in}}{\pgfqpoint{0.000000in}{-0.020833in}}%
\pgfpathclose%
\pgfusepath{stroke,fill}%
}%
\begin{pgfscope}%
\pgfsys@transformshift{3.516296in}{1.230492in}%
\pgfsys@useobject{currentmarker}{}%
\end{pgfscope}%
\begin{pgfscope}%
\pgfsys@transformshift{4.679697in}{1.316216in}%
\pgfsys@useobject{currentmarker}{}%
\end{pgfscope}%
\begin{pgfscope}%
\pgfsys@transformshift{3.765600in}{1.126582in}%
\pgfsys@useobject{currentmarker}{}%
\end{pgfscope}%
\begin{pgfscope}%
\pgfsys@transformshift{4.131244in}{1.104739in}%
\pgfsys@useobject{currentmarker}{}%
\end{pgfscope}%
\begin{pgfscope}%
\pgfsys@transformshift{4.051712in}{1.103542in}%
\pgfsys@useobject{currentmarker}{}%
\end{pgfscope}%
\begin{pgfscope}%
\pgfsys@transformshift{3.873031in}{1.192854in}%
\pgfsys@useobject{currentmarker}{}%
\end{pgfscope}%
\begin{pgfscope}%
\pgfsys@transformshift{3.677888in}{1.034515in}%
\pgfsys@useobject{currentmarker}{}%
\end{pgfscope}%
\end{pgfscope}%
\begin{pgfscope}%
\pgfpathrectangle{\pgfqpoint{2.918777in}{0.881091in}}{\pgfqpoint{2.478207in}{0.922812in}}%
\pgfusepath{clip}%
\pgfsetrectcap%
\pgfsetroundjoin%
\pgfsetlinewidth{0.752812pt}%
\definecolor{currentstroke}{rgb}{0.000000,0.329412,0.623529}%
\pgfsetstrokecolor{currentstroke}%
\pgfsetdash{}{0pt}%
\pgfpathmoveto{\pgfqpoint{2.918777in}{1.342498in}}%
\pgfpathlineto{\pgfqpoint{2.943809in}{1.342499in}}%
\pgfpathlineto{\pgfqpoint{2.968842in}{1.342501in}}%
\pgfpathlineto{\pgfqpoint{2.993874in}{1.342507in}}%
\pgfpathlineto{\pgfqpoint{3.018906in}{1.342520in}}%
\pgfpathlineto{\pgfqpoint{3.043939in}{1.342548in}}%
\pgfpathlineto{\pgfqpoint{3.068971in}{1.342605in}}%
\pgfpathlineto{\pgfqpoint{3.094004in}{1.342714in}}%
\pgfpathlineto{\pgfqpoint{3.119036in}{1.342916in}}%
\pgfpathlineto{\pgfqpoint{3.144068in}{1.343270in}}%
\pgfpathlineto{\pgfqpoint{3.169101in}{1.343861in}}%
\pgfpathlineto{\pgfqpoint{3.194133in}{1.344794in}}%
\pgfpathlineto{\pgfqpoint{3.219166in}{1.346180in}}%
\pgfpathlineto{\pgfqpoint{3.244198in}{1.348102in}}%
\pgfpathlineto{\pgfqpoint{3.269230in}{1.350559in}}%
\pgfpathlineto{\pgfqpoint{3.294263in}{1.353376in}}%
\pgfpathlineto{\pgfqpoint{3.319295in}{1.356107in}}%
\pgfpathlineto{\pgfqpoint{3.344327in}{1.357938in}}%
\pgfpathlineto{\pgfqpoint{3.369360in}{1.357627in}}%
\pgfpathlineto{\pgfqpoint{3.394392in}{1.353546in}}%
\pgfpathlineto{\pgfqpoint{3.419425in}{1.343837in}}%
\pgfpathlineto{\pgfqpoint{3.444457in}{1.326737in}}%
\pgfpathlineto{\pgfqpoint{3.469489in}{1.301021in}}%
\pgfpathlineto{\pgfqpoint{3.494522in}{1.266498in}}%
\pgfpathlineto{\pgfqpoint{3.519554in}{1.224425in}}%
\pgfpathlineto{\pgfqpoint{3.544587in}{1.177682in}}%
\pgfpathlineto{\pgfqpoint{3.569619in}{1.130588in}}%
\pgfpathlineto{\pgfqpoint{3.594651in}{1.088293in}}%
\pgfpathlineto{\pgfqpoint{3.619684in}{1.055846in}}%
\pgfpathlineto{\pgfqpoint{3.644716in}{1.037138in}}%
\pgfpathlineto{\pgfqpoint{3.669749in}{1.033988in}}%
\pgfpathlineto{\pgfqpoint{3.694781in}{1.045678in}}%
\pgfpathlineto{\pgfqpoint{3.719813in}{1.069063in}}%
\pgfpathlineto{\pgfqpoint{3.744846in}{1.099276in}}%
\pgfpathlineto{\pgfqpoint{3.769878in}{1.130814in}}%
\pgfpathlineto{\pgfqpoint{3.794911in}{1.158688in}}%
\pgfpathlineto{\pgfqpoint{3.819943in}{1.179320in}}%
\pgfpathlineto{\pgfqpoint{3.844975in}{1.190970in}}%
\pgfpathlineto{\pgfqpoint{3.870008in}{1.193658in}}%
\pgfpathlineto{\pgfqpoint{3.895040in}{1.188690in}}%
\pgfpathlineto{\pgfqpoint{3.920072in}{1.178044in}}%
\pgfpathlineto{\pgfqpoint{3.945105in}{1.163824in}}%
\pgfpathlineto{\pgfqpoint{3.970137in}{1.147946in}}%
\pgfpathlineto{\pgfqpoint{3.995170in}{1.132054in}}%
\pgfpathlineto{\pgfqpoint{4.020202in}{1.117617in}}%
\pgfpathlineto{\pgfqpoint{4.045234in}{1.106039in}}%
\pgfpathlineto{\pgfqpoint{4.070267in}{1.098694in}}%
\pgfpathlineto{\pgfqpoint{4.095299in}{1.096818in}}%
\pgfpathlineto{\pgfqpoint{4.120332in}{1.101307in}}%
\pgfpathlineto{\pgfqpoint{4.145364in}{1.112483in}}%
\pgfpathlineto{\pgfqpoint{4.170396in}{1.129946in}}%
\pgfpathlineto{\pgfqpoint{4.195429in}{1.152568in}}%
\pgfpathlineto{\pgfqpoint{4.220461in}{1.178651in}}%
\pgfpathlineto{\pgfqpoint{4.245494in}{1.206204in}}%
\pgfpathlineto{\pgfqpoint{4.270526in}{1.233256in}}%
\pgfpathlineto{\pgfqpoint{4.295558in}{1.258148in}}%
\pgfpathlineto{\pgfqpoint{4.320591in}{1.279706in}}%
\pgfpathlineto{\pgfqpoint{4.345623in}{1.297308in}}%
\pgfpathlineto{\pgfqpoint{4.370655in}{1.310826in}}%
\pgfpathlineto{\pgfqpoint{4.395688in}{1.320507in}}%
\pgfpathlineto{\pgfqpoint{4.420720in}{1.326828in}}%
\pgfpathlineto{\pgfqpoint{4.445753in}{1.330355in}}%
\pgfpathlineto{\pgfqpoint{4.470785in}{1.331656in}}%
\pgfpathlineto{\pgfqpoint{4.495817in}{1.331255in}}%
\pgfpathlineto{\pgfqpoint{4.520850in}{1.329618in}}%
\pgfpathlineto{\pgfqpoint{4.545882in}{1.327175in}}%
\pgfpathlineto{\pgfqpoint{4.570915in}{1.324336in}}%
\pgfpathlineto{\pgfqpoint{4.595947in}{1.321493in}}%
\pgfpathlineto{\pgfqpoint{4.620979in}{1.319018in}}%
\pgfpathlineto{\pgfqpoint{4.646012in}{1.317230in}}%
\pgfpathlineto{\pgfqpoint{4.671044in}{1.316364in}}%
\pgfpathlineto{\pgfqpoint{4.696077in}{1.316536in}}%
\pgfpathlineto{\pgfqpoint{4.721109in}{1.317734in}}%
\pgfpathlineto{\pgfqpoint{4.746141in}{1.319820in}}%
\pgfpathlineto{\pgfqpoint{4.771174in}{1.322559in}}%
\pgfpathlineto{\pgfqpoint{4.796206in}{1.325669in}}%
\pgfpathlineto{\pgfqpoint{4.821238in}{1.328861in}}%
\pgfpathlineto{\pgfqpoint{4.846271in}{1.331890in}}%
\pgfpathlineto{\pgfqpoint{4.871303in}{1.334576in}}%
\pgfpathlineto{\pgfqpoint{4.896336in}{1.336818in}}%
\pgfpathlineto{\pgfqpoint{4.921368in}{1.338589in}}%
\pgfpathlineto{\pgfqpoint{4.946400in}{1.339915in}}%
\pgfpathlineto{\pgfqpoint{4.971433in}{1.340859in}}%
\pgfpathlineto{\pgfqpoint{4.996465in}{1.341500in}}%
\pgfpathlineto{\pgfqpoint{5.021498in}{1.341914in}}%
\pgfpathlineto{\pgfqpoint{5.046530in}{1.342170in}}%
\pgfpathlineto{\pgfqpoint{5.071562in}{1.342321in}}%
\pgfpathlineto{\pgfqpoint{5.096595in}{1.342406in}}%
\pgfpathlineto{\pgfqpoint{5.121627in}{1.342452in}}%
\pgfpathlineto{\pgfqpoint{5.146660in}{1.342476in}}%
\pgfpathlineto{\pgfqpoint{5.171692in}{1.342487in}}%
\pgfpathlineto{\pgfqpoint{5.196724in}{1.342493in}}%
\pgfpathlineto{\pgfqpoint{5.221757in}{1.342495in}}%
\pgfpathlineto{\pgfqpoint{5.246789in}{1.342496in}}%
\pgfpathlineto{\pgfqpoint{5.271822in}{1.342497in}}%
\pgfpathlineto{\pgfqpoint{5.296854in}{1.342497in}}%
\pgfpathlineto{\pgfqpoint{5.321886in}{1.342497in}}%
\pgfpathlineto{\pgfqpoint{5.346919in}{1.342497in}}%
\pgfpathlineto{\pgfqpoint{5.371951in}{1.342497in}}%
\pgfpathlineto{\pgfqpoint{5.396984in}{1.342497in}}%
\pgfusepath{stroke}%
\end{pgfscope}%
\begin{pgfscope}%
\pgfpathrectangle{\pgfqpoint{2.918777in}{0.881091in}}{\pgfqpoint{2.478207in}{0.922812in}}%
\pgfusepath{clip}%
\pgfsetrectcap%
\pgfsetroundjoin%
\pgfsetlinewidth{0.752812pt}%
\definecolor{currentstroke}{rgb}{0.000000,0.000000,0.000000}%
\pgfsetstrokecolor{currentstroke}%
\pgfsetdash{}{0pt}%
\pgfpathmoveto{\pgfqpoint{3.101702in}{1.813903in}}%
\pgfpathlineto{\pgfqpoint{3.119036in}{1.744910in}}%
\pgfpathlineto{\pgfqpoint{3.144068in}{1.678187in}}%
\pgfpathlineto{\pgfqpoint{3.169101in}{1.641080in}}%
\pgfpathlineto{\pgfqpoint{3.194133in}{1.628273in}}%
\pgfpathlineto{\pgfqpoint{3.219166in}{1.632987in}}%
\pgfpathlineto{\pgfqpoint{3.244198in}{1.647656in}}%
\pgfpathlineto{\pgfqpoint{3.269230in}{1.664651in}}%
\pgfpathlineto{\pgfqpoint{3.294263in}{1.676966in}}%
\pgfpathlineto{\pgfqpoint{3.319295in}{1.678831in}}%
\pgfpathlineto{\pgfqpoint{3.344327in}{1.666167in}}%
\pgfpathlineto{\pgfqpoint{3.369360in}{1.636859in}}%
\pgfpathlineto{\pgfqpoint{3.394392in}{1.590834in}}%
\pgfpathlineto{\pgfqpoint{3.419425in}{1.529933in}}%
\pgfpathlineto{\pgfqpoint{3.444457in}{1.457606in}}%
\pgfpathlineto{\pgfqpoint{3.469489in}{1.378473in}}%
\pgfpathlineto{\pgfqpoint{3.494522in}{1.297788in}}%
\pgfpathlineto{\pgfqpoint{3.519554in}{1.220873in}}%
\pgfpathlineto{\pgfqpoint{3.544587in}{1.152579in}}%
\pgfpathlineto{\pgfqpoint{3.569619in}{1.096823in}}%
\pgfpathlineto{\pgfqpoint{3.594651in}{1.056247in}}%
\pgfpathlineto{\pgfqpoint{3.619684in}{1.032021in}}%
\pgfpathlineto{\pgfqpoint{3.644716in}{1.023811in}}%
\pgfpathlineto{\pgfqpoint{3.669749in}{1.029902in}}%
\pgfpathlineto{\pgfqpoint{3.694781in}{1.047453in}}%
\pgfpathlineto{\pgfqpoint{3.719813in}{1.072855in}}%
\pgfpathlineto{\pgfqpoint{3.744846in}{1.102143in}}%
\pgfpathlineto{\pgfqpoint{3.769878in}{1.131431in}}%
\pgfpathlineto{\pgfqpoint{3.794911in}{1.157291in}}%
\pgfpathlineto{\pgfqpoint{3.819943in}{1.177083in}}%
\pgfpathlineto{\pgfqpoint{3.844975in}{1.189157in}}%
\pgfpathlineto{\pgfqpoint{3.870008in}{1.192944in}}%
\pgfpathlineto{\pgfqpoint{3.895040in}{1.188923in}}%
\pgfpathlineto{\pgfqpoint{3.920072in}{1.178463in}}%
\pgfpathlineto{\pgfqpoint{3.945105in}{1.163581in}}%
\pgfpathlineto{\pgfqpoint{3.970137in}{1.146634in}}%
\pgfpathlineto{\pgfqpoint{3.995170in}{1.129999in}}%
\pgfpathlineto{\pgfqpoint{4.020202in}{1.115759in}}%
\pgfpathlineto{\pgfqpoint{4.045234in}{1.105451in}}%
\pgfpathlineto{\pgfqpoint{4.070267in}{1.099892in}}%
\pgfpathlineto{\pgfqpoint{4.095299in}{1.099109in}}%
\pgfpathlineto{\pgfqpoint{4.120332in}{1.102374in}}%
\pgfpathlineto{\pgfqpoint{4.145364in}{1.108339in}}%
\pgfpathlineto{\pgfqpoint{4.170396in}{1.115249in}}%
\pgfpathlineto{\pgfqpoint{4.195429in}{1.121217in}}%
\pgfpathlineto{\pgfqpoint{4.220461in}{1.124504in}}%
\pgfpathlineto{\pgfqpoint{4.245494in}{1.123790in}}%
\pgfpathlineto{\pgfqpoint{4.270526in}{1.118392in}}%
\pgfpathlineto{\pgfqpoint{4.295558in}{1.108398in}}%
\pgfpathlineto{\pgfqpoint{4.320591in}{1.094700in}}%
\pgfpathlineto{\pgfqpoint{4.345623in}{1.078924in}}%
\pgfpathlineto{\pgfqpoint{4.370655in}{1.063257in}}%
\pgfpathlineto{\pgfqpoint{4.395688in}{1.050195in}}%
\pgfpathlineto{\pgfqpoint{4.420720in}{1.042231in}}%
\pgfpathlineto{\pgfqpoint{4.445753in}{1.041531in}}%
\pgfpathlineto{\pgfqpoint{4.470785in}{1.049629in}}%
\pgfpathlineto{\pgfqpoint{4.495817in}{1.067181in}}%
\pgfpathlineto{\pgfqpoint{4.520850in}{1.093815in}}%
\pgfpathlineto{\pgfqpoint{4.545882in}{1.128089in}}%
\pgfpathlineto{\pgfqpoint{4.570915in}{1.167585in}}%
\pgfpathlineto{\pgfqpoint{4.595947in}{1.209118in}}%
\pgfpathlineto{\pgfqpoint{4.620979in}{1.249053in}}%
\pgfpathlineto{\pgfqpoint{4.646012in}{1.283695in}}%
\pgfpathlineto{\pgfqpoint{4.671044in}{1.309719in}}%
\pgfpathlineto{\pgfqpoint{4.696077in}{1.324582in}}%
\pgfpathlineto{\pgfqpoint{4.721109in}{1.326884in}}%
\pgfpathlineto{\pgfqpoint{4.746141in}{1.316619in}}%
\pgfpathlineto{\pgfqpoint{4.771174in}{1.295300in}}%
\pgfpathlineto{\pgfqpoint{4.796206in}{1.265929in}}%
\pgfpathlineto{\pgfqpoint{4.821238in}{1.232795in}}%
\pgfpathlineto{\pgfqpoint{4.846271in}{1.201138in}}%
\pgfpathlineto{\pgfqpoint{4.871303in}{1.176684in}}%
\pgfpathlineto{\pgfqpoint{4.896336in}{1.165105in}}%
\pgfpathlineto{\pgfqpoint{4.921368in}{1.171454in}}%
\pgfpathlineto{\pgfqpoint{4.946400in}{1.199629in}}%
\pgfpathlineto{\pgfqpoint{4.971433in}{1.251935in}}%
\pgfpathlineto{\pgfqpoint{4.996465in}{1.328776in}}%
\pgfpathlineto{\pgfqpoint{5.021498in}{1.428532in}}%
\pgfpathlineto{\pgfqpoint{5.046530in}{1.547629in}}%
\pgfpathlineto{\pgfqpoint{5.071562in}{1.680821in}}%
\pgfpathlineto{\pgfqpoint{5.095219in}{1.813903in}}%
\pgfusepath{stroke}%
\end{pgfscope}%
\begin{pgfscope}%
\pgfpathrectangle{\pgfqpoint{2.918777in}{0.881091in}}{\pgfqpoint{2.478207in}{0.922812in}}%
\pgfusepath{clip}%
\pgfsetbuttcap%
\pgfsetroundjoin%
\definecolor{currentfill}{rgb}{0.631373,0.062745,0.207843}%
\pgfsetfillcolor{currentfill}%
\pgfsetlinewidth{1.003750pt}%
\definecolor{currentstroke}{rgb}{0.631373,0.062745,0.207843}%
\pgfsetstrokecolor{currentstroke}%
\pgfsetdash{}{0pt}%
\pgfsys@defobject{currentmarker}{\pgfqpoint{-0.027778in}{-0.027778in}}{\pgfqpoint{0.027778in}{0.027778in}}{%
\pgfpathmoveto{\pgfqpoint{0.000000in}{-0.027778in}}%
\pgfpathcurveto{\pgfqpoint{0.007367in}{-0.027778in}}{\pgfqpoint{0.014433in}{-0.024851in}}{\pgfqpoint{0.019642in}{-0.019642in}}%
\pgfpathcurveto{\pgfqpoint{0.024851in}{-0.014433in}}{\pgfqpoint{0.027778in}{-0.007367in}}{\pgfqpoint{0.027778in}{0.000000in}}%
\pgfpathcurveto{\pgfqpoint{0.027778in}{0.007367in}}{\pgfqpoint{0.024851in}{0.014433in}}{\pgfqpoint{0.019642in}{0.019642in}}%
\pgfpathcurveto{\pgfqpoint{0.014433in}{0.024851in}}{\pgfqpoint{0.007367in}{0.027778in}}{\pgfqpoint{0.000000in}{0.027778in}}%
\pgfpathcurveto{\pgfqpoint{-0.007367in}{0.027778in}}{\pgfqpoint{-0.014433in}{0.024851in}}{\pgfqpoint{-0.019642in}{0.019642in}}%
\pgfpathcurveto{\pgfqpoint{-0.024851in}{0.014433in}}{\pgfqpoint{-0.027778in}{0.007367in}}{\pgfqpoint{-0.027778in}{0.000000in}}%
\pgfpathcurveto{\pgfqpoint{-0.027778in}{-0.007367in}}{\pgfqpoint{-0.024851in}{-0.014433in}}{\pgfqpoint{-0.019642in}{-0.019642in}}%
\pgfpathcurveto{\pgfqpoint{-0.014433in}{-0.024851in}}{\pgfqpoint{-0.007367in}{-0.027778in}}{\pgfqpoint{0.000000in}{-0.027778in}}%
\pgfpathclose%
\pgfusepath{stroke,fill}%
}%
\begin{pgfscope}%
\pgfsys@transformshift{3.652297in}{1.024257in}%
\pgfsys@useobject{currentmarker}{}%
\end{pgfscope}%
\end{pgfscope}%
\begin{pgfscope}%
\pgfsetrectcap%
\pgfsetmiterjoin%
\pgfsetlinewidth{0.752812pt}%
\definecolor{currentstroke}{rgb}{0.000000,0.000000,0.000000}%
\pgfsetstrokecolor{currentstroke}%
\pgfsetdash{}{0pt}%
\pgfpathmoveto{\pgfqpoint{2.918777in}{0.881091in}}%
\pgfpathlineto{\pgfqpoint{2.918777in}{1.803903in}}%
\pgfusepath{stroke}%
\end{pgfscope}%
\begin{pgfscope}%
\pgfsetrectcap%
\pgfsetmiterjoin%
\pgfsetlinewidth{0.752812pt}%
\definecolor{currentstroke}{rgb}{0.000000,0.000000,0.000000}%
\pgfsetstrokecolor{currentstroke}%
\pgfsetdash{}{0pt}%
\pgfpathmoveto{\pgfqpoint{5.396984in}{0.881091in}}%
\pgfpathlineto{\pgfqpoint{5.396984in}{1.803903in}}%
\pgfusepath{stroke}%
\end{pgfscope}%
\begin{pgfscope}%
\pgfsetrectcap%
\pgfsetmiterjoin%
\pgfsetlinewidth{0.752812pt}%
\definecolor{currentstroke}{rgb}{0.000000,0.000000,0.000000}%
\pgfsetstrokecolor{currentstroke}%
\pgfsetdash{}{0pt}%
\pgfpathmoveto{\pgfqpoint{2.918777in}{0.881091in}}%
\pgfpathlineto{\pgfqpoint{5.396984in}{0.881091in}}%
\pgfusepath{stroke}%
\end{pgfscope}%
\begin{pgfscope}%
\pgfsetrectcap%
\pgfsetmiterjoin%
\pgfsetlinewidth{0.752812pt}%
\definecolor{currentstroke}{rgb}{0.000000,0.000000,0.000000}%
\pgfsetstrokecolor{currentstroke}%
\pgfsetdash{}{0pt}%
\pgfpathmoveto{\pgfqpoint{2.918777in}{1.803903in}}%
\pgfpathlineto{\pgfqpoint{5.396984in}{1.803903in}}%
\pgfusepath{stroke}%
\end{pgfscope}%
\begin{pgfscope}%
\definecolor{textcolor}{rgb}{0.000000,0.000000,0.000000}%
\pgfsetstrokecolor{textcolor}%
\pgfsetfillcolor{textcolor}%
\pgftext[x=4.157880in,y=1.665481in,,base]{\color{textcolor}\rmfamily\fontsize{10.000000}{12.000000}\selectfont (4)}%
\end{pgfscope}%
\begin{pgfscope}%
\pgfsetbuttcap%
\pgfsetmiterjoin%
\definecolor{currentfill}{rgb}{1.000000,1.000000,1.000000}%
\pgfsetfillcolor{currentfill}%
\pgfsetlinewidth{0.000000pt}%
\definecolor{currentstroke}{rgb}{0.000000,0.000000,0.000000}%
\pgfsetstrokecolor{currentstroke}%
\pgfsetstrokeopacity{0.000000}%
\pgfsetdash{}{0pt}%
\pgfpathmoveto{\pgfqpoint{2.918777in}{0.408431in}}%
\pgfpathlineto{\pgfqpoint{5.396984in}{0.408431in}}%
\pgfpathlineto{\pgfqpoint{5.396984in}{0.858583in}}%
\pgfpathlineto{\pgfqpoint{2.918777in}{0.858583in}}%
\pgfpathclose%
\pgfusepath{fill}%
\end{pgfscope}%
\begin{pgfscope}%
\pgfsetbuttcap%
\pgfsetroundjoin%
\definecolor{currentfill}{rgb}{0.000000,0.000000,0.000000}%
\pgfsetfillcolor{currentfill}%
\pgfsetlinewidth{0.803000pt}%
\definecolor{currentstroke}{rgb}{0.000000,0.000000,0.000000}%
\pgfsetstrokecolor{currentstroke}%
\pgfsetdash{}{0pt}%
\pgfsys@defobject{currentmarker}{\pgfqpoint{0.000000in}{-0.048611in}}{\pgfqpoint{0.000000in}{0.000000in}}{%
\pgfpathmoveto{\pgfqpoint{0.000000in}{0.000000in}}%
\pgfpathlineto{\pgfqpoint{0.000000in}{-0.048611in}}%
\pgfusepath{stroke,fill}%
}%
\begin{pgfscope}%
\pgfsys@transformshift{2.918777in}{0.408431in}%
\pgfsys@useobject{currentmarker}{}%
\end{pgfscope}%
\end{pgfscope}%
\begin{pgfscope}%
\definecolor{textcolor}{rgb}{0.000000,0.000000,0.000000}%
\pgfsetstrokecolor{textcolor}%
\pgfsetfillcolor{textcolor}%
\pgftext[x=2.918777in,y=0.311209in,,top]{\color{textcolor}\rmfamily\fontsize{10.000000}{12.000000}\selectfont \(\displaystyle {\ensuremath{-}3}\)}%
\end{pgfscope}%
\begin{pgfscope}%
\pgfsetbuttcap%
\pgfsetroundjoin%
\definecolor{currentfill}{rgb}{0.000000,0.000000,0.000000}%
\pgfsetfillcolor{currentfill}%
\pgfsetlinewidth{0.803000pt}%
\definecolor{currentstroke}{rgb}{0.000000,0.000000,0.000000}%
\pgfsetstrokecolor{currentstroke}%
\pgfsetdash{}{0pt}%
\pgfsys@defobject{currentmarker}{\pgfqpoint{0.000000in}{-0.048611in}}{\pgfqpoint{0.000000in}{0.000000in}}{%
\pgfpathmoveto{\pgfqpoint{0.000000in}{0.000000in}}%
\pgfpathlineto{\pgfqpoint{0.000000in}{-0.048611in}}%
\pgfusepath{stroke,fill}%
}%
\begin{pgfscope}%
\pgfsys@transformshift{3.331811in}{0.408431in}%
\pgfsys@useobject{currentmarker}{}%
\end{pgfscope}%
\end{pgfscope}%
\begin{pgfscope}%
\definecolor{textcolor}{rgb}{0.000000,0.000000,0.000000}%
\pgfsetstrokecolor{textcolor}%
\pgfsetfillcolor{textcolor}%
\pgftext[x=3.331811in,y=0.311209in,,top]{\color{textcolor}\rmfamily\fontsize{10.000000}{12.000000}\selectfont \(\displaystyle {\ensuremath{-}2}\)}%
\end{pgfscope}%
\begin{pgfscope}%
\pgfsetbuttcap%
\pgfsetroundjoin%
\definecolor{currentfill}{rgb}{0.000000,0.000000,0.000000}%
\pgfsetfillcolor{currentfill}%
\pgfsetlinewidth{0.803000pt}%
\definecolor{currentstroke}{rgb}{0.000000,0.000000,0.000000}%
\pgfsetstrokecolor{currentstroke}%
\pgfsetdash{}{0pt}%
\pgfsys@defobject{currentmarker}{\pgfqpoint{0.000000in}{-0.048611in}}{\pgfqpoint{0.000000in}{0.000000in}}{%
\pgfpathmoveto{\pgfqpoint{0.000000in}{0.000000in}}%
\pgfpathlineto{\pgfqpoint{0.000000in}{-0.048611in}}%
\pgfusepath{stroke,fill}%
}%
\begin{pgfscope}%
\pgfsys@transformshift{3.744846in}{0.408431in}%
\pgfsys@useobject{currentmarker}{}%
\end{pgfscope}%
\end{pgfscope}%
\begin{pgfscope}%
\definecolor{textcolor}{rgb}{0.000000,0.000000,0.000000}%
\pgfsetstrokecolor{textcolor}%
\pgfsetfillcolor{textcolor}%
\pgftext[x=3.744846in,y=0.311209in,,top]{\color{textcolor}\rmfamily\fontsize{10.000000}{12.000000}\selectfont \(\displaystyle {\ensuremath{-}1}\)}%
\end{pgfscope}%
\begin{pgfscope}%
\pgfsetbuttcap%
\pgfsetroundjoin%
\definecolor{currentfill}{rgb}{0.000000,0.000000,0.000000}%
\pgfsetfillcolor{currentfill}%
\pgfsetlinewidth{0.803000pt}%
\definecolor{currentstroke}{rgb}{0.000000,0.000000,0.000000}%
\pgfsetstrokecolor{currentstroke}%
\pgfsetdash{}{0pt}%
\pgfsys@defobject{currentmarker}{\pgfqpoint{0.000000in}{-0.048611in}}{\pgfqpoint{0.000000in}{0.000000in}}{%
\pgfpathmoveto{\pgfqpoint{0.000000in}{0.000000in}}%
\pgfpathlineto{\pgfqpoint{0.000000in}{-0.048611in}}%
\pgfusepath{stroke,fill}%
}%
\begin{pgfscope}%
\pgfsys@transformshift{4.157880in}{0.408431in}%
\pgfsys@useobject{currentmarker}{}%
\end{pgfscope}%
\end{pgfscope}%
\begin{pgfscope}%
\definecolor{textcolor}{rgb}{0.000000,0.000000,0.000000}%
\pgfsetstrokecolor{textcolor}%
\pgfsetfillcolor{textcolor}%
\pgftext[x=4.157880in,y=0.311209in,,top]{\color{textcolor}\rmfamily\fontsize{10.000000}{12.000000}\selectfont \(\displaystyle {0}\)}%
\end{pgfscope}%
\begin{pgfscope}%
\pgfsetbuttcap%
\pgfsetroundjoin%
\definecolor{currentfill}{rgb}{0.000000,0.000000,0.000000}%
\pgfsetfillcolor{currentfill}%
\pgfsetlinewidth{0.803000pt}%
\definecolor{currentstroke}{rgb}{0.000000,0.000000,0.000000}%
\pgfsetstrokecolor{currentstroke}%
\pgfsetdash{}{0pt}%
\pgfsys@defobject{currentmarker}{\pgfqpoint{0.000000in}{-0.048611in}}{\pgfqpoint{0.000000in}{0.000000in}}{%
\pgfpathmoveto{\pgfqpoint{0.000000in}{0.000000in}}%
\pgfpathlineto{\pgfqpoint{0.000000in}{-0.048611in}}%
\pgfusepath{stroke,fill}%
}%
\begin{pgfscope}%
\pgfsys@transformshift{4.570915in}{0.408431in}%
\pgfsys@useobject{currentmarker}{}%
\end{pgfscope}%
\end{pgfscope}%
\begin{pgfscope}%
\definecolor{textcolor}{rgb}{0.000000,0.000000,0.000000}%
\pgfsetstrokecolor{textcolor}%
\pgfsetfillcolor{textcolor}%
\pgftext[x=4.570915in,y=0.311209in,,top]{\color{textcolor}\rmfamily\fontsize{10.000000}{12.000000}\selectfont \(\displaystyle {1}\)}%
\end{pgfscope}%
\begin{pgfscope}%
\pgfsetbuttcap%
\pgfsetroundjoin%
\definecolor{currentfill}{rgb}{0.000000,0.000000,0.000000}%
\pgfsetfillcolor{currentfill}%
\pgfsetlinewidth{0.803000pt}%
\definecolor{currentstroke}{rgb}{0.000000,0.000000,0.000000}%
\pgfsetstrokecolor{currentstroke}%
\pgfsetdash{}{0pt}%
\pgfsys@defobject{currentmarker}{\pgfqpoint{0.000000in}{-0.048611in}}{\pgfqpoint{0.000000in}{0.000000in}}{%
\pgfpathmoveto{\pgfqpoint{0.000000in}{0.000000in}}%
\pgfpathlineto{\pgfqpoint{0.000000in}{-0.048611in}}%
\pgfusepath{stroke,fill}%
}%
\begin{pgfscope}%
\pgfsys@transformshift{4.983949in}{0.408431in}%
\pgfsys@useobject{currentmarker}{}%
\end{pgfscope}%
\end{pgfscope}%
\begin{pgfscope}%
\definecolor{textcolor}{rgb}{0.000000,0.000000,0.000000}%
\pgfsetstrokecolor{textcolor}%
\pgfsetfillcolor{textcolor}%
\pgftext[x=4.983949in,y=0.311209in,,top]{\color{textcolor}\rmfamily\fontsize{10.000000}{12.000000}\selectfont \(\displaystyle {2}\)}%
\end{pgfscope}%
\begin{pgfscope}%
\pgfsetbuttcap%
\pgfsetroundjoin%
\definecolor{currentfill}{rgb}{0.000000,0.000000,0.000000}%
\pgfsetfillcolor{currentfill}%
\pgfsetlinewidth{0.803000pt}%
\definecolor{currentstroke}{rgb}{0.000000,0.000000,0.000000}%
\pgfsetstrokecolor{currentstroke}%
\pgfsetdash{}{0pt}%
\pgfsys@defobject{currentmarker}{\pgfqpoint{0.000000in}{-0.048611in}}{\pgfqpoint{0.000000in}{0.000000in}}{%
\pgfpathmoveto{\pgfqpoint{0.000000in}{0.000000in}}%
\pgfpathlineto{\pgfqpoint{0.000000in}{-0.048611in}}%
\pgfusepath{stroke,fill}%
}%
\begin{pgfscope}%
\pgfsys@transformshift{5.396984in}{0.408431in}%
\pgfsys@useobject{currentmarker}{}%
\end{pgfscope}%
\end{pgfscope}%
\begin{pgfscope}%
\definecolor{textcolor}{rgb}{0.000000,0.000000,0.000000}%
\pgfsetstrokecolor{textcolor}%
\pgfsetfillcolor{textcolor}%
\pgftext[x=5.396984in,y=0.311209in,,top]{\color{textcolor}\rmfamily\fontsize{10.000000}{12.000000}\selectfont \(\displaystyle {3}\)}%
\end{pgfscope}%
\begin{pgfscope}%
\definecolor{textcolor}{rgb}{0.000000,0.000000,0.000000}%
\pgfsetstrokecolor{textcolor}%
\pgfsetfillcolor{textcolor}%
\pgftext[x=4.157880in,y=0.132320in,,top]{\color{textcolor}\rmfamily\fontsize{10.000000}{12.000000}\selectfont \(\displaystyle x\)}%
\end{pgfscope}%
\begin{pgfscope}%
\pgfpathrectangle{\pgfqpoint{2.918777in}{0.408431in}}{\pgfqpoint{2.478207in}{0.450152in}}%
\pgfusepath{clip}%
\pgfsetbuttcap%
\pgfsetroundjoin%
\pgfsetlinewidth{0.752812pt}%
\definecolor{currentstroke}{rgb}{0.000000,0.000000,0.000000}%
\pgfsetstrokecolor{currentstroke}%
\pgfsetdash{{2.775000pt}{1.200000pt}}{0.000000pt}%
\pgfpathmoveto{\pgfqpoint{3.331811in}{0.408431in}}%
\pgfpathlineto{\pgfqpoint{3.331811in}{0.858583in}}%
\pgfusepath{stroke}%
\end{pgfscope}%
\begin{pgfscope}%
\pgfpathrectangle{\pgfqpoint{2.918777in}{0.408431in}}{\pgfqpoint{2.478207in}{0.450152in}}%
\pgfusepath{clip}%
\pgfsetbuttcap%
\pgfsetroundjoin%
\pgfsetlinewidth{0.752812pt}%
\definecolor{currentstroke}{rgb}{0.000000,0.000000,0.000000}%
\pgfsetstrokecolor{currentstroke}%
\pgfsetdash{{2.775000pt}{1.200000pt}}{0.000000pt}%
\pgfpathmoveto{\pgfqpoint{4.983949in}{0.408431in}}%
\pgfpathlineto{\pgfqpoint{4.983949in}{0.858583in}}%
\pgfusepath{stroke}%
\end{pgfscope}%
\begin{pgfscope}%
\pgfpathrectangle{\pgfqpoint{2.918777in}{0.408431in}}{\pgfqpoint{2.478207in}{0.450152in}}%
\pgfusepath{clip}%
\pgfsetbuttcap%
\pgfsetroundjoin%
\pgfsetlinewidth{1.505625pt}%
\definecolor{currentstroke}{rgb}{0.631373,0.062745,0.207843}%
\pgfsetstrokecolor{currentstroke}%
\pgfsetdash{{1.500000pt}{2.475000pt}}{0.000000pt}%
\pgfpathmoveto{\pgfqpoint{3.652297in}{0.408431in}}%
\pgfpathlineto{\pgfqpoint{3.652297in}{0.858583in}}%
\pgfusepath{stroke}%
\end{pgfscope}%
\begin{pgfscope}%
\pgfpathrectangle{\pgfqpoint{2.918777in}{0.408431in}}{\pgfqpoint{2.478207in}{0.450152in}}%
\pgfusepath{clip}%
\pgfsetrectcap%
\pgfsetroundjoin%
\pgfsetlinewidth{0.752812pt}%
\definecolor{currentstroke}{rgb}{0.964706,0.658824,0.000000}%
\pgfsetstrokecolor{currentstroke}%
\pgfsetdash{}{0pt}%
\pgfpathmoveto{\pgfqpoint{2.918777in}{0.694212in}}%
\pgfpathlineto{\pgfqpoint{2.943809in}{0.694213in}}%
\pgfpathlineto{\pgfqpoint{2.968842in}{0.694216in}}%
\pgfpathlineto{\pgfqpoint{2.993874in}{0.694224in}}%
\pgfpathlineto{\pgfqpoint{3.018906in}{0.694241in}}%
\pgfpathlineto{\pgfqpoint{3.043939in}{0.694278in}}%
\pgfpathlineto{\pgfqpoint{3.068971in}{0.694351in}}%
\pgfpathlineto{\pgfqpoint{3.094004in}{0.694494in}}%
\pgfpathlineto{\pgfqpoint{3.119036in}{0.694760in}}%
\pgfpathlineto{\pgfqpoint{3.144068in}{0.695231in}}%
\pgfpathlineto{\pgfqpoint{3.169101in}{0.696032in}}%
\pgfpathlineto{\pgfqpoint{3.194133in}{0.697336in}}%
\pgfpathlineto{\pgfqpoint{3.219166in}{0.699376in}}%
\pgfpathlineto{\pgfqpoint{3.244198in}{0.702441in}}%
\pgfpathlineto{\pgfqpoint{3.269230in}{0.706866in}}%
\pgfpathlineto{\pgfqpoint{3.294263in}{0.712977in}}%
\pgfpathlineto{\pgfqpoint{3.319295in}{0.720984in}}%
\pgfpathlineto{\pgfqpoint{3.344327in}{0.730797in}}%
\pgfpathlineto{\pgfqpoint{3.369360in}{0.741772in}}%
\pgfpathlineto{\pgfqpoint{3.394392in}{0.752481in}}%
\pgfpathlineto{\pgfqpoint{3.419425in}{0.760586in}}%
\pgfpathlineto{\pgfqpoint{3.444457in}{0.762973in}}%
\pgfpathlineto{\pgfqpoint{3.469489in}{0.756169in}}%
\pgfpathlineto{\pgfqpoint{3.494522in}{0.736749in}}%
\pgfpathlineto{\pgfqpoint{3.519554in}{0.695266in}}%
\pgfpathlineto{\pgfqpoint{3.544587in}{0.622593in}}%
\pgfpathlineto{\pgfqpoint{3.569619in}{0.552716in}}%
\pgfpathlineto{\pgfqpoint{3.594651in}{0.496591in}}%
\pgfpathlineto{\pgfqpoint{3.619684in}{0.459610in}}%
\pgfpathlineto{\pgfqpoint{3.644716in}{0.443772in}}%
\pgfpathlineto{\pgfqpoint{3.669749in}{0.446798in}}%
\pgfpathlineto{\pgfqpoint{3.694781in}{0.463190in}}%
\pgfpathlineto{\pgfqpoint{3.719813in}{0.492630in}}%
\pgfpathlineto{\pgfqpoint{3.744846in}{0.532522in}}%
\pgfpathlineto{\pgfqpoint{3.769878in}{0.573734in}}%
\pgfpathlineto{\pgfqpoint{3.794911in}{0.608073in}}%
\pgfpathlineto{\pgfqpoint{3.819943in}{0.633593in}}%
\pgfpathlineto{\pgfqpoint{3.844975in}{0.650202in}}%
\pgfpathlineto{\pgfqpoint{3.870008in}{0.655657in}}%
\pgfpathlineto{\pgfqpoint{3.895040in}{0.644798in}}%
\pgfpathlineto{\pgfqpoint{3.920072in}{0.622878in}}%
\pgfpathlineto{\pgfqpoint{3.945105in}{0.598294in}}%
\pgfpathlineto{\pgfqpoint{3.970137in}{0.576363in}}%
\pgfpathlineto{\pgfqpoint{3.995170in}{0.559828in}}%
\pgfpathlineto{\pgfqpoint{4.020202in}{0.548906in}}%
\pgfpathlineto{\pgfqpoint{4.045234in}{0.540847in}}%
\pgfpathlineto{\pgfqpoint{4.070267in}{0.531304in}}%
\pgfpathlineto{\pgfqpoint{4.095299in}{0.527427in}}%
\pgfpathlineto{\pgfqpoint{4.120332in}{0.535340in}}%
\pgfpathlineto{\pgfqpoint{4.145364in}{0.545700in}}%
\pgfpathlineto{\pgfqpoint{4.170396in}{0.551335in}}%
\pgfpathlineto{\pgfqpoint{4.195429in}{0.557876in}}%
\pgfpathlineto{\pgfqpoint{4.220461in}{0.567218in}}%
\pgfpathlineto{\pgfqpoint{4.245494in}{0.579735in}}%
\pgfpathlineto{\pgfqpoint{4.270526in}{0.594934in}}%
\pgfpathlineto{\pgfqpoint{4.295558in}{0.611697in}}%
\pgfpathlineto{\pgfqpoint{4.320591in}{0.628595in}}%
\pgfpathlineto{\pgfqpoint{4.345623in}{0.644241in}}%
\pgfpathlineto{\pgfqpoint{4.370655in}{0.657583in}}%
\pgfpathlineto{\pgfqpoint{4.395688in}{0.668079in}}%
\pgfpathlineto{\pgfqpoint{4.420720in}{0.675741in}}%
\pgfpathlineto{\pgfqpoint{4.445753in}{0.681094in}}%
\pgfpathlineto{\pgfqpoint{4.470785in}{0.685079in}}%
\pgfpathlineto{\pgfqpoint{4.495817in}{0.688949in}}%
\pgfpathlineto{\pgfqpoint{4.520850in}{0.694153in}}%
\pgfpathlineto{\pgfqpoint{4.545882in}{0.702168in}}%
\pgfpathlineto{\pgfqpoint{4.570915in}{0.714299in}}%
\pgfpathlineto{\pgfqpoint{4.595947in}{0.731438in}}%
\pgfpathlineto{\pgfqpoint{4.620979in}{0.753841in}}%
\pgfpathlineto{\pgfqpoint{4.646012in}{0.780904in}}%
\pgfpathlineto{\pgfqpoint{4.671044in}{0.809732in}}%
\pgfpathlineto{\pgfqpoint{4.696077in}{0.801229in}}%
\pgfpathlineto{\pgfqpoint{4.721109in}{0.772190in}}%
\pgfpathlineto{\pgfqpoint{4.746141in}{0.746506in}}%
\pgfpathlineto{\pgfqpoint{4.771174in}{0.725857in}}%
\pgfpathlineto{\pgfqpoint{4.796206in}{0.710619in}}%
\pgfpathlineto{\pgfqpoint{4.821238in}{0.700436in}}%
\pgfpathlineto{\pgfqpoint{4.846271in}{0.694456in}}%
\pgfpathlineto{\pgfqpoint{4.871303in}{0.691583in}}%
\pgfpathlineto{\pgfqpoint{4.896336in}{0.690723in}}%
\pgfpathlineto{\pgfqpoint{4.921368in}{0.690962in}}%
\pgfpathlineto{\pgfqpoint{4.946400in}{0.691650in}}%
\pgfpathlineto{\pgfqpoint{4.971433in}{0.692401in}}%
\pgfpathlineto{\pgfqpoint{4.996465in}{0.693032in}}%
\pgfpathlineto{\pgfqpoint{5.021498in}{0.693493in}}%
\pgfpathlineto{\pgfqpoint{5.046530in}{0.693798in}}%
\pgfpathlineto{\pgfqpoint{5.071562in}{0.693985in}}%
\pgfpathlineto{\pgfqpoint{5.096595in}{0.694093in}}%
\pgfpathlineto{\pgfqpoint{5.121627in}{0.694152in}}%
\pgfpathlineto{\pgfqpoint{5.146660in}{0.694183in}}%
\pgfpathlineto{\pgfqpoint{5.171692in}{0.694198in}}%
\pgfpathlineto{\pgfqpoint{5.196724in}{0.694205in}}%
\pgfpathlineto{\pgfqpoint{5.221757in}{0.694209in}}%
\pgfpathlineto{\pgfqpoint{5.246789in}{0.694210in}}%
\pgfpathlineto{\pgfqpoint{5.271822in}{0.694211in}}%
\pgfpathlineto{\pgfqpoint{5.296854in}{0.694211in}}%
\pgfpathlineto{\pgfqpoint{5.321886in}{0.694211in}}%
\pgfpathlineto{\pgfqpoint{5.346919in}{0.694211in}}%
\pgfpathlineto{\pgfqpoint{5.371951in}{0.694211in}}%
\pgfpathlineto{\pgfqpoint{5.396984in}{0.694211in}}%
\pgfusepath{stroke}%
\end{pgfscope}%
\begin{pgfscope}%
\pgfsetrectcap%
\pgfsetmiterjoin%
\pgfsetlinewidth{0.752812pt}%
\definecolor{currentstroke}{rgb}{0.000000,0.000000,0.000000}%
\pgfsetstrokecolor{currentstroke}%
\pgfsetdash{}{0pt}%
\pgfpathmoveto{\pgfqpoint{2.918777in}{0.408431in}}%
\pgfpathlineto{\pgfqpoint{2.918777in}{0.858583in}}%
\pgfusepath{stroke}%
\end{pgfscope}%
\begin{pgfscope}%
\pgfsetrectcap%
\pgfsetmiterjoin%
\pgfsetlinewidth{0.752812pt}%
\definecolor{currentstroke}{rgb}{0.000000,0.000000,0.000000}%
\pgfsetstrokecolor{currentstroke}%
\pgfsetdash{}{0pt}%
\pgfpathmoveto{\pgfqpoint{5.396984in}{0.408431in}}%
\pgfpathlineto{\pgfqpoint{5.396984in}{0.858583in}}%
\pgfusepath{stroke}%
\end{pgfscope}%
\begin{pgfscope}%
\pgfsetrectcap%
\pgfsetmiterjoin%
\pgfsetlinewidth{0.752812pt}%
\definecolor{currentstroke}{rgb}{0.000000,0.000000,0.000000}%
\pgfsetstrokecolor{currentstroke}%
\pgfsetdash{}{0pt}%
\pgfpathmoveto{\pgfqpoint{2.918777in}{0.408431in}}%
\pgfpathlineto{\pgfqpoint{5.396984in}{0.408431in}}%
\pgfusepath{stroke}%
\end{pgfscope}%
\begin{pgfscope}%
\pgfsetrectcap%
\pgfsetmiterjoin%
\pgfsetlinewidth{0.752812pt}%
\definecolor{currentstroke}{rgb}{0.000000,0.000000,0.000000}%
\pgfsetstrokecolor{currentstroke}%
\pgfsetdash{}{0pt}%
\pgfpathmoveto{\pgfqpoint{2.918777in}{0.858583in}}%
\pgfpathlineto{\pgfqpoint{5.396984in}{0.858583in}}%
\pgfusepath{stroke}%
\end{pgfscope}%
\end{pgfpicture}%
\makeatother%
\endgroup%

    \caption[Visualization of \gls{bo} iterations.]{Visualization of \gls{bo} iterations with the objective function (black line), previous queries (black points) and the chosen query (red points). The query is chosen by minimizing the acquisition function $\alpha(\mathbf{x}|\mathcal{D})$ (orange) based on the current model (blue) within the feasible set $\mathbf{x} \in \mathcal{X} = [-2, 2]$. This is indicated by the red dashed line.}
    \label{fig:bo_example}
\end{figure}

As the objective function is unknown, \gls{bo} requires a surrogate model which captures the prior believe of the objective function and can be updated from the observations. This model can be a parametric model such as a linear model, however the most common choice in current \gls{bo} algorithms is the use of a nonparametric model in form of a \gls{gp} as $f(\mathbf{x}) \sim \mathcal{GP}(m, k)$ as discussed in the previous \Cref{sec:gaussian_process}. 

\begin{algorithm}[b]
\centering
\caption{\gls{bo} \cite{Shahriari_2016}}
\begin{algorithmic}[1]
\Require prior $f \sim \mathcal{GP}(m(\mathbf{x}), k(\mathbf{x},\mathbf{x}'))$; feasible set $\mathcal{X}\in \R^D$; data set ${\mathcal{D}_{N} = \{y_j, \mathbf{x}_j\}_{j=0}^{N}}$ 
\For{$k = N, 2, \dots, K$}
    \State Train \gls{gp} model with $\mathcal{D}_k$
    \State choose next query $\mathbf{x}_{k+1} = \underset{\mathbf{x}\in \mathcal{X}}{\argmin} \,\alpha(\mathbf{x}|\mathcal{D})$
    \State query objective function $y_{k+1} = f(\mathbf{x}_{k+1}) + w$
    \State update data set $\mathcal{D}_{k+1} = \mathcal{D}_{k} \cup \{y_{k+1}, \mathbf{x}_{k+1}\}$
\EndFor
\end{algorithmic}
\label{algo:bo}
\end{algorithm}

Besides the model, \gls{bo} requires the definition of an acquisition function in the form of $\alpha(\mathbf{x}|\mathcal{D}) \colon \mathcal{X} \mapsto \R$ which maps a query $\mathbf{x}$ to a corresponding value defining the utility of the query. Optimizing the acquisition function and updating the model is performed sequentially as shown in Algorithm \ref{algo:bo} up to a terminal condition or after exhausting a predefined observation budget. A few \gls{bo} steps are also visualized on an example in Figure \ref{fig:bo_example}.


The sampling efficiency of \gls{bo} highly depends on choosing a suitable acquisition function for the problem at hand. Different acquisition functions have been proposed such as \gls{pi}\cite{Kushner_1964}, \gls{ei}\cite{} and \gls{ucb}\cite{Auer_2002} (or in the case of minimizing an objective function \gls{lcb}), each with different characteristics regarding exploration and exploitation. Every acquisition function has to balance the exploration-exploitation trade-off in exploring the objective function by querying at locations with high variance or exploiting the model's mean. The mentioned acquisition functions are myopic, as they try to optimize in a one-step-look-ahead fashion considering only the model's current state as opposed to planing a sequence of queries.


\subsubsection{Regret}

To evaluate the performance of different \gls{bo} algorithms a metric called regret is used. It defines the cost of choosing a query at iteration $k$ which deviates from the optimum. The objective of a \gls{bo} algorithm is then to minimize the cumulative regret.
\begin{definition}[Cumulative regret]
Let $\mathbf{x}^*$ be the optimizer to the function $f(\mathbf{x})$ and let $\mathbf{x}_k$ be the queried point by the algorithm at iteration $k$. The cumulative regret after $K$ iterations is then given by
\begin{equation}
    R_K \coloneqq \sum_{k=1}^K (f(\mathbf{x}_k) - f(\mathbf{x}^*)).
\end{equation}
\end{definition}
A desirable characteristic of a \gls{bo} algorithm is to achieve sub-linear regret (also called \emph{no-regret}) as
\begin{equation}
    \lim_{T \to \infty} \frac{R_K}{T} = 0.
\end{equation}
It defines that for large $T$ the cumulative regret converges to a constant implying the convergence of the algorithm to the true optimum $\mathbf{x}^*$.
Such a no-regret algorithm is \gls{gp}-\gls{lcb} as defined in \cite{Srinivas_2010} with the acquisition function
\begin{equation}
    \mathbf{x}_{k+1} = \argmin_{\mathbf{x} \in \mathcal{X}} \alpha_{\gls{gp}-\gls{lcb}}(\mathbf{x}|\mathcal{D}) = \argmin_{\mathbf{x} \in \mathcal{X}} \mu_k(\mathbf{x}) - \sqrt{\beta_{k+1}}\, \sigma_k(\mathbf{x})
    \label{eq:lcb}
\end{equation}
with $\mu_k$ and $\sigma_k$ describing the posterior mean and standard deviation of the \gls{gp} model from the previous iteration, respectively. Hence, $\beta_{k+1}$ defines the mentioned exploration-exploitation trade-off at the current iteration for \gls{gp}-\gls{lcb}. Setting $\beta_{k+1}$ according to \textcite[Theorem 1]{Srinivas_2010} results in proven sub-linear regret. For a deeper introduction to \gls{bo} it is referred to \textcite{Shahriari_2016}.

\subsubsection{Time-Varying Bayesian Optimization}
\label{sec:tvbo}

The following notation of \gls{tvbo} is based on the notation in \cite{Wang_2021}. In \gls{tvbo} the unknown objective function is time-varying as $f\colon \mathcal{X} \times \mathcal{T} \mapsto \R$ where $\mathcal{T}$ represents the time domain as an increasing sequence $\mathcal{T} = \{1,2, \dots, T\}$ with $T$ as the time horizon. To include the time dependency into the \gls{gp} model, the current state-of-the-art is to use a product composite kernel of $k_{S}\colon \mathcal{X} \times \mathcal{X} \mapsto \R$ and $k_{T}\colon \mathcal{T} \times \mathcal{T} \mapsto \R$ resulting in the \emph{spatio-temporal kernel}
\begin{equation}
    k\colon \mathcal{X} \times \mathcal{T} \mapsto \R, \quad k(\{\mathbf{x},t\},\{\mathbf{x}',t'\}) = k_{S}(\mathbf{x}, \mathbf{x}') \otimes k_{T}(t, t')
    \label{eq:spatio_temporal_kernel}
\end{equation}
with $\otimes$ as the Hadamard product.
The kernel $k_S$ embeds the spatial correlations within $\mathcal{X}$ and implies the Bayesian regularity assumptions on $f_t(\mathbf{x})$ as being a sample from a \gls{gp} prior with kernel $k_S$ at each time step. The spatial kernel $k_S$ is often chosen to be a kernel from the Matérn class such as the \gls{se} kernel. The kernel $k_T$ characterizes the temporal correlations and defines how to treat data from the past. As defined in \eqref{eq:compsite_kernel_diff} the resulting kernel $k(\{\mathbf{x},t\},\{\mathbf{x}',t'\})$ is a valid kernel, as long as $k_S$ and $k_T$ are valid kernels in $\mathcal{X}$ and $\mathcal{T}$, respectively. 
The base algorithm for \gls{tvbo} is displayed in Algorithm \ref{algo:tvbo}.

\begin{algorithm}[h]
\centering
\caption{Base \gls{tvbo}}
\begin{algorithmic}[1]
\Require prior $\mathcal{GP}(m(\mathbf{x}), k_S(\mathbf{x},\mathbf{x}') \otimes k_T(t, t'))$ and hyperparameter; feasible set ${\mathcal{X}\in \R^D}$; data set $\mathcal{D}_{N} = \{y_j, \mathbf{x}_j, t_j\}_{j=0}^{N}$ 
\State $t_0=N$
\For{$t = t_0, t_0+1, t_0+2, \dots, T$}
    \State Train \gls{gp} model with $\mathcal{D}_t$
    \State choose next query $\mathbf{x}_{t+1} = \underset{\mathbf{x}\in \mathcal{X}}{\argmin} \,\alpha(\mathbf{x}, t+1|\mathcal{D})$
    \State query objective function $y_{t+1} = f_{t+1}(\mathbf{x}_{t+1}) + w$
    \State update data set $\mathcal{D}_{t+1} = \mathcal{D}_{t} \cup \{y_{t+1}, \mathbf{x}_{t+1}, t+1\}$
\EndFor
\end{algorithmic}
\label{algo:tvbo}
\end{algorithm}

In contrast to standard \gls{bo}, in \gls{tvbo} the acquisition function is constrained to only choose a query for the next time step $t+1$, given the \gls{gp} model of the current time step, even though the \gls{gp} model is defined over the whole domain $\mathcal{X}\times\mathcal{T}$ through the kernel $k$. Moreover, in this time-varying environment, there is no longer a single optimizer for the objective function, but an optimizer for each time step, which may vary over time. Therefore, a different notion of regret has to be defined, to capture the performance of a \gls{tvbo} algorithm. For this purpose, the dynamic cumulative regret metric is introduced (Definition \ref{def:dynamic_regret}).
\begin{definition}[Dynamic cumulative regret]
Let $\mathbf{x}_t^*$ be the optimizer to the time-varying function $f_t(\mathbf{x})$ as $\mathbf{x}_t^* = \argmin_{\mathbf{x} \in \mathcal{X}} f_t(\mathbf{x})$ at time step $t$ and let $\mathbf{x}_t$ be the queried point by the algorithm at time step $t$. Than the dynamic cumulative regret after $T$ time steps is
\begin{equation}
    R_T \coloneqq \sum_{t=1}^T (f_t(\mathbf{x}_t) - f_t(\mathbf{x}_t^*)).
\end{equation}
\label{def:dynamic_regret}
\end{definition}
\vspace{-0.5cm}
In the following, the dynamic cumulative regret will only be referred to as regret for convenience. Achieving sub-linear regret in a general time-varying setting is not possible without stating assumptions on the amount of change over the time horizon $T$ \cite{Besbes_2015}. The intuition behind is that it is not possible to track the optimum with arbitrary precision if the objective function changes significantly at each time step \cite{Bogunovic_2016}. However, when $f_t(\mathbf{x})$ is a function in an \gls{rkhs} $\mathcal{H}_K$ with a bounded norm and the amount of change is limited and known a-priori within a variation budget $P_T$ \cite{Besbes_2014} as
\begin{equation}
    \sum_{t=1}^{T-1} ||f_t(\mathbf{x}) - f_{t+1}(\mathbf{x})||_{\mathcal{H}_K} \leq P_T,
\end{equation}
\gls{tvbo} algorithms have been developed that have been proven to achieve sub-linear regret \cite{Zhou_2021}.