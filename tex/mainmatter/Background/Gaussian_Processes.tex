\section{Gaussian Process Regression}
\label{sec:gaussian_process}

\glspl{gp} are widely used for regression and build the foundation of many \gls{bo} algorithms by modeling its objective function. To introduce \gls{gp} regression, this section will follow \textcite{Rasmussen_2006} to which is also referred to for further details.

\glspl{gp} are a nonparametric Bayesian approach to regression which explicitly incorporate uncertainty. The goal of \gls{gp} regression is to model the function $f \colon \mathcal{X} \mapsto \R$ with $\mathcal{X} \subset \R^D$ which is corrupted by zero mean Gaussian noise $w \sim \mathcal{N}\left( 0, \sigma_n^2 \right)$. An observation $y$ from $f(\mathbf{x})$ can therefore be expressed as
\begin{equation}
    y = f(\mathbf{x}) + w.
    \label{eq:gp_objective_function}
\end{equation}
Taking $N$ observation of the objective function \eqref{eq:gp_objective_function} at the training points $\mathbf{X} = [\mathbf{x}_1, \dots, \mathbf{x}_N] \in R^{D \times N}$ results in a data set $\mathcal{D} \coloneqq \left\{ (\mathbf{x}_i, y_i)|i = 1,\dots, N \right\} = (\mathbf{X}, \mathbf{y})$ with $\mathbf{y} = [y_1,\dots,y_N] \in \R^N$ as the training targets in vector notation. 

A \gls{gp} can be interpreted as modeling $f(\mathbf{x})$ as a distribution over functions and is fully defined as $f(\mathbf{x}) \sim \mathcal{GP}\left(m(\mathbf{x}), k(\mathbf{x}, \mathbf{x}') \right)$
with the mean function $m$ and kernel $k$ as
\begin{alignat}{2}
    m \colon \mathcal{X} &\mapsto \R, \quad &&m(\mathbf{x}) = \EX\left[ f(\mathbf{x}) \right] \\
    k \colon \mathcal{X} \times \mathcal{X} &\mapsto \R, \quad &&k(\mathbf{x}, \mathbf{x}') = \EX\left[ \left( f(\mathbf{x})- m(\mathbf{x})\right) \left(f(\mathbf{x}')- m(\mathbf{x}')\right) \right].
\end{alignat}
The mean function $m$ is often set to be constant as $m(\mathbf{x}) = \mu_0$ with $\mu_0$ either as zero or as the mean of the training targets $\mathbf{y}$. However, more complex mean functions are also possible.

With the information contained in the data set $\mathcal{D}$, predictions $\MatBold{f}_*$ at $N_*$ test locations ${\mathbf{X}_* = [\MatBold{x}^*_{1}, \dots ,\MatBold{x}^*_{N_*}] \in \R^{D \times N_*}}$ can be made by setting up the multivariate Gaussian joint distribution over training targets and predictions as
\begin{equation}
    \left[ \begin{array}{c}
    \mathbf{y}\\ 
    \mathbf{f_*} 
    \end{array}\right] \sim \mathcal{N} \left( \left[ \begin{array}{c}
    m(\mathbf{X})\\ 
    m(\mathbf{X}_*)
    \end{array}\right], \left[\begin{array}{cc}
    K(\mathbf{X},\mathbf{X}) + \sigma_n^2 \mathbf{I} & K(\mathbf{X},\mathbf{X}_*) \\ 
    K(\mathbf{X}_*,\mathbf{X}) & K(\mathbf{X}_*,\mathbf{X}_*)
    \end{array}\right] \right).
    \label{eq:gp_joint_distribution}
\end{equation}
Here, the notation is used that $K(\mathbf{X}, \mathbf{X}')$ denotes a matrix with entries $K(\mathbf{X}, \mathbf{X}')_{i,j} = k(\mathbf{X}_i, \mathbf{X}'_j)$. The marginal distribution $\MatBold{f}_* \sim \mathcal{N}\left(m(\mathbf{X}_*), K(\mathbf{X}_*,\mathbf{X}_*) \right)$ is the prior prediction over the test locations. Conditioning the joint distribution \eqref{eq:gp_joint_distribution} on the data set $\mathcal{D}$ yields the conditioned posterior distribution
\begin{equation}
    \mathbf{f_*} | \mathcal{D} \sim \mathcal{N}(\boldsymbol\mu_*, \boldsymbol\Sigma_*) \label{eq:posterior_distribution}
\end{equation}
which again is a multivariate Gaussian distribution with mean $\boldsymbol\mu_*$ and covariance matrix $\boldsymbol\Sigma_*$ as
\begin{alignat}{3}
    &\boldsymbol\mu_* &&= \quad m(\mathbf{X}_*) &&+ K(\mathbf{X}_*,\mathbf{X}) \left(K(\mathbf{X},\mathbf{X})+ \sigma_n^2 \mathbf{I} \right)^{-1} \left(\mathbf{y} - m(\mathbf{X}) \right), \label{eq:gp_mean}\\
    &\boldsymbol\Sigma_* &&= \underbrace{\vphantom{\left(K(\mathbf{X},\mathbf{X})+ \sigma_n^2 \mathbf{I} \right)^{-1}}K(\mathbf{X}_*,\mathbf{X}_*)}_\text{prior knowledge} &&- \underbrace{K(\mathbf{X}_*,\mathbf{X}) \left(K(\mathbf{X},\mathbf{X})+ \sigma_n^2 \mathbf{I} \right)^{-1} K(\mathbf{X},\mathbf{X}_*)\quad}_\text{obtained knowledge}. \label{eq:gp_cov}
\end{alignat}
The calculation of the mean and covariance of the posterior distribution can each be divided into a prior knowledge part defined though the \gls{gp} prior and an update part to this prior obtained through the information in data set $\mathcal{D}$.

\subsubsection{Kernels}

Selecting a suitable kernel for the regression task is crucial. A commonly chosen kernel is the \gls{se} kernel as 
\begin{equation}
    k(\mathbf{x}, \mathbf{x}') = \sigma_k^2 \exp\left(-\frac{1}{2} (\mathbf{x} - \mathbf{x}')^T \MatBold{\Lambda}^{-1} (\mathbf{x} - \mathbf{x}')\right)
\end{equation}
with the length scales $\MatBold{\Lambda} = \mathrm{diag}(\sigma_{l,1}^2, \dots, \sigma_{l,D}^2)$ and the output scale $\sigma_k^2$ as hyperparameters. It defines a \gls{rkhs} $\mathcal{H}_k(\mathcal{X})$ with the property that any function $f \in \mathcal{H}_k(\mathcal{X})$ is within the set of infinitely differentiable functions $C^\infty(\mathcal{X})$. Therefore, the mean estimate \eqref{eq:gp_mean} of the \gls{gp} posterior will also be in $C^\infty(\mathcal{X})$. Furthermore, the \gls{se} kernel is a \emph{universal} kernel meaning it can approximate any continuous function. However, since the hypothesis space of the \gls{gp} model with a \gls{se} kernel is restricted to functions in $C^\infty(\mathcal{X})$, choosing the \gls{se} kernel can be problematic in some practical applications as this smoothness assumption induces a strong bias and the universal property is only given in the presence of infinite data.

Kernels can be tailed to specific problems as long as they satisfy the conditions of being symmetric and resulting in a positive definite Gram matrix $K$ (for more information see \textcite[Chap. 4.1]{Rasmussen_2006}). For example, in \textcite{Marco_2017} a kernel was specifically designed to imply a distribution over \gls{lqr} cost functions. Valid kernels can also be recombined resulting, e.g., in a product composite kernel as
\begin{equation}
    k_{comp}(\mathbf{x}, \mathbf{x}') = k_1(\mathbf{x}, \mathbf{x}') \otimes k_2(\mathbf{x}, \mathbf{x}').
\end{equation} 
This can be beneficial for capturing different characteristic in the data set $\mathcal{D}$ \cite{Duvenaud_2014}. Furthermore, if valid kernels $k_1(\mathbf{x}_1, \mathbf{x}_1')$ and $k_2(\mathbf{x}_2, \mathbf{x}_2')$ mapping to $\R$ are defined over different input spaces $\mathcal{X}_1$ and $\mathcal{X}_2$, multiplying $k_1$ and $k_2$ also results in a valid product composite kernel as
\begin{equation}
    k_{comp}\colon \mathcal{X}_1 \times \mathcal{X}_2 \mapsto \R, \quad k_{comp}(\{\mathbf{x}_1,\mathbf{x}_2\}, \{\mathbf{x}_1',\mathbf{x}_2'\}) = k_1(\mathbf{x}_1, \mathbf{x}_1') \otimes k_2(\mathbf{x}_2, \mathbf{x}_2').
    \label{eq:compsite_kernel_diff}
\end{equation}
This property enables the use of different kernels for individual dimensions of the input $\mathbf{x}$ which might have different context \cite{Krause_2011}.

\subsubsection{Linear Operators on Gaussian Processes}
\label{sec:linear_operator}

\glspl{gp} are closed under linear operators $\mathcal{L}$ such as differentiation \cite{Rasmussen_2006}. Considering a \gls{gp} as $f(\mathbf{x}) \sim \mathcal{GP}\left(m(\mathbf{x}),k(\mathbf{x}, \mathbf{x}')\right)$, this means applying the linear operator to $f(\mathbf{x})$ results again in a \gls{gp} as
\begin{equation}
    \mathcal{L}f(\mathbf{x}) \sim \mathcal{GP}\left(\mathcal{L}m(\mathbf{x}),\mathcal{L}k(\mathbf{x}, \mathbf{x}')\mathcal{L}^T\right)
\end{equation}
using the notation by \textcite{Agrell_2019}. Here, $\mathcal{L}k(\mathbf{x}, \mathbf{x}')$ and $k(\mathbf{x}, \mathbf{x}')\mathcal{L}^T$ indicate the operator $\mathcal{L}$ acting on $\mathbf{x}$ and $\mathbf{x}'$, respectively. The property of staying closed under linear operators is used by \textcite{Geist_2020} and \textcite{Jidling_2017} to embed prior knowledge in the form of physical insights into the \gls{gp} regression task as equality constaints on the \gls{gp}.


\subsection{Linear Inequality Constraints}
\label{sec:linear_constraints}

Assumption \ref{ass:prior_knowledge_convex} in the problem formulation states that the objective function remains convex thought time. This means that the Hessian $\nabla_{\mathbf{x}} f_t$ remains positive definite throughout the feasible set at each time step. Therefore, this thesis uses a linear operator $\mathcal{L} = \frac{\pdiff^2}{\pdiff x_i^2}$ for every spatial dimension $i\in [1,\dots,D]$ as introduced above and apply inequality constraints on the posterior distribution $\mathbf{f}_*$ in \eqref{eq:posterior_distribution}. In combination with a smooth kernel such as the \gls{se} kernel, the positive definiteness of the Hessian of $f_t$ can be approximated. Applying such linear inequality constraints in \gls{gp} regression has been discussed among others by \textcite{Agrell_2019} and \textcite{Wang_2016} which build the theoretical background for this section.

A \gls{gp} under linear inequality constraints requires the \gls{gp} posterior conditioned on the data set $\mathcal{D}$ to satisfy
\begin{equation}
    a(\mathbf{x}) \leq \mathcal{L} f(\mathbf{x}) \leq b(\mathbf{x})
    \label{eq:linear_inequality_constraints_complete}
\end{equation}
for two bounding functions $a(\mathbf{x}), b(\mathbf{x})\colon \R^D \mapsto (\R \cup \{-\infty, \infty\})$ with $a(\mathbf{x}) < b(\mathbf{x}),\, \forall \mathbf{x} \in \mathcal{X}$. \textcite{Agrell_2019} and \textcite{Wang_2016} introduce a method to achieve this approximately by considering only a finite set of $N_v$ inputs ${\mathbf{X}_v = [\MatBold{x}^v_{1}, \dots ,\MatBold{x}^v_{N_v}] \in \R^{D \times N_v}}$, called the \glspl{vop}, at which \eqref{eq:linear_inequality_constraints_complete} has to hold. Furthermore, in \textcite{Agrell_2019} the assumption is made that the virtual observations $\mathcal{L} f(\MatBold{x}^v_{i})$ are corrupted by Gaussian noise $w_{v,i} \sim \mathcal{N}(0, \sigma_v^2)$. The reason for this is numerical stability in calculating the constrained posterior distribution. This yields in a relaxed formulation of \eqref{eq:linear_inequality_constraints_complete} as
\begin{equation}
    a(\mathbf{X}_v) \leq \mathcal{L} f(\mathbf{X}_v) + w_v \leq b(\mathbf{X}_v), \quad w_{v} \sim \mathcal{N}(\mathbf{0}, \sigma_v^2\MatBold{I}),
    \label{eq:linear_inequality_constraints}
\end{equation}
where the constraints $a(\mathbf{x}), b(\mathbf{x})$ no longer have to hold for all $\mathbf{x} \in \mathcal{X}$. The \glspl{vop} do not have to be within the feasible set $\mathcal{X}$ but $\R^D$.

To simplify notation, the corrupted virtual observations at the \glspl{vop} will be denoted as ${\tilde{C}(\mathbf{X}_v) \coloneqq \mathcal{L} f(\mathbf{X}_v) + w_v}$, the Gram matrix $K(\MatBold{X},\MatBold{X}')$ as $K_{\MatBold{X},\MatBold{X}'}$, and the mean function $m(\mathbf{x})$ as $\mu_\mathbf{x}$. Furthermore, $C(\mathbf{X}_v)$ will denote the event of $\tilde{C}$ satisfying \eqref{eq:linear_inequality_constraints} for all $\MatBold{x}^v_{i} \in \mathbf{X}_v$.

Since applying a linear operator $\mathcal{L}$ to $f(\mathbf{x})$ results again in a \gls{gp} as described in \Cref{sec:linear_operator}, the joint distribution of the predictions $\mathbf{f}_*$, observations $\mathbf{y}$ and virtual observations $\tilde{C}$ can be set up as
\begin{equation}
    \left[ \begin{array}{c}
    \mathbf{f_*} \\
    \mathbf{y}\\ 
    \tilde{C}
    \end{array}\right] \sim \mathcal{N} \left( \left[ \begin{array}{c}
    \mu_{\mathbf{X}_*}\\ 
    \mu_{\mathbf{X}} \\
    \mathcal{L}\mu_{\mathbf{X}_v}
    \end{array}\right],
    \left[\begin{array}{ccc}
    K_{\mathbf{X}_*,\mathbf{X}_*}& K_{\mathbf{X}_*,\mathbf{X}} & K_{\mathbf{X}_*,\mathbf{X}_v}\mathcal{L}^T \\ 
    K_{\mathbf{X},\mathbf{X}_*} & K_{\mathbf{X},\mathbf{X}} + \sigma_n^2\mathbf{I} & K_{\mathbf{X},\mathbf{X}_v}\mathcal{L}^T  \\
    \mathcal{L}K_{\mathbf{X}_v,\mathbf{X}_*} & \mathcal{L}K_{\mathbf{X}_v,\mathbf{X}} & \mathcal{L}K_{\mathbf{X}_v,\mathbf{X}_v}\mathcal{L}^T + \sigma_v^2\mathbf{I}
    \end{array}\right] \right).
    \label{eq:unconstrained_joint_distribution}
\end{equation}
Conditioning the joint distribution on the data set $\mathcal{D}=(\mathbf{X}, \mathbf{y})$ results in
\begin{equation}
    \left.\left[ \begin{array}{c}
    \mathbf{f_*} \\
    \tilde{C}
    \end{array}\right] \right\vert \mathcal{D} \sim \mathcal{N} \left( \left[ \begin{array}{c}
    \mu_{\mathbf{X}_*} + A_2 \left( \mathbf{y} - \mu_{\mathbf{X}}\right) \\ 
    \mathcal{L}\mu_{\mathbf{X}_v} + A_1 \left( \mathbf{y} - \mu_{\mathbf{X}}\right)
    \end{array}\right],
    \left[\begin{array}{cc}
    B_2 & B_3 \\
    B_3^T & B_1
    \end{array}\right] \right)
    \label{eq:unconstrained_joint_distribution2}
\end{equation}
with
\begin{align}
    A_1 &= (\mathcal{L}K_{\mathbf{X}_v,\mathbf{X}}) \left(K_{\mathbf{X},\mathbf{X}} + \sigma_n^2\mathbf{I} \right)^{-1} \label{eq:A1}\\
    A_2 &= K_{\mathbf{X}_*,\mathbf{X}} \left(K_{\mathbf{X},\mathbf{X}} + \sigma_n^2\mathbf{I} \right)^{-1} \\
    B_1 &= \mathcal{L}K_{\mathbf{X}_v,\mathbf{X}_v}\mathcal{L}^T + \sigma_v^2\mathbf{I} - A_1 K_{\mathbf{X},\mathbf{X}_v}\mathcal{L}^T \\
    B_2 &= K_{\mathbf{X}_*,\mathbf{X}_*} - A_2 K_{\mathbf{X},\mathbf{X}_*} \\
    B_3 &= K_{\mathbf{X}_*,\mathbf{X}_v}\mathcal{L}^T - A_2 K_{\mathbf{X},\mathbf{X}_v}\mathcal{L}^T\label{eq:B3}.
\end{align}
Further conditioning on $\tilde{C}$ yields in the multivariate Gaussian distribution
\begin{equation}
    \mathbf{f_*} | \mathcal{D}, \tilde{C} \sim \mathcal{N} \left(\mu_{\mathbf{X}_*} + A (\tilde{C} - \mathcal{L}\mu_{\mathbf{X}_v}) + B (\mathbf{y} - \mu_{\mathbf{X}}), \Sigma \right)
\end{equation}
with
\begin{align}
    A &= B_3 B_1^{-1} \label{eq:constrained_A}\\
    B &= A_2 - A A_1 \\
    \Sigma &= B_2 - A B_3^T.
\end{align}
Looking at the calculation of $A$ in \eqref{eq:constrained_A}, the role of the additional virtual observation noise $w_{v,i} \sim \mathcal{N}(0, \sigma_v^2)$ as a numerical regularization for calculating $B_1^{-1}$ becomes clear. An interpretation of $\sigma_v^2$ is that the probability of satisfying the constraints at the virtual locations is slightly reduced. In practice, $\sigma_v^2$ is set to be very small ($\sigma_v^2\approx 10^{-6}$).

Up to this point, the marginalization of the virtual observations from \eqref{eq:unconstrained_joint_distribution2} remained Gaussian as ${\tilde{C} \sim \mathcal{N}(\mu_c, \Sigma_c)}$. By now conditioning on the event $C$ we define $\MatBold{C} = \tilde{C} | \mathcal{D}, C$ resulting in a \emph{truncated} multivariate normal distribution as
\begin{equation}
    \MatBold{C} \sim \mathcal{TN} \big(\underbrace{\mathcal{L}\mu_{\mathbf{X}_v} + A_1 \left( \mathbf{y} - \mu_{\mathbf{X}}\right)}_{\mu_{\mathcal{TN}} \in \R^{P \times 1}}, \underbrace{\vphantom{\mathcal{L}\mu_{\mathbf{X}_v} + A_1 \left( \mathbf{y} - \mu_{\mathbf{X}}\right)}B_1}_{\Sigma_{\mathcal{TN}} \in \R^{P \times P}} ,a(\mathbf{X}_v), b(\mathbf{X}_v) \big)
    \label{eq:truncated_mvn}
\end{equation}
with $\mathcal{TN}(\mu_{\mathcal{TN}}, \Sigma_{\mathcal{TN}}, a, b)$ as the Gaussian $\mathcal{N}(\mu_{\mathcal{TN}}, \Sigma_{\mathcal{TN}})$ conditioned on the hyperbox $[a_1, b_1]\times\dots\times[a_{N_v}, b_{N_v}]$. Following \textcite[Lemma 1]{Agrell_2019} the resulting constrained posterior distribution is a compound Gaussian with a truncated mean as
\begin{equation}
    \mathbf{f_*} | \mathcal{D}, C \sim \mathcal{N} \left(\mu_{\mathbf{X}_*} + A (\MatBold{C} - \mathcal{L}\mu_{\mathbf{X}_v}) + B (\mathbf{y} - \mu_{\mathbf{X}}), \Sigma \right).
    \label{eq:constrained_posterior_distribution}
\end{equation}
This posterior distribution is guaranteed to satisfy \eqref{eq:linear_inequality_constraints_complete} at the \glspl{vop} for ${\sigma_v^2 \to 0}$. However, \textcite{Wang_2016} observed that using a sufficient amount of \glspl{vop} throughout $\mathcal{X}$ results in a high probability of $\mathbf{f}_*$ satisfying the constraints in \eqref{eq:linear_inequality_constraints_complete} in $\mathcal{X}$.
Lemma 2 of \textcite{Agrell_2019} provides a numerically more stable implementation of the factors \eqref{eq:A1} to \eqref{eq:B3} based on Cholesky factorization and is summarized in Appendix \ref{apx:numerically_stable_factors}.

\subsection{Sampling from the Constrained Posterior Distribution}
\label{sec:sampling_posterior}

Even though the posterior distribution \eqref{eq:constrained_posterior_distribution} is Gaussian, it can no longer be calculated in closed form as the mean is truncated. Therefore, the posterior has to be approximated using sampling. This can be achieved following Algorithm~\ref{algo:constrained_posterior} proposed by \textcite[Algorithm~3]{Agrell_2019}.

\begin{algorithm}[h]
\centering
\caption{Sampling form the constrained posterior distribution \cite{Agrell_2019}}
\begin{algorithmic}[1]
\Require Calculate factors $A$, $B$, $\Sigma$, $A_1$, $B_1$
\State Find a matrix $\MatBold{Q}$ s.t. $\MatBold{Q}^T \MatBold{Q} = \Sigma \in \R^{M \times M}$ using Cholesky decomposition.
\State Generate $\tilde{\MatBold{C}}_k$, a $P \times k$ matrix where each column is a sample of $\tilde{C} | \mathcal{D}, C$ from the truncated multivariate normal distribution \eqref{eq:truncated_mvn}.
\State Generate $\MatBold{U}_k$, a $M \times k$ matrix with k samples of the multivariate standard normal distribution $\mathcal{N}(\mathbf{0}, \mathbf{I}_M)$ with $\mathbf{I}_M \in \R^{M \times M}$.
\State Calculate the $M \times k$ matrix where each column is a sample from the distribution $\mathbf{f_*} | \mathcal{D}, C$ in \eqref{eq:constrained_posterior_distribution} as
\begin{equation}
    \left[\mu_{\mathbf{X}_*} + B (\mathbf{y} - \mu_{\mathbf{X}}) \right] \oplus \left[A(- \mathcal{L}\mu_{\mathbf{X}_v} \oplus \tilde{\MatBold{C}}_k) +  \MatBold{Q}\MatBold{U}_k \right]
\end{equation}
with $\oplus$ representing the operation of adding the $M \times 1$ vector on the left hand side to each column of the $M\times k$ matrix on the right hand side.
\end{algorithmic}
\label{algo:constrained_posterior}
\end{algorithm}
The difficulty in sampling from the posterior lies in sampling from the truncated multivariate normal distribution. An approach to rejection sampling via a minimax tilting method was proposed by \textcite{Botev2016} resulting in iid samples of \eqref{eq:truncated_mvn}. The algorithm has shown to perform well with minimal error up to a dimension of $P \approx 100$. However, rejection sampling suffers from the curse of dimensionality as the acceptance rate drops exponentially with growing dimensions. Therefore, one has to fall back to approximate sampling using \gls{mcmc} methods. In the case of sampling from the truncated multivariate normal distribution, Gibbs sampling can be used, as calculating the necessary conditional distributions is possible. For the case of monotonicity constraints on the posterior distribution ($\mathcal{L}=\frac{\pdiff}{\pdiff x_i}$) a Gibbs sampling method has been proposed by \textcite{Wang_2016}. It has been adapted to work with any constraints $a(\cdot), b(\cdot)$ as well as any operator $\mathcal{L}$ as described in this section and is shown in Algorithm~\ref{algo:gibbssampling}. 
The truncated normal distribution from which has to be sampled in line 5, Algorithm~\ref{algo:gibbssampling}, is one dimensional. Therefore, the rejection sampling method by \textcite{Botev2016} can again be used for this sub-task to efficiently generate the iid samples.

\begin{algorithm}[h]
\centering
\caption{Gibbs sampling for the truncated multivariate normal distribution (adapted form \textcite[Section~3.1.2]{Wang_2016})}
\begin{algorithmic}[1]
\Require Calculate mean $\mu_{\mathcal{TN}}$ and covariance $\Sigma_{\mathcal{TN}}$ of the truncated multivariate normal distribution \eqref{eq:truncated_mvn}
\For{$k = 1,\dots, K$}
\For{$i = 1,\dots, P$}
\State $\mu^k_{(i)} = \mu_{\mathcal{TN},(i)} + \Sigma_{\mathcal{TN},(i, \mathbf{-i})} \,  \Sigma^{-1}_{\mathcal{TN}, (\mathbf{-i}, \mathbf{-i})} \left( \tilde{\MatBold{C}}^k_{(\mathbf{-i})} - \mu_{\mathcal{TN},(\mathbf{-i})}\right)$
\State $\sigma_{(i)} = \Sigma_{\mathcal{TN},(i, i)} - \Sigma_{\mathcal{TN},(i, \mathbf{-i})} \,\Sigma^{-1}_{\mathcal{TN}, (\mathbf{-i}, \mathbf{-i})} \,\Sigma^T_{\mathcal{TN},(i, \mathbf{-i})}$
\State Draw a sample $\tilde{\mathrm{C}}^{k+1}_{(i)}|\tilde{\MatBold{C}}^k_{(\mathbf{-i})}, \mathcal{D} \sim \mathcal{TN}(\mu^k_{(i)}, \sigma_{(i)}, a_{(i)}, b_{(i)})$
\EndFor
\EndFor

\noindent with $\tilde{\MatBold{C}}^k_{(\mathbf{-i})} = (\tilde{\mathrm{C}}^{k+1}_{(0)},\dots, \tilde{\mathrm{C}}^{k+1}_{(i-1)}, \tilde{\mathrm{C}}^{k}_{(i+1)},\dots,\tilde{\mathrm{C}}^{k}_{(N_v)})$, $\mu_{\mathcal{TN},(\mathbf{-i})}$ as the mean vector without the $i$th element, and $\Sigma^{-1}_{\mathcal{TN}, (\mathbf{-i}, \mathbf{-i})}$ as the covariance matrix without the $i$th row and $i$th column.
\end{algorithmic}
\label{algo:gibbssampling}
\end{algorithm}

\subsubsection{On the Virtual Observation Locations}

As mentioned, sampling from the truncated multivariate normal distribution \eqref{eq:truncated_mvn} is the most difficult and computationally demanding part of the presented approach to constrain the \gls{gp} posterior. Assuming that one linear operator $\mathcal{L}$ for each of the spatial dimensions $D$ of a \gls{gp} is used, then the dimension of \eqref{eq:truncated_mvn} would be $P = D \cdot N_v$. Furthermore, the number of \glspl{vop} $N_v$ depends exponentially on the dimensions D if the whole hyper cube should be filled equidistantly. Therefore, the dimensions of $P$ are
\begin{equation}
    P = D \cdot N_{v/D}^D .
    \label{eq:scaling_of_P}    
\end{equation}
with $N_{v/D}$ as the number of \glspl{vop} per dimension 
This is visualized for different spatial dimensions in Figure~\ref{fig:dims_vops}.
\begin{figure}[t]
    \centering
    %% Creator: Matplotlib, PGF backend
%%
%% To include the figure in your LaTeX document, write
%%   \input{<filename>.pgf}
%%
%% Make sure the required packages are loaded in your preamble
%%   \usepackage{pgf}
%%
%% Figures using additional raster images can only be included by \input if
%% they are in the same directory as the main LaTeX file. For loading figures
%% from other directories you can use the `import` package
%%   \usepackage{import}
%%
%% and then include the figures with
%%   \import{<path to file>}{<filename>.pgf}
%%
%% Matplotlib used the following preamble
%%   \usepackage{fontspec}
%%
\begingroup%
\makeatletter%
\begin{pgfpicture}%
\pgfpathrectangle{\pgfpointorigin}{\pgfqpoint{4.956413in}{2.756909in}}%
\pgfusepath{use as bounding box, clip}%
\begin{pgfscope}%
\pgfsetbuttcap%
\pgfsetmiterjoin%
\definecolor{currentfill}{rgb}{1.000000,1.000000,1.000000}%
\pgfsetfillcolor{currentfill}%
\pgfsetlinewidth{0.000000pt}%
\definecolor{currentstroke}{rgb}{1.000000,1.000000,1.000000}%
\pgfsetstrokecolor{currentstroke}%
\pgfsetdash{}{0pt}%
\pgfpathmoveto{\pgfqpoint{0.000000in}{0.000000in}}%
\pgfpathlineto{\pgfqpoint{4.956413in}{0.000000in}}%
\pgfpathlineto{\pgfqpoint{4.956413in}{2.756909in}}%
\pgfpathlineto{\pgfqpoint{0.000000in}{2.756909in}}%
\pgfpathclose%
\pgfusepath{fill}%
\end{pgfscope}%
\begin{pgfscope}%
\pgfsetbuttcap%
\pgfsetmiterjoin%
\definecolor{currentfill}{rgb}{1.000000,1.000000,1.000000}%
\pgfsetfillcolor{currentfill}%
\pgfsetlinewidth{0.000000pt}%
\definecolor{currentstroke}{rgb}{0.000000,0.000000,0.000000}%
\pgfsetstrokecolor{currentstroke}%
\pgfsetstrokeopacity{0.000000}%
\pgfsetdash{}{0pt}%
\pgfpathmoveto{\pgfqpoint{0.594770in}{0.413536in}}%
\pgfpathlineto{\pgfqpoint{4.361644in}{0.413536in}}%
\pgfpathlineto{\pgfqpoint{4.361644in}{2.619063in}}%
\pgfpathlineto{\pgfqpoint{0.594770in}{2.619063in}}%
\pgfpathclose%
\pgfusepath{fill}%
\end{pgfscope}%
\begin{pgfscope}%
\pgfpathrectangle{\pgfqpoint{0.594770in}{0.413536in}}{\pgfqpoint{3.766874in}{2.205527in}}%
\pgfusepath{clip}%
\pgfsetbuttcap%
\pgfsetroundjoin%
\definecolor{currentfill}{rgb}{0.611765,0.619608,0.623529}%
\pgfsetfillcolor{currentfill}%
\pgfsetfillopacity{0.300000}%
\pgfsetlinewidth{1.003750pt}%
\definecolor{currentstroke}{rgb}{0.611765,0.619608,0.623529}%
\pgfsetstrokecolor{currentstroke}%
\pgfsetstrokeopacity{0.300000}%
\pgfsetdash{}{0pt}%
\pgfsys@defobject{currentmarker}{\pgfqpoint{0.594770in}{0.964918in}}{\pgfqpoint{4.361644in}{2.619063in}}{%
\pgfpathmoveto{\pgfqpoint{0.594770in}{2.619063in}}%
\pgfpathlineto{\pgfqpoint{0.594770in}{0.964918in}}%
\pgfpathlineto{\pgfqpoint{1.013311in}{0.964918in}}%
\pgfpathlineto{\pgfqpoint{1.431853in}{0.964918in}}%
\pgfpathlineto{\pgfqpoint{1.850394in}{0.964918in}}%
\pgfpathlineto{\pgfqpoint{2.268936in}{0.964918in}}%
\pgfpathlineto{\pgfqpoint{2.687478in}{0.964918in}}%
\pgfpathlineto{\pgfqpoint{3.106019in}{0.964918in}}%
\pgfpathlineto{\pgfqpoint{3.524561in}{0.964918in}}%
\pgfpathlineto{\pgfqpoint{3.943102in}{0.964918in}}%
\pgfpathlineto{\pgfqpoint{4.361644in}{0.964918in}}%
\pgfpathlineto{\pgfqpoint{4.361644in}{2.619063in}}%
\pgfpathlineto{\pgfqpoint{4.361644in}{2.619063in}}%
\pgfpathlineto{\pgfqpoint{3.943102in}{2.619063in}}%
\pgfpathlineto{\pgfqpoint{3.524561in}{2.619063in}}%
\pgfpathlineto{\pgfqpoint{3.106019in}{2.619063in}}%
\pgfpathlineto{\pgfqpoint{2.687478in}{2.619063in}}%
\pgfpathlineto{\pgfqpoint{2.268936in}{2.619063in}}%
\pgfpathlineto{\pgfqpoint{1.850394in}{2.619063in}}%
\pgfpathlineto{\pgfqpoint{1.431853in}{2.619063in}}%
\pgfpathlineto{\pgfqpoint{1.013311in}{2.619063in}}%
\pgfpathlineto{\pgfqpoint{0.594770in}{2.619063in}}%
\pgfpathclose%
\pgfusepath{stroke,fill}%
}%
\begin{pgfscope}%
\pgfsys@transformshift{0.000000in}{0.000000in}%
\pgfsys@useobject{currentmarker}{}%
\end{pgfscope}%
\end{pgfscope}%
\begin{pgfscope}%
\pgfpathrectangle{\pgfqpoint{0.594770in}{0.413536in}}{\pgfqpoint{3.766874in}{2.205527in}}%
\pgfusepath{clip}%
\pgfsetbuttcap%
\pgfsetroundjoin%
\definecolor{currentfill}{rgb}{0.611765,0.619608,0.623529}%
\pgfsetfillcolor{currentfill}%
\pgfsetfillopacity{0.500000}%
\pgfsetlinewidth{1.003750pt}%
\definecolor{currentstroke}{rgb}{0.611765,0.619608,0.623529}%
\pgfsetstrokecolor{currentstroke}%
\pgfsetstrokeopacity{0.500000}%
\pgfsetdash{}{0pt}%
\pgfsys@defobject{currentmarker}{\pgfqpoint{0.594770in}{1.791991in}}{\pgfqpoint{4.361644in}{2.619063in}}{%
\pgfpathmoveto{\pgfqpoint{0.594770in}{2.619063in}}%
\pgfpathlineto{\pgfqpoint{0.594770in}{1.791991in}}%
\pgfpathlineto{\pgfqpoint{1.013311in}{1.791991in}}%
\pgfpathlineto{\pgfqpoint{1.431853in}{1.791991in}}%
\pgfpathlineto{\pgfqpoint{1.850394in}{1.791991in}}%
\pgfpathlineto{\pgfqpoint{2.268936in}{1.791991in}}%
\pgfpathlineto{\pgfqpoint{2.687478in}{1.791991in}}%
\pgfpathlineto{\pgfqpoint{3.106019in}{1.791991in}}%
\pgfpathlineto{\pgfqpoint{3.524561in}{1.791991in}}%
\pgfpathlineto{\pgfqpoint{3.943102in}{1.791991in}}%
\pgfpathlineto{\pgfqpoint{4.361644in}{1.791991in}}%
\pgfpathlineto{\pgfqpoint{4.361644in}{2.619063in}}%
\pgfpathlineto{\pgfqpoint{4.361644in}{2.619063in}}%
\pgfpathlineto{\pgfqpoint{3.943102in}{2.619063in}}%
\pgfpathlineto{\pgfqpoint{3.524561in}{2.619063in}}%
\pgfpathlineto{\pgfqpoint{3.106019in}{2.619063in}}%
\pgfpathlineto{\pgfqpoint{2.687478in}{2.619063in}}%
\pgfpathlineto{\pgfqpoint{2.268936in}{2.619063in}}%
\pgfpathlineto{\pgfqpoint{1.850394in}{2.619063in}}%
\pgfpathlineto{\pgfqpoint{1.431853in}{2.619063in}}%
\pgfpathlineto{\pgfqpoint{1.013311in}{2.619063in}}%
\pgfpathlineto{\pgfqpoint{0.594770in}{2.619063in}}%
\pgfpathclose%
\pgfusepath{stroke,fill}%
}%
\begin{pgfscope}%
\pgfsys@transformshift{0.000000in}{0.000000in}%
\pgfsys@useobject{currentmarker}{}%
\end{pgfscope}%
\end{pgfscope}%
\begin{pgfscope}%
\pgfsetbuttcap%
\pgfsetroundjoin%
\definecolor{currentfill}{rgb}{0.000000,0.000000,0.000000}%
\pgfsetfillcolor{currentfill}%
\pgfsetlinewidth{0.803000pt}%
\definecolor{currentstroke}{rgb}{0.000000,0.000000,0.000000}%
\pgfsetstrokecolor{currentstroke}%
\pgfsetdash{}{0pt}%
\pgfsys@defobject{currentmarker}{\pgfqpoint{0.000000in}{-0.048611in}}{\pgfqpoint{0.000000in}{0.000000in}}{%
\pgfpathmoveto{\pgfqpoint{0.000000in}{0.000000in}}%
\pgfpathlineto{\pgfqpoint{0.000000in}{-0.048611in}}%
\pgfusepath{stroke,fill}%
}%
\begin{pgfscope}%
\pgfsys@transformshift{0.594770in}{0.413536in}%
\pgfsys@useobject{currentmarker}{}%
\end{pgfscope}%
\end{pgfscope}%
\begin{pgfscope}%
\definecolor{textcolor}{rgb}{0.000000,0.000000,0.000000}%
\pgfsetstrokecolor{textcolor}%
\pgfsetfillcolor{textcolor}%
\pgftext[x=0.594770in,y=0.316314in,,top]{\color{textcolor}\rmfamily\fontsize{10.000000}{12.000000}\selectfont \(\displaystyle {1}\)}%
\end{pgfscope}%
\begin{pgfscope}%
\pgfsetbuttcap%
\pgfsetroundjoin%
\definecolor{currentfill}{rgb}{0.000000,0.000000,0.000000}%
\pgfsetfillcolor{currentfill}%
\pgfsetlinewidth{0.803000pt}%
\definecolor{currentstroke}{rgb}{0.000000,0.000000,0.000000}%
\pgfsetstrokecolor{currentstroke}%
\pgfsetdash{}{0pt}%
\pgfsys@defobject{currentmarker}{\pgfqpoint{0.000000in}{-0.048611in}}{\pgfqpoint{0.000000in}{0.000000in}}{%
\pgfpathmoveto{\pgfqpoint{0.000000in}{0.000000in}}%
\pgfpathlineto{\pgfqpoint{0.000000in}{-0.048611in}}%
\pgfusepath{stroke,fill}%
}%
\begin{pgfscope}%
\pgfsys@transformshift{1.013311in}{0.413536in}%
\pgfsys@useobject{currentmarker}{}%
\end{pgfscope}%
\end{pgfscope}%
\begin{pgfscope}%
\definecolor{textcolor}{rgb}{0.000000,0.000000,0.000000}%
\pgfsetstrokecolor{textcolor}%
\pgfsetfillcolor{textcolor}%
\pgftext[x=1.013311in,y=0.316314in,,top]{\color{textcolor}\rmfamily\fontsize{10.000000}{12.000000}\selectfont \(\displaystyle {2}\)}%
\end{pgfscope}%
\begin{pgfscope}%
\pgfsetbuttcap%
\pgfsetroundjoin%
\definecolor{currentfill}{rgb}{0.000000,0.000000,0.000000}%
\pgfsetfillcolor{currentfill}%
\pgfsetlinewidth{0.803000pt}%
\definecolor{currentstroke}{rgb}{0.000000,0.000000,0.000000}%
\pgfsetstrokecolor{currentstroke}%
\pgfsetdash{}{0pt}%
\pgfsys@defobject{currentmarker}{\pgfqpoint{0.000000in}{-0.048611in}}{\pgfqpoint{0.000000in}{0.000000in}}{%
\pgfpathmoveto{\pgfqpoint{0.000000in}{0.000000in}}%
\pgfpathlineto{\pgfqpoint{0.000000in}{-0.048611in}}%
\pgfusepath{stroke,fill}%
}%
\begin{pgfscope}%
\pgfsys@transformshift{1.431853in}{0.413536in}%
\pgfsys@useobject{currentmarker}{}%
\end{pgfscope}%
\end{pgfscope}%
\begin{pgfscope}%
\definecolor{textcolor}{rgb}{0.000000,0.000000,0.000000}%
\pgfsetstrokecolor{textcolor}%
\pgfsetfillcolor{textcolor}%
\pgftext[x=1.431853in,y=0.316314in,,top]{\color{textcolor}\rmfamily\fontsize{10.000000}{12.000000}\selectfont \(\displaystyle {3}\)}%
\end{pgfscope}%
\begin{pgfscope}%
\pgfsetbuttcap%
\pgfsetroundjoin%
\definecolor{currentfill}{rgb}{0.000000,0.000000,0.000000}%
\pgfsetfillcolor{currentfill}%
\pgfsetlinewidth{0.803000pt}%
\definecolor{currentstroke}{rgb}{0.000000,0.000000,0.000000}%
\pgfsetstrokecolor{currentstroke}%
\pgfsetdash{}{0pt}%
\pgfsys@defobject{currentmarker}{\pgfqpoint{0.000000in}{-0.048611in}}{\pgfqpoint{0.000000in}{0.000000in}}{%
\pgfpathmoveto{\pgfqpoint{0.000000in}{0.000000in}}%
\pgfpathlineto{\pgfqpoint{0.000000in}{-0.048611in}}%
\pgfusepath{stroke,fill}%
}%
\begin{pgfscope}%
\pgfsys@transformshift{1.850394in}{0.413536in}%
\pgfsys@useobject{currentmarker}{}%
\end{pgfscope}%
\end{pgfscope}%
\begin{pgfscope}%
\definecolor{textcolor}{rgb}{0.000000,0.000000,0.000000}%
\pgfsetstrokecolor{textcolor}%
\pgfsetfillcolor{textcolor}%
\pgftext[x=1.850394in,y=0.316314in,,top]{\color{textcolor}\rmfamily\fontsize{10.000000}{12.000000}\selectfont \(\displaystyle {4}\)}%
\end{pgfscope}%
\begin{pgfscope}%
\pgfsetbuttcap%
\pgfsetroundjoin%
\definecolor{currentfill}{rgb}{0.000000,0.000000,0.000000}%
\pgfsetfillcolor{currentfill}%
\pgfsetlinewidth{0.803000pt}%
\definecolor{currentstroke}{rgb}{0.000000,0.000000,0.000000}%
\pgfsetstrokecolor{currentstroke}%
\pgfsetdash{}{0pt}%
\pgfsys@defobject{currentmarker}{\pgfqpoint{0.000000in}{-0.048611in}}{\pgfqpoint{0.000000in}{0.000000in}}{%
\pgfpathmoveto{\pgfqpoint{0.000000in}{0.000000in}}%
\pgfpathlineto{\pgfqpoint{0.000000in}{-0.048611in}}%
\pgfusepath{stroke,fill}%
}%
\begin{pgfscope}%
\pgfsys@transformshift{2.268936in}{0.413536in}%
\pgfsys@useobject{currentmarker}{}%
\end{pgfscope}%
\end{pgfscope}%
\begin{pgfscope}%
\definecolor{textcolor}{rgb}{0.000000,0.000000,0.000000}%
\pgfsetstrokecolor{textcolor}%
\pgfsetfillcolor{textcolor}%
\pgftext[x=2.268936in,y=0.316314in,,top]{\color{textcolor}\rmfamily\fontsize{10.000000}{12.000000}\selectfont \(\displaystyle {5}\)}%
\end{pgfscope}%
\begin{pgfscope}%
\pgfsetbuttcap%
\pgfsetroundjoin%
\definecolor{currentfill}{rgb}{0.000000,0.000000,0.000000}%
\pgfsetfillcolor{currentfill}%
\pgfsetlinewidth{0.803000pt}%
\definecolor{currentstroke}{rgb}{0.000000,0.000000,0.000000}%
\pgfsetstrokecolor{currentstroke}%
\pgfsetdash{}{0pt}%
\pgfsys@defobject{currentmarker}{\pgfqpoint{0.000000in}{-0.048611in}}{\pgfqpoint{0.000000in}{0.000000in}}{%
\pgfpathmoveto{\pgfqpoint{0.000000in}{0.000000in}}%
\pgfpathlineto{\pgfqpoint{0.000000in}{-0.048611in}}%
\pgfusepath{stroke,fill}%
}%
\begin{pgfscope}%
\pgfsys@transformshift{2.687478in}{0.413536in}%
\pgfsys@useobject{currentmarker}{}%
\end{pgfscope}%
\end{pgfscope}%
\begin{pgfscope}%
\definecolor{textcolor}{rgb}{0.000000,0.000000,0.000000}%
\pgfsetstrokecolor{textcolor}%
\pgfsetfillcolor{textcolor}%
\pgftext[x=2.687478in,y=0.316314in,,top]{\color{textcolor}\rmfamily\fontsize{10.000000}{12.000000}\selectfont \(\displaystyle {6}\)}%
\end{pgfscope}%
\begin{pgfscope}%
\pgfsetbuttcap%
\pgfsetroundjoin%
\definecolor{currentfill}{rgb}{0.000000,0.000000,0.000000}%
\pgfsetfillcolor{currentfill}%
\pgfsetlinewidth{0.803000pt}%
\definecolor{currentstroke}{rgb}{0.000000,0.000000,0.000000}%
\pgfsetstrokecolor{currentstroke}%
\pgfsetdash{}{0pt}%
\pgfsys@defobject{currentmarker}{\pgfqpoint{0.000000in}{-0.048611in}}{\pgfqpoint{0.000000in}{0.000000in}}{%
\pgfpathmoveto{\pgfqpoint{0.000000in}{0.000000in}}%
\pgfpathlineto{\pgfqpoint{0.000000in}{-0.048611in}}%
\pgfusepath{stroke,fill}%
}%
\begin{pgfscope}%
\pgfsys@transformshift{3.106019in}{0.413536in}%
\pgfsys@useobject{currentmarker}{}%
\end{pgfscope}%
\end{pgfscope}%
\begin{pgfscope}%
\definecolor{textcolor}{rgb}{0.000000,0.000000,0.000000}%
\pgfsetstrokecolor{textcolor}%
\pgfsetfillcolor{textcolor}%
\pgftext[x=3.106019in,y=0.316314in,,top]{\color{textcolor}\rmfamily\fontsize{10.000000}{12.000000}\selectfont \(\displaystyle {7}\)}%
\end{pgfscope}%
\begin{pgfscope}%
\pgfsetbuttcap%
\pgfsetroundjoin%
\definecolor{currentfill}{rgb}{0.000000,0.000000,0.000000}%
\pgfsetfillcolor{currentfill}%
\pgfsetlinewidth{0.803000pt}%
\definecolor{currentstroke}{rgb}{0.000000,0.000000,0.000000}%
\pgfsetstrokecolor{currentstroke}%
\pgfsetdash{}{0pt}%
\pgfsys@defobject{currentmarker}{\pgfqpoint{0.000000in}{-0.048611in}}{\pgfqpoint{0.000000in}{0.000000in}}{%
\pgfpathmoveto{\pgfqpoint{0.000000in}{0.000000in}}%
\pgfpathlineto{\pgfqpoint{0.000000in}{-0.048611in}}%
\pgfusepath{stroke,fill}%
}%
\begin{pgfscope}%
\pgfsys@transformshift{3.524561in}{0.413536in}%
\pgfsys@useobject{currentmarker}{}%
\end{pgfscope}%
\end{pgfscope}%
\begin{pgfscope}%
\definecolor{textcolor}{rgb}{0.000000,0.000000,0.000000}%
\pgfsetstrokecolor{textcolor}%
\pgfsetfillcolor{textcolor}%
\pgftext[x=3.524561in,y=0.316314in,,top]{\color{textcolor}\rmfamily\fontsize{10.000000}{12.000000}\selectfont \(\displaystyle {8}\)}%
\end{pgfscope}%
\begin{pgfscope}%
\pgfsetbuttcap%
\pgfsetroundjoin%
\definecolor{currentfill}{rgb}{0.000000,0.000000,0.000000}%
\pgfsetfillcolor{currentfill}%
\pgfsetlinewidth{0.803000pt}%
\definecolor{currentstroke}{rgb}{0.000000,0.000000,0.000000}%
\pgfsetstrokecolor{currentstroke}%
\pgfsetdash{}{0pt}%
\pgfsys@defobject{currentmarker}{\pgfqpoint{0.000000in}{-0.048611in}}{\pgfqpoint{0.000000in}{0.000000in}}{%
\pgfpathmoveto{\pgfqpoint{0.000000in}{0.000000in}}%
\pgfpathlineto{\pgfqpoint{0.000000in}{-0.048611in}}%
\pgfusepath{stroke,fill}%
}%
\begin{pgfscope}%
\pgfsys@transformshift{3.943102in}{0.413536in}%
\pgfsys@useobject{currentmarker}{}%
\end{pgfscope}%
\end{pgfscope}%
\begin{pgfscope}%
\definecolor{textcolor}{rgb}{0.000000,0.000000,0.000000}%
\pgfsetstrokecolor{textcolor}%
\pgfsetfillcolor{textcolor}%
\pgftext[x=3.943102in,y=0.316314in,,top]{\color{textcolor}\rmfamily\fontsize{10.000000}{12.000000}\selectfont \(\displaystyle {9}\)}%
\end{pgfscope}%
\begin{pgfscope}%
\pgfsetbuttcap%
\pgfsetroundjoin%
\definecolor{currentfill}{rgb}{0.000000,0.000000,0.000000}%
\pgfsetfillcolor{currentfill}%
\pgfsetlinewidth{0.803000pt}%
\definecolor{currentstroke}{rgb}{0.000000,0.000000,0.000000}%
\pgfsetstrokecolor{currentstroke}%
\pgfsetdash{}{0pt}%
\pgfsys@defobject{currentmarker}{\pgfqpoint{0.000000in}{-0.048611in}}{\pgfqpoint{0.000000in}{0.000000in}}{%
\pgfpathmoveto{\pgfqpoint{0.000000in}{0.000000in}}%
\pgfpathlineto{\pgfqpoint{0.000000in}{-0.048611in}}%
\pgfusepath{stroke,fill}%
}%
\begin{pgfscope}%
\pgfsys@transformshift{4.361644in}{0.413536in}%
\pgfsys@useobject{currentmarker}{}%
\end{pgfscope}%
\end{pgfscope}%
\begin{pgfscope}%
\definecolor{textcolor}{rgb}{0.000000,0.000000,0.000000}%
\pgfsetstrokecolor{textcolor}%
\pgfsetfillcolor{textcolor}%
\pgftext[x=4.361644in,y=0.316314in,,top]{\color{textcolor}\rmfamily\fontsize{10.000000}{12.000000}\selectfont \(\displaystyle {10}\)}%
\end{pgfscope}%
\begin{pgfscope}%
\definecolor{textcolor}{rgb}{0.000000,0.000000,0.000000}%
\pgfsetstrokecolor{textcolor}%
\pgfsetfillcolor{textcolor}%
\pgftext[x=2.478207in,y=0.137425in,,top]{\color{textcolor}\rmfamily\fontsize{10.000000}{12.000000}\selectfont \# VOPs per dimension}%
\end{pgfscope}%
\begin{pgfscope}%
\pgfsetbuttcap%
\pgfsetroundjoin%
\definecolor{currentfill}{rgb}{0.000000,0.000000,0.000000}%
\pgfsetfillcolor{currentfill}%
\pgfsetlinewidth{0.803000pt}%
\definecolor{currentstroke}{rgb}{0.000000,0.000000,0.000000}%
\pgfsetstrokecolor{currentstroke}%
\pgfsetdash{}{0pt}%
\pgfsys@defobject{currentmarker}{\pgfqpoint{-0.048611in}{0.000000in}}{\pgfqpoint{-0.000000in}{0.000000in}}{%
\pgfpathmoveto{\pgfqpoint{-0.000000in}{0.000000in}}%
\pgfpathlineto{\pgfqpoint{-0.048611in}{0.000000in}}%
\pgfusepath{stroke,fill}%
}%
\begin{pgfscope}%
\pgfsys@transformshift{0.594770in}{0.413536in}%
\pgfsys@useobject{currentmarker}{}%
\end{pgfscope}%
\end{pgfscope}%
\begin{pgfscope}%
\definecolor{textcolor}{rgb}{0.000000,0.000000,0.000000}%
\pgfsetstrokecolor{textcolor}%
\pgfsetfillcolor{textcolor}%
\pgftext[x=0.428103in, y=0.365342in, left, base]{\color{textcolor}\rmfamily\fontsize{10.000000}{12.000000}\selectfont \(\displaystyle {0}\)}%
\end{pgfscope}%
\begin{pgfscope}%
\pgfsetbuttcap%
\pgfsetroundjoin%
\definecolor{currentfill}{rgb}{0.000000,0.000000,0.000000}%
\pgfsetfillcolor{currentfill}%
\pgfsetlinewidth{0.803000pt}%
\definecolor{currentstroke}{rgb}{0.000000,0.000000,0.000000}%
\pgfsetstrokecolor{currentstroke}%
\pgfsetdash{}{0pt}%
\pgfsys@defobject{currentmarker}{\pgfqpoint{-0.048611in}{0.000000in}}{\pgfqpoint{-0.000000in}{0.000000in}}{%
\pgfpathmoveto{\pgfqpoint{-0.000000in}{0.000000in}}%
\pgfpathlineto{\pgfqpoint{-0.048611in}{0.000000in}}%
\pgfusepath{stroke,fill}%
}%
\begin{pgfscope}%
\pgfsys@transformshift{0.594770in}{0.964918in}%
\pgfsys@useobject{currentmarker}{}%
\end{pgfscope}%
\end{pgfscope}%
\begin{pgfscope}%
\definecolor{textcolor}{rgb}{0.000000,0.000000,0.000000}%
\pgfsetstrokecolor{textcolor}%
\pgfsetfillcolor{textcolor}%
\pgftext[x=0.289213in, y=0.916724in, left, base]{\color{textcolor}\rmfamily\fontsize{10.000000}{12.000000}\selectfont \(\displaystyle {100}\)}%
\end{pgfscope}%
\begin{pgfscope}%
\pgfsetbuttcap%
\pgfsetroundjoin%
\definecolor{currentfill}{rgb}{0.000000,0.000000,0.000000}%
\pgfsetfillcolor{currentfill}%
\pgfsetlinewidth{0.803000pt}%
\definecolor{currentstroke}{rgb}{0.000000,0.000000,0.000000}%
\pgfsetstrokecolor{currentstroke}%
\pgfsetdash{}{0pt}%
\pgfsys@defobject{currentmarker}{\pgfqpoint{-0.048611in}{0.000000in}}{\pgfqpoint{-0.000000in}{0.000000in}}{%
\pgfpathmoveto{\pgfqpoint{-0.000000in}{0.000000in}}%
\pgfpathlineto{\pgfqpoint{-0.048611in}{0.000000in}}%
\pgfusepath{stroke,fill}%
}%
\begin{pgfscope}%
\pgfsys@transformshift{0.594770in}{1.516300in}%
\pgfsys@useobject{currentmarker}{}%
\end{pgfscope}%
\end{pgfscope}%
\begin{pgfscope}%
\definecolor{textcolor}{rgb}{0.000000,0.000000,0.000000}%
\pgfsetstrokecolor{textcolor}%
\pgfsetfillcolor{textcolor}%
\pgftext[x=0.289213in, y=1.468105in, left, base]{\color{textcolor}\rmfamily\fontsize{10.000000}{12.000000}\selectfont \(\displaystyle {200}\)}%
\end{pgfscope}%
\begin{pgfscope}%
\pgfsetbuttcap%
\pgfsetroundjoin%
\definecolor{currentfill}{rgb}{0.000000,0.000000,0.000000}%
\pgfsetfillcolor{currentfill}%
\pgfsetlinewidth{0.803000pt}%
\definecolor{currentstroke}{rgb}{0.000000,0.000000,0.000000}%
\pgfsetstrokecolor{currentstroke}%
\pgfsetdash{}{0pt}%
\pgfsys@defobject{currentmarker}{\pgfqpoint{-0.048611in}{0.000000in}}{\pgfqpoint{-0.000000in}{0.000000in}}{%
\pgfpathmoveto{\pgfqpoint{-0.000000in}{0.000000in}}%
\pgfpathlineto{\pgfqpoint{-0.048611in}{0.000000in}}%
\pgfusepath{stroke,fill}%
}%
\begin{pgfscope}%
\pgfsys@transformshift{0.594770in}{2.067682in}%
\pgfsys@useobject{currentmarker}{}%
\end{pgfscope}%
\end{pgfscope}%
\begin{pgfscope}%
\definecolor{textcolor}{rgb}{0.000000,0.000000,0.000000}%
\pgfsetstrokecolor{textcolor}%
\pgfsetfillcolor{textcolor}%
\pgftext[x=0.289213in, y=2.019487in, left, base]{\color{textcolor}\rmfamily\fontsize{10.000000}{12.000000}\selectfont \(\displaystyle {300}\)}%
\end{pgfscope}%
\begin{pgfscope}%
\pgfsetbuttcap%
\pgfsetroundjoin%
\definecolor{currentfill}{rgb}{0.000000,0.000000,0.000000}%
\pgfsetfillcolor{currentfill}%
\pgfsetlinewidth{0.803000pt}%
\definecolor{currentstroke}{rgb}{0.000000,0.000000,0.000000}%
\pgfsetstrokecolor{currentstroke}%
\pgfsetdash{}{0pt}%
\pgfsys@defobject{currentmarker}{\pgfqpoint{-0.048611in}{0.000000in}}{\pgfqpoint{-0.000000in}{0.000000in}}{%
\pgfpathmoveto{\pgfqpoint{-0.000000in}{0.000000in}}%
\pgfpathlineto{\pgfqpoint{-0.048611in}{0.000000in}}%
\pgfusepath{stroke,fill}%
}%
\begin{pgfscope}%
\pgfsys@transformshift{0.594770in}{2.619063in}%
\pgfsys@useobject{currentmarker}{}%
\end{pgfscope}%
\end{pgfscope}%
\begin{pgfscope}%
\definecolor{textcolor}{rgb}{0.000000,0.000000,0.000000}%
\pgfsetstrokecolor{textcolor}%
\pgfsetfillcolor{textcolor}%
\pgftext[x=0.289213in, y=2.570869in, left, base]{\color{textcolor}\rmfamily\fontsize{10.000000}{12.000000}\selectfont \(\displaystyle {400}\)}%
\end{pgfscope}%
\begin{pgfscope}%
\definecolor{textcolor}{rgb}{0.000000,0.000000,0.000000}%
\pgfsetstrokecolor{textcolor}%
\pgfsetfillcolor{textcolor}%
\pgftext[x=0.233658in,y=1.516300in,,bottom,rotate=90.000000]{\color{textcolor}\rmfamily\fontsize{10.000000}{12.000000}\selectfont \(\displaystyle P\)}%
\end{pgfscope}%
\begin{pgfscope}%
\pgfpathrectangle{\pgfqpoint{0.594770in}{0.413536in}}{\pgfqpoint{3.766874in}{2.205527in}}%
\pgfusepath{clip}%
\pgfsetbuttcap%
\pgfsetroundjoin%
\pgfsetlinewidth{1.505625pt}%
\definecolor{currentstroke}{rgb}{0.392157,0.396078,0.403922}%
\pgfsetstrokecolor{currentstroke}%
\pgfsetdash{{5.550000pt}{2.400000pt}}{0.000000pt}%
\pgfpathmoveto{\pgfqpoint{0.594770in}{0.964918in}}%
\pgfpathlineto{\pgfqpoint{4.361644in}{0.964918in}}%
\pgfusepath{stroke}%
\end{pgfscope}%
\begin{pgfscope}%
\pgfpathrectangle{\pgfqpoint{0.594770in}{0.413536in}}{\pgfqpoint{3.766874in}{2.205527in}}%
\pgfusepath{clip}%
\pgfsetbuttcap%
\pgfsetroundjoin%
\pgfsetlinewidth{1.505625pt}%
\definecolor{currentstroke}{rgb}{0.392157,0.396078,0.403922}%
\pgfsetstrokecolor{currentstroke}%
\pgfsetdash{{5.550000pt}{2.400000pt}}{0.000000pt}%
\pgfpathmoveto{\pgfqpoint{0.594770in}{1.791991in}}%
\pgfpathlineto{\pgfqpoint{4.361644in}{1.791991in}}%
\pgfusepath{stroke}%
\end{pgfscope}%
\begin{pgfscope}%
\pgfpathrectangle{\pgfqpoint{0.594770in}{0.413536in}}{\pgfqpoint{3.766874in}{2.205527in}}%
\pgfusepath{clip}%
\pgfsetrectcap%
\pgfsetroundjoin%
\pgfsetlinewidth{1.505625pt}%
\definecolor{currentstroke}{rgb}{0.000000,0.329412,0.623529}%
\pgfsetstrokecolor{currentstroke}%
\pgfsetdash{}{0pt}%
\pgfpathmoveto{\pgfqpoint{0.594770in}{0.419050in}}%
\pgfpathlineto{\pgfqpoint{1.013311in}{0.424564in}}%
\pgfpathlineto{\pgfqpoint{1.431853in}{0.430078in}}%
\pgfpathlineto{\pgfqpoint{1.850394in}{0.435592in}}%
\pgfpathlineto{\pgfqpoint{2.268936in}{0.441105in}}%
\pgfpathlineto{\pgfqpoint{2.687478in}{0.446619in}}%
\pgfpathlineto{\pgfqpoint{3.106019in}{0.452133in}}%
\pgfpathlineto{\pgfqpoint{3.524561in}{0.457647in}}%
\pgfpathlineto{\pgfqpoint{3.943102in}{0.463161in}}%
\pgfpathlineto{\pgfqpoint{4.361644in}{0.468674in}}%
\pgfusepath{stroke}%
\end{pgfscope}%
\begin{pgfscope}%
\pgfpathrectangle{\pgfqpoint{0.594770in}{0.413536in}}{\pgfqpoint{3.766874in}{2.205527in}}%
\pgfusepath{clip}%
\pgfsetbuttcap%
\pgfsetroundjoin%
\definecolor{currentfill}{rgb}{0.000000,0.329412,0.623529}%
\pgfsetfillcolor{currentfill}%
\pgfsetlinewidth{1.003750pt}%
\definecolor{currentstroke}{rgb}{0.000000,0.329412,0.623529}%
\pgfsetstrokecolor{currentstroke}%
\pgfsetdash{}{0pt}%
\pgfsys@defobject{currentmarker}{\pgfqpoint{-0.041667in}{-0.041667in}}{\pgfqpoint{0.041667in}{0.041667in}}{%
\pgfpathmoveto{\pgfqpoint{0.000000in}{-0.041667in}}%
\pgfpathcurveto{\pgfqpoint{0.011050in}{-0.041667in}}{\pgfqpoint{0.021649in}{-0.037276in}}{\pgfqpoint{0.029463in}{-0.029463in}}%
\pgfpathcurveto{\pgfqpoint{0.037276in}{-0.021649in}}{\pgfqpoint{0.041667in}{-0.011050in}}{\pgfqpoint{0.041667in}{0.000000in}}%
\pgfpathcurveto{\pgfqpoint{0.041667in}{0.011050in}}{\pgfqpoint{0.037276in}{0.021649in}}{\pgfqpoint{0.029463in}{0.029463in}}%
\pgfpathcurveto{\pgfqpoint{0.021649in}{0.037276in}}{\pgfqpoint{0.011050in}{0.041667in}}{\pgfqpoint{0.000000in}{0.041667in}}%
\pgfpathcurveto{\pgfqpoint{-0.011050in}{0.041667in}}{\pgfqpoint{-0.021649in}{0.037276in}}{\pgfqpoint{-0.029463in}{0.029463in}}%
\pgfpathcurveto{\pgfqpoint{-0.037276in}{0.021649in}}{\pgfqpoint{-0.041667in}{0.011050in}}{\pgfqpoint{-0.041667in}{0.000000in}}%
\pgfpathcurveto{\pgfqpoint{-0.041667in}{-0.011050in}}{\pgfqpoint{-0.037276in}{-0.021649in}}{\pgfqpoint{-0.029463in}{-0.029463in}}%
\pgfpathcurveto{\pgfqpoint{-0.021649in}{-0.037276in}}{\pgfqpoint{-0.011050in}{-0.041667in}}{\pgfqpoint{0.000000in}{-0.041667in}}%
\pgfpathclose%
\pgfusepath{stroke,fill}%
}%
\begin{pgfscope}%
\pgfsys@transformshift{0.594770in}{0.419050in}%
\pgfsys@useobject{currentmarker}{}%
\end{pgfscope}%
\begin{pgfscope}%
\pgfsys@transformshift{1.013311in}{0.424564in}%
\pgfsys@useobject{currentmarker}{}%
\end{pgfscope}%
\begin{pgfscope}%
\pgfsys@transformshift{1.431853in}{0.430078in}%
\pgfsys@useobject{currentmarker}{}%
\end{pgfscope}%
\begin{pgfscope}%
\pgfsys@transformshift{1.850394in}{0.435592in}%
\pgfsys@useobject{currentmarker}{}%
\end{pgfscope}%
\begin{pgfscope}%
\pgfsys@transformshift{2.268936in}{0.441105in}%
\pgfsys@useobject{currentmarker}{}%
\end{pgfscope}%
\begin{pgfscope}%
\pgfsys@transformshift{2.687478in}{0.446619in}%
\pgfsys@useobject{currentmarker}{}%
\end{pgfscope}%
\begin{pgfscope}%
\pgfsys@transformshift{3.106019in}{0.452133in}%
\pgfsys@useobject{currentmarker}{}%
\end{pgfscope}%
\begin{pgfscope}%
\pgfsys@transformshift{3.524561in}{0.457647in}%
\pgfsys@useobject{currentmarker}{}%
\end{pgfscope}%
\begin{pgfscope}%
\pgfsys@transformshift{3.943102in}{0.463161in}%
\pgfsys@useobject{currentmarker}{}%
\end{pgfscope}%
\begin{pgfscope}%
\pgfsys@transformshift{4.361644in}{0.468674in}%
\pgfsys@useobject{currentmarker}{}%
\end{pgfscope}%
\end{pgfscope}%
\begin{pgfscope}%
\pgfpathrectangle{\pgfqpoint{0.594770in}{0.413536in}}{\pgfqpoint{3.766874in}{2.205527in}}%
\pgfusepath{clip}%
\pgfsetrectcap%
\pgfsetroundjoin%
\pgfsetlinewidth{1.505625pt}%
\definecolor{currentstroke}{rgb}{0.964706,0.658824,0.000000}%
\pgfsetstrokecolor{currentstroke}%
\pgfsetdash{}{0pt}%
\pgfpathmoveto{\pgfqpoint{0.594770in}{0.424564in}}%
\pgfpathlineto{\pgfqpoint{1.013311in}{0.457647in}}%
\pgfpathlineto{\pgfqpoint{1.431853in}{0.512785in}}%
\pgfpathlineto{\pgfqpoint{1.850394in}{0.589978in}}%
\pgfpathlineto{\pgfqpoint{2.268936in}{0.689227in}}%
\pgfpathlineto{\pgfqpoint{2.687478in}{0.810531in}}%
\pgfpathlineto{\pgfqpoint{3.106019in}{0.953890in}}%
\pgfpathlineto{\pgfqpoint{3.524561in}{1.119305in}}%
\pgfpathlineto{\pgfqpoint{3.943102in}{1.306775in}}%
\pgfpathlineto{\pgfqpoint{4.361644in}{1.516300in}}%
\pgfusepath{stroke}%
\end{pgfscope}%
\begin{pgfscope}%
\pgfpathrectangle{\pgfqpoint{0.594770in}{0.413536in}}{\pgfqpoint{3.766874in}{2.205527in}}%
\pgfusepath{clip}%
\pgfsetbuttcap%
\pgfsetroundjoin%
\definecolor{currentfill}{rgb}{0.964706,0.658824,0.000000}%
\pgfsetfillcolor{currentfill}%
\pgfsetlinewidth{1.003750pt}%
\definecolor{currentstroke}{rgb}{0.964706,0.658824,0.000000}%
\pgfsetstrokecolor{currentstroke}%
\pgfsetdash{}{0pt}%
\pgfsys@defobject{currentmarker}{\pgfqpoint{-0.041667in}{-0.041667in}}{\pgfqpoint{0.041667in}{0.041667in}}{%
\pgfpathmoveto{\pgfqpoint{0.000000in}{-0.041667in}}%
\pgfpathcurveto{\pgfqpoint{0.011050in}{-0.041667in}}{\pgfqpoint{0.021649in}{-0.037276in}}{\pgfqpoint{0.029463in}{-0.029463in}}%
\pgfpathcurveto{\pgfqpoint{0.037276in}{-0.021649in}}{\pgfqpoint{0.041667in}{-0.011050in}}{\pgfqpoint{0.041667in}{0.000000in}}%
\pgfpathcurveto{\pgfqpoint{0.041667in}{0.011050in}}{\pgfqpoint{0.037276in}{0.021649in}}{\pgfqpoint{0.029463in}{0.029463in}}%
\pgfpathcurveto{\pgfqpoint{0.021649in}{0.037276in}}{\pgfqpoint{0.011050in}{0.041667in}}{\pgfqpoint{0.000000in}{0.041667in}}%
\pgfpathcurveto{\pgfqpoint{-0.011050in}{0.041667in}}{\pgfqpoint{-0.021649in}{0.037276in}}{\pgfqpoint{-0.029463in}{0.029463in}}%
\pgfpathcurveto{\pgfqpoint{-0.037276in}{0.021649in}}{\pgfqpoint{-0.041667in}{0.011050in}}{\pgfqpoint{-0.041667in}{0.000000in}}%
\pgfpathcurveto{\pgfqpoint{-0.041667in}{-0.011050in}}{\pgfqpoint{-0.037276in}{-0.021649in}}{\pgfqpoint{-0.029463in}{-0.029463in}}%
\pgfpathcurveto{\pgfqpoint{-0.021649in}{-0.037276in}}{\pgfqpoint{-0.011050in}{-0.041667in}}{\pgfqpoint{0.000000in}{-0.041667in}}%
\pgfpathclose%
\pgfusepath{stroke,fill}%
}%
\begin{pgfscope}%
\pgfsys@transformshift{0.594770in}{0.424564in}%
\pgfsys@useobject{currentmarker}{}%
\end{pgfscope}%
\begin{pgfscope}%
\pgfsys@transformshift{1.013311in}{0.457647in}%
\pgfsys@useobject{currentmarker}{}%
\end{pgfscope}%
\begin{pgfscope}%
\pgfsys@transformshift{1.431853in}{0.512785in}%
\pgfsys@useobject{currentmarker}{}%
\end{pgfscope}%
\begin{pgfscope}%
\pgfsys@transformshift{1.850394in}{0.589978in}%
\pgfsys@useobject{currentmarker}{}%
\end{pgfscope}%
\begin{pgfscope}%
\pgfsys@transformshift{2.268936in}{0.689227in}%
\pgfsys@useobject{currentmarker}{}%
\end{pgfscope}%
\begin{pgfscope}%
\pgfsys@transformshift{2.687478in}{0.810531in}%
\pgfsys@useobject{currentmarker}{}%
\end{pgfscope}%
\begin{pgfscope}%
\pgfsys@transformshift{3.106019in}{0.953890in}%
\pgfsys@useobject{currentmarker}{}%
\end{pgfscope}%
\begin{pgfscope}%
\pgfsys@transformshift{3.524561in}{1.119305in}%
\pgfsys@useobject{currentmarker}{}%
\end{pgfscope}%
\begin{pgfscope}%
\pgfsys@transformshift{3.943102in}{1.306775in}%
\pgfsys@useobject{currentmarker}{}%
\end{pgfscope}%
\begin{pgfscope}%
\pgfsys@transformshift{4.361644in}{1.516300in}%
\pgfsys@useobject{currentmarker}{}%
\end{pgfscope}%
\end{pgfscope}%
\begin{pgfscope}%
\pgfpathrectangle{\pgfqpoint{0.594770in}{0.413536in}}{\pgfqpoint{3.766874in}{2.205527in}}%
\pgfusepath{clip}%
\pgfsetrectcap%
\pgfsetroundjoin%
\pgfsetlinewidth{1.505625pt}%
\definecolor{currentstroke}{rgb}{0.341176,0.670588,0.152941}%
\pgfsetstrokecolor{currentstroke}%
\pgfsetdash{}{0pt}%
\pgfpathmoveto{\pgfqpoint{0.594770in}{0.430078in}}%
\pgfpathlineto{\pgfqpoint{1.013311in}{0.545868in}}%
\pgfpathlineto{\pgfqpoint{1.431853in}{0.860156in}}%
\pgfpathlineto{\pgfqpoint{1.850394in}{1.472189in}}%
\pgfpathlineto{\pgfqpoint{2.268936in}{2.481218in}}%
\pgfpathlineto{\pgfqpoint{2.310044in}{2.629063in}}%
\pgfusepath{stroke}%
\end{pgfscope}%
\begin{pgfscope}%
\pgfpathrectangle{\pgfqpoint{0.594770in}{0.413536in}}{\pgfqpoint{3.766874in}{2.205527in}}%
\pgfusepath{clip}%
\pgfsetbuttcap%
\pgfsetroundjoin%
\definecolor{currentfill}{rgb}{0.341176,0.670588,0.152941}%
\pgfsetfillcolor{currentfill}%
\pgfsetlinewidth{1.003750pt}%
\definecolor{currentstroke}{rgb}{0.341176,0.670588,0.152941}%
\pgfsetstrokecolor{currentstroke}%
\pgfsetdash{}{0pt}%
\pgfsys@defobject{currentmarker}{\pgfqpoint{-0.041667in}{-0.041667in}}{\pgfqpoint{0.041667in}{0.041667in}}{%
\pgfpathmoveto{\pgfqpoint{0.000000in}{-0.041667in}}%
\pgfpathcurveto{\pgfqpoint{0.011050in}{-0.041667in}}{\pgfqpoint{0.021649in}{-0.037276in}}{\pgfqpoint{0.029463in}{-0.029463in}}%
\pgfpathcurveto{\pgfqpoint{0.037276in}{-0.021649in}}{\pgfqpoint{0.041667in}{-0.011050in}}{\pgfqpoint{0.041667in}{0.000000in}}%
\pgfpathcurveto{\pgfqpoint{0.041667in}{0.011050in}}{\pgfqpoint{0.037276in}{0.021649in}}{\pgfqpoint{0.029463in}{0.029463in}}%
\pgfpathcurveto{\pgfqpoint{0.021649in}{0.037276in}}{\pgfqpoint{0.011050in}{0.041667in}}{\pgfqpoint{0.000000in}{0.041667in}}%
\pgfpathcurveto{\pgfqpoint{-0.011050in}{0.041667in}}{\pgfqpoint{-0.021649in}{0.037276in}}{\pgfqpoint{-0.029463in}{0.029463in}}%
\pgfpathcurveto{\pgfqpoint{-0.037276in}{0.021649in}}{\pgfqpoint{-0.041667in}{0.011050in}}{\pgfqpoint{-0.041667in}{0.000000in}}%
\pgfpathcurveto{\pgfqpoint{-0.041667in}{-0.011050in}}{\pgfqpoint{-0.037276in}{-0.021649in}}{\pgfqpoint{-0.029463in}{-0.029463in}}%
\pgfpathcurveto{\pgfqpoint{-0.021649in}{-0.037276in}}{\pgfqpoint{-0.011050in}{-0.041667in}}{\pgfqpoint{0.000000in}{-0.041667in}}%
\pgfpathclose%
\pgfusepath{stroke,fill}%
}%
\begin{pgfscope}%
\pgfsys@transformshift{0.594770in}{0.430078in}%
\pgfsys@useobject{currentmarker}{}%
\end{pgfscope}%
\begin{pgfscope}%
\pgfsys@transformshift{1.013311in}{0.545868in}%
\pgfsys@useobject{currentmarker}{}%
\end{pgfscope}%
\begin{pgfscope}%
\pgfsys@transformshift{1.431853in}{0.860156in}%
\pgfsys@useobject{currentmarker}{}%
\end{pgfscope}%
\begin{pgfscope}%
\pgfsys@transformshift{1.850394in}{1.472189in}%
\pgfsys@useobject{currentmarker}{}%
\end{pgfscope}%
\begin{pgfscope}%
\pgfsys@transformshift{2.268936in}{2.481218in}%
\pgfsys@useobject{currentmarker}{}%
\end{pgfscope}%
\begin{pgfscope}%
\pgfsys@transformshift{2.687478in}{3.986490in}%
\pgfsys@useobject{currentmarker}{}%
\end{pgfscope}%
\begin{pgfscope}%
\pgfsys@transformshift{3.106019in}{6.087255in}%
\pgfsys@useobject{currentmarker}{}%
\end{pgfscope}%
\begin{pgfscope}%
\pgfsys@transformshift{3.524561in}{8.882760in}%
\pgfsys@useobject{currentmarker}{}%
\end{pgfscope}%
\begin{pgfscope}%
\pgfsys@transformshift{3.943102in}{12.472255in}%
\pgfsys@useobject{currentmarker}{}%
\end{pgfscope}%
\begin{pgfscope}%
\pgfsys@transformshift{4.361644in}{16.954989in}%
\pgfsys@useobject{currentmarker}{}%
\end{pgfscope}%
\end{pgfscope}%
\begin{pgfscope}%
\pgfpathrectangle{\pgfqpoint{0.594770in}{0.413536in}}{\pgfqpoint{3.766874in}{2.205527in}}%
\pgfusepath{clip}%
\pgfsetrectcap%
\pgfsetroundjoin%
\pgfsetlinewidth{1.505625pt}%
\definecolor{currentstroke}{rgb}{0.631373,0.062745,0.207843}%
\pgfsetstrokecolor{currentstroke}%
\pgfsetdash{}{0pt}%
\pgfpathmoveto{\pgfqpoint{0.594770in}{0.435592in}}%
\pgfpathlineto{\pgfqpoint{1.013311in}{0.766421in}}%
\pgfpathlineto{\pgfqpoint{1.431853in}{2.200013in}}%
\pgfpathlineto{\pgfqpoint{1.478379in}{2.629063in}}%
\pgfusepath{stroke}%
\end{pgfscope}%
\begin{pgfscope}%
\pgfpathrectangle{\pgfqpoint{0.594770in}{0.413536in}}{\pgfqpoint{3.766874in}{2.205527in}}%
\pgfusepath{clip}%
\pgfsetbuttcap%
\pgfsetroundjoin%
\definecolor{currentfill}{rgb}{0.631373,0.062745,0.207843}%
\pgfsetfillcolor{currentfill}%
\pgfsetlinewidth{1.003750pt}%
\definecolor{currentstroke}{rgb}{0.631373,0.062745,0.207843}%
\pgfsetstrokecolor{currentstroke}%
\pgfsetdash{}{0pt}%
\pgfsys@defobject{currentmarker}{\pgfqpoint{-0.041667in}{-0.041667in}}{\pgfqpoint{0.041667in}{0.041667in}}{%
\pgfpathmoveto{\pgfqpoint{0.000000in}{-0.041667in}}%
\pgfpathcurveto{\pgfqpoint{0.011050in}{-0.041667in}}{\pgfqpoint{0.021649in}{-0.037276in}}{\pgfqpoint{0.029463in}{-0.029463in}}%
\pgfpathcurveto{\pgfqpoint{0.037276in}{-0.021649in}}{\pgfqpoint{0.041667in}{-0.011050in}}{\pgfqpoint{0.041667in}{0.000000in}}%
\pgfpathcurveto{\pgfqpoint{0.041667in}{0.011050in}}{\pgfqpoint{0.037276in}{0.021649in}}{\pgfqpoint{0.029463in}{0.029463in}}%
\pgfpathcurveto{\pgfqpoint{0.021649in}{0.037276in}}{\pgfqpoint{0.011050in}{0.041667in}}{\pgfqpoint{0.000000in}{0.041667in}}%
\pgfpathcurveto{\pgfqpoint{-0.011050in}{0.041667in}}{\pgfqpoint{-0.021649in}{0.037276in}}{\pgfqpoint{-0.029463in}{0.029463in}}%
\pgfpathcurveto{\pgfqpoint{-0.037276in}{0.021649in}}{\pgfqpoint{-0.041667in}{0.011050in}}{\pgfqpoint{-0.041667in}{0.000000in}}%
\pgfpathcurveto{\pgfqpoint{-0.041667in}{-0.011050in}}{\pgfqpoint{-0.037276in}{-0.021649in}}{\pgfqpoint{-0.029463in}{-0.029463in}}%
\pgfpathcurveto{\pgfqpoint{-0.021649in}{-0.037276in}}{\pgfqpoint{-0.011050in}{-0.041667in}}{\pgfqpoint{0.000000in}{-0.041667in}}%
\pgfpathclose%
\pgfusepath{stroke,fill}%
}%
\begin{pgfscope}%
\pgfsys@transformshift{0.594770in}{0.435592in}%
\pgfsys@useobject{currentmarker}{}%
\end{pgfscope}%
\begin{pgfscope}%
\pgfsys@transformshift{1.013311in}{0.766421in}%
\pgfsys@useobject{currentmarker}{}%
\end{pgfscope}%
\begin{pgfscope}%
\pgfsys@transformshift{1.431853in}{2.200013in}%
\pgfsys@useobject{currentmarker}{}%
\end{pgfscope}%
\begin{pgfscope}%
\pgfsys@transformshift{1.850394in}{6.059685in}%
\pgfsys@useobject{currentmarker}{}%
\end{pgfscope}%
\begin{pgfscope}%
\pgfsys@transformshift{2.268936in}{14.198080in}%
\pgfsys@useobject{currentmarker}{}%
\end{pgfscope}%
\begin{pgfscope}%
\pgfsys@transformshift{2.687478in}{28.997167in}%
\pgfsys@useobject{currentmarker}{}%
\end{pgfscope}%
\begin{pgfscope}%
\pgfsys@transformshift{3.106019in}{53.368240in}%
\pgfsys@useobject{currentmarker}{}%
\end{pgfscope}%
\begin{pgfscope}%
\pgfsys@transformshift{3.524561in}{90.751923in}%
\pgfsys@useobject{currentmarker}{}%
\end{pgfscope}%
\begin{pgfscope}%
\pgfsys@transformshift{3.943102in}{145.118164in}%
\pgfsys@useobject{currentmarker}{}%
\end{pgfscope}%
\begin{pgfscope}%
\pgfsys@transformshift{4.361644in}{220.966238in}%
\pgfsys@useobject{currentmarker}{}%
\end{pgfscope}%
\end{pgfscope}%
\begin{pgfscope}%
\pgfsetrectcap%
\pgfsetmiterjoin%
\pgfsetlinewidth{0.803000pt}%
\definecolor{currentstroke}{rgb}{0.000000,0.000000,0.000000}%
\pgfsetstrokecolor{currentstroke}%
\pgfsetdash{}{0pt}%
\pgfpathmoveto{\pgfqpoint{0.594770in}{0.413536in}}%
\pgfpathlineto{\pgfqpoint{0.594770in}{2.619063in}}%
\pgfusepath{stroke}%
\end{pgfscope}%
\begin{pgfscope}%
\pgfsetrectcap%
\pgfsetmiterjoin%
\pgfsetlinewidth{0.803000pt}%
\definecolor{currentstroke}{rgb}{0.000000,0.000000,0.000000}%
\pgfsetstrokecolor{currentstroke}%
\pgfsetdash{}{0pt}%
\pgfpathmoveto{\pgfqpoint{4.361644in}{0.413536in}}%
\pgfpathlineto{\pgfqpoint{4.361644in}{2.619063in}}%
\pgfusepath{stroke}%
\end{pgfscope}%
\begin{pgfscope}%
\pgfsetrectcap%
\pgfsetmiterjoin%
\pgfsetlinewidth{0.803000pt}%
\definecolor{currentstroke}{rgb}{0.000000,0.000000,0.000000}%
\pgfsetstrokecolor{currentstroke}%
\pgfsetdash{}{0pt}%
\pgfpathmoveto{\pgfqpoint{0.594770in}{0.413536in}}%
\pgfpathlineto{\pgfqpoint{4.361644in}{0.413536in}}%
\pgfusepath{stroke}%
\end{pgfscope}%
\begin{pgfscope}%
\pgfsetrectcap%
\pgfsetmiterjoin%
\pgfsetlinewidth{0.803000pt}%
\definecolor{currentstroke}{rgb}{0.000000,0.000000,0.000000}%
\pgfsetstrokecolor{currentstroke}%
\pgfsetdash{}{0pt}%
\pgfpathmoveto{\pgfqpoint{0.594770in}{2.619063in}}%
\pgfpathlineto{\pgfqpoint{4.361644in}{2.619063in}}%
\pgfusepath{stroke}%
\end{pgfscope}%
\begin{pgfscope}%
\definecolor{textcolor}{rgb}{0.000000,0.000000,0.000000}%
\pgfsetstrokecolor{textcolor}%
\pgfsetfillcolor{textcolor}%
\pgftext[x=3.733831in,y=0.689227in,,base]{\color{textcolor}\rmfamily\fontsize{10.000000}{12.000000}\selectfont Minimax Tilting \cite{Botev2016}}%
\end{pgfscope}%
\begin{pgfscope}%
\definecolor{textcolor}{rgb}{0.000000,0.000000,0.000000}%
\pgfsetstrokecolor{textcolor}%
\pgfsetfillcolor{textcolor}%
\pgftext[x=2.896748in,y=1.378454in,,base]{\color{textcolor}\rmfamily\fontsize{10.000000}{12.000000}\selectfont Gibbs Sampling (Alg. \ref{algo:gibbssampling})}%
\end{pgfscope}%
\begin{pgfscope}%
\definecolor{textcolor}{rgb}{0.000000,0.000000,0.000000}%
\pgfsetstrokecolor{textcolor}%
\pgfsetfillcolor{textcolor}%
\pgftext[x=2.478207in,y=2.205527in,,base]{\color{textcolor}\rmfamily\fontsize{10.000000}{12.000000}\selectfont Computationally too expensive}%
\end{pgfscope}%
\begin{pgfscope}%
\pgfsetrectcap%
\pgfsetroundjoin%
\pgfsetlinewidth{1.505625pt}%
\definecolor{currentstroke}{rgb}{0.000000,0.329412,0.623529}%
\pgfsetstrokecolor{currentstroke}%
\pgfsetdash{}{0pt}%
\pgfpathmoveto{\pgfqpoint{3.549618in}{2.511618in}}%
\pgfpathlineto{\pgfqpoint{3.827395in}{2.511618in}}%
\pgfusepath{stroke}%
\end{pgfscope}%
\begin{pgfscope}%
\pgfsetbuttcap%
\pgfsetroundjoin%
\definecolor{currentfill}{rgb}{0.000000,0.329412,0.623529}%
\pgfsetfillcolor{currentfill}%
\pgfsetlinewidth{1.003750pt}%
\definecolor{currentstroke}{rgb}{0.000000,0.329412,0.623529}%
\pgfsetstrokecolor{currentstroke}%
\pgfsetdash{}{0pt}%
\pgfsys@defobject{currentmarker}{\pgfqpoint{-0.041667in}{-0.041667in}}{\pgfqpoint{0.041667in}{0.041667in}}{%
\pgfpathmoveto{\pgfqpoint{0.000000in}{-0.041667in}}%
\pgfpathcurveto{\pgfqpoint{0.011050in}{-0.041667in}}{\pgfqpoint{0.021649in}{-0.037276in}}{\pgfqpoint{0.029463in}{-0.029463in}}%
\pgfpathcurveto{\pgfqpoint{0.037276in}{-0.021649in}}{\pgfqpoint{0.041667in}{-0.011050in}}{\pgfqpoint{0.041667in}{0.000000in}}%
\pgfpathcurveto{\pgfqpoint{0.041667in}{0.011050in}}{\pgfqpoint{0.037276in}{0.021649in}}{\pgfqpoint{0.029463in}{0.029463in}}%
\pgfpathcurveto{\pgfqpoint{0.021649in}{0.037276in}}{\pgfqpoint{0.011050in}{0.041667in}}{\pgfqpoint{0.000000in}{0.041667in}}%
\pgfpathcurveto{\pgfqpoint{-0.011050in}{0.041667in}}{\pgfqpoint{-0.021649in}{0.037276in}}{\pgfqpoint{-0.029463in}{0.029463in}}%
\pgfpathcurveto{\pgfqpoint{-0.037276in}{0.021649in}}{\pgfqpoint{-0.041667in}{0.011050in}}{\pgfqpoint{-0.041667in}{0.000000in}}%
\pgfpathcurveto{\pgfqpoint{-0.041667in}{-0.011050in}}{\pgfqpoint{-0.037276in}{-0.021649in}}{\pgfqpoint{-0.029463in}{-0.029463in}}%
\pgfpathcurveto{\pgfqpoint{-0.021649in}{-0.037276in}}{\pgfqpoint{-0.011050in}{-0.041667in}}{\pgfqpoint{0.000000in}{-0.041667in}}%
\pgfpathclose%
\pgfusepath{stroke,fill}%
}%
\begin{pgfscope}%
\pgfsys@transformshift{3.688507in}{2.511618in}%
\pgfsys@useobject{currentmarker}{}%
\end{pgfscope}%
\end{pgfscope}%
\begin{pgfscope}%
\definecolor{textcolor}{rgb}{0.000000,0.000000,0.000000}%
\pgfsetstrokecolor{textcolor}%
\pgfsetfillcolor{textcolor}%
\pgftext[x=3.938507in,y=2.463007in,left,base]{\color{textcolor}\rmfamily\fontsize{10.000000}{12.000000}\selectfont \(\displaystyle D=1\)}%
\end{pgfscope}%
\begin{pgfscope}%
\pgfsetrectcap%
\pgfsetroundjoin%
\pgfsetlinewidth{1.505625pt}%
\definecolor{currentstroke}{rgb}{0.964706,0.658824,0.000000}%
\pgfsetstrokecolor{currentstroke}%
\pgfsetdash{}{0pt}%
\pgfpathmoveto{\pgfqpoint{3.549618in}{2.318007in}}%
\pgfpathlineto{\pgfqpoint{3.827395in}{2.318007in}}%
\pgfusepath{stroke}%
\end{pgfscope}%
\begin{pgfscope}%
\pgfsetbuttcap%
\pgfsetroundjoin%
\definecolor{currentfill}{rgb}{0.964706,0.658824,0.000000}%
\pgfsetfillcolor{currentfill}%
\pgfsetlinewidth{1.003750pt}%
\definecolor{currentstroke}{rgb}{0.964706,0.658824,0.000000}%
\pgfsetstrokecolor{currentstroke}%
\pgfsetdash{}{0pt}%
\pgfsys@defobject{currentmarker}{\pgfqpoint{-0.041667in}{-0.041667in}}{\pgfqpoint{0.041667in}{0.041667in}}{%
\pgfpathmoveto{\pgfqpoint{0.000000in}{-0.041667in}}%
\pgfpathcurveto{\pgfqpoint{0.011050in}{-0.041667in}}{\pgfqpoint{0.021649in}{-0.037276in}}{\pgfqpoint{0.029463in}{-0.029463in}}%
\pgfpathcurveto{\pgfqpoint{0.037276in}{-0.021649in}}{\pgfqpoint{0.041667in}{-0.011050in}}{\pgfqpoint{0.041667in}{0.000000in}}%
\pgfpathcurveto{\pgfqpoint{0.041667in}{0.011050in}}{\pgfqpoint{0.037276in}{0.021649in}}{\pgfqpoint{0.029463in}{0.029463in}}%
\pgfpathcurveto{\pgfqpoint{0.021649in}{0.037276in}}{\pgfqpoint{0.011050in}{0.041667in}}{\pgfqpoint{0.000000in}{0.041667in}}%
\pgfpathcurveto{\pgfqpoint{-0.011050in}{0.041667in}}{\pgfqpoint{-0.021649in}{0.037276in}}{\pgfqpoint{-0.029463in}{0.029463in}}%
\pgfpathcurveto{\pgfqpoint{-0.037276in}{0.021649in}}{\pgfqpoint{-0.041667in}{0.011050in}}{\pgfqpoint{-0.041667in}{0.000000in}}%
\pgfpathcurveto{\pgfqpoint{-0.041667in}{-0.011050in}}{\pgfqpoint{-0.037276in}{-0.021649in}}{\pgfqpoint{-0.029463in}{-0.029463in}}%
\pgfpathcurveto{\pgfqpoint{-0.021649in}{-0.037276in}}{\pgfqpoint{-0.011050in}{-0.041667in}}{\pgfqpoint{0.000000in}{-0.041667in}}%
\pgfpathclose%
\pgfusepath{stroke,fill}%
}%
\begin{pgfscope}%
\pgfsys@transformshift{3.688507in}{2.318007in}%
\pgfsys@useobject{currentmarker}{}%
\end{pgfscope}%
\end{pgfscope}%
\begin{pgfscope}%
\definecolor{textcolor}{rgb}{0.000000,0.000000,0.000000}%
\pgfsetstrokecolor{textcolor}%
\pgfsetfillcolor{textcolor}%
\pgftext[x=3.938507in,y=2.269396in,left,base]{\color{textcolor}\rmfamily\fontsize{10.000000}{12.000000}\selectfont \(\displaystyle D=2\)}%
\end{pgfscope}%
\begin{pgfscope}%
\pgfsetrectcap%
\pgfsetroundjoin%
\pgfsetlinewidth{1.505625pt}%
\definecolor{currentstroke}{rgb}{0.341176,0.670588,0.152941}%
\pgfsetstrokecolor{currentstroke}%
\pgfsetdash{}{0pt}%
\pgfpathmoveto{\pgfqpoint{3.549618in}{2.124396in}}%
\pgfpathlineto{\pgfqpoint{3.827395in}{2.124396in}}%
\pgfusepath{stroke}%
\end{pgfscope}%
\begin{pgfscope}%
\pgfsetbuttcap%
\pgfsetroundjoin%
\definecolor{currentfill}{rgb}{0.341176,0.670588,0.152941}%
\pgfsetfillcolor{currentfill}%
\pgfsetlinewidth{1.003750pt}%
\definecolor{currentstroke}{rgb}{0.341176,0.670588,0.152941}%
\pgfsetstrokecolor{currentstroke}%
\pgfsetdash{}{0pt}%
\pgfsys@defobject{currentmarker}{\pgfqpoint{-0.041667in}{-0.041667in}}{\pgfqpoint{0.041667in}{0.041667in}}{%
\pgfpathmoveto{\pgfqpoint{0.000000in}{-0.041667in}}%
\pgfpathcurveto{\pgfqpoint{0.011050in}{-0.041667in}}{\pgfqpoint{0.021649in}{-0.037276in}}{\pgfqpoint{0.029463in}{-0.029463in}}%
\pgfpathcurveto{\pgfqpoint{0.037276in}{-0.021649in}}{\pgfqpoint{0.041667in}{-0.011050in}}{\pgfqpoint{0.041667in}{0.000000in}}%
\pgfpathcurveto{\pgfqpoint{0.041667in}{0.011050in}}{\pgfqpoint{0.037276in}{0.021649in}}{\pgfqpoint{0.029463in}{0.029463in}}%
\pgfpathcurveto{\pgfqpoint{0.021649in}{0.037276in}}{\pgfqpoint{0.011050in}{0.041667in}}{\pgfqpoint{0.000000in}{0.041667in}}%
\pgfpathcurveto{\pgfqpoint{-0.011050in}{0.041667in}}{\pgfqpoint{-0.021649in}{0.037276in}}{\pgfqpoint{-0.029463in}{0.029463in}}%
\pgfpathcurveto{\pgfqpoint{-0.037276in}{0.021649in}}{\pgfqpoint{-0.041667in}{0.011050in}}{\pgfqpoint{-0.041667in}{0.000000in}}%
\pgfpathcurveto{\pgfqpoint{-0.041667in}{-0.011050in}}{\pgfqpoint{-0.037276in}{-0.021649in}}{\pgfqpoint{-0.029463in}{-0.029463in}}%
\pgfpathcurveto{\pgfqpoint{-0.021649in}{-0.037276in}}{\pgfqpoint{-0.011050in}{-0.041667in}}{\pgfqpoint{0.000000in}{-0.041667in}}%
\pgfpathclose%
\pgfusepath{stroke,fill}%
}%
\begin{pgfscope}%
\pgfsys@transformshift{3.688507in}{2.124396in}%
\pgfsys@useobject{currentmarker}{}%
\end{pgfscope}%
\end{pgfscope}%
\begin{pgfscope}%
\definecolor{textcolor}{rgb}{0.000000,0.000000,0.000000}%
\pgfsetstrokecolor{textcolor}%
\pgfsetfillcolor{textcolor}%
\pgftext[x=3.938507in,y=2.075785in,left,base]{\color{textcolor}\rmfamily\fontsize{10.000000}{12.000000}\selectfont \(\displaystyle D=3\)}%
\end{pgfscope}%
\begin{pgfscope}%
\pgfsetrectcap%
\pgfsetroundjoin%
\pgfsetlinewidth{1.505625pt}%
\definecolor{currentstroke}{rgb}{0.631373,0.062745,0.207843}%
\pgfsetstrokecolor{currentstroke}%
\pgfsetdash{}{0pt}%
\pgfpathmoveto{\pgfqpoint{3.549618in}{1.930785in}}%
\pgfpathlineto{\pgfqpoint{3.827395in}{1.930785in}}%
\pgfusepath{stroke}%
\end{pgfscope}%
\begin{pgfscope}%
\pgfsetbuttcap%
\pgfsetroundjoin%
\definecolor{currentfill}{rgb}{0.631373,0.062745,0.207843}%
\pgfsetfillcolor{currentfill}%
\pgfsetlinewidth{1.003750pt}%
\definecolor{currentstroke}{rgb}{0.631373,0.062745,0.207843}%
\pgfsetstrokecolor{currentstroke}%
\pgfsetdash{}{0pt}%
\pgfsys@defobject{currentmarker}{\pgfqpoint{-0.041667in}{-0.041667in}}{\pgfqpoint{0.041667in}{0.041667in}}{%
\pgfpathmoveto{\pgfqpoint{0.000000in}{-0.041667in}}%
\pgfpathcurveto{\pgfqpoint{0.011050in}{-0.041667in}}{\pgfqpoint{0.021649in}{-0.037276in}}{\pgfqpoint{0.029463in}{-0.029463in}}%
\pgfpathcurveto{\pgfqpoint{0.037276in}{-0.021649in}}{\pgfqpoint{0.041667in}{-0.011050in}}{\pgfqpoint{0.041667in}{0.000000in}}%
\pgfpathcurveto{\pgfqpoint{0.041667in}{0.011050in}}{\pgfqpoint{0.037276in}{0.021649in}}{\pgfqpoint{0.029463in}{0.029463in}}%
\pgfpathcurveto{\pgfqpoint{0.021649in}{0.037276in}}{\pgfqpoint{0.011050in}{0.041667in}}{\pgfqpoint{0.000000in}{0.041667in}}%
\pgfpathcurveto{\pgfqpoint{-0.011050in}{0.041667in}}{\pgfqpoint{-0.021649in}{0.037276in}}{\pgfqpoint{-0.029463in}{0.029463in}}%
\pgfpathcurveto{\pgfqpoint{-0.037276in}{0.021649in}}{\pgfqpoint{-0.041667in}{0.011050in}}{\pgfqpoint{-0.041667in}{0.000000in}}%
\pgfpathcurveto{\pgfqpoint{-0.041667in}{-0.011050in}}{\pgfqpoint{-0.037276in}{-0.021649in}}{\pgfqpoint{-0.029463in}{-0.029463in}}%
\pgfpathcurveto{\pgfqpoint{-0.021649in}{-0.037276in}}{\pgfqpoint{-0.011050in}{-0.041667in}}{\pgfqpoint{0.000000in}{-0.041667in}}%
\pgfpathclose%
\pgfusepath{stroke,fill}%
}%
\begin{pgfscope}%
\pgfsys@transformshift{3.688507in}{1.930785in}%
\pgfsys@useobject{currentmarker}{}%
\end{pgfscope}%
\end{pgfscope}%
\begin{pgfscope}%
\definecolor{textcolor}{rgb}{0.000000,0.000000,0.000000}%
\pgfsetstrokecolor{textcolor}%
\pgfsetfillcolor{textcolor}%
\pgftext[x=3.938507in,y=1.882174in,left,base]{\color{textcolor}\rmfamily\fontsize{10.000000}{12.000000}\selectfont \(\displaystyle D=4\)}%
\end{pgfscope}%
\end{pgfpicture}%
\makeatother%
\endgroup%

    \caption[Dependency of the number of \glspl{vop} per dimension on the dimension $P$ of the truncated multivariate normal distribution.]{Dependency of the number of \glspl{vop} per dimension on the dimension $P$ of the truncated multivariate normal distribution \eqref{eq:truncated_mvn} for different spatial dimensions $D$. The displayed bounds are not fixed as sampling strongly depends on the covariance matrix in \eqref{eq:truncated_mvn}.}
   \label{fig:dims_vops}
\end{figure}
Depending on $P$, different sampling methods are suitable. Up to $P\approx 100$ the rejection sampling method by \textcite{Botev2016} showed to perform well. However, at higher dimensions $P$ the acceptance rate becomes too low and Gibbs sampling as proposed in Algorithm~\ref{algo:gibbssampling} showed to perform well. While Gibbs sampling is not limited by an acceptance rate because every sample is accepted, sampling at in dimensions $P>250$ showed to be also not feasible as the computational effort is too high since the inner loop of Algorithm~\ref{algo:gibbssampling} scales with $P$. However, these bounds on the algorithms are not fixed as sampling strongly depends on the covariance matrix of \eqref{eq:truncated_mvn}. 

There have been proposals for choosing the VOPs optimally. These methods try to find the locations within the feasible set $\mathcal{X}$ at which the probability of satisfying \eqref{eq:linear_inequality_constraints_complete} is the lowest. These locations with low probability are then added to the set of \glspl{vop} \cite{Agrell_2019}\cite{Wang_2016}. However, this can still result in a large set of \glspl{vop} in higher dimensions. Instead, a different approach could be to choose the \glspl{vop} based on a sensor placement problem, thus choosing the locations which maximize the probability of satisfying \eqref{eq:linear_inequality_constraints_complete} for all $\mathbf{x} \in \mathcal{X}$. This would reduce the number of \glspl{vop} needed but also increase the computational complexity significantly. 

\subsection{Example: 1D Convexity Constrained Gaussian Process}

To show the concept of constraining a \gls{gp} and the influence of choosing the \glspl{vop}, a short one dimensional example is presented. Considering the prior knowledge in Assumption \ref{ass:prior_knowledge_convex} of the objective function staying convex through time, the constrained posterior distribution can be constructed using a linear operator $\mathcal{L} = \frac{\pdiff^2}{\pdiff x_i^2}$ and defining the constraint functions $a(\mathbf{x}) = 0$ and $b(\mathbf{x}) = +\infty$ to enforce convexity. Furthermore, nine \glspl{vop} are defined in an equidistant grid $\mathbf{X}_v = [-4, \dots, 4] \in \R^9$. To construct the posterior, the Gram matrices with the applied linear operator and the mean with the linear operator have to be constructed. As the mean function is assumed to be constant applying the linear operator results in $\mathcal{L}\mu_{\mathbf{X}_v} = \mathbf{0}$. The Gram matrices are constructed as
\begin{align}
    K_{\mathbf{X},\mathbf{X}_v}\mathcal{L}^T &= \left[(K^{1,0}_{\MatBold{X}_{v},\mathbf{X}})^T, \dots, (K^{D,0}_{\MatBold{X}_{v},\mathbf{X}})^T \right]\label{eq:constrained_gram1}\\
    K_{\mathbf{X}_*,\mathbf{X}_v}\mathcal{L}^T &= \left[(K^{1,0}_{\MatBold{X}_{v},\mathbf{X}_*})^T, \dots, (K^{D,0}_{\MatBold{X}_{v},\mathbf{X}_*})^T \right] \label{eq:constrained_gram2}\\
    \mathcal{L}K_{\mathbf{X}_v,\mathbf{X}_v}\mathcal{L}^T &= \left[\begin{array}{ccc}
    K^{1,1}_{\MatBold{X}_{v},\MatBold{X}_{v}} & \cdots & K^{1,D}_{\MatBold{X}_{v},\MatBold{X}_{v}} \\
    \vdots & \ddots & \vdots\\
     K^{D,1}_{\MatBold{X}_{v},\MatBold{X}_{v}} & \cdots & K^{D,D}_{\MatBold{X}_{v},\MatBold{X}_{v}}
    \end{array}\right] \label{eq:constrained_gram3}
\end{align}
with the notation of $K^{i,0}_{\mathbf{x},\mathbf{x}'} = \frac{\pdiff^2}{\pdiff x_i^2}K(\mathbf{x},\mathbf{x}')$ and $K^{i,j}_{\mathbf{x},\mathbf{x}'} = \frac{\pdiff^4}{\pdiff x_i^2 {x'}_j^2}K(\mathbf{x},\mathbf{x}')$ as partial derivatives, which are listed in Appendix \ref{apx:derivatives} for the \gls{se} kernel.
\begin{figure}[h]
    \centering
    %% Creator: Matplotlib, PGF backend
%%
%% To include the figure in your LaTeX document, write
%%   \input{<filename>.pgf}
%%
%% Make sure the required packages are loaded in your preamble
%%   \usepackage{pgf}
%%
%% Figures using additional raster images can only be included by \input if
%% they are in the same directory as the main LaTeX file. For loading figures
%% from other directories you can use the `import` package
%%   \usepackage{import}
%%
%% and then include the figures with
%%   \import{<path to file>}{<filename>.pgf}
%%
%% Matplotlib used the following preamble
%%   \usepackage{fontspec}
%%
\begingroup%
\makeatletter%
\begin{pgfpicture}%
\pgfpathrectangle{\pgfpointorigin}{\pgfqpoint{5.507126in}{2.212334in}}%
\pgfusepath{use as bounding box, clip}%
\begin{pgfscope}%
\pgfsetbuttcap%
\pgfsetmiterjoin%
\definecolor{currentfill}{rgb}{1.000000,1.000000,1.000000}%
\pgfsetfillcolor{currentfill}%
\pgfsetlinewidth{0.000000pt}%
\definecolor{currentstroke}{rgb}{1.000000,1.000000,1.000000}%
\pgfsetstrokecolor{currentstroke}%
\pgfsetdash{}{0pt}%
\pgfpathmoveto{\pgfqpoint{0.000000in}{0.000000in}}%
\pgfpathlineto{\pgfqpoint{5.507126in}{0.000000in}}%
\pgfpathlineto{\pgfqpoint{5.507126in}{2.212334in}}%
\pgfpathlineto{\pgfqpoint{0.000000in}{2.212334in}}%
\pgfpathclose%
\pgfusepath{fill}%
\end{pgfscope}%
\begin{pgfscope}%
\pgfsetbuttcap%
\pgfsetmiterjoin%
\definecolor{currentfill}{rgb}{1.000000,1.000000,1.000000}%
\pgfsetfillcolor{currentfill}%
\pgfsetlinewidth{0.000000pt}%
\definecolor{currentstroke}{rgb}{0.000000,0.000000,0.000000}%
\pgfsetstrokecolor{currentstroke}%
\pgfsetstrokeopacity{0.000000}%
\pgfsetdash{}{0pt}%
\pgfpathmoveto{\pgfqpoint{0.550713in}{0.398220in}}%
\pgfpathlineto{\pgfqpoint{2.914747in}{0.398220in}}%
\pgfpathlineto{\pgfqpoint{2.914747in}{2.101717in}}%
\pgfpathlineto{\pgfqpoint{0.550713in}{2.101717in}}%
\pgfpathclose%
\pgfusepath{fill}%
\end{pgfscope}%
\begin{pgfscope}%
\pgfpathrectangle{\pgfqpoint{0.550713in}{0.398220in}}{\pgfqpoint{2.364035in}{1.703497in}}%
\pgfusepath{clip}%
\pgfsetbuttcap%
\pgfsetroundjoin%
\definecolor{currentfill}{rgb}{0.803922,0.545098,0.529412}%
\pgfsetfillcolor{currentfill}%
\pgfsetfillopacity{0.700000}%
\pgfsetlinewidth{0.000000pt}%
\definecolor{currentstroke}{rgb}{0.803922,0.545098,0.529412}%
\pgfsetstrokecolor{currentstroke}%
\pgfsetstrokeopacity{0.700000}%
\pgfsetdash{}{0pt}%
\pgfsys@defobject{currentmarker}{\pgfqpoint{0.550713in}{0.909269in}}{\pgfqpoint{2.914747in}{1.590668in}}{%
\pgfpathmoveto{\pgfqpoint{0.550713in}{1.590668in}}%
\pgfpathlineto{\pgfqpoint{0.550713in}{0.909269in}}%
\pgfpathlineto{\pgfqpoint{0.574592in}{0.909269in}}%
\pgfpathlineto{\pgfqpoint{0.598471in}{0.909269in}}%
\pgfpathlineto{\pgfqpoint{0.622350in}{0.909269in}}%
\pgfpathlineto{\pgfqpoint{0.646229in}{0.909269in}}%
\pgfpathlineto{\pgfqpoint{0.670108in}{0.909269in}}%
\pgfpathlineto{\pgfqpoint{0.693987in}{0.909269in}}%
\pgfpathlineto{\pgfqpoint{0.717867in}{0.909269in}}%
\pgfpathlineto{\pgfqpoint{0.741746in}{0.909269in}}%
\pgfpathlineto{\pgfqpoint{0.765625in}{0.909269in}}%
\pgfpathlineto{\pgfqpoint{0.789504in}{0.909269in}}%
\pgfpathlineto{\pgfqpoint{0.813383in}{0.909269in}}%
\pgfpathlineto{\pgfqpoint{0.837262in}{0.909269in}}%
\pgfpathlineto{\pgfqpoint{0.861141in}{0.909269in}}%
\pgfpathlineto{\pgfqpoint{0.885021in}{0.909269in}}%
\pgfpathlineto{\pgfqpoint{0.908900in}{0.909269in}}%
\pgfpathlineto{\pgfqpoint{0.932779in}{0.909269in}}%
\pgfpathlineto{\pgfqpoint{0.956658in}{0.909269in}}%
\pgfpathlineto{\pgfqpoint{0.980537in}{0.909269in}}%
\pgfpathlineto{\pgfqpoint{1.004416in}{0.909269in}}%
\pgfpathlineto{\pgfqpoint{1.028295in}{0.909269in}}%
\pgfpathlineto{\pgfqpoint{1.052174in}{0.909269in}}%
\pgfpathlineto{\pgfqpoint{1.076054in}{0.909269in}}%
\pgfpathlineto{\pgfqpoint{1.099933in}{0.909269in}}%
\pgfpathlineto{\pgfqpoint{1.123812in}{0.909269in}}%
\pgfpathlineto{\pgfqpoint{1.147691in}{0.909269in}}%
\pgfpathlineto{\pgfqpoint{1.171570in}{0.909269in}}%
\pgfpathlineto{\pgfqpoint{1.195449in}{0.909269in}}%
\pgfpathlineto{\pgfqpoint{1.219328in}{0.909269in}}%
\pgfpathlineto{\pgfqpoint{1.243208in}{0.909269in}}%
\pgfpathlineto{\pgfqpoint{1.267087in}{0.909269in}}%
\pgfpathlineto{\pgfqpoint{1.290966in}{0.909269in}}%
\pgfpathlineto{\pgfqpoint{1.314845in}{0.909269in}}%
\pgfpathlineto{\pgfqpoint{1.338724in}{0.909269in}}%
\pgfpathlineto{\pgfqpoint{1.362603in}{0.909269in}}%
\pgfpathlineto{\pgfqpoint{1.386482in}{0.909269in}}%
\pgfpathlineto{\pgfqpoint{1.410362in}{0.909269in}}%
\pgfpathlineto{\pgfqpoint{1.434241in}{0.909269in}}%
\pgfpathlineto{\pgfqpoint{1.458120in}{0.909269in}}%
\pgfpathlineto{\pgfqpoint{1.481999in}{0.909269in}}%
\pgfpathlineto{\pgfqpoint{1.505878in}{0.909269in}}%
\pgfpathlineto{\pgfqpoint{1.529757in}{0.909269in}}%
\pgfpathlineto{\pgfqpoint{1.553636in}{0.909269in}}%
\pgfpathlineto{\pgfqpoint{1.577516in}{0.909269in}}%
\pgfpathlineto{\pgfqpoint{1.601395in}{0.909269in}}%
\pgfpathlineto{\pgfqpoint{1.625274in}{0.909269in}}%
\pgfpathlineto{\pgfqpoint{1.649153in}{0.909269in}}%
\pgfpathlineto{\pgfqpoint{1.673032in}{0.909269in}}%
\pgfpathlineto{\pgfqpoint{1.696911in}{0.909269in}}%
\pgfpathlineto{\pgfqpoint{1.720790in}{0.909269in}}%
\pgfpathlineto{\pgfqpoint{1.744669in}{0.909269in}}%
\pgfpathlineto{\pgfqpoint{1.768549in}{0.909269in}}%
\pgfpathlineto{\pgfqpoint{1.792428in}{0.909269in}}%
\pgfpathlineto{\pgfqpoint{1.816307in}{0.909269in}}%
\pgfpathlineto{\pgfqpoint{1.840186in}{0.909269in}}%
\pgfpathlineto{\pgfqpoint{1.864065in}{0.909269in}}%
\pgfpathlineto{\pgfqpoint{1.887944in}{0.909269in}}%
\pgfpathlineto{\pgfqpoint{1.911823in}{0.909269in}}%
\pgfpathlineto{\pgfqpoint{1.935703in}{0.909269in}}%
\pgfpathlineto{\pgfqpoint{1.959582in}{0.909269in}}%
\pgfpathlineto{\pgfqpoint{1.983461in}{0.909269in}}%
\pgfpathlineto{\pgfqpoint{2.007340in}{0.909269in}}%
\pgfpathlineto{\pgfqpoint{2.031219in}{0.909269in}}%
\pgfpathlineto{\pgfqpoint{2.055098in}{0.909269in}}%
\pgfpathlineto{\pgfqpoint{2.078977in}{0.909269in}}%
\pgfpathlineto{\pgfqpoint{2.102857in}{0.909269in}}%
\pgfpathlineto{\pgfqpoint{2.126736in}{0.909269in}}%
\pgfpathlineto{\pgfqpoint{2.150615in}{0.909269in}}%
\pgfpathlineto{\pgfqpoint{2.174494in}{0.909269in}}%
\pgfpathlineto{\pgfqpoint{2.198373in}{0.909269in}}%
\pgfpathlineto{\pgfqpoint{2.222252in}{0.909269in}}%
\pgfpathlineto{\pgfqpoint{2.246131in}{0.909269in}}%
\pgfpathlineto{\pgfqpoint{2.270010in}{0.909269in}}%
\pgfpathlineto{\pgfqpoint{2.293890in}{0.909269in}}%
\pgfpathlineto{\pgfqpoint{2.317769in}{0.909269in}}%
\pgfpathlineto{\pgfqpoint{2.341648in}{0.909269in}}%
\pgfpathlineto{\pgfqpoint{2.365527in}{0.909269in}}%
\pgfpathlineto{\pgfqpoint{2.389406in}{0.909269in}}%
\pgfpathlineto{\pgfqpoint{2.413285in}{0.909269in}}%
\pgfpathlineto{\pgfqpoint{2.437164in}{0.909269in}}%
\pgfpathlineto{\pgfqpoint{2.461044in}{0.909269in}}%
\pgfpathlineto{\pgfqpoint{2.484923in}{0.909269in}}%
\pgfpathlineto{\pgfqpoint{2.508802in}{0.909269in}}%
\pgfpathlineto{\pgfqpoint{2.532681in}{0.909269in}}%
\pgfpathlineto{\pgfqpoint{2.556560in}{0.909269in}}%
\pgfpathlineto{\pgfqpoint{2.580439in}{0.909269in}}%
\pgfpathlineto{\pgfqpoint{2.604318in}{0.909269in}}%
\pgfpathlineto{\pgfqpoint{2.628198in}{0.909269in}}%
\pgfpathlineto{\pgfqpoint{2.652077in}{0.909269in}}%
\pgfpathlineto{\pgfqpoint{2.675956in}{0.909269in}}%
\pgfpathlineto{\pgfqpoint{2.699835in}{0.909269in}}%
\pgfpathlineto{\pgfqpoint{2.723714in}{0.909269in}}%
\pgfpathlineto{\pgfqpoint{2.747593in}{0.909269in}}%
\pgfpathlineto{\pgfqpoint{2.771472in}{0.909269in}}%
\pgfpathlineto{\pgfqpoint{2.795352in}{0.909269in}}%
\pgfpathlineto{\pgfqpoint{2.819231in}{0.909269in}}%
\pgfpathlineto{\pgfqpoint{2.843110in}{0.909269in}}%
\pgfpathlineto{\pgfqpoint{2.866989in}{0.909269in}}%
\pgfpathlineto{\pgfqpoint{2.890868in}{0.909269in}}%
\pgfpathlineto{\pgfqpoint{2.914747in}{0.909269in}}%
\pgfpathlineto{\pgfqpoint{2.914747in}{1.590668in}}%
\pgfpathlineto{\pgfqpoint{2.914747in}{1.590668in}}%
\pgfpathlineto{\pgfqpoint{2.890868in}{1.590668in}}%
\pgfpathlineto{\pgfqpoint{2.866989in}{1.590668in}}%
\pgfpathlineto{\pgfqpoint{2.843110in}{1.590668in}}%
\pgfpathlineto{\pgfqpoint{2.819231in}{1.590668in}}%
\pgfpathlineto{\pgfqpoint{2.795352in}{1.590668in}}%
\pgfpathlineto{\pgfqpoint{2.771472in}{1.590668in}}%
\pgfpathlineto{\pgfqpoint{2.747593in}{1.590668in}}%
\pgfpathlineto{\pgfqpoint{2.723714in}{1.590668in}}%
\pgfpathlineto{\pgfqpoint{2.699835in}{1.590668in}}%
\pgfpathlineto{\pgfqpoint{2.675956in}{1.590668in}}%
\pgfpathlineto{\pgfqpoint{2.652077in}{1.590668in}}%
\pgfpathlineto{\pgfqpoint{2.628198in}{1.590668in}}%
\pgfpathlineto{\pgfqpoint{2.604318in}{1.590668in}}%
\pgfpathlineto{\pgfqpoint{2.580439in}{1.590668in}}%
\pgfpathlineto{\pgfqpoint{2.556560in}{1.590668in}}%
\pgfpathlineto{\pgfqpoint{2.532681in}{1.590668in}}%
\pgfpathlineto{\pgfqpoint{2.508802in}{1.590668in}}%
\pgfpathlineto{\pgfqpoint{2.484923in}{1.590668in}}%
\pgfpathlineto{\pgfqpoint{2.461044in}{1.590668in}}%
\pgfpathlineto{\pgfqpoint{2.437164in}{1.590668in}}%
\pgfpathlineto{\pgfqpoint{2.413285in}{1.590668in}}%
\pgfpathlineto{\pgfqpoint{2.389406in}{1.590668in}}%
\pgfpathlineto{\pgfqpoint{2.365527in}{1.590668in}}%
\pgfpathlineto{\pgfqpoint{2.341648in}{1.590668in}}%
\pgfpathlineto{\pgfqpoint{2.317769in}{1.590668in}}%
\pgfpathlineto{\pgfqpoint{2.293890in}{1.590668in}}%
\pgfpathlineto{\pgfqpoint{2.270010in}{1.590668in}}%
\pgfpathlineto{\pgfqpoint{2.246131in}{1.590668in}}%
\pgfpathlineto{\pgfqpoint{2.222252in}{1.590668in}}%
\pgfpathlineto{\pgfqpoint{2.198373in}{1.590668in}}%
\pgfpathlineto{\pgfqpoint{2.174494in}{1.590668in}}%
\pgfpathlineto{\pgfqpoint{2.150615in}{1.590668in}}%
\pgfpathlineto{\pgfqpoint{2.126736in}{1.590668in}}%
\pgfpathlineto{\pgfqpoint{2.102857in}{1.590668in}}%
\pgfpathlineto{\pgfqpoint{2.078977in}{1.590668in}}%
\pgfpathlineto{\pgfqpoint{2.055098in}{1.590668in}}%
\pgfpathlineto{\pgfqpoint{2.031219in}{1.590668in}}%
\pgfpathlineto{\pgfqpoint{2.007340in}{1.590668in}}%
\pgfpathlineto{\pgfqpoint{1.983461in}{1.590668in}}%
\pgfpathlineto{\pgfqpoint{1.959582in}{1.590668in}}%
\pgfpathlineto{\pgfqpoint{1.935703in}{1.590668in}}%
\pgfpathlineto{\pgfqpoint{1.911823in}{1.590668in}}%
\pgfpathlineto{\pgfqpoint{1.887944in}{1.590668in}}%
\pgfpathlineto{\pgfqpoint{1.864065in}{1.590668in}}%
\pgfpathlineto{\pgfqpoint{1.840186in}{1.590668in}}%
\pgfpathlineto{\pgfqpoint{1.816307in}{1.590668in}}%
\pgfpathlineto{\pgfqpoint{1.792428in}{1.590668in}}%
\pgfpathlineto{\pgfqpoint{1.768549in}{1.590668in}}%
\pgfpathlineto{\pgfqpoint{1.744669in}{1.590668in}}%
\pgfpathlineto{\pgfqpoint{1.720790in}{1.590668in}}%
\pgfpathlineto{\pgfqpoint{1.696911in}{1.590668in}}%
\pgfpathlineto{\pgfqpoint{1.673032in}{1.590668in}}%
\pgfpathlineto{\pgfqpoint{1.649153in}{1.590668in}}%
\pgfpathlineto{\pgfqpoint{1.625274in}{1.590668in}}%
\pgfpathlineto{\pgfqpoint{1.601395in}{1.590668in}}%
\pgfpathlineto{\pgfqpoint{1.577516in}{1.590668in}}%
\pgfpathlineto{\pgfqpoint{1.553636in}{1.590668in}}%
\pgfpathlineto{\pgfqpoint{1.529757in}{1.590668in}}%
\pgfpathlineto{\pgfqpoint{1.505878in}{1.590668in}}%
\pgfpathlineto{\pgfqpoint{1.481999in}{1.590668in}}%
\pgfpathlineto{\pgfqpoint{1.458120in}{1.590668in}}%
\pgfpathlineto{\pgfqpoint{1.434241in}{1.590668in}}%
\pgfpathlineto{\pgfqpoint{1.410362in}{1.590668in}}%
\pgfpathlineto{\pgfqpoint{1.386482in}{1.590668in}}%
\pgfpathlineto{\pgfqpoint{1.362603in}{1.590668in}}%
\pgfpathlineto{\pgfqpoint{1.338724in}{1.590668in}}%
\pgfpathlineto{\pgfqpoint{1.314845in}{1.590668in}}%
\pgfpathlineto{\pgfqpoint{1.290966in}{1.590668in}}%
\pgfpathlineto{\pgfqpoint{1.267087in}{1.590668in}}%
\pgfpathlineto{\pgfqpoint{1.243208in}{1.590668in}}%
\pgfpathlineto{\pgfqpoint{1.219328in}{1.590668in}}%
\pgfpathlineto{\pgfqpoint{1.195449in}{1.590668in}}%
\pgfpathlineto{\pgfqpoint{1.171570in}{1.590668in}}%
\pgfpathlineto{\pgfqpoint{1.147691in}{1.590668in}}%
\pgfpathlineto{\pgfqpoint{1.123812in}{1.590668in}}%
\pgfpathlineto{\pgfqpoint{1.099933in}{1.590668in}}%
\pgfpathlineto{\pgfqpoint{1.076054in}{1.590668in}}%
\pgfpathlineto{\pgfqpoint{1.052174in}{1.590668in}}%
\pgfpathlineto{\pgfqpoint{1.028295in}{1.590668in}}%
\pgfpathlineto{\pgfqpoint{1.004416in}{1.590668in}}%
\pgfpathlineto{\pgfqpoint{0.980537in}{1.590668in}}%
\pgfpathlineto{\pgfqpoint{0.956658in}{1.590668in}}%
\pgfpathlineto{\pgfqpoint{0.932779in}{1.590668in}}%
\pgfpathlineto{\pgfqpoint{0.908900in}{1.590668in}}%
\pgfpathlineto{\pgfqpoint{0.885021in}{1.590668in}}%
\pgfpathlineto{\pgfqpoint{0.861141in}{1.590668in}}%
\pgfpathlineto{\pgfqpoint{0.837262in}{1.590668in}}%
\pgfpathlineto{\pgfqpoint{0.813383in}{1.590668in}}%
\pgfpathlineto{\pgfqpoint{0.789504in}{1.590668in}}%
\pgfpathlineto{\pgfqpoint{0.765625in}{1.590668in}}%
\pgfpathlineto{\pgfqpoint{0.741746in}{1.590668in}}%
\pgfpathlineto{\pgfqpoint{0.717867in}{1.590668in}}%
\pgfpathlineto{\pgfqpoint{0.693987in}{1.590668in}}%
\pgfpathlineto{\pgfqpoint{0.670108in}{1.590668in}}%
\pgfpathlineto{\pgfqpoint{0.646229in}{1.590668in}}%
\pgfpathlineto{\pgfqpoint{0.622350in}{1.590668in}}%
\pgfpathlineto{\pgfqpoint{0.598471in}{1.590668in}}%
\pgfpathlineto{\pgfqpoint{0.574592in}{1.590668in}}%
\pgfpathlineto{\pgfqpoint{0.550713in}{1.590668in}}%
\pgfpathclose%
\pgfusepath{fill}%
}%
\begin{pgfscope}%
\pgfsys@transformshift{0.000000in}{0.000000in}%
\pgfsys@useobject{currentmarker}{}%
\end{pgfscope}%
\end{pgfscope}%
\begin{pgfscope}%
\pgfpathrectangle{\pgfqpoint{0.550713in}{0.398220in}}{\pgfqpoint{2.364035in}{1.703497in}}%
\pgfusepath{clip}%
\pgfsetbuttcap%
\pgfsetroundjoin%
\definecolor{currentfill}{rgb}{0.556863,0.729412,0.898039}%
\pgfsetfillcolor{currentfill}%
\pgfsetfillopacity{0.700000}%
\pgfsetlinewidth{0.000000pt}%
\definecolor{currentstroke}{rgb}{0.556863,0.729412,0.898039}%
\pgfsetstrokecolor{currentstroke}%
\pgfsetstrokeopacity{0.700000}%
\pgfsetdash{}{0pt}%
\pgfsys@defobject{currentmarker}{\pgfqpoint{0.550713in}{0.802303in}}{\pgfqpoint{2.914747in}{1.813084in}}{%
\pgfpathmoveto{\pgfqpoint{0.550713in}{1.590207in}}%
\pgfpathlineto{\pgfqpoint{0.550713in}{0.909701in}}%
\pgfpathlineto{\pgfqpoint{0.574592in}{0.909870in}}%
\pgfpathlineto{\pgfqpoint{0.598471in}{0.910135in}}%
\pgfpathlineto{\pgfqpoint{0.622350in}{0.910584in}}%
\pgfpathlineto{\pgfqpoint{0.646229in}{0.911367in}}%
\pgfpathlineto{\pgfqpoint{0.670108in}{0.912731in}}%
\pgfpathlineto{\pgfqpoint{0.693987in}{0.915009in}}%
\pgfpathlineto{\pgfqpoint{0.717867in}{0.918645in}}%
\pgfpathlineto{\pgfqpoint{0.741746in}{0.924190in}}%
\pgfpathlineto{\pgfqpoint{0.765625in}{0.932290in}}%
\pgfpathlineto{\pgfqpoint{0.789504in}{0.943678in}}%
\pgfpathlineto{\pgfqpoint{0.813383in}{0.959118in}}%
\pgfpathlineto{\pgfqpoint{0.837262in}{0.979290in}}%
\pgfpathlineto{\pgfqpoint{0.861141in}{1.004636in}}%
\pgfpathlineto{\pgfqpoint{0.885021in}{1.035108in}}%
\pgfpathlineto{\pgfqpoint{0.908900in}{1.069916in}}%
\pgfpathlineto{\pgfqpoint{0.932779in}{1.107304in}}%
\pgfpathlineto{\pgfqpoint{0.956658in}{1.144481in}}%
\pgfpathlineto{\pgfqpoint{0.980537in}{1.177873in}}%
\pgfpathlineto{\pgfqpoint{1.004416in}{1.203759in}}%
\pgfpathlineto{\pgfqpoint{1.028295in}{1.219195in}}%
\pgfpathlineto{\pgfqpoint{1.052174in}{1.222853in}}%
\pgfpathlineto{\pgfqpoint{1.076054in}{1.215263in}}%
\pgfpathlineto{\pgfqpoint{1.099933in}{1.198418in}}%
\pgfpathlineto{\pgfqpoint{1.123812in}{1.174948in}}%
\pgfpathlineto{\pgfqpoint{1.147691in}{1.147390in}}%
\pgfpathlineto{\pgfqpoint{1.171570in}{1.117750in}}%
\pgfpathlineto{\pgfqpoint{1.195449in}{1.087470in}}%
\pgfpathlineto{\pgfqpoint{1.219328in}{1.057637in}}%
\pgfpathlineto{\pgfqpoint{1.243208in}{1.029204in}}%
\pgfpathlineto{\pgfqpoint{1.267087in}{1.002980in}}%
\pgfpathlineto{\pgfqpoint{1.290966in}{0.979418in}}%
\pgfpathlineto{\pgfqpoint{1.314845in}{0.958501in}}%
\pgfpathlineto{\pgfqpoint{1.338724in}{0.939848in}}%
\pgfpathlineto{\pgfqpoint{1.362603in}{0.922964in}}%
\pgfpathlineto{\pgfqpoint{1.386482in}{0.907429in}}%
\pgfpathlineto{\pgfqpoint{1.410362in}{0.892957in}}%
\pgfpathlineto{\pgfqpoint{1.434241in}{0.879394in}}%
\pgfpathlineto{\pgfqpoint{1.458120in}{0.866765in}}%
\pgfpathlineto{\pgfqpoint{1.481999in}{0.855254in}}%
\pgfpathlineto{\pgfqpoint{1.505878in}{0.845083in}}%
\pgfpathlineto{\pgfqpoint{1.529757in}{0.836343in}}%
\pgfpathlineto{\pgfqpoint{1.553636in}{0.828924in}}%
\pgfpathlineto{\pgfqpoint{1.577516in}{0.822603in}}%
\pgfpathlineto{\pgfqpoint{1.601395in}{0.817196in}}%
\pgfpathlineto{\pgfqpoint{1.625274in}{0.812593in}}%
\pgfpathlineto{\pgfqpoint{1.649153in}{0.808749in}}%
\pgfpathlineto{\pgfqpoint{1.673032in}{0.805673in}}%
\pgfpathlineto{\pgfqpoint{1.696911in}{0.803464in}}%
\pgfpathlineto{\pgfqpoint{1.720790in}{0.802303in}}%
\pgfpathlineto{\pgfqpoint{1.744669in}{0.802343in}}%
\pgfpathlineto{\pgfqpoint{1.768549in}{0.803585in}}%
\pgfpathlineto{\pgfqpoint{1.792428in}{0.805878in}}%
\pgfpathlineto{\pgfqpoint{1.816307in}{0.809038in}}%
\pgfpathlineto{\pgfqpoint{1.840186in}{0.812964in}}%
\pgfpathlineto{\pgfqpoint{1.864065in}{0.817633in}}%
\pgfpathlineto{\pgfqpoint{1.887944in}{0.823091in}}%
\pgfpathlineto{\pgfqpoint{1.911823in}{0.829438in}}%
\pgfpathlineto{\pgfqpoint{1.935703in}{0.836868in}}%
\pgfpathlineto{\pgfqpoint{1.959582in}{0.845616in}}%
\pgfpathlineto{\pgfqpoint{1.983461in}{0.855800in}}%
\pgfpathlineto{\pgfqpoint{2.007340in}{0.867337in}}%
\pgfpathlineto{\pgfqpoint{2.031219in}{0.880004in}}%
\pgfpathlineto{\pgfqpoint{2.055098in}{0.893611in}}%
\pgfpathlineto{\pgfqpoint{2.078977in}{0.908130in}}%
\pgfpathlineto{\pgfqpoint{2.102857in}{0.923701in}}%
\pgfpathlineto{\pgfqpoint{2.126736in}{0.940618in}}%
\pgfpathlineto{\pgfqpoint{2.150615in}{0.959310in}}%
\pgfpathlineto{\pgfqpoint{2.174494in}{0.980281in}}%
\pgfpathlineto{\pgfqpoint{2.198373in}{1.003928in}}%
\pgfpathlineto{\pgfqpoint{2.222252in}{1.030266in}}%
\pgfpathlineto{\pgfqpoint{2.246131in}{1.058818in}}%
\pgfpathlineto{\pgfqpoint{2.270010in}{1.088743in}}%
\pgfpathlineto{\pgfqpoint{2.293890in}{1.119054in}}%
\pgfpathlineto{\pgfqpoint{2.317769in}{1.148638in}}%
\pgfpathlineto{\pgfqpoint{2.341648in}{1.176045in}}%
\pgfpathlineto{\pgfqpoint{2.365527in}{1.199284in}}%
\pgfpathlineto{\pgfqpoint{2.389406in}{1.215854in}}%
\pgfpathlineto{\pgfqpoint{2.413285in}{1.223174in}}%
\pgfpathlineto{\pgfqpoint{2.437164in}{1.219311in}}%
\pgfpathlineto{\pgfqpoint{2.461044in}{1.203752in}}%
\pgfpathlineto{\pgfqpoint{2.484923in}{1.177811in}}%
\pgfpathlineto{\pgfqpoint{2.508802in}{1.144386in}}%
\pgfpathlineto{\pgfqpoint{2.532681in}{1.107163in}}%
\pgfpathlineto{\pgfqpoint{2.556560in}{1.069719in}}%
\pgfpathlineto{\pgfqpoint{2.580439in}{1.034868in}}%
\pgfpathlineto{\pgfqpoint{2.604318in}{1.004402in}}%
\pgfpathlineto{\pgfqpoint{2.628198in}{0.979145in}}%
\pgfpathlineto{\pgfqpoint{2.652077in}{0.959134in}}%
\pgfpathlineto{\pgfqpoint{2.675956in}{0.943904in}}%
\pgfpathlineto{\pgfqpoint{2.699835in}{0.932721in}}%
\pgfpathlineto{\pgfqpoint{2.723714in}{0.924770in}}%
\pgfpathlineto{\pgfqpoint{2.747593in}{0.919290in}}%
\pgfpathlineto{\pgfqpoint{2.771472in}{0.915622in}}%
\pgfpathlineto{\pgfqpoint{2.795352in}{0.913241in}}%
\pgfpathlineto{\pgfqpoint{2.819231in}{0.911726in}}%
\pgfpathlineto{\pgfqpoint{2.843110in}{0.910751in}}%
\pgfpathlineto{\pgfqpoint{2.866989in}{0.910091in}}%
\pgfpathlineto{\pgfqpoint{2.890868in}{0.909610in}}%
\pgfpathlineto{\pgfqpoint{2.914747in}{0.909255in}}%
\pgfpathlineto{\pgfqpoint{2.914747in}{1.589067in}}%
\pgfpathlineto{\pgfqpoint{2.914747in}{1.589067in}}%
\pgfpathlineto{\pgfqpoint{2.890868in}{1.588821in}}%
\pgfpathlineto{\pgfqpoint{2.866989in}{1.588862in}}%
\pgfpathlineto{\pgfqpoint{2.843110in}{1.589342in}}%
\pgfpathlineto{\pgfqpoint{2.819231in}{1.590417in}}%
\pgfpathlineto{\pgfqpoint{2.795352in}{1.592263in}}%
\pgfpathlineto{\pgfqpoint{2.771472in}{1.595108in}}%
\pgfpathlineto{\pgfqpoint{2.747593in}{1.599261in}}%
\pgfpathlineto{\pgfqpoint{2.723714in}{1.605109in}}%
\pgfpathlineto{\pgfqpoint{2.699835in}{1.613073in}}%
\pgfpathlineto{\pgfqpoint{2.675956in}{1.623559in}}%
\pgfpathlineto{\pgfqpoint{2.652077in}{1.636848in}}%
\pgfpathlineto{\pgfqpoint{2.628198in}{1.653027in}}%
\pgfpathlineto{\pgfqpoint{2.604318in}{1.671904in}}%
\pgfpathlineto{\pgfqpoint{2.580439in}{1.692965in}}%
\pgfpathlineto{\pgfqpoint{2.556560in}{1.715417in}}%
\pgfpathlineto{\pgfqpoint{2.532681in}{1.738241in}}%
\pgfpathlineto{\pgfqpoint{2.508802in}{1.760323in}}%
\pgfpathlineto{\pgfqpoint{2.484923in}{1.780465in}}%
\pgfpathlineto{\pgfqpoint{2.461044in}{1.797259in}}%
\pgfpathlineto{\pgfqpoint{2.437164in}{1.808864in}}%
\pgfpathlineto{\pgfqpoint{2.413285in}{1.812956in}}%
\pgfpathlineto{\pgfqpoint{2.389406in}{1.807126in}}%
\pgfpathlineto{\pgfqpoint{2.365527in}{1.789681in}}%
\pgfpathlineto{\pgfqpoint{2.341648in}{1.760456in}}%
\pgfpathlineto{\pgfqpoint{2.317769in}{1.721192in}}%
\pgfpathlineto{\pgfqpoint{2.293890in}{1.675258in}}%
\pgfpathlineto{\pgfqpoint{2.270010in}{1.626820in}}%
\pgfpathlineto{\pgfqpoint{2.246131in}{1.579735in}}%
\pgfpathlineto{\pgfqpoint{2.222252in}{1.536666in}}%
\pgfpathlineto{\pgfqpoint{2.198373in}{1.498765in}}%
\pgfpathlineto{\pgfqpoint{2.174494in}{1.465934in}}%
\pgfpathlineto{\pgfqpoint{2.150615in}{1.437329in}}%
\pgfpathlineto{\pgfqpoint{2.126736in}{1.411829in}}%
\pgfpathlineto{\pgfqpoint{2.102857in}{1.388361in}}%
\pgfpathlineto{\pgfqpoint{2.078977in}{1.366178in}}%
\pgfpathlineto{\pgfqpoint{2.055098in}{1.345015in}}%
\pgfpathlineto{\pgfqpoint{2.031219in}{1.325058in}}%
\pgfpathlineto{\pgfqpoint{2.007340in}{1.306705in}}%
\pgfpathlineto{\pgfqpoint{1.983461in}{1.290285in}}%
\pgfpathlineto{\pgfqpoint{1.959582in}{1.275916in}}%
\pgfpathlineto{\pgfqpoint{1.935703in}{1.263544in}}%
\pgfpathlineto{\pgfqpoint{1.911823in}{1.252999in}}%
\pgfpathlineto{\pgfqpoint{1.887944in}{1.244034in}}%
\pgfpathlineto{\pgfqpoint{1.864065in}{1.236350in}}%
\pgfpathlineto{\pgfqpoint{1.840186in}{1.229717in}}%
\pgfpathlineto{\pgfqpoint{1.816307in}{1.224095in}}%
\pgfpathlineto{\pgfqpoint{1.792428in}{1.219634in}}%
\pgfpathlineto{\pgfqpoint{1.768549in}{1.216550in}}%
\pgfpathlineto{\pgfqpoint{1.744669in}{1.214994in}}%
\pgfpathlineto{\pgfqpoint{1.720790in}{1.215022in}}%
\pgfpathlineto{\pgfqpoint{1.696911in}{1.216633in}}%
\pgfpathlineto{\pgfqpoint{1.673032in}{1.219768in}}%
\pgfpathlineto{\pgfqpoint{1.649153in}{1.224274in}}%
\pgfpathlineto{\pgfqpoint{1.625274in}{1.229930in}}%
\pgfpathlineto{\pgfqpoint{1.601395in}{1.236579in}}%
\pgfpathlineto{\pgfqpoint{1.577516in}{1.244262in}}%
\pgfpathlineto{\pgfqpoint{1.553636in}{1.253203in}}%
\pgfpathlineto{\pgfqpoint{1.529757in}{1.263705in}}%
\pgfpathlineto{\pgfqpoint{1.505878in}{1.276028in}}%
\pgfpathlineto{\pgfqpoint{1.481999in}{1.290343in}}%
\pgfpathlineto{\pgfqpoint{1.458120in}{1.306707in}}%
\pgfpathlineto{\pgfqpoint{1.434241in}{1.325000in}}%
\pgfpathlineto{\pgfqpoint{1.410362in}{1.344887in}}%
\pgfpathlineto{\pgfqpoint{1.386482in}{1.365972in}}%
\pgfpathlineto{\pgfqpoint{1.362603in}{1.388068in}}%
\pgfpathlineto{\pgfqpoint{1.338724in}{1.411448in}}%
\pgfpathlineto{\pgfqpoint{1.314845in}{1.436880in}}%
\pgfpathlineto{\pgfqpoint{1.290966in}{1.465449in}}%
\pgfpathlineto{\pgfqpoint{1.267087in}{1.498293in}}%
\pgfpathlineto{\pgfqpoint{1.243208in}{1.536256in}}%
\pgfpathlineto{\pgfqpoint{1.219328in}{1.579418in}}%
\pgfpathlineto{\pgfqpoint{1.195449in}{1.626605in}}%
\pgfpathlineto{\pgfqpoint{1.171570in}{1.675124in}}%
\pgfpathlineto{\pgfqpoint{1.147691in}{1.721108in}}%
\pgfpathlineto{\pgfqpoint{1.123812in}{1.760396in}}%
\pgfpathlineto{\pgfqpoint{1.099933in}{1.789639in}}%
\pgfpathlineto{\pgfqpoint{1.076054in}{1.807136in}}%
\pgfpathlineto{\pgfqpoint{1.052174in}{1.813084in}}%
\pgfpathlineto{\pgfqpoint{1.028295in}{1.809206in}}%
\pgfpathlineto{\pgfqpoint{1.004416in}{1.797893in}}%
\pgfpathlineto{\pgfqpoint{0.980537in}{1.781409in}}%
\pgfpathlineto{\pgfqpoint{0.956658in}{1.761502in}}%
\pgfpathlineto{\pgfqpoint{0.932779in}{1.739522in}}%
\pgfpathlineto{\pgfqpoint{0.908900in}{1.716655in}}%
\pgfpathlineto{\pgfqpoint{0.885021in}{1.694070in}}%
\pgfpathlineto{\pgfqpoint{0.861141in}{1.672859in}}%
\pgfpathlineto{\pgfqpoint{0.837262in}{1.653909in}}%
\pgfpathlineto{\pgfqpoint{0.813383in}{1.637789in}}%
\pgfpathlineto{\pgfqpoint{0.789504in}{1.624712in}}%
\pgfpathlineto{\pgfqpoint{0.765625in}{1.614564in}}%
\pgfpathlineto{\pgfqpoint{0.741746in}{1.607003in}}%
\pgfpathlineto{\pgfqpoint{0.717867in}{1.601562in}}%
\pgfpathlineto{\pgfqpoint{0.693987in}{1.597751in}}%
\pgfpathlineto{\pgfqpoint{0.670108in}{1.595131in}}%
\pgfpathlineto{\pgfqpoint{0.646229in}{1.593345in}}%
\pgfpathlineto{\pgfqpoint{0.622350in}{1.592123in}}%
\pgfpathlineto{\pgfqpoint{0.598471in}{1.591275in}}%
\pgfpathlineto{\pgfqpoint{0.574592in}{1.590667in}}%
\pgfpathlineto{\pgfqpoint{0.550713in}{1.590207in}}%
\pgfpathclose%
\pgfusepath{fill}%
}%
\begin{pgfscope}%
\pgfsys@transformshift{0.000000in}{0.000000in}%
\pgfsys@useobject{currentmarker}{}%
\end{pgfscope}%
\end{pgfscope}%
\begin{pgfscope}%
\pgfsetbuttcap%
\pgfsetroundjoin%
\definecolor{currentfill}{rgb}{0.000000,0.000000,0.000000}%
\pgfsetfillcolor{currentfill}%
\pgfsetlinewidth{0.803000pt}%
\definecolor{currentstroke}{rgb}{0.000000,0.000000,0.000000}%
\pgfsetstrokecolor{currentstroke}%
\pgfsetdash{}{0pt}%
\pgfsys@defobject{currentmarker}{\pgfqpoint{0.000000in}{-0.048611in}}{\pgfqpoint{0.000000in}{0.000000in}}{%
\pgfpathmoveto{\pgfqpoint{0.000000in}{0.000000in}}%
\pgfpathlineto{\pgfqpoint{0.000000in}{-0.048611in}}%
\pgfusepath{stroke,fill}%
}%
\begin{pgfscope}%
\pgfsys@transformshift{0.993969in}{0.398220in}%
\pgfsys@useobject{currentmarker}{}%
\end{pgfscope}%
\end{pgfscope}%
\begin{pgfscope}%
\definecolor{textcolor}{rgb}{0.000000,0.000000,0.000000}%
\pgfsetstrokecolor{textcolor}%
\pgfsetfillcolor{textcolor}%
\pgftext[x=0.993969in,y=0.300998in,,top]{\color{textcolor}\rmfamily\fontsize{10.000000}{12.000000}\selectfont \(\displaystyle {\ensuremath{-}5}\)}%
\end{pgfscope}%
\begin{pgfscope}%
\pgfsetbuttcap%
\pgfsetroundjoin%
\definecolor{currentfill}{rgb}{0.000000,0.000000,0.000000}%
\pgfsetfillcolor{currentfill}%
\pgfsetlinewidth{0.803000pt}%
\definecolor{currentstroke}{rgb}{0.000000,0.000000,0.000000}%
\pgfsetstrokecolor{currentstroke}%
\pgfsetdash{}{0pt}%
\pgfsys@defobject{currentmarker}{\pgfqpoint{0.000000in}{-0.048611in}}{\pgfqpoint{0.000000in}{0.000000in}}{%
\pgfpathmoveto{\pgfqpoint{0.000000in}{0.000000in}}%
\pgfpathlineto{\pgfqpoint{0.000000in}{-0.048611in}}%
\pgfusepath{stroke,fill}%
}%
\begin{pgfscope}%
\pgfsys@transformshift{1.732730in}{0.398220in}%
\pgfsys@useobject{currentmarker}{}%
\end{pgfscope}%
\end{pgfscope}%
\begin{pgfscope}%
\definecolor{textcolor}{rgb}{0.000000,0.000000,0.000000}%
\pgfsetstrokecolor{textcolor}%
\pgfsetfillcolor{textcolor}%
\pgftext[x=1.732730in,y=0.300998in,,top]{\color{textcolor}\rmfamily\fontsize{10.000000}{12.000000}\selectfont \(\displaystyle {0}\)}%
\end{pgfscope}%
\begin{pgfscope}%
\pgfsetbuttcap%
\pgfsetroundjoin%
\definecolor{currentfill}{rgb}{0.000000,0.000000,0.000000}%
\pgfsetfillcolor{currentfill}%
\pgfsetlinewidth{0.803000pt}%
\definecolor{currentstroke}{rgb}{0.000000,0.000000,0.000000}%
\pgfsetstrokecolor{currentstroke}%
\pgfsetdash{}{0pt}%
\pgfsys@defobject{currentmarker}{\pgfqpoint{0.000000in}{-0.048611in}}{\pgfqpoint{0.000000in}{0.000000in}}{%
\pgfpathmoveto{\pgfqpoint{0.000000in}{0.000000in}}%
\pgfpathlineto{\pgfqpoint{0.000000in}{-0.048611in}}%
\pgfusepath{stroke,fill}%
}%
\begin{pgfscope}%
\pgfsys@transformshift{2.471491in}{0.398220in}%
\pgfsys@useobject{currentmarker}{}%
\end{pgfscope}%
\end{pgfscope}%
\begin{pgfscope}%
\definecolor{textcolor}{rgb}{0.000000,0.000000,0.000000}%
\pgfsetstrokecolor{textcolor}%
\pgfsetfillcolor{textcolor}%
\pgftext[x=2.471491in,y=0.300998in,,top]{\color{textcolor}\rmfamily\fontsize{10.000000}{12.000000}\selectfont \(\displaystyle {5}\)}%
\end{pgfscope}%
\begin{pgfscope}%
\definecolor{textcolor}{rgb}{0.000000,0.000000,0.000000}%
\pgfsetstrokecolor{textcolor}%
\pgfsetfillcolor{textcolor}%
\pgftext[x=1.732730in,y=0.122109in,,top]{\color{textcolor}\rmfamily\fontsize{10.000000}{12.000000}\selectfont \(\displaystyle x\)}%
\end{pgfscope}%
\begin{pgfscope}%
\pgfsetbuttcap%
\pgfsetroundjoin%
\definecolor{currentfill}{rgb}{0.000000,0.000000,0.000000}%
\pgfsetfillcolor{currentfill}%
\pgfsetlinewidth{0.803000pt}%
\definecolor{currentstroke}{rgb}{0.000000,0.000000,0.000000}%
\pgfsetstrokecolor{currentstroke}%
\pgfsetdash{}{0pt}%
\pgfsys@defobject{currentmarker}{\pgfqpoint{-0.048611in}{0.000000in}}{\pgfqpoint{-0.000000in}{0.000000in}}{%
\pgfpathmoveto{\pgfqpoint{-0.000000in}{0.000000in}}%
\pgfpathlineto{\pgfqpoint{-0.048611in}{0.000000in}}%
\pgfusepath{stroke,fill}%
}%
\begin{pgfscope}%
\pgfsys@transformshift{0.550713in}{0.398220in}%
\pgfsys@useobject{currentmarker}{}%
\end{pgfscope}%
\end{pgfscope}%
\begin{pgfscope}%
\definecolor{textcolor}{rgb}{0.000000,0.000000,0.000000}%
\pgfsetstrokecolor{textcolor}%
\pgfsetfillcolor{textcolor}%
\pgftext[x=0.206576in, y=0.350026in, left, base]{\color{textcolor}\rmfamily\fontsize{10.000000}{12.000000}\selectfont \(\displaystyle {\ensuremath{-}10}\)}%
\end{pgfscope}%
\begin{pgfscope}%
\pgfsetbuttcap%
\pgfsetroundjoin%
\definecolor{currentfill}{rgb}{0.000000,0.000000,0.000000}%
\pgfsetfillcolor{currentfill}%
\pgfsetlinewidth{0.803000pt}%
\definecolor{currentstroke}{rgb}{0.000000,0.000000,0.000000}%
\pgfsetstrokecolor{currentstroke}%
\pgfsetdash{}{0pt}%
\pgfsys@defobject{currentmarker}{\pgfqpoint{-0.048611in}{0.000000in}}{\pgfqpoint{-0.000000in}{0.000000in}}{%
\pgfpathmoveto{\pgfqpoint{-0.000000in}{0.000000in}}%
\pgfpathlineto{\pgfqpoint{-0.048611in}{0.000000in}}%
\pgfusepath{stroke,fill}%
}%
\begin{pgfscope}%
\pgfsys@transformshift{0.550713in}{0.824094in}%
\pgfsys@useobject{currentmarker}{}%
\end{pgfscope}%
\end{pgfscope}%
\begin{pgfscope}%
\definecolor{textcolor}{rgb}{0.000000,0.000000,0.000000}%
\pgfsetstrokecolor{textcolor}%
\pgfsetfillcolor{textcolor}%
\pgftext[x=0.276021in, y=0.775900in, left, base]{\color{textcolor}\rmfamily\fontsize{10.000000}{12.000000}\selectfont \(\displaystyle {\ensuremath{-}5}\)}%
\end{pgfscope}%
\begin{pgfscope}%
\pgfsetbuttcap%
\pgfsetroundjoin%
\definecolor{currentfill}{rgb}{0.000000,0.000000,0.000000}%
\pgfsetfillcolor{currentfill}%
\pgfsetlinewidth{0.803000pt}%
\definecolor{currentstroke}{rgb}{0.000000,0.000000,0.000000}%
\pgfsetstrokecolor{currentstroke}%
\pgfsetdash{}{0pt}%
\pgfsys@defobject{currentmarker}{\pgfqpoint{-0.048611in}{0.000000in}}{\pgfqpoint{-0.000000in}{0.000000in}}{%
\pgfpathmoveto{\pgfqpoint{-0.000000in}{0.000000in}}%
\pgfpathlineto{\pgfqpoint{-0.048611in}{0.000000in}}%
\pgfusepath{stroke,fill}%
}%
\begin{pgfscope}%
\pgfsys@transformshift{0.550713in}{1.249969in}%
\pgfsys@useobject{currentmarker}{}%
\end{pgfscope}%
\end{pgfscope}%
\begin{pgfscope}%
\definecolor{textcolor}{rgb}{0.000000,0.000000,0.000000}%
\pgfsetstrokecolor{textcolor}%
\pgfsetfillcolor{textcolor}%
\pgftext[x=0.384046in, y=1.201774in, left, base]{\color{textcolor}\rmfamily\fontsize{10.000000}{12.000000}\selectfont \(\displaystyle {0}\)}%
\end{pgfscope}%
\begin{pgfscope}%
\pgfsetbuttcap%
\pgfsetroundjoin%
\definecolor{currentfill}{rgb}{0.000000,0.000000,0.000000}%
\pgfsetfillcolor{currentfill}%
\pgfsetlinewidth{0.803000pt}%
\definecolor{currentstroke}{rgb}{0.000000,0.000000,0.000000}%
\pgfsetstrokecolor{currentstroke}%
\pgfsetdash{}{0pt}%
\pgfsys@defobject{currentmarker}{\pgfqpoint{-0.048611in}{0.000000in}}{\pgfqpoint{-0.000000in}{0.000000in}}{%
\pgfpathmoveto{\pgfqpoint{-0.000000in}{0.000000in}}%
\pgfpathlineto{\pgfqpoint{-0.048611in}{0.000000in}}%
\pgfusepath{stroke,fill}%
}%
\begin{pgfscope}%
\pgfsys@transformshift{0.550713in}{1.675843in}%
\pgfsys@useobject{currentmarker}{}%
\end{pgfscope}%
\end{pgfscope}%
\begin{pgfscope}%
\definecolor{textcolor}{rgb}{0.000000,0.000000,0.000000}%
\pgfsetstrokecolor{textcolor}%
\pgfsetfillcolor{textcolor}%
\pgftext[x=0.384046in, y=1.627649in, left, base]{\color{textcolor}\rmfamily\fontsize{10.000000}{12.000000}\selectfont \(\displaystyle {5}\)}%
\end{pgfscope}%
\begin{pgfscope}%
\pgfsetbuttcap%
\pgfsetroundjoin%
\definecolor{currentfill}{rgb}{0.000000,0.000000,0.000000}%
\pgfsetfillcolor{currentfill}%
\pgfsetlinewidth{0.803000pt}%
\definecolor{currentstroke}{rgb}{0.000000,0.000000,0.000000}%
\pgfsetstrokecolor{currentstroke}%
\pgfsetdash{}{0pt}%
\pgfsys@defobject{currentmarker}{\pgfqpoint{-0.048611in}{0.000000in}}{\pgfqpoint{-0.000000in}{0.000000in}}{%
\pgfpathmoveto{\pgfqpoint{-0.000000in}{0.000000in}}%
\pgfpathlineto{\pgfqpoint{-0.048611in}{0.000000in}}%
\pgfusepath{stroke,fill}%
}%
\begin{pgfscope}%
\pgfsys@transformshift{0.550713in}{2.101717in}%
\pgfsys@useobject{currentmarker}{}%
\end{pgfscope}%
\end{pgfscope}%
\begin{pgfscope}%
\definecolor{textcolor}{rgb}{0.000000,0.000000,0.000000}%
\pgfsetstrokecolor{textcolor}%
\pgfsetfillcolor{textcolor}%
\pgftext[x=0.314601in, y=2.053523in, left, base]{\color{textcolor}\rmfamily\fontsize{10.000000}{12.000000}\selectfont \(\displaystyle {10}\)}%
\end{pgfscope}%
\begin{pgfscope}%
\definecolor{textcolor}{rgb}{0.000000,0.000000,0.000000}%
\pgfsetstrokecolor{textcolor}%
\pgfsetfillcolor{textcolor}%
\pgftext[x=0.151021in,y=1.249969in,,bottom,rotate=90.000000]{\color{textcolor}\rmfamily\fontsize{10.000000}{12.000000}\selectfont \(\displaystyle f(x)\)}%
\end{pgfscope}%
\begin{pgfscope}%
\pgfpathrectangle{\pgfqpoint{0.550713in}{0.398220in}}{\pgfqpoint{2.364035in}{1.703497in}}%
\pgfusepath{clip}%
\pgfsetrectcap%
\pgfsetroundjoin%
\pgfsetlinewidth{0.752812pt}%
\definecolor{currentstroke}{rgb}{0.000000,0.000000,0.000000}%
\pgfsetstrokecolor{currentstroke}%
\pgfsetdash{}{0pt}%
\pgfpathmoveto{\pgfqpoint{0.649782in}{2.111717in}}%
\pgfpathlineto{\pgfqpoint{0.670108in}{2.060587in}}%
\pgfpathlineto{\pgfqpoint{0.693987in}{2.001854in}}%
\pgfpathlineto{\pgfqpoint{0.717867in}{1.944456in}}%
\pgfpathlineto{\pgfqpoint{0.741746in}{1.888392in}}%
\pgfpathlineto{\pgfqpoint{0.765625in}{1.833663in}}%
\pgfpathlineto{\pgfqpoint{0.789504in}{1.780269in}}%
\pgfpathlineto{\pgfqpoint{0.813383in}{1.728210in}}%
\pgfpathlineto{\pgfqpoint{0.837262in}{1.677486in}}%
\pgfpathlineto{\pgfqpoint{0.861141in}{1.628096in}}%
\pgfpathlineto{\pgfqpoint{0.885021in}{1.580042in}}%
\pgfpathlineto{\pgfqpoint{0.908900in}{1.533322in}}%
\pgfpathlineto{\pgfqpoint{0.932779in}{1.487937in}}%
\pgfpathlineto{\pgfqpoint{0.956658in}{1.443887in}}%
\pgfpathlineto{\pgfqpoint{0.980537in}{1.401172in}}%
\pgfpathlineto{\pgfqpoint{1.004416in}{1.359791in}}%
\pgfpathlineto{\pgfqpoint{1.028295in}{1.319746in}}%
\pgfpathlineto{\pgfqpoint{1.052174in}{1.281035in}}%
\pgfpathlineto{\pgfqpoint{1.076054in}{1.243660in}}%
\pgfpathlineto{\pgfqpoint{1.099933in}{1.207619in}}%
\pgfpathlineto{\pgfqpoint{1.123812in}{1.172913in}}%
\pgfpathlineto{\pgfqpoint{1.147691in}{1.139541in}}%
\pgfpathlineto{\pgfqpoint{1.171570in}{1.107505in}}%
\pgfpathlineto{\pgfqpoint{1.195449in}{1.076803in}}%
\pgfpathlineto{\pgfqpoint{1.219328in}{1.047437in}}%
\pgfpathlineto{\pgfqpoint{1.243208in}{1.019405in}}%
\pgfpathlineto{\pgfqpoint{1.267087in}{0.992708in}}%
\pgfpathlineto{\pgfqpoint{1.290966in}{0.967346in}}%
\pgfpathlineto{\pgfqpoint{1.314845in}{0.943318in}}%
\pgfpathlineto{\pgfqpoint{1.338724in}{0.920626in}}%
\pgfpathlineto{\pgfqpoint{1.362603in}{0.899268in}}%
\pgfpathlineto{\pgfqpoint{1.386482in}{0.879246in}}%
\pgfpathlineto{\pgfqpoint{1.410362in}{0.860558in}}%
\pgfpathlineto{\pgfqpoint{1.434241in}{0.843205in}}%
\pgfpathlineto{\pgfqpoint{1.458120in}{0.827187in}}%
\pgfpathlineto{\pgfqpoint{1.481999in}{0.812503in}}%
\pgfpathlineto{\pgfqpoint{1.505878in}{0.799155in}}%
\pgfpathlineto{\pgfqpoint{1.529757in}{0.787141in}}%
\pgfpathlineto{\pgfqpoint{1.553636in}{0.776462in}}%
\pgfpathlineto{\pgfqpoint{1.577516in}{0.767118in}}%
\pgfpathlineto{\pgfqpoint{1.601395in}{0.759109in}}%
\pgfpathlineto{\pgfqpoint{1.625274in}{0.752435in}}%
\pgfpathlineto{\pgfqpoint{1.649153in}{0.747096in}}%
\pgfpathlineto{\pgfqpoint{1.673032in}{0.743091in}}%
\pgfpathlineto{\pgfqpoint{1.696911in}{0.740421in}}%
\pgfpathlineto{\pgfqpoint{1.720790in}{0.739086in}}%
\pgfpathlineto{\pgfqpoint{1.744669in}{0.739086in}}%
\pgfpathlineto{\pgfqpoint{1.768549in}{0.740421in}}%
\pgfpathlineto{\pgfqpoint{1.792428in}{0.743091in}}%
\pgfpathlineto{\pgfqpoint{1.816307in}{0.747096in}}%
\pgfpathlineto{\pgfqpoint{1.840186in}{0.752435in}}%
\pgfpathlineto{\pgfqpoint{1.864065in}{0.759109in}}%
\pgfpathlineto{\pgfqpoint{1.887944in}{0.767118in}}%
\pgfpathlineto{\pgfqpoint{1.911823in}{0.776462in}}%
\pgfpathlineto{\pgfqpoint{1.935703in}{0.787141in}}%
\pgfpathlineto{\pgfqpoint{1.959582in}{0.799155in}}%
\pgfpathlineto{\pgfqpoint{1.983461in}{0.812503in}}%
\pgfpathlineto{\pgfqpoint{2.007340in}{0.827187in}}%
\pgfpathlineto{\pgfqpoint{2.031219in}{0.843205in}}%
\pgfpathlineto{\pgfqpoint{2.055098in}{0.860558in}}%
\pgfpathlineto{\pgfqpoint{2.078977in}{0.879246in}}%
\pgfpathlineto{\pgfqpoint{2.102857in}{0.899268in}}%
\pgfpathlineto{\pgfqpoint{2.126736in}{0.920626in}}%
\pgfpathlineto{\pgfqpoint{2.150615in}{0.943318in}}%
\pgfpathlineto{\pgfqpoint{2.174494in}{0.967346in}}%
\pgfpathlineto{\pgfqpoint{2.198373in}{0.992708in}}%
\pgfpathlineto{\pgfqpoint{2.222252in}{1.019405in}}%
\pgfpathlineto{\pgfqpoint{2.246131in}{1.047437in}}%
\pgfpathlineto{\pgfqpoint{2.270010in}{1.076803in}}%
\pgfpathlineto{\pgfqpoint{2.293890in}{1.107505in}}%
\pgfpathlineto{\pgfqpoint{2.317769in}{1.139541in}}%
\pgfpathlineto{\pgfqpoint{2.341648in}{1.172913in}}%
\pgfpathlineto{\pgfqpoint{2.365527in}{1.207619in}}%
\pgfpathlineto{\pgfqpoint{2.389406in}{1.243660in}}%
\pgfpathlineto{\pgfqpoint{2.413285in}{1.281035in}}%
\pgfpathlineto{\pgfqpoint{2.437164in}{1.319746in}}%
\pgfpathlineto{\pgfqpoint{2.461044in}{1.359791in}}%
\pgfpathlineto{\pgfqpoint{2.484923in}{1.401172in}}%
\pgfpathlineto{\pgfqpoint{2.508802in}{1.443887in}}%
\pgfpathlineto{\pgfqpoint{2.532681in}{1.487937in}}%
\pgfpathlineto{\pgfqpoint{2.556560in}{1.533322in}}%
\pgfpathlineto{\pgfqpoint{2.580439in}{1.580042in}}%
\pgfpathlineto{\pgfqpoint{2.604318in}{1.628096in}}%
\pgfpathlineto{\pgfqpoint{2.628198in}{1.677486in}}%
\pgfpathlineto{\pgfqpoint{2.652077in}{1.728210in}}%
\pgfpathlineto{\pgfqpoint{2.675956in}{1.780269in}}%
\pgfpathlineto{\pgfqpoint{2.699835in}{1.833663in}}%
\pgfpathlineto{\pgfqpoint{2.723714in}{1.888392in}}%
\pgfpathlineto{\pgfqpoint{2.747593in}{1.944456in}}%
\pgfpathlineto{\pgfqpoint{2.771472in}{2.001854in}}%
\pgfpathlineto{\pgfqpoint{2.795352in}{2.060587in}}%
\pgfpathlineto{\pgfqpoint{2.815677in}{2.111717in}}%
\pgfusepath{stroke}%
\end{pgfscope}%
\begin{pgfscope}%
\pgfpathrectangle{\pgfqpoint{0.550713in}{0.398220in}}{\pgfqpoint{2.364035in}{1.703497in}}%
\pgfusepath{clip}%
\pgfsetbuttcap%
\pgfsetroundjoin%
\definecolor{currentfill}{rgb}{0.000000,0.500000,0.000000}%
\pgfsetfillcolor{currentfill}%
\pgfsetlinewidth{1.003750pt}%
\definecolor{currentstroke}{rgb}{0.000000,0.500000,0.000000}%
\pgfsetstrokecolor{currentstroke}%
\pgfsetdash{}{0pt}%
\pgfsys@defobject{currentmarker}{\pgfqpoint{-0.020833in}{-0.020833in}}{\pgfqpoint{0.020833in}{0.020833in}}{%
\pgfpathmoveto{\pgfqpoint{0.000000in}{-0.020833in}}%
\pgfpathcurveto{\pgfqpoint{0.005525in}{-0.020833in}}{\pgfqpoint{0.010825in}{-0.018638in}}{\pgfqpoint{0.014731in}{-0.014731in}}%
\pgfpathcurveto{\pgfqpoint{0.018638in}{-0.010825in}}{\pgfqpoint{0.020833in}{-0.005525in}}{\pgfqpoint{0.020833in}{0.000000in}}%
\pgfpathcurveto{\pgfqpoint{0.020833in}{0.005525in}}{\pgfqpoint{0.018638in}{0.010825in}}{\pgfqpoint{0.014731in}{0.014731in}}%
\pgfpathcurveto{\pgfqpoint{0.010825in}{0.018638in}}{\pgfqpoint{0.005525in}{0.020833in}}{\pgfqpoint{0.000000in}{0.020833in}}%
\pgfpathcurveto{\pgfqpoint{-0.005525in}{0.020833in}}{\pgfqpoint{-0.010825in}{0.018638in}}{\pgfqpoint{-0.014731in}{0.014731in}}%
\pgfpathcurveto{\pgfqpoint{-0.018638in}{0.010825in}}{\pgfqpoint{-0.020833in}{0.005525in}}{\pgfqpoint{-0.020833in}{0.000000in}}%
\pgfpathcurveto{\pgfqpoint{-0.020833in}{-0.005525in}}{\pgfqpoint{-0.018638in}{-0.010825in}}{\pgfqpoint{-0.014731in}{-0.014731in}}%
\pgfpathcurveto{\pgfqpoint{-0.010825in}{-0.018638in}}{\pgfqpoint{-0.005525in}{-0.020833in}}{\pgfqpoint{0.000000in}{-0.020833in}}%
\pgfpathclose%
\pgfusepath{stroke,fill}%
}%
\begin{pgfscope}%
\pgfsys@transformshift{1.215597in}{0.483395in}%
\pgfsys@useobject{currentmarker}{}%
\end{pgfscope}%
\begin{pgfscope}%
\pgfsys@transformshift{1.330516in}{0.483395in}%
\pgfsys@useobject{currentmarker}{}%
\end{pgfscope}%
\begin{pgfscope}%
\pgfsys@transformshift{1.445434in}{0.483395in}%
\pgfsys@useobject{currentmarker}{}%
\end{pgfscope}%
\begin{pgfscope}%
\pgfsys@transformshift{1.560352in}{0.483395in}%
\pgfsys@useobject{currentmarker}{}%
\end{pgfscope}%
\begin{pgfscope}%
\pgfsys@transformshift{1.675271in}{0.483395in}%
\pgfsys@useobject{currentmarker}{}%
\end{pgfscope}%
\begin{pgfscope}%
\pgfsys@transformshift{1.790189in}{0.483395in}%
\pgfsys@useobject{currentmarker}{}%
\end{pgfscope}%
\begin{pgfscope}%
\pgfsys@transformshift{1.905107in}{0.483395in}%
\pgfsys@useobject{currentmarker}{}%
\end{pgfscope}%
\begin{pgfscope}%
\pgfsys@transformshift{2.020026in}{0.483395in}%
\pgfsys@useobject{currentmarker}{}%
\end{pgfscope}%
\begin{pgfscope}%
\pgfsys@transformshift{2.134944in}{0.483395in}%
\pgfsys@useobject{currentmarker}{}%
\end{pgfscope}%
\begin{pgfscope}%
\pgfsys@transformshift{2.249862in}{0.483395in}%
\pgfsys@useobject{currentmarker}{}%
\end{pgfscope}%
\end{pgfscope}%
\begin{pgfscope}%
\pgfpathrectangle{\pgfqpoint{0.550713in}{0.398220in}}{\pgfqpoint{2.364035in}{1.703497in}}%
\pgfusepath{clip}%
\pgfsetrectcap%
\pgfsetroundjoin%
\pgfsetlinewidth{0.752812pt}%
\definecolor{currentstroke}{rgb}{0.631373,0.062745,0.207843}%
\pgfsetstrokecolor{currentstroke}%
\pgfsetdash{}{0pt}%
\pgfpathmoveto{\pgfqpoint{0.550713in}{1.249969in}}%
\pgfpathlineto{\pgfqpoint{0.574592in}{1.249969in}}%
\pgfpathlineto{\pgfqpoint{0.598471in}{1.249969in}}%
\pgfpathlineto{\pgfqpoint{0.622350in}{1.249969in}}%
\pgfpathlineto{\pgfqpoint{0.646229in}{1.249969in}}%
\pgfpathlineto{\pgfqpoint{0.670108in}{1.249969in}}%
\pgfpathlineto{\pgfqpoint{0.693987in}{1.249969in}}%
\pgfpathlineto{\pgfqpoint{0.717867in}{1.249969in}}%
\pgfpathlineto{\pgfqpoint{0.741746in}{1.249969in}}%
\pgfpathlineto{\pgfqpoint{0.765625in}{1.249969in}}%
\pgfpathlineto{\pgfqpoint{0.789504in}{1.249969in}}%
\pgfpathlineto{\pgfqpoint{0.813383in}{1.249969in}}%
\pgfpathlineto{\pgfqpoint{0.837262in}{1.249969in}}%
\pgfpathlineto{\pgfqpoint{0.861141in}{1.249969in}}%
\pgfpathlineto{\pgfqpoint{0.885021in}{1.249969in}}%
\pgfpathlineto{\pgfqpoint{0.908900in}{1.249969in}}%
\pgfpathlineto{\pgfqpoint{0.932779in}{1.249969in}}%
\pgfpathlineto{\pgfqpoint{0.956658in}{1.249969in}}%
\pgfpathlineto{\pgfqpoint{0.980537in}{1.249969in}}%
\pgfpathlineto{\pgfqpoint{1.004416in}{1.249969in}}%
\pgfpathlineto{\pgfqpoint{1.028295in}{1.249969in}}%
\pgfpathlineto{\pgfqpoint{1.052174in}{1.249969in}}%
\pgfpathlineto{\pgfqpoint{1.076054in}{1.249969in}}%
\pgfpathlineto{\pgfqpoint{1.099933in}{1.249969in}}%
\pgfpathlineto{\pgfqpoint{1.123812in}{1.249969in}}%
\pgfpathlineto{\pgfqpoint{1.147691in}{1.249969in}}%
\pgfpathlineto{\pgfqpoint{1.171570in}{1.249969in}}%
\pgfpathlineto{\pgfqpoint{1.195449in}{1.249969in}}%
\pgfpathlineto{\pgfqpoint{1.219328in}{1.249969in}}%
\pgfpathlineto{\pgfqpoint{1.243208in}{1.249969in}}%
\pgfpathlineto{\pgfqpoint{1.267087in}{1.249969in}}%
\pgfpathlineto{\pgfqpoint{1.290966in}{1.249969in}}%
\pgfpathlineto{\pgfqpoint{1.314845in}{1.249969in}}%
\pgfpathlineto{\pgfqpoint{1.338724in}{1.249969in}}%
\pgfpathlineto{\pgfqpoint{1.362603in}{1.249969in}}%
\pgfpathlineto{\pgfqpoint{1.386482in}{1.249969in}}%
\pgfpathlineto{\pgfqpoint{1.410362in}{1.249969in}}%
\pgfpathlineto{\pgfqpoint{1.434241in}{1.249969in}}%
\pgfpathlineto{\pgfqpoint{1.458120in}{1.249969in}}%
\pgfpathlineto{\pgfqpoint{1.481999in}{1.249969in}}%
\pgfpathlineto{\pgfqpoint{1.505878in}{1.249969in}}%
\pgfpathlineto{\pgfqpoint{1.529757in}{1.249969in}}%
\pgfpathlineto{\pgfqpoint{1.553636in}{1.249969in}}%
\pgfpathlineto{\pgfqpoint{1.577516in}{1.249969in}}%
\pgfpathlineto{\pgfqpoint{1.601395in}{1.249969in}}%
\pgfpathlineto{\pgfqpoint{1.625274in}{1.249969in}}%
\pgfpathlineto{\pgfqpoint{1.649153in}{1.249969in}}%
\pgfpathlineto{\pgfqpoint{1.673032in}{1.249969in}}%
\pgfpathlineto{\pgfqpoint{1.696911in}{1.249969in}}%
\pgfpathlineto{\pgfqpoint{1.720790in}{1.249969in}}%
\pgfpathlineto{\pgfqpoint{1.744669in}{1.249969in}}%
\pgfpathlineto{\pgfqpoint{1.768549in}{1.249969in}}%
\pgfpathlineto{\pgfqpoint{1.792428in}{1.249969in}}%
\pgfpathlineto{\pgfqpoint{1.816307in}{1.249969in}}%
\pgfpathlineto{\pgfqpoint{1.840186in}{1.249969in}}%
\pgfpathlineto{\pgfqpoint{1.864065in}{1.249969in}}%
\pgfpathlineto{\pgfqpoint{1.887944in}{1.249969in}}%
\pgfpathlineto{\pgfqpoint{1.911823in}{1.249969in}}%
\pgfpathlineto{\pgfqpoint{1.935703in}{1.249969in}}%
\pgfpathlineto{\pgfqpoint{1.959582in}{1.249969in}}%
\pgfpathlineto{\pgfqpoint{1.983461in}{1.249969in}}%
\pgfpathlineto{\pgfqpoint{2.007340in}{1.249969in}}%
\pgfpathlineto{\pgfqpoint{2.031219in}{1.249969in}}%
\pgfpathlineto{\pgfqpoint{2.055098in}{1.249969in}}%
\pgfpathlineto{\pgfqpoint{2.078977in}{1.249969in}}%
\pgfpathlineto{\pgfqpoint{2.102857in}{1.249969in}}%
\pgfpathlineto{\pgfqpoint{2.126736in}{1.249969in}}%
\pgfpathlineto{\pgfqpoint{2.150615in}{1.249969in}}%
\pgfpathlineto{\pgfqpoint{2.174494in}{1.249969in}}%
\pgfpathlineto{\pgfqpoint{2.198373in}{1.249969in}}%
\pgfpathlineto{\pgfqpoint{2.222252in}{1.249969in}}%
\pgfpathlineto{\pgfqpoint{2.246131in}{1.249969in}}%
\pgfpathlineto{\pgfqpoint{2.270010in}{1.249969in}}%
\pgfpathlineto{\pgfqpoint{2.293890in}{1.249969in}}%
\pgfpathlineto{\pgfqpoint{2.317769in}{1.249969in}}%
\pgfpathlineto{\pgfqpoint{2.341648in}{1.249969in}}%
\pgfpathlineto{\pgfqpoint{2.365527in}{1.249969in}}%
\pgfpathlineto{\pgfqpoint{2.389406in}{1.249969in}}%
\pgfpathlineto{\pgfqpoint{2.413285in}{1.249969in}}%
\pgfpathlineto{\pgfqpoint{2.437164in}{1.249969in}}%
\pgfpathlineto{\pgfqpoint{2.461044in}{1.249969in}}%
\pgfpathlineto{\pgfqpoint{2.484923in}{1.249969in}}%
\pgfpathlineto{\pgfqpoint{2.508802in}{1.249969in}}%
\pgfpathlineto{\pgfqpoint{2.532681in}{1.249969in}}%
\pgfpathlineto{\pgfqpoint{2.556560in}{1.249969in}}%
\pgfpathlineto{\pgfqpoint{2.580439in}{1.249969in}}%
\pgfpathlineto{\pgfqpoint{2.604318in}{1.249969in}}%
\pgfpathlineto{\pgfqpoint{2.628198in}{1.249969in}}%
\pgfpathlineto{\pgfqpoint{2.652077in}{1.249969in}}%
\pgfpathlineto{\pgfqpoint{2.675956in}{1.249969in}}%
\pgfpathlineto{\pgfqpoint{2.699835in}{1.249969in}}%
\pgfpathlineto{\pgfqpoint{2.723714in}{1.249969in}}%
\pgfpathlineto{\pgfqpoint{2.747593in}{1.249969in}}%
\pgfpathlineto{\pgfqpoint{2.771472in}{1.249969in}}%
\pgfpathlineto{\pgfqpoint{2.795352in}{1.249969in}}%
\pgfpathlineto{\pgfqpoint{2.819231in}{1.249969in}}%
\pgfpathlineto{\pgfqpoint{2.843110in}{1.249969in}}%
\pgfpathlineto{\pgfqpoint{2.866989in}{1.249969in}}%
\pgfpathlineto{\pgfqpoint{2.890868in}{1.249969in}}%
\pgfpathlineto{\pgfqpoint{2.914747in}{1.249969in}}%
\pgfusepath{stroke}%
\end{pgfscope}%
\begin{pgfscope}%
\pgfpathrectangle{\pgfqpoint{0.550713in}{0.398220in}}{\pgfqpoint{2.364035in}{1.703497in}}%
\pgfusepath{clip}%
\pgfsetrectcap%
\pgfsetroundjoin%
\pgfsetlinewidth{0.501875pt}%
\definecolor{currentstroke}{rgb}{0.713725,0.321569,0.337255}%
\pgfsetstrokecolor{currentstroke}%
\pgfsetdash{}{0pt}%
\pgfpathmoveto{\pgfqpoint{0.550713in}{1.171769in}}%
\pgfpathlineto{\pgfqpoint{0.574592in}{1.225987in}}%
\pgfpathlineto{\pgfqpoint{0.598471in}{1.275650in}}%
\pgfpathlineto{\pgfqpoint{0.622350in}{1.315028in}}%
\pgfpathlineto{\pgfqpoint{0.646229in}{1.340845in}}%
\pgfpathlineto{\pgfqpoint{0.670108in}{1.352730in}}%
\pgfpathlineto{\pgfqpoint{0.693987in}{1.353009in}}%
\pgfpathlineto{\pgfqpoint{0.717867in}{1.345952in}}%
\pgfpathlineto{\pgfqpoint{0.741746in}{1.336667in}}%
\pgfpathlineto{\pgfqpoint{0.765625in}{1.329898in}}%
\pgfpathlineto{\pgfqpoint{0.789504in}{1.328982in}}%
\pgfpathlineto{\pgfqpoint{0.813383in}{1.335152in}}%
\pgfpathlineto{\pgfqpoint{0.837262in}{1.347282in}}%
\pgfpathlineto{\pgfqpoint{0.861141in}{1.362103in}}%
\pgfpathlineto{\pgfqpoint{0.885021in}{1.374830in}}%
\pgfpathlineto{\pgfqpoint{0.908900in}{1.380103in}}%
\pgfpathlineto{\pgfqpoint{0.932779in}{1.373072in}}%
\pgfpathlineto{\pgfqpoint{0.956658in}{1.350426in}}%
\pgfpathlineto{\pgfqpoint{0.980537in}{1.311156in}}%
\pgfpathlineto{\pgfqpoint{1.004416in}{1.256865in}}%
\pgfpathlineto{\pgfqpoint{1.028295in}{1.191544in}}%
\pgfpathlineto{\pgfqpoint{1.052174in}{1.120842in}}%
\pgfpathlineto{\pgfqpoint{1.076054in}{1.051000in}}%
\pgfpathlineto{\pgfqpoint{1.099933in}{0.987707in}}%
\pgfpathlineto{\pgfqpoint{1.123812in}{0.935136in}}%
\pgfpathlineto{\pgfqpoint{1.147691in}{0.895372in}}%
\pgfpathlineto{\pgfqpoint{1.171570in}{0.868333in}}%
\pgfpathlineto{\pgfqpoint{1.195449in}{0.852148in}}%
\pgfpathlineto{\pgfqpoint{1.219328in}{0.843883in}}%
\pgfpathlineto{\pgfqpoint{1.243208in}{0.840424in}}%
\pgfpathlineto{\pgfqpoint{1.267087in}{0.839338in}}%
\pgfpathlineto{\pgfqpoint{1.290966in}{0.839538in}}%
\pgfpathlineto{\pgfqpoint{1.314845in}{0.841590in}}%
\pgfpathlineto{\pgfqpoint{1.338724in}{0.847617in}}%
\pgfpathlineto{\pgfqpoint{1.362603in}{0.860770in}}%
\pgfpathlineto{\pgfqpoint{1.386482in}{0.884383in}}%
\pgfpathlineto{\pgfqpoint{1.410362in}{0.920982in}}%
\pgfpathlineto{\pgfqpoint{1.434241in}{0.971380in}}%
\pgfpathlineto{\pgfqpoint{1.458120in}{1.034090in}}%
\pgfpathlineto{\pgfqpoint{1.481999in}{1.105205in}}%
\pgfpathlineto{\pgfqpoint{1.505878in}{1.178815in}}%
\pgfpathlineto{\pgfqpoint{1.529757in}{1.247895in}}%
\pgfpathlineto{\pgfqpoint{1.553636in}{1.305494in}}%
\pgfpathlineto{\pgfqpoint{1.577516in}{1.345987in}}%
\pgfpathlineto{\pgfqpoint{1.601395in}{1.366105in}}%
\pgfpathlineto{\pgfqpoint{1.625274in}{1.365533in}}%
\pgfpathlineto{\pgfqpoint{1.649153in}{1.346917in}}%
\pgfpathlineto{\pgfqpoint{1.673032in}{1.315284in}}%
\pgfpathlineto{\pgfqpoint{1.696911in}{1.277013in}}%
\pgfpathlineto{\pgfqpoint{1.720790in}{1.238608in}}%
\pgfpathlineto{\pgfqpoint{1.744669in}{1.205571in}}%
\pgfpathlineto{\pgfqpoint{1.768549in}{1.181596in}}%
\pgfpathlineto{\pgfqpoint{1.792428in}{1.168239in}}%
\pgfpathlineto{\pgfqpoint{1.816307in}{1.165031in}}%
\pgfpathlineto{\pgfqpoint{1.840186in}{1.169937in}}%
\pgfpathlineto{\pgfqpoint{1.864065in}{1.180008in}}%
\pgfpathlineto{\pgfqpoint{1.887944in}{1.192083in}}%
\pgfpathlineto{\pgfqpoint{1.911823in}{1.203436in}}%
\pgfpathlineto{\pgfqpoint{1.935703in}{1.212280in}}%
\pgfpathlineto{\pgfqpoint{1.959582in}{1.218030in}}%
\pgfpathlineto{\pgfqpoint{1.983461in}{1.221254in}}%
\pgfpathlineto{\pgfqpoint{2.007340in}{1.223274in}}%
\pgfpathlineto{\pgfqpoint{2.031219in}{1.225499in}}%
\pgfpathlineto{\pgfqpoint{2.055098in}{1.228676in}}%
\pgfpathlineto{\pgfqpoint{2.078977in}{1.232328in}}%
\pgfpathlineto{\pgfqpoint{2.102857in}{1.234611in}}%
\pgfpathlineto{\pgfqpoint{2.126736in}{1.232685in}}%
\pgfpathlineto{\pgfqpoint{2.150615in}{1.223518in}}%
\pgfpathlineto{\pgfqpoint{2.174494in}{1.204847in}}%
\pgfpathlineto{\pgfqpoint{2.198373in}{1.176000in}}%
\pgfpathlineto{\pgfqpoint{2.222252in}{1.138325in}}%
\pgfpathlineto{\pgfqpoint{2.246131in}{1.095100in}}%
\pgfpathlineto{\pgfqpoint{2.270010in}{1.051001in}}%
\pgfpathlineto{\pgfqpoint{2.293890in}{1.011274in}}%
\pgfpathlineto{\pgfqpoint{2.317769in}{0.980825in}}%
\pgfpathlineto{\pgfqpoint{2.341648in}{0.963410in}}%
\pgfpathlineto{\pgfqpoint{2.365527in}{0.961057in}}%
\pgfpathlineto{\pgfqpoint{2.389406in}{0.973777in}}%
\pgfpathlineto{\pgfqpoint{2.413285in}{0.999596in}}%
\pgfpathlineto{\pgfqpoint{2.437164in}{1.034868in}}%
\pgfpathlineto{\pgfqpoint{2.461044in}{1.074831in}}%
\pgfpathlineto{\pgfqpoint{2.484923in}{1.114305in}}%
\pgfpathlineto{\pgfqpoint{2.508802in}{1.148440in}}%
\pgfpathlineto{\pgfqpoint{2.532681in}{1.173386in}}%
\pgfpathlineto{\pgfqpoint{2.556560in}{1.186789in}}%
\pgfpathlineto{\pgfqpoint{2.580439in}{1.188029in}}%
\pgfpathlineto{\pgfqpoint{2.604318in}{1.178169in}}%
\pgfpathlineto{\pgfqpoint{2.628198in}{1.159648in}}%
\pgfpathlineto{\pgfqpoint{2.652077in}{1.135790in}}%
\pgfpathlineto{\pgfqpoint{2.675956in}{1.110251in}}%
\pgfpathlineto{\pgfqpoint{2.699835in}{1.086519in}}%
\pgfpathlineto{\pgfqpoint{2.723714in}{1.067528in}}%
\pgfpathlineto{\pgfqpoint{2.747593in}{1.055437in}}%
\pgfpathlineto{\pgfqpoint{2.771472in}{1.051517in}}%
\pgfpathlineto{\pgfqpoint{2.795352in}{1.056126in}}%
\pgfpathlineto{\pgfqpoint{2.819231in}{1.068719in}}%
\pgfpathlineto{\pgfqpoint{2.843110in}{1.087879in}}%
\pgfpathlineto{\pgfqpoint{2.866989in}{1.111378in}}%
\pgfpathlineto{\pgfqpoint{2.890868in}{1.136323in}}%
\pgfpathlineto{\pgfqpoint{2.914747in}{1.159399in}}%
\pgfusepath{stroke}%
\end{pgfscope}%
\begin{pgfscope}%
\pgfpathrectangle{\pgfqpoint{0.550713in}{0.398220in}}{\pgfqpoint{2.364035in}{1.703497in}}%
\pgfusepath{clip}%
\pgfsetrectcap%
\pgfsetroundjoin%
\pgfsetlinewidth{0.501875pt}%
\definecolor{currentstroke}{rgb}{0.713725,0.321569,0.337255}%
\pgfsetstrokecolor{currentstroke}%
\pgfsetdash{}{0pt}%
\pgfpathmoveto{\pgfqpoint{0.550713in}{1.243729in}}%
\pgfpathlineto{\pgfqpoint{0.574592in}{1.211354in}}%
\pgfpathlineto{\pgfqpoint{0.598471in}{1.191923in}}%
\pgfpathlineto{\pgfqpoint{0.622350in}{1.187652in}}%
\pgfpathlineto{\pgfqpoint{0.646229in}{1.197096in}}%
\pgfpathlineto{\pgfqpoint{0.670108in}{1.215390in}}%
\pgfpathlineto{\pgfqpoint{0.693987in}{1.235514in}}%
\pgfpathlineto{\pgfqpoint{0.717867in}{1.250213in}}%
\pgfpathlineto{\pgfqpoint{0.741746in}{1.253981in}}%
\pgfpathlineto{\pgfqpoint{0.765625in}{1.244497in}}%
\pgfpathlineto{\pgfqpoint{0.789504in}{1.223130in}}%
\pgfpathlineto{\pgfqpoint{0.813383in}{1.194441in}}%
\pgfpathlineto{\pgfqpoint{0.837262in}{1.164899in}}%
\pgfpathlineto{\pgfqpoint{0.861141in}{1.141252in}}%
\pgfpathlineto{\pgfqpoint{0.885021in}{1.128973in}}%
\pgfpathlineto{\pgfqpoint{0.908900in}{1.131144in}}%
\pgfpathlineto{\pgfqpoint{0.932779in}{1.147931in}}%
\pgfpathlineto{\pgfqpoint{0.956658in}{1.176698in}}%
\pgfpathlineto{\pgfqpoint{0.980537in}{1.212653in}}%
\pgfpathlineto{\pgfqpoint{1.004416in}{1.249847in}}%
\pgfpathlineto{\pgfqpoint{1.028295in}{1.282312in}}%
\pgfpathlineto{\pgfqpoint{1.052174in}{1.305101in}}%
\pgfpathlineto{\pgfqpoint{1.076054in}{1.315047in}}%
\pgfpathlineto{\pgfqpoint{1.099933in}{1.311119in}}%
\pgfpathlineto{\pgfqpoint{1.123812in}{1.294364in}}%
\pgfpathlineto{\pgfqpoint{1.147691in}{1.267518in}}%
\pgfpathlineto{\pgfqpoint{1.171570in}{1.234425in}}%
\pgfpathlineto{\pgfqpoint{1.195449in}{1.199400in}}%
\pgfpathlineto{\pgfqpoint{1.219328in}{1.166642in}}%
\pgfpathlineto{\pgfqpoint{1.243208in}{1.139750in}}%
\pgfpathlineto{\pgfqpoint{1.267087in}{1.121382in}}%
\pgfpathlineto{\pgfqpoint{1.290966in}{1.113038in}}%
\pgfpathlineto{\pgfqpoint{1.314845in}{1.115010in}}%
\pgfpathlineto{\pgfqpoint{1.338724in}{1.126465in}}%
\pgfpathlineto{\pgfqpoint{1.362603in}{1.145681in}}%
\pgfpathlineto{\pgfqpoint{1.386482in}{1.170378in}}%
\pgfpathlineto{\pgfqpoint{1.410362in}{1.198112in}}%
\pgfpathlineto{\pgfqpoint{1.434241in}{1.226659in}}%
\pgfpathlineto{\pgfqpoint{1.458120in}{1.254305in}}%
\pgfpathlineto{\pgfqpoint{1.481999in}{1.279979in}}%
\pgfpathlineto{\pgfqpoint{1.505878in}{1.303172in}}%
\pgfpathlineto{\pgfqpoint{1.529757in}{1.323666in}}%
\pgfpathlineto{\pgfqpoint{1.553636in}{1.341143in}}%
\pgfpathlineto{\pgfqpoint{1.577516in}{1.354845in}}%
\pgfpathlineto{\pgfqpoint{1.601395in}{1.363418in}}%
\pgfpathlineto{\pgfqpoint{1.625274in}{1.365057in}}%
\pgfpathlineto{\pgfqpoint{1.649153in}{1.357932in}}%
\pgfpathlineto{\pgfqpoint{1.673032in}{1.340796in}}%
\pgfpathlineto{\pgfqpoint{1.696911in}{1.313580in}}%
\pgfpathlineto{\pgfqpoint{1.720790in}{1.277779in}}%
\pgfpathlineto{\pgfqpoint{1.744669in}{1.236510in}}%
\pgfpathlineto{\pgfqpoint{1.768549in}{1.194171in}}%
\pgfpathlineto{\pgfqpoint{1.792428in}{1.155771in}}%
\pgfpathlineto{\pgfqpoint{1.816307in}{1.126069in}}%
\pgfpathlineto{\pgfqpoint{1.840186in}{1.108737in}}%
\pgfpathlineto{\pgfqpoint{1.864065in}{1.105725in}}%
\pgfpathlineto{\pgfqpoint{1.887944in}{1.117019in}}%
\pgfpathlineto{\pgfqpoint{1.911823in}{1.140786in}}%
\pgfpathlineto{\pgfqpoint{1.935703in}{1.173869in}}%
\pgfpathlineto{\pgfqpoint{1.959582in}{1.212427in}}%
\pgfpathlineto{\pgfqpoint{1.983461in}{1.252552in}}%
\pgfpathlineto{\pgfqpoint{2.007340in}{1.290720in}}%
\pgfpathlineto{\pgfqpoint{2.031219in}{1.324026in}}%
\pgfpathlineto{\pgfqpoint{2.055098in}{1.350261in}}%
\pgfpathlineto{\pgfqpoint{2.078977in}{1.367889in}}%
\pgfpathlineto{\pgfqpoint{2.102857in}{1.376011in}}%
\pgfpathlineto{\pgfqpoint{2.126736in}{1.374329in}}%
\pgfpathlineto{\pgfqpoint{2.150615in}{1.363122in}}%
\pgfpathlineto{\pgfqpoint{2.174494in}{1.343202in}}%
\pgfpathlineto{\pgfqpoint{2.198373in}{1.315855in}}%
\pgfpathlineto{\pgfqpoint{2.222252in}{1.282759in}}%
\pgfpathlineto{\pgfqpoint{2.246131in}{1.245926in}}%
\pgfpathlineto{\pgfqpoint{2.270010in}{1.207655in}}%
\pgfpathlineto{\pgfqpoint{2.293890in}{1.170519in}}%
\pgfpathlineto{\pgfqpoint{2.317769in}{1.137323in}}%
\pgfpathlineto{\pgfqpoint{2.341648in}{1.110996in}}%
\pgfpathlineto{\pgfqpoint{2.365527in}{1.094354in}}%
\pgfpathlineto{\pgfqpoint{2.389406in}{1.089719in}}%
\pgfpathlineto{\pgfqpoint{2.413285in}{1.098442in}}%
\pgfpathlineto{\pgfqpoint{2.437164in}{1.120436in}}%
\pgfpathlineto{\pgfqpoint{2.461044in}{1.153885in}}%
\pgfpathlineto{\pgfqpoint{2.484923in}{1.195250in}}%
\pgfpathlineto{\pgfqpoint{2.508802in}{1.239648in}}%
\pgfpathlineto{\pgfqpoint{2.532681in}{1.281547in}}%
\pgfpathlineto{\pgfqpoint{2.556560in}{1.315629in}}%
\pgfpathlineto{\pgfqpoint{2.580439in}{1.337638in}}%
\pgfpathlineto{\pgfqpoint{2.604318in}{1.344989in}}%
\pgfpathlineto{\pgfqpoint{2.628198in}{1.337061in}}%
\pgfpathlineto{\pgfqpoint{2.652077in}{1.315116in}}%
\pgfpathlineto{\pgfqpoint{2.675956in}{1.281937in}}%
\pgfpathlineto{\pgfqpoint{2.699835in}{1.241305in}}%
\pgfpathlineto{\pgfqpoint{2.723714in}{1.197457in}}%
\pgfpathlineto{\pgfqpoint{2.747593in}{1.154645in}}%
\pgfpathlineto{\pgfqpoint{2.771472in}{1.116809in}}%
\pgfpathlineto{\pgfqpoint{2.795352in}{1.087362in}}%
\pgfpathlineto{\pgfqpoint{2.819231in}{1.068970in}}%
\pgfpathlineto{\pgfqpoint{2.843110in}{1.063308in}}%
\pgfpathlineto{\pgfqpoint{2.866989in}{1.070792in}}%
\pgfpathlineto{\pgfqpoint{2.890868in}{1.090386in}}%
\pgfpathlineto{\pgfqpoint{2.914747in}{1.119620in}}%
\pgfusepath{stroke}%
\end{pgfscope}%
\begin{pgfscope}%
\pgfpathrectangle{\pgfqpoint{0.550713in}{0.398220in}}{\pgfqpoint{2.364035in}{1.703497in}}%
\pgfusepath{clip}%
\pgfsetrectcap%
\pgfsetroundjoin%
\pgfsetlinewidth{0.501875pt}%
\definecolor{currentstroke}{rgb}{0.713725,0.321569,0.337255}%
\pgfsetstrokecolor{currentstroke}%
\pgfsetdash{}{0pt}%
\pgfpathmoveto{\pgfqpoint{0.550713in}{1.179410in}}%
\pgfpathlineto{\pgfqpoint{0.574592in}{1.181412in}}%
\pgfpathlineto{\pgfqpoint{0.598471in}{1.179075in}}%
\pgfpathlineto{\pgfqpoint{0.622350in}{1.173459in}}%
\pgfpathlineto{\pgfqpoint{0.646229in}{1.166413in}}%
\pgfpathlineto{\pgfqpoint{0.670108in}{1.160285in}}%
\pgfpathlineto{\pgfqpoint{0.693987in}{1.157566in}}%
\pgfpathlineto{\pgfqpoint{0.717867in}{1.160498in}}%
\pgfpathlineto{\pgfqpoint{0.741746in}{1.170692in}}%
\pgfpathlineto{\pgfqpoint{0.765625in}{1.188750in}}%
\pgfpathlineto{\pgfqpoint{0.789504in}{1.213953in}}%
\pgfpathlineto{\pgfqpoint{0.813383in}{1.244128in}}%
\pgfpathlineto{\pgfqpoint{0.837262in}{1.275813in}}%
\pgfpathlineto{\pgfqpoint{0.861141in}{1.304786in}}%
\pgfpathlineto{\pgfqpoint{0.885021in}{1.326918in}}%
\pgfpathlineto{\pgfqpoint{0.908900in}{1.339127in}}%
\pgfpathlineto{\pgfqpoint{0.932779in}{1.340162in}}%
\pgfpathlineto{\pgfqpoint{0.956658in}{1.330953in}}%
\pgfpathlineto{\pgfqpoint{0.980537in}{1.314420in}}%
\pgfpathlineto{\pgfqpoint{1.004416in}{1.294789in}}%
\pgfpathlineto{\pgfqpoint{1.028295in}{1.276619in}}%
\pgfpathlineto{\pgfqpoint{1.052174in}{1.263824in}}%
\pgfpathlineto{\pgfqpoint{1.076054in}{1.258905in}}%
\pgfpathlineto{\pgfqpoint{1.099933in}{1.262583in}}%
\pgfpathlineto{\pgfqpoint{1.123812in}{1.273854in}}%
\pgfpathlineto{\pgfqpoint{1.147691in}{1.290426in}}%
\pgfpathlineto{\pgfqpoint{1.171570in}{1.309393in}}%
\pgfpathlineto{\pgfqpoint{1.195449in}{1.327971in}}%
\pgfpathlineto{\pgfqpoint{1.219328in}{1.344089in}}%
\pgfpathlineto{\pgfqpoint{1.243208in}{1.356688in}}%
\pgfpathlineto{\pgfqpoint{1.267087in}{1.365678in}}%
\pgfpathlineto{\pgfqpoint{1.290966in}{1.371603in}}%
\pgfpathlineto{\pgfqpoint{1.314845in}{1.375158in}}%
\pgfpathlineto{\pgfqpoint{1.338724in}{1.376772in}}%
\pgfpathlineto{\pgfqpoint{1.362603in}{1.376389in}}%
\pgfpathlineto{\pgfqpoint{1.386482in}{1.373525in}}%
\pgfpathlineto{\pgfqpoint{1.410362in}{1.367526in}}%
\pgfpathlineto{\pgfqpoint{1.434241in}{1.357910in}}%
\pgfpathlineto{\pgfqpoint{1.458120in}{1.344655in}}%
\pgfpathlineto{\pgfqpoint{1.481999in}{1.328330in}}%
\pgfpathlineto{\pgfqpoint{1.505878in}{1.310060in}}%
\pgfpathlineto{\pgfqpoint{1.529757in}{1.291337in}}%
\pgfpathlineto{\pgfqpoint{1.553636in}{1.273768in}}%
\pgfpathlineto{\pgfqpoint{1.577516in}{1.258809in}}%
\pgfpathlineto{\pgfqpoint{1.601395in}{1.247550in}}%
\pgfpathlineto{\pgfqpoint{1.625274in}{1.240557in}}%
\pgfpathlineto{\pgfqpoint{1.649153in}{1.237807in}}%
\pgfpathlineto{\pgfqpoint{1.673032in}{1.238694in}}%
\pgfpathlineto{\pgfqpoint{1.696911in}{1.242105in}}%
\pgfpathlineto{\pgfqpoint{1.720790in}{1.246561in}}%
\pgfpathlineto{\pgfqpoint{1.744669in}{1.250412in}}%
\pgfpathlineto{\pgfqpoint{1.768549in}{1.252067in}}%
\pgfpathlineto{\pgfqpoint{1.792428in}{1.250272in}}%
\pgfpathlineto{\pgfqpoint{1.816307in}{1.244379in}}%
\pgfpathlineto{\pgfqpoint{1.840186in}{1.234603in}}%
\pgfpathlineto{\pgfqpoint{1.864065in}{1.222167in}}%
\pgfpathlineto{\pgfqpoint{1.887944in}{1.209305in}}%
\pgfpathlineto{\pgfqpoint{1.911823in}{1.199001in}}%
\pgfpathlineto{\pgfqpoint{1.935703in}{1.194493in}}%
\pgfpathlineto{\pgfqpoint{1.959582in}{1.198567in}}%
\pgfpathlineto{\pgfqpoint{1.983461in}{1.212817in}}%
\pgfpathlineto{\pgfqpoint{2.007340in}{1.237099in}}%
\pgfpathlineto{\pgfqpoint{2.031219in}{1.269384in}}%
\pgfpathlineto{\pgfqpoint{2.055098in}{1.306091in}}%
\pgfpathlineto{\pgfqpoint{2.078977in}{1.342848in}}%
\pgfpathlineto{\pgfqpoint{2.102857in}{1.375420in}}%
\pgfpathlineto{\pgfqpoint{2.126736in}{1.400529in}}%
\pgfpathlineto{\pgfqpoint{2.150615in}{1.416302in}}%
\pgfpathlineto{\pgfqpoint{2.174494in}{1.422275in}}%
\pgfpathlineto{\pgfqpoint{2.198373in}{1.419003in}}%
\pgfpathlineto{\pgfqpoint{2.222252in}{1.407512in}}%
\pgfpathlineto{\pgfqpoint{2.246131in}{1.388826in}}%
\pgfpathlineto{\pgfqpoint{2.270010in}{1.363765in}}%
\pgfpathlineto{\pgfqpoint{2.293890in}{1.333065in}}%
\pgfpathlineto{\pgfqpoint{2.317769in}{1.297750in}}%
\pgfpathlineto{\pgfqpoint{2.341648in}{1.259559in}}%
\pgfpathlineto{\pgfqpoint{2.365527in}{1.221214in}}%
\pgfpathlineto{\pgfqpoint{2.389406in}{1.186362in}}%
\pgfpathlineto{\pgfqpoint{2.413285in}{1.159131in}}%
\pgfpathlineto{\pgfqpoint{2.437164in}{1.143393in}}%
\pgfpathlineto{\pgfqpoint{2.461044in}{1.141909in}}%
\pgfpathlineto{\pgfqpoint{2.484923in}{1.155631in}}%
\pgfpathlineto{\pgfqpoint{2.508802in}{1.183352in}}%
\pgfpathlineto{\pgfqpoint{2.532681in}{1.221851in}}%
\pgfpathlineto{\pgfqpoint{2.556560in}{1.266496in}}%
\pgfpathlineto{\pgfqpoint{2.580439in}{1.312150in}}%
\pgfpathlineto{\pgfqpoint{2.604318in}{1.354122in}}%
\pgfpathlineto{\pgfqpoint{2.628198in}{1.388882in}}%
\pgfpathlineto{\pgfqpoint{2.652077in}{1.414364in}}%
\pgfpathlineto{\pgfqpoint{2.675956in}{1.429815in}}%
\pgfpathlineto{\pgfqpoint{2.699835in}{1.435351in}}%
\pgfpathlineto{\pgfqpoint{2.723714in}{1.431463in}}%
\pgfpathlineto{\pgfqpoint{2.747593in}{1.418756in}}%
\pgfpathlineto{\pgfqpoint{2.771472in}{1.398030in}}%
\pgfpathlineto{\pgfqpoint{2.795352in}{1.370682in}}%
\pgfpathlineto{\pgfqpoint{2.819231in}{1.339152in}}%
\pgfpathlineto{\pgfqpoint{2.843110in}{1.307131in}}%
\pgfpathlineto{\pgfqpoint{2.866989in}{1.279282in}}%
\pgfpathlineto{\pgfqpoint{2.890868in}{1.260459in}}%
\pgfpathlineto{\pgfqpoint{2.914747in}{1.254610in}}%
\pgfusepath{stroke}%
\end{pgfscope}%
\begin{pgfscope}%
\pgfpathrectangle{\pgfqpoint{0.550713in}{0.398220in}}{\pgfqpoint{2.364035in}{1.703497in}}%
\pgfusepath{clip}%
\pgfsetrectcap%
\pgfsetroundjoin%
\pgfsetlinewidth{0.752812pt}%
\definecolor{currentstroke}{rgb}{0.000000,0.329412,0.623529}%
\pgfsetstrokecolor{currentstroke}%
\pgfsetdash{}{0pt}%
\pgfpathmoveto{\pgfqpoint{0.550713in}{1.249954in}}%
\pgfpathlineto{\pgfqpoint{0.574592in}{1.250268in}}%
\pgfpathlineto{\pgfqpoint{0.598471in}{1.250705in}}%
\pgfpathlineto{\pgfqpoint{0.622350in}{1.251353in}}%
\pgfpathlineto{\pgfqpoint{0.646229in}{1.252356in}}%
\pgfpathlineto{\pgfqpoint{0.670108in}{1.253931in}}%
\pgfpathlineto{\pgfqpoint{0.693987in}{1.256380in}}%
\pgfpathlineto{\pgfqpoint{0.717867in}{1.260104in}}%
\pgfpathlineto{\pgfqpoint{0.741746in}{1.265596in}}%
\pgfpathlineto{\pgfqpoint{0.765625in}{1.273427in}}%
\pgfpathlineto{\pgfqpoint{0.789504in}{1.284195in}}%
\pgfpathlineto{\pgfqpoint{0.813383in}{1.298454in}}%
\pgfpathlineto{\pgfqpoint{0.837262in}{1.316600in}}%
\pgfpathlineto{\pgfqpoint{0.861141in}{1.338748in}}%
\pgfpathlineto{\pgfqpoint{0.885021in}{1.364589in}}%
\pgfpathlineto{\pgfqpoint{0.908900in}{1.393286in}}%
\pgfpathlineto{\pgfqpoint{0.932779in}{1.423413in}}%
\pgfpathlineto{\pgfqpoint{0.956658in}{1.452992in}}%
\pgfpathlineto{\pgfqpoint{0.980537in}{1.479641in}}%
\pgfpathlineto{\pgfqpoint{1.004416in}{1.500826in}}%
\pgfpathlineto{\pgfqpoint{1.028295in}{1.514201in}}%
\pgfpathlineto{\pgfqpoint{1.052174in}{1.517969in}}%
\pgfpathlineto{\pgfqpoint{1.076054in}{1.511199in}}%
\pgfpathlineto{\pgfqpoint{1.099933in}{1.494029in}}%
\pgfpathlineto{\pgfqpoint{1.123812in}{1.467672in}}%
\pgfpathlineto{\pgfqpoint{1.147691in}{1.434249in}}%
\pgfpathlineto{\pgfqpoint{1.171570in}{1.396437in}}%
\pgfpathlineto{\pgfqpoint{1.195449in}{1.357037in}}%
\pgfpathlineto{\pgfqpoint{1.219328in}{1.318528in}}%
\pgfpathlineto{\pgfqpoint{1.243208in}{1.282730in}}%
\pgfpathlineto{\pgfqpoint{1.267087in}{1.250637in}}%
\pgfpathlineto{\pgfqpoint{1.290966in}{1.222433in}}%
\pgfpathlineto{\pgfqpoint{1.314845in}{1.197690in}}%
\pgfpathlineto{\pgfqpoint{1.338724in}{1.175648in}}%
\pgfpathlineto{\pgfqpoint{1.362603in}{1.155516in}}%
\pgfpathlineto{\pgfqpoint{1.386482in}{1.136701in}}%
\pgfpathlineto{\pgfqpoint{1.410362in}{1.118922in}}%
\pgfpathlineto{\pgfqpoint{1.434241in}{1.102197in}}%
\pgfpathlineto{\pgfqpoint{1.458120in}{1.086736in}}%
\pgfpathlineto{\pgfqpoint{1.481999in}{1.072798in}}%
\pgfpathlineto{\pgfqpoint{1.505878in}{1.060556in}}%
\pgfpathlineto{\pgfqpoint{1.529757in}{1.050024in}}%
\pgfpathlineto{\pgfqpoint{1.553636in}{1.041063in}}%
\pgfpathlineto{\pgfqpoint{1.577516in}{1.033432in}}%
\pgfpathlineto{\pgfqpoint{1.601395in}{1.026887in}}%
\pgfpathlineto{\pgfqpoint{1.625274in}{1.021261in}}%
\pgfpathlineto{\pgfqpoint{1.649153in}{1.016511in}}%
\pgfpathlineto{\pgfqpoint{1.673032in}{1.012720in}}%
\pgfpathlineto{\pgfqpoint{1.696911in}{1.010048in}}%
\pgfpathlineto{\pgfqpoint{1.720790in}{1.008663in}}%
\pgfpathlineto{\pgfqpoint{1.744669in}{1.008668in}}%
\pgfpathlineto{\pgfqpoint{1.768549in}{1.010067in}}%
\pgfpathlineto{\pgfqpoint{1.792428in}{1.012756in}}%
\pgfpathlineto{\pgfqpoint{1.816307in}{1.016567in}}%
\pgfpathlineto{\pgfqpoint{1.840186in}{1.021340in}}%
\pgfpathlineto{\pgfqpoint{1.864065in}{1.026991in}}%
\pgfpathlineto{\pgfqpoint{1.887944in}{1.033562in}}%
\pgfpathlineto{\pgfqpoint{1.911823in}{1.041219in}}%
\pgfpathlineto{\pgfqpoint{1.935703in}{1.050206in}}%
\pgfpathlineto{\pgfqpoint{1.959582in}{1.060766in}}%
\pgfpathlineto{\pgfqpoint{1.983461in}{1.073042in}}%
\pgfpathlineto{\pgfqpoint{2.007340in}{1.087021in}}%
\pgfpathlineto{\pgfqpoint{2.031219in}{1.102531in}}%
\pgfpathlineto{\pgfqpoint{2.055098in}{1.119313in}}%
\pgfpathlineto{\pgfqpoint{2.078977in}{1.137154in}}%
\pgfpathlineto{\pgfqpoint{2.102857in}{1.156031in}}%
\pgfpathlineto{\pgfqpoint{2.126736in}{1.176224in}}%
\pgfpathlineto{\pgfqpoint{2.150615in}{1.198319in}}%
\pgfpathlineto{\pgfqpoint{2.174494in}{1.223107in}}%
\pgfpathlineto{\pgfqpoint{2.198373in}{1.251347in}}%
\pgfpathlineto{\pgfqpoint{2.222252in}{1.283466in}}%
\pgfpathlineto{\pgfqpoint{2.246131in}{1.319276in}}%
\pgfpathlineto{\pgfqpoint{2.270010in}{1.357782in}}%
\pgfpathlineto{\pgfqpoint{2.293890in}{1.397156in}}%
\pgfpathlineto{\pgfqpoint{2.317769in}{1.434915in}}%
\pgfpathlineto{\pgfqpoint{2.341648in}{1.468250in}}%
\pgfpathlineto{\pgfqpoint{2.365527in}{1.494483in}}%
\pgfpathlineto{\pgfqpoint{2.389406in}{1.511490in}}%
\pgfpathlineto{\pgfqpoint{2.413285in}{1.518065in}}%
\pgfpathlineto{\pgfqpoint{2.437164in}{1.514088in}}%
\pgfpathlineto{\pgfqpoint{2.461044in}{1.500505in}}%
\pgfpathlineto{\pgfqpoint{2.484923in}{1.479138in}}%
\pgfpathlineto{\pgfqpoint{2.508802in}{1.452354in}}%
\pgfpathlineto{\pgfqpoint{2.532681in}{1.422702in}}%
\pgfpathlineto{\pgfqpoint{2.556560in}{1.392568in}}%
\pgfpathlineto{\pgfqpoint{2.580439in}{1.363916in}}%
\pgfpathlineto{\pgfqpoint{2.604318in}{1.338153in}}%
\pgfpathlineto{\pgfqpoint{2.628198in}{1.316086in}}%
\pgfpathlineto{\pgfqpoint{2.652077in}{1.297991in}}%
\pgfpathlineto{\pgfqpoint{2.675956in}{1.283731in}}%
\pgfpathlineto{\pgfqpoint{2.699835in}{1.272897in}}%
\pgfpathlineto{\pgfqpoint{2.723714in}{1.264940in}}%
\pgfpathlineto{\pgfqpoint{2.747593in}{1.259275in}}%
\pgfpathlineto{\pgfqpoint{2.771472in}{1.255365in}}%
\pgfpathlineto{\pgfqpoint{2.795352in}{1.252752in}}%
\pgfpathlineto{\pgfqpoint{2.819231in}{1.251071in}}%
\pgfpathlineto{\pgfqpoint{2.843110in}{1.250046in}}%
\pgfpathlineto{\pgfqpoint{2.866989in}{1.249476in}}%
\pgfpathlineto{\pgfqpoint{2.890868in}{1.249216in}}%
\pgfpathlineto{\pgfqpoint{2.914747in}{1.249161in}}%
\pgfusepath{stroke}%
\end{pgfscope}%
\begin{pgfscope}%
\pgfpathrectangle{\pgfqpoint{0.550713in}{0.398220in}}{\pgfqpoint{2.364035in}{1.703497in}}%
\pgfusepath{clip}%
\pgfsetrectcap%
\pgfsetroundjoin%
\pgfsetlinewidth{0.501875pt}%
\definecolor{currentstroke}{rgb}{0.250980,0.498039,0.717647}%
\pgfsetstrokecolor{currentstroke}%
\pgfsetdash{}{0pt}%
\pgfpathmoveto{\pgfqpoint{0.550713in}{1.343417in}}%
\pgfpathlineto{\pgfqpoint{0.574592in}{1.393850in}}%
\pgfpathlineto{\pgfqpoint{0.598471in}{1.442088in}}%
\pgfpathlineto{\pgfqpoint{0.622350in}{1.485287in}}%
\pgfpathlineto{\pgfqpoint{0.646229in}{1.522034in}}%
\pgfpathlineto{\pgfqpoint{0.670108in}{1.551901in}}%
\pgfpathlineto{\pgfqpoint{0.693987in}{1.576212in}}%
\pgfpathlineto{\pgfqpoint{0.717867in}{1.596320in}}%
\pgfpathlineto{\pgfqpoint{0.741746in}{1.614523in}}%
\pgfpathlineto{\pgfqpoint{0.765625in}{1.632519in}}%
\pgfpathlineto{\pgfqpoint{0.789504in}{1.652556in}}%
\pgfpathlineto{\pgfqpoint{0.813383in}{1.674443in}}%
\pgfpathlineto{\pgfqpoint{0.837262in}{1.698285in}}%
\pgfpathlineto{\pgfqpoint{0.861141in}{1.722793in}}%
\pgfpathlineto{\pgfqpoint{0.885021in}{1.745986in}}%
\pgfpathlineto{\pgfqpoint{0.908900in}{1.765229in}}%
\pgfpathlineto{\pgfqpoint{0.932779in}{1.778028in}}%
\pgfpathlineto{\pgfqpoint{0.956658in}{1.781548in}}%
\pgfpathlineto{\pgfqpoint{0.980537in}{1.774007in}}%
\pgfpathlineto{\pgfqpoint{1.004416in}{1.754763in}}%
\pgfpathlineto{\pgfqpoint{1.028295in}{1.724347in}}%
\pgfpathlineto{\pgfqpoint{1.052174in}{1.684750in}}%
\pgfpathlineto{\pgfqpoint{1.076054in}{1.638735in}}%
\pgfpathlineto{\pgfqpoint{1.099933in}{1.589755in}}%
\pgfpathlineto{\pgfqpoint{1.123812in}{1.540462in}}%
\pgfpathlineto{\pgfqpoint{1.147691in}{1.493536in}}%
\pgfpathlineto{\pgfqpoint{1.171570in}{1.449528in}}%
\pgfpathlineto{\pgfqpoint{1.195449in}{1.409231in}}%
\pgfpathlineto{\pgfqpoint{1.219328in}{1.372114in}}%
\pgfpathlineto{\pgfqpoint{1.243208in}{1.337554in}}%
\pgfpathlineto{\pgfqpoint{1.267087in}{1.305442in}}%
\pgfpathlineto{\pgfqpoint{1.290966in}{1.276022in}}%
\pgfpathlineto{\pgfqpoint{1.314845in}{1.249705in}}%
\pgfpathlineto{\pgfqpoint{1.338724in}{1.226826in}}%
\pgfpathlineto{\pgfqpoint{1.362603in}{1.208119in}}%
\pgfpathlineto{\pgfqpoint{1.386482in}{1.192596in}}%
\pgfpathlineto{\pgfqpoint{1.410362in}{1.180139in}}%
\pgfpathlineto{\pgfqpoint{1.434241in}{1.169386in}}%
\pgfpathlineto{\pgfqpoint{1.458120in}{1.159351in}}%
\pgfpathlineto{\pgfqpoint{1.481999in}{1.149447in}}%
\pgfpathlineto{\pgfqpoint{1.505878in}{1.139087in}}%
\pgfpathlineto{\pgfqpoint{1.529757in}{1.128693in}}%
\pgfpathlineto{\pgfqpoint{1.553636in}{1.118194in}}%
\pgfpathlineto{\pgfqpoint{1.577516in}{1.108724in}}%
\pgfpathlineto{\pgfqpoint{1.601395in}{1.099961in}}%
\pgfpathlineto{\pgfqpoint{1.625274in}{1.092212in}}%
\pgfpathlineto{\pgfqpoint{1.649153in}{1.085648in}}%
\pgfpathlineto{\pgfqpoint{1.673032in}{1.080248in}}%
\pgfpathlineto{\pgfqpoint{1.696911in}{1.076054in}}%
\pgfpathlineto{\pgfqpoint{1.720790in}{1.073210in}}%
\pgfpathlineto{\pgfqpoint{1.744669in}{1.072221in}}%
\pgfpathlineto{\pgfqpoint{1.768549in}{1.073784in}}%
\pgfpathlineto{\pgfqpoint{1.792428in}{1.077899in}}%
\pgfpathlineto{\pgfqpoint{1.816307in}{1.085348in}}%
\pgfpathlineto{\pgfqpoint{1.840186in}{1.095864in}}%
\pgfpathlineto{\pgfqpoint{1.864065in}{1.109771in}}%
\pgfpathlineto{\pgfqpoint{1.887944in}{1.125828in}}%
\pgfpathlineto{\pgfqpoint{1.911823in}{1.143375in}}%
\pgfpathlineto{\pgfqpoint{1.935703in}{1.161815in}}%
\pgfpathlineto{\pgfqpoint{1.959582in}{1.180013in}}%
\pgfpathlineto{\pgfqpoint{1.983461in}{1.198124in}}%
\pgfpathlineto{\pgfqpoint{2.007340in}{1.215913in}}%
\pgfpathlineto{\pgfqpoint{2.031219in}{1.234343in}}%
\pgfpathlineto{\pgfqpoint{2.055098in}{1.253765in}}%
\pgfpathlineto{\pgfqpoint{2.078977in}{1.274967in}}%
\pgfpathlineto{\pgfqpoint{2.102857in}{1.297786in}}%
\pgfpathlineto{\pgfqpoint{2.126736in}{1.321947in}}%
\pgfpathlineto{\pgfqpoint{2.150615in}{1.347149in}}%
\pgfpathlineto{\pgfqpoint{2.174494in}{1.373220in}}%
\pgfpathlineto{\pgfqpoint{2.198373in}{1.399861in}}%
\pgfpathlineto{\pgfqpoint{2.222252in}{1.427098in}}%
\pgfpathlineto{\pgfqpoint{2.246131in}{1.454642in}}%
\pgfpathlineto{\pgfqpoint{2.270010in}{1.482722in}}%
\pgfpathlineto{\pgfqpoint{2.293890in}{1.509458in}}%
\pgfpathlineto{\pgfqpoint{2.317769in}{1.533402in}}%
\pgfpathlineto{\pgfqpoint{2.341648in}{1.552194in}}%
\pgfpathlineto{\pgfqpoint{2.365527in}{1.563416in}}%
\pgfpathlineto{\pgfqpoint{2.389406in}{1.564921in}}%
\pgfpathlineto{\pgfqpoint{2.413285in}{1.555598in}}%
\pgfpathlineto{\pgfqpoint{2.437164in}{1.536412in}}%
\pgfpathlineto{\pgfqpoint{2.461044in}{1.509775in}}%
\pgfpathlineto{\pgfqpoint{2.484923in}{1.479788in}}%
\pgfpathlineto{\pgfqpoint{2.508802in}{1.451682in}}%
\pgfpathlineto{\pgfqpoint{2.532681in}{1.430111in}}%
\pgfpathlineto{\pgfqpoint{2.556560in}{1.419464in}}%
\pgfpathlineto{\pgfqpoint{2.580439in}{1.421249in}}%
\pgfpathlineto{\pgfqpoint{2.604318in}{1.434756in}}%
\pgfpathlineto{\pgfqpoint{2.628198in}{1.456174in}}%
\pgfpathlineto{\pgfqpoint{2.652077in}{1.479674in}}%
\pgfpathlineto{\pgfqpoint{2.675956in}{1.498575in}}%
\pgfpathlineto{\pgfqpoint{2.699835in}{1.507047in}}%
\pgfpathlineto{\pgfqpoint{2.723714in}{1.500302in}}%
\pgfpathlineto{\pgfqpoint{2.747593in}{1.475633in}}%
\pgfpathlineto{\pgfqpoint{2.771472in}{1.434315in}}%
\pgfpathlineto{\pgfqpoint{2.795352in}{1.379082in}}%
\pgfpathlineto{\pgfqpoint{2.819231in}{1.315237in}}%
\pgfpathlineto{\pgfqpoint{2.843110in}{1.249036in}}%
\pgfpathlineto{\pgfqpoint{2.866989in}{1.186707in}}%
\pgfpathlineto{\pgfqpoint{2.890868in}{1.133610in}}%
\pgfpathlineto{\pgfqpoint{2.914747in}{1.093595in}}%
\pgfusepath{stroke}%
\end{pgfscope}%
\begin{pgfscope}%
\pgfpathrectangle{\pgfqpoint{0.550713in}{0.398220in}}{\pgfqpoint{2.364035in}{1.703497in}}%
\pgfusepath{clip}%
\pgfsetrectcap%
\pgfsetroundjoin%
\pgfsetlinewidth{0.501875pt}%
\definecolor{currentstroke}{rgb}{0.250980,0.498039,0.717647}%
\pgfsetstrokecolor{currentstroke}%
\pgfsetdash{}{0pt}%
\pgfpathmoveto{\pgfqpoint{0.550713in}{1.434766in}}%
\pgfpathlineto{\pgfqpoint{0.574592in}{1.435182in}}%
\pgfpathlineto{\pgfqpoint{0.598471in}{1.434062in}}%
\pgfpathlineto{\pgfqpoint{0.622350in}{1.431071in}}%
\pgfpathlineto{\pgfqpoint{0.646229in}{1.424558in}}%
\pgfpathlineto{\pgfqpoint{0.670108in}{1.413584in}}%
\pgfpathlineto{\pgfqpoint{0.693987in}{1.396459in}}%
\pgfpathlineto{\pgfqpoint{0.717867in}{1.373119in}}%
\pgfpathlineto{\pgfqpoint{0.741746in}{1.344262in}}%
\pgfpathlineto{\pgfqpoint{0.765625in}{1.312452in}}%
\pgfpathlineto{\pgfqpoint{0.789504in}{1.283039in}}%
\pgfpathlineto{\pgfqpoint{0.813383in}{1.261191in}}%
\pgfpathlineto{\pgfqpoint{0.837262in}{1.253770in}}%
\pgfpathlineto{\pgfqpoint{0.861141in}{1.265661in}}%
\pgfpathlineto{\pgfqpoint{0.885021in}{1.299487in}}%
\pgfpathlineto{\pgfqpoint{0.908900in}{1.355233in}}%
\pgfpathlineto{\pgfqpoint{0.932779in}{1.429206in}}%
\pgfpathlineto{\pgfqpoint{0.956658in}{1.515161in}}%
\pgfpathlineto{\pgfqpoint{0.980537in}{1.605387in}}%
\pgfpathlineto{\pgfqpoint{1.004416in}{1.691320in}}%
\pgfpathlineto{\pgfqpoint{1.028295in}{1.764738in}}%
\pgfpathlineto{\pgfqpoint{1.052174in}{1.819885in}}%
\pgfpathlineto{\pgfqpoint{1.076054in}{1.852676in}}%
\pgfpathlineto{\pgfqpoint{1.099933in}{1.861892in}}%
\pgfpathlineto{\pgfqpoint{1.123812in}{1.848399in}}%
\pgfpathlineto{\pgfqpoint{1.147691in}{1.816179in}}%
\pgfpathlineto{\pgfqpoint{1.171570in}{1.769411in}}%
\pgfpathlineto{\pgfqpoint{1.195449in}{1.713528in}}%
\pgfpathlineto{\pgfqpoint{1.219328in}{1.653919in}}%
\pgfpathlineto{\pgfqpoint{1.243208in}{1.594816in}}%
\pgfpathlineto{\pgfqpoint{1.267087in}{1.539941in}}%
\pgfpathlineto{\pgfqpoint{1.290966in}{1.490697in}}%
\pgfpathlineto{\pgfqpoint{1.314845in}{1.448068in}}%
\pgfpathlineto{\pgfqpoint{1.338724in}{1.411467in}}%
\pgfpathlineto{\pgfqpoint{1.362603in}{1.379512in}}%
\pgfpathlineto{\pgfqpoint{1.386482in}{1.350808in}}%
\pgfpathlineto{\pgfqpoint{1.410362in}{1.324003in}}%
\pgfpathlineto{\pgfqpoint{1.434241in}{1.298014in}}%
\pgfpathlineto{\pgfqpoint{1.458120in}{1.273051in}}%
\pgfpathlineto{\pgfqpoint{1.481999in}{1.248807in}}%
\pgfpathlineto{\pgfqpoint{1.505878in}{1.225469in}}%
\pgfpathlineto{\pgfqpoint{1.529757in}{1.203644in}}%
\pgfpathlineto{\pgfqpoint{1.553636in}{1.183667in}}%
\pgfpathlineto{\pgfqpoint{1.577516in}{1.165041in}}%
\pgfpathlineto{\pgfqpoint{1.601395in}{1.148100in}}%
\pgfpathlineto{\pgfqpoint{1.625274in}{1.132481in}}%
\pgfpathlineto{\pgfqpoint{1.649153in}{1.118214in}}%
\pgfpathlineto{\pgfqpoint{1.673032in}{1.106399in}}%
\pgfpathlineto{\pgfqpoint{1.696911in}{1.097647in}}%
\pgfpathlineto{\pgfqpoint{1.720790in}{1.093163in}}%
\pgfpathlineto{\pgfqpoint{1.744669in}{1.093011in}}%
\pgfpathlineto{\pgfqpoint{1.768549in}{1.096425in}}%
\pgfpathlineto{\pgfqpoint{1.792428in}{1.101472in}}%
\pgfpathlineto{\pgfqpoint{1.816307in}{1.106846in}}%
\pgfpathlineto{\pgfqpoint{1.840186in}{1.110815in}}%
\pgfpathlineto{\pgfqpoint{1.864065in}{1.112829in}}%
\pgfpathlineto{\pgfqpoint{1.887944in}{1.113633in}}%
\pgfpathlineto{\pgfqpoint{1.911823in}{1.114720in}}%
\pgfpathlineto{\pgfqpoint{1.935703in}{1.117877in}}%
\pgfpathlineto{\pgfqpoint{1.959582in}{1.124171in}}%
\pgfpathlineto{\pgfqpoint{1.983461in}{1.133335in}}%
\pgfpathlineto{\pgfqpoint{2.007340in}{1.144689in}}%
\pgfpathlineto{\pgfqpoint{2.031219in}{1.156689in}}%
\pgfpathlineto{\pgfqpoint{2.055098in}{1.168437in}}%
\pgfpathlineto{\pgfqpoint{2.078977in}{1.179371in}}%
\pgfpathlineto{\pgfqpoint{2.102857in}{1.189945in}}%
\pgfpathlineto{\pgfqpoint{2.126736in}{1.201487in}}%
\pgfpathlineto{\pgfqpoint{2.150615in}{1.215006in}}%
\pgfpathlineto{\pgfqpoint{2.174494in}{1.231913in}}%
\pgfpathlineto{\pgfqpoint{2.198373in}{1.252993in}}%
\pgfpathlineto{\pgfqpoint{2.222252in}{1.276886in}}%
\pgfpathlineto{\pgfqpoint{2.246131in}{1.303006in}}%
\pgfpathlineto{\pgfqpoint{2.270010in}{1.329664in}}%
\pgfpathlineto{\pgfqpoint{2.293890in}{1.355561in}}%
\pgfpathlineto{\pgfqpoint{2.317769in}{1.379390in}}%
\pgfpathlineto{\pgfqpoint{2.341648in}{1.400336in}}%
\pgfpathlineto{\pgfqpoint{2.365527in}{1.417514in}}%
\pgfpathlineto{\pgfqpoint{2.389406in}{1.431450in}}%
\pgfpathlineto{\pgfqpoint{2.413285in}{1.441545in}}%
\pgfpathlineto{\pgfqpoint{2.437164in}{1.449070in}}%
\pgfpathlineto{\pgfqpoint{2.461044in}{1.454471in}}%
\pgfpathlineto{\pgfqpoint{2.484923in}{1.459378in}}%
\pgfpathlineto{\pgfqpoint{2.508802in}{1.465236in}}%
\pgfpathlineto{\pgfqpoint{2.532681in}{1.473154in}}%
\pgfpathlineto{\pgfqpoint{2.556560in}{1.483980in}}%
\pgfpathlineto{\pgfqpoint{2.580439in}{1.497592in}}%
\pgfpathlineto{\pgfqpoint{2.604318in}{1.512398in}}%
\pgfpathlineto{\pgfqpoint{2.628198in}{1.526071in}}%
\pgfpathlineto{\pgfqpoint{2.652077in}{1.535954in}}%
\pgfpathlineto{\pgfqpoint{2.675956in}{1.538759in}}%
\pgfpathlineto{\pgfqpoint{2.699835in}{1.531604in}}%
\pgfpathlineto{\pgfqpoint{2.723714in}{1.513535in}}%
\pgfpathlineto{\pgfqpoint{2.747593in}{1.484184in}}%
\pgfpathlineto{\pgfqpoint{2.771472in}{1.444885in}}%
\pgfpathlineto{\pgfqpoint{2.795352in}{1.398657in}}%
\pgfpathlineto{\pgfqpoint{2.819231in}{1.349203in}}%
\pgfpathlineto{\pgfqpoint{2.843110in}{1.301234in}}%
\pgfpathlineto{\pgfqpoint{2.866989in}{1.259563in}}%
\pgfpathlineto{\pgfqpoint{2.890868in}{1.228361in}}%
\pgfpathlineto{\pgfqpoint{2.914747in}{1.210522in}}%
\pgfusepath{stroke}%
\end{pgfscope}%
\begin{pgfscope}%
\pgfpathrectangle{\pgfqpoint{0.550713in}{0.398220in}}{\pgfqpoint{2.364035in}{1.703497in}}%
\pgfusepath{clip}%
\pgfsetrectcap%
\pgfsetroundjoin%
\pgfsetlinewidth{0.501875pt}%
\definecolor{currentstroke}{rgb}{0.250980,0.498039,0.717647}%
\pgfsetstrokecolor{currentstroke}%
\pgfsetdash{}{0pt}%
\pgfpathmoveto{\pgfqpoint{0.550713in}{1.122679in}}%
\pgfpathlineto{\pgfqpoint{0.574592in}{1.085304in}}%
\pgfpathlineto{\pgfqpoint{0.598471in}{1.062022in}}%
\pgfpathlineto{\pgfqpoint{0.622350in}{1.053820in}}%
\pgfpathlineto{\pgfqpoint{0.646229in}{1.059127in}}%
\pgfpathlineto{\pgfqpoint{0.670108in}{1.076240in}}%
\pgfpathlineto{\pgfqpoint{0.693987in}{1.101532in}}%
\pgfpathlineto{\pgfqpoint{0.717867in}{1.130944in}}%
\pgfpathlineto{\pgfqpoint{0.741746in}{1.160588in}}%
\pgfpathlineto{\pgfqpoint{0.765625in}{1.186764in}}%
\pgfpathlineto{\pgfqpoint{0.789504in}{1.207100in}}%
\pgfpathlineto{\pgfqpoint{0.813383in}{1.221161in}}%
\pgfpathlineto{\pgfqpoint{0.837262in}{1.229986in}}%
\pgfpathlineto{\pgfqpoint{0.861141in}{1.235973in}}%
\pgfpathlineto{\pgfqpoint{0.885021in}{1.242855in}}%
\pgfpathlineto{\pgfqpoint{0.908900in}{1.252604in}}%
\pgfpathlineto{\pgfqpoint{0.932779in}{1.267331in}}%
\pgfpathlineto{\pgfqpoint{0.956658in}{1.286354in}}%
\pgfpathlineto{\pgfqpoint{0.980537in}{1.307579in}}%
\pgfpathlineto{\pgfqpoint{1.004416in}{1.327225in}}%
\pgfpathlineto{\pgfqpoint{1.028295in}{1.341237in}}%
\pgfpathlineto{\pgfqpoint{1.052174in}{1.346380in}}%
\pgfpathlineto{\pgfqpoint{1.076054in}{1.340763in}}%
\pgfpathlineto{\pgfqpoint{1.099933in}{1.324294in}}%
\pgfpathlineto{\pgfqpoint{1.123812in}{1.299651in}}%
\pgfpathlineto{\pgfqpoint{1.147691in}{1.269135in}}%
\pgfpathlineto{\pgfqpoint{1.171570in}{1.236567in}}%
\pgfpathlineto{\pgfqpoint{1.195449in}{1.204918in}}%
\pgfpathlineto{\pgfqpoint{1.219328in}{1.176614in}}%
\pgfpathlineto{\pgfqpoint{1.243208in}{1.151996in}}%
\pgfpathlineto{\pgfqpoint{1.267087in}{1.131660in}}%
\pgfpathlineto{\pgfqpoint{1.290966in}{1.114703in}}%
\pgfpathlineto{\pgfqpoint{1.314845in}{1.099878in}}%
\pgfpathlineto{\pgfqpoint{1.338724in}{1.086647in}}%
\pgfpathlineto{\pgfqpoint{1.362603in}{1.073998in}}%
\pgfpathlineto{\pgfqpoint{1.386482in}{1.062002in}}%
\pgfpathlineto{\pgfqpoint{1.410362in}{1.050280in}}%
\pgfpathlineto{\pgfqpoint{1.434241in}{1.039134in}}%
\pgfpathlineto{\pgfqpoint{1.458120in}{1.028445in}}%
\pgfpathlineto{\pgfqpoint{1.481999in}{1.018270in}}%
\pgfpathlineto{\pgfqpoint{1.505878in}{1.008504in}}%
\pgfpathlineto{\pgfqpoint{1.529757in}{0.998659in}}%
\pgfpathlineto{\pgfqpoint{1.553636in}{0.988976in}}%
\pgfpathlineto{\pgfqpoint{1.577516in}{0.979384in}}%
\pgfpathlineto{\pgfqpoint{1.601395in}{0.970046in}}%
\pgfpathlineto{\pgfqpoint{1.625274in}{0.961051in}}%
\pgfpathlineto{\pgfqpoint{1.649153in}{0.952844in}}%
\pgfpathlineto{\pgfqpoint{1.673032in}{0.944721in}}%
\pgfpathlineto{\pgfqpoint{1.696911in}{0.937100in}}%
\pgfpathlineto{\pgfqpoint{1.720790in}{0.929602in}}%
\pgfpathlineto{\pgfqpoint{1.744669in}{0.922241in}}%
\pgfpathlineto{\pgfqpoint{1.768549in}{0.916166in}}%
\pgfpathlineto{\pgfqpoint{1.792428in}{0.911213in}}%
\pgfpathlineto{\pgfqpoint{1.816307in}{0.908010in}}%
\pgfpathlineto{\pgfqpoint{1.840186in}{0.906464in}}%
\pgfpathlineto{\pgfqpoint{1.864065in}{0.906403in}}%
\pgfpathlineto{\pgfqpoint{1.887944in}{0.906859in}}%
\pgfpathlineto{\pgfqpoint{1.911823in}{0.908408in}}%
\pgfpathlineto{\pgfqpoint{1.935703in}{0.910815in}}%
\pgfpathlineto{\pgfqpoint{1.959582in}{0.915241in}}%
\pgfpathlineto{\pgfqpoint{1.983461in}{0.922757in}}%
\pgfpathlineto{\pgfqpoint{2.007340in}{0.935943in}}%
\pgfpathlineto{\pgfqpoint{2.031219in}{0.955707in}}%
\pgfpathlineto{\pgfqpoint{2.055098in}{0.983126in}}%
\pgfpathlineto{\pgfqpoint{2.078977in}{1.018145in}}%
\pgfpathlineto{\pgfqpoint{2.102857in}{1.060068in}}%
\pgfpathlineto{\pgfqpoint{2.126736in}{1.108279in}}%
\pgfpathlineto{\pgfqpoint{2.150615in}{1.161028in}}%
\pgfpathlineto{\pgfqpoint{2.174494in}{1.217620in}}%
\pgfpathlineto{\pgfqpoint{2.198373in}{1.277390in}}%
\pgfpathlineto{\pgfqpoint{2.222252in}{1.340493in}}%
\pgfpathlineto{\pgfqpoint{2.246131in}{1.406512in}}%
\pgfpathlineto{\pgfqpoint{2.270010in}{1.474469in}}%
\pgfpathlineto{\pgfqpoint{2.293890in}{1.543079in}}%
\pgfpathlineto{\pgfqpoint{2.317769in}{1.609818in}}%
\pgfpathlineto{\pgfqpoint{2.341648in}{1.670097in}}%
\pgfpathlineto{\pgfqpoint{2.365527in}{1.719305in}}%
\pgfpathlineto{\pgfqpoint{2.389406in}{1.752238in}}%
\pgfpathlineto{\pgfqpoint{2.413285in}{1.764205in}}%
\pgfpathlineto{\pgfqpoint{2.437164in}{1.752326in}}%
\pgfpathlineto{\pgfqpoint{2.461044in}{1.716650in}}%
\pgfpathlineto{\pgfqpoint{2.484923in}{1.659900in}}%
\pgfpathlineto{\pgfqpoint{2.508802in}{1.587552in}}%
\pgfpathlineto{\pgfqpoint{2.532681in}{1.507591in}}%
\pgfpathlineto{\pgfqpoint{2.556560in}{1.428906in}}%
\pgfpathlineto{\pgfqpoint{2.580439in}{1.359246in}}%
\pgfpathlineto{\pgfqpoint{2.604318in}{1.305254in}}%
\pgfpathlineto{\pgfqpoint{2.628198in}{1.269458in}}%
\pgfpathlineto{\pgfqpoint{2.652077in}{1.252353in}}%
\pgfpathlineto{\pgfqpoint{2.675956in}{1.250998in}}%
\pgfpathlineto{\pgfqpoint{2.699835in}{1.260013in}}%
\pgfpathlineto{\pgfqpoint{2.723714in}{1.273018in}}%
\pgfpathlineto{\pgfqpoint{2.747593in}{1.284493in}}%
\pgfpathlineto{\pgfqpoint{2.771472in}{1.290410in}}%
\pgfpathlineto{\pgfqpoint{2.795352in}{1.288741in}}%
\pgfpathlineto{\pgfqpoint{2.819231in}{1.279553in}}%
\pgfpathlineto{\pgfqpoint{2.843110in}{1.265281in}}%
\pgfpathlineto{\pgfqpoint{2.866989in}{1.247974in}}%
\pgfpathlineto{\pgfqpoint{2.890868in}{1.231096in}}%
\pgfpathlineto{\pgfqpoint{2.914747in}{1.216661in}}%
\pgfusepath{stroke}%
\end{pgfscope}%
\begin{pgfscope}%
\pgfsetrectcap%
\pgfsetmiterjoin%
\pgfsetlinewidth{0.752812pt}%
\definecolor{currentstroke}{rgb}{0.000000,0.000000,0.000000}%
\pgfsetstrokecolor{currentstroke}%
\pgfsetdash{}{0pt}%
\pgfpathmoveto{\pgfqpoint{0.550713in}{0.398220in}}%
\pgfpathlineto{\pgfqpoint{0.550713in}{2.101717in}}%
\pgfusepath{stroke}%
\end{pgfscope}%
\begin{pgfscope}%
\pgfsetrectcap%
\pgfsetmiterjoin%
\pgfsetlinewidth{0.752812pt}%
\definecolor{currentstroke}{rgb}{0.000000,0.000000,0.000000}%
\pgfsetstrokecolor{currentstroke}%
\pgfsetdash{}{0pt}%
\pgfpathmoveto{\pgfqpoint{2.914747in}{0.398220in}}%
\pgfpathlineto{\pgfqpoint{2.914747in}{2.101717in}}%
\pgfusepath{stroke}%
\end{pgfscope}%
\begin{pgfscope}%
\pgfsetrectcap%
\pgfsetmiterjoin%
\pgfsetlinewidth{0.752812pt}%
\definecolor{currentstroke}{rgb}{0.000000,0.000000,0.000000}%
\pgfsetstrokecolor{currentstroke}%
\pgfsetdash{}{0pt}%
\pgfpathmoveto{\pgfqpoint{0.550713in}{0.398220in}}%
\pgfpathlineto{\pgfqpoint{2.914747in}{0.398220in}}%
\pgfusepath{stroke}%
\end{pgfscope}%
\begin{pgfscope}%
\pgfsetrectcap%
\pgfsetmiterjoin%
\pgfsetlinewidth{0.752812pt}%
\definecolor{currentstroke}{rgb}{0.000000,0.000000,0.000000}%
\pgfsetstrokecolor{currentstroke}%
\pgfsetdash{}{0pt}%
\pgfpathmoveto{\pgfqpoint{0.550713in}{2.101717in}}%
\pgfpathlineto{\pgfqpoint{2.914747in}{2.101717in}}%
\pgfusepath{stroke}%
\end{pgfscope}%
\begin{pgfscope}%
\definecolor{textcolor}{rgb}{0.000000,0.000000,0.000000}%
\pgfsetstrokecolor{textcolor}%
\pgfsetfillcolor{textcolor}%
\pgftext[x=1.732730in,y=1.931368in,,base]{\color{textcolor}\rmfamily\fontsize{10.000000}{12.000000}\selectfont Prior}%
\end{pgfscope}%
\begin{pgfscope}%
\pgfsetbuttcap%
\pgfsetmiterjoin%
\definecolor{currentfill}{rgb}{1.000000,1.000000,1.000000}%
\pgfsetfillcolor{currentfill}%
\pgfsetlinewidth{0.000000pt}%
\definecolor{currentstroke}{rgb}{0.000000,0.000000,0.000000}%
\pgfsetstrokecolor{currentstroke}%
\pgfsetstrokeopacity{0.000000}%
\pgfsetdash{}{0pt}%
\pgfpathmoveto{\pgfqpoint{3.032949in}{0.398220in}}%
\pgfpathlineto{\pgfqpoint{5.396984in}{0.398220in}}%
\pgfpathlineto{\pgfqpoint{5.396984in}{2.101717in}}%
\pgfpathlineto{\pgfqpoint{3.032949in}{2.101717in}}%
\pgfpathclose%
\pgfusepath{fill}%
\end{pgfscope}%
\begin{pgfscope}%
\pgfpathrectangle{\pgfqpoint{3.032949in}{0.398220in}}{\pgfqpoint{2.364035in}{1.703497in}}%
\pgfusepath{clip}%
\pgfsetbuttcap%
\pgfsetroundjoin%
\definecolor{currentfill}{rgb}{0.803922,0.545098,0.529412}%
\pgfsetfillcolor{currentfill}%
\pgfsetfillopacity{0.700000}%
\pgfsetlinewidth{0.000000pt}%
\definecolor{currentstroke}{rgb}{0.803922,0.545098,0.529412}%
\pgfsetstrokecolor{currentstroke}%
\pgfsetstrokeopacity{0.700000}%
\pgfsetdash{}{0pt}%
\pgfsys@defobject{currentmarker}{\pgfqpoint{3.032949in}{0.712552in}}{\pgfqpoint{5.396984in}{1.603989in}}{%
\pgfpathmoveto{\pgfqpoint{3.032949in}{1.590669in}}%
\pgfpathlineto{\pgfqpoint{3.032949in}{0.909271in}}%
\pgfpathlineto{\pgfqpoint{3.056828in}{0.909272in}}%
\pgfpathlineto{\pgfqpoint{3.080707in}{0.909274in}}%
\pgfpathlineto{\pgfqpoint{3.104586in}{0.909277in}}%
\pgfpathlineto{\pgfqpoint{3.128465in}{0.909283in}}%
\pgfpathlineto{\pgfqpoint{3.152345in}{0.909295in}}%
\pgfpathlineto{\pgfqpoint{3.176224in}{0.909316in}}%
\pgfpathlineto{\pgfqpoint{3.200103in}{0.909358in}}%
\pgfpathlineto{\pgfqpoint{3.223982in}{0.909442in}}%
\pgfpathlineto{\pgfqpoint{3.247861in}{0.909612in}}%
\pgfpathlineto{\pgfqpoint{3.271740in}{0.909949in}}%
\pgfpathlineto{\pgfqpoint{3.295619in}{0.910596in}}%
\pgfpathlineto{\pgfqpoint{3.319499in}{0.911794in}}%
\pgfpathlineto{\pgfqpoint{3.343378in}{0.913907in}}%
\pgfpathlineto{\pgfqpoint{3.367257in}{0.917445in}}%
\pgfpathlineto{\pgfqpoint{3.391136in}{0.923060in}}%
\pgfpathlineto{\pgfqpoint{3.415015in}{0.931495in}}%
\pgfpathlineto{\pgfqpoint{3.438894in}{0.943485in}}%
\pgfpathlineto{\pgfqpoint{3.462773in}{0.959608in}}%
\pgfpathlineto{\pgfqpoint{3.486653in}{0.980107in}}%
\pgfpathlineto{\pgfqpoint{3.510532in}{1.004698in}}%
\pgfpathlineto{\pgfqpoint{3.534411in}{1.032407in}}%
\pgfpathlineto{\pgfqpoint{3.558290in}{1.061409in}}%
\pgfpathlineto{\pgfqpoint{3.582169in}{1.088658in}}%
\pgfpathlineto{\pgfqpoint{3.606048in}{1.107652in}}%
\pgfpathlineto{\pgfqpoint{3.629927in}{1.094807in}}%
\pgfpathlineto{\pgfqpoint{3.653807in}{1.038695in}}%
\pgfpathlineto{\pgfqpoint{3.677686in}{0.975455in}}%
\pgfpathlineto{\pgfqpoint{3.701565in}{0.918603in}}%
\pgfpathlineto{\pgfqpoint{3.725444in}{0.873372in}}%
\pgfpathlineto{\pgfqpoint{3.749323in}{0.843014in}}%
\pgfpathlineto{\pgfqpoint{3.773202in}{0.829379in}}%
\pgfpathlineto{\pgfqpoint{3.797081in}{0.832639in}}%
\pgfpathlineto{\pgfqpoint{3.820960in}{0.849872in}}%
\pgfpathlineto{\pgfqpoint{3.844840in}{0.865041in}}%
\pgfpathlineto{\pgfqpoint{3.868719in}{0.845319in}}%
\pgfpathlineto{\pgfqpoint{3.892598in}{0.813667in}}%
\pgfpathlineto{\pgfqpoint{3.916477in}{0.787045in}}%
\pgfpathlineto{\pgfqpoint{3.940356in}{0.768608in}}%
\pgfpathlineto{\pgfqpoint{3.964235in}{0.758383in}}%
\pgfpathlineto{\pgfqpoint{3.988114in}{0.755023in}}%
\pgfpathlineto{\pgfqpoint{4.011994in}{0.756392in}}%
\pgfpathlineto{\pgfqpoint{4.035873in}{0.760015in}}%
\pgfpathlineto{\pgfqpoint{4.059752in}{0.763495in}}%
\pgfpathlineto{\pgfqpoint{4.083631in}{0.764882in}}%
\pgfpathlineto{\pgfqpoint{4.107510in}{0.762962in}}%
\pgfpathlineto{\pgfqpoint{4.131389in}{0.757411in}}%
\pgfpathlineto{\pgfqpoint{4.155268in}{0.748801in}}%
\pgfpathlineto{\pgfqpoint{4.179148in}{0.738445in}}%
\pgfpathlineto{\pgfqpoint{4.203027in}{0.728123in}}%
\pgfpathlineto{\pgfqpoint{4.226906in}{0.719722in}}%
\pgfpathlineto{\pgfqpoint{4.250785in}{0.714849in}}%
\pgfpathlineto{\pgfqpoint{4.274664in}{0.714469in}}%
\pgfpathlineto{\pgfqpoint{4.298543in}{0.718563in}}%
\pgfpathlineto{\pgfqpoint{4.322422in}{0.725628in}}%
\pgfpathlineto{\pgfqpoint{4.346301in}{0.731201in}}%
\pgfpathlineto{\pgfqpoint{4.370181in}{0.726547in}}%
\pgfpathlineto{\pgfqpoint{4.394060in}{0.715187in}}%
\pgfpathlineto{\pgfqpoint{4.417939in}{0.712552in}}%
\pgfpathlineto{\pgfqpoint{4.441818in}{0.723483in}}%
\pgfpathlineto{\pgfqpoint{4.465697in}{0.745852in}}%
\pgfpathlineto{\pgfqpoint{4.489576in}{0.774308in}}%
\pgfpathlineto{\pgfqpoint{4.513455in}{0.802136in}}%
\pgfpathlineto{\pgfqpoint{4.537335in}{0.825232in}}%
\pgfpathlineto{\pgfqpoint{4.561214in}{0.845099in}}%
\pgfpathlineto{\pgfqpoint{4.585093in}{0.865048in}}%
\pgfpathlineto{\pgfqpoint{4.608972in}{0.887044in}}%
\pgfpathlineto{\pgfqpoint{4.632851in}{0.911650in}}%
\pgfpathlineto{\pgfqpoint{4.656730in}{0.938391in}}%
\pgfpathlineto{\pgfqpoint{4.680609in}{0.965701in}}%
\pgfpathlineto{\pgfqpoint{4.704489in}{0.991191in}}%
\pgfpathlineto{\pgfqpoint{4.728368in}{1.014226in}}%
\pgfpathlineto{\pgfqpoint{4.752247in}{1.037365in}}%
\pgfpathlineto{\pgfqpoint{4.776126in}{1.063521in}}%
\pgfpathlineto{\pgfqpoint{4.800005in}{1.094426in}}%
\pgfpathlineto{\pgfqpoint{4.823884in}{1.130440in}}%
\pgfpathlineto{\pgfqpoint{4.847763in}{1.169755in}}%
\pgfpathlineto{\pgfqpoint{4.871643in}{1.205146in}}%
\pgfpathlineto{\pgfqpoint{4.895522in}{1.221264in}}%
\pgfpathlineto{\pgfqpoint{4.919401in}{1.214153in}}%
\pgfpathlineto{\pgfqpoint{4.943280in}{1.192028in}}%
\pgfpathlineto{\pgfqpoint{4.967159in}{1.160865in}}%
\pgfpathlineto{\pgfqpoint{4.991038in}{1.124941in}}%
\pgfpathlineto{\pgfqpoint{5.014917in}{1.087787in}}%
\pgfpathlineto{\pgfqpoint{5.038796in}{1.052216in}}%
\pgfpathlineto{\pgfqpoint{5.062676in}{1.020214in}}%
\pgfpathlineto{\pgfqpoint{5.086555in}{0.992911in}}%
\pgfpathlineto{\pgfqpoint{5.110434in}{0.970676in}}%
\pgfpathlineto{\pgfqpoint{5.134313in}{0.953291in}}%
\pgfpathlineto{\pgfqpoint{5.158192in}{0.940169in}}%
\pgfpathlineto{\pgfqpoint{5.182071in}{0.930556in}}%
\pgfpathlineto{\pgfqpoint{5.205950in}{0.923686in}}%
\pgfpathlineto{\pgfqpoint{5.229830in}{0.918874in}}%
\pgfpathlineto{\pgfqpoint{5.253709in}{0.915563in}}%
\pgfpathlineto{\pgfqpoint{5.277588in}{0.913320in}}%
\pgfpathlineto{\pgfqpoint{5.301467in}{0.911828in}}%
\pgfpathlineto{\pgfqpoint{5.325346in}{0.910851in}}%
\pgfpathlineto{\pgfqpoint{5.349225in}{0.910226in}}%
\pgfpathlineto{\pgfqpoint{5.373104in}{0.909834in}}%
\pgfpathlineto{\pgfqpoint{5.396984in}{0.909595in}}%
\pgfpathlineto{\pgfqpoint{5.396984in}{1.590990in}}%
\pgfpathlineto{\pgfqpoint{5.396984in}{1.590990in}}%
\pgfpathlineto{\pgfqpoint{5.373104in}{1.591223in}}%
\pgfpathlineto{\pgfqpoint{5.349225in}{1.591595in}}%
\pgfpathlineto{\pgfqpoint{5.325346in}{1.592170in}}%
\pgfpathlineto{\pgfqpoint{5.301467in}{1.593023in}}%
\pgfpathlineto{\pgfqpoint{5.277588in}{1.594229in}}%
\pgfpathlineto{\pgfqpoint{5.253709in}{1.595842in}}%
\pgfpathlineto{\pgfqpoint{5.229830in}{1.597853in}}%
\pgfpathlineto{\pgfqpoint{5.205950in}{1.600133in}}%
\pgfpathlineto{\pgfqpoint{5.182071in}{1.602358in}}%
\pgfpathlineto{\pgfqpoint{5.158192in}{1.603942in}}%
\pgfpathlineto{\pgfqpoint{5.134313in}{1.603989in}}%
\pgfpathlineto{\pgfqpoint{5.110434in}{1.601298in}}%
\pgfpathlineto{\pgfqpoint{5.086555in}{1.594447in}}%
\pgfpathlineto{\pgfqpoint{5.062676in}{1.581953in}}%
\pgfpathlineto{\pgfqpoint{5.038796in}{1.562503in}}%
\pgfpathlineto{\pgfqpoint{5.014917in}{1.535229in}}%
\pgfpathlineto{\pgfqpoint{4.991038in}{1.499960in}}%
\pgfpathlineto{\pgfqpoint{4.967159in}{1.457440in}}%
\pgfpathlineto{\pgfqpoint{4.943280in}{1.409492in}}%
\pgfpathlineto{\pgfqpoint{4.919401in}{1.359336in}}%
\pgfpathlineto{\pgfqpoint{4.895522in}{1.312758in}}%
\pgfpathlineto{\pgfqpoint{4.871643in}{1.278784in}}%
\pgfpathlineto{\pgfqpoint{4.847763in}{1.255244in}}%
\pgfpathlineto{\pgfqpoint{4.823884in}{1.229382in}}%
\pgfpathlineto{\pgfqpoint{4.800005in}{1.197050in}}%
\pgfpathlineto{\pgfqpoint{4.776126in}{1.159587in}}%
\pgfpathlineto{\pgfqpoint{4.752247in}{1.120133in}}%
\pgfpathlineto{\pgfqpoint{4.728368in}{1.082434in}}%
\pgfpathlineto{\pgfqpoint{4.704489in}{1.050391in}}%
\pgfpathlineto{\pgfqpoint{4.680609in}{1.026438in}}%
\pgfpathlineto{\pgfqpoint{4.656730in}{1.008818in}}%
\pgfpathlineto{\pgfqpoint{4.632851in}{0.993316in}}%
\pgfpathlineto{\pgfqpoint{4.608972in}{0.976284in}}%
\pgfpathlineto{\pgfqpoint{4.585093in}{0.955386in}}%
\pgfpathlineto{\pgfqpoint{4.561214in}{0.930026in}}%
\pgfpathlineto{\pgfqpoint{4.537335in}{0.902066in}}%
\pgfpathlineto{\pgfqpoint{4.513455in}{0.875947in}}%
\pgfpathlineto{\pgfqpoint{4.489576in}{0.855523in}}%
\pgfpathlineto{\pgfqpoint{4.465697in}{0.840008in}}%
\pgfpathlineto{\pgfqpoint{4.441818in}{0.826567in}}%
\pgfpathlineto{\pgfqpoint{4.417939in}{0.813752in}}%
\pgfpathlineto{\pgfqpoint{4.394060in}{0.802793in}}%
\pgfpathlineto{\pgfqpoint{4.370181in}{0.800834in}}%
\pgfpathlineto{\pgfqpoint{4.346301in}{0.824168in}}%
\pgfpathlineto{\pgfqpoint{4.322422in}{0.875558in}}%
\pgfpathlineto{\pgfqpoint{4.298543in}{0.943895in}}%
\pgfpathlineto{\pgfqpoint{4.274664in}{1.020960in}}%
\pgfpathlineto{\pgfqpoint{4.250785in}{1.100483in}}%
\pgfpathlineto{\pgfqpoint{4.226906in}{1.177135in}}%
\pgfpathlineto{\pgfqpoint{4.203027in}{1.246527in}}%
\pgfpathlineto{\pgfqpoint{4.179148in}{1.305316in}}%
\pgfpathlineto{\pgfqpoint{4.155268in}{1.351220in}}%
\pgfpathlineto{\pgfqpoint{4.131389in}{1.382901in}}%
\pgfpathlineto{\pgfqpoint{4.107510in}{1.399777in}}%
\pgfpathlineto{\pgfqpoint{4.083631in}{1.401818in}}%
\pgfpathlineto{\pgfqpoint{4.059752in}{1.389403in}}%
\pgfpathlineto{\pgfqpoint{4.035873in}{1.363267in}}%
\pgfpathlineto{\pgfqpoint{4.011994in}{1.324541in}}%
\pgfpathlineto{\pgfqpoint{3.988114in}{1.274848in}}%
\pgfpathlineto{\pgfqpoint{3.964235in}{1.216419in}}%
\pgfpathlineto{\pgfqpoint{3.940356in}{1.152174in}}%
\pgfpathlineto{\pgfqpoint{3.916477in}{1.085772in}}%
\pgfpathlineto{\pgfqpoint{3.892598in}{1.021769in}}%
\pgfpathlineto{\pgfqpoint{3.868719in}{0.966953in}}%
\pgfpathlineto{\pgfqpoint{3.844840in}{0.940688in}}%
\pgfpathlineto{\pgfqpoint{3.820960in}{0.966886in}}%
\pgfpathlineto{\pgfqpoint{3.797081in}{1.012172in}}%
\pgfpathlineto{\pgfqpoint{3.773202in}{1.058579in}}%
\pgfpathlineto{\pgfqpoint{3.749323in}{1.100155in}}%
\pgfpathlineto{\pgfqpoint{3.725444in}{1.133331in}}%
\pgfpathlineto{\pgfqpoint{3.701565in}{1.155947in}}%
\pgfpathlineto{\pgfqpoint{3.677686in}{1.167400in}}%
\pgfpathlineto{\pgfqpoint{3.653807in}{1.169582in}}%
\pgfpathlineto{\pgfqpoint{3.629927in}{1.173414in}}%
\pgfpathlineto{\pgfqpoint{3.606048in}{1.213308in}}%
\pgfpathlineto{\pgfqpoint{3.582169in}{1.276961in}}%
\pgfpathlineto{\pgfqpoint{3.558290in}{1.340667in}}%
\pgfpathlineto{\pgfqpoint{3.534411in}{1.398387in}}%
\pgfpathlineto{\pgfqpoint{3.510532in}{1.447936in}}%
\pgfpathlineto{\pgfqpoint{3.486653in}{1.488560in}}%
\pgfpathlineto{\pgfqpoint{3.462773in}{1.520415in}}%
\pgfpathlineto{\pgfqpoint{3.438894in}{1.544290in}}%
\pgfpathlineto{\pgfqpoint{3.415015in}{1.561367in}}%
\pgfpathlineto{\pgfqpoint{3.391136in}{1.573003in}}%
\pgfpathlineto{\pgfqpoint{3.367257in}{1.580538in}}%
\pgfpathlineto{\pgfqpoint{3.343378in}{1.585165in}}%
\pgfpathlineto{\pgfqpoint{3.319499in}{1.587851in}}%
\pgfpathlineto{\pgfqpoint{3.295619in}{1.589322in}}%
\pgfpathlineto{\pgfqpoint{3.271740in}{1.590077in}}%
\pgfpathlineto{\pgfqpoint{3.247861in}{1.590438in}}%
\pgfpathlineto{\pgfqpoint{3.223982in}{1.590596in}}%
\pgfpathlineto{\pgfqpoint{3.200103in}{1.590657in}}%
\pgfpathlineto{\pgfqpoint{3.176224in}{1.590677in}}%
\pgfpathlineto{\pgfqpoint{3.152345in}{1.590679in}}%
\pgfpathlineto{\pgfqpoint{3.128465in}{1.590677in}}%
\pgfpathlineto{\pgfqpoint{3.104586in}{1.590674in}}%
\pgfpathlineto{\pgfqpoint{3.080707in}{1.590672in}}%
\pgfpathlineto{\pgfqpoint{3.056828in}{1.590670in}}%
\pgfpathlineto{\pgfqpoint{3.032949in}{1.590669in}}%
\pgfpathclose%
\pgfusepath{fill}%
}%
\begin{pgfscope}%
\pgfsys@transformshift{0.000000in}{0.000000in}%
\pgfsys@useobject{currentmarker}{}%
\end{pgfscope}%
\end{pgfscope}%
\begin{pgfscope}%
\pgfpathrectangle{\pgfqpoint{3.032949in}{0.398220in}}{\pgfqpoint{2.364035in}{1.703497in}}%
\pgfusepath{clip}%
\pgfsetbuttcap%
\pgfsetroundjoin%
\definecolor{currentfill}{rgb}{0.556863,0.729412,0.898039}%
\pgfsetfillcolor{currentfill}%
\pgfsetfillopacity{0.700000}%
\pgfsetlinewidth{0.000000pt}%
\definecolor{currentstroke}{rgb}{0.556863,0.729412,0.898039}%
\pgfsetstrokecolor{currentstroke}%
\pgfsetstrokeopacity{0.700000}%
\pgfsetdash{}{0pt}%
\pgfsys@defobject{currentmarker}{\pgfqpoint{3.032949in}{0.694900in}}{\pgfqpoint{5.396984in}{1.597797in}}{%
\pgfpathmoveto{\pgfqpoint{3.032949in}{1.590864in}}%
\pgfpathlineto{\pgfqpoint{3.032949in}{0.909999in}}%
\pgfpathlineto{\pgfqpoint{3.056828in}{0.909971in}}%
\pgfpathlineto{\pgfqpoint{3.080707in}{0.909938in}}%
\pgfpathlineto{\pgfqpoint{3.104586in}{0.909931in}}%
\pgfpathlineto{\pgfqpoint{3.128465in}{0.910031in}}%
\pgfpathlineto{\pgfqpoint{3.152345in}{0.910348in}}%
\pgfpathlineto{\pgfqpoint{3.176224in}{0.911030in}}%
\pgfpathlineto{\pgfqpoint{3.200103in}{0.912260in}}%
\pgfpathlineto{\pgfqpoint{3.223982in}{0.914272in}}%
\pgfpathlineto{\pgfqpoint{3.247861in}{0.917355in}}%
\pgfpathlineto{\pgfqpoint{3.271740in}{0.921888in}}%
\pgfpathlineto{\pgfqpoint{3.295619in}{0.928357in}}%
\pgfpathlineto{\pgfqpoint{3.319499in}{0.937353in}}%
\pgfpathlineto{\pgfqpoint{3.343378in}{0.949529in}}%
\pgfpathlineto{\pgfqpoint{3.367257in}{0.965492in}}%
\pgfpathlineto{\pgfqpoint{3.391136in}{0.985626in}}%
\pgfpathlineto{\pgfqpoint{3.415015in}{1.009853in}}%
\pgfpathlineto{\pgfqpoint{3.438894in}{1.037424in}}%
\pgfpathlineto{\pgfqpoint{3.462773in}{1.066770in}}%
\pgfpathlineto{\pgfqpoint{3.486653in}{1.095519in}}%
\pgfpathlineto{\pgfqpoint{3.510532in}{1.120697in}}%
\pgfpathlineto{\pgfqpoint{3.534411in}{1.139110in}}%
\pgfpathlineto{\pgfqpoint{3.558290in}{1.147814in}}%
\pgfpathlineto{\pgfqpoint{3.582169in}{1.144569in}}%
\pgfpathlineto{\pgfqpoint{3.606048in}{1.128369in}}%
\pgfpathlineto{\pgfqpoint{3.629927in}{1.100782in}}%
\pgfpathlineto{\pgfqpoint{3.653807in}{1.066604in}}%
\pgfpathlineto{\pgfqpoint{3.677686in}{1.030807in}}%
\pgfpathlineto{\pgfqpoint{3.701565in}{0.996580in}}%
\pgfpathlineto{\pgfqpoint{3.725444in}{0.965934in}}%
\pgfpathlineto{\pgfqpoint{3.749323in}{0.939932in}}%
\pgfpathlineto{\pgfqpoint{3.773202in}{0.918364in}}%
\pgfpathlineto{\pgfqpoint{3.797081in}{0.899748in}}%
\pgfpathlineto{\pgfqpoint{3.820960in}{0.882047in}}%
\pgfpathlineto{\pgfqpoint{3.844840in}{0.863881in}}%
\pgfpathlineto{\pgfqpoint{3.868719in}{0.845110in}}%
\pgfpathlineto{\pgfqpoint{3.892598in}{0.826167in}}%
\pgfpathlineto{\pgfqpoint{3.916477in}{0.807371in}}%
\pgfpathlineto{\pgfqpoint{3.940356in}{0.789063in}}%
\pgfpathlineto{\pgfqpoint{3.964235in}{0.771855in}}%
\pgfpathlineto{\pgfqpoint{3.988114in}{0.756396in}}%
\pgfpathlineto{\pgfqpoint{4.011994in}{0.742960in}}%
\pgfpathlineto{\pgfqpoint{4.035873in}{0.731384in}}%
\pgfpathlineto{\pgfqpoint{4.059752in}{0.721388in}}%
\pgfpathlineto{\pgfqpoint{4.083631in}{0.712899in}}%
\pgfpathlineto{\pgfqpoint{4.107510in}{0.706020in}}%
\pgfpathlineto{\pgfqpoint{4.131389in}{0.700803in}}%
\pgfpathlineto{\pgfqpoint{4.155268in}{0.697174in}}%
\pgfpathlineto{\pgfqpoint{4.179148in}{0.695136in}}%
\pgfpathlineto{\pgfqpoint{4.203027in}{0.694900in}}%
\pgfpathlineto{\pgfqpoint{4.226906in}{0.696713in}}%
\pgfpathlineto{\pgfqpoint{4.250785in}{0.700537in}}%
\pgfpathlineto{\pgfqpoint{4.274664in}{0.705968in}}%
\pgfpathlineto{\pgfqpoint{4.298543in}{0.712521in}}%
\pgfpathlineto{\pgfqpoint{4.322422in}{0.719955in}}%
\pgfpathlineto{\pgfqpoint{4.346301in}{0.728230in}}%
\pgfpathlineto{\pgfqpoint{4.370181in}{0.737238in}}%
\pgfpathlineto{\pgfqpoint{4.394060in}{0.746801in}}%
\pgfpathlineto{\pgfqpoint{4.417939in}{0.757075in}}%
\pgfpathlineto{\pgfqpoint{4.441818in}{0.768643in}}%
\pgfpathlineto{\pgfqpoint{4.465697in}{0.781944in}}%
\pgfpathlineto{\pgfqpoint{4.489576in}{0.796826in}}%
\pgfpathlineto{\pgfqpoint{4.513455in}{0.812830in}}%
\pgfpathlineto{\pgfqpoint{4.537335in}{0.829887in}}%
\pgfpathlineto{\pgfqpoint{4.561214in}{0.848412in}}%
\pgfpathlineto{\pgfqpoint{4.585093in}{0.868716in}}%
\pgfpathlineto{\pgfqpoint{4.608972in}{0.890618in}}%
\pgfpathlineto{\pgfqpoint{4.632851in}{0.913841in}}%
\pgfpathlineto{\pgfqpoint{4.656730in}{0.938639in}}%
\pgfpathlineto{\pgfqpoint{4.680609in}{0.965555in}}%
\pgfpathlineto{\pgfqpoint{4.704489in}{0.994250in}}%
\pgfpathlineto{\pgfqpoint{4.728368in}{1.023119in}}%
\pgfpathlineto{\pgfqpoint{4.752247in}{1.051268in}}%
\pgfpathlineto{\pgfqpoint{4.776126in}{1.079919in}}%
\pgfpathlineto{\pgfqpoint{4.800005in}{1.110921in}}%
\pgfpathlineto{\pgfqpoint{4.823884in}{1.145081in}}%
\pgfpathlineto{\pgfqpoint{4.847763in}{1.180792in}}%
\pgfpathlineto{\pgfqpoint{4.871643in}{1.210954in}}%
\pgfpathlineto{\pgfqpoint{4.895522in}{1.222059in}}%
\pgfpathlineto{\pgfqpoint{4.919401in}{1.211750in}}%
\pgfpathlineto{\pgfqpoint{4.943280in}{1.187299in}}%
\pgfpathlineto{\pgfqpoint{4.967159in}{1.154262in}}%
\pgfpathlineto{\pgfqpoint{4.991038in}{1.116813in}}%
\pgfpathlineto{\pgfqpoint{5.014917in}{1.078429in}}%
\pgfpathlineto{\pgfqpoint{5.038796in}{1.041894in}}%
\pgfpathlineto{\pgfqpoint{5.062676in}{1.009205in}}%
\pgfpathlineto{\pgfqpoint{5.086555in}{0.981542in}}%
\pgfpathlineto{\pgfqpoint{5.110434in}{0.959323in}}%
\pgfpathlineto{\pgfqpoint{5.134313in}{0.942368in}}%
\pgfpathlineto{\pgfqpoint{5.158192in}{0.930086in}}%
\pgfpathlineto{\pgfqpoint{5.182071in}{0.921658in}}%
\pgfpathlineto{\pgfqpoint{5.205950in}{0.916216in}}%
\pgfpathlineto{\pgfqpoint{5.229830in}{0.912938in}}%
\pgfpathlineto{\pgfqpoint{5.253709in}{0.911129in}}%
\pgfpathlineto{\pgfqpoint{5.277588in}{0.910240in}}%
\pgfpathlineto{\pgfqpoint{5.301467in}{0.909871in}}%
\pgfpathlineto{\pgfqpoint{5.325346in}{0.909748in}}%
\pgfpathlineto{\pgfqpoint{5.349225in}{0.909709in}}%
\pgfpathlineto{\pgfqpoint{5.373104in}{0.909670in}}%
\pgfpathlineto{\pgfqpoint{5.396984in}{0.909617in}}%
\pgfpathlineto{\pgfqpoint{5.396984in}{1.589656in}}%
\pgfpathlineto{\pgfqpoint{5.396984in}{1.589656in}}%
\pgfpathlineto{\pgfqpoint{5.373104in}{1.589817in}}%
\pgfpathlineto{\pgfqpoint{5.349225in}{1.590069in}}%
\pgfpathlineto{\pgfqpoint{5.325346in}{1.590426in}}%
\pgfpathlineto{\pgfqpoint{5.301467in}{1.590876in}}%
\pgfpathlineto{\pgfqpoint{5.277588in}{1.591376in}}%
\pgfpathlineto{\pgfqpoint{5.253709in}{1.591839in}}%
\pgfpathlineto{\pgfqpoint{5.229830in}{1.592120in}}%
\pgfpathlineto{\pgfqpoint{5.205950in}{1.591979in}}%
\pgfpathlineto{\pgfqpoint{5.182071in}{1.591063in}}%
\pgfpathlineto{\pgfqpoint{5.158192in}{1.588879in}}%
\pgfpathlineto{\pgfqpoint{5.134313in}{1.584789in}}%
\pgfpathlineto{\pgfqpoint{5.110434in}{1.578010in}}%
\pgfpathlineto{\pgfqpoint{5.086555in}{1.567674in}}%
\pgfpathlineto{\pgfqpoint{5.062676in}{1.552891in}}%
\pgfpathlineto{\pgfqpoint{5.038796in}{1.532873in}}%
\pgfpathlineto{\pgfqpoint{5.014917in}{1.507076in}}%
\pgfpathlineto{\pgfqpoint{4.991038in}{1.475355in}}%
\pgfpathlineto{\pgfqpoint{4.967159in}{1.438134in}}%
\pgfpathlineto{\pgfqpoint{4.943280in}{1.396605in}}%
\pgfpathlineto{\pgfqpoint{4.919401in}{1.353101in}}%
\pgfpathlineto{\pgfqpoint{4.895522in}{1.312106in}}%
\pgfpathlineto{\pgfqpoint{4.871643in}{1.280848in}}%
\pgfpathlineto{\pgfqpoint{4.847763in}{1.258043in}}%
\pgfpathlineto{\pgfqpoint{4.823884in}{1.232264in}}%
\pgfpathlineto{\pgfqpoint{4.800005in}{1.199211in}}%
\pgfpathlineto{\pgfqpoint{4.776126in}{1.160370in}}%
\pgfpathlineto{\pgfqpoint{4.752247in}{1.119470in}}%
\pgfpathlineto{\pgfqpoint{4.728368in}{1.080725in}}%
\pgfpathlineto{\pgfqpoint{4.704489in}{1.046909in}}%
\pgfpathlineto{\pgfqpoint{4.680609in}{1.017813in}}%
\pgfpathlineto{\pgfqpoint{4.656730in}{0.991780in}}%
\pgfpathlineto{\pgfqpoint{4.632851in}{0.967924in}}%
\pgfpathlineto{\pgfqpoint{4.608972in}{0.946057in}}%
\pgfpathlineto{\pgfqpoint{4.585093in}{0.925772in}}%
\pgfpathlineto{\pgfqpoint{4.561214in}{0.906384in}}%
\pgfpathlineto{\pgfqpoint{4.537335in}{0.887604in}}%
\pgfpathlineto{\pgfqpoint{4.513455in}{0.869889in}}%
\pgfpathlineto{\pgfqpoint{4.489576in}{0.853942in}}%
\pgfpathlineto{\pgfqpoint{4.465697in}{0.839989in}}%
\pgfpathlineto{\pgfqpoint{4.441818in}{0.827766in}}%
\pgfpathlineto{\pgfqpoint{4.417939in}{0.817170in}}%
\pgfpathlineto{\pgfqpoint{4.394060in}{0.808533in}}%
\pgfpathlineto{\pgfqpoint{4.370181in}{0.802229in}}%
\pgfpathlineto{\pgfqpoint{4.346301in}{0.798177in}}%
\pgfpathlineto{\pgfqpoint{4.322422in}{0.795993in}}%
\pgfpathlineto{\pgfqpoint{4.298543in}{0.795436in}}%
\pgfpathlineto{\pgfqpoint{4.274664in}{0.796431in}}%
\pgfpathlineto{\pgfqpoint{4.250785in}{0.798761in}}%
\pgfpathlineto{\pgfqpoint{4.226906in}{0.801992in}}%
\pgfpathlineto{\pgfqpoint{4.203027in}{0.805743in}}%
\pgfpathlineto{\pgfqpoint{4.179148in}{0.809945in}}%
\pgfpathlineto{\pgfqpoint{4.155268in}{0.814734in}}%
\pgfpathlineto{\pgfqpoint{4.131389in}{0.820169in}}%
\pgfpathlineto{\pgfqpoint{4.107510in}{0.826090in}}%
\pgfpathlineto{\pgfqpoint{4.083631in}{0.832325in}}%
\pgfpathlineto{\pgfqpoint{4.059752in}{0.838922in}}%
\pgfpathlineto{\pgfqpoint{4.035873in}{0.846097in}}%
\pgfpathlineto{\pgfqpoint{4.011994in}{0.853967in}}%
\pgfpathlineto{\pgfqpoint{3.988114in}{0.862454in}}%
\pgfpathlineto{\pgfqpoint{3.964235in}{0.871510in}}%
\pgfpathlineto{\pgfqpoint{3.940356in}{0.881370in}}%
\pgfpathlineto{\pgfqpoint{3.916477in}{0.892478in}}%
\pgfpathlineto{\pgfqpoint{3.892598in}{0.905125in}}%
\pgfpathlineto{\pgfqpoint{3.868719in}{0.919349in}}%
\pgfpathlineto{\pgfqpoint{3.844840in}{0.935349in}}%
\pgfpathlineto{\pgfqpoint{3.820960in}{0.953769in}}%
\pgfpathlineto{\pgfqpoint{3.797081in}{0.975084in}}%
\pgfpathlineto{\pgfqpoint{3.773202in}{0.998864in}}%
\pgfpathlineto{\pgfqpoint{3.749323in}{1.024157in}}%
\pgfpathlineto{\pgfqpoint{3.725444in}{1.050394in}}%
\pgfpathlineto{\pgfqpoint{3.701565in}{1.077747in}}%
\pgfpathlineto{\pgfqpoint{3.677686in}{1.106848in}}%
\pgfpathlineto{\pgfqpoint{3.653807in}{1.138317in}}%
\pgfpathlineto{\pgfqpoint{3.629927in}{1.173032in}}%
\pgfpathlineto{\pgfqpoint{3.606048in}{1.213004in}}%
\pgfpathlineto{\pgfqpoint{3.582169in}{1.259818in}}%
\pgfpathlineto{\pgfqpoint{3.558290in}{1.312060in}}%
\pgfpathlineto{\pgfqpoint{3.534411in}{1.366416in}}%
\pgfpathlineto{\pgfqpoint{3.510532in}{1.419309in}}%
\pgfpathlineto{\pgfqpoint{3.486653in}{1.467565in}}%
\pgfpathlineto{\pgfqpoint{3.462773in}{1.508795in}}%
\pgfpathlineto{\pgfqpoint{3.438894in}{1.541643in}}%
\pgfpathlineto{\pgfqpoint{3.415015in}{1.565840in}}%
\pgfpathlineto{\pgfqpoint{3.391136in}{1.582072in}}%
\pgfpathlineto{\pgfqpoint{3.367257in}{1.591696in}}%
\pgfpathlineto{\pgfqpoint{3.343378in}{1.596384in}}%
\pgfpathlineto{\pgfqpoint{3.319499in}{1.597797in}}%
\pgfpathlineto{\pgfqpoint{3.295619in}{1.597350in}}%
\pgfpathlineto{\pgfqpoint{3.271740in}{1.596086in}}%
\pgfpathlineto{\pgfqpoint{3.247861in}{1.594658in}}%
\pgfpathlineto{\pgfqpoint{3.223982in}{1.593404in}}%
\pgfpathlineto{\pgfqpoint{3.200103in}{1.592445in}}%
\pgfpathlineto{\pgfqpoint{3.176224in}{1.591768in}}%
\pgfpathlineto{\pgfqpoint{3.152345in}{1.591318in}}%
\pgfpathlineto{\pgfqpoint{3.128465in}{1.591032in}}%
\pgfpathlineto{\pgfqpoint{3.104586in}{1.590869in}}%
\pgfpathlineto{\pgfqpoint{3.080707in}{1.590796in}}%
\pgfpathlineto{\pgfqpoint{3.056828in}{1.590801in}}%
\pgfpathlineto{\pgfqpoint{3.032949in}{1.590864in}}%
\pgfpathclose%
\pgfusepath{fill}%
}%
\begin{pgfscope}%
\pgfsys@transformshift{0.000000in}{0.000000in}%
\pgfsys@useobject{currentmarker}{}%
\end{pgfscope}%
\end{pgfscope}%
\begin{pgfscope}%
\pgfsetbuttcap%
\pgfsetroundjoin%
\definecolor{currentfill}{rgb}{0.000000,0.000000,0.000000}%
\pgfsetfillcolor{currentfill}%
\pgfsetlinewidth{0.803000pt}%
\definecolor{currentstroke}{rgb}{0.000000,0.000000,0.000000}%
\pgfsetstrokecolor{currentstroke}%
\pgfsetdash{}{0pt}%
\pgfsys@defobject{currentmarker}{\pgfqpoint{0.000000in}{-0.048611in}}{\pgfqpoint{0.000000in}{0.000000in}}{%
\pgfpathmoveto{\pgfqpoint{0.000000in}{0.000000in}}%
\pgfpathlineto{\pgfqpoint{0.000000in}{-0.048611in}}%
\pgfusepath{stroke,fill}%
}%
\begin{pgfscope}%
\pgfsys@transformshift{3.476205in}{0.398220in}%
\pgfsys@useobject{currentmarker}{}%
\end{pgfscope}%
\end{pgfscope}%
\begin{pgfscope}%
\definecolor{textcolor}{rgb}{0.000000,0.000000,0.000000}%
\pgfsetstrokecolor{textcolor}%
\pgfsetfillcolor{textcolor}%
\pgftext[x=3.476205in,y=0.300998in,,top]{\color{textcolor}\rmfamily\fontsize{10.000000}{12.000000}\selectfont \(\displaystyle {\ensuremath{-}5}\)}%
\end{pgfscope}%
\begin{pgfscope}%
\pgfsetbuttcap%
\pgfsetroundjoin%
\definecolor{currentfill}{rgb}{0.000000,0.000000,0.000000}%
\pgfsetfillcolor{currentfill}%
\pgfsetlinewidth{0.803000pt}%
\definecolor{currentstroke}{rgb}{0.000000,0.000000,0.000000}%
\pgfsetstrokecolor{currentstroke}%
\pgfsetdash{}{0pt}%
\pgfsys@defobject{currentmarker}{\pgfqpoint{0.000000in}{-0.048611in}}{\pgfqpoint{0.000000in}{0.000000in}}{%
\pgfpathmoveto{\pgfqpoint{0.000000in}{0.000000in}}%
\pgfpathlineto{\pgfqpoint{0.000000in}{-0.048611in}}%
\pgfusepath{stroke,fill}%
}%
\begin{pgfscope}%
\pgfsys@transformshift{4.214966in}{0.398220in}%
\pgfsys@useobject{currentmarker}{}%
\end{pgfscope}%
\end{pgfscope}%
\begin{pgfscope}%
\definecolor{textcolor}{rgb}{0.000000,0.000000,0.000000}%
\pgfsetstrokecolor{textcolor}%
\pgfsetfillcolor{textcolor}%
\pgftext[x=4.214966in,y=0.300998in,,top]{\color{textcolor}\rmfamily\fontsize{10.000000}{12.000000}\selectfont \(\displaystyle {0}\)}%
\end{pgfscope}%
\begin{pgfscope}%
\pgfsetbuttcap%
\pgfsetroundjoin%
\definecolor{currentfill}{rgb}{0.000000,0.000000,0.000000}%
\pgfsetfillcolor{currentfill}%
\pgfsetlinewidth{0.803000pt}%
\definecolor{currentstroke}{rgb}{0.000000,0.000000,0.000000}%
\pgfsetstrokecolor{currentstroke}%
\pgfsetdash{}{0pt}%
\pgfsys@defobject{currentmarker}{\pgfqpoint{0.000000in}{-0.048611in}}{\pgfqpoint{0.000000in}{0.000000in}}{%
\pgfpathmoveto{\pgfqpoint{0.000000in}{0.000000in}}%
\pgfpathlineto{\pgfqpoint{0.000000in}{-0.048611in}}%
\pgfusepath{stroke,fill}%
}%
\begin{pgfscope}%
\pgfsys@transformshift{4.953727in}{0.398220in}%
\pgfsys@useobject{currentmarker}{}%
\end{pgfscope}%
\end{pgfscope}%
\begin{pgfscope}%
\definecolor{textcolor}{rgb}{0.000000,0.000000,0.000000}%
\pgfsetstrokecolor{textcolor}%
\pgfsetfillcolor{textcolor}%
\pgftext[x=4.953727in,y=0.300998in,,top]{\color{textcolor}\rmfamily\fontsize{10.000000}{12.000000}\selectfont \(\displaystyle {5}\)}%
\end{pgfscope}%
\begin{pgfscope}%
\definecolor{textcolor}{rgb}{0.000000,0.000000,0.000000}%
\pgfsetstrokecolor{textcolor}%
\pgfsetfillcolor{textcolor}%
\pgftext[x=4.214966in,y=0.122109in,,top]{\color{textcolor}\rmfamily\fontsize{10.000000}{12.000000}\selectfont \(\displaystyle x\)}%
\end{pgfscope}%
\begin{pgfscope}%
\pgfsetbuttcap%
\pgfsetroundjoin%
\definecolor{currentfill}{rgb}{0.000000,0.000000,0.000000}%
\pgfsetfillcolor{currentfill}%
\pgfsetlinewidth{0.803000pt}%
\definecolor{currentstroke}{rgb}{0.000000,0.000000,0.000000}%
\pgfsetstrokecolor{currentstroke}%
\pgfsetdash{}{0pt}%
\pgfsys@defobject{currentmarker}{\pgfqpoint{-0.048611in}{0.000000in}}{\pgfqpoint{-0.000000in}{0.000000in}}{%
\pgfpathmoveto{\pgfqpoint{-0.000000in}{0.000000in}}%
\pgfpathlineto{\pgfqpoint{-0.048611in}{0.000000in}}%
\pgfusepath{stroke,fill}%
}%
\begin{pgfscope}%
\pgfsys@transformshift{3.032949in}{0.398220in}%
\pgfsys@useobject{currentmarker}{}%
\end{pgfscope}%
\end{pgfscope}%
\begin{pgfscope}%
\pgfsetbuttcap%
\pgfsetroundjoin%
\definecolor{currentfill}{rgb}{0.000000,0.000000,0.000000}%
\pgfsetfillcolor{currentfill}%
\pgfsetlinewidth{0.803000pt}%
\definecolor{currentstroke}{rgb}{0.000000,0.000000,0.000000}%
\pgfsetstrokecolor{currentstroke}%
\pgfsetdash{}{0pt}%
\pgfsys@defobject{currentmarker}{\pgfqpoint{-0.048611in}{0.000000in}}{\pgfqpoint{-0.000000in}{0.000000in}}{%
\pgfpathmoveto{\pgfqpoint{-0.000000in}{0.000000in}}%
\pgfpathlineto{\pgfqpoint{-0.048611in}{0.000000in}}%
\pgfusepath{stroke,fill}%
}%
\begin{pgfscope}%
\pgfsys@transformshift{3.032949in}{0.824094in}%
\pgfsys@useobject{currentmarker}{}%
\end{pgfscope}%
\end{pgfscope}%
\begin{pgfscope}%
\pgfsetbuttcap%
\pgfsetroundjoin%
\definecolor{currentfill}{rgb}{0.000000,0.000000,0.000000}%
\pgfsetfillcolor{currentfill}%
\pgfsetlinewidth{0.803000pt}%
\definecolor{currentstroke}{rgb}{0.000000,0.000000,0.000000}%
\pgfsetstrokecolor{currentstroke}%
\pgfsetdash{}{0pt}%
\pgfsys@defobject{currentmarker}{\pgfqpoint{-0.048611in}{0.000000in}}{\pgfqpoint{-0.000000in}{0.000000in}}{%
\pgfpathmoveto{\pgfqpoint{-0.000000in}{0.000000in}}%
\pgfpathlineto{\pgfqpoint{-0.048611in}{0.000000in}}%
\pgfusepath{stroke,fill}%
}%
\begin{pgfscope}%
\pgfsys@transformshift{3.032949in}{1.249969in}%
\pgfsys@useobject{currentmarker}{}%
\end{pgfscope}%
\end{pgfscope}%
\begin{pgfscope}%
\pgfsetbuttcap%
\pgfsetroundjoin%
\definecolor{currentfill}{rgb}{0.000000,0.000000,0.000000}%
\pgfsetfillcolor{currentfill}%
\pgfsetlinewidth{0.803000pt}%
\definecolor{currentstroke}{rgb}{0.000000,0.000000,0.000000}%
\pgfsetstrokecolor{currentstroke}%
\pgfsetdash{}{0pt}%
\pgfsys@defobject{currentmarker}{\pgfqpoint{-0.048611in}{0.000000in}}{\pgfqpoint{-0.000000in}{0.000000in}}{%
\pgfpathmoveto{\pgfqpoint{-0.000000in}{0.000000in}}%
\pgfpathlineto{\pgfqpoint{-0.048611in}{0.000000in}}%
\pgfusepath{stroke,fill}%
}%
\begin{pgfscope}%
\pgfsys@transformshift{3.032949in}{1.675843in}%
\pgfsys@useobject{currentmarker}{}%
\end{pgfscope}%
\end{pgfscope}%
\begin{pgfscope}%
\pgfsetbuttcap%
\pgfsetroundjoin%
\definecolor{currentfill}{rgb}{0.000000,0.000000,0.000000}%
\pgfsetfillcolor{currentfill}%
\pgfsetlinewidth{0.803000pt}%
\definecolor{currentstroke}{rgb}{0.000000,0.000000,0.000000}%
\pgfsetstrokecolor{currentstroke}%
\pgfsetdash{}{0pt}%
\pgfsys@defobject{currentmarker}{\pgfqpoint{-0.048611in}{0.000000in}}{\pgfqpoint{-0.000000in}{0.000000in}}{%
\pgfpathmoveto{\pgfqpoint{-0.000000in}{0.000000in}}%
\pgfpathlineto{\pgfqpoint{-0.048611in}{0.000000in}}%
\pgfusepath{stroke,fill}%
}%
\begin{pgfscope}%
\pgfsys@transformshift{3.032949in}{2.101717in}%
\pgfsys@useobject{currentmarker}{}%
\end{pgfscope}%
\end{pgfscope}%
\begin{pgfscope}%
\pgfpathrectangle{\pgfqpoint{3.032949in}{0.398220in}}{\pgfqpoint{2.364035in}{1.703497in}}%
\pgfusepath{clip}%
\pgfsetrectcap%
\pgfsetroundjoin%
\pgfsetlinewidth{0.752812pt}%
\definecolor{currentstroke}{rgb}{0.000000,0.000000,0.000000}%
\pgfsetstrokecolor{currentstroke}%
\pgfsetdash{}{0pt}%
\pgfpathmoveto{\pgfqpoint{3.132019in}{2.111717in}}%
\pgfpathlineto{\pgfqpoint{3.152345in}{2.060587in}}%
\pgfpathlineto{\pgfqpoint{3.176224in}{2.001854in}}%
\pgfpathlineto{\pgfqpoint{3.200103in}{1.944456in}}%
\pgfpathlineto{\pgfqpoint{3.223982in}{1.888392in}}%
\pgfpathlineto{\pgfqpoint{3.247861in}{1.833663in}}%
\pgfpathlineto{\pgfqpoint{3.271740in}{1.780269in}}%
\pgfpathlineto{\pgfqpoint{3.295619in}{1.728210in}}%
\pgfpathlineto{\pgfqpoint{3.319499in}{1.677486in}}%
\pgfpathlineto{\pgfqpoint{3.343378in}{1.628096in}}%
\pgfpathlineto{\pgfqpoint{3.367257in}{1.580042in}}%
\pgfpathlineto{\pgfqpoint{3.391136in}{1.533322in}}%
\pgfpathlineto{\pgfqpoint{3.415015in}{1.487937in}}%
\pgfpathlineto{\pgfqpoint{3.438894in}{1.443887in}}%
\pgfpathlineto{\pgfqpoint{3.462773in}{1.401172in}}%
\pgfpathlineto{\pgfqpoint{3.486653in}{1.359791in}}%
\pgfpathlineto{\pgfqpoint{3.510532in}{1.319746in}}%
\pgfpathlineto{\pgfqpoint{3.534411in}{1.281035in}}%
\pgfpathlineto{\pgfqpoint{3.558290in}{1.243660in}}%
\pgfpathlineto{\pgfqpoint{3.582169in}{1.207619in}}%
\pgfpathlineto{\pgfqpoint{3.606048in}{1.172913in}}%
\pgfpathlineto{\pgfqpoint{3.629927in}{1.139541in}}%
\pgfpathlineto{\pgfqpoint{3.653807in}{1.107505in}}%
\pgfpathlineto{\pgfqpoint{3.677686in}{1.076803in}}%
\pgfpathlineto{\pgfqpoint{3.701565in}{1.047437in}}%
\pgfpathlineto{\pgfqpoint{3.725444in}{1.019405in}}%
\pgfpathlineto{\pgfqpoint{3.749323in}{0.992708in}}%
\pgfpathlineto{\pgfqpoint{3.773202in}{0.967346in}}%
\pgfpathlineto{\pgfqpoint{3.797081in}{0.943318in}}%
\pgfpathlineto{\pgfqpoint{3.820960in}{0.920626in}}%
\pgfpathlineto{\pgfqpoint{3.844840in}{0.899268in}}%
\pgfpathlineto{\pgfqpoint{3.868719in}{0.879246in}}%
\pgfpathlineto{\pgfqpoint{3.892598in}{0.860558in}}%
\pgfpathlineto{\pgfqpoint{3.916477in}{0.843205in}}%
\pgfpathlineto{\pgfqpoint{3.940356in}{0.827187in}}%
\pgfpathlineto{\pgfqpoint{3.964235in}{0.812503in}}%
\pgfpathlineto{\pgfqpoint{3.988114in}{0.799155in}}%
\pgfpathlineto{\pgfqpoint{4.011994in}{0.787141in}}%
\pgfpathlineto{\pgfqpoint{4.035873in}{0.776462in}}%
\pgfpathlineto{\pgfqpoint{4.059752in}{0.767118in}}%
\pgfpathlineto{\pgfqpoint{4.083631in}{0.759109in}}%
\pgfpathlineto{\pgfqpoint{4.107510in}{0.752435in}}%
\pgfpathlineto{\pgfqpoint{4.131389in}{0.747096in}}%
\pgfpathlineto{\pgfqpoint{4.155268in}{0.743091in}}%
\pgfpathlineto{\pgfqpoint{4.179148in}{0.740421in}}%
\pgfpathlineto{\pgfqpoint{4.203027in}{0.739086in}}%
\pgfpathlineto{\pgfqpoint{4.226906in}{0.739086in}}%
\pgfpathlineto{\pgfqpoint{4.250785in}{0.740421in}}%
\pgfpathlineto{\pgfqpoint{4.274664in}{0.743091in}}%
\pgfpathlineto{\pgfqpoint{4.298543in}{0.747096in}}%
\pgfpathlineto{\pgfqpoint{4.322422in}{0.752435in}}%
\pgfpathlineto{\pgfqpoint{4.346301in}{0.759109in}}%
\pgfpathlineto{\pgfqpoint{4.370181in}{0.767118in}}%
\pgfpathlineto{\pgfqpoint{4.394060in}{0.776462in}}%
\pgfpathlineto{\pgfqpoint{4.417939in}{0.787141in}}%
\pgfpathlineto{\pgfqpoint{4.441818in}{0.799155in}}%
\pgfpathlineto{\pgfqpoint{4.465697in}{0.812503in}}%
\pgfpathlineto{\pgfqpoint{4.489576in}{0.827187in}}%
\pgfpathlineto{\pgfqpoint{4.513455in}{0.843205in}}%
\pgfpathlineto{\pgfqpoint{4.537335in}{0.860558in}}%
\pgfpathlineto{\pgfqpoint{4.561214in}{0.879246in}}%
\pgfpathlineto{\pgfqpoint{4.585093in}{0.899268in}}%
\pgfpathlineto{\pgfqpoint{4.608972in}{0.920626in}}%
\pgfpathlineto{\pgfqpoint{4.632851in}{0.943318in}}%
\pgfpathlineto{\pgfqpoint{4.656730in}{0.967346in}}%
\pgfpathlineto{\pgfqpoint{4.680609in}{0.992708in}}%
\pgfpathlineto{\pgfqpoint{4.704489in}{1.019405in}}%
\pgfpathlineto{\pgfqpoint{4.728368in}{1.047437in}}%
\pgfpathlineto{\pgfqpoint{4.752247in}{1.076803in}}%
\pgfpathlineto{\pgfqpoint{4.776126in}{1.107505in}}%
\pgfpathlineto{\pgfqpoint{4.800005in}{1.139541in}}%
\pgfpathlineto{\pgfqpoint{4.823884in}{1.172913in}}%
\pgfpathlineto{\pgfqpoint{4.847763in}{1.207619in}}%
\pgfpathlineto{\pgfqpoint{4.871643in}{1.243660in}}%
\pgfpathlineto{\pgfqpoint{4.895522in}{1.281035in}}%
\pgfpathlineto{\pgfqpoint{4.919401in}{1.319746in}}%
\pgfpathlineto{\pgfqpoint{4.943280in}{1.359791in}}%
\pgfpathlineto{\pgfqpoint{4.967159in}{1.401172in}}%
\pgfpathlineto{\pgfqpoint{4.991038in}{1.443887in}}%
\pgfpathlineto{\pgfqpoint{5.014917in}{1.487937in}}%
\pgfpathlineto{\pgfqpoint{5.038796in}{1.533322in}}%
\pgfpathlineto{\pgfqpoint{5.062676in}{1.580042in}}%
\pgfpathlineto{\pgfqpoint{5.086555in}{1.628096in}}%
\pgfpathlineto{\pgfqpoint{5.110434in}{1.677486in}}%
\pgfpathlineto{\pgfqpoint{5.134313in}{1.728210in}}%
\pgfpathlineto{\pgfqpoint{5.158192in}{1.780269in}}%
\pgfpathlineto{\pgfqpoint{5.182071in}{1.833663in}}%
\pgfpathlineto{\pgfqpoint{5.205950in}{1.888392in}}%
\pgfpathlineto{\pgfqpoint{5.229830in}{1.944456in}}%
\pgfpathlineto{\pgfqpoint{5.253709in}{2.001854in}}%
\pgfpathlineto{\pgfqpoint{5.277588in}{2.060587in}}%
\pgfpathlineto{\pgfqpoint{5.297914in}{2.111717in}}%
\pgfusepath{stroke}%
\end{pgfscope}%
\begin{pgfscope}%
\pgfpathrectangle{\pgfqpoint{3.032949in}{0.398220in}}{\pgfqpoint{2.364035in}{1.703497in}}%
\pgfusepath{clip}%
\pgfsetbuttcap%
\pgfsetroundjoin%
\definecolor{currentfill}{rgb}{0.000000,0.500000,0.000000}%
\pgfsetfillcolor{currentfill}%
\pgfsetlinewidth{1.003750pt}%
\definecolor{currentstroke}{rgb}{0.000000,0.500000,0.000000}%
\pgfsetstrokecolor{currentstroke}%
\pgfsetdash{}{0pt}%
\pgfsys@defobject{currentmarker}{\pgfqpoint{-0.020833in}{-0.020833in}}{\pgfqpoint{0.020833in}{0.020833in}}{%
\pgfpathmoveto{\pgfqpoint{0.000000in}{-0.020833in}}%
\pgfpathcurveto{\pgfqpoint{0.005525in}{-0.020833in}}{\pgfqpoint{0.010825in}{-0.018638in}}{\pgfqpoint{0.014731in}{-0.014731in}}%
\pgfpathcurveto{\pgfqpoint{0.018638in}{-0.010825in}}{\pgfqpoint{0.020833in}{-0.005525in}}{\pgfqpoint{0.020833in}{0.000000in}}%
\pgfpathcurveto{\pgfqpoint{0.020833in}{0.005525in}}{\pgfqpoint{0.018638in}{0.010825in}}{\pgfqpoint{0.014731in}{0.014731in}}%
\pgfpathcurveto{\pgfqpoint{0.010825in}{0.018638in}}{\pgfqpoint{0.005525in}{0.020833in}}{\pgfqpoint{0.000000in}{0.020833in}}%
\pgfpathcurveto{\pgfqpoint{-0.005525in}{0.020833in}}{\pgfqpoint{-0.010825in}{0.018638in}}{\pgfqpoint{-0.014731in}{0.014731in}}%
\pgfpathcurveto{\pgfqpoint{-0.018638in}{0.010825in}}{\pgfqpoint{-0.020833in}{0.005525in}}{\pgfqpoint{-0.020833in}{0.000000in}}%
\pgfpathcurveto{\pgfqpoint{-0.020833in}{-0.005525in}}{\pgfqpoint{-0.018638in}{-0.010825in}}{\pgfqpoint{-0.014731in}{-0.014731in}}%
\pgfpathcurveto{\pgfqpoint{-0.010825in}{-0.018638in}}{\pgfqpoint{-0.005525in}{-0.020833in}}{\pgfqpoint{0.000000in}{-0.020833in}}%
\pgfpathclose%
\pgfusepath{stroke,fill}%
}%
\begin{pgfscope}%
\pgfsys@transformshift{3.697834in}{0.483395in}%
\pgfsys@useobject{currentmarker}{}%
\end{pgfscope}%
\begin{pgfscope}%
\pgfsys@transformshift{3.812752in}{0.483395in}%
\pgfsys@useobject{currentmarker}{}%
\end{pgfscope}%
\begin{pgfscope}%
\pgfsys@transformshift{3.927670in}{0.483395in}%
\pgfsys@useobject{currentmarker}{}%
\end{pgfscope}%
\begin{pgfscope}%
\pgfsys@transformshift{4.042589in}{0.483395in}%
\pgfsys@useobject{currentmarker}{}%
\end{pgfscope}%
\begin{pgfscope}%
\pgfsys@transformshift{4.157507in}{0.483395in}%
\pgfsys@useobject{currentmarker}{}%
\end{pgfscope}%
\begin{pgfscope}%
\pgfsys@transformshift{4.272425in}{0.483395in}%
\pgfsys@useobject{currentmarker}{}%
\end{pgfscope}%
\begin{pgfscope}%
\pgfsys@transformshift{4.387344in}{0.483395in}%
\pgfsys@useobject{currentmarker}{}%
\end{pgfscope}%
\begin{pgfscope}%
\pgfsys@transformshift{4.502262in}{0.483395in}%
\pgfsys@useobject{currentmarker}{}%
\end{pgfscope}%
\begin{pgfscope}%
\pgfsys@transformshift{4.617180in}{0.483395in}%
\pgfsys@useobject{currentmarker}{}%
\end{pgfscope}%
\begin{pgfscope}%
\pgfsys@transformshift{4.732099in}{0.483395in}%
\pgfsys@useobject{currentmarker}{}%
\end{pgfscope}%
\end{pgfscope}%
\begin{pgfscope}%
\pgfpathrectangle{\pgfqpoint{3.032949in}{0.398220in}}{\pgfqpoint{2.364035in}{1.703497in}}%
\pgfusepath{clip}%
\pgfsetrectcap%
\pgfsetroundjoin%
\pgfsetlinewidth{0.752812pt}%
\definecolor{currentstroke}{rgb}{0.631373,0.062745,0.207843}%
\pgfsetstrokecolor{currentstroke}%
\pgfsetdash{}{0pt}%
\pgfpathmoveto{\pgfqpoint{3.032949in}{1.249970in}}%
\pgfpathlineto{\pgfqpoint{3.056828in}{1.249971in}}%
\pgfpathlineto{\pgfqpoint{3.080707in}{1.249973in}}%
\pgfpathlineto{\pgfqpoint{3.104586in}{1.249976in}}%
\pgfpathlineto{\pgfqpoint{3.128465in}{1.249980in}}%
\pgfpathlineto{\pgfqpoint{3.152345in}{1.249987in}}%
\pgfpathlineto{\pgfqpoint{3.176224in}{1.249996in}}%
\pgfpathlineto{\pgfqpoint{3.200103in}{1.250008in}}%
\pgfpathlineto{\pgfqpoint{3.223982in}{1.250019in}}%
\pgfpathlineto{\pgfqpoint{3.247861in}{1.250025in}}%
\pgfpathlineto{\pgfqpoint{3.271740in}{1.250013in}}%
\pgfpathlineto{\pgfqpoint{3.295619in}{1.249959in}}%
\pgfpathlineto{\pgfqpoint{3.319499in}{1.249823in}}%
\pgfpathlineto{\pgfqpoint{3.343378in}{1.249536in}}%
\pgfpathlineto{\pgfqpoint{3.367257in}{1.248992in}}%
\pgfpathlineto{\pgfqpoint{3.391136in}{1.248032in}}%
\pgfpathlineto{\pgfqpoint{3.415015in}{1.246431in}}%
\pgfpathlineto{\pgfqpoint{3.438894in}{1.243887in}}%
\pgfpathlineto{\pgfqpoint{3.462773in}{1.240011in}}%
\pgfpathlineto{\pgfqpoint{3.486653in}{1.234333in}}%
\pgfpathlineto{\pgfqpoint{3.510532in}{1.226317in}}%
\pgfpathlineto{\pgfqpoint{3.534411in}{1.215397in}}%
\pgfpathlineto{\pgfqpoint{3.558290in}{1.201038in}}%
\pgfpathlineto{\pgfqpoint{3.582169in}{1.182810in}}%
\pgfpathlineto{\pgfqpoint{3.606048in}{1.160480in}}%
\pgfpathlineto{\pgfqpoint{3.629927in}{1.134111in}}%
\pgfpathlineto{\pgfqpoint{3.653807in}{1.104138in}}%
\pgfpathlineto{\pgfqpoint{3.677686in}{1.071428in}}%
\pgfpathlineto{\pgfqpoint{3.701565in}{1.037275in}}%
\pgfpathlineto{\pgfqpoint{3.725444in}{1.003352in}}%
\pgfpathlineto{\pgfqpoint{3.749323in}{0.971585in}}%
\pgfpathlineto{\pgfqpoint{3.773202in}{0.943979in}}%
\pgfpathlineto{\pgfqpoint{3.797081in}{0.922405in}}%
\pgfpathlineto{\pgfqpoint{3.820960in}{0.908379in}}%
\pgfpathlineto{\pgfqpoint{3.844840in}{0.902864in}}%
\pgfpathlineto{\pgfqpoint{3.868719in}{0.906136in}}%
\pgfpathlineto{\pgfqpoint{3.892598in}{0.917718in}}%
\pgfpathlineto{\pgfqpoint{3.916477in}{0.936408in}}%
\pgfpathlineto{\pgfqpoint{3.940356in}{0.960391in}}%
\pgfpathlineto{\pgfqpoint{3.964235in}{0.987401in}}%
\pgfpathlineto{\pgfqpoint{3.988114in}{1.014935in}}%
\pgfpathlineto{\pgfqpoint{4.011994in}{1.040467in}}%
\pgfpathlineto{\pgfqpoint{4.035873in}{1.061641in}}%
\pgfpathlineto{\pgfqpoint{4.059752in}{1.076449in}}%
\pgfpathlineto{\pgfqpoint{4.083631in}{1.083350in}}%
\pgfpathlineto{\pgfqpoint{4.107510in}{1.081370in}}%
\pgfpathlineto{\pgfqpoint{4.131389in}{1.070156in}}%
\pgfpathlineto{\pgfqpoint{4.155268in}{1.050010in}}%
\pgfpathlineto{\pgfqpoint{4.179148in}{1.021880in}}%
\pgfpathlineto{\pgfqpoint{4.203027in}{0.987325in}}%
\pgfpathlineto{\pgfqpoint{4.226906in}{0.948429in}}%
\pgfpathlineto{\pgfqpoint{4.250785in}{0.907666in}}%
\pgfpathlineto{\pgfqpoint{4.274664in}{0.867715in}}%
\pgfpathlineto{\pgfqpoint{4.298543in}{0.831229in}}%
\pgfpathlineto{\pgfqpoint{4.322422in}{0.800593in}}%
\pgfpathlineto{\pgfqpoint{4.346301in}{0.777685in}}%
\pgfpathlineto{\pgfqpoint{4.370181in}{0.763690in}}%
\pgfpathlineto{\pgfqpoint{4.394060in}{0.758990in}}%
\pgfpathlineto{\pgfqpoint{4.417939in}{0.763152in}}%
\pgfpathlineto{\pgfqpoint{4.441818in}{0.775025in}}%
\pgfpathlineto{\pgfqpoint{4.465697in}{0.792930in}}%
\pgfpathlineto{\pgfqpoint{4.489576in}{0.814915in}}%
\pgfpathlineto{\pgfqpoint{4.513455in}{0.839042in}}%
\pgfpathlineto{\pgfqpoint{4.537335in}{0.863649in}}%
\pgfpathlineto{\pgfqpoint{4.561214in}{0.887563in}}%
\pgfpathlineto{\pgfqpoint{4.585093in}{0.910217in}}%
\pgfpathlineto{\pgfqpoint{4.608972in}{0.931664in}}%
\pgfpathlineto{\pgfqpoint{4.632851in}{0.952483in}}%
\pgfpathlineto{\pgfqpoint{4.656730in}{0.973604in}}%
\pgfpathlineto{\pgfqpoint{4.680609in}{0.996070in}}%
\pgfpathlineto{\pgfqpoint{4.704489in}{1.020791in}}%
\pgfpathlineto{\pgfqpoint{4.728368in}{1.048330in}}%
\pgfpathlineto{\pgfqpoint{4.752247in}{1.078749in}}%
\pgfpathlineto{\pgfqpoint{4.776126in}{1.111554in}}%
\pgfpathlineto{\pgfqpoint{4.800005in}{1.145738in}}%
\pgfpathlineto{\pgfqpoint{4.823884in}{1.179911in}}%
\pgfpathlineto{\pgfqpoint{4.847763in}{1.212500in}}%
\pgfpathlineto{\pgfqpoint{4.871643in}{1.241965in}}%
\pgfpathlineto{\pgfqpoint{4.895522in}{1.267011in}}%
\pgfpathlineto{\pgfqpoint{4.919401in}{1.286744in}}%
\pgfpathlineto{\pgfqpoint{4.943280in}{1.300760in}}%
\pgfpathlineto{\pgfqpoint{4.967159in}{1.309152in}}%
\pgfpathlineto{\pgfqpoint{4.991038in}{1.312451in}}%
\pgfpathlineto{\pgfqpoint{5.014917in}{1.311508in}}%
\pgfpathlineto{\pgfqpoint{5.038796in}{1.307360in}}%
\pgfpathlineto{\pgfqpoint{5.062676in}{1.301083in}}%
\pgfpathlineto{\pgfqpoint{5.086555in}{1.293679in}}%
\pgfpathlineto{\pgfqpoint{5.110434in}{1.285987in}}%
\pgfpathlineto{\pgfqpoint{5.134313in}{1.278640in}}%
\pgfpathlineto{\pgfqpoint{5.158192in}{1.272056in}}%
\pgfpathlineto{\pgfqpoint{5.182071in}{1.266457in}}%
\pgfpathlineto{\pgfqpoint{5.205950in}{1.261909in}}%
\pgfpathlineto{\pgfqpoint{5.229830in}{1.258364in}}%
\pgfpathlineto{\pgfqpoint{5.253709in}{1.255702in}}%
\pgfpathlineto{\pgfqpoint{5.277588in}{1.253775in}}%
\pgfpathlineto{\pgfqpoint{5.301467in}{1.252425in}}%
\pgfpathlineto{\pgfqpoint{5.325346in}{1.251511in}}%
\pgfpathlineto{\pgfqpoint{5.349225in}{1.250911in}}%
\pgfpathlineto{\pgfqpoint{5.373104in}{1.250529in}}%
\pgfpathlineto{\pgfqpoint{5.396984in}{1.250293in}}%
\pgfusepath{stroke}%
\end{pgfscope}%
\begin{pgfscope}%
\pgfpathrectangle{\pgfqpoint{3.032949in}{0.398220in}}{\pgfqpoint{2.364035in}{1.703497in}}%
\pgfusepath{clip}%
\pgfsetrectcap%
\pgfsetroundjoin%
\pgfsetlinewidth{0.501875pt}%
\definecolor{currentstroke}{rgb}{0.713725,0.321569,0.337255}%
\pgfsetstrokecolor{currentstroke}%
\pgfsetdash{}{0pt}%
\pgfpathmoveto{\pgfqpoint{3.032949in}{1.175210in}}%
\pgfpathlineto{\pgfqpoint{3.056828in}{1.218741in}}%
\pgfpathlineto{\pgfqpoint{3.080707in}{1.263744in}}%
\pgfpathlineto{\pgfqpoint{3.104586in}{1.306894in}}%
\pgfpathlineto{\pgfqpoint{3.128465in}{1.346378in}}%
\pgfpathlineto{\pgfqpoint{3.152345in}{1.381904in}}%
\pgfpathlineto{\pgfqpoint{3.176224in}{1.414273in}}%
\pgfpathlineto{\pgfqpoint{3.200103in}{1.444690in}}%
\pgfpathlineto{\pgfqpoint{3.223982in}{1.474053in}}%
\pgfpathlineto{\pgfqpoint{3.247861in}{1.502435in}}%
\pgfpathlineto{\pgfqpoint{3.271740in}{1.528895in}}%
\pgfpathlineto{\pgfqpoint{3.295619in}{1.551645in}}%
\pgfpathlineto{\pgfqpoint{3.319499in}{1.568468in}}%
\pgfpathlineto{\pgfqpoint{3.343378in}{1.577243in}}%
\pgfpathlineto{\pgfqpoint{3.367257in}{1.576435in}}%
\pgfpathlineto{\pgfqpoint{3.391136in}{1.565406in}}%
\pgfpathlineto{\pgfqpoint{3.415015in}{1.544519in}}%
\pgfpathlineto{\pgfqpoint{3.438894in}{1.515010in}}%
\pgfpathlineto{\pgfqpoint{3.462773in}{1.478665in}}%
\pgfpathlineto{\pgfqpoint{3.486653in}{1.437400in}}%
\pgfpathlineto{\pgfqpoint{3.510532in}{1.392812in}}%
\pgfpathlineto{\pgfqpoint{3.534411in}{1.345848in}}%
\pgfpathlineto{\pgfqpoint{3.558290in}{1.296689in}}%
\pgfpathlineto{\pgfqpoint{3.582169in}{1.244910in}}%
\pgfpathlineto{\pgfqpoint{3.606048in}{1.189914in}}%
\pgfpathlineto{\pgfqpoint{3.629927in}{1.131511in}}%
\pgfpathlineto{\pgfqpoint{3.653807in}{1.070481in}}%
\pgfpathlineto{\pgfqpoint{3.677686in}{1.008908in}}%
\pgfpathlineto{\pgfqpoint{3.701565in}{0.950166in}}%
\pgfpathlineto{\pgfqpoint{3.725444in}{0.898507in}}%
\pgfpathlineto{\pgfqpoint{3.749323in}{0.858344in}}%
\pgfpathlineto{\pgfqpoint{3.773202in}{0.833384in}}%
\pgfpathlineto{\pgfqpoint{3.797081in}{0.825823in}}%
\pgfpathlineto{\pgfqpoint{3.820960in}{0.835782in}}%
\pgfpathlineto{\pgfqpoint{3.844840in}{0.861084in}}%
\pgfpathlineto{\pgfqpoint{3.868719in}{0.897447in}}%
\pgfpathlineto{\pgfqpoint{3.892598in}{0.939051in}}%
\pgfpathlineto{\pgfqpoint{3.916477in}{0.979399in}}%
\pgfpathlineto{\pgfqpoint{3.940356in}{1.012331in}}%
\pgfpathlineto{\pgfqpoint{3.964235in}{1.033015in}}%
\pgfpathlineto{\pgfqpoint{3.988114in}{1.038723in}}%
\pgfpathlineto{\pgfqpoint{4.011994in}{1.029254in}}%
\pgfpathlineto{\pgfqpoint{4.035873in}{1.006892in}}%
\pgfpathlineto{\pgfqpoint{4.059752in}{0.975904in}}%
\pgfpathlineto{\pgfqpoint{4.083631in}{0.941662in}}%
\pgfpathlineto{\pgfqpoint{4.107510in}{0.909547in}}%
\pgfpathlineto{\pgfqpoint{4.131389in}{0.883862in}}%
\pgfpathlineto{\pgfqpoint{4.155268in}{0.866984in}}%
\pgfpathlineto{\pgfqpoint{4.179148in}{0.858961in}}%
\pgfpathlineto{\pgfqpoint{4.203027in}{0.857663in}}%
\pgfpathlineto{\pgfqpoint{4.226906in}{0.859474in}}%
\pgfpathlineto{\pgfqpoint{4.250785in}{0.860351in}}%
\pgfpathlineto{\pgfqpoint{4.274664in}{0.856961in}}%
\pgfpathlineto{\pgfqpoint{4.298543in}{0.847571in}}%
\pgfpathlineto{\pgfqpoint{4.322422in}{0.832443in}}%
\pgfpathlineto{\pgfqpoint{4.346301in}{0.813635in}}%
\pgfpathlineto{\pgfqpoint{4.370181in}{0.794341in}}%
\pgfpathlineto{\pgfqpoint{4.394060in}{0.778010in}}%
\pgfpathlineto{\pgfqpoint{4.417939in}{0.767550in}}%
\pgfpathlineto{\pgfqpoint{4.441818in}{0.764822in}}%
\pgfpathlineto{\pgfqpoint{4.465697in}{0.770474in}}%
\pgfpathlineto{\pgfqpoint{4.489576in}{0.784060in}}%
\pgfpathlineto{\pgfqpoint{4.513455in}{0.804284in}}%
\pgfpathlineto{\pgfqpoint{4.537335in}{0.829301in}}%
\pgfpathlineto{\pgfqpoint{4.561214in}{0.857019in}}%
\pgfpathlineto{\pgfqpoint{4.585093in}{0.885426in}}%
\pgfpathlineto{\pgfqpoint{4.608972in}{0.912929in}}%
\pgfpathlineto{\pgfqpoint{4.632851in}{0.938666in}}%
\pgfpathlineto{\pgfqpoint{4.656730in}{0.962681in}}%
\pgfpathlineto{\pgfqpoint{4.680609in}{0.985871in}}%
\pgfpathlineto{\pgfqpoint{4.704489in}{1.009691in}}%
\pgfpathlineto{\pgfqpoint{4.728368in}{1.035667in}}%
\pgfpathlineto{\pgfqpoint{4.752247in}{1.064886in}}%
\pgfpathlineto{\pgfqpoint{4.776126in}{1.097624in}}%
\pgfpathlineto{\pgfqpoint{4.800005in}{1.133216in}}%
\pgfpathlineto{\pgfqpoint{4.823884in}{1.170215in}}%
\pgfpathlineto{\pgfqpoint{4.847763in}{1.206762in}}%
\pgfpathlineto{\pgfqpoint{4.871643in}{1.241042in}}%
\pgfpathlineto{\pgfqpoint{4.895522in}{1.271707in}}%
\pgfpathlineto{\pgfqpoint{4.919401in}{1.298133in}}%
\pgfpathlineto{\pgfqpoint{4.943280in}{1.320467in}}%
\pgfpathlineto{\pgfqpoint{4.967159in}{1.339457in}}%
\pgfpathlineto{\pgfqpoint{4.991038in}{1.356108in}}%
\pgfpathlineto{\pgfqpoint{5.014917in}{1.371272in}}%
\pgfpathlineto{\pgfqpoint{5.038796in}{1.385257in}}%
\pgfpathlineto{\pgfqpoint{5.062676in}{1.397574in}}%
\pgfpathlineto{\pgfqpoint{5.086555in}{1.406867in}}%
\pgfpathlineto{\pgfqpoint{5.110434in}{1.411092in}}%
\pgfpathlineto{\pgfqpoint{5.134313in}{1.407889in}}%
\pgfpathlineto{\pgfqpoint{5.158192in}{1.395130in}}%
\pgfpathlineto{\pgfqpoint{5.182071in}{1.371505in}}%
\pgfpathlineto{\pgfqpoint{5.205950in}{1.337020in}}%
\pgfpathlineto{\pgfqpoint{5.229830in}{1.293272in}}%
\pgfpathlineto{\pgfqpoint{5.253709in}{1.243399in}}%
\pgfpathlineto{\pgfqpoint{5.277588in}{1.191664in}}%
\pgfpathlineto{\pgfqpoint{5.301467in}{1.142756in}}%
\pgfpathlineto{\pgfqpoint{5.325346in}{1.100928in}}%
\pgfpathlineto{\pgfqpoint{5.349225in}{1.069197in}}%
\pgfpathlineto{\pgfqpoint{5.373104in}{1.048771in}}%
\pgfpathlineto{\pgfqpoint{5.396984in}{1.038861in}}%
\pgfusepath{stroke}%
\end{pgfscope}%
\begin{pgfscope}%
\pgfpathrectangle{\pgfqpoint{3.032949in}{0.398220in}}{\pgfqpoint{2.364035in}{1.703497in}}%
\pgfusepath{clip}%
\pgfsetrectcap%
\pgfsetroundjoin%
\pgfsetlinewidth{0.501875pt}%
\definecolor{currentstroke}{rgb}{0.713725,0.321569,0.337255}%
\pgfsetstrokecolor{currentstroke}%
\pgfsetdash{}{0pt}%
\pgfpathmoveto{\pgfqpoint{3.032949in}{1.328397in}}%
\pgfpathlineto{\pgfqpoint{3.056828in}{1.289199in}}%
\pgfpathlineto{\pgfqpoint{3.080707in}{1.260229in}}%
\pgfpathlineto{\pgfqpoint{3.104586in}{1.245231in}}%
\pgfpathlineto{\pgfqpoint{3.128465in}{1.244796in}}%
\pgfpathlineto{\pgfqpoint{3.152345in}{1.256326in}}%
\pgfpathlineto{\pgfqpoint{3.176224in}{1.274905in}}%
\pgfpathlineto{\pgfqpoint{3.200103in}{1.294770in}}%
\pgfpathlineto{\pgfqpoint{3.223982in}{1.310869in}}%
\pgfpathlineto{\pgfqpoint{3.247861in}{1.320021in}}%
\pgfpathlineto{\pgfqpoint{3.271740in}{1.321353in}}%
\pgfpathlineto{\pgfqpoint{3.295619in}{1.316002in}}%
\pgfpathlineto{\pgfqpoint{3.319499in}{1.306318in}}%
\pgfpathlineto{\pgfqpoint{3.343378in}{1.294915in}}%
\pgfpathlineto{\pgfqpoint{3.367257in}{1.283878in}}%
\pgfpathlineto{\pgfqpoint{3.391136in}{1.274325in}}%
\pgfpathlineto{\pgfqpoint{3.415015in}{1.266328in}}%
\pgfpathlineto{\pgfqpoint{3.438894in}{1.259117in}}%
\pgfpathlineto{\pgfqpoint{3.462773in}{1.251424in}}%
\pgfpathlineto{\pgfqpoint{3.486653in}{1.241847in}}%
\pgfpathlineto{\pgfqpoint{3.510532in}{1.229135in}}%
\pgfpathlineto{\pgfqpoint{3.534411in}{1.212372in}}%
\pgfpathlineto{\pgfqpoint{3.558290in}{1.191034in}}%
\pgfpathlineto{\pgfqpoint{3.582169in}{1.164989in}}%
\pgfpathlineto{\pgfqpoint{3.606048in}{1.134475in}}%
\pgfpathlineto{\pgfqpoint{3.629927in}{1.100125in}}%
\pgfpathlineto{\pgfqpoint{3.653807in}{1.063054in}}%
\pgfpathlineto{\pgfqpoint{3.677686in}{1.024955in}}%
\pgfpathlineto{\pgfqpoint{3.701565in}{0.988141in}}%
\pgfpathlineto{\pgfqpoint{3.725444in}{0.955408in}}%
\pgfpathlineto{\pgfqpoint{3.749323in}{0.929678in}}%
\pgfpathlineto{\pgfqpoint{3.773202in}{0.913464in}}%
\pgfpathlineto{\pgfqpoint{3.797081in}{0.908262in}}%
\pgfpathlineto{\pgfqpoint{3.820960in}{0.914043in}}%
\pgfpathlineto{\pgfqpoint{3.844840in}{0.929035in}}%
\pgfpathlineto{\pgfqpoint{3.868719in}{0.949889in}}%
\pgfpathlineto{\pgfqpoint{3.892598in}{0.972258in}}%
\pgfpathlineto{\pgfqpoint{3.916477in}{0.991677in}}%
\pgfpathlineto{\pgfqpoint{3.940356in}{1.004540in}}%
\pgfpathlineto{\pgfqpoint{3.964235in}{1.008924in}}%
\pgfpathlineto{\pgfqpoint{3.988114in}{1.004992in}}%
\pgfpathlineto{\pgfqpoint{4.011994in}{0.994856in}}%
\pgfpathlineto{\pgfqpoint{4.035873in}{0.981903in}}%
\pgfpathlineto{\pgfqpoint{4.059752in}{0.969781in}}%
\pgfpathlineto{\pgfqpoint{4.083631in}{0.961328in}}%
\pgfpathlineto{\pgfqpoint{4.107510in}{0.957763in}}%
\pgfpathlineto{\pgfqpoint{4.131389in}{0.958351in}}%
\pgfpathlineto{\pgfqpoint{4.155268in}{0.960617in}}%
\pgfpathlineto{\pgfqpoint{4.179148in}{0.961037in}}%
\pgfpathlineto{\pgfqpoint{4.203027in}{0.955989in}}%
\pgfpathlineto{\pgfqpoint{4.226906in}{0.942738in}}%
\pgfpathlineto{\pgfqpoint{4.250785in}{0.920185in}}%
\pgfpathlineto{\pgfqpoint{4.274664in}{0.889219in}}%
\pgfpathlineto{\pgfqpoint{4.298543in}{0.852587in}}%
\pgfpathlineto{\pgfqpoint{4.322422in}{0.814359in}}%
\pgfpathlineto{\pgfqpoint{4.346301in}{0.779115in}}%
\pgfpathlineto{\pgfqpoint{4.370181in}{0.751104in}}%
\pgfpathlineto{\pgfqpoint{4.394060in}{0.733535in}}%
\pgfpathlineto{\pgfqpoint{4.417939in}{0.728148in}}%
\pgfpathlineto{\pgfqpoint{4.441818in}{0.735081in}}%
\pgfpathlineto{\pgfqpoint{4.465697in}{0.753003in}}%
\pgfpathlineto{\pgfqpoint{4.489576in}{0.779455in}}%
\pgfpathlineto{\pgfqpoint{4.513455in}{0.811313in}}%
\pgfpathlineto{\pgfqpoint{4.537335in}{0.845330in}}%
\pgfpathlineto{\pgfqpoint{4.561214in}{0.878668in}}%
\pgfpathlineto{\pgfqpoint{4.585093in}{0.909338in}}%
\pgfpathlineto{\pgfqpoint{4.608972in}{0.936461in}}%
\pgfpathlineto{\pgfqpoint{4.632851in}{0.960283in}}%
\pgfpathlineto{\pgfqpoint{4.656730in}{0.981933in}}%
\pgfpathlineto{\pgfqpoint{4.680609in}{1.002999in}}%
\pgfpathlineto{\pgfqpoint{4.704489in}{1.025031in}}%
\pgfpathlineto{\pgfqpoint{4.728368in}{1.049114in}}%
\pgfpathlineto{\pgfqpoint{4.752247in}{1.075624in}}%
\pgfpathlineto{\pgfqpoint{4.776126in}{1.104210in}}%
\pgfpathlineto{\pgfqpoint{4.800005in}{1.133975in}}%
\pgfpathlineto{\pgfqpoint{4.823884in}{1.163767in}}%
\pgfpathlineto{\pgfqpoint{4.847763in}{1.192453in}}%
\pgfpathlineto{\pgfqpoint{4.871643in}{1.219076in}}%
\pgfpathlineto{\pgfqpoint{4.895522in}{1.242860in}}%
\pgfpathlineto{\pgfqpoint{4.919401in}{1.263104in}}%
\pgfpathlineto{\pgfqpoint{4.943280in}{1.279080in}}%
\pgfpathlineto{\pgfqpoint{4.967159in}{1.290025in}}%
\pgfpathlineto{\pgfqpoint{4.991038in}{1.295314in}}%
\pgfpathlineto{\pgfqpoint{5.014917in}{1.294744in}}%
\pgfpathlineto{\pgfqpoint{5.038796in}{1.288843in}}%
\pgfpathlineto{\pgfqpoint{5.062676in}{1.279025in}}%
\pgfpathlineto{\pgfqpoint{5.086555in}{1.267495in}}%
\pgfpathlineto{\pgfqpoint{5.110434in}{1.256857in}}%
\pgfpathlineto{\pgfqpoint{5.134313in}{1.249513in}}%
\pgfpathlineto{\pgfqpoint{5.158192in}{1.247034in}}%
\pgfpathlineto{\pgfqpoint{5.182071in}{1.249687in}}%
\pgfpathlineto{\pgfqpoint{5.205950in}{1.256311in}}%
\pgfpathlineto{\pgfqpoint{5.229830in}{1.264604in}}%
\pgfpathlineto{\pgfqpoint{5.253709in}{1.271751in}}%
\pgfpathlineto{\pgfqpoint{5.277588in}{1.275206in}}%
\pgfpathlineto{\pgfqpoint{5.301467in}{1.273367in}}%
\pgfpathlineto{\pgfqpoint{5.325346in}{1.265932in}}%
\pgfpathlineto{\pgfqpoint{5.349225in}{1.253841in}}%
\pgfpathlineto{\pgfqpoint{5.373104in}{1.238877in}}%
\pgfpathlineto{\pgfqpoint{5.396984in}{1.223110in}}%
\pgfusepath{stroke}%
\end{pgfscope}%
\begin{pgfscope}%
\pgfpathrectangle{\pgfqpoint{3.032949in}{0.398220in}}{\pgfqpoint{2.364035in}{1.703497in}}%
\pgfusepath{clip}%
\pgfsetrectcap%
\pgfsetroundjoin%
\pgfsetlinewidth{0.501875pt}%
\definecolor{currentstroke}{rgb}{0.713725,0.321569,0.337255}%
\pgfsetstrokecolor{currentstroke}%
\pgfsetdash{}{0pt}%
\pgfpathmoveto{\pgfqpoint{3.032949in}{1.238354in}}%
\pgfpathlineto{\pgfqpoint{3.056828in}{1.240820in}}%
\pgfpathlineto{\pgfqpoint{3.080707in}{1.237889in}}%
\pgfpathlineto{\pgfqpoint{3.104586in}{1.230230in}}%
\pgfpathlineto{\pgfqpoint{3.128465in}{1.219027in}}%
\pgfpathlineto{\pgfqpoint{3.152345in}{1.205717in}}%
\pgfpathlineto{\pgfqpoint{3.176224in}{1.191755in}}%
\pgfpathlineto{\pgfqpoint{3.200103in}{1.178442in}}%
\pgfpathlineto{\pgfqpoint{3.223982in}{1.166821in}}%
\pgfpathlineto{\pgfqpoint{3.247861in}{1.157604in}}%
\pgfpathlineto{\pgfqpoint{3.271740in}{1.151082in}}%
\pgfpathlineto{\pgfqpoint{3.295619in}{1.147075in}}%
\pgfpathlineto{\pgfqpoint{3.319499in}{1.144978in}}%
\pgfpathlineto{\pgfqpoint{3.343378in}{1.143943in}}%
\pgfpathlineto{\pgfqpoint{3.367257in}{1.143198in}}%
\pgfpathlineto{\pgfqpoint{3.391136in}{1.142361in}}%
\pgfpathlineto{\pgfqpoint{3.415015in}{1.141611in}}%
\pgfpathlineto{\pgfqpoint{3.438894in}{1.141597in}}%
\pgfpathlineto{\pgfqpoint{3.462773in}{1.143085in}}%
\pgfpathlineto{\pgfqpoint{3.486653in}{1.146473in}}%
\pgfpathlineto{\pgfqpoint{3.510532in}{1.151369in}}%
\pgfpathlineto{\pgfqpoint{3.534411in}{1.156409in}}%
\pgfpathlineto{\pgfqpoint{3.558290in}{1.159408in}}%
\pgfpathlineto{\pgfqpoint{3.582169in}{1.157809in}}%
\pgfpathlineto{\pgfqpoint{3.606048in}{1.149314in}}%
\pgfpathlineto{\pgfqpoint{3.629927in}{1.132499in}}%
\pgfpathlineto{\pgfqpoint{3.653807in}{1.107267in}}%
\pgfpathlineto{\pgfqpoint{3.677686in}{1.074994in}}%
\pgfpathlineto{\pgfqpoint{3.701565in}{1.038334in}}%
\pgfpathlineto{\pgfqpoint{3.725444in}{1.000704in}}%
\pgfpathlineto{\pgfqpoint{3.749323in}{0.965592in}}%
\pgfpathlineto{\pgfqpoint{3.773202in}{0.935874in}}%
\pgfpathlineto{\pgfqpoint{3.797081in}{0.913333in}}%
\pgfpathlineto{\pgfqpoint{3.820960in}{0.898533in}}%
\pgfpathlineto{\pgfqpoint{3.844840in}{0.891074in}}%
\pgfpathlineto{\pgfqpoint{3.868719in}{0.890107in}}%
\pgfpathlineto{\pgfqpoint{3.892598in}{0.894893in}}%
\pgfpathlineto{\pgfqpoint{3.916477in}{0.905187in}}%
\pgfpathlineto{\pgfqpoint{3.940356in}{0.921264in}}%
\pgfpathlineto{\pgfqpoint{3.964235in}{0.943598in}}%
\pgfpathlineto{\pgfqpoint{3.988114in}{0.972296in}}%
\pgfpathlineto{\pgfqpoint{4.011994in}{1.006512in}}%
\pgfpathlineto{\pgfqpoint{4.035873in}{1.044069in}}%
\pgfpathlineto{\pgfqpoint{4.059752in}{1.081440in}}%
\pgfpathlineto{\pgfqpoint{4.083631in}{1.114139in}}%
\pgfpathlineto{\pgfqpoint{4.107510in}{1.137442in}}%
\pgfpathlineto{\pgfqpoint{4.131389in}{1.147267in}}%
\pgfpathlineto{\pgfqpoint{4.155268in}{1.141003in}}%
\pgfpathlineto{\pgfqpoint{4.179148in}{1.118070in}}%
\pgfpathlineto{\pgfqpoint{4.203027in}{1.080096in}}%
\pgfpathlineto{\pgfqpoint{4.226906in}{1.030653in}}%
\pgfpathlineto{\pgfqpoint{4.250785in}{0.974639in}}%
\pgfpathlineto{\pgfqpoint{4.274664in}{0.917459in}}%
\pgfpathlineto{\pgfqpoint{4.298543in}{0.864209in}}%
\pgfpathlineto{\pgfqpoint{4.322422in}{0.819032in}}%
\pgfpathlineto{\pgfqpoint{4.346301in}{0.784768in}}%
\pgfpathlineto{\pgfqpoint{4.370181in}{0.762897in}}%
\pgfpathlineto{\pgfqpoint{4.394060in}{0.753686in}}%
\pgfpathlineto{\pgfqpoint{4.417939in}{0.756408in}}%
\pgfpathlineto{\pgfqpoint{4.441818in}{0.769528in}}%
\pgfpathlineto{\pgfqpoint{4.465697in}{0.790819in}}%
\pgfpathlineto{\pgfqpoint{4.489576in}{0.817484in}}%
\pgfpathlineto{\pgfqpoint{4.513455in}{0.846383in}}%
\pgfpathlineto{\pgfqpoint{4.537335in}{0.874438in}}%
\pgfpathlineto{\pgfqpoint{4.561214in}{0.899180in}}%
\pgfpathlineto{\pgfqpoint{4.585093in}{0.919291in}}%
\pgfpathlineto{\pgfqpoint{4.608972in}{0.934956in}}%
\pgfpathlineto{\pgfqpoint{4.632851in}{0.947855in}}%
\pgfpathlineto{\pgfqpoint{4.656730in}{0.960754in}}%
\pgfpathlineto{\pgfqpoint{4.680609in}{0.976796in}}%
\pgfpathlineto{\pgfqpoint{4.704489in}{0.998668in}}%
\pgfpathlineto{\pgfqpoint{4.728368in}{1.027890in}}%
\pgfpathlineto{\pgfqpoint{4.752247in}{1.064402in}}%
\pgfpathlineto{\pgfqpoint{4.776126in}{1.106536in}}%
\pgfpathlineto{\pgfqpoint{4.800005in}{1.151349in}}%
\pgfpathlineto{\pgfqpoint{4.823884in}{1.195213in}}%
\pgfpathlineto{\pgfqpoint{4.847763in}{1.234479in}}%
\pgfpathlineto{\pgfqpoint{4.871643in}{1.266080in}}%
\pgfpathlineto{\pgfqpoint{4.895522in}{1.287953in}}%
\pgfpathlineto{\pgfqpoint{4.919401in}{1.299240in}}%
\pgfpathlineto{\pgfqpoint{4.943280in}{1.300293in}}%
\pgfpathlineto{\pgfqpoint{4.967159in}{1.292536in}}%
\pgfpathlineto{\pgfqpoint{4.991038in}{1.278249in}}%
\pgfpathlineto{\pgfqpoint{5.014917in}{1.260335in}}%
\pgfpathlineto{\pgfqpoint{5.038796in}{1.242055in}}%
\pgfpathlineto{\pgfqpoint{5.062676in}{1.226741in}}%
\pgfpathlineto{\pgfqpoint{5.086555in}{1.217418in}}%
\pgfpathlineto{\pgfqpoint{5.110434in}{1.216344in}}%
\pgfpathlineto{\pgfqpoint{5.134313in}{1.224490in}}%
\pgfpathlineto{\pgfqpoint{5.158192in}{1.241091in}}%
\pgfpathlineto{\pgfqpoint{5.182071in}{1.263451in}}%
\pgfpathlineto{\pgfqpoint{5.205950in}{1.287185in}}%
\pgfpathlineto{\pgfqpoint{5.229830in}{1.306977in}}%
\pgfpathlineto{\pgfqpoint{5.253709in}{1.317787in}}%
\pgfpathlineto{\pgfqpoint{5.277588in}{1.316213in}}%
\pgfpathlineto{\pgfqpoint{5.301467in}{1.301606in}}%
\pgfpathlineto{\pgfqpoint{5.325346in}{1.276543in}}%
\pgfpathlineto{\pgfqpoint{5.349225in}{1.246426in}}%
\pgfpathlineto{\pgfqpoint{5.373104in}{1.218261in}}%
\pgfpathlineto{\pgfqpoint{5.396984in}{1.198958in}}%
\pgfusepath{stroke}%
\end{pgfscope}%
\begin{pgfscope}%
\pgfpathrectangle{\pgfqpoint{3.032949in}{0.398220in}}{\pgfqpoint{2.364035in}{1.703497in}}%
\pgfusepath{clip}%
\pgfsetrectcap%
\pgfsetroundjoin%
\pgfsetlinewidth{0.752812pt}%
\definecolor{currentstroke}{rgb}{0.000000,0.329412,0.623529}%
\pgfsetstrokecolor{currentstroke}%
\pgfsetdash{}{0pt}%
\pgfpathmoveto{\pgfqpoint{3.032949in}{1.250431in}}%
\pgfpathlineto{\pgfqpoint{3.056828in}{1.250386in}}%
\pgfpathlineto{\pgfqpoint{3.080707in}{1.250367in}}%
\pgfpathlineto{\pgfqpoint{3.104586in}{1.250400in}}%
\pgfpathlineto{\pgfqpoint{3.128465in}{1.250532in}}%
\pgfpathlineto{\pgfqpoint{3.152345in}{1.250833in}}%
\pgfpathlineto{\pgfqpoint{3.176224in}{1.251399in}}%
\pgfpathlineto{\pgfqpoint{3.200103in}{1.252352in}}%
\pgfpathlineto{\pgfqpoint{3.223982in}{1.253838in}}%
\pgfpathlineto{\pgfqpoint{3.247861in}{1.256007in}}%
\pgfpathlineto{\pgfqpoint{3.271740in}{1.258987in}}%
\pgfpathlineto{\pgfqpoint{3.295619in}{1.262854in}}%
\pgfpathlineto{\pgfqpoint{3.319499in}{1.267575in}}%
\pgfpathlineto{\pgfqpoint{3.343378in}{1.272956in}}%
\pgfpathlineto{\pgfqpoint{3.367257in}{1.278594in}}%
\pgfpathlineto{\pgfqpoint{3.391136in}{1.283849in}}%
\pgfpathlineto{\pgfqpoint{3.415015in}{1.287846in}}%
\pgfpathlineto{\pgfqpoint{3.438894in}{1.289533in}}%
\pgfpathlineto{\pgfqpoint{3.462773in}{1.287782in}}%
\pgfpathlineto{\pgfqpoint{3.486653in}{1.281542in}}%
\pgfpathlineto{\pgfqpoint{3.510532in}{1.270003in}}%
\pgfpathlineto{\pgfqpoint{3.534411in}{1.252763in}}%
\pgfpathlineto{\pgfqpoint{3.558290in}{1.229937in}}%
\pgfpathlineto{\pgfqpoint{3.582169in}{1.202193in}}%
\pgfpathlineto{\pgfqpoint{3.606048in}{1.170686in}}%
\pgfpathlineto{\pgfqpoint{3.629927in}{1.136907in}}%
\pgfpathlineto{\pgfqpoint{3.653807in}{1.102460in}}%
\pgfpathlineto{\pgfqpoint{3.677686in}{1.068828in}}%
\pgfpathlineto{\pgfqpoint{3.701565in}{1.037163in}}%
\pgfpathlineto{\pgfqpoint{3.725444in}{1.008164in}}%
\pgfpathlineto{\pgfqpoint{3.749323in}{0.982044in}}%
\pgfpathlineto{\pgfqpoint{3.773202in}{0.958614in}}%
\pgfpathlineto{\pgfqpoint{3.797081in}{0.937416in}}%
\pgfpathlineto{\pgfqpoint{3.820960in}{0.917908in}}%
\pgfpathlineto{\pgfqpoint{3.844840in}{0.899615in}}%
\pgfpathlineto{\pgfqpoint{3.868719in}{0.882229in}}%
\pgfpathlineto{\pgfqpoint{3.892598in}{0.865646in}}%
\pgfpathlineto{\pgfqpoint{3.916477in}{0.849924in}}%
\pgfpathlineto{\pgfqpoint{3.940356in}{0.835217in}}%
\pgfpathlineto{\pgfqpoint{3.964235in}{0.821682in}}%
\pgfpathlineto{\pgfqpoint{3.988114in}{0.809425in}}%
\pgfpathlineto{\pgfqpoint{4.011994in}{0.798464in}}%
\pgfpathlineto{\pgfqpoint{4.035873in}{0.788741in}}%
\pgfpathlineto{\pgfqpoint{4.059752in}{0.780155in}}%
\pgfpathlineto{\pgfqpoint{4.083631in}{0.772612in}}%
\pgfpathlineto{\pgfqpoint{4.107510in}{0.766055in}}%
\pgfpathlineto{\pgfqpoint{4.131389in}{0.760486in}}%
\pgfpathlineto{\pgfqpoint{4.155268in}{0.755954in}}%
\pgfpathlineto{\pgfqpoint{4.179148in}{0.752540in}}%
\pgfpathlineto{\pgfqpoint{4.203027in}{0.750321in}}%
\pgfpathlineto{\pgfqpoint{4.226906in}{0.749352in}}%
\pgfpathlineto{\pgfqpoint{4.250785in}{0.749649in}}%
\pgfpathlineto{\pgfqpoint{4.274664in}{0.751199in}}%
\pgfpathlineto{\pgfqpoint{4.298543in}{0.753979in}}%
\pgfpathlineto{\pgfqpoint{4.322422in}{0.757974in}}%
\pgfpathlineto{\pgfqpoint{4.346301in}{0.763204in}}%
\pgfpathlineto{\pgfqpoint{4.370181in}{0.769733in}}%
\pgfpathlineto{\pgfqpoint{4.394060in}{0.777667in}}%
\pgfpathlineto{\pgfqpoint{4.417939in}{0.787122in}}%
\pgfpathlineto{\pgfqpoint{4.441818in}{0.798205in}}%
\pgfpathlineto{\pgfqpoint{4.465697in}{0.810966in}}%
\pgfpathlineto{\pgfqpoint{4.489576in}{0.825384in}}%
\pgfpathlineto{\pgfqpoint{4.513455in}{0.841360in}}%
\pgfpathlineto{\pgfqpoint{4.537335in}{0.858746in}}%
\pgfpathlineto{\pgfqpoint{4.561214in}{0.877398in}}%
\pgfpathlineto{\pgfqpoint{4.585093in}{0.897244in}}%
\pgfpathlineto{\pgfqpoint{4.608972in}{0.918337in}}%
\pgfpathlineto{\pgfqpoint{4.632851in}{0.940883in}}%
\pgfpathlineto{\pgfqpoint{4.656730in}{0.965209in}}%
\pgfpathlineto{\pgfqpoint{4.680609in}{0.991684in}}%
\pgfpathlineto{\pgfqpoint{4.704489in}{1.020579in}}%
\pgfpathlineto{\pgfqpoint{4.728368in}{1.051922in}}%
\pgfpathlineto{\pgfqpoint{4.752247in}{1.085369in}}%
\pgfpathlineto{\pgfqpoint{4.776126in}{1.120144in}}%
\pgfpathlineto{\pgfqpoint{4.800005in}{1.155066in}}%
\pgfpathlineto{\pgfqpoint{4.823884in}{1.188673in}}%
\pgfpathlineto{\pgfqpoint{4.847763in}{1.219417in}}%
\pgfpathlineto{\pgfqpoint{4.871643in}{1.245901in}}%
\pgfpathlineto{\pgfqpoint{4.895522in}{1.267083in}}%
\pgfpathlineto{\pgfqpoint{4.919401in}{1.282425in}}%
\pgfpathlineto{\pgfqpoint{4.943280in}{1.291952in}}%
\pgfpathlineto{\pgfqpoint{4.967159in}{1.296198in}}%
\pgfpathlineto{\pgfqpoint{4.991038in}{1.296084in}}%
\pgfpathlineto{\pgfqpoint{5.014917in}{1.292752in}}%
\pgfpathlineto{\pgfqpoint{5.038796in}{1.287383in}}%
\pgfpathlineto{\pgfqpoint{5.062676in}{1.281048in}}%
\pgfpathlineto{\pgfqpoint{5.086555in}{1.274608in}}%
\pgfpathlineto{\pgfqpoint{5.110434in}{1.268667in}}%
\pgfpathlineto{\pgfqpoint{5.134313in}{1.263579in}}%
\pgfpathlineto{\pgfqpoint{5.158192in}{1.259482in}}%
\pgfpathlineto{\pgfqpoint{5.182071in}{1.256361in}}%
\pgfpathlineto{\pgfqpoint{5.205950in}{1.254098in}}%
\pgfpathlineto{\pgfqpoint{5.229830in}{1.252529in}}%
\pgfpathlineto{\pgfqpoint{5.253709in}{1.251484in}}%
\pgfpathlineto{\pgfqpoint{5.277588in}{1.250808in}}%
\pgfpathlineto{\pgfqpoint{5.301467in}{1.250373in}}%
\pgfpathlineto{\pgfqpoint{5.325346in}{1.250087in}}%
\pgfpathlineto{\pgfqpoint{5.349225in}{1.249889in}}%
\pgfpathlineto{\pgfqpoint{5.373104in}{1.249743in}}%
\pgfpathlineto{\pgfqpoint{5.396984in}{1.249637in}}%
\pgfusepath{stroke}%
\end{pgfscope}%
\begin{pgfscope}%
\pgfpathrectangle{\pgfqpoint{3.032949in}{0.398220in}}{\pgfqpoint{2.364035in}{1.703497in}}%
\pgfusepath{clip}%
\pgfsetrectcap%
\pgfsetroundjoin%
\pgfsetlinewidth{0.501875pt}%
\definecolor{currentstroke}{rgb}{0.250980,0.498039,0.717647}%
\pgfsetstrokecolor{currentstroke}%
\pgfsetdash{}{0pt}%
\pgfpathmoveto{\pgfqpoint{3.032949in}{1.481763in}}%
\pgfpathlineto{\pgfqpoint{3.056828in}{1.477244in}}%
\pgfpathlineto{\pgfqpoint{3.080707in}{1.462219in}}%
\pgfpathlineto{\pgfqpoint{3.104586in}{1.436403in}}%
\pgfpathlineto{\pgfqpoint{3.128465in}{1.399883in}}%
\pgfpathlineto{\pgfqpoint{3.152345in}{1.354706in}}%
\pgfpathlineto{\pgfqpoint{3.176224in}{1.303162in}}%
\pgfpathlineto{\pgfqpoint{3.200103in}{1.248796in}}%
\pgfpathlineto{\pgfqpoint{3.223982in}{1.195484in}}%
\pgfpathlineto{\pgfqpoint{3.247861in}{1.146812in}}%
\pgfpathlineto{\pgfqpoint{3.271740in}{1.107301in}}%
\pgfpathlineto{\pgfqpoint{3.295619in}{1.080309in}}%
\pgfpathlineto{\pgfqpoint{3.319499in}{1.068524in}}%
\pgfpathlineto{\pgfqpoint{3.343378in}{1.072972in}}%
\pgfpathlineto{\pgfqpoint{3.367257in}{1.092688in}}%
\pgfpathlineto{\pgfqpoint{3.391136in}{1.124304in}}%
\pgfpathlineto{\pgfqpoint{3.415015in}{1.162761in}}%
\pgfpathlineto{\pgfqpoint{3.438894in}{1.201108in}}%
\pgfpathlineto{\pgfqpoint{3.462773in}{1.232864in}}%
\pgfpathlineto{\pgfqpoint{3.486653in}{1.252550in}}%
\pgfpathlineto{\pgfqpoint{3.510532in}{1.257408in}}%
\pgfpathlineto{\pgfqpoint{3.534411in}{1.246787in}}%
\pgfpathlineto{\pgfqpoint{3.558290in}{1.222759in}}%
\pgfpathlineto{\pgfqpoint{3.582169in}{1.188999in}}%
\pgfpathlineto{\pgfqpoint{3.606048in}{1.149930in}}%
\pgfpathlineto{\pgfqpoint{3.629927in}{1.109700in}}%
\pgfpathlineto{\pgfqpoint{3.653807in}{1.071225in}}%
\pgfpathlineto{\pgfqpoint{3.677686in}{1.036919in}}%
\pgfpathlineto{\pgfqpoint{3.701565in}{1.006815in}}%
\pgfpathlineto{\pgfqpoint{3.725444in}{0.980701in}}%
\pgfpathlineto{\pgfqpoint{3.749323in}{0.958417in}}%
\pgfpathlineto{\pgfqpoint{3.773202in}{0.938237in}}%
\pgfpathlineto{\pgfqpoint{3.797081in}{0.920081in}}%
\pgfpathlineto{\pgfqpoint{3.820960in}{0.902969in}}%
\pgfpathlineto{\pgfqpoint{3.844840in}{0.886664in}}%
\pgfpathlineto{\pgfqpoint{3.868719in}{0.871185in}}%
\pgfpathlineto{\pgfqpoint{3.892598in}{0.856481in}}%
\pgfpathlineto{\pgfqpoint{3.916477in}{0.842410in}}%
\pgfpathlineto{\pgfqpoint{3.940356in}{0.829510in}}%
\pgfpathlineto{\pgfqpoint{3.964235in}{0.817541in}}%
\pgfpathlineto{\pgfqpoint{3.988114in}{0.806998in}}%
\pgfpathlineto{\pgfqpoint{4.011994in}{0.797732in}}%
\pgfpathlineto{\pgfqpoint{4.035873in}{0.789626in}}%
\pgfpathlineto{\pgfqpoint{4.059752in}{0.782397in}}%
\pgfpathlineto{\pgfqpoint{4.083631in}{0.775968in}}%
\pgfpathlineto{\pgfqpoint{4.107510in}{0.769972in}}%
\pgfpathlineto{\pgfqpoint{4.131389in}{0.764575in}}%
\pgfpathlineto{\pgfqpoint{4.155268in}{0.759707in}}%
\pgfpathlineto{\pgfqpoint{4.179148in}{0.755917in}}%
\pgfpathlineto{\pgfqpoint{4.203027in}{0.752888in}}%
\pgfpathlineto{\pgfqpoint{4.226906in}{0.751139in}}%
\pgfpathlineto{\pgfqpoint{4.250785in}{0.750819in}}%
\pgfpathlineto{\pgfqpoint{4.274664in}{0.752347in}}%
\pgfpathlineto{\pgfqpoint{4.298543in}{0.755861in}}%
\pgfpathlineto{\pgfqpoint{4.322422in}{0.761895in}}%
\pgfpathlineto{\pgfqpoint{4.346301in}{0.770517in}}%
\pgfpathlineto{\pgfqpoint{4.370181in}{0.781158in}}%
\pgfpathlineto{\pgfqpoint{4.394060in}{0.793720in}}%
\pgfpathlineto{\pgfqpoint{4.417939in}{0.807348in}}%
\pgfpathlineto{\pgfqpoint{4.441818in}{0.821371in}}%
\pgfpathlineto{\pgfqpoint{4.465697in}{0.835863in}}%
\pgfpathlineto{\pgfqpoint{4.489576in}{0.850524in}}%
\pgfpathlineto{\pgfqpoint{4.513455in}{0.865410in}}%
\pgfpathlineto{\pgfqpoint{4.537335in}{0.880717in}}%
\pgfpathlineto{\pgfqpoint{4.561214in}{0.897059in}}%
\pgfpathlineto{\pgfqpoint{4.585093in}{0.914306in}}%
\pgfpathlineto{\pgfqpoint{4.608972in}{0.932864in}}%
\pgfpathlineto{\pgfqpoint{4.632851in}{0.952943in}}%
\pgfpathlineto{\pgfqpoint{4.656730in}{0.975235in}}%
\pgfpathlineto{\pgfqpoint{4.680609in}{1.000141in}}%
\pgfpathlineto{\pgfqpoint{4.704489in}{1.028501in}}%
\pgfpathlineto{\pgfqpoint{4.728368in}{1.060692in}}%
\pgfpathlineto{\pgfqpoint{4.752247in}{1.096768in}}%
\pgfpathlineto{\pgfqpoint{4.776126in}{1.135478in}}%
\pgfpathlineto{\pgfqpoint{4.800005in}{1.175393in}}%
\pgfpathlineto{\pgfqpoint{4.823884in}{1.213087in}}%
\pgfpathlineto{\pgfqpoint{4.847763in}{1.245736in}}%
\pgfpathlineto{\pgfqpoint{4.871643in}{1.269791in}}%
\pgfpathlineto{\pgfqpoint{4.895522in}{1.281958in}}%
\pgfpathlineto{\pgfqpoint{4.919401in}{1.281523in}}%
\pgfpathlineto{\pgfqpoint{4.943280in}{1.268984in}}%
\pgfpathlineto{\pgfqpoint{4.967159in}{1.246174in}}%
\pgfpathlineto{\pgfqpoint{4.991038in}{1.216532in}}%
\pgfpathlineto{\pgfqpoint{5.014917in}{1.184399in}}%
\pgfpathlineto{\pgfqpoint{5.038796in}{1.153177in}}%
\pgfpathlineto{\pgfqpoint{5.062676in}{1.126554in}}%
\pgfpathlineto{\pgfqpoint{5.086555in}{1.107179in}}%
\pgfpathlineto{\pgfqpoint{5.110434in}{1.096848in}}%
\pgfpathlineto{\pgfqpoint{5.134313in}{1.096669in}}%
\pgfpathlineto{\pgfqpoint{5.158192in}{1.106926in}}%
\pgfpathlineto{\pgfqpoint{5.182071in}{1.126477in}}%
\pgfpathlineto{\pgfqpoint{5.205950in}{1.152613in}}%
\pgfpathlineto{\pgfqpoint{5.229830in}{1.182534in}}%
\pgfpathlineto{\pgfqpoint{5.253709in}{1.210781in}}%
\pgfpathlineto{\pgfqpoint{5.277588in}{1.232994in}}%
\pgfpathlineto{\pgfqpoint{5.301467in}{1.244798in}}%
\pgfpathlineto{\pgfqpoint{5.325346in}{1.243860in}}%
\pgfpathlineto{\pgfqpoint{5.349225in}{1.230378in}}%
\pgfpathlineto{\pgfqpoint{5.373104in}{1.206915in}}%
\pgfpathlineto{\pgfqpoint{5.396984in}{1.177708in}}%
\pgfusepath{stroke}%
\end{pgfscope}%
\begin{pgfscope}%
\pgfpathrectangle{\pgfqpoint{3.032949in}{0.398220in}}{\pgfqpoint{2.364035in}{1.703497in}}%
\pgfusepath{clip}%
\pgfsetrectcap%
\pgfsetroundjoin%
\pgfsetlinewidth{0.501875pt}%
\definecolor{currentstroke}{rgb}{0.250980,0.498039,0.717647}%
\pgfsetstrokecolor{currentstroke}%
\pgfsetdash{}{0pt}%
\pgfpathmoveto{\pgfqpoint{3.032949in}{1.302049in}}%
\pgfpathlineto{\pgfqpoint{3.056828in}{1.348001in}}%
\pgfpathlineto{\pgfqpoint{3.080707in}{1.386332in}}%
\pgfpathlineto{\pgfqpoint{3.104586in}{1.412778in}}%
\pgfpathlineto{\pgfqpoint{3.128465in}{1.424651in}}%
\pgfpathlineto{\pgfqpoint{3.152345in}{1.421918in}}%
\pgfpathlineto{\pgfqpoint{3.176224in}{1.406870in}}%
\pgfpathlineto{\pgfqpoint{3.200103in}{1.383381in}}%
\pgfpathlineto{\pgfqpoint{3.223982in}{1.355646in}}%
\pgfpathlineto{\pgfqpoint{3.247861in}{1.327921in}}%
\pgfpathlineto{\pgfqpoint{3.271740in}{1.304357in}}%
\pgfpathlineto{\pgfqpoint{3.295619in}{1.286971in}}%
\pgfpathlineto{\pgfqpoint{3.319499in}{1.276347in}}%
\pgfpathlineto{\pgfqpoint{3.343378in}{1.272607in}}%
\pgfpathlineto{\pgfqpoint{3.367257in}{1.274247in}}%
\pgfpathlineto{\pgfqpoint{3.391136in}{1.279120in}}%
\pgfpathlineto{\pgfqpoint{3.415015in}{1.285095in}}%
\pgfpathlineto{\pgfqpoint{3.438894in}{1.289510in}}%
\pgfpathlineto{\pgfqpoint{3.462773in}{1.290224in}}%
\pgfpathlineto{\pgfqpoint{3.486653in}{1.284614in}}%
\pgfpathlineto{\pgfqpoint{3.510532in}{1.271927in}}%
\pgfpathlineto{\pgfqpoint{3.534411in}{1.251942in}}%
\pgfpathlineto{\pgfqpoint{3.558290in}{1.226254in}}%
\pgfpathlineto{\pgfqpoint{3.582169in}{1.196668in}}%
\pgfpathlineto{\pgfqpoint{3.606048in}{1.165652in}}%
\pgfpathlineto{\pgfqpoint{3.629927in}{1.135188in}}%
\pgfpathlineto{\pgfqpoint{3.653807in}{1.106433in}}%
\pgfpathlineto{\pgfqpoint{3.677686in}{1.079562in}}%
\pgfpathlineto{\pgfqpoint{3.701565in}{1.053618in}}%
\pgfpathlineto{\pgfqpoint{3.725444in}{1.027922in}}%
\pgfpathlineto{\pgfqpoint{3.749323in}{1.002110in}}%
\pgfpathlineto{\pgfqpoint{3.773202in}{0.975894in}}%
\pgfpathlineto{\pgfqpoint{3.797081in}{0.949919in}}%
\pgfpathlineto{\pgfqpoint{3.820960in}{0.925878in}}%
\pgfpathlineto{\pgfqpoint{3.844840in}{0.904190in}}%
\pgfpathlineto{\pgfqpoint{3.868719in}{0.885933in}}%
\pgfpathlineto{\pgfqpoint{3.892598in}{0.871104in}}%
\pgfpathlineto{\pgfqpoint{3.916477in}{0.858579in}}%
\pgfpathlineto{\pgfqpoint{3.940356in}{0.847289in}}%
\pgfpathlineto{\pgfqpoint{3.964235in}{0.835603in}}%
\pgfpathlineto{\pgfqpoint{3.988114in}{0.822963in}}%
\pgfpathlineto{\pgfqpoint{4.011994in}{0.808988in}}%
\pgfpathlineto{\pgfqpoint{4.035873in}{0.794191in}}%
\pgfpathlineto{\pgfqpoint{4.059752in}{0.779585in}}%
\pgfpathlineto{\pgfqpoint{4.083631in}{0.766108in}}%
\pgfpathlineto{\pgfqpoint{4.107510in}{0.754424in}}%
\pgfpathlineto{\pgfqpoint{4.131389in}{0.744262in}}%
\pgfpathlineto{\pgfqpoint{4.155268in}{0.735679in}}%
\pgfpathlineto{\pgfqpoint{4.179148in}{0.727756in}}%
\pgfpathlineto{\pgfqpoint{4.203027in}{0.720779in}}%
\pgfpathlineto{\pgfqpoint{4.226906in}{0.714946in}}%
\pgfpathlineto{\pgfqpoint{4.250785in}{0.710937in}}%
\pgfpathlineto{\pgfqpoint{4.274664in}{0.710518in}}%
\pgfpathlineto{\pgfqpoint{4.298543in}{0.714185in}}%
\pgfpathlineto{\pgfqpoint{4.322422in}{0.722655in}}%
\pgfpathlineto{\pgfqpoint{4.346301in}{0.735156in}}%
\pgfpathlineto{\pgfqpoint{4.370181in}{0.750773in}}%
\pgfpathlineto{\pgfqpoint{4.394060in}{0.767851in}}%
\pgfpathlineto{\pgfqpoint{4.417939in}{0.784833in}}%
\pgfpathlineto{\pgfqpoint{4.441818in}{0.800808in}}%
\pgfpathlineto{\pgfqpoint{4.465697in}{0.815850in}}%
\pgfpathlineto{\pgfqpoint{4.489576in}{0.830513in}}%
\pgfpathlineto{\pgfqpoint{4.513455in}{0.845528in}}%
\pgfpathlineto{\pgfqpoint{4.537335in}{0.861330in}}%
\pgfpathlineto{\pgfqpoint{4.561214in}{0.878583in}}%
\pgfpathlineto{\pgfqpoint{4.585093in}{0.897520in}}%
\pgfpathlineto{\pgfqpoint{4.608972in}{0.918094in}}%
\pgfpathlineto{\pgfqpoint{4.632851in}{0.940016in}}%
\pgfpathlineto{\pgfqpoint{4.656730in}{0.964076in}}%
\pgfpathlineto{\pgfqpoint{4.680609in}{0.990364in}}%
\pgfpathlineto{\pgfqpoint{4.704489in}{1.020455in}}%
\pgfpathlineto{\pgfqpoint{4.728368in}{1.053950in}}%
\pgfpathlineto{\pgfqpoint{4.752247in}{1.090089in}}%
\pgfpathlineto{\pgfqpoint{4.776126in}{1.128260in}}%
\pgfpathlineto{\pgfqpoint{4.800005in}{1.166087in}}%
\pgfpathlineto{\pgfqpoint{4.823884in}{1.201582in}}%
\pgfpathlineto{\pgfqpoint{4.847763in}{1.233544in}}%
\pgfpathlineto{\pgfqpoint{4.871643in}{1.261259in}}%
\pgfpathlineto{\pgfqpoint{4.895522in}{1.284939in}}%
\pgfpathlineto{\pgfqpoint{4.919401in}{1.305708in}}%
\pgfpathlineto{\pgfqpoint{4.943280in}{1.323885in}}%
\pgfpathlineto{\pgfqpoint{4.967159in}{1.341036in}}%
\pgfpathlineto{\pgfqpoint{4.991038in}{1.355595in}}%
\pgfpathlineto{\pgfqpoint{5.014917in}{1.367098in}}%
\pgfpathlineto{\pgfqpoint{5.038796in}{1.373878in}}%
\pgfpathlineto{\pgfqpoint{5.062676in}{1.374014in}}%
\pgfpathlineto{\pgfqpoint{5.086555in}{1.366612in}}%
\pgfpathlineto{\pgfqpoint{5.110434in}{1.352054in}}%
\pgfpathlineto{\pgfqpoint{5.134313in}{1.331294in}}%
\pgfpathlineto{\pgfqpoint{5.158192in}{1.306599in}}%
\pgfpathlineto{\pgfqpoint{5.182071in}{1.280137in}}%
\pgfpathlineto{\pgfqpoint{5.205950in}{1.253790in}}%
\pgfpathlineto{\pgfqpoint{5.229830in}{1.228840in}}%
\pgfpathlineto{\pgfqpoint{5.253709in}{1.206009in}}%
\pgfpathlineto{\pgfqpoint{5.277588in}{1.184968in}}%
\pgfpathlineto{\pgfqpoint{5.301467in}{1.165132in}}%
\pgfpathlineto{\pgfqpoint{5.325346in}{1.145525in}}%
\pgfpathlineto{\pgfqpoint{5.349225in}{1.125029in}}%
\pgfpathlineto{\pgfqpoint{5.373104in}{1.102862in}}%
\pgfpathlineto{\pgfqpoint{5.396984in}{1.078621in}}%
\pgfusepath{stroke}%
\end{pgfscope}%
\begin{pgfscope}%
\pgfpathrectangle{\pgfqpoint{3.032949in}{0.398220in}}{\pgfqpoint{2.364035in}{1.703497in}}%
\pgfusepath{clip}%
\pgfsetrectcap%
\pgfsetroundjoin%
\pgfsetlinewidth{0.501875pt}%
\definecolor{currentstroke}{rgb}{0.250980,0.498039,0.717647}%
\pgfsetstrokecolor{currentstroke}%
\pgfsetdash{}{0pt}%
\pgfpathmoveto{\pgfqpoint{3.032949in}{1.337708in}}%
\pgfpathlineto{\pgfqpoint{3.056828in}{1.345272in}}%
\pgfpathlineto{\pgfqpoint{3.080707in}{1.346459in}}%
\pgfpathlineto{\pgfqpoint{3.104586in}{1.341626in}}%
\pgfpathlineto{\pgfqpoint{3.128465in}{1.332429in}}%
\pgfpathlineto{\pgfqpoint{3.152345in}{1.321898in}}%
\pgfpathlineto{\pgfqpoint{3.176224in}{1.313722in}}%
\pgfpathlineto{\pgfqpoint{3.200103in}{1.310937in}}%
\pgfpathlineto{\pgfqpoint{3.223982in}{1.317500in}}%
\pgfpathlineto{\pgfqpoint{3.247861in}{1.334101in}}%
\pgfpathlineto{\pgfqpoint{3.271740in}{1.361627in}}%
\pgfpathlineto{\pgfqpoint{3.295619in}{1.397264in}}%
\pgfpathlineto{\pgfqpoint{3.319499in}{1.437222in}}%
\pgfpathlineto{\pgfqpoint{3.343378in}{1.476370in}}%
\pgfpathlineto{\pgfqpoint{3.367257in}{1.508815in}}%
\pgfpathlineto{\pgfqpoint{3.391136in}{1.529371in}}%
\pgfpathlineto{\pgfqpoint{3.415015in}{1.534132in}}%
\pgfpathlineto{\pgfqpoint{3.438894in}{1.520951in}}%
\pgfpathlineto{\pgfqpoint{3.462773in}{1.490635in}}%
\pgfpathlineto{\pgfqpoint{3.486653in}{1.445737in}}%
\pgfpathlineto{\pgfqpoint{3.510532in}{1.389885in}}%
\pgfpathlineto{\pgfqpoint{3.534411in}{1.328371in}}%
\pgfpathlineto{\pgfqpoint{3.558290in}{1.265471in}}%
\pgfpathlineto{\pgfqpoint{3.582169in}{1.204982in}}%
\pgfpathlineto{\pgfqpoint{3.606048in}{1.148923in}}%
\pgfpathlineto{\pgfqpoint{3.629927in}{1.098413in}}%
\pgfpathlineto{\pgfqpoint{3.653807in}{1.054483in}}%
\pgfpathlineto{\pgfqpoint{3.677686in}{1.016791in}}%
\pgfpathlineto{\pgfqpoint{3.701565in}{0.985697in}}%
\pgfpathlineto{\pgfqpoint{3.725444in}{0.961244in}}%
\pgfpathlineto{\pgfqpoint{3.749323in}{0.943231in}}%
\pgfpathlineto{\pgfqpoint{3.773202in}{0.930879in}}%
\pgfpathlineto{\pgfqpoint{3.797081in}{0.922508in}}%
\pgfpathlineto{\pgfqpoint{3.820960in}{0.916316in}}%
\pgfpathlineto{\pgfqpoint{3.844840in}{0.909439in}}%
\pgfpathlineto{\pgfqpoint{3.868719in}{0.901171in}}%
\pgfpathlineto{\pgfqpoint{3.892598in}{0.890256in}}%
\pgfpathlineto{\pgfqpoint{3.916477in}{0.877016in}}%
\pgfpathlineto{\pgfqpoint{3.940356in}{0.863230in}}%
\pgfpathlineto{\pgfqpoint{3.964235in}{0.850192in}}%
\pgfpathlineto{\pgfqpoint{3.988114in}{0.839471in}}%
\pgfpathlineto{\pgfqpoint{4.011994in}{0.831706in}}%
\pgfpathlineto{\pgfqpoint{4.035873in}{0.826402in}}%
\pgfpathlineto{\pgfqpoint{4.059752in}{0.822407in}}%
\pgfpathlineto{\pgfqpoint{4.083631in}{0.818315in}}%
\pgfpathlineto{\pgfqpoint{4.107510in}{0.813053in}}%
\pgfpathlineto{\pgfqpoint{4.131389in}{0.806537in}}%
\pgfpathlineto{\pgfqpoint{4.155268in}{0.799069in}}%
\pgfpathlineto{\pgfqpoint{4.179148in}{0.791464in}}%
\pgfpathlineto{\pgfqpoint{4.203027in}{0.785418in}}%
\pgfpathlineto{\pgfqpoint{4.226906in}{0.781674in}}%
\pgfpathlineto{\pgfqpoint{4.250785in}{0.779619in}}%
\pgfpathlineto{\pgfqpoint{4.274664in}{0.778581in}}%
\pgfpathlineto{\pgfqpoint{4.298543in}{0.777557in}}%
\pgfpathlineto{\pgfqpoint{4.322422in}{0.775892in}}%
\pgfpathlineto{\pgfqpoint{4.346301in}{0.773665in}}%
\pgfpathlineto{\pgfqpoint{4.370181in}{0.770926in}}%
\pgfpathlineto{\pgfqpoint{4.394060in}{0.769853in}}%
\pgfpathlineto{\pgfqpoint{4.417939in}{0.771837in}}%
\pgfpathlineto{\pgfqpoint{4.441818in}{0.778045in}}%
\pgfpathlineto{\pgfqpoint{4.465697in}{0.788770in}}%
\pgfpathlineto{\pgfqpoint{4.489576in}{0.804376in}}%
\pgfpathlineto{\pgfqpoint{4.513455in}{0.823051in}}%
\pgfpathlineto{\pgfqpoint{4.537335in}{0.843928in}}%
\pgfpathlineto{\pgfqpoint{4.561214in}{0.865617in}}%
\pgfpathlineto{\pgfqpoint{4.585093in}{0.887524in}}%
\pgfpathlineto{\pgfqpoint{4.608972in}{0.909238in}}%
\pgfpathlineto{\pgfqpoint{4.632851in}{0.930904in}}%
\pgfpathlineto{\pgfqpoint{4.656730in}{0.953483in}}%
\pgfpathlineto{\pgfqpoint{4.680609in}{0.977926in}}%
\pgfpathlineto{\pgfqpoint{4.704489in}{1.005310in}}%
\pgfpathlineto{\pgfqpoint{4.728368in}{1.036205in}}%
\pgfpathlineto{\pgfqpoint{4.752247in}{1.070977in}}%
\pgfpathlineto{\pgfqpoint{4.776126in}{1.108641in}}%
\pgfpathlineto{\pgfqpoint{4.800005in}{1.147485in}}%
\pgfpathlineto{\pgfqpoint{4.823884in}{1.184992in}}%
\pgfpathlineto{\pgfqpoint{4.847763in}{1.218141in}}%
\pgfpathlineto{\pgfqpoint{4.871643in}{1.244362in}}%
\pgfpathlineto{\pgfqpoint{4.895522in}{1.261986in}}%
\pgfpathlineto{\pgfqpoint{4.919401in}{1.270253in}}%
\pgfpathlineto{\pgfqpoint{4.943280in}{1.270073in}}%
\pgfpathlineto{\pgfqpoint{4.967159in}{1.262970in}}%
\pgfpathlineto{\pgfqpoint{4.991038in}{1.252076in}}%
\pgfpathlineto{\pgfqpoint{5.014917in}{1.240054in}}%
\pgfpathlineto{\pgfqpoint{5.038796in}{1.230402in}}%
\pgfpathlineto{\pgfqpoint{5.062676in}{1.225709in}}%
\pgfpathlineto{\pgfqpoint{5.086555in}{1.227329in}}%
\pgfpathlineto{\pgfqpoint{5.110434in}{1.236303in}}%
\pgfpathlineto{\pgfqpoint{5.134313in}{1.252444in}}%
\pgfpathlineto{\pgfqpoint{5.158192in}{1.274600in}}%
\pgfpathlineto{\pgfqpoint{5.182071in}{1.301189in}}%
\pgfpathlineto{\pgfqpoint{5.205950in}{1.329546in}}%
\pgfpathlineto{\pgfqpoint{5.229830in}{1.358050in}}%
\pgfpathlineto{\pgfqpoint{5.253709in}{1.383591in}}%
\pgfpathlineto{\pgfqpoint{5.277588in}{1.403770in}}%
\pgfpathlineto{\pgfqpoint{5.301467in}{1.416051in}}%
\pgfpathlineto{\pgfqpoint{5.325346in}{1.418720in}}%
\pgfpathlineto{\pgfqpoint{5.349225in}{1.410839in}}%
\pgfpathlineto{\pgfqpoint{5.373104in}{1.393150in}}%
\pgfpathlineto{\pgfqpoint{5.396984in}{1.367935in}}%
\pgfusepath{stroke}%
\end{pgfscope}%
\begin{pgfscope}%
\pgfsetrectcap%
\pgfsetmiterjoin%
\pgfsetlinewidth{0.752812pt}%
\definecolor{currentstroke}{rgb}{0.000000,0.000000,0.000000}%
\pgfsetstrokecolor{currentstroke}%
\pgfsetdash{}{0pt}%
\pgfpathmoveto{\pgfqpoint{3.032949in}{0.398220in}}%
\pgfpathlineto{\pgfqpoint{3.032949in}{2.101717in}}%
\pgfusepath{stroke}%
\end{pgfscope}%
\begin{pgfscope}%
\pgfsetrectcap%
\pgfsetmiterjoin%
\pgfsetlinewidth{0.752812pt}%
\definecolor{currentstroke}{rgb}{0.000000,0.000000,0.000000}%
\pgfsetstrokecolor{currentstroke}%
\pgfsetdash{}{0pt}%
\pgfpathmoveto{\pgfqpoint{5.396984in}{0.398220in}}%
\pgfpathlineto{\pgfqpoint{5.396984in}{2.101717in}}%
\pgfusepath{stroke}%
\end{pgfscope}%
\begin{pgfscope}%
\pgfsetrectcap%
\pgfsetmiterjoin%
\pgfsetlinewidth{0.752812pt}%
\definecolor{currentstroke}{rgb}{0.000000,0.000000,0.000000}%
\pgfsetstrokecolor{currentstroke}%
\pgfsetdash{}{0pt}%
\pgfpathmoveto{\pgfqpoint{3.032949in}{0.398220in}}%
\pgfpathlineto{\pgfqpoint{5.396984in}{0.398220in}}%
\pgfusepath{stroke}%
\end{pgfscope}%
\begin{pgfscope}%
\pgfsetrectcap%
\pgfsetmiterjoin%
\pgfsetlinewidth{0.752812pt}%
\definecolor{currentstroke}{rgb}{0.000000,0.000000,0.000000}%
\pgfsetstrokecolor{currentstroke}%
\pgfsetdash{}{0pt}%
\pgfpathmoveto{\pgfqpoint{3.032949in}{2.101717in}}%
\pgfpathlineto{\pgfqpoint{5.396984in}{2.101717in}}%
\pgfusepath{stroke}%
\end{pgfscope}%
\begin{pgfscope}%
\definecolor{textcolor}{rgb}{0.000000,0.000000,0.000000}%
\pgfsetstrokecolor{textcolor}%
\pgfsetfillcolor{textcolor}%
\pgftext[x=4.214966in,y=1.931368in,,base]{\color{textcolor}\rmfamily\fontsize{10.000000}{12.000000}\selectfont Posterior}%
\end{pgfscope}%
\begin{pgfscope}%
\pgfpathrectangle{\pgfqpoint{3.032949in}{0.398220in}}{\pgfqpoint{2.364035in}{1.703497in}}%
\pgfusepath{clip}%
\pgfsetbuttcap%
\pgfsetroundjoin%
\definecolor{currentfill}{rgb}{0.000000,0.000000,0.000000}%
\pgfsetfillcolor{currentfill}%
\pgfsetlinewidth{1.003750pt}%
\definecolor{currentstroke}{rgb}{0.000000,0.000000,0.000000}%
\pgfsetstrokecolor{currentstroke}%
\pgfsetdash{}{0pt}%
\pgfsys@defobject{currentmarker}{\pgfqpoint{-0.020833in}{-0.020833in}}{\pgfqpoint{0.020833in}{0.020833in}}{%
\pgfpathmoveto{\pgfqpoint{0.000000in}{-0.020833in}}%
\pgfpathcurveto{\pgfqpoint{0.005525in}{-0.020833in}}{\pgfqpoint{0.010825in}{-0.018638in}}{\pgfqpoint{0.014731in}{-0.014731in}}%
\pgfpathcurveto{\pgfqpoint{0.018638in}{-0.010825in}}{\pgfqpoint{0.020833in}{-0.005525in}}{\pgfqpoint{0.020833in}{0.000000in}}%
\pgfpathcurveto{\pgfqpoint{0.020833in}{0.005525in}}{\pgfqpoint{0.018638in}{0.010825in}}{\pgfqpoint{0.014731in}{0.014731in}}%
\pgfpathcurveto{\pgfqpoint{0.010825in}{0.018638in}}{\pgfqpoint{0.005525in}{0.020833in}}{\pgfqpoint{0.000000in}{0.020833in}}%
\pgfpathcurveto{\pgfqpoint{-0.005525in}{0.020833in}}{\pgfqpoint{-0.010825in}{0.018638in}}{\pgfqpoint{-0.014731in}{0.014731in}}%
\pgfpathcurveto{\pgfqpoint{-0.018638in}{0.010825in}}{\pgfqpoint{-0.020833in}{0.005525in}}{\pgfqpoint{-0.020833in}{0.000000in}}%
\pgfpathcurveto{\pgfqpoint{-0.020833in}{-0.005525in}}{\pgfqpoint{-0.018638in}{-0.010825in}}{\pgfqpoint{-0.014731in}{-0.014731in}}%
\pgfpathcurveto{\pgfqpoint{-0.010825in}{-0.018638in}}{\pgfqpoint{-0.005525in}{-0.020833in}}{\pgfqpoint{0.000000in}{-0.020833in}}%
\pgfpathclose%
\pgfusepath{stroke,fill}%
}%
\begin{pgfscope}%
\pgfsys@transformshift{3.623958in}{1.141114in}%
\pgfsys@useobject{currentmarker}{}%
\end{pgfscope}%
\begin{pgfscope}%
\pgfsys@transformshift{3.845586in}{0.898491in}%
\pgfsys@useobject{currentmarker}{}%
\end{pgfscope}%
\begin{pgfscope}%
\pgfsys@transformshift{4.362718in}{0.761828in}%
\pgfsys@useobject{currentmarker}{}%
\end{pgfscope}%
\begin{pgfscope}%
\pgfsys@transformshift{4.510471in}{0.834892in}%
\pgfsys@useobject{currentmarker}{}%
\end{pgfscope}%
\begin{pgfscope}%
\pgfsys@transformshift{4.658223in}{0.974905in}%
\pgfsys@useobject{currentmarker}{}%
\end{pgfscope}%
\begin{pgfscope}%
\pgfsys@transformshift{4.732099in}{1.049907in}%
\pgfsys@useobject{currentmarker}{}%
\end{pgfscope}%
\begin{pgfscope}%
\pgfsys@transformshift{4.879851in}{1.253033in}%
\pgfsys@useobject{currentmarker}{}%
\end{pgfscope}%
\end{pgfscope}%
\end{pgfpicture}%
\makeatother%
\endgroup%

    \caption[Comparing unconstrained and constrained prior and posterior distributions.]{Unconstrained (red) and constrained (blue) prior distribution on the left as well as the unconstrained (red) \eqref{eq:posterior_distribution} and constrained (blue) \eqref{eq:constrained_posterior_distribution} posterior distribution conditioned on the training data from the objective function (black) on the right. The green points are the \glspl{vop}. The thin lines are samples from the corresponding distribution. The hyperparameters are $\mu_0=0$, $\sigma
   ^2_l = 1$, $\sigma_k^2 = 4$, $\sigma_v^2 = 10^{-8}$, $\sigma_n^2 = 0.05^2$.}
   \label{fig:constrained_gp_example}
\end{figure}
The resulting posterior distribution by applying Algorithm~\ref{algo:constrained_posterior} is displayed in Figure~\ref{fig:constrained_gp_example}.
The sampling algorithm for the constrained \gls{gp} prior distribution as displayed in Figure~\ref{fig:constrained_gp_example} (left) is given in Appendix \ref{apx:sampling_from_prior}. It can be observed that at the \glspl{vop} the posterior is convex. However, outside of the \glspl{vop} the posterior converges back to the unconstrained posterior, highlighting the importance of choosing the \glspl{vop}.
