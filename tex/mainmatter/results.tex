\chapter{Results}
\label{chap:results}

In this chapter, the proposed methods are evaluated empirically and the Hypotheses~\ref{hyp:ui_structural_information}--\ref{hyp:ctvbo} are tested on different experiments. For this purpose, the proposed modeling approach \gls{uitvbo} is compared to TV-GP-UCB\footnotemark by \textcite{Bogunovic_2016} using standard \gls{tvbo} as well as the proposed method \gls{ctvbo}, both with and without data selection. This results in the different variations summarized in Table \ref{tab:models}.

\bgroup
\def\arraystretch{1.2}
\begin{table}[h]
\begin{center}
\begin{tabular}{ c || c | c | c}
 & \textbf{\gls{tvbo}} & \textbf{Data}& \textbf{Forgetting}\vspace{-0.1cm}\\
\textbf{Variation} & \textbf{Algorithm} & \textbf{Selection} & \textbf{Strategy}\\\hline\hline
\textbf{\gls{uitvbo}} & standard \gls{tvbo} & -- & \gls{ui} \\\hline
\textbf{B \gls{uitvbo}} & standard \gls{tvbo} & binning & \gls{ui} \\\hline
\textbf{TV-GP-UCB}\footnotemark[\value{footnote}] & standard \gls{tvbo} & -- & \gls{b2p} \\\hline
\textbf{SW TV-GP-UCB}\footnotemark[\value{footnote}] & standard \gls{tvbo} & sliding window & \gls{b2p} \\\hline
\textbf{C-\gls{uitvbo}} & \gls{ctvbo} & -- & \gls{ui} \\\hline
\textbf{B C-\gls{uitvbo}} & \gls{ctvbo} & binning & \gls{ui} \\\hline
\textbf{C-TV-GP-UCB}\footnotemark[\value{footnote}] & \gls{ctvbo} & -- & \gls{b2p} \\\hline
\textbf{SW C-TV-GP-UCB}\footnotemark[\value{footnote}] & \gls{ctvbo} & sliding window & \gls{b2p} \\\hline
\end{tabular}
\end{center}
\caption[Evaluated model variations.]{Variations which are evaluated in this chapter. Standard \gls{tvbo} denotes Algorithm~\ref{algo:tvbo}, \gls{ctvbo} denotes the proposed method in Algorithm~\ref{algo:constrained_tvbo}. The data selection strategies are as discussed in \Cref{sec:data_selection}.}
\label{tab:models}
\end{table}
\egroup
\footnotetext{Note that only the \textit{model} as introduced by \textcite{Bogunovic_2016} will be used for comparison, \underline{not} the \gls{ucb} algorithm.}

Furthermore, to test Hypothesis~\ref{hyp:ui_structural_information}, each experiment is performed with a well-defined and an optimistic prior mean.
An optimistic prior mean investigates the robustness of the variations regarding a misspecified prior distribution. Such robustness is desirable, especially regarding real-world applications where the mean of an objective function changes over time.

The acquisition function throughout this chapter will be \gls{lcb} as
\begin{equation}
    \alpha(\mathbf{x}, t+1|\mathcal{D}) = \mu_{t+1}(\mathbf{x}) - \sqrt{\beta_{t+1}}\, \sigma_{t+1}(\mathbf{x})
    \label{eq:results_lcb}
\end{equation}
with a constant exploration-exploitation factor of $\beta_{t+1}=2$. Different choices for $\beta_{t+1}$ may be appropriate for different variations in Table~\ref{tab:models}, however, the exploration-exploitation also depends on the forgetting factors. Fixing the acquisition function increases the emphasis on the modeling approaches and algorithms.

The variations are implemented in Python and are based on PyTorch \cite{Paszke_2019} using GPyTorch \cite{Gardner_2018} for modeling the \gls{gp} and BoTorch \cite{Balandat_2020} for optimizing the acquisition function.

\section{Synthetic Experiments}
\label{sec:synthetic_experiments}

To compare the different variations in Table~\ref{tab:models} different synthetic experiments are performed. First, the variations are compared qualitatively in the within-model as well as out-of-model comparison. These investigate the behavior of the proposed methods if all assumptions are satisfied by the objective function.
Then quantitative comparisons are conducted using benchmarks similar to those of \textcite{Renganathan_2020}.
The comparisons are performed both one-dimensional ($D=1$) and two-dimensional ($D=2$). For the method \gls{ctvbo} $\delta=1.5$ \eqref{eq:delta} is chosen and for $D=1$ the number of \glspl{vop} per dimension is set to $N_{v/D}=10$, for $D=2$ to $N_{v/D}=5$. For the data selection strategies, the number of bins per dimension is set to $20$, and the sliding window size is set to $W=30$ for $D=1$ and $W=80$ for $D=2$. The sliding window sizes were chosen to account for the same amount of training data as binning since approximately $80$ bins were filled in the two-dimensional experiments. The same number of training points allows for a straightforward comparison.

\subsection{Within-Model Comparison}
\label{sec:within-model}

The first experiments conducted are within-model comparisons motivated by \textcite{Hennig_2012} and previous work by \textcite{Bogunovic_2016} where the objective function is generated according to the model assumptions at hand. In this thesis, the assumptions are the objective function staying convex through time captured by the proposed method \gls{ctvbo} as well as temporal change according to a Wiener process as embedded in \gls{uitvbo}. For the within-model comparisons, the hyperparameters are known a-priori to the algorithm. Therefore no hyperparameter optimization is performed.

\begin{algorithm}[h]
\centering
\caption{Generate Within-Model Objective Function}
\begin{algorithmic}[1]
\Require prior $\mathcal{GP}(m(\mathbf{x}), k_S(\mathbf{x},\mathbf{x}') \otimes k_T(t, t'))$ and hyperparameter; feasible set ${\mathcal{X}\in \R^D}$; number of \glspl{vop} per dimension $N_{v/D}$, truncation bounds $a(\cdot), b(\cdot)$
\State Sample $f_0$ from constrained prior distribution \Comment{Appendix \ref{apx:sampling_from_prior}}
\State $f = [f_0]$
\For{$t = 0, 1, \dots, T$}
    \State Learn previous sample $f_t$
    \State Place \glspl{vop} at time steps $t$ and $t+1$
    \State Sample $f_{t+1}$ from the posterior at $t+1$
    \State $f = [f;f_{t+1}]$
\EndFor
\Ensure $f$
\end{algorithmic}
\label{algo:within_model_comparison}
\end{algorithm}

Usually, the within-model objective function is generated by sampling from the \gls{gp} prior distribution with fixed hyperparameters. However, the number of \glspl{vop} needed to span across the whole domain $\mathcal{X}\times\mathcal{T}$ to generate one sample is too high as discussed in \Cref{sec:model_convex_functions}. Therefore, the objective function is generated in an iterative fashion according to Algorithm~\ref{algo:within_model_comparison} with the temporal kernel $k_{T,wp}$ and a fixed \gls{ui} forgetting factor $\hat{\sigma}_w^2$. The generated output is a time-varying objective function satisfying Assumption~\ref{ass:prior_knowledge_convex} and the temporal change of a Wiener process which are assumptions relevant for real-world applications.

For the one-dimensional within-model objective functions, the samples at each time step are generated on the feasible set $\mathcal{X} = [-5, 9]$ with a length scale $\boldsymbol\Lambda_{11} = 3$, output variance of $\sigma_k^2 = 1$, and a prior mean of $m(\mathbf{x}) = \mathbf{0}$. Furthermore, the \gls{ui} forgetting factor is set to $\hat{\sigma}_w^2=0.03$, the number of \glspl{vop} per dimension is set to $N_{v/D}=10$, and the bounding functions are $a(\mathbf{X}_v)=0$ and $b(\mathbf{X}_v)=1$. For creating the two-dimensional within-model objective functions, the same settings are chosen except with $N_{v/D}=6$, the feasible set as $\mathcal{X}=[-7,7]^2$, and $\boldsymbol\Lambda_{11} =\boldsymbol\Lambda_{22} = 3$.

The generated objective functions are not \emph{within-model} for the variations using \gls{b2p} forgetting which has to be accounted for. The \gls{ui} forgetting factor $\hat{\sigma}_w^2$ implies the increase in variance after one time step. This is also implied by $\epsilon$ in the Markov chain model of TV-GP-UCB in \eqref{eq:markov_chain} as shown in Appendix \ref{apx:forgetting_factors}. Therefore, $\epsilon$ is set to be $\epsilon=\hat{\sigma}_w^2$.
Furthermore, the Wiener process kernel $k_{T,wp}$ causes the output variance of the composite kernel to increase with $\hat{\sigma}_w^2$ at each time step (see \eqref{eq:sigma_w_hat}). Therefore, to allow a comparison, at each time step, the output variance of the models using \gls{b2p} forgetting is also increased by $\hat{\sigma}_w^2$ as
\begin{equation}
    \sigma_{k,t}^2 = \sigma_{k}^2 + \hat{\sigma}_w^2 \cdot t \text{ (for \gls{b2p} forgetting)}
\end{equation}
with $\sigma_{k,t}^2$ as the output variance at time step $t$.
Nevertheless, caution is needed when quantitatively comparing the forgetting strategies. Here, the synthetic examples in \Cref{sec:1D,sec:2D} are more appropriate. However, qualitative trends between the forgetting strategies can be highlighted from the within-model comparisons.

The variations of the Table~\ref{tab:models} were evaluated on five different objective functions for $D=1$ and $D=2$ generated according to Algorithm~\ref{algo:within_model_comparison}. Five simulations are performed on each objective function using different initializations of $N=15$ initial training points. The initializations were consistent for each variation. Figure~\ref{fig:WMC_cumulative_regret_1D} shows the results for the one-dimensional within-model comparison.
\begin{figure}[h]
    \centering
    %% Creator: Matplotlib, PGF backend
%%
%% To include the figure in your LaTeX document, write
%%   \input{<filename>.pgf}
%%
%% Make sure the required packages are loaded in your preamble
%%   \usepackage{pgf}
%%
%% Figures using additional raster images can only be included by \input if
%% they are in the same directory as the main LaTeX file. For loading figures
%% from other directories you can use the `import` package
%%   \usepackage{import}
%%
%% and then include the figures with
%%   \import{<path to file>}{<filename>.pgf}
%%
%% Matplotlib used the following preamble
%%   \usepackage{fontspec}
%%
\begingroup%
\makeatletter%
\begin{pgfpicture}%
\pgfpathrectangle{\pgfpointorigin}{\pgfqpoint{5.507126in}{2.552693in}}%
\pgfusepath{use as bounding box, clip}%
\begin{pgfscope}%
\pgfsetbuttcap%
\pgfsetmiterjoin%
\definecolor{currentfill}{rgb}{1.000000,1.000000,1.000000}%
\pgfsetfillcolor{currentfill}%
\pgfsetlinewidth{0.000000pt}%
\definecolor{currentstroke}{rgb}{1.000000,1.000000,1.000000}%
\pgfsetstrokecolor{currentstroke}%
\pgfsetdash{}{0pt}%
\pgfpathmoveto{\pgfqpoint{0.000000in}{0.000000in}}%
\pgfpathlineto{\pgfqpoint{5.507126in}{0.000000in}}%
\pgfpathlineto{\pgfqpoint{5.507126in}{2.552693in}}%
\pgfpathlineto{\pgfqpoint{0.000000in}{2.552693in}}%
\pgfpathclose%
\pgfusepath{fill}%
\end{pgfscope}%
\begin{pgfscope}%
\pgfsetbuttcap%
\pgfsetmiterjoin%
\definecolor{currentfill}{rgb}{1.000000,1.000000,1.000000}%
\pgfsetfillcolor{currentfill}%
\pgfsetlinewidth{0.000000pt}%
\definecolor{currentstroke}{rgb}{0.000000,0.000000,0.000000}%
\pgfsetstrokecolor{currentstroke}%
\pgfsetstrokeopacity{0.000000}%
\pgfsetdash{}{0pt}%
\pgfpathmoveto{\pgfqpoint{0.550713in}{0.127635in}}%
\pgfpathlineto{\pgfqpoint{3.744846in}{0.127635in}}%
\pgfpathlineto{\pgfqpoint{3.744846in}{2.425059in}}%
\pgfpathlineto{\pgfqpoint{0.550713in}{2.425059in}}%
\pgfpathclose%
\pgfusepath{fill}%
\end{pgfscope}%
\begin{pgfscope}%
\pgfpathrectangle{\pgfqpoint{0.550713in}{0.127635in}}{\pgfqpoint{3.194133in}{2.297424in}}%
\pgfusepath{clip}%
\pgfsetbuttcap%
\pgfsetmiterjoin%
\definecolor{currentfill}{rgb}{0.631373,0.062745,0.207843}%
\pgfsetfillcolor{currentfill}%
\pgfsetlinewidth{0.752812pt}%
\definecolor{currentstroke}{rgb}{0.000000,0.000000,0.000000}%
\pgfsetstrokecolor{currentstroke}%
\pgfsetdash{}{0pt}%
\pgfpathmoveto{\pgfqpoint{0.592236in}{0.887551in}}%
\pgfpathlineto{\pgfqpoint{0.748749in}{0.887551in}}%
\pgfpathlineto{\pgfqpoint{0.748749in}{1.043926in}}%
\pgfpathlineto{\pgfqpoint{0.592236in}{1.043926in}}%
\pgfpathlineto{\pgfqpoint{0.592236in}{0.887551in}}%
\pgfpathclose%
\pgfusepath{stroke,fill}%
\end{pgfscope}%
\begin{pgfscope}%
\pgfpathrectangle{\pgfqpoint{0.550713in}{0.127635in}}{\pgfqpoint{3.194133in}{2.297424in}}%
\pgfusepath{clip}%
\pgfsetbuttcap%
\pgfsetmiterjoin%
\definecolor{currentfill}{rgb}{0.898039,0.772549,0.752941}%
\pgfsetfillcolor{currentfill}%
\pgfsetlinewidth{0.752812pt}%
\definecolor{currentstroke}{rgb}{0.000000,0.000000,0.000000}%
\pgfsetstrokecolor{currentstroke}%
\pgfsetdash{}{0pt}%
\pgfpathmoveto{\pgfqpoint{0.751943in}{1.430961in}}%
\pgfpathlineto{\pgfqpoint{0.908456in}{1.430961in}}%
\pgfpathlineto{\pgfqpoint{0.908456in}{1.531102in}}%
\pgfpathlineto{\pgfqpoint{0.751943in}{1.531102in}}%
\pgfpathlineto{\pgfqpoint{0.751943in}{1.430961in}}%
\pgfpathclose%
\pgfusepath{stroke,fill}%
\end{pgfscope}%
\begin{pgfscope}%
\pgfpathrectangle{\pgfqpoint{0.550713in}{0.127635in}}{\pgfqpoint{3.194133in}{2.297424in}}%
\pgfusepath{clip}%
\pgfsetbuttcap%
\pgfsetmiterjoin%
\definecolor{currentfill}{rgb}{0.890196,0.000000,0.400000}%
\pgfsetfillcolor{currentfill}%
\pgfsetlinewidth{0.752812pt}%
\definecolor{currentstroke}{rgb}{0.000000,0.000000,0.000000}%
\pgfsetstrokecolor{currentstroke}%
\pgfsetdash{}{0pt}%
\pgfpathmoveto{\pgfqpoint{0.991503in}{0.906798in}}%
\pgfpathlineto{\pgfqpoint{1.148015in}{0.906798in}}%
\pgfpathlineto{\pgfqpoint{1.148015in}{1.107928in}}%
\pgfpathlineto{\pgfqpoint{0.991503in}{1.107928in}}%
\pgfpathlineto{\pgfqpoint{0.991503in}{0.906798in}}%
\pgfpathclose%
\pgfusepath{stroke,fill}%
\end{pgfscope}%
\begin{pgfscope}%
\pgfpathrectangle{\pgfqpoint{0.550713in}{0.127635in}}{\pgfqpoint{3.194133in}{2.297424in}}%
\pgfusepath{clip}%
\pgfsetbuttcap%
\pgfsetmiterjoin%
\definecolor{currentfill}{rgb}{0.976471,0.823529,0.854902}%
\pgfsetfillcolor{currentfill}%
\pgfsetlinewidth{0.752812pt}%
\definecolor{currentstroke}{rgb}{0.000000,0.000000,0.000000}%
\pgfsetstrokecolor{currentstroke}%
\pgfsetdash{}{0pt}%
\pgfpathmoveto{\pgfqpoint{1.151210in}{1.868477in}}%
\pgfpathlineto{\pgfqpoint{1.307722in}{1.868477in}}%
\pgfpathlineto{\pgfqpoint{1.307722in}{2.399686in}}%
\pgfpathlineto{\pgfqpoint{1.151210in}{2.399686in}}%
\pgfpathlineto{\pgfqpoint{1.151210in}{1.868477in}}%
\pgfpathclose%
\pgfusepath{stroke,fill}%
\end{pgfscope}%
\begin{pgfscope}%
\pgfpathrectangle{\pgfqpoint{0.550713in}{0.127635in}}{\pgfqpoint{3.194133in}{2.297424in}}%
\pgfusepath{clip}%
\pgfsetbuttcap%
\pgfsetmiterjoin%
\definecolor{currentfill}{rgb}{0.000000,0.329412,0.623529}%
\pgfsetfillcolor{currentfill}%
\pgfsetlinewidth{0.752812pt}%
\definecolor{currentstroke}{rgb}{0.000000,0.000000,0.000000}%
\pgfsetstrokecolor{currentstroke}%
\pgfsetdash{}{0pt}%
\pgfpathmoveto{\pgfqpoint{1.390770in}{0.759835in}}%
\pgfpathlineto{\pgfqpoint{1.547282in}{0.759835in}}%
\pgfpathlineto{\pgfqpoint{1.547282in}{0.990198in}}%
\pgfpathlineto{\pgfqpoint{1.390770in}{0.990198in}}%
\pgfpathlineto{\pgfqpoint{1.390770in}{0.759835in}}%
\pgfpathclose%
\pgfusepath{stroke,fill}%
\end{pgfscope}%
\begin{pgfscope}%
\pgfpathrectangle{\pgfqpoint{0.550713in}{0.127635in}}{\pgfqpoint{3.194133in}{2.297424in}}%
\pgfusepath{clip}%
\pgfsetbuttcap%
\pgfsetmiterjoin%
\definecolor{currentfill}{rgb}{0.780392,0.866667,0.949020}%
\pgfsetfillcolor{currentfill}%
\pgfsetlinewidth{0.752812pt}%
\definecolor{currentstroke}{rgb}{0.000000,0.000000,0.000000}%
\pgfsetstrokecolor{currentstroke}%
\pgfsetdash{}{0pt}%
\pgfpathmoveto{\pgfqpoint{1.550476in}{0.835710in}}%
\pgfpathlineto{\pgfqpoint{1.706989in}{0.835710in}}%
\pgfpathlineto{\pgfqpoint{1.706989in}{1.062230in}}%
\pgfpathlineto{\pgfqpoint{1.550476in}{1.062230in}}%
\pgfpathlineto{\pgfqpoint{1.550476in}{0.835710in}}%
\pgfpathclose%
\pgfusepath{stroke,fill}%
\end{pgfscope}%
\begin{pgfscope}%
\pgfpathrectangle{\pgfqpoint{0.550713in}{0.127635in}}{\pgfqpoint{3.194133in}{2.297424in}}%
\pgfusepath{clip}%
\pgfsetbuttcap%
\pgfsetmiterjoin%
\definecolor{currentfill}{rgb}{0.000000,0.380392,0.396078}%
\pgfsetfillcolor{currentfill}%
\pgfsetlinewidth{0.752812pt}%
\definecolor{currentstroke}{rgb}{0.000000,0.000000,0.000000}%
\pgfsetstrokecolor{currentstroke}%
\pgfsetdash{}{0pt}%
\pgfpathmoveto{\pgfqpoint{1.790036in}{0.909735in}}%
\pgfpathlineto{\pgfqpoint{1.946549in}{0.909735in}}%
\pgfpathlineto{\pgfqpoint{1.946549in}{1.052177in}}%
\pgfpathlineto{\pgfqpoint{1.790036in}{1.052177in}}%
\pgfpathlineto{\pgfqpoint{1.790036in}{0.909735in}}%
\pgfpathclose%
\pgfusepath{stroke,fill}%
\end{pgfscope}%
\begin{pgfscope}%
\pgfpathrectangle{\pgfqpoint{0.550713in}{0.127635in}}{\pgfqpoint{3.194133in}{2.297424in}}%
\pgfusepath{clip}%
\pgfsetbuttcap%
\pgfsetmiterjoin%
\definecolor{currentfill}{rgb}{0.749020,0.815686,0.819608}%
\pgfsetfillcolor{currentfill}%
\pgfsetlinewidth{0.752812pt}%
\definecolor{currentstroke}{rgb}{0.000000,0.000000,0.000000}%
\pgfsetstrokecolor{currentstroke}%
\pgfsetdash{}{0pt}%
\pgfpathmoveto{\pgfqpoint{1.949743in}{0.939635in}}%
\pgfpathlineto{\pgfqpoint{2.106255in}{0.939635in}}%
\pgfpathlineto{\pgfqpoint{2.106255in}{1.168116in}}%
\pgfpathlineto{\pgfqpoint{1.949743in}{1.168116in}}%
\pgfpathlineto{\pgfqpoint{1.949743in}{0.939635in}}%
\pgfpathclose%
\pgfusepath{stroke,fill}%
\end{pgfscope}%
\begin{pgfscope}%
\pgfpathrectangle{\pgfqpoint{0.550713in}{0.127635in}}{\pgfqpoint{3.194133in}{2.297424in}}%
\pgfusepath{clip}%
\pgfsetbuttcap%
\pgfsetmiterjoin%
\definecolor{currentfill}{rgb}{0.380392,0.129412,0.345098}%
\pgfsetfillcolor{currentfill}%
\pgfsetlinewidth{0.752812pt}%
\definecolor{currentstroke}{rgb}{0.000000,0.000000,0.000000}%
\pgfsetstrokecolor{currentstroke}%
\pgfsetdash{}{0pt}%
\pgfpathmoveto{\pgfqpoint{2.189303in}{0.630518in}}%
\pgfpathlineto{\pgfqpoint{2.345815in}{0.630518in}}%
\pgfpathlineto{\pgfqpoint{2.345815in}{0.689453in}}%
\pgfpathlineto{\pgfqpoint{2.189303in}{0.689453in}}%
\pgfpathlineto{\pgfqpoint{2.189303in}{0.630518in}}%
\pgfpathclose%
\pgfusepath{stroke,fill}%
\end{pgfscope}%
\begin{pgfscope}%
\pgfpathrectangle{\pgfqpoint{0.550713in}{0.127635in}}{\pgfqpoint{3.194133in}{2.297424in}}%
\pgfusepath{clip}%
\pgfsetbuttcap%
\pgfsetmiterjoin%
\definecolor{currentfill}{rgb}{0.823529,0.752941,0.803922}%
\pgfsetfillcolor{currentfill}%
\pgfsetlinewidth{0.752812pt}%
\definecolor{currentstroke}{rgb}{0.000000,0.000000,0.000000}%
\pgfsetstrokecolor{currentstroke}%
\pgfsetdash{}{0pt}%
\pgfpathmoveto{\pgfqpoint{2.349010in}{0.709846in}}%
\pgfpathlineto{\pgfqpoint{2.505522in}{0.709846in}}%
\pgfpathlineto{\pgfqpoint{2.505522in}{0.746785in}}%
\pgfpathlineto{\pgfqpoint{2.349010in}{0.746785in}}%
\pgfpathlineto{\pgfqpoint{2.349010in}{0.709846in}}%
\pgfpathclose%
\pgfusepath{stroke,fill}%
\end{pgfscope}%
\begin{pgfscope}%
\pgfpathrectangle{\pgfqpoint{0.550713in}{0.127635in}}{\pgfqpoint{3.194133in}{2.297424in}}%
\pgfusepath{clip}%
\pgfsetbuttcap%
\pgfsetmiterjoin%
\definecolor{currentfill}{rgb}{0.964706,0.658824,0.000000}%
\pgfsetfillcolor{currentfill}%
\pgfsetlinewidth{0.752812pt}%
\definecolor{currentstroke}{rgb}{0.000000,0.000000,0.000000}%
\pgfsetstrokecolor{currentstroke}%
\pgfsetdash{}{0pt}%
\pgfpathmoveto{\pgfqpoint{2.588570in}{0.646303in}}%
\pgfpathlineto{\pgfqpoint{2.745082in}{0.646303in}}%
\pgfpathlineto{\pgfqpoint{2.745082in}{0.680680in}}%
\pgfpathlineto{\pgfqpoint{2.588570in}{0.680680in}}%
\pgfpathlineto{\pgfqpoint{2.588570in}{0.646303in}}%
\pgfpathclose%
\pgfusepath{stroke,fill}%
\end{pgfscope}%
\begin{pgfscope}%
\pgfpathrectangle{\pgfqpoint{0.550713in}{0.127635in}}{\pgfqpoint{3.194133in}{2.297424in}}%
\pgfusepath{clip}%
\pgfsetbuttcap%
\pgfsetmiterjoin%
\definecolor{currentfill}{rgb}{0.996078,0.917647,0.788235}%
\pgfsetfillcolor{currentfill}%
\pgfsetlinewidth{0.752812pt}%
\definecolor{currentstroke}{rgb}{0.000000,0.000000,0.000000}%
\pgfsetstrokecolor{currentstroke}%
\pgfsetdash{}{0pt}%
\pgfpathmoveto{\pgfqpoint{2.748276in}{0.714258in}}%
\pgfpathlineto{\pgfqpoint{2.904789in}{0.714258in}}%
\pgfpathlineto{\pgfqpoint{2.904789in}{0.765359in}}%
\pgfpathlineto{\pgfqpoint{2.748276in}{0.765359in}}%
\pgfpathlineto{\pgfqpoint{2.748276in}{0.714258in}}%
\pgfpathclose%
\pgfusepath{stroke,fill}%
\end{pgfscope}%
\begin{pgfscope}%
\pgfpathrectangle{\pgfqpoint{0.550713in}{0.127635in}}{\pgfqpoint{3.194133in}{2.297424in}}%
\pgfusepath{clip}%
\pgfsetbuttcap%
\pgfsetmiterjoin%
\definecolor{currentfill}{rgb}{0.341176,0.670588,0.152941}%
\pgfsetfillcolor{currentfill}%
\pgfsetlinewidth{0.752812pt}%
\definecolor{currentstroke}{rgb}{0.000000,0.000000,0.000000}%
\pgfsetstrokecolor{currentstroke}%
\pgfsetdash{}{0pt}%
\pgfpathmoveto{\pgfqpoint{2.987836in}{0.592972in}}%
\pgfpathlineto{\pgfqpoint{3.144349in}{0.592972in}}%
\pgfpathlineto{\pgfqpoint{3.144349in}{0.694924in}}%
\pgfpathlineto{\pgfqpoint{2.987836in}{0.694924in}}%
\pgfpathlineto{\pgfqpoint{2.987836in}{0.592972in}}%
\pgfpathclose%
\pgfusepath{stroke,fill}%
\end{pgfscope}%
\begin{pgfscope}%
\pgfpathrectangle{\pgfqpoint{0.550713in}{0.127635in}}{\pgfqpoint{3.194133in}{2.297424in}}%
\pgfusepath{clip}%
\pgfsetbuttcap%
\pgfsetmiterjoin%
\definecolor{currentfill}{rgb}{0.866667,0.921569,0.807843}%
\pgfsetfillcolor{currentfill}%
\pgfsetlinewidth{0.752812pt}%
\definecolor{currentstroke}{rgb}{0.000000,0.000000,0.000000}%
\pgfsetstrokecolor{currentstroke}%
\pgfsetdash{}{0pt}%
\pgfpathmoveto{\pgfqpoint{3.147543in}{0.603692in}}%
\pgfpathlineto{\pgfqpoint{3.304055in}{0.603692in}}%
\pgfpathlineto{\pgfqpoint{3.304055in}{0.680073in}}%
\pgfpathlineto{\pgfqpoint{3.147543in}{0.680073in}}%
\pgfpathlineto{\pgfqpoint{3.147543in}{0.603692in}}%
\pgfpathclose%
\pgfusepath{stroke,fill}%
\end{pgfscope}%
\begin{pgfscope}%
\pgfpathrectangle{\pgfqpoint{0.550713in}{0.127635in}}{\pgfqpoint{3.194133in}{2.297424in}}%
\pgfusepath{clip}%
\pgfsetbuttcap%
\pgfsetmiterjoin%
\definecolor{currentfill}{rgb}{0.478431,0.435294,0.674510}%
\pgfsetfillcolor{currentfill}%
\pgfsetlinewidth{0.752812pt}%
\definecolor{currentstroke}{rgb}{0.000000,0.000000,0.000000}%
\pgfsetstrokecolor{currentstroke}%
\pgfsetdash{}{0pt}%
\pgfpathmoveto{\pgfqpoint{3.387103in}{0.619842in}}%
\pgfpathlineto{\pgfqpoint{3.543615in}{0.619842in}}%
\pgfpathlineto{\pgfqpoint{3.543615in}{0.699327in}}%
\pgfpathlineto{\pgfqpoint{3.387103in}{0.699327in}}%
\pgfpathlineto{\pgfqpoint{3.387103in}{0.619842in}}%
\pgfpathclose%
\pgfusepath{stroke,fill}%
\end{pgfscope}%
\begin{pgfscope}%
\pgfpathrectangle{\pgfqpoint{0.550713in}{0.127635in}}{\pgfqpoint{3.194133in}{2.297424in}}%
\pgfusepath{clip}%
\pgfsetbuttcap%
\pgfsetmiterjoin%
\definecolor{currentfill}{rgb}{0.870588,0.854902,0.921569}%
\pgfsetfillcolor{currentfill}%
\pgfsetlinewidth{0.752812pt}%
\definecolor{currentstroke}{rgb}{0.000000,0.000000,0.000000}%
\pgfsetstrokecolor{currentstroke}%
\pgfsetdash{}{0pt}%
\pgfpathmoveto{\pgfqpoint{3.546809in}{0.629160in}}%
\pgfpathlineto{\pgfqpoint{3.703322in}{0.629160in}}%
\pgfpathlineto{\pgfqpoint{3.703322in}{0.687534in}}%
\pgfpathlineto{\pgfqpoint{3.546809in}{0.687534in}}%
\pgfpathlineto{\pgfqpoint{3.546809in}{0.629160in}}%
\pgfpathclose%
\pgfusepath{stroke,fill}%
\end{pgfscope}%
\begin{pgfscope}%
\pgfpathrectangle{\pgfqpoint{0.550713in}{0.127635in}}{\pgfqpoint{3.194133in}{2.297424in}}%
\pgfusepath{clip}%
\pgfsetbuttcap%
\pgfsetmiterjoin%
\definecolor{currentfill}{rgb}{0.000000,0.000000,0.000000}%
\pgfsetfillcolor{currentfill}%
\pgfsetlinewidth{0.376406pt}%
\definecolor{currentstroke}{rgb}{0.000000,0.000000,0.000000}%
\pgfsetstrokecolor{currentstroke}%
\pgfsetdash{}{0pt}%
\pgfpathmoveto{\pgfqpoint{0.750346in}{0.127635in}}%
\pgfpathlineto{\pgfqpoint{0.750346in}{0.127635in}}%
\pgfpathlineto{\pgfqpoint{0.750346in}{0.127635in}}%
\pgfpathlineto{\pgfqpoint{0.750346in}{0.127635in}}%
\pgfpathclose%
\pgfusepath{stroke,fill}%
\end{pgfscope}%
\begin{pgfscope}%
\pgfpathrectangle{\pgfqpoint{0.550713in}{0.127635in}}{\pgfqpoint{3.194133in}{2.297424in}}%
\pgfusepath{clip}%
\pgfsetbuttcap%
\pgfsetmiterjoin%
\definecolor{currentfill}{rgb}{0.813235,0.819118,0.822059}%
\pgfsetfillcolor{currentfill}%
\pgfsetlinewidth{0.376406pt}%
\definecolor{currentstroke}{rgb}{0.000000,0.000000,0.000000}%
\pgfsetstrokecolor{currentstroke}%
\pgfsetdash{}{0pt}%
\pgfpathmoveto{\pgfqpoint{0.750346in}{0.127635in}}%
\pgfpathlineto{\pgfqpoint{0.750346in}{0.127635in}}%
\pgfpathlineto{\pgfqpoint{0.750346in}{0.127635in}}%
\pgfpathlineto{\pgfqpoint{0.750346in}{0.127635in}}%
\pgfpathclose%
\pgfusepath{stroke,fill}%
\end{pgfscope}%
\begin{pgfscope}%
\pgfsetbuttcap%
\pgfsetroundjoin%
\definecolor{currentfill}{rgb}{0.000000,0.000000,0.000000}%
\pgfsetfillcolor{currentfill}%
\pgfsetlinewidth{0.803000pt}%
\definecolor{currentstroke}{rgb}{0.000000,0.000000,0.000000}%
\pgfsetstrokecolor{currentstroke}%
\pgfsetdash{}{0pt}%
\pgfsys@defobject{currentmarker}{\pgfqpoint{-0.048611in}{0.000000in}}{\pgfqpoint{-0.000000in}{0.000000in}}{%
\pgfpathmoveto{\pgfqpoint{-0.000000in}{0.000000in}}%
\pgfpathlineto{\pgfqpoint{-0.048611in}{0.000000in}}%
\pgfusepath{stroke,fill}%
}%
\begin{pgfscope}%
\pgfsys@transformshift{0.550713in}{0.127635in}%
\pgfsys@useobject{currentmarker}{}%
\end{pgfscope}%
\end{pgfscope}%
\begin{pgfscope}%
\definecolor{textcolor}{rgb}{0.000000,0.000000,0.000000}%
\pgfsetstrokecolor{textcolor}%
\pgfsetfillcolor{textcolor}%
\pgftext[x=0.384046in, y=0.079440in, left, base]{\color{textcolor}\rmfamily\fontsize{10.000000}{12.000000}\selectfont \(\displaystyle {0}\)}%
\end{pgfscope}%
\begin{pgfscope}%
\pgfsetbuttcap%
\pgfsetroundjoin%
\definecolor{currentfill}{rgb}{0.000000,0.000000,0.000000}%
\pgfsetfillcolor{currentfill}%
\pgfsetlinewidth{0.803000pt}%
\definecolor{currentstroke}{rgb}{0.000000,0.000000,0.000000}%
\pgfsetstrokecolor{currentstroke}%
\pgfsetdash{}{0pt}%
\pgfsys@defobject{currentmarker}{\pgfqpoint{-0.048611in}{0.000000in}}{\pgfqpoint{-0.000000in}{0.000000in}}{%
\pgfpathmoveto{\pgfqpoint{-0.000000in}{0.000000in}}%
\pgfpathlineto{\pgfqpoint{-0.048611in}{0.000000in}}%
\pgfusepath{stroke,fill}%
}%
\begin{pgfscope}%
\pgfsys@transformshift{0.550713in}{0.587119in}%
\pgfsys@useobject{currentmarker}{}%
\end{pgfscope}%
\end{pgfscope}%
\begin{pgfscope}%
\definecolor{textcolor}{rgb}{0.000000,0.000000,0.000000}%
\pgfsetstrokecolor{textcolor}%
\pgfsetfillcolor{textcolor}%
\pgftext[x=0.314601in, y=0.538925in, left, base]{\color{textcolor}\rmfamily\fontsize{10.000000}{12.000000}\selectfont \(\displaystyle {20}\)}%
\end{pgfscope}%
\begin{pgfscope}%
\pgfsetbuttcap%
\pgfsetroundjoin%
\definecolor{currentfill}{rgb}{0.000000,0.000000,0.000000}%
\pgfsetfillcolor{currentfill}%
\pgfsetlinewidth{0.803000pt}%
\definecolor{currentstroke}{rgb}{0.000000,0.000000,0.000000}%
\pgfsetstrokecolor{currentstroke}%
\pgfsetdash{}{0pt}%
\pgfsys@defobject{currentmarker}{\pgfqpoint{-0.048611in}{0.000000in}}{\pgfqpoint{-0.000000in}{0.000000in}}{%
\pgfpathmoveto{\pgfqpoint{-0.000000in}{0.000000in}}%
\pgfpathlineto{\pgfqpoint{-0.048611in}{0.000000in}}%
\pgfusepath{stroke,fill}%
}%
\begin{pgfscope}%
\pgfsys@transformshift{0.550713in}{1.046604in}%
\pgfsys@useobject{currentmarker}{}%
\end{pgfscope}%
\end{pgfscope}%
\begin{pgfscope}%
\definecolor{textcolor}{rgb}{0.000000,0.000000,0.000000}%
\pgfsetstrokecolor{textcolor}%
\pgfsetfillcolor{textcolor}%
\pgftext[x=0.314601in, y=0.998410in, left, base]{\color{textcolor}\rmfamily\fontsize{10.000000}{12.000000}\selectfont \(\displaystyle {40}\)}%
\end{pgfscope}%
\begin{pgfscope}%
\pgfsetbuttcap%
\pgfsetroundjoin%
\definecolor{currentfill}{rgb}{0.000000,0.000000,0.000000}%
\pgfsetfillcolor{currentfill}%
\pgfsetlinewidth{0.803000pt}%
\definecolor{currentstroke}{rgb}{0.000000,0.000000,0.000000}%
\pgfsetstrokecolor{currentstroke}%
\pgfsetdash{}{0pt}%
\pgfsys@defobject{currentmarker}{\pgfqpoint{-0.048611in}{0.000000in}}{\pgfqpoint{-0.000000in}{0.000000in}}{%
\pgfpathmoveto{\pgfqpoint{-0.000000in}{0.000000in}}%
\pgfpathlineto{\pgfqpoint{-0.048611in}{0.000000in}}%
\pgfusepath{stroke,fill}%
}%
\begin{pgfscope}%
\pgfsys@transformshift{0.550713in}{1.506089in}%
\pgfsys@useobject{currentmarker}{}%
\end{pgfscope}%
\end{pgfscope}%
\begin{pgfscope}%
\definecolor{textcolor}{rgb}{0.000000,0.000000,0.000000}%
\pgfsetstrokecolor{textcolor}%
\pgfsetfillcolor{textcolor}%
\pgftext[x=0.314601in, y=1.457895in, left, base]{\color{textcolor}\rmfamily\fontsize{10.000000}{12.000000}\selectfont \(\displaystyle {60}\)}%
\end{pgfscope}%
\begin{pgfscope}%
\pgfsetbuttcap%
\pgfsetroundjoin%
\definecolor{currentfill}{rgb}{0.000000,0.000000,0.000000}%
\pgfsetfillcolor{currentfill}%
\pgfsetlinewidth{0.803000pt}%
\definecolor{currentstroke}{rgb}{0.000000,0.000000,0.000000}%
\pgfsetstrokecolor{currentstroke}%
\pgfsetdash{}{0pt}%
\pgfsys@defobject{currentmarker}{\pgfqpoint{-0.048611in}{0.000000in}}{\pgfqpoint{-0.000000in}{0.000000in}}{%
\pgfpathmoveto{\pgfqpoint{-0.000000in}{0.000000in}}%
\pgfpathlineto{\pgfqpoint{-0.048611in}{0.000000in}}%
\pgfusepath{stroke,fill}%
}%
\begin{pgfscope}%
\pgfsys@transformshift{0.550713in}{1.965574in}%
\pgfsys@useobject{currentmarker}{}%
\end{pgfscope}%
\end{pgfscope}%
\begin{pgfscope}%
\definecolor{textcolor}{rgb}{0.000000,0.000000,0.000000}%
\pgfsetstrokecolor{textcolor}%
\pgfsetfillcolor{textcolor}%
\pgftext[x=0.314601in, y=1.917379in, left, base]{\color{textcolor}\rmfamily\fontsize{10.000000}{12.000000}\selectfont \(\displaystyle {80}\)}%
\end{pgfscope}%
\begin{pgfscope}%
\pgfsetbuttcap%
\pgfsetroundjoin%
\definecolor{currentfill}{rgb}{0.000000,0.000000,0.000000}%
\pgfsetfillcolor{currentfill}%
\pgfsetlinewidth{0.803000pt}%
\definecolor{currentstroke}{rgb}{0.000000,0.000000,0.000000}%
\pgfsetstrokecolor{currentstroke}%
\pgfsetdash{}{0pt}%
\pgfsys@defobject{currentmarker}{\pgfqpoint{-0.048611in}{0.000000in}}{\pgfqpoint{-0.000000in}{0.000000in}}{%
\pgfpathmoveto{\pgfqpoint{-0.000000in}{0.000000in}}%
\pgfpathlineto{\pgfqpoint{-0.048611in}{0.000000in}}%
\pgfusepath{stroke,fill}%
}%
\begin{pgfscope}%
\pgfsys@transformshift{0.550713in}{2.425059in}%
\pgfsys@useobject{currentmarker}{}%
\end{pgfscope}%
\end{pgfscope}%
\begin{pgfscope}%
\definecolor{textcolor}{rgb}{0.000000,0.000000,0.000000}%
\pgfsetstrokecolor{textcolor}%
\pgfsetfillcolor{textcolor}%
\pgftext[x=0.245156in, y=2.376864in, left, base]{\color{textcolor}\rmfamily\fontsize{10.000000}{12.000000}\selectfont \(\displaystyle {100}\)}%
\end{pgfscope}%
\begin{pgfscope}%
\definecolor{textcolor}{rgb}{0.000000,0.000000,0.000000}%
\pgfsetstrokecolor{textcolor}%
\pgfsetfillcolor{textcolor}%
\pgftext[x=0.189601in,y=1.276347in,,bottom,rotate=90.000000]{\color{textcolor}\rmfamily\fontsize{10.000000}{12.000000}\selectfont \(\displaystyle R_T\)}%
\end{pgfscope}%
\begin{pgfscope}%
\pgfpathrectangle{\pgfqpoint{0.550713in}{0.127635in}}{\pgfqpoint{3.194133in}{2.297424in}}%
\pgfusepath{clip}%
\pgfsetbuttcap%
\pgfsetroundjoin%
\pgfsetlinewidth{0.501875pt}%
\definecolor{currentstroke}{rgb}{0.392157,0.396078,0.403922}%
\pgfsetstrokecolor{currentstroke}%
\pgfsetdash{}{0pt}%
\pgfpathmoveto{\pgfqpoint{0.949979in}{0.127635in}}%
\pgfpathlineto{\pgfqpoint{0.949979in}{2.425059in}}%
\pgfusepath{stroke}%
\end{pgfscope}%
\begin{pgfscope}%
\pgfpathrectangle{\pgfqpoint{0.550713in}{0.127635in}}{\pgfqpoint{3.194133in}{2.297424in}}%
\pgfusepath{clip}%
\pgfsetbuttcap%
\pgfsetroundjoin%
\pgfsetlinewidth{0.501875pt}%
\definecolor{currentstroke}{rgb}{0.392157,0.396078,0.403922}%
\pgfsetstrokecolor{currentstroke}%
\pgfsetdash{}{0pt}%
\pgfpathmoveto{\pgfqpoint{1.349246in}{0.127635in}}%
\pgfpathlineto{\pgfqpoint{1.349246in}{2.425059in}}%
\pgfusepath{stroke}%
\end{pgfscope}%
\begin{pgfscope}%
\pgfpathrectangle{\pgfqpoint{0.550713in}{0.127635in}}{\pgfqpoint{3.194133in}{2.297424in}}%
\pgfusepath{clip}%
\pgfsetbuttcap%
\pgfsetroundjoin%
\pgfsetlinewidth{0.501875pt}%
\definecolor{currentstroke}{rgb}{0.392157,0.396078,0.403922}%
\pgfsetstrokecolor{currentstroke}%
\pgfsetdash{}{0pt}%
\pgfpathmoveto{\pgfqpoint{1.748513in}{0.127635in}}%
\pgfpathlineto{\pgfqpoint{1.748513in}{2.425059in}}%
\pgfusepath{stroke}%
\end{pgfscope}%
\begin{pgfscope}%
\pgfpathrectangle{\pgfqpoint{0.550713in}{0.127635in}}{\pgfqpoint{3.194133in}{2.297424in}}%
\pgfusepath{clip}%
\pgfsetbuttcap%
\pgfsetroundjoin%
\pgfsetlinewidth{0.501875pt}%
\definecolor{currentstroke}{rgb}{0.392157,0.396078,0.403922}%
\pgfsetstrokecolor{currentstroke}%
\pgfsetdash{}{0pt}%
\pgfpathmoveto{\pgfqpoint{2.147779in}{0.127635in}}%
\pgfpathlineto{\pgfqpoint{2.147779in}{2.425059in}}%
\pgfusepath{stroke}%
\end{pgfscope}%
\begin{pgfscope}%
\pgfpathrectangle{\pgfqpoint{0.550713in}{0.127635in}}{\pgfqpoint{3.194133in}{2.297424in}}%
\pgfusepath{clip}%
\pgfsetbuttcap%
\pgfsetroundjoin%
\pgfsetlinewidth{0.501875pt}%
\definecolor{currentstroke}{rgb}{0.392157,0.396078,0.403922}%
\pgfsetstrokecolor{currentstroke}%
\pgfsetdash{}{0pt}%
\pgfpathmoveto{\pgfqpoint{2.547046in}{0.127635in}}%
\pgfpathlineto{\pgfqpoint{2.547046in}{2.425059in}}%
\pgfusepath{stroke}%
\end{pgfscope}%
\begin{pgfscope}%
\pgfpathrectangle{\pgfqpoint{0.550713in}{0.127635in}}{\pgfqpoint{3.194133in}{2.297424in}}%
\pgfusepath{clip}%
\pgfsetbuttcap%
\pgfsetroundjoin%
\pgfsetlinewidth{0.501875pt}%
\definecolor{currentstroke}{rgb}{0.392157,0.396078,0.403922}%
\pgfsetstrokecolor{currentstroke}%
\pgfsetdash{}{0pt}%
\pgfpathmoveto{\pgfqpoint{2.946312in}{0.127635in}}%
\pgfpathlineto{\pgfqpoint{2.946312in}{2.425059in}}%
\pgfusepath{stroke}%
\end{pgfscope}%
\begin{pgfscope}%
\pgfpathrectangle{\pgfqpoint{0.550713in}{0.127635in}}{\pgfqpoint{3.194133in}{2.297424in}}%
\pgfusepath{clip}%
\pgfsetbuttcap%
\pgfsetroundjoin%
\pgfsetlinewidth{0.501875pt}%
\definecolor{currentstroke}{rgb}{0.392157,0.396078,0.403922}%
\pgfsetstrokecolor{currentstroke}%
\pgfsetdash{}{0pt}%
\pgfpathmoveto{\pgfqpoint{3.345579in}{0.127635in}}%
\pgfpathlineto{\pgfqpoint{3.345579in}{2.425059in}}%
\pgfusepath{stroke}%
\end{pgfscope}%
\begin{pgfscope}%
\pgfpathrectangle{\pgfqpoint{0.550713in}{0.127635in}}{\pgfqpoint{3.194133in}{2.297424in}}%
\pgfusepath{clip}%
\pgfsetbuttcap%
\pgfsetroundjoin%
\pgfsetlinewidth{0.853187pt}%
\definecolor{currentstroke}{rgb}{0.392157,0.396078,0.403922}%
\pgfsetstrokecolor{currentstroke}%
\pgfsetdash{{3.145000pt}{1.360000pt}}{0.000000pt}%
\pgfpathmoveto{\pgfqpoint{0.540713in}{1.795456in}}%
\pgfpathlineto{\pgfqpoint{3.754846in}{1.795456in}}%
\pgfusepath{stroke}%
\end{pgfscope}%
\begin{pgfscope}%
\pgfpathrectangle{\pgfqpoint{0.550713in}{0.127635in}}{\pgfqpoint{3.194133in}{2.297424in}}%
\pgfusepath{clip}%
\pgfsetrectcap%
\pgfsetroundjoin%
\pgfsetlinewidth{0.752812pt}%
\definecolor{currentstroke}{rgb}{0.000000,0.000000,0.000000}%
\pgfsetstrokecolor{currentstroke}%
\pgfsetdash{}{0pt}%
\pgfpathmoveto{\pgfqpoint{0.670493in}{0.887551in}}%
\pgfpathlineto{\pgfqpoint{0.670493in}{0.770698in}}%
\pgfusepath{stroke}%
\end{pgfscope}%
\begin{pgfscope}%
\pgfpathrectangle{\pgfqpoint{0.550713in}{0.127635in}}{\pgfqpoint{3.194133in}{2.297424in}}%
\pgfusepath{clip}%
\pgfsetrectcap%
\pgfsetroundjoin%
\pgfsetlinewidth{0.752812pt}%
\definecolor{currentstroke}{rgb}{0.000000,0.000000,0.000000}%
\pgfsetstrokecolor{currentstroke}%
\pgfsetdash{}{0pt}%
\pgfpathmoveto{\pgfqpoint{0.670493in}{1.043926in}}%
\pgfpathlineto{\pgfqpoint{0.670493in}{1.050800in}}%
\pgfusepath{stroke}%
\end{pgfscope}%
\begin{pgfscope}%
\pgfpathrectangle{\pgfqpoint{0.550713in}{0.127635in}}{\pgfqpoint{3.194133in}{2.297424in}}%
\pgfusepath{clip}%
\pgfsetrectcap%
\pgfsetroundjoin%
\pgfsetlinewidth{0.752812pt}%
\definecolor{currentstroke}{rgb}{0.000000,0.000000,0.000000}%
\pgfsetstrokecolor{currentstroke}%
\pgfsetdash{}{0pt}%
\pgfpathmoveto{\pgfqpoint{0.631364in}{0.770698in}}%
\pgfpathlineto{\pgfqpoint{0.709621in}{0.770698in}}%
\pgfusepath{stroke}%
\end{pgfscope}%
\begin{pgfscope}%
\pgfpathrectangle{\pgfqpoint{0.550713in}{0.127635in}}{\pgfqpoint{3.194133in}{2.297424in}}%
\pgfusepath{clip}%
\pgfsetrectcap%
\pgfsetroundjoin%
\pgfsetlinewidth{0.752812pt}%
\definecolor{currentstroke}{rgb}{0.000000,0.000000,0.000000}%
\pgfsetstrokecolor{currentstroke}%
\pgfsetdash{}{0pt}%
\pgfpathmoveto{\pgfqpoint{0.631364in}{1.050800in}}%
\pgfpathlineto{\pgfqpoint{0.709621in}{1.050800in}}%
\pgfusepath{stroke}%
\end{pgfscope}%
\begin{pgfscope}%
\pgfpathrectangle{\pgfqpoint{0.550713in}{0.127635in}}{\pgfqpoint{3.194133in}{2.297424in}}%
\pgfusepath{clip}%
\pgfsetbuttcap%
\pgfsetmiterjoin%
\definecolor{currentfill}{rgb}{0.000000,0.000000,0.000000}%
\pgfsetfillcolor{currentfill}%
\pgfsetlinewidth{1.003750pt}%
\definecolor{currentstroke}{rgb}{0.000000,0.000000,0.000000}%
\pgfsetstrokecolor{currentstroke}%
\pgfsetdash{}{0pt}%
\pgfsys@defobject{currentmarker}{\pgfqpoint{-0.011785in}{-0.019642in}}{\pgfqpoint{0.011785in}{0.019642in}}{%
\pgfpathmoveto{\pgfqpoint{-0.000000in}{-0.019642in}}%
\pgfpathlineto{\pgfqpoint{0.011785in}{0.000000in}}%
\pgfpathlineto{\pgfqpoint{0.000000in}{0.019642in}}%
\pgfpathlineto{\pgfqpoint{-0.011785in}{0.000000in}}%
\pgfpathclose%
\pgfusepath{stroke,fill}%
}%
\begin{pgfscope}%
\pgfsys@transformshift{0.670493in}{1.663224in}%
\pgfsys@useobject{currentmarker}{}%
\end{pgfscope}%
\begin{pgfscope}%
\pgfsys@transformshift{0.670493in}{1.619733in}%
\pgfsys@useobject{currentmarker}{}%
\end{pgfscope}%
\begin{pgfscope}%
\pgfsys@transformshift{0.670493in}{1.637833in}%
\pgfsys@useobject{currentmarker}{}%
\end{pgfscope}%
\begin{pgfscope}%
\pgfsys@transformshift{0.670493in}{1.619384in}%
\pgfsys@useobject{currentmarker}{}%
\end{pgfscope}%
\begin{pgfscope}%
\pgfsys@transformshift{0.670493in}{1.594631in}%
\pgfsys@useobject{currentmarker}{}%
\end{pgfscope}%
\end{pgfscope}%
\begin{pgfscope}%
\pgfpathrectangle{\pgfqpoint{0.550713in}{0.127635in}}{\pgfqpoint{3.194133in}{2.297424in}}%
\pgfusepath{clip}%
\pgfsetrectcap%
\pgfsetroundjoin%
\pgfsetlinewidth{0.752812pt}%
\definecolor{currentstroke}{rgb}{0.000000,0.000000,0.000000}%
\pgfsetstrokecolor{currentstroke}%
\pgfsetdash{}{0pt}%
\pgfpathmoveto{\pgfqpoint{0.830199in}{1.430961in}}%
\pgfpathlineto{\pgfqpoint{0.830199in}{1.331462in}}%
\pgfusepath{stroke}%
\end{pgfscope}%
\begin{pgfscope}%
\pgfpathrectangle{\pgfqpoint{0.550713in}{0.127635in}}{\pgfqpoint{3.194133in}{2.297424in}}%
\pgfusepath{clip}%
\pgfsetrectcap%
\pgfsetroundjoin%
\pgfsetlinewidth{0.752812pt}%
\definecolor{currentstroke}{rgb}{0.000000,0.000000,0.000000}%
\pgfsetstrokecolor{currentstroke}%
\pgfsetdash{}{0pt}%
\pgfpathmoveto{\pgfqpoint{0.830199in}{1.531102in}}%
\pgfpathlineto{\pgfqpoint{0.830199in}{1.560454in}}%
\pgfusepath{stroke}%
\end{pgfscope}%
\begin{pgfscope}%
\pgfpathrectangle{\pgfqpoint{0.550713in}{0.127635in}}{\pgfqpoint{3.194133in}{2.297424in}}%
\pgfusepath{clip}%
\pgfsetrectcap%
\pgfsetroundjoin%
\pgfsetlinewidth{0.752812pt}%
\definecolor{currentstroke}{rgb}{0.000000,0.000000,0.000000}%
\pgfsetstrokecolor{currentstroke}%
\pgfsetdash{}{0pt}%
\pgfpathmoveto{\pgfqpoint{0.791071in}{1.331462in}}%
\pgfpathlineto{\pgfqpoint{0.869327in}{1.331462in}}%
\pgfusepath{stroke}%
\end{pgfscope}%
\begin{pgfscope}%
\pgfpathrectangle{\pgfqpoint{0.550713in}{0.127635in}}{\pgfqpoint{3.194133in}{2.297424in}}%
\pgfusepath{clip}%
\pgfsetrectcap%
\pgfsetroundjoin%
\pgfsetlinewidth{0.752812pt}%
\definecolor{currentstroke}{rgb}{0.000000,0.000000,0.000000}%
\pgfsetstrokecolor{currentstroke}%
\pgfsetdash{}{0pt}%
\pgfpathmoveto{\pgfqpoint{0.791071in}{1.560454in}}%
\pgfpathlineto{\pgfqpoint{0.869327in}{1.560454in}}%
\pgfusepath{stroke}%
\end{pgfscope}%
\begin{pgfscope}%
\pgfpathrectangle{\pgfqpoint{0.550713in}{0.127635in}}{\pgfqpoint{3.194133in}{2.297424in}}%
\pgfusepath{clip}%
\pgfsetbuttcap%
\pgfsetmiterjoin%
\definecolor{currentfill}{rgb}{0.000000,0.000000,0.000000}%
\pgfsetfillcolor{currentfill}%
\pgfsetlinewidth{1.003750pt}%
\definecolor{currentstroke}{rgb}{0.000000,0.000000,0.000000}%
\pgfsetstrokecolor{currentstroke}%
\pgfsetdash{}{0pt}%
\pgfsys@defobject{currentmarker}{\pgfqpoint{-0.011785in}{-0.019642in}}{\pgfqpoint{0.011785in}{0.019642in}}{%
\pgfpathmoveto{\pgfqpoint{-0.000000in}{-0.019642in}}%
\pgfpathlineto{\pgfqpoint{0.011785in}{0.000000in}}%
\pgfpathlineto{\pgfqpoint{0.000000in}{0.019642in}}%
\pgfpathlineto{\pgfqpoint{-0.011785in}{0.000000in}}%
\pgfpathclose%
\pgfusepath{stroke,fill}%
}%
\begin{pgfscope}%
\pgfsys@transformshift{0.830199in}{2.171362in}%
\pgfsys@useobject{currentmarker}{}%
\end{pgfscope}%
\begin{pgfscope}%
\pgfsys@transformshift{0.830199in}{2.212629in}%
\pgfsys@useobject{currentmarker}{}%
\end{pgfscope}%
\begin{pgfscope}%
\pgfsys@transformshift{0.830199in}{2.090462in}%
\pgfsys@useobject{currentmarker}{}%
\end{pgfscope}%
\begin{pgfscope}%
\pgfsys@transformshift{0.830199in}{2.055574in}%
\pgfsys@useobject{currentmarker}{}%
\end{pgfscope}%
\begin{pgfscope}%
\pgfsys@transformshift{0.830199in}{2.153689in}%
\pgfsys@useobject{currentmarker}{}%
\end{pgfscope}%
\end{pgfscope}%
\begin{pgfscope}%
\pgfpathrectangle{\pgfqpoint{0.550713in}{0.127635in}}{\pgfqpoint{3.194133in}{2.297424in}}%
\pgfusepath{clip}%
\pgfsetrectcap%
\pgfsetroundjoin%
\pgfsetlinewidth{0.752812pt}%
\definecolor{currentstroke}{rgb}{0.000000,0.000000,0.000000}%
\pgfsetstrokecolor{currentstroke}%
\pgfsetdash{}{0pt}%
\pgfpathmoveto{\pgfqpoint{1.069759in}{0.906798in}}%
\pgfpathlineto{\pgfqpoint{1.069759in}{0.807448in}}%
\pgfusepath{stroke}%
\end{pgfscope}%
\begin{pgfscope}%
\pgfpathrectangle{\pgfqpoint{0.550713in}{0.127635in}}{\pgfqpoint{3.194133in}{2.297424in}}%
\pgfusepath{clip}%
\pgfsetrectcap%
\pgfsetroundjoin%
\pgfsetlinewidth{0.752812pt}%
\definecolor{currentstroke}{rgb}{0.000000,0.000000,0.000000}%
\pgfsetstrokecolor{currentstroke}%
\pgfsetdash{}{0pt}%
\pgfpathmoveto{\pgfqpoint{1.069759in}{1.107928in}}%
\pgfpathlineto{\pgfqpoint{1.069759in}{1.128800in}}%
\pgfusepath{stroke}%
\end{pgfscope}%
\begin{pgfscope}%
\pgfpathrectangle{\pgfqpoint{0.550713in}{0.127635in}}{\pgfqpoint{3.194133in}{2.297424in}}%
\pgfusepath{clip}%
\pgfsetrectcap%
\pgfsetroundjoin%
\pgfsetlinewidth{0.752812pt}%
\definecolor{currentstroke}{rgb}{0.000000,0.000000,0.000000}%
\pgfsetstrokecolor{currentstroke}%
\pgfsetdash{}{0pt}%
\pgfpathmoveto{\pgfqpoint{1.030631in}{0.807448in}}%
\pgfpathlineto{\pgfqpoint{1.108887in}{0.807448in}}%
\pgfusepath{stroke}%
\end{pgfscope}%
\begin{pgfscope}%
\pgfpathrectangle{\pgfqpoint{0.550713in}{0.127635in}}{\pgfqpoint{3.194133in}{2.297424in}}%
\pgfusepath{clip}%
\pgfsetrectcap%
\pgfsetroundjoin%
\pgfsetlinewidth{0.752812pt}%
\definecolor{currentstroke}{rgb}{0.000000,0.000000,0.000000}%
\pgfsetstrokecolor{currentstroke}%
\pgfsetdash{}{0pt}%
\pgfpathmoveto{\pgfqpoint{1.030631in}{1.128800in}}%
\pgfpathlineto{\pgfqpoint{1.108887in}{1.128800in}}%
\pgfusepath{stroke}%
\end{pgfscope}%
\begin{pgfscope}%
\pgfpathrectangle{\pgfqpoint{0.550713in}{0.127635in}}{\pgfqpoint{3.194133in}{2.297424in}}%
\pgfusepath{clip}%
\pgfsetbuttcap%
\pgfsetmiterjoin%
\definecolor{currentfill}{rgb}{0.000000,0.000000,0.000000}%
\pgfsetfillcolor{currentfill}%
\pgfsetlinewidth{1.003750pt}%
\definecolor{currentstroke}{rgb}{0.000000,0.000000,0.000000}%
\pgfsetstrokecolor{currentstroke}%
\pgfsetdash{}{0pt}%
\pgfsys@defobject{currentmarker}{\pgfqpoint{-0.011785in}{-0.019642in}}{\pgfqpoint{0.011785in}{0.019642in}}{%
\pgfpathmoveto{\pgfqpoint{-0.000000in}{-0.019642in}}%
\pgfpathlineto{\pgfqpoint{0.011785in}{0.000000in}}%
\pgfpathlineto{\pgfqpoint{0.000000in}{0.019642in}}%
\pgfpathlineto{\pgfqpoint{-0.011785in}{0.000000in}}%
\pgfpathclose%
\pgfusepath{stroke,fill}%
}%
\begin{pgfscope}%
\pgfsys@transformshift{1.069759in}{2.131281in}%
\pgfsys@useobject{currentmarker}{}%
\end{pgfscope}%
\begin{pgfscope}%
\pgfsys@transformshift{1.069759in}{2.115583in}%
\pgfsys@useobject{currentmarker}{}%
\end{pgfscope}%
\begin{pgfscope}%
\pgfsys@transformshift{1.069759in}{2.051907in}%
\pgfsys@useobject{currentmarker}{}%
\end{pgfscope}%
\begin{pgfscope}%
\pgfsys@transformshift{1.069759in}{2.124396in}%
\pgfsys@useobject{currentmarker}{}%
\end{pgfscope}%
\begin{pgfscope}%
\pgfsys@transformshift{1.069759in}{2.150964in}%
\pgfsys@useobject{currentmarker}{}%
\end{pgfscope}%
\end{pgfscope}%
\begin{pgfscope}%
\pgfpathrectangle{\pgfqpoint{0.550713in}{0.127635in}}{\pgfqpoint{3.194133in}{2.297424in}}%
\pgfusepath{clip}%
\pgfsetrectcap%
\pgfsetroundjoin%
\pgfsetlinewidth{0.752812pt}%
\definecolor{currentstroke}{rgb}{0.000000,0.000000,0.000000}%
\pgfsetstrokecolor{currentstroke}%
\pgfsetdash{}{0pt}%
\pgfpathmoveto{\pgfqpoint{1.229466in}{1.868477in}}%
\pgfpathlineto{\pgfqpoint{1.229466in}{1.734854in}}%
\pgfusepath{stroke}%
\end{pgfscope}%
\begin{pgfscope}%
\pgfpathrectangle{\pgfqpoint{0.550713in}{0.127635in}}{\pgfqpoint{3.194133in}{2.297424in}}%
\pgfusepath{clip}%
\pgfsetrectcap%
\pgfsetroundjoin%
\pgfsetlinewidth{0.752812pt}%
\definecolor{currentstroke}{rgb}{0.000000,0.000000,0.000000}%
\pgfsetstrokecolor{currentstroke}%
\pgfsetdash{}{0pt}%
\pgfpathmoveto{\pgfqpoint{1.229466in}{2.399686in}}%
\pgfpathlineto{\pgfqpoint{1.229466in}{2.435059in}}%
\pgfusepath{stroke}%
\end{pgfscope}%
\begin{pgfscope}%
\pgfpathrectangle{\pgfqpoint{0.550713in}{0.127635in}}{\pgfqpoint{3.194133in}{2.297424in}}%
\pgfusepath{clip}%
\pgfsetrectcap%
\pgfsetroundjoin%
\pgfsetlinewidth{0.752812pt}%
\definecolor{currentstroke}{rgb}{0.000000,0.000000,0.000000}%
\pgfsetstrokecolor{currentstroke}%
\pgfsetdash{}{0pt}%
\pgfpathmoveto{\pgfqpoint{1.190338in}{1.734854in}}%
\pgfpathlineto{\pgfqpoint{1.268594in}{1.734854in}}%
\pgfusepath{stroke}%
\end{pgfscope}%
\begin{pgfscope}%
\pgfpathrectangle{\pgfqpoint{0.550713in}{0.127635in}}{\pgfqpoint{3.194133in}{2.297424in}}%
\pgfusepath{clip}%
\pgfsetrectcap%
\pgfsetroundjoin%
\pgfsetlinewidth{0.752812pt}%
\definecolor{currentstroke}{rgb}{0.000000,0.000000,0.000000}%
\pgfsetstrokecolor{currentstroke}%
\pgfsetdash{}{0pt}%
\pgfusepath{stroke}%
\end{pgfscope}%
\begin{pgfscope}%
\pgfpathrectangle{\pgfqpoint{0.550713in}{0.127635in}}{\pgfqpoint{3.194133in}{2.297424in}}%
\pgfusepath{clip}%
\pgfsetrectcap%
\pgfsetroundjoin%
\pgfsetlinewidth{0.752812pt}%
\definecolor{currentstroke}{rgb}{0.000000,0.000000,0.000000}%
\pgfsetstrokecolor{currentstroke}%
\pgfsetdash{}{0pt}%
\pgfpathmoveto{\pgfqpoint{1.469026in}{0.759835in}}%
\pgfpathlineto{\pgfqpoint{1.469026in}{0.604199in}}%
\pgfusepath{stroke}%
\end{pgfscope}%
\begin{pgfscope}%
\pgfpathrectangle{\pgfqpoint{0.550713in}{0.127635in}}{\pgfqpoint{3.194133in}{2.297424in}}%
\pgfusepath{clip}%
\pgfsetrectcap%
\pgfsetroundjoin%
\pgfsetlinewidth{0.752812pt}%
\definecolor{currentstroke}{rgb}{0.000000,0.000000,0.000000}%
\pgfsetstrokecolor{currentstroke}%
\pgfsetdash{}{0pt}%
\pgfpathmoveto{\pgfqpoint{1.469026in}{0.990198in}}%
\pgfpathlineto{\pgfqpoint{1.469026in}{1.275169in}}%
\pgfusepath{stroke}%
\end{pgfscope}%
\begin{pgfscope}%
\pgfpathrectangle{\pgfqpoint{0.550713in}{0.127635in}}{\pgfqpoint{3.194133in}{2.297424in}}%
\pgfusepath{clip}%
\pgfsetrectcap%
\pgfsetroundjoin%
\pgfsetlinewidth{0.752812pt}%
\definecolor{currentstroke}{rgb}{0.000000,0.000000,0.000000}%
\pgfsetstrokecolor{currentstroke}%
\pgfsetdash{}{0pt}%
\pgfpathmoveto{\pgfqpoint{1.429898in}{0.604199in}}%
\pgfpathlineto{\pgfqpoint{1.508154in}{0.604199in}}%
\pgfusepath{stroke}%
\end{pgfscope}%
\begin{pgfscope}%
\pgfpathrectangle{\pgfqpoint{0.550713in}{0.127635in}}{\pgfqpoint{3.194133in}{2.297424in}}%
\pgfusepath{clip}%
\pgfsetrectcap%
\pgfsetroundjoin%
\pgfsetlinewidth{0.752812pt}%
\definecolor{currentstroke}{rgb}{0.000000,0.000000,0.000000}%
\pgfsetstrokecolor{currentstroke}%
\pgfsetdash{}{0pt}%
\pgfpathmoveto{\pgfqpoint{1.429898in}{1.275169in}}%
\pgfpathlineto{\pgfqpoint{1.508154in}{1.275169in}}%
\pgfusepath{stroke}%
\end{pgfscope}%
\begin{pgfscope}%
\pgfpathrectangle{\pgfqpoint{0.550713in}{0.127635in}}{\pgfqpoint{3.194133in}{2.297424in}}%
\pgfusepath{clip}%
\pgfsetrectcap%
\pgfsetroundjoin%
\pgfsetlinewidth{0.752812pt}%
\definecolor{currentstroke}{rgb}{0.000000,0.000000,0.000000}%
\pgfsetstrokecolor{currentstroke}%
\pgfsetdash{}{0pt}%
\pgfpathmoveto{\pgfqpoint{1.628733in}{0.835710in}}%
\pgfpathlineto{\pgfqpoint{1.628733in}{0.667283in}}%
\pgfusepath{stroke}%
\end{pgfscope}%
\begin{pgfscope}%
\pgfpathrectangle{\pgfqpoint{0.550713in}{0.127635in}}{\pgfqpoint{3.194133in}{2.297424in}}%
\pgfusepath{clip}%
\pgfsetrectcap%
\pgfsetroundjoin%
\pgfsetlinewidth{0.752812pt}%
\definecolor{currentstroke}{rgb}{0.000000,0.000000,0.000000}%
\pgfsetstrokecolor{currentstroke}%
\pgfsetdash{}{0pt}%
\pgfpathmoveto{\pgfqpoint{1.628733in}{1.062230in}}%
\pgfpathlineto{\pgfqpoint{1.628733in}{1.133928in}}%
\pgfusepath{stroke}%
\end{pgfscope}%
\begin{pgfscope}%
\pgfpathrectangle{\pgfqpoint{0.550713in}{0.127635in}}{\pgfqpoint{3.194133in}{2.297424in}}%
\pgfusepath{clip}%
\pgfsetrectcap%
\pgfsetroundjoin%
\pgfsetlinewidth{0.752812pt}%
\definecolor{currentstroke}{rgb}{0.000000,0.000000,0.000000}%
\pgfsetstrokecolor{currentstroke}%
\pgfsetdash{}{0pt}%
\pgfpathmoveto{\pgfqpoint{1.589604in}{0.667283in}}%
\pgfpathlineto{\pgfqpoint{1.667861in}{0.667283in}}%
\pgfusepath{stroke}%
\end{pgfscope}%
\begin{pgfscope}%
\pgfpathrectangle{\pgfqpoint{0.550713in}{0.127635in}}{\pgfqpoint{3.194133in}{2.297424in}}%
\pgfusepath{clip}%
\pgfsetrectcap%
\pgfsetroundjoin%
\pgfsetlinewidth{0.752812pt}%
\definecolor{currentstroke}{rgb}{0.000000,0.000000,0.000000}%
\pgfsetstrokecolor{currentstroke}%
\pgfsetdash{}{0pt}%
\pgfpathmoveto{\pgfqpoint{1.589604in}{1.133928in}}%
\pgfpathlineto{\pgfqpoint{1.667861in}{1.133928in}}%
\pgfusepath{stroke}%
\end{pgfscope}%
\begin{pgfscope}%
\pgfpathrectangle{\pgfqpoint{0.550713in}{0.127635in}}{\pgfqpoint{3.194133in}{2.297424in}}%
\pgfusepath{clip}%
\pgfsetrectcap%
\pgfsetroundjoin%
\pgfsetlinewidth{0.752812pt}%
\definecolor{currentstroke}{rgb}{0.000000,0.000000,0.000000}%
\pgfsetstrokecolor{currentstroke}%
\pgfsetdash{}{0pt}%
\pgfpathmoveto{\pgfqpoint{1.868293in}{0.909735in}}%
\pgfpathlineto{\pgfqpoint{1.868293in}{0.746318in}}%
\pgfusepath{stroke}%
\end{pgfscope}%
\begin{pgfscope}%
\pgfpathrectangle{\pgfqpoint{0.550713in}{0.127635in}}{\pgfqpoint{3.194133in}{2.297424in}}%
\pgfusepath{clip}%
\pgfsetrectcap%
\pgfsetroundjoin%
\pgfsetlinewidth{0.752812pt}%
\definecolor{currentstroke}{rgb}{0.000000,0.000000,0.000000}%
\pgfsetstrokecolor{currentstroke}%
\pgfsetdash{}{0pt}%
\pgfpathmoveto{\pgfqpoint{1.868293in}{1.052177in}}%
\pgfpathlineto{\pgfqpoint{1.868293in}{1.142278in}}%
\pgfusepath{stroke}%
\end{pgfscope}%
\begin{pgfscope}%
\pgfpathrectangle{\pgfqpoint{0.550713in}{0.127635in}}{\pgfqpoint{3.194133in}{2.297424in}}%
\pgfusepath{clip}%
\pgfsetrectcap%
\pgfsetroundjoin%
\pgfsetlinewidth{0.752812pt}%
\definecolor{currentstroke}{rgb}{0.000000,0.000000,0.000000}%
\pgfsetstrokecolor{currentstroke}%
\pgfsetdash{}{0pt}%
\pgfpathmoveto{\pgfqpoint{1.829164in}{0.746318in}}%
\pgfpathlineto{\pgfqpoint{1.907421in}{0.746318in}}%
\pgfusepath{stroke}%
\end{pgfscope}%
\begin{pgfscope}%
\pgfpathrectangle{\pgfqpoint{0.550713in}{0.127635in}}{\pgfqpoint{3.194133in}{2.297424in}}%
\pgfusepath{clip}%
\pgfsetrectcap%
\pgfsetroundjoin%
\pgfsetlinewidth{0.752812pt}%
\definecolor{currentstroke}{rgb}{0.000000,0.000000,0.000000}%
\pgfsetstrokecolor{currentstroke}%
\pgfsetdash{}{0pt}%
\pgfpathmoveto{\pgfqpoint{1.829164in}{1.142278in}}%
\pgfpathlineto{\pgfqpoint{1.907421in}{1.142278in}}%
\pgfusepath{stroke}%
\end{pgfscope}%
\begin{pgfscope}%
\pgfpathrectangle{\pgfqpoint{0.550713in}{0.127635in}}{\pgfqpoint{3.194133in}{2.297424in}}%
\pgfusepath{clip}%
\pgfsetbuttcap%
\pgfsetmiterjoin%
\definecolor{currentfill}{rgb}{0.000000,0.000000,0.000000}%
\pgfsetfillcolor{currentfill}%
\pgfsetlinewidth{1.003750pt}%
\definecolor{currentstroke}{rgb}{0.000000,0.000000,0.000000}%
\pgfsetstrokecolor{currentstroke}%
\pgfsetdash{}{0pt}%
\pgfsys@defobject{currentmarker}{\pgfqpoint{-0.011785in}{-0.019642in}}{\pgfqpoint{0.011785in}{0.019642in}}{%
\pgfpathmoveto{\pgfqpoint{-0.000000in}{-0.019642in}}%
\pgfpathlineto{\pgfqpoint{0.011785in}{0.000000in}}%
\pgfpathlineto{\pgfqpoint{0.000000in}{0.019642in}}%
\pgfpathlineto{\pgfqpoint{-0.011785in}{0.000000in}}%
\pgfpathclose%
\pgfusepath{stroke,fill}%
}%
\begin{pgfscope}%
\pgfsys@transformshift{1.868293in}{1.268659in}%
\pgfsys@useobject{currentmarker}{}%
\end{pgfscope}%
\begin{pgfscope}%
\pgfsys@transformshift{1.868293in}{1.419077in}%
\pgfsys@useobject{currentmarker}{}%
\end{pgfscope}%
\begin{pgfscope}%
\pgfsys@transformshift{1.868293in}{1.371069in}%
\pgfsys@useobject{currentmarker}{}%
\end{pgfscope}%
\end{pgfscope}%
\begin{pgfscope}%
\pgfpathrectangle{\pgfqpoint{0.550713in}{0.127635in}}{\pgfqpoint{3.194133in}{2.297424in}}%
\pgfusepath{clip}%
\pgfsetrectcap%
\pgfsetroundjoin%
\pgfsetlinewidth{0.752812pt}%
\definecolor{currentstroke}{rgb}{0.000000,0.000000,0.000000}%
\pgfsetstrokecolor{currentstroke}%
\pgfsetdash{}{0pt}%
\pgfpathmoveto{\pgfqpoint{2.027999in}{0.939635in}}%
\pgfpathlineto{\pgfqpoint{2.027999in}{0.755181in}}%
\pgfusepath{stroke}%
\end{pgfscope}%
\begin{pgfscope}%
\pgfpathrectangle{\pgfqpoint{0.550713in}{0.127635in}}{\pgfqpoint{3.194133in}{2.297424in}}%
\pgfusepath{clip}%
\pgfsetrectcap%
\pgfsetroundjoin%
\pgfsetlinewidth{0.752812pt}%
\definecolor{currentstroke}{rgb}{0.000000,0.000000,0.000000}%
\pgfsetstrokecolor{currentstroke}%
\pgfsetdash{}{0pt}%
\pgfpathmoveto{\pgfqpoint{2.027999in}{1.168116in}}%
\pgfpathlineto{\pgfqpoint{2.027999in}{1.369107in}}%
\pgfusepath{stroke}%
\end{pgfscope}%
\begin{pgfscope}%
\pgfpathrectangle{\pgfqpoint{0.550713in}{0.127635in}}{\pgfqpoint{3.194133in}{2.297424in}}%
\pgfusepath{clip}%
\pgfsetrectcap%
\pgfsetroundjoin%
\pgfsetlinewidth{0.752812pt}%
\definecolor{currentstroke}{rgb}{0.000000,0.000000,0.000000}%
\pgfsetstrokecolor{currentstroke}%
\pgfsetdash{}{0pt}%
\pgfpathmoveto{\pgfqpoint{1.988871in}{0.755181in}}%
\pgfpathlineto{\pgfqpoint{2.067127in}{0.755181in}}%
\pgfusepath{stroke}%
\end{pgfscope}%
\begin{pgfscope}%
\pgfpathrectangle{\pgfqpoint{0.550713in}{0.127635in}}{\pgfqpoint{3.194133in}{2.297424in}}%
\pgfusepath{clip}%
\pgfsetrectcap%
\pgfsetroundjoin%
\pgfsetlinewidth{0.752812pt}%
\definecolor{currentstroke}{rgb}{0.000000,0.000000,0.000000}%
\pgfsetstrokecolor{currentstroke}%
\pgfsetdash{}{0pt}%
\pgfpathmoveto{\pgfqpoint{1.988871in}{1.369107in}}%
\pgfpathlineto{\pgfqpoint{2.067127in}{1.369107in}}%
\pgfusepath{stroke}%
\end{pgfscope}%
\begin{pgfscope}%
\pgfpathrectangle{\pgfqpoint{0.550713in}{0.127635in}}{\pgfqpoint{3.194133in}{2.297424in}}%
\pgfusepath{clip}%
\pgfsetrectcap%
\pgfsetroundjoin%
\pgfsetlinewidth{0.752812pt}%
\definecolor{currentstroke}{rgb}{0.000000,0.000000,0.000000}%
\pgfsetstrokecolor{currentstroke}%
\pgfsetdash{}{0pt}%
\pgfpathmoveto{\pgfqpoint{2.267559in}{0.630518in}}%
\pgfpathlineto{\pgfqpoint{2.267559in}{0.580012in}}%
\pgfusepath{stroke}%
\end{pgfscope}%
\begin{pgfscope}%
\pgfpathrectangle{\pgfqpoint{0.550713in}{0.127635in}}{\pgfqpoint{3.194133in}{2.297424in}}%
\pgfusepath{clip}%
\pgfsetrectcap%
\pgfsetroundjoin%
\pgfsetlinewidth{0.752812pt}%
\definecolor{currentstroke}{rgb}{0.000000,0.000000,0.000000}%
\pgfsetstrokecolor{currentstroke}%
\pgfsetdash{}{0pt}%
\pgfpathmoveto{\pgfqpoint{2.267559in}{0.689453in}}%
\pgfpathlineto{\pgfqpoint{2.267559in}{0.720036in}}%
\pgfusepath{stroke}%
\end{pgfscope}%
\begin{pgfscope}%
\pgfpathrectangle{\pgfqpoint{0.550713in}{0.127635in}}{\pgfqpoint{3.194133in}{2.297424in}}%
\pgfusepath{clip}%
\pgfsetrectcap%
\pgfsetroundjoin%
\pgfsetlinewidth{0.752812pt}%
\definecolor{currentstroke}{rgb}{0.000000,0.000000,0.000000}%
\pgfsetstrokecolor{currentstroke}%
\pgfsetdash{}{0pt}%
\pgfpathmoveto{\pgfqpoint{2.228431in}{0.580012in}}%
\pgfpathlineto{\pgfqpoint{2.306687in}{0.580012in}}%
\pgfusepath{stroke}%
\end{pgfscope}%
\begin{pgfscope}%
\pgfpathrectangle{\pgfqpoint{0.550713in}{0.127635in}}{\pgfqpoint{3.194133in}{2.297424in}}%
\pgfusepath{clip}%
\pgfsetrectcap%
\pgfsetroundjoin%
\pgfsetlinewidth{0.752812pt}%
\definecolor{currentstroke}{rgb}{0.000000,0.000000,0.000000}%
\pgfsetstrokecolor{currentstroke}%
\pgfsetdash{}{0pt}%
\pgfpathmoveto{\pgfqpoint{2.228431in}{0.720036in}}%
\pgfpathlineto{\pgfqpoint{2.306687in}{0.720036in}}%
\pgfusepath{stroke}%
\end{pgfscope}%
\begin{pgfscope}%
\pgfpathrectangle{\pgfqpoint{0.550713in}{0.127635in}}{\pgfqpoint{3.194133in}{2.297424in}}%
\pgfusepath{clip}%
\pgfsetrectcap%
\pgfsetroundjoin%
\pgfsetlinewidth{0.752812pt}%
\definecolor{currentstroke}{rgb}{0.000000,0.000000,0.000000}%
\pgfsetstrokecolor{currentstroke}%
\pgfsetdash{}{0pt}%
\pgfpathmoveto{\pgfqpoint{2.427266in}{0.709846in}}%
\pgfpathlineto{\pgfqpoint{2.427266in}{0.660438in}}%
\pgfusepath{stroke}%
\end{pgfscope}%
\begin{pgfscope}%
\pgfpathrectangle{\pgfqpoint{0.550713in}{0.127635in}}{\pgfqpoint{3.194133in}{2.297424in}}%
\pgfusepath{clip}%
\pgfsetrectcap%
\pgfsetroundjoin%
\pgfsetlinewidth{0.752812pt}%
\definecolor{currentstroke}{rgb}{0.000000,0.000000,0.000000}%
\pgfsetstrokecolor{currentstroke}%
\pgfsetdash{}{0pt}%
\pgfpathmoveto{\pgfqpoint{2.427266in}{0.746785in}}%
\pgfpathlineto{\pgfqpoint{2.427266in}{0.801397in}}%
\pgfusepath{stroke}%
\end{pgfscope}%
\begin{pgfscope}%
\pgfpathrectangle{\pgfqpoint{0.550713in}{0.127635in}}{\pgfqpoint{3.194133in}{2.297424in}}%
\pgfusepath{clip}%
\pgfsetrectcap%
\pgfsetroundjoin%
\pgfsetlinewidth{0.752812pt}%
\definecolor{currentstroke}{rgb}{0.000000,0.000000,0.000000}%
\pgfsetstrokecolor{currentstroke}%
\pgfsetdash{}{0pt}%
\pgfpathmoveto{\pgfqpoint{2.388138in}{0.660438in}}%
\pgfpathlineto{\pgfqpoint{2.466394in}{0.660438in}}%
\pgfusepath{stroke}%
\end{pgfscope}%
\begin{pgfscope}%
\pgfpathrectangle{\pgfqpoint{0.550713in}{0.127635in}}{\pgfqpoint{3.194133in}{2.297424in}}%
\pgfusepath{clip}%
\pgfsetrectcap%
\pgfsetroundjoin%
\pgfsetlinewidth{0.752812pt}%
\definecolor{currentstroke}{rgb}{0.000000,0.000000,0.000000}%
\pgfsetstrokecolor{currentstroke}%
\pgfsetdash{}{0pt}%
\pgfpathmoveto{\pgfqpoint{2.388138in}{0.801397in}}%
\pgfpathlineto{\pgfqpoint{2.466394in}{0.801397in}}%
\pgfusepath{stroke}%
\end{pgfscope}%
\begin{pgfscope}%
\pgfpathrectangle{\pgfqpoint{0.550713in}{0.127635in}}{\pgfqpoint{3.194133in}{2.297424in}}%
\pgfusepath{clip}%
\pgfsetrectcap%
\pgfsetroundjoin%
\pgfsetlinewidth{0.752812pt}%
\definecolor{currentstroke}{rgb}{0.000000,0.000000,0.000000}%
\pgfsetstrokecolor{currentstroke}%
\pgfsetdash{}{0pt}%
\pgfpathmoveto{\pgfqpoint{2.666826in}{0.646303in}}%
\pgfpathlineto{\pgfqpoint{2.666826in}{0.601030in}}%
\pgfusepath{stroke}%
\end{pgfscope}%
\begin{pgfscope}%
\pgfpathrectangle{\pgfqpoint{0.550713in}{0.127635in}}{\pgfqpoint{3.194133in}{2.297424in}}%
\pgfusepath{clip}%
\pgfsetrectcap%
\pgfsetroundjoin%
\pgfsetlinewidth{0.752812pt}%
\definecolor{currentstroke}{rgb}{0.000000,0.000000,0.000000}%
\pgfsetstrokecolor{currentstroke}%
\pgfsetdash{}{0pt}%
\pgfpathmoveto{\pgfqpoint{2.666826in}{0.680680in}}%
\pgfpathlineto{\pgfqpoint{2.666826in}{0.729195in}}%
\pgfusepath{stroke}%
\end{pgfscope}%
\begin{pgfscope}%
\pgfpathrectangle{\pgfqpoint{0.550713in}{0.127635in}}{\pgfqpoint{3.194133in}{2.297424in}}%
\pgfusepath{clip}%
\pgfsetrectcap%
\pgfsetroundjoin%
\pgfsetlinewidth{0.752812pt}%
\definecolor{currentstroke}{rgb}{0.000000,0.000000,0.000000}%
\pgfsetstrokecolor{currentstroke}%
\pgfsetdash{}{0pt}%
\pgfpathmoveto{\pgfqpoint{2.627698in}{0.601030in}}%
\pgfpathlineto{\pgfqpoint{2.705954in}{0.601030in}}%
\pgfusepath{stroke}%
\end{pgfscope}%
\begin{pgfscope}%
\pgfpathrectangle{\pgfqpoint{0.550713in}{0.127635in}}{\pgfqpoint{3.194133in}{2.297424in}}%
\pgfusepath{clip}%
\pgfsetrectcap%
\pgfsetroundjoin%
\pgfsetlinewidth{0.752812pt}%
\definecolor{currentstroke}{rgb}{0.000000,0.000000,0.000000}%
\pgfsetstrokecolor{currentstroke}%
\pgfsetdash{}{0pt}%
\pgfpathmoveto{\pgfqpoint{2.627698in}{0.729195in}}%
\pgfpathlineto{\pgfqpoint{2.705954in}{0.729195in}}%
\pgfusepath{stroke}%
\end{pgfscope}%
\begin{pgfscope}%
\pgfpathrectangle{\pgfqpoint{0.550713in}{0.127635in}}{\pgfqpoint{3.194133in}{2.297424in}}%
\pgfusepath{clip}%
\pgfsetbuttcap%
\pgfsetmiterjoin%
\definecolor{currentfill}{rgb}{0.000000,0.000000,0.000000}%
\pgfsetfillcolor{currentfill}%
\pgfsetlinewidth{1.003750pt}%
\definecolor{currentstroke}{rgb}{0.000000,0.000000,0.000000}%
\pgfsetstrokecolor{currentstroke}%
\pgfsetdash{}{0pt}%
\pgfsys@defobject{currentmarker}{\pgfqpoint{-0.011785in}{-0.019642in}}{\pgfqpoint{0.011785in}{0.019642in}}{%
\pgfpathmoveto{\pgfqpoint{-0.000000in}{-0.019642in}}%
\pgfpathlineto{\pgfqpoint{0.011785in}{0.000000in}}%
\pgfpathlineto{\pgfqpoint{0.000000in}{0.019642in}}%
\pgfpathlineto{\pgfqpoint{-0.011785in}{0.000000in}}%
\pgfpathclose%
\pgfusepath{stroke,fill}%
}%
\begin{pgfscope}%
\pgfsys@transformshift{2.666826in}{0.590426in}%
\pgfsys@useobject{currentmarker}{}%
\end{pgfscope}%
\begin{pgfscope}%
\pgfsys@transformshift{2.666826in}{0.573387in}%
\pgfsys@useobject{currentmarker}{}%
\end{pgfscope}%
\end{pgfscope}%
\begin{pgfscope}%
\pgfpathrectangle{\pgfqpoint{0.550713in}{0.127635in}}{\pgfqpoint{3.194133in}{2.297424in}}%
\pgfusepath{clip}%
\pgfsetrectcap%
\pgfsetroundjoin%
\pgfsetlinewidth{0.752812pt}%
\definecolor{currentstroke}{rgb}{0.000000,0.000000,0.000000}%
\pgfsetstrokecolor{currentstroke}%
\pgfsetdash{}{0pt}%
\pgfpathmoveto{\pgfqpoint{2.826532in}{0.714258in}}%
\pgfpathlineto{\pgfqpoint{2.826532in}{0.666177in}}%
\pgfusepath{stroke}%
\end{pgfscope}%
\begin{pgfscope}%
\pgfpathrectangle{\pgfqpoint{0.550713in}{0.127635in}}{\pgfqpoint{3.194133in}{2.297424in}}%
\pgfusepath{clip}%
\pgfsetrectcap%
\pgfsetroundjoin%
\pgfsetlinewidth{0.752812pt}%
\definecolor{currentstroke}{rgb}{0.000000,0.000000,0.000000}%
\pgfsetstrokecolor{currentstroke}%
\pgfsetdash{}{0pt}%
\pgfpathmoveto{\pgfqpoint{2.826532in}{0.765359in}}%
\pgfpathlineto{\pgfqpoint{2.826532in}{0.808570in}}%
\pgfusepath{stroke}%
\end{pgfscope}%
\begin{pgfscope}%
\pgfpathrectangle{\pgfqpoint{0.550713in}{0.127635in}}{\pgfqpoint{3.194133in}{2.297424in}}%
\pgfusepath{clip}%
\pgfsetrectcap%
\pgfsetroundjoin%
\pgfsetlinewidth{0.752812pt}%
\definecolor{currentstroke}{rgb}{0.000000,0.000000,0.000000}%
\pgfsetstrokecolor{currentstroke}%
\pgfsetdash{}{0pt}%
\pgfpathmoveto{\pgfqpoint{2.787404in}{0.666177in}}%
\pgfpathlineto{\pgfqpoint{2.865661in}{0.666177in}}%
\pgfusepath{stroke}%
\end{pgfscope}%
\begin{pgfscope}%
\pgfpathrectangle{\pgfqpoint{0.550713in}{0.127635in}}{\pgfqpoint{3.194133in}{2.297424in}}%
\pgfusepath{clip}%
\pgfsetrectcap%
\pgfsetroundjoin%
\pgfsetlinewidth{0.752812pt}%
\definecolor{currentstroke}{rgb}{0.000000,0.000000,0.000000}%
\pgfsetstrokecolor{currentstroke}%
\pgfsetdash{}{0pt}%
\pgfpathmoveto{\pgfqpoint{2.787404in}{0.808570in}}%
\pgfpathlineto{\pgfqpoint{2.865661in}{0.808570in}}%
\pgfusepath{stroke}%
\end{pgfscope}%
\begin{pgfscope}%
\pgfpathrectangle{\pgfqpoint{0.550713in}{0.127635in}}{\pgfqpoint{3.194133in}{2.297424in}}%
\pgfusepath{clip}%
\pgfsetrectcap%
\pgfsetroundjoin%
\pgfsetlinewidth{0.752812pt}%
\definecolor{currentstroke}{rgb}{0.000000,0.000000,0.000000}%
\pgfsetstrokecolor{currentstroke}%
\pgfsetdash{}{0pt}%
\pgfpathmoveto{\pgfqpoint{3.066092in}{0.592972in}}%
\pgfpathlineto{\pgfqpoint{3.066092in}{0.472187in}}%
\pgfusepath{stroke}%
\end{pgfscope}%
\begin{pgfscope}%
\pgfpathrectangle{\pgfqpoint{0.550713in}{0.127635in}}{\pgfqpoint{3.194133in}{2.297424in}}%
\pgfusepath{clip}%
\pgfsetrectcap%
\pgfsetroundjoin%
\pgfsetlinewidth{0.752812pt}%
\definecolor{currentstroke}{rgb}{0.000000,0.000000,0.000000}%
\pgfsetstrokecolor{currentstroke}%
\pgfsetdash{}{0pt}%
\pgfpathmoveto{\pgfqpoint{3.066092in}{0.694924in}}%
\pgfpathlineto{\pgfqpoint{3.066092in}{0.732462in}}%
\pgfusepath{stroke}%
\end{pgfscope}%
\begin{pgfscope}%
\pgfpathrectangle{\pgfqpoint{0.550713in}{0.127635in}}{\pgfqpoint{3.194133in}{2.297424in}}%
\pgfusepath{clip}%
\pgfsetrectcap%
\pgfsetroundjoin%
\pgfsetlinewidth{0.752812pt}%
\definecolor{currentstroke}{rgb}{0.000000,0.000000,0.000000}%
\pgfsetstrokecolor{currentstroke}%
\pgfsetdash{}{0pt}%
\pgfpathmoveto{\pgfqpoint{3.026964in}{0.472187in}}%
\pgfpathlineto{\pgfqpoint{3.105221in}{0.472187in}}%
\pgfusepath{stroke}%
\end{pgfscope}%
\begin{pgfscope}%
\pgfpathrectangle{\pgfqpoint{0.550713in}{0.127635in}}{\pgfqpoint{3.194133in}{2.297424in}}%
\pgfusepath{clip}%
\pgfsetrectcap%
\pgfsetroundjoin%
\pgfsetlinewidth{0.752812pt}%
\definecolor{currentstroke}{rgb}{0.000000,0.000000,0.000000}%
\pgfsetstrokecolor{currentstroke}%
\pgfsetdash{}{0pt}%
\pgfpathmoveto{\pgfqpoint{3.026964in}{0.732462in}}%
\pgfpathlineto{\pgfqpoint{3.105221in}{0.732462in}}%
\pgfusepath{stroke}%
\end{pgfscope}%
\begin{pgfscope}%
\pgfpathrectangle{\pgfqpoint{0.550713in}{0.127635in}}{\pgfqpoint{3.194133in}{2.297424in}}%
\pgfusepath{clip}%
\pgfsetrectcap%
\pgfsetroundjoin%
\pgfsetlinewidth{0.752812pt}%
\definecolor{currentstroke}{rgb}{0.000000,0.000000,0.000000}%
\pgfsetstrokecolor{currentstroke}%
\pgfsetdash{}{0pt}%
\pgfpathmoveto{\pgfqpoint{3.225799in}{0.603692in}}%
\pgfpathlineto{\pgfqpoint{3.225799in}{0.538115in}}%
\pgfusepath{stroke}%
\end{pgfscope}%
\begin{pgfscope}%
\pgfpathrectangle{\pgfqpoint{0.550713in}{0.127635in}}{\pgfqpoint{3.194133in}{2.297424in}}%
\pgfusepath{clip}%
\pgfsetrectcap%
\pgfsetroundjoin%
\pgfsetlinewidth{0.752812pt}%
\definecolor{currentstroke}{rgb}{0.000000,0.000000,0.000000}%
\pgfsetstrokecolor{currentstroke}%
\pgfsetdash{}{0pt}%
\pgfpathmoveto{\pgfqpoint{3.225799in}{0.680073in}}%
\pgfpathlineto{\pgfqpoint{3.225799in}{0.755436in}}%
\pgfusepath{stroke}%
\end{pgfscope}%
\begin{pgfscope}%
\pgfpathrectangle{\pgfqpoint{0.550713in}{0.127635in}}{\pgfqpoint{3.194133in}{2.297424in}}%
\pgfusepath{clip}%
\pgfsetrectcap%
\pgfsetroundjoin%
\pgfsetlinewidth{0.752812pt}%
\definecolor{currentstroke}{rgb}{0.000000,0.000000,0.000000}%
\pgfsetstrokecolor{currentstroke}%
\pgfsetdash{}{0pt}%
\pgfpathmoveto{\pgfqpoint{3.186671in}{0.538115in}}%
\pgfpathlineto{\pgfqpoint{3.264927in}{0.538115in}}%
\pgfusepath{stroke}%
\end{pgfscope}%
\begin{pgfscope}%
\pgfpathrectangle{\pgfqpoint{0.550713in}{0.127635in}}{\pgfqpoint{3.194133in}{2.297424in}}%
\pgfusepath{clip}%
\pgfsetrectcap%
\pgfsetroundjoin%
\pgfsetlinewidth{0.752812pt}%
\definecolor{currentstroke}{rgb}{0.000000,0.000000,0.000000}%
\pgfsetstrokecolor{currentstroke}%
\pgfsetdash{}{0pt}%
\pgfpathmoveto{\pgfqpoint{3.186671in}{0.755436in}}%
\pgfpathlineto{\pgfqpoint{3.264927in}{0.755436in}}%
\pgfusepath{stroke}%
\end{pgfscope}%
\begin{pgfscope}%
\pgfpathrectangle{\pgfqpoint{0.550713in}{0.127635in}}{\pgfqpoint{3.194133in}{2.297424in}}%
\pgfusepath{clip}%
\pgfsetrectcap%
\pgfsetroundjoin%
\pgfsetlinewidth{0.752812pt}%
\definecolor{currentstroke}{rgb}{0.000000,0.000000,0.000000}%
\pgfsetstrokecolor{currentstroke}%
\pgfsetdash{}{0pt}%
\pgfpathmoveto{\pgfqpoint{3.465359in}{0.619842in}}%
\pgfpathlineto{\pgfqpoint{3.465359in}{0.532649in}}%
\pgfusepath{stroke}%
\end{pgfscope}%
\begin{pgfscope}%
\pgfpathrectangle{\pgfqpoint{0.550713in}{0.127635in}}{\pgfqpoint{3.194133in}{2.297424in}}%
\pgfusepath{clip}%
\pgfsetrectcap%
\pgfsetroundjoin%
\pgfsetlinewidth{0.752812pt}%
\definecolor{currentstroke}{rgb}{0.000000,0.000000,0.000000}%
\pgfsetstrokecolor{currentstroke}%
\pgfsetdash{}{0pt}%
\pgfpathmoveto{\pgfqpoint{3.465359in}{0.699327in}}%
\pgfpathlineto{\pgfqpoint{3.465359in}{0.739099in}}%
\pgfusepath{stroke}%
\end{pgfscope}%
\begin{pgfscope}%
\pgfpathrectangle{\pgfqpoint{0.550713in}{0.127635in}}{\pgfqpoint{3.194133in}{2.297424in}}%
\pgfusepath{clip}%
\pgfsetrectcap%
\pgfsetroundjoin%
\pgfsetlinewidth{0.752812pt}%
\definecolor{currentstroke}{rgb}{0.000000,0.000000,0.000000}%
\pgfsetstrokecolor{currentstroke}%
\pgfsetdash{}{0pt}%
\pgfpathmoveto{\pgfqpoint{3.426231in}{0.532649in}}%
\pgfpathlineto{\pgfqpoint{3.504487in}{0.532649in}}%
\pgfusepath{stroke}%
\end{pgfscope}%
\begin{pgfscope}%
\pgfpathrectangle{\pgfqpoint{0.550713in}{0.127635in}}{\pgfqpoint{3.194133in}{2.297424in}}%
\pgfusepath{clip}%
\pgfsetrectcap%
\pgfsetroundjoin%
\pgfsetlinewidth{0.752812pt}%
\definecolor{currentstroke}{rgb}{0.000000,0.000000,0.000000}%
\pgfsetstrokecolor{currentstroke}%
\pgfsetdash{}{0pt}%
\pgfpathmoveto{\pgfqpoint{3.426231in}{0.739099in}}%
\pgfpathlineto{\pgfqpoint{3.504487in}{0.739099in}}%
\pgfusepath{stroke}%
\end{pgfscope}%
\begin{pgfscope}%
\pgfpathrectangle{\pgfqpoint{0.550713in}{0.127635in}}{\pgfqpoint{3.194133in}{2.297424in}}%
\pgfusepath{clip}%
\pgfsetrectcap%
\pgfsetroundjoin%
\pgfsetlinewidth{0.752812pt}%
\definecolor{currentstroke}{rgb}{0.000000,0.000000,0.000000}%
\pgfsetstrokecolor{currentstroke}%
\pgfsetdash{}{0pt}%
\pgfpathmoveto{\pgfqpoint{3.625066in}{0.629160in}}%
\pgfpathlineto{\pgfqpoint{3.625066in}{0.552887in}}%
\pgfusepath{stroke}%
\end{pgfscope}%
\begin{pgfscope}%
\pgfpathrectangle{\pgfqpoint{0.550713in}{0.127635in}}{\pgfqpoint{3.194133in}{2.297424in}}%
\pgfusepath{clip}%
\pgfsetrectcap%
\pgfsetroundjoin%
\pgfsetlinewidth{0.752812pt}%
\definecolor{currentstroke}{rgb}{0.000000,0.000000,0.000000}%
\pgfsetstrokecolor{currentstroke}%
\pgfsetdash{}{0pt}%
\pgfpathmoveto{\pgfqpoint{3.625066in}{0.687534in}}%
\pgfpathlineto{\pgfqpoint{3.625066in}{0.744806in}}%
\pgfusepath{stroke}%
\end{pgfscope}%
\begin{pgfscope}%
\pgfpathrectangle{\pgfqpoint{0.550713in}{0.127635in}}{\pgfqpoint{3.194133in}{2.297424in}}%
\pgfusepath{clip}%
\pgfsetrectcap%
\pgfsetroundjoin%
\pgfsetlinewidth{0.752812pt}%
\definecolor{currentstroke}{rgb}{0.000000,0.000000,0.000000}%
\pgfsetstrokecolor{currentstroke}%
\pgfsetdash{}{0pt}%
\pgfpathmoveto{\pgfqpoint{3.585938in}{0.552887in}}%
\pgfpathlineto{\pgfqpoint{3.664194in}{0.552887in}}%
\pgfusepath{stroke}%
\end{pgfscope}%
\begin{pgfscope}%
\pgfpathrectangle{\pgfqpoint{0.550713in}{0.127635in}}{\pgfqpoint{3.194133in}{2.297424in}}%
\pgfusepath{clip}%
\pgfsetrectcap%
\pgfsetroundjoin%
\pgfsetlinewidth{0.752812pt}%
\definecolor{currentstroke}{rgb}{0.000000,0.000000,0.000000}%
\pgfsetstrokecolor{currentstroke}%
\pgfsetdash{}{0pt}%
\pgfpathmoveto{\pgfqpoint{3.585938in}{0.744806in}}%
\pgfpathlineto{\pgfqpoint{3.664194in}{0.744806in}}%
\pgfusepath{stroke}%
\end{pgfscope}%
\begin{pgfscope}%
\pgfpathrectangle{\pgfqpoint{0.550713in}{0.127635in}}{\pgfqpoint{3.194133in}{2.297424in}}%
\pgfusepath{clip}%
\pgfsetbuttcap%
\pgfsetmiterjoin%
\definecolor{currentfill}{rgb}{0.000000,0.000000,0.000000}%
\pgfsetfillcolor{currentfill}%
\pgfsetlinewidth{1.003750pt}%
\definecolor{currentstroke}{rgb}{0.000000,0.000000,0.000000}%
\pgfsetstrokecolor{currentstroke}%
\pgfsetdash{}{0pt}%
\pgfsys@defobject{currentmarker}{\pgfqpoint{-0.011785in}{-0.019642in}}{\pgfqpoint{0.011785in}{0.019642in}}{%
\pgfpathmoveto{\pgfqpoint{-0.000000in}{-0.019642in}}%
\pgfpathlineto{\pgfqpoint{0.011785in}{0.000000in}}%
\pgfpathlineto{\pgfqpoint{0.000000in}{0.019642in}}%
\pgfpathlineto{\pgfqpoint{-0.011785in}{0.000000in}}%
\pgfpathclose%
\pgfusepath{stroke,fill}%
}%
\begin{pgfscope}%
\pgfsys@transformshift{3.625066in}{0.791540in}%
\pgfsys@useobject{currentmarker}{}%
\end{pgfscope}%
\end{pgfscope}%
\begin{pgfscope}%
\pgfpathrectangle{\pgfqpoint{0.550713in}{0.127635in}}{\pgfqpoint{3.194133in}{2.297424in}}%
\pgfusepath{clip}%
\pgfsetrectcap%
\pgfsetroundjoin%
\pgfsetlinewidth{0.752812pt}%
\definecolor{currentstroke}{rgb}{0.000000,0.000000,0.000000}%
\pgfsetstrokecolor{currentstroke}%
\pgfsetdash{}{0pt}%
\pgfpathmoveto{\pgfqpoint{0.592236in}{0.958511in}}%
\pgfpathlineto{\pgfqpoint{0.748749in}{0.958511in}}%
\pgfusepath{stroke}%
\end{pgfscope}%
\begin{pgfscope}%
\pgfpathrectangle{\pgfqpoint{0.550713in}{0.127635in}}{\pgfqpoint{3.194133in}{2.297424in}}%
\pgfusepath{clip}%
\pgfsetbuttcap%
\pgfsetroundjoin%
\definecolor{currentfill}{rgb}{1.000000,1.000000,1.000000}%
\pgfsetfillcolor{currentfill}%
\pgfsetlinewidth{1.003750pt}%
\definecolor{currentstroke}{rgb}{0.000000,0.000000,0.000000}%
\pgfsetstrokecolor{currentstroke}%
\pgfsetdash{}{0pt}%
\pgfsys@defobject{currentmarker}{\pgfqpoint{-0.027778in}{-0.027778in}}{\pgfqpoint{0.027778in}{0.027778in}}{%
\pgfpathmoveto{\pgfqpoint{0.000000in}{-0.027778in}}%
\pgfpathcurveto{\pgfqpoint{0.007367in}{-0.027778in}}{\pgfqpoint{0.014433in}{-0.024851in}}{\pgfqpoint{0.019642in}{-0.019642in}}%
\pgfpathcurveto{\pgfqpoint{0.024851in}{-0.014433in}}{\pgfqpoint{0.027778in}{-0.007367in}}{\pgfqpoint{0.027778in}{0.000000in}}%
\pgfpathcurveto{\pgfqpoint{0.027778in}{0.007367in}}{\pgfqpoint{0.024851in}{0.014433in}}{\pgfqpoint{0.019642in}{0.019642in}}%
\pgfpathcurveto{\pgfqpoint{0.014433in}{0.024851in}}{\pgfqpoint{0.007367in}{0.027778in}}{\pgfqpoint{0.000000in}{0.027778in}}%
\pgfpathcurveto{\pgfqpoint{-0.007367in}{0.027778in}}{\pgfqpoint{-0.014433in}{0.024851in}}{\pgfqpoint{-0.019642in}{0.019642in}}%
\pgfpathcurveto{\pgfqpoint{-0.024851in}{0.014433in}}{\pgfqpoint{-0.027778in}{0.007367in}}{\pgfqpoint{-0.027778in}{0.000000in}}%
\pgfpathcurveto{\pgfqpoint{-0.027778in}{-0.007367in}}{\pgfqpoint{-0.024851in}{-0.014433in}}{\pgfqpoint{-0.019642in}{-0.019642in}}%
\pgfpathcurveto{\pgfqpoint{-0.014433in}{-0.024851in}}{\pgfqpoint{-0.007367in}{-0.027778in}}{\pgfqpoint{0.000000in}{-0.027778in}}%
\pgfpathclose%
\pgfusepath{stroke,fill}%
}%
\begin{pgfscope}%
\pgfsys@transformshift{0.670493in}{1.069306in}%
\pgfsys@useobject{currentmarker}{}%
\end{pgfscope}%
\end{pgfscope}%
\begin{pgfscope}%
\pgfpathrectangle{\pgfqpoint{0.550713in}{0.127635in}}{\pgfqpoint{3.194133in}{2.297424in}}%
\pgfusepath{clip}%
\pgfsetrectcap%
\pgfsetroundjoin%
\pgfsetlinewidth{0.752812pt}%
\definecolor{currentstroke}{rgb}{0.000000,0.000000,0.000000}%
\pgfsetstrokecolor{currentstroke}%
\pgfsetdash{}{0pt}%
\pgfpathmoveto{\pgfqpoint{0.751943in}{1.463181in}}%
\pgfpathlineto{\pgfqpoint{0.908456in}{1.463181in}}%
\pgfusepath{stroke}%
\end{pgfscope}%
\begin{pgfscope}%
\pgfpathrectangle{\pgfqpoint{0.550713in}{0.127635in}}{\pgfqpoint{3.194133in}{2.297424in}}%
\pgfusepath{clip}%
\pgfsetbuttcap%
\pgfsetroundjoin%
\definecolor{currentfill}{rgb}{1.000000,1.000000,1.000000}%
\pgfsetfillcolor{currentfill}%
\pgfsetlinewidth{1.003750pt}%
\definecolor{currentstroke}{rgb}{0.000000,0.000000,0.000000}%
\pgfsetstrokecolor{currentstroke}%
\pgfsetdash{}{0pt}%
\pgfsys@defobject{currentmarker}{\pgfqpoint{-0.027778in}{-0.027778in}}{\pgfqpoint{0.027778in}{0.027778in}}{%
\pgfpathmoveto{\pgfqpoint{0.000000in}{-0.027778in}}%
\pgfpathcurveto{\pgfqpoint{0.007367in}{-0.027778in}}{\pgfqpoint{0.014433in}{-0.024851in}}{\pgfqpoint{0.019642in}{-0.019642in}}%
\pgfpathcurveto{\pgfqpoint{0.024851in}{-0.014433in}}{\pgfqpoint{0.027778in}{-0.007367in}}{\pgfqpoint{0.027778in}{0.000000in}}%
\pgfpathcurveto{\pgfqpoint{0.027778in}{0.007367in}}{\pgfqpoint{0.024851in}{0.014433in}}{\pgfqpoint{0.019642in}{0.019642in}}%
\pgfpathcurveto{\pgfqpoint{0.014433in}{0.024851in}}{\pgfqpoint{0.007367in}{0.027778in}}{\pgfqpoint{0.000000in}{0.027778in}}%
\pgfpathcurveto{\pgfqpoint{-0.007367in}{0.027778in}}{\pgfqpoint{-0.014433in}{0.024851in}}{\pgfqpoint{-0.019642in}{0.019642in}}%
\pgfpathcurveto{\pgfqpoint{-0.024851in}{0.014433in}}{\pgfqpoint{-0.027778in}{0.007367in}}{\pgfqpoint{-0.027778in}{0.000000in}}%
\pgfpathcurveto{\pgfqpoint{-0.027778in}{-0.007367in}}{\pgfqpoint{-0.024851in}{-0.014433in}}{\pgfqpoint{-0.019642in}{-0.019642in}}%
\pgfpathcurveto{\pgfqpoint{-0.014433in}{-0.024851in}}{\pgfqpoint{-0.007367in}{-0.027778in}}{\pgfqpoint{0.000000in}{-0.027778in}}%
\pgfpathclose%
\pgfusepath{stroke,fill}%
}%
\begin{pgfscope}%
\pgfsys@transformshift{0.830199in}{1.590759in}%
\pgfsys@useobject{currentmarker}{}%
\end{pgfscope}%
\end{pgfscope}%
\begin{pgfscope}%
\pgfpathrectangle{\pgfqpoint{0.550713in}{0.127635in}}{\pgfqpoint{3.194133in}{2.297424in}}%
\pgfusepath{clip}%
\pgfsetrectcap%
\pgfsetroundjoin%
\pgfsetlinewidth{0.752812pt}%
\definecolor{currentstroke}{rgb}{0.000000,0.000000,0.000000}%
\pgfsetstrokecolor{currentstroke}%
\pgfsetdash{}{0pt}%
\pgfpathmoveto{\pgfqpoint{0.991503in}{1.038084in}}%
\pgfpathlineto{\pgfqpoint{1.148015in}{1.038084in}}%
\pgfusepath{stroke}%
\end{pgfscope}%
\begin{pgfscope}%
\pgfpathrectangle{\pgfqpoint{0.550713in}{0.127635in}}{\pgfqpoint{3.194133in}{2.297424in}}%
\pgfusepath{clip}%
\pgfsetbuttcap%
\pgfsetroundjoin%
\definecolor{currentfill}{rgb}{1.000000,1.000000,1.000000}%
\pgfsetfillcolor{currentfill}%
\pgfsetlinewidth{1.003750pt}%
\definecolor{currentstroke}{rgb}{0.000000,0.000000,0.000000}%
\pgfsetstrokecolor{currentstroke}%
\pgfsetdash{}{0pt}%
\pgfsys@defobject{currentmarker}{\pgfqpoint{-0.027778in}{-0.027778in}}{\pgfqpoint{0.027778in}{0.027778in}}{%
\pgfpathmoveto{\pgfqpoint{0.000000in}{-0.027778in}}%
\pgfpathcurveto{\pgfqpoint{0.007367in}{-0.027778in}}{\pgfqpoint{0.014433in}{-0.024851in}}{\pgfqpoint{0.019642in}{-0.019642in}}%
\pgfpathcurveto{\pgfqpoint{0.024851in}{-0.014433in}}{\pgfqpoint{0.027778in}{-0.007367in}}{\pgfqpoint{0.027778in}{0.000000in}}%
\pgfpathcurveto{\pgfqpoint{0.027778in}{0.007367in}}{\pgfqpoint{0.024851in}{0.014433in}}{\pgfqpoint{0.019642in}{0.019642in}}%
\pgfpathcurveto{\pgfqpoint{0.014433in}{0.024851in}}{\pgfqpoint{0.007367in}{0.027778in}}{\pgfqpoint{0.000000in}{0.027778in}}%
\pgfpathcurveto{\pgfqpoint{-0.007367in}{0.027778in}}{\pgfqpoint{-0.014433in}{0.024851in}}{\pgfqpoint{-0.019642in}{0.019642in}}%
\pgfpathcurveto{\pgfqpoint{-0.024851in}{0.014433in}}{\pgfqpoint{-0.027778in}{0.007367in}}{\pgfqpoint{-0.027778in}{0.000000in}}%
\pgfpathcurveto{\pgfqpoint{-0.027778in}{-0.007367in}}{\pgfqpoint{-0.024851in}{-0.014433in}}{\pgfqpoint{-0.019642in}{-0.019642in}}%
\pgfpathcurveto{\pgfqpoint{-0.014433in}{-0.024851in}}{\pgfqpoint{-0.007367in}{-0.027778in}}{\pgfqpoint{0.000000in}{-0.027778in}}%
\pgfpathclose%
\pgfusepath{stroke,fill}%
}%
\begin{pgfscope}%
\pgfsys@transformshift{1.069759in}{1.195661in}%
\pgfsys@useobject{currentmarker}{}%
\end{pgfscope}%
\end{pgfscope}%
\begin{pgfscope}%
\pgfpathrectangle{\pgfqpoint{0.550713in}{0.127635in}}{\pgfqpoint{3.194133in}{2.297424in}}%
\pgfusepath{clip}%
\pgfsetrectcap%
\pgfsetroundjoin%
\pgfsetlinewidth{0.752812pt}%
\definecolor{currentstroke}{rgb}{0.000000,0.000000,0.000000}%
\pgfsetstrokecolor{currentstroke}%
\pgfsetdash{}{0pt}%
\pgfpathmoveto{\pgfqpoint{1.151210in}{2.163439in}}%
\pgfpathlineto{\pgfqpoint{1.307722in}{2.163439in}}%
\pgfusepath{stroke}%
\end{pgfscope}%
\begin{pgfscope}%
\pgfpathrectangle{\pgfqpoint{0.550713in}{0.127635in}}{\pgfqpoint{3.194133in}{2.297424in}}%
\pgfusepath{clip}%
\pgfsetbuttcap%
\pgfsetroundjoin%
\definecolor{currentfill}{rgb}{1.000000,1.000000,1.000000}%
\pgfsetfillcolor{currentfill}%
\pgfsetlinewidth{1.003750pt}%
\definecolor{currentstroke}{rgb}{0.000000,0.000000,0.000000}%
\pgfsetstrokecolor{currentstroke}%
\pgfsetdash{}{0pt}%
\pgfsys@defobject{currentmarker}{\pgfqpoint{-0.027778in}{-0.027778in}}{\pgfqpoint{0.027778in}{0.027778in}}{%
\pgfpathmoveto{\pgfqpoint{0.000000in}{-0.027778in}}%
\pgfpathcurveto{\pgfqpoint{0.007367in}{-0.027778in}}{\pgfqpoint{0.014433in}{-0.024851in}}{\pgfqpoint{0.019642in}{-0.019642in}}%
\pgfpathcurveto{\pgfqpoint{0.024851in}{-0.014433in}}{\pgfqpoint{0.027778in}{-0.007367in}}{\pgfqpoint{0.027778in}{0.000000in}}%
\pgfpathcurveto{\pgfqpoint{0.027778in}{0.007367in}}{\pgfqpoint{0.024851in}{0.014433in}}{\pgfqpoint{0.019642in}{0.019642in}}%
\pgfpathcurveto{\pgfqpoint{0.014433in}{0.024851in}}{\pgfqpoint{0.007367in}{0.027778in}}{\pgfqpoint{0.000000in}{0.027778in}}%
\pgfpathcurveto{\pgfqpoint{-0.007367in}{0.027778in}}{\pgfqpoint{-0.014433in}{0.024851in}}{\pgfqpoint{-0.019642in}{0.019642in}}%
\pgfpathcurveto{\pgfqpoint{-0.024851in}{0.014433in}}{\pgfqpoint{-0.027778in}{0.007367in}}{\pgfqpoint{-0.027778in}{0.000000in}}%
\pgfpathcurveto{\pgfqpoint{-0.027778in}{-0.007367in}}{\pgfqpoint{-0.024851in}{-0.014433in}}{\pgfqpoint{-0.019642in}{-0.019642in}}%
\pgfpathcurveto{\pgfqpoint{-0.014433in}{-0.024851in}}{\pgfqpoint{-0.007367in}{-0.027778in}}{\pgfqpoint{0.000000in}{-0.027778in}}%
\pgfpathclose%
\pgfusepath{stroke,fill}%
}%
\begin{pgfscope}%
\pgfsys@transformshift{1.229466in}{2.155704in}%
\pgfsys@useobject{currentmarker}{}%
\end{pgfscope}%
\end{pgfscope}%
\begin{pgfscope}%
\pgfpathrectangle{\pgfqpoint{0.550713in}{0.127635in}}{\pgfqpoint{3.194133in}{2.297424in}}%
\pgfusepath{clip}%
\pgfsetrectcap%
\pgfsetroundjoin%
\pgfsetlinewidth{0.752812pt}%
\definecolor{currentstroke}{rgb}{0.000000,0.000000,0.000000}%
\pgfsetstrokecolor{currentstroke}%
\pgfsetdash{}{0pt}%
\pgfpathmoveto{\pgfqpoint{1.390770in}{0.883323in}}%
\pgfpathlineto{\pgfqpoint{1.547282in}{0.883323in}}%
\pgfusepath{stroke}%
\end{pgfscope}%
\begin{pgfscope}%
\pgfpathrectangle{\pgfqpoint{0.550713in}{0.127635in}}{\pgfqpoint{3.194133in}{2.297424in}}%
\pgfusepath{clip}%
\pgfsetbuttcap%
\pgfsetroundjoin%
\definecolor{currentfill}{rgb}{1.000000,1.000000,1.000000}%
\pgfsetfillcolor{currentfill}%
\pgfsetlinewidth{1.003750pt}%
\definecolor{currentstroke}{rgb}{0.000000,0.000000,0.000000}%
\pgfsetstrokecolor{currentstroke}%
\pgfsetdash{}{0pt}%
\pgfsys@defobject{currentmarker}{\pgfqpoint{-0.027778in}{-0.027778in}}{\pgfqpoint{0.027778in}{0.027778in}}{%
\pgfpathmoveto{\pgfqpoint{0.000000in}{-0.027778in}}%
\pgfpathcurveto{\pgfqpoint{0.007367in}{-0.027778in}}{\pgfqpoint{0.014433in}{-0.024851in}}{\pgfqpoint{0.019642in}{-0.019642in}}%
\pgfpathcurveto{\pgfqpoint{0.024851in}{-0.014433in}}{\pgfqpoint{0.027778in}{-0.007367in}}{\pgfqpoint{0.027778in}{0.000000in}}%
\pgfpathcurveto{\pgfqpoint{0.027778in}{0.007367in}}{\pgfqpoint{0.024851in}{0.014433in}}{\pgfqpoint{0.019642in}{0.019642in}}%
\pgfpathcurveto{\pgfqpoint{0.014433in}{0.024851in}}{\pgfqpoint{0.007367in}{0.027778in}}{\pgfqpoint{0.000000in}{0.027778in}}%
\pgfpathcurveto{\pgfqpoint{-0.007367in}{0.027778in}}{\pgfqpoint{-0.014433in}{0.024851in}}{\pgfqpoint{-0.019642in}{0.019642in}}%
\pgfpathcurveto{\pgfqpoint{-0.024851in}{0.014433in}}{\pgfqpoint{-0.027778in}{0.007367in}}{\pgfqpoint{-0.027778in}{0.000000in}}%
\pgfpathcurveto{\pgfqpoint{-0.027778in}{-0.007367in}}{\pgfqpoint{-0.024851in}{-0.014433in}}{\pgfqpoint{-0.019642in}{-0.019642in}}%
\pgfpathcurveto{\pgfqpoint{-0.014433in}{-0.024851in}}{\pgfqpoint{-0.007367in}{-0.027778in}}{\pgfqpoint{0.000000in}{-0.027778in}}%
\pgfpathclose%
\pgfusepath{stroke,fill}%
}%
\begin{pgfscope}%
\pgfsys@transformshift{1.469026in}{0.887695in}%
\pgfsys@useobject{currentmarker}{}%
\end{pgfscope}%
\end{pgfscope}%
\begin{pgfscope}%
\pgfpathrectangle{\pgfqpoint{0.550713in}{0.127635in}}{\pgfqpoint{3.194133in}{2.297424in}}%
\pgfusepath{clip}%
\pgfsetrectcap%
\pgfsetroundjoin%
\pgfsetlinewidth{0.752812pt}%
\definecolor{currentstroke}{rgb}{0.000000,0.000000,0.000000}%
\pgfsetstrokecolor{currentstroke}%
\pgfsetdash{}{0pt}%
\pgfpathmoveto{\pgfqpoint{1.550476in}{0.912645in}}%
\pgfpathlineto{\pgfqpoint{1.706989in}{0.912645in}}%
\pgfusepath{stroke}%
\end{pgfscope}%
\begin{pgfscope}%
\pgfpathrectangle{\pgfqpoint{0.550713in}{0.127635in}}{\pgfqpoint{3.194133in}{2.297424in}}%
\pgfusepath{clip}%
\pgfsetbuttcap%
\pgfsetroundjoin%
\definecolor{currentfill}{rgb}{1.000000,1.000000,1.000000}%
\pgfsetfillcolor{currentfill}%
\pgfsetlinewidth{1.003750pt}%
\definecolor{currentstroke}{rgb}{0.000000,0.000000,0.000000}%
\pgfsetstrokecolor{currentstroke}%
\pgfsetdash{}{0pt}%
\pgfsys@defobject{currentmarker}{\pgfqpoint{-0.027778in}{-0.027778in}}{\pgfqpoint{0.027778in}{0.027778in}}{%
\pgfpathmoveto{\pgfqpoint{0.000000in}{-0.027778in}}%
\pgfpathcurveto{\pgfqpoint{0.007367in}{-0.027778in}}{\pgfqpoint{0.014433in}{-0.024851in}}{\pgfqpoint{0.019642in}{-0.019642in}}%
\pgfpathcurveto{\pgfqpoint{0.024851in}{-0.014433in}}{\pgfqpoint{0.027778in}{-0.007367in}}{\pgfqpoint{0.027778in}{0.000000in}}%
\pgfpathcurveto{\pgfqpoint{0.027778in}{0.007367in}}{\pgfqpoint{0.024851in}{0.014433in}}{\pgfqpoint{0.019642in}{0.019642in}}%
\pgfpathcurveto{\pgfqpoint{0.014433in}{0.024851in}}{\pgfqpoint{0.007367in}{0.027778in}}{\pgfqpoint{0.000000in}{0.027778in}}%
\pgfpathcurveto{\pgfqpoint{-0.007367in}{0.027778in}}{\pgfqpoint{-0.014433in}{0.024851in}}{\pgfqpoint{-0.019642in}{0.019642in}}%
\pgfpathcurveto{\pgfqpoint{-0.024851in}{0.014433in}}{\pgfqpoint{-0.027778in}{0.007367in}}{\pgfqpoint{-0.027778in}{0.000000in}}%
\pgfpathcurveto{\pgfqpoint{-0.027778in}{-0.007367in}}{\pgfqpoint{-0.024851in}{-0.014433in}}{\pgfqpoint{-0.019642in}{-0.019642in}}%
\pgfpathcurveto{\pgfqpoint{-0.014433in}{-0.024851in}}{\pgfqpoint{-0.007367in}{-0.027778in}}{\pgfqpoint{0.000000in}{-0.027778in}}%
\pgfpathclose%
\pgfusepath{stroke,fill}%
}%
\begin{pgfscope}%
\pgfsys@transformshift{1.628733in}{0.919127in}%
\pgfsys@useobject{currentmarker}{}%
\end{pgfscope}%
\end{pgfscope}%
\begin{pgfscope}%
\pgfpathrectangle{\pgfqpoint{0.550713in}{0.127635in}}{\pgfqpoint{3.194133in}{2.297424in}}%
\pgfusepath{clip}%
\pgfsetrectcap%
\pgfsetroundjoin%
\pgfsetlinewidth{0.752812pt}%
\definecolor{currentstroke}{rgb}{0.000000,0.000000,0.000000}%
\pgfsetstrokecolor{currentstroke}%
\pgfsetdash{}{0pt}%
\pgfpathmoveto{\pgfqpoint{1.790036in}{1.025120in}}%
\pgfpathlineto{\pgfqpoint{1.946549in}{1.025120in}}%
\pgfusepath{stroke}%
\end{pgfscope}%
\begin{pgfscope}%
\pgfpathrectangle{\pgfqpoint{0.550713in}{0.127635in}}{\pgfqpoint{3.194133in}{2.297424in}}%
\pgfusepath{clip}%
\pgfsetbuttcap%
\pgfsetroundjoin%
\definecolor{currentfill}{rgb}{1.000000,1.000000,1.000000}%
\pgfsetfillcolor{currentfill}%
\pgfsetlinewidth{1.003750pt}%
\definecolor{currentstroke}{rgb}{0.000000,0.000000,0.000000}%
\pgfsetstrokecolor{currentstroke}%
\pgfsetdash{}{0pt}%
\pgfsys@defobject{currentmarker}{\pgfqpoint{-0.027778in}{-0.027778in}}{\pgfqpoint{0.027778in}{0.027778in}}{%
\pgfpathmoveto{\pgfqpoint{0.000000in}{-0.027778in}}%
\pgfpathcurveto{\pgfqpoint{0.007367in}{-0.027778in}}{\pgfqpoint{0.014433in}{-0.024851in}}{\pgfqpoint{0.019642in}{-0.019642in}}%
\pgfpathcurveto{\pgfqpoint{0.024851in}{-0.014433in}}{\pgfqpoint{0.027778in}{-0.007367in}}{\pgfqpoint{0.027778in}{0.000000in}}%
\pgfpathcurveto{\pgfqpoint{0.027778in}{0.007367in}}{\pgfqpoint{0.024851in}{0.014433in}}{\pgfqpoint{0.019642in}{0.019642in}}%
\pgfpathcurveto{\pgfqpoint{0.014433in}{0.024851in}}{\pgfqpoint{0.007367in}{0.027778in}}{\pgfqpoint{0.000000in}{0.027778in}}%
\pgfpathcurveto{\pgfqpoint{-0.007367in}{0.027778in}}{\pgfqpoint{-0.014433in}{0.024851in}}{\pgfqpoint{-0.019642in}{0.019642in}}%
\pgfpathcurveto{\pgfqpoint{-0.024851in}{0.014433in}}{\pgfqpoint{-0.027778in}{0.007367in}}{\pgfqpoint{-0.027778in}{0.000000in}}%
\pgfpathcurveto{\pgfqpoint{-0.027778in}{-0.007367in}}{\pgfqpoint{-0.024851in}{-0.014433in}}{\pgfqpoint{-0.019642in}{-0.019642in}}%
\pgfpathcurveto{\pgfqpoint{-0.014433in}{-0.024851in}}{\pgfqpoint{-0.007367in}{-0.027778in}}{\pgfqpoint{0.000000in}{-0.027778in}}%
\pgfpathclose%
\pgfusepath{stroke,fill}%
}%
\begin{pgfscope}%
\pgfsys@transformshift{1.868293in}{1.005424in}%
\pgfsys@useobject{currentmarker}{}%
\end{pgfscope}%
\end{pgfscope}%
\begin{pgfscope}%
\pgfpathrectangle{\pgfqpoint{0.550713in}{0.127635in}}{\pgfqpoint{3.194133in}{2.297424in}}%
\pgfusepath{clip}%
\pgfsetrectcap%
\pgfsetroundjoin%
\pgfsetlinewidth{0.752812pt}%
\definecolor{currentstroke}{rgb}{0.000000,0.000000,0.000000}%
\pgfsetstrokecolor{currentstroke}%
\pgfsetdash{}{0pt}%
\pgfpathmoveto{\pgfqpoint{1.949743in}{1.042862in}}%
\pgfpathlineto{\pgfqpoint{2.106255in}{1.042862in}}%
\pgfusepath{stroke}%
\end{pgfscope}%
\begin{pgfscope}%
\pgfpathrectangle{\pgfqpoint{0.550713in}{0.127635in}}{\pgfqpoint{3.194133in}{2.297424in}}%
\pgfusepath{clip}%
\pgfsetbuttcap%
\pgfsetroundjoin%
\definecolor{currentfill}{rgb}{1.000000,1.000000,1.000000}%
\pgfsetfillcolor{currentfill}%
\pgfsetlinewidth{1.003750pt}%
\definecolor{currentstroke}{rgb}{0.000000,0.000000,0.000000}%
\pgfsetstrokecolor{currentstroke}%
\pgfsetdash{}{0pt}%
\pgfsys@defobject{currentmarker}{\pgfqpoint{-0.027778in}{-0.027778in}}{\pgfqpoint{0.027778in}{0.027778in}}{%
\pgfpathmoveto{\pgfqpoint{0.000000in}{-0.027778in}}%
\pgfpathcurveto{\pgfqpoint{0.007367in}{-0.027778in}}{\pgfqpoint{0.014433in}{-0.024851in}}{\pgfqpoint{0.019642in}{-0.019642in}}%
\pgfpathcurveto{\pgfqpoint{0.024851in}{-0.014433in}}{\pgfqpoint{0.027778in}{-0.007367in}}{\pgfqpoint{0.027778in}{0.000000in}}%
\pgfpathcurveto{\pgfqpoint{0.027778in}{0.007367in}}{\pgfqpoint{0.024851in}{0.014433in}}{\pgfqpoint{0.019642in}{0.019642in}}%
\pgfpathcurveto{\pgfqpoint{0.014433in}{0.024851in}}{\pgfqpoint{0.007367in}{0.027778in}}{\pgfqpoint{0.000000in}{0.027778in}}%
\pgfpathcurveto{\pgfqpoint{-0.007367in}{0.027778in}}{\pgfqpoint{-0.014433in}{0.024851in}}{\pgfqpoint{-0.019642in}{0.019642in}}%
\pgfpathcurveto{\pgfqpoint{-0.024851in}{0.014433in}}{\pgfqpoint{-0.027778in}{0.007367in}}{\pgfqpoint{-0.027778in}{0.000000in}}%
\pgfpathcurveto{\pgfqpoint{-0.027778in}{-0.007367in}}{\pgfqpoint{-0.024851in}{-0.014433in}}{\pgfqpoint{-0.019642in}{-0.019642in}}%
\pgfpathcurveto{\pgfqpoint{-0.014433in}{-0.024851in}}{\pgfqpoint{-0.007367in}{-0.027778in}}{\pgfqpoint{0.000000in}{-0.027778in}}%
\pgfpathclose%
\pgfusepath{stroke,fill}%
}%
\begin{pgfscope}%
\pgfsys@transformshift{2.027999in}{1.046940in}%
\pgfsys@useobject{currentmarker}{}%
\end{pgfscope}%
\end{pgfscope}%
\begin{pgfscope}%
\pgfpathrectangle{\pgfqpoint{0.550713in}{0.127635in}}{\pgfqpoint{3.194133in}{2.297424in}}%
\pgfusepath{clip}%
\pgfsetrectcap%
\pgfsetroundjoin%
\pgfsetlinewidth{0.752812pt}%
\definecolor{currentstroke}{rgb}{0.000000,0.000000,0.000000}%
\pgfsetstrokecolor{currentstroke}%
\pgfsetdash{}{0pt}%
\pgfpathmoveto{\pgfqpoint{2.189303in}{0.655768in}}%
\pgfpathlineto{\pgfqpoint{2.345815in}{0.655768in}}%
\pgfusepath{stroke}%
\end{pgfscope}%
\begin{pgfscope}%
\pgfpathrectangle{\pgfqpoint{0.550713in}{0.127635in}}{\pgfqpoint{3.194133in}{2.297424in}}%
\pgfusepath{clip}%
\pgfsetbuttcap%
\pgfsetroundjoin%
\definecolor{currentfill}{rgb}{1.000000,1.000000,1.000000}%
\pgfsetfillcolor{currentfill}%
\pgfsetlinewidth{1.003750pt}%
\definecolor{currentstroke}{rgb}{0.000000,0.000000,0.000000}%
\pgfsetstrokecolor{currentstroke}%
\pgfsetdash{}{0pt}%
\pgfsys@defobject{currentmarker}{\pgfqpoint{-0.027778in}{-0.027778in}}{\pgfqpoint{0.027778in}{0.027778in}}{%
\pgfpathmoveto{\pgfqpoint{0.000000in}{-0.027778in}}%
\pgfpathcurveto{\pgfqpoint{0.007367in}{-0.027778in}}{\pgfqpoint{0.014433in}{-0.024851in}}{\pgfqpoint{0.019642in}{-0.019642in}}%
\pgfpathcurveto{\pgfqpoint{0.024851in}{-0.014433in}}{\pgfqpoint{0.027778in}{-0.007367in}}{\pgfqpoint{0.027778in}{0.000000in}}%
\pgfpathcurveto{\pgfqpoint{0.027778in}{0.007367in}}{\pgfqpoint{0.024851in}{0.014433in}}{\pgfqpoint{0.019642in}{0.019642in}}%
\pgfpathcurveto{\pgfqpoint{0.014433in}{0.024851in}}{\pgfqpoint{0.007367in}{0.027778in}}{\pgfqpoint{0.000000in}{0.027778in}}%
\pgfpathcurveto{\pgfqpoint{-0.007367in}{0.027778in}}{\pgfqpoint{-0.014433in}{0.024851in}}{\pgfqpoint{-0.019642in}{0.019642in}}%
\pgfpathcurveto{\pgfqpoint{-0.024851in}{0.014433in}}{\pgfqpoint{-0.027778in}{0.007367in}}{\pgfqpoint{-0.027778in}{0.000000in}}%
\pgfpathcurveto{\pgfqpoint{-0.027778in}{-0.007367in}}{\pgfqpoint{-0.024851in}{-0.014433in}}{\pgfqpoint{-0.019642in}{-0.019642in}}%
\pgfpathcurveto{\pgfqpoint{-0.014433in}{-0.024851in}}{\pgfqpoint{-0.007367in}{-0.027778in}}{\pgfqpoint{0.000000in}{-0.027778in}}%
\pgfpathclose%
\pgfusepath{stroke,fill}%
}%
\begin{pgfscope}%
\pgfsys@transformshift{2.267559in}{0.657912in}%
\pgfsys@useobject{currentmarker}{}%
\end{pgfscope}%
\end{pgfscope}%
\begin{pgfscope}%
\pgfpathrectangle{\pgfqpoint{0.550713in}{0.127635in}}{\pgfqpoint{3.194133in}{2.297424in}}%
\pgfusepath{clip}%
\pgfsetrectcap%
\pgfsetroundjoin%
\pgfsetlinewidth{0.752812pt}%
\definecolor{currentstroke}{rgb}{0.000000,0.000000,0.000000}%
\pgfsetstrokecolor{currentstroke}%
\pgfsetdash{}{0pt}%
\pgfpathmoveto{\pgfqpoint{2.349010in}{0.725637in}}%
\pgfpathlineto{\pgfqpoint{2.505522in}{0.725637in}}%
\pgfusepath{stroke}%
\end{pgfscope}%
\begin{pgfscope}%
\pgfpathrectangle{\pgfqpoint{0.550713in}{0.127635in}}{\pgfqpoint{3.194133in}{2.297424in}}%
\pgfusepath{clip}%
\pgfsetbuttcap%
\pgfsetroundjoin%
\definecolor{currentfill}{rgb}{1.000000,1.000000,1.000000}%
\pgfsetfillcolor{currentfill}%
\pgfsetlinewidth{1.003750pt}%
\definecolor{currentstroke}{rgb}{0.000000,0.000000,0.000000}%
\pgfsetstrokecolor{currentstroke}%
\pgfsetdash{}{0pt}%
\pgfsys@defobject{currentmarker}{\pgfqpoint{-0.027778in}{-0.027778in}}{\pgfqpoint{0.027778in}{0.027778in}}{%
\pgfpathmoveto{\pgfqpoint{0.000000in}{-0.027778in}}%
\pgfpathcurveto{\pgfqpoint{0.007367in}{-0.027778in}}{\pgfqpoint{0.014433in}{-0.024851in}}{\pgfqpoint{0.019642in}{-0.019642in}}%
\pgfpathcurveto{\pgfqpoint{0.024851in}{-0.014433in}}{\pgfqpoint{0.027778in}{-0.007367in}}{\pgfqpoint{0.027778in}{0.000000in}}%
\pgfpathcurveto{\pgfqpoint{0.027778in}{0.007367in}}{\pgfqpoint{0.024851in}{0.014433in}}{\pgfqpoint{0.019642in}{0.019642in}}%
\pgfpathcurveto{\pgfqpoint{0.014433in}{0.024851in}}{\pgfqpoint{0.007367in}{0.027778in}}{\pgfqpoint{0.000000in}{0.027778in}}%
\pgfpathcurveto{\pgfqpoint{-0.007367in}{0.027778in}}{\pgfqpoint{-0.014433in}{0.024851in}}{\pgfqpoint{-0.019642in}{0.019642in}}%
\pgfpathcurveto{\pgfqpoint{-0.024851in}{0.014433in}}{\pgfqpoint{-0.027778in}{0.007367in}}{\pgfqpoint{-0.027778in}{0.000000in}}%
\pgfpathcurveto{\pgfqpoint{-0.027778in}{-0.007367in}}{\pgfqpoint{-0.024851in}{-0.014433in}}{\pgfqpoint{-0.019642in}{-0.019642in}}%
\pgfpathcurveto{\pgfqpoint{-0.014433in}{-0.024851in}}{\pgfqpoint{-0.007367in}{-0.027778in}}{\pgfqpoint{0.000000in}{-0.027778in}}%
\pgfpathclose%
\pgfusepath{stroke,fill}%
}%
\begin{pgfscope}%
\pgfsys@transformshift{2.427266in}{0.729630in}%
\pgfsys@useobject{currentmarker}{}%
\end{pgfscope}%
\end{pgfscope}%
\begin{pgfscope}%
\pgfpathrectangle{\pgfqpoint{0.550713in}{0.127635in}}{\pgfqpoint{3.194133in}{2.297424in}}%
\pgfusepath{clip}%
\pgfsetrectcap%
\pgfsetroundjoin%
\pgfsetlinewidth{0.752812pt}%
\definecolor{currentstroke}{rgb}{0.000000,0.000000,0.000000}%
\pgfsetstrokecolor{currentstroke}%
\pgfsetdash{}{0pt}%
\pgfpathmoveto{\pgfqpoint{2.588570in}{0.660672in}}%
\pgfpathlineto{\pgfqpoint{2.745082in}{0.660672in}}%
\pgfusepath{stroke}%
\end{pgfscope}%
\begin{pgfscope}%
\pgfpathrectangle{\pgfqpoint{0.550713in}{0.127635in}}{\pgfqpoint{3.194133in}{2.297424in}}%
\pgfusepath{clip}%
\pgfsetbuttcap%
\pgfsetroundjoin%
\definecolor{currentfill}{rgb}{1.000000,1.000000,1.000000}%
\pgfsetfillcolor{currentfill}%
\pgfsetlinewidth{1.003750pt}%
\definecolor{currentstroke}{rgb}{0.000000,0.000000,0.000000}%
\pgfsetstrokecolor{currentstroke}%
\pgfsetdash{}{0pt}%
\pgfsys@defobject{currentmarker}{\pgfqpoint{-0.027778in}{-0.027778in}}{\pgfqpoint{0.027778in}{0.027778in}}{%
\pgfpathmoveto{\pgfqpoint{0.000000in}{-0.027778in}}%
\pgfpathcurveto{\pgfqpoint{0.007367in}{-0.027778in}}{\pgfqpoint{0.014433in}{-0.024851in}}{\pgfqpoint{0.019642in}{-0.019642in}}%
\pgfpathcurveto{\pgfqpoint{0.024851in}{-0.014433in}}{\pgfqpoint{0.027778in}{-0.007367in}}{\pgfqpoint{0.027778in}{0.000000in}}%
\pgfpathcurveto{\pgfqpoint{0.027778in}{0.007367in}}{\pgfqpoint{0.024851in}{0.014433in}}{\pgfqpoint{0.019642in}{0.019642in}}%
\pgfpathcurveto{\pgfqpoint{0.014433in}{0.024851in}}{\pgfqpoint{0.007367in}{0.027778in}}{\pgfqpoint{0.000000in}{0.027778in}}%
\pgfpathcurveto{\pgfqpoint{-0.007367in}{0.027778in}}{\pgfqpoint{-0.014433in}{0.024851in}}{\pgfqpoint{-0.019642in}{0.019642in}}%
\pgfpathcurveto{\pgfqpoint{-0.024851in}{0.014433in}}{\pgfqpoint{-0.027778in}{0.007367in}}{\pgfqpoint{-0.027778in}{0.000000in}}%
\pgfpathcurveto{\pgfqpoint{-0.027778in}{-0.007367in}}{\pgfqpoint{-0.024851in}{-0.014433in}}{\pgfqpoint{-0.019642in}{-0.019642in}}%
\pgfpathcurveto{\pgfqpoint{-0.014433in}{-0.024851in}}{\pgfqpoint{-0.007367in}{-0.027778in}}{\pgfqpoint{0.000000in}{-0.027778in}}%
\pgfpathclose%
\pgfusepath{stroke,fill}%
}%
\begin{pgfscope}%
\pgfsys@transformshift{2.666826in}{0.660811in}%
\pgfsys@useobject{currentmarker}{}%
\end{pgfscope}%
\end{pgfscope}%
\begin{pgfscope}%
\pgfpathrectangle{\pgfqpoint{0.550713in}{0.127635in}}{\pgfqpoint{3.194133in}{2.297424in}}%
\pgfusepath{clip}%
\pgfsetrectcap%
\pgfsetroundjoin%
\pgfsetlinewidth{0.752812pt}%
\definecolor{currentstroke}{rgb}{0.000000,0.000000,0.000000}%
\pgfsetstrokecolor{currentstroke}%
\pgfsetdash{}{0pt}%
\pgfpathmoveto{\pgfqpoint{2.748276in}{0.733173in}}%
\pgfpathlineto{\pgfqpoint{2.904789in}{0.733173in}}%
\pgfusepath{stroke}%
\end{pgfscope}%
\begin{pgfscope}%
\pgfpathrectangle{\pgfqpoint{0.550713in}{0.127635in}}{\pgfqpoint{3.194133in}{2.297424in}}%
\pgfusepath{clip}%
\pgfsetbuttcap%
\pgfsetroundjoin%
\definecolor{currentfill}{rgb}{1.000000,1.000000,1.000000}%
\pgfsetfillcolor{currentfill}%
\pgfsetlinewidth{1.003750pt}%
\definecolor{currentstroke}{rgb}{0.000000,0.000000,0.000000}%
\pgfsetstrokecolor{currentstroke}%
\pgfsetdash{}{0pt}%
\pgfsys@defobject{currentmarker}{\pgfqpoint{-0.027778in}{-0.027778in}}{\pgfqpoint{0.027778in}{0.027778in}}{%
\pgfpathmoveto{\pgfqpoint{0.000000in}{-0.027778in}}%
\pgfpathcurveto{\pgfqpoint{0.007367in}{-0.027778in}}{\pgfqpoint{0.014433in}{-0.024851in}}{\pgfqpoint{0.019642in}{-0.019642in}}%
\pgfpathcurveto{\pgfqpoint{0.024851in}{-0.014433in}}{\pgfqpoint{0.027778in}{-0.007367in}}{\pgfqpoint{0.027778in}{0.000000in}}%
\pgfpathcurveto{\pgfqpoint{0.027778in}{0.007367in}}{\pgfqpoint{0.024851in}{0.014433in}}{\pgfqpoint{0.019642in}{0.019642in}}%
\pgfpathcurveto{\pgfqpoint{0.014433in}{0.024851in}}{\pgfqpoint{0.007367in}{0.027778in}}{\pgfqpoint{0.000000in}{0.027778in}}%
\pgfpathcurveto{\pgfqpoint{-0.007367in}{0.027778in}}{\pgfqpoint{-0.014433in}{0.024851in}}{\pgfqpoint{-0.019642in}{0.019642in}}%
\pgfpathcurveto{\pgfqpoint{-0.024851in}{0.014433in}}{\pgfqpoint{-0.027778in}{0.007367in}}{\pgfqpoint{-0.027778in}{0.000000in}}%
\pgfpathcurveto{\pgfqpoint{-0.027778in}{-0.007367in}}{\pgfqpoint{-0.024851in}{-0.014433in}}{\pgfqpoint{-0.019642in}{-0.019642in}}%
\pgfpathcurveto{\pgfqpoint{-0.014433in}{-0.024851in}}{\pgfqpoint{-0.007367in}{-0.027778in}}{\pgfqpoint{0.000000in}{-0.027778in}}%
\pgfpathclose%
\pgfusepath{stroke,fill}%
}%
\begin{pgfscope}%
\pgfsys@transformshift{2.826532in}{0.738632in}%
\pgfsys@useobject{currentmarker}{}%
\end{pgfscope}%
\end{pgfscope}%
\begin{pgfscope}%
\pgfpathrectangle{\pgfqpoint{0.550713in}{0.127635in}}{\pgfqpoint{3.194133in}{2.297424in}}%
\pgfusepath{clip}%
\pgfsetrectcap%
\pgfsetroundjoin%
\pgfsetlinewidth{0.752812pt}%
\definecolor{currentstroke}{rgb}{0.000000,0.000000,0.000000}%
\pgfsetstrokecolor{currentstroke}%
\pgfsetdash{}{0pt}%
\pgfpathmoveto{\pgfqpoint{2.987836in}{0.663250in}}%
\pgfpathlineto{\pgfqpoint{3.144349in}{0.663250in}}%
\pgfusepath{stroke}%
\end{pgfscope}%
\begin{pgfscope}%
\pgfpathrectangle{\pgfqpoint{0.550713in}{0.127635in}}{\pgfqpoint{3.194133in}{2.297424in}}%
\pgfusepath{clip}%
\pgfsetbuttcap%
\pgfsetroundjoin%
\definecolor{currentfill}{rgb}{1.000000,1.000000,1.000000}%
\pgfsetfillcolor{currentfill}%
\pgfsetlinewidth{1.003750pt}%
\definecolor{currentstroke}{rgb}{0.000000,0.000000,0.000000}%
\pgfsetstrokecolor{currentstroke}%
\pgfsetdash{}{0pt}%
\pgfsys@defobject{currentmarker}{\pgfqpoint{-0.027778in}{-0.027778in}}{\pgfqpoint{0.027778in}{0.027778in}}{%
\pgfpathmoveto{\pgfqpoint{0.000000in}{-0.027778in}}%
\pgfpathcurveto{\pgfqpoint{0.007367in}{-0.027778in}}{\pgfqpoint{0.014433in}{-0.024851in}}{\pgfqpoint{0.019642in}{-0.019642in}}%
\pgfpathcurveto{\pgfqpoint{0.024851in}{-0.014433in}}{\pgfqpoint{0.027778in}{-0.007367in}}{\pgfqpoint{0.027778in}{0.000000in}}%
\pgfpathcurveto{\pgfqpoint{0.027778in}{0.007367in}}{\pgfqpoint{0.024851in}{0.014433in}}{\pgfqpoint{0.019642in}{0.019642in}}%
\pgfpathcurveto{\pgfqpoint{0.014433in}{0.024851in}}{\pgfqpoint{0.007367in}{0.027778in}}{\pgfqpoint{0.000000in}{0.027778in}}%
\pgfpathcurveto{\pgfqpoint{-0.007367in}{0.027778in}}{\pgfqpoint{-0.014433in}{0.024851in}}{\pgfqpoint{-0.019642in}{0.019642in}}%
\pgfpathcurveto{\pgfqpoint{-0.024851in}{0.014433in}}{\pgfqpoint{-0.027778in}{0.007367in}}{\pgfqpoint{-0.027778in}{0.000000in}}%
\pgfpathcurveto{\pgfqpoint{-0.027778in}{-0.007367in}}{\pgfqpoint{-0.024851in}{-0.014433in}}{\pgfqpoint{-0.019642in}{-0.019642in}}%
\pgfpathcurveto{\pgfqpoint{-0.014433in}{-0.024851in}}{\pgfqpoint{-0.007367in}{-0.027778in}}{\pgfqpoint{0.000000in}{-0.027778in}}%
\pgfpathclose%
\pgfusepath{stroke,fill}%
}%
\begin{pgfscope}%
\pgfsys@transformshift{3.066092in}{0.637011in}%
\pgfsys@useobject{currentmarker}{}%
\end{pgfscope}%
\end{pgfscope}%
\begin{pgfscope}%
\pgfpathrectangle{\pgfqpoint{0.550713in}{0.127635in}}{\pgfqpoint{3.194133in}{2.297424in}}%
\pgfusepath{clip}%
\pgfsetrectcap%
\pgfsetroundjoin%
\pgfsetlinewidth{0.752812pt}%
\definecolor{currentstroke}{rgb}{0.000000,0.000000,0.000000}%
\pgfsetstrokecolor{currentstroke}%
\pgfsetdash{}{0pt}%
\pgfpathmoveto{\pgfqpoint{3.147543in}{0.645569in}}%
\pgfpathlineto{\pgfqpoint{3.304055in}{0.645569in}}%
\pgfusepath{stroke}%
\end{pgfscope}%
\begin{pgfscope}%
\pgfpathrectangle{\pgfqpoint{0.550713in}{0.127635in}}{\pgfqpoint{3.194133in}{2.297424in}}%
\pgfusepath{clip}%
\pgfsetbuttcap%
\pgfsetroundjoin%
\definecolor{currentfill}{rgb}{1.000000,1.000000,1.000000}%
\pgfsetfillcolor{currentfill}%
\pgfsetlinewidth{1.003750pt}%
\definecolor{currentstroke}{rgb}{0.000000,0.000000,0.000000}%
\pgfsetstrokecolor{currentstroke}%
\pgfsetdash{}{0pt}%
\pgfsys@defobject{currentmarker}{\pgfqpoint{-0.027778in}{-0.027778in}}{\pgfqpoint{0.027778in}{0.027778in}}{%
\pgfpathmoveto{\pgfqpoint{0.000000in}{-0.027778in}}%
\pgfpathcurveto{\pgfqpoint{0.007367in}{-0.027778in}}{\pgfqpoint{0.014433in}{-0.024851in}}{\pgfqpoint{0.019642in}{-0.019642in}}%
\pgfpathcurveto{\pgfqpoint{0.024851in}{-0.014433in}}{\pgfqpoint{0.027778in}{-0.007367in}}{\pgfqpoint{0.027778in}{0.000000in}}%
\pgfpathcurveto{\pgfqpoint{0.027778in}{0.007367in}}{\pgfqpoint{0.024851in}{0.014433in}}{\pgfqpoint{0.019642in}{0.019642in}}%
\pgfpathcurveto{\pgfqpoint{0.014433in}{0.024851in}}{\pgfqpoint{0.007367in}{0.027778in}}{\pgfqpoint{0.000000in}{0.027778in}}%
\pgfpathcurveto{\pgfqpoint{-0.007367in}{0.027778in}}{\pgfqpoint{-0.014433in}{0.024851in}}{\pgfqpoint{-0.019642in}{0.019642in}}%
\pgfpathcurveto{\pgfqpoint{-0.024851in}{0.014433in}}{\pgfqpoint{-0.027778in}{0.007367in}}{\pgfqpoint{-0.027778in}{0.000000in}}%
\pgfpathcurveto{\pgfqpoint{-0.027778in}{-0.007367in}}{\pgfqpoint{-0.024851in}{-0.014433in}}{\pgfqpoint{-0.019642in}{-0.019642in}}%
\pgfpathcurveto{\pgfqpoint{-0.014433in}{-0.024851in}}{\pgfqpoint{-0.007367in}{-0.027778in}}{\pgfqpoint{0.000000in}{-0.027778in}}%
\pgfpathclose%
\pgfusepath{stroke,fill}%
}%
\begin{pgfscope}%
\pgfsys@transformshift{3.225799in}{0.639497in}%
\pgfsys@useobject{currentmarker}{}%
\end{pgfscope}%
\end{pgfscope}%
\begin{pgfscope}%
\pgfpathrectangle{\pgfqpoint{0.550713in}{0.127635in}}{\pgfqpoint{3.194133in}{2.297424in}}%
\pgfusepath{clip}%
\pgfsetrectcap%
\pgfsetroundjoin%
\pgfsetlinewidth{0.752812pt}%
\definecolor{currentstroke}{rgb}{0.000000,0.000000,0.000000}%
\pgfsetstrokecolor{currentstroke}%
\pgfsetdash{}{0pt}%
\pgfpathmoveto{\pgfqpoint{3.387103in}{0.655642in}}%
\pgfpathlineto{\pgfqpoint{3.543615in}{0.655642in}}%
\pgfusepath{stroke}%
\end{pgfscope}%
\begin{pgfscope}%
\pgfpathrectangle{\pgfqpoint{0.550713in}{0.127635in}}{\pgfqpoint{3.194133in}{2.297424in}}%
\pgfusepath{clip}%
\pgfsetbuttcap%
\pgfsetroundjoin%
\definecolor{currentfill}{rgb}{1.000000,1.000000,1.000000}%
\pgfsetfillcolor{currentfill}%
\pgfsetlinewidth{1.003750pt}%
\definecolor{currentstroke}{rgb}{0.000000,0.000000,0.000000}%
\pgfsetstrokecolor{currentstroke}%
\pgfsetdash{}{0pt}%
\pgfsys@defobject{currentmarker}{\pgfqpoint{-0.027778in}{-0.027778in}}{\pgfqpoint{0.027778in}{0.027778in}}{%
\pgfpathmoveto{\pgfqpoint{0.000000in}{-0.027778in}}%
\pgfpathcurveto{\pgfqpoint{0.007367in}{-0.027778in}}{\pgfqpoint{0.014433in}{-0.024851in}}{\pgfqpoint{0.019642in}{-0.019642in}}%
\pgfpathcurveto{\pgfqpoint{0.024851in}{-0.014433in}}{\pgfqpoint{0.027778in}{-0.007367in}}{\pgfqpoint{0.027778in}{0.000000in}}%
\pgfpathcurveto{\pgfqpoint{0.027778in}{0.007367in}}{\pgfqpoint{0.024851in}{0.014433in}}{\pgfqpoint{0.019642in}{0.019642in}}%
\pgfpathcurveto{\pgfqpoint{0.014433in}{0.024851in}}{\pgfqpoint{0.007367in}{0.027778in}}{\pgfqpoint{0.000000in}{0.027778in}}%
\pgfpathcurveto{\pgfqpoint{-0.007367in}{0.027778in}}{\pgfqpoint{-0.014433in}{0.024851in}}{\pgfqpoint{-0.019642in}{0.019642in}}%
\pgfpathcurveto{\pgfqpoint{-0.024851in}{0.014433in}}{\pgfqpoint{-0.027778in}{0.007367in}}{\pgfqpoint{-0.027778in}{0.000000in}}%
\pgfpathcurveto{\pgfqpoint{-0.027778in}{-0.007367in}}{\pgfqpoint{-0.024851in}{-0.014433in}}{\pgfqpoint{-0.019642in}{-0.019642in}}%
\pgfpathcurveto{\pgfqpoint{-0.014433in}{-0.024851in}}{\pgfqpoint{-0.007367in}{-0.027778in}}{\pgfqpoint{0.000000in}{-0.027778in}}%
\pgfpathclose%
\pgfusepath{stroke,fill}%
}%
\begin{pgfscope}%
\pgfsys@transformshift{3.465359in}{0.655332in}%
\pgfsys@useobject{currentmarker}{}%
\end{pgfscope}%
\end{pgfscope}%
\begin{pgfscope}%
\pgfpathrectangle{\pgfqpoint{0.550713in}{0.127635in}}{\pgfqpoint{3.194133in}{2.297424in}}%
\pgfusepath{clip}%
\pgfsetrectcap%
\pgfsetroundjoin%
\pgfsetlinewidth{0.752812pt}%
\definecolor{currentstroke}{rgb}{0.000000,0.000000,0.000000}%
\pgfsetstrokecolor{currentstroke}%
\pgfsetdash{}{0pt}%
\pgfpathmoveto{\pgfqpoint{3.546809in}{0.661830in}}%
\pgfpathlineto{\pgfqpoint{3.703322in}{0.661830in}}%
\pgfusepath{stroke}%
\end{pgfscope}%
\begin{pgfscope}%
\pgfpathrectangle{\pgfqpoint{0.550713in}{0.127635in}}{\pgfqpoint{3.194133in}{2.297424in}}%
\pgfusepath{clip}%
\pgfsetbuttcap%
\pgfsetroundjoin%
\definecolor{currentfill}{rgb}{1.000000,1.000000,1.000000}%
\pgfsetfillcolor{currentfill}%
\pgfsetlinewidth{1.003750pt}%
\definecolor{currentstroke}{rgb}{0.000000,0.000000,0.000000}%
\pgfsetstrokecolor{currentstroke}%
\pgfsetdash{}{0pt}%
\pgfsys@defobject{currentmarker}{\pgfqpoint{-0.027778in}{-0.027778in}}{\pgfqpoint{0.027778in}{0.027778in}}{%
\pgfpathmoveto{\pgfqpoint{0.000000in}{-0.027778in}}%
\pgfpathcurveto{\pgfqpoint{0.007367in}{-0.027778in}}{\pgfqpoint{0.014433in}{-0.024851in}}{\pgfqpoint{0.019642in}{-0.019642in}}%
\pgfpathcurveto{\pgfqpoint{0.024851in}{-0.014433in}}{\pgfqpoint{0.027778in}{-0.007367in}}{\pgfqpoint{0.027778in}{0.000000in}}%
\pgfpathcurveto{\pgfqpoint{0.027778in}{0.007367in}}{\pgfqpoint{0.024851in}{0.014433in}}{\pgfqpoint{0.019642in}{0.019642in}}%
\pgfpathcurveto{\pgfqpoint{0.014433in}{0.024851in}}{\pgfqpoint{0.007367in}{0.027778in}}{\pgfqpoint{0.000000in}{0.027778in}}%
\pgfpathcurveto{\pgfqpoint{-0.007367in}{0.027778in}}{\pgfqpoint{-0.014433in}{0.024851in}}{\pgfqpoint{-0.019642in}{0.019642in}}%
\pgfpathcurveto{\pgfqpoint{-0.024851in}{0.014433in}}{\pgfqpoint{-0.027778in}{0.007367in}}{\pgfqpoint{-0.027778in}{0.000000in}}%
\pgfpathcurveto{\pgfqpoint{-0.027778in}{-0.007367in}}{\pgfqpoint{-0.024851in}{-0.014433in}}{\pgfqpoint{-0.019642in}{-0.019642in}}%
\pgfpathcurveto{\pgfqpoint{-0.014433in}{-0.024851in}}{\pgfqpoint{-0.007367in}{-0.027778in}}{\pgfqpoint{0.000000in}{-0.027778in}}%
\pgfpathclose%
\pgfusepath{stroke,fill}%
}%
\begin{pgfscope}%
\pgfsys@transformshift{3.625066in}{0.659661in}%
\pgfsys@useobject{currentmarker}{}%
\end{pgfscope}%
\end{pgfscope}%
\begin{pgfscope}%
\pgfsetrectcap%
\pgfsetmiterjoin%
\pgfsetlinewidth{0.752812pt}%
\definecolor{currentstroke}{rgb}{0.000000,0.000000,0.000000}%
\pgfsetstrokecolor{currentstroke}%
\pgfsetdash{}{0pt}%
\pgfpathmoveto{\pgfqpoint{0.550713in}{0.127635in}}%
\pgfpathlineto{\pgfqpoint{0.550713in}{2.425059in}}%
\pgfusepath{stroke}%
\end{pgfscope}%
\begin{pgfscope}%
\pgfsetrectcap%
\pgfsetmiterjoin%
\pgfsetlinewidth{0.752812pt}%
\definecolor{currentstroke}{rgb}{0.000000,0.000000,0.000000}%
\pgfsetstrokecolor{currentstroke}%
\pgfsetdash{}{0pt}%
\pgfpathmoveto{\pgfqpoint{3.744846in}{0.127635in}}%
\pgfpathlineto{\pgfqpoint{3.744846in}{2.425059in}}%
\pgfusepath{stroke}%
\end{pgfscope}%
\begin{pgfscope}%
\pgfsetrectcap%
\pgfsetmiterjoin%
\pgfsetlinewidth{0.752812pt}%
\definecolor{currentstroke}{rgb}{0.000000,0.000000,0.000000}%
\pgfsetstrokecolor{currentstroke}%
\pgfsetdash{}{0pt}%
\pgfpathmoveto{\pgfqpoint{0.550713in}{0.127635in}}%
\pgfpathlineto{\pgfqpoint{3.744846in}{0.127635in}}%
\pgfusepath{stroke}%
\end{pgfscope}%
\begin{pgfscope}%
\pgfsetrectcap%
\pgfsetmiterjoin%
\pgfsetlinewidth{0.752812pt}%
\definecolor{currentstroke}{rgb}{0.000000,0.000000,0.000000}%
\pgfsetstrokecolor{currentstroke}%
\pgfsetdash{}{0pt}%
\pgfpathmoveto{\pgfqpoint{0.550713in}{2.425059in}}%
\pgfpathlineto{\pgfqpoint{3.744846in}{2.425059in}}%
\pgfusepath{stroke}%
\end{pgfscope}%
\begin{pgfscope}%
\pgfsetbuttcap%
\pgfsetroundjoin%
\pgfsetlinewidth{1.003750pt}%
\definecolor{currentstroke}{rgb}{0.392157,0.396078,0.403922}%
\pgfsetstrokecolor{currentstroke}%
\pgfsetdash{{3.700000pt}{1.600000pt}}{0.000000pt}%
\pgfpathmoveto{\pgfqpoint{3.869846in}{2.064679in}}%
\pgfpathlineto{\pgfqpoint{4.147623in}{2.064679in}}%
\pgfusepath{stroke}%
\end{pgfscope}%
\begin{pgfscope}%
\definecolor{textcolor}{rgb}{0.000000,0.000000,0.000000}%
\pgfsetstrokecolor{textcolor}%
\pgfsetfillcolor{textcolor}%
\pgftext[x=4.258735in,y=2.016068in,left,base]{\color{textcolor}\rmfamily\fontsize{10.000000}{12.000000}\selectfont Only Exploitation}%
\end{pgfscope}%
\begin{pgfscope}%
\pgfsetbuttcap%
\pgfsetmiterjoin%
\definecolor{currentfill}{rgb}{0.631373,0.062745,0.207843}%
\pgfsetfillcolor{currentfill}%
\pgfsetlinewidth{0.000000pt}%
\definecolor{currentstroke}{rgb}{0.000000,0.000000,0.000000}%
\pgfsetstrokecolor{currentstroke}%
\pgfsetstrokeopacity{0.000000}%
\pgfsetdash{}{0pt}%
\pgfpathmoveto{\pgfqpoint{3.869846in}{1.820930in}}%
\pgfpathlineto{\pgfqpoint{4.147623in}{1.820930in}}%
\pgfpathlineto{\pgfqpoint{4.147623in}{1.918152in}}%
\pgfpathlineto{\pgfqpoint{3.869846in}{1.918152in}}%
\pgfpathclose%
\pgfusepath{fill}%
\end{pgfscope}%
\begin{pgfscope}%
\definecolor{textcolor}{rgb}{0.000000,0.000000,0.000000}%
\pgfsetstrokecolor{textcolor}%
\pgfsetfillcolor{textcolor}%
\pgftext[x=4.258735in,y=1.820930in,left,base]{\color{textcolor}\rmfamily\fontsize{10.000000}{12.000000}\selectfont TV-GP-UCB}%
\end{pgfscope}%
\begin{pgfscope}%
\pgfsetbuttcap%
\pgfsetmiterjoin%
\definecolor{currentfill}{rgb}{0.890196,0.000000,0.400000}%
\pgfsetfillcolor{currentfill}%
\pgfsetlinewidth{0.000000pt}%
\definecolor{currentstroke}{rgb}{0.000000,0.000000,0.000000}%
\pgfsetstrokecolor{currentstroke}%
\pgfsetstrokeopacity{0.000000}%
\pgfsetdash{}{0pt}%
\pgfpathmoveto{\pgfqpoint{3.869846in}{1.627319in}}%
\pgfpathlineto{\pgfqpoint{4.147623in}{1.627319in}}%
\pgfpathlineto{\pgfqpoint{4.147623in}{1.724541in}}%
\pgfpathlineto{\pgfqpoint{3.869846in}{1.724541in}}%
\pgfpathclose%
\pgfusepath{fill}%
\end{pgfscope}%
\begin{pgfscope}%
\definecolor{textcolor}{rgb}{0.000000,0.000000,0.000000}%
\pgfsetstrokecolor{textcolor}%
\pgfsetfillcolor{textcolor}%
\pgftext[x=4.258735in,y=1.627319in,left,base]{\color{textcolor}\rmfamily\fontsize{10.000000}{12.000000}\selectfont SW TV-GP-UCB}%
\end{pgfscope}%
\begin{pgfscope}%
\pgfsetbuttcap%
\pgfsetmiterjoin%
\definecolor{currentfill}{rgb}{0.000000,0.329412,0.623529}%
\pgfsetfillcolor{currentfill}%
\pgfsetlinewidth{0.000000pt}%
\definecolor{currentstroke}{rgb}{0.000000,0.000000,0.000000}%
\pgfsetstrokecolor{currentstroke}%
\pgfsetstrokeopacity{0.000000}%
\pgfsetdash{}{0pt}%
\pgfpathmoveto{\pgfqpoint{3.869846in}{1.433708in}}%
\pgfpathlineto{\pgfqpoint{4.147623in}{1.433708in}}%
\pgfpathlineto{\pgfqpoint{4.147623in}{1.530930in}}%
\pgfpathlineto{\pgfqpoint{3.869846in}{1.530930in}}%
\pgfpathclose%
\pgfusepath{fill}%
\end{pgfscope}%
\begin{pgfscope}%
\definecolor{textcolor}{rgb}{0.000000,0.000000,0.000000}%
\pgfsetstrokecolor{textcolor}%
\pgfsetfillcolor{textcolor}%
\pgftext[x=4.258735in,y=1.433708in,left,base]{\color{textcolor}\rmfamily\fontsize{10.000000}{12.000000}\selectfont UI-TVBO}%
\end{pgfscope}%
\begin{pgfscope}%
\pgfsetbuttcap%
\pgfsetmiterjoin%
\definecolor{currentfill}{rgb}{0.000000,0.380392,0.396078}%
\pgfsetfillcolor{currentfill}%
\pgfsetlinewidth{0.000000pt}%
\definecolor{currentstroke}{rgb}{0.000000,0.000000,0.000000}%
\pgfsetstrokecolor{currentstroke}%
\pgfsetstrokeopacity{0.000000}%
\pgfsetdash{}{0pt}%
\pgfpathmoveto{\pgfqpoint{3.869846in}{1.240097in}}%
\pgfpathlineto{\pgfqpoint{4.147623in}{1.240097in}}%
\pgfpathlineto{\pgfqpoint{4.147623in}{1.337319in}}%
\pgfpathlineto{\pgfqpoint{3.869846in}{1.337319in}}%
\pgfpathclose%
\pgfusepath{fill}%
\end{pgfscope}%
\begin{pgfscope}%
\definecolor{textcolor}{rgb}{0.000000,0.000000,0.000000}%
\pgfsetstrokecolor{textcolor}%
\pgfsetfillcolor{textcolor}%
\pgftext[x=4.258735in,y=1.240097in,left,base]{\color{textcolor}\rmfamily\fontsize{10.000000}{12.000000}\selectfont B UI-TVBO}%
\end{pgfscope}%
\begin{pgfscope}%
\pgfsetbuttcap%
\pgfsetmiterjoin%
\definecolor{currentfill}{rgb}{0.380392,0.129412,0.345098}%
\pgfsetfillcolor{currentfill}%
\pgfsetlinewidth{0.000000pt}%
\definecolor{currentstroke}{rgb}{0.000000,0.000000,0.000000}%
\pgfsetstrokecolor{currentstroke}%
\pgfsetstrokeopacity{0.000000}%
\pgfsetdash{}{0pt}%
\pgfpathmoveto{\pgfqpoint{3.869846in}{1.046486in}}%
\pgfpathlineto{\pgfqpoint{4.147623in}{1.046486in}}%
\pgfpathlineto{\pgfqpoint{4.147623in}{1.143708in}}%
\pgfpathlineto{\pgfqpoint{3.869846in}{1.143708in}}%
\pgfpathclose%
\pgfusepath{fill}%
\end{pgfscope}%
\begin{pgfscope}%
\definecolor{textcolor}{rgb}{0.000000,0.000000,0.000000}%
\pgfsetstrokecolor{textcolor}%
\pgfsetfillcolor{textcolor}%
\pgftext[x=4.258735in,y=1.046486in,left,base]{\color{textcolor}\rmfamily\fontsize{10.000000}{12.000000}\selectfont C-TV-GP-UCB}%
\end{pgfscope}%
\begin{pgfscope}%
\pgfsetbuttcap%
\pgfsetmiterjoin%
\definecolor{currentfill}{rgb}{0.964706,0.658824,0.000000}%
\pgfsetfillcolor{currentfill}%
\pgfsetlinewidth{0.000000pt}%
\definecolor{currentstroke}{rgb}{0.000000,0.000000,0.000000}%
\pgfsetstrokecolor{currentstroke}%
\pgfsetstrokeopacity{0.000000}%
\pgfsetdash{}{0pt}%
\pgfpathmoveto{\pgfqpoint{3.869846in}{0.852875in}}%
\pgfpathlineto{\pgfqpoint{4.147623in}{0.852875in}}%
\pgfpathlineto{\pgfqpoint{4.147623in}{0.950097in}}%
\pgfpathlineto{\pgfqpoint{3.869846in}{0.950097in}}%
\pgfpathclose%
\pgfusepath{fill}%
\end{pgfscope}%
\begin{pgfscope}%
\definecolor{textcolor}{rgb}{0.000000,0.000000,0.000000}%
\pgfsetstrokecolor{textcolor}%
\pgfsetfillcolor{textcolor}%
\pgftext[x=4.258735in,y=0.852875in,left,base]{\color{textcolor}\rmfamily\fontsize{10.000000}{12.000000}\selectfont SW C-TV-GP-UCB}%
\end{pgfscope}%
\begin{pgfscope}%
\pgfsetbuttcap%
\pgfsetmiterjoin%
\definecolor{currentfill}{rgb}{0.341176,0.670588,0.152941}%
\pgfsetfillcolor{currentfill}%
\pgfsetlinewidth{0.000000pt}%
\definecolor{currentstroke}{rgb}{0.000000,0.000000,0.000000}%
\pgfsetstrokecolor{currentstroke}%
\pgfsetstrokeopacity{0.000000}%
\pgfsetdash{}{0pt}%
\pgfpathmoveto{\pgfqpoint{3.869846in}{0.659264in}}%
\pgfpathlineto{\pgfqpoint{4.147623in}{0.659264in}}%
\pgfpathlineto{\pgfqpoint{4.147623in}{0.756486in}}%
\pgfpathlineto{\pgfqpoint{3.869846in}{0.756486in}}%
\pgfpathclose%
\pgfusepath{fill}%
\end{pgfscope}%
\begin{pgfscope}%
\definecolor{textcolor}{rgb}{0.000000,0.000000,0.000000}%
\pgfsetstrokecolor{textcolor}%
\pgfsetfillcolor{textcolor}%
\pgftext[x=4.258735in,y=0.659264in,left,base]{\color{textcolor}\rmfamily\fontsize{10.000000}{12.000000}\selectfont C-UI-TVBO}%
\end{pgfscope}%
\begin{pgfscope}%
\pgfsetbuttcap%
\pgfsetmiterjoin%
\definecolor{currentfill}{rgb}{0.478431,0.435294,0.674510}%
\pgfsetfillcolor{currentfill}%
\pgfsetlinewidth{0.000000pt}%
\definecolor{currentstroke}{rgb}{0.000000,0.000000,0.000000}%
\pgfsetstrokecolor{currentstroke}%
\pgfsetstrokeopacity{0.000000}%
\pgfsetdash{}{0pt}%
\pgfpathmoveto{\pgfqpoint{3.869846in}{0.465653in}}%
\pgfpathlineto{\pgfqpoint{4.147623in}{0.465653in}}%
\pgfpathlineto{\pgfqpoint{4.147623in}{0.562875in}}%
\pgfpathlineto{\pgfqpoint{3.869846in}{0.562875in}}%
\pgfpathclose%
\pgfusepath{fill}%
\end{pgfscope}%
\begin{pgfscope}%
\definecolor{textcolor}{rgb}{0.000000,0.000000,0.000000}%
\pgfsetstrokecolor{textcolor}%
\pgfsetfillcolor{textcolor}%
\pgftext[x=4.258735in,y=0.465653in,left,base]{\color{textcolor}\rmfamily\fontsize{10.000000}{12.000000}\selectfont B C-UI-TVBO}%
\end{pgfscope}%
\begin{pgfscope}%
\pgfsetbuttcap%
\pgfsetmiterjoin%
\definecolor{currentfill}{rgb}{1.000000,1.000000,1.000000}%
\pgfsetfillcolor{currentfill}%
\pgfsetlinewidth{1.003750pt}%
\definecolor{currentstroke}{rgb}{1.000000,1.000000,1.000000}%
\pgfsetstrokecolor{currentstroke}%
\pgfsetdash{}{0pt}%
\pgfpathmoveto{\pgfqpoint{2.736198in}{1.968269in}}%
\pgfpathlineto{\pgfqpoint{3.689290in}{1.968269in}}%
\pgfpathquadraticcurveto{\pgfqpoint{3.717068in}{1.968269in}}{\pgfqpoint{3.717068in}{1.996046in}}%
\pgfpathlineto{\pgfqpoint{3.717068in}{2.369503in}}%
\pgfpathquadraticcurveto{\pgfqpoint{3.717068in}{2.397281in}}{\pgfqpoint{3.689290in}{2.397281in}}%
\pgfpathlineto{\pgfqpoint{2.736198in}{2.397281in}}%
\pgfpathquadraticcurveto{\pgfqpoint{2.708420in}{2.397281in}}{\pgfqpoint{2.708420in}{2.369503in}}%
\pgfpathlineto{\pgfqpoint{2.708420in}{1.996046in}}%
\pgfpathquadraticcurveto{\pgfqpoint{2.708420in}{1.968269in}}{\pgfqpoint{2.736198in}{1.968269in}}%
\pgfpathclose%
\pgfusepath{stroke,fill}%
\end{pgfscope}%
\begin{pgfscope}%
\pgfsetbuttcap%
\pgfsetmiterjoin%
\definecolor{currentfill}{rgb}{0.000000,0.000000,0.000000}%
\pgfsetfillcolor{currentfill}%
\pgfsetlinewidth{0.000000pt}%
\definecolor{currentstroke}{rgb}{0.000000,0.000000,0.000000}%
\pgfsetstrokecolor{currentstroke}%
\pgfsetstrokeopacity{0.000000}%
\pgfsetdash{}{0pt}%
\pgfpathmoveto{\pgfqpoint{2.763976in}{2.244503in}}%
\pgfpathlineto{\pgfqpoint{3.041753in}{2.244503in}}%
\pgfpathlineto{\pgfqpoint{3.041753in}{2.341725in}}%
\pgfpathlineto{\pgfqpoint{2.763976in}{2.341725in}}%
\pgfpathclose%
\pgfusepath{fill}%
\end{pgfscope}%
\begin{pgfscope}%
\definecolor{textcolor}{rgb}{0.000000,0.000000,0.000000}%
\pgfsetstrokecolor{textcolor}%
\pgfsetfillcolor{textcolor}%
\pgftext[x=3.152864in,y=2.244503in,left,base]{\color{textcolor}\rmfamily\fontsize{10.000000}{12.000000}\selectfont \(\displaystyle \mu_0=0\)}%
\end{pgfscope}%
\begin{pgfscope}%
\pgfsetbuttcap%
\pgfsetmiterjoin%
\definecolor{currentfill}{rgb}{0.811765,0.819608,0.823529}%
\pgfsetfillcolor{currentfill}%
\pgfsetlinewidth{0.000000pt}%
\definecolor{currentstroke}{rgb}{0.000000,0.000000,0.000000}%
\pgfsetstrokecolor{currentstroke}%
\pgfsetstrokeopacity{0.000000}%
\pgfsetdash{}{0pt}%
\pgfpathmoveto{\pgfqpoint{2.763976in}{2.050830in}}%
\pgfpathlineto{\pgfqpoint{3.041753in}{2.050830in}}%
\pgfpathlineto{\pgfqpoint{3.041753in}{2.148053in}}%
\pgfpathlineto{\pgfqpoint{2.763976in}{2.148053in}}%
\pgfpathclose%
\pgfusepath{fill}%
\end{pgfscope}%
\begin{pgfscope}%
\definecolor{textcolor}{rgb}{0.000000,0.000000,0.000000}%
\pgfsetstrokecolor{textcolor}%
\pgfsetfillcolor{textcolor}%
\pgftext[x=3.152864in,y=2.050830in,left,base]{\color{textcolor}\rmfamily\fontsize{10.000000}{12.000000}\selectfont \(\displaystyle \mu_0=-2\)}%
\end{pgfscope}%
\begin{pgfscope}%
\pgfsetbuttcap%
\pgfsetmiterjoin%
\definecolor{currentfill}{rgb}{1.000000,1.000000,1.000000}%
\pgfsetfillcolor{currentfill}%
\pgfsetlinewidth{1.003750pt}%
\definecolor{currentstroke}{rgb}{1.000000,1.000000,1.000000}%
\pgfsetstrokecolor{currentstroke}%
\pgfsetdash{}{0pt}%
\pgfpathmoveto{\pgfqpoint{2.736198in}{1.968269in}}%
\pgfpathlineto{\pgfqpoint{3.689290in}{1.968269in}}%
\pgfpathquadraticcurveto{\pgfqpoint{3.717068in}{1.968269in}}{\pgfqpoint{3.717068in}{1.996046in}}%
\pgfpathlineto{\pgfqpoint{3.717068in}{2.369503in}}%
\pgfpathquadraticcurveto{\pgfqpoint{3.717068in}{2.397281in}}{\pgfqpoint{3.689290in}{2.397281in}}%
\pgfpathlineto{\pgfqpoint{2.736198in}{2.397281in}}%
\pgfpathquadraticcurveto{\pgfqpoint{2.708420in}{2.397281in}}{\pgfqpoint{2.708420in}{2.369503in}}%
\pgfpathlineto{\pgfqpoint{2.708420in}{1.996046in}}%
\pgfpathquadraticcurveto{\pgfqpoint{2.708420in}{1.968269in}}{\pgfqpoint{2.736198in}{1.968269in}}%
\pgfpathclose%
\pgfusepath{stroke,fill}%
\end{pgfscope}%
\begin{pgfscope}%
\pgfsetbuttcap%
\pgfsetmiterjoin%
\definecolor{currentfill}{rgb}{0.000000,0.000000,0.000000}%
\pgfsetfillcolor{currentfill}%
\pgfsetlinewidth{0.000000pt}%
\definecolor{currentstroke}{rgb}{0.000000,0.000000,0.000000}%
\pgfsetstrokecolor{currentstroke}%
\pgfsetstrokeopacity{0.000000}%
\pgfsetdash{}{0pt}%
\pgfpathmoveto{\pgfqpoint{2.763976in}{2.244503in}}%
\pgfpathlineto{\pgfqpoint{3.041753in}{2.244503in}}%
\pgfpathlineto{\pgfqpoint{3.041753in}{2.341725in}}%
\pgfpathlineto{\pgfqpoint{2.763976in}{2.341725in}}%
\pgfpathclose%
\pgfusepath{fill}%
\end{pgfscope}%
\begin{pgfscope}%
\definecolor{textcolor}{rgb}{0.000000,0.000000,0.000000}%
\pgfsetstrokecolor{textcolor}%
\pgfsetfillcolor{textcolor}%
\pgftext[x=3.152864in,y=2.244503in,left,base]{\color{textcolor}\rmfamily\fontsize{10.000000}{12.000000}\selectfont \(\displaystyle \mu_0=0\)}%
\end{pgfscope}%
\begin{pgfscope}%
\pgfsetbuttcap%
\pgfsetmiterjoin%
\definecolor{currentfill}{rgb}{0.811765,0.819608,0.823529}%
\pgfsetfillcolor{currentfill}%
\pgfsetlinewidth{0.000000pt}%
\definecolor{currentstroke}{rgb}{0.000000,0.000000,0.000000}%
\pgfsetstrokecolor{currentstroke}%
\pgfsetstrokeopacity{0.000000}%
\pgfsetdash{}{0pt}%
\pgfpathmoveto{\pgfqpoint{2.763976in}{2.050830in}}%
\pgfpathlineto{\pgfqpoint{3.041753in}{2.050830in}}%
\pgfpathlineto{\pgfqpoint{3.041753in}{2.148053in}}%
\pgfpathlineto{\pgfqpoint{2.763976in}{2.148053in}}%
\pgfpathclose%
\pgfusepath{fill}%
\end{pgfscope}%
\begin{pgfscope}%
\definecolor{textcolor}{rgb}{0.000000,0.000000,0.000000}%
\pgfsetstrokecolor{textcolor}%
\pgfsetfillcolor{textcolor}%
\pgftext[x=3.152864in,y=2.050830in,left,base]{\color{textcolor}\rmfamily\fontsize{10.000000}{12.000000}\selectfont \(\displaystyle \mu_0=-2\)}%
\end{pgfscope}%
\begin{pgfscope}%
\pgfpathrectangle{\pgfqpoint{0.550713in}{0.127635in}}{\pgfqpoint{3.194133in}{2.297424in}}%
\pgfusepath{clip}%
\pgfsetbuttcap%
\pgfsetmiterjoin%
\definecolor{currentfill}{rgb}{1.000000,1.000000,1.000000}%
\pgfsetfillcolor{currentfill}%
\pgfsetlinewidth{0.000000pt}%
\definecolor{currentstroke}{rgb}{0.000000,0.000000,0.000000}%
\pgfsetstrokecolor{currentstroke}%
\pgfsetstrokeopacity{0.000000}%
\pgfsetdash{}{0pt}%
\pgfpathmoveto{\pgfqpoint{2.227632in}{1.735831in}}%
\pgfpathlineto{\pgfqpoint{3.704919in}{1.735831in}}%
\pgfpathlineto{\pgfqpoint{3.704919in}{1.965574in}}%
\pgfpathlineto{\pgfqpoint{2.227632in}{1.965574in}}%
\pgfpathclose%
\pgfusepath{fill}%
\end{pgfscope}%
\begin{pgfscope}%
\pgfsetbuttcap%
\pgfsetmiterjoin%
\definecolor{currentfill}{rgb}{1.000000,1.000000,1.000000}%
\pgfsetfillcolor{currentfill}%
\pgfsetlinewidth{0.000000pt}%
\definecolor{currentstroke}{rgb}{0.000000,0.000000,0.000000}%
\pgfsetstrokecolor{currentstroke}%
\pgfsetstrokeopacity{0.000000}%
\pgfsetdash{}{0pt}%
\pgfpathmoveto{\pgfqpoint{2.243603in}{0.862810in}}%
\pgfpathlineto{\pgfqpoint{3.649022in}{0.862810in}}%
\pgfpathlineto{\pgfqpoint{3.649022in}{1.781780in}}%
\pgfpathlineto{\pgfqpoint{2.243603in}{1.781780in}}%
\pgfpathclose%
\pgfusepath{fill}%
\end{pgfscope}%
\begin{pgfscope}%
\pgfpathrectangle{\pgfqpoint{2.243603in}{0.862810in}}{\pgfqpoint{1.405419in}{0.918970in}}%
\pgfusepath{clip}%
\pgfsetbuttcap%
\pgfsetmiterjoin%
\definecolor{currentfill}{rgb}{0.380392,0.129412,0.345098}%
\pgfsetfillcolor{currentfill}%
\pgfsetlinewidth{0.752812pt}%
\definecolor{currentstroke}{rgb}{0.000000,0.000000,0.000000}%
\pgfsetstrokecolor{currentstroke}%
\pgfsetdash{}{0pt}%
\pgfpathmoveto{\pgfqpoint{2.280144in}{1.271781in}}%
\pgfpathlineto{\pgfqpoint{2.417875in}{1.271781in}}%
\pgfpathlineto{\pgfqpoint{2.417875in}{1.451084in}}%
\pgfpathlineto{\pgfqpoint{2.280144in}{1.451084in}}%
\pgfpathlineto{\pgfqpoint{2.280144in}{1.271781in}}%
\pgfpathclose%
\pgfusepath{stroke,fill}%
\end{pgfscope}%
\begin{pgfscope}%
\pgfpathrectangle{\pgfqpoint{2.243603in}{0.862810in}}{\pgfqpoint{1.405419in}{0.918970in}}%
\pgfusepath{clip}%
\pgfsetbuttcap%
\pgfsetmiterjoin%
\definecolor{currentfill}{rgb}{0.823529,0.752941,0.803922}%
\pgfsetfillcolor{currentfill}%
\pgfsetlinewidth{0.752812pt}%
\definecolor{currentstroke}{rgb}{0.000000,0.000000,0.000000}%
\pgfsetstrokecolor{currentstroke}%
\pgfsetdash{}{0pt}%
\pgfpathmoveto{\pgfqpoint{2.420686in}{1.508771in}}%
\pgfpathlineto{\pgfqpoint{2.558417in}{1.508771in}}%
\pgfpathlineto{\pgfqpoint{2.558417in}{1.608299in}}%
\pgfpathlineto{\pgfqpoint{2.420686in}{1.608299in}}%
\pgfpathlineto{\pgfqpoint{2.420686in}{1.508771in}}%
\pgfpathclose%
\pgfusepath{stroke,fill}%
\end{pgfscope}%
\begin{pgfscope}%
\pgfpathrectangle{\pgfqpoint{2.243603in}{0.862810in}}{\pgfqpoint{1.405419in}{0.918970in}}%
\pgfusepath{clip}%
\pgfsetbuttcap%
\pgfsetmiterjoin%
\definecolor{currentfill}{rgb}{0.964706,0.658824,0.000000}%
\pgfsetfillcolor{currentfill}%
\pgfsetlinewidth{0.752812pt}%
\definecolor{currentstroke}{rgb}{0.000000,0.000000,0.000000}%
\pgfsetstrokecolor{currentstroke}%
\pgfsetdash{}{0pt}%
\pgfpathmoveto{\pgfqpoint{2.631499in}{1.321786in}}%
\pgfpathlineto{\pgfqpoint{2.769230in}{1.321786in}}%
\pgfpathlineto{\pgfqpoint{2.769230in}{1.425620in}}%
\pgfpathlineto{\pgfqpoint{2.631499in}{1.425620in}}%
\pgfpathlineto{\pgfqpoint{2.631499in}{1.321786in}}%
\pgfpathclose%
\pgfusepath{stroke,fill}%
\end{pgfscope}%
\begin{pgfscope}%
\pgfpathrectangle{\pgfqpoint{2.243603in}{0.862810in}}{\pgfqpoint{1.405419in}{0.918970in}}%
\pgfusepath{clip}%
\pgfsetbuttcap%
\pgfsetmiterjoin%
\definecolor{currentfill}{rgb}{0.996078,0.917647,0.788235}%
\pgfsetfillcolor{currentfill}%
\pgfsetlinewidth{0.752812pt}%
\definecolor{currentstroke}{rgb}{0.000000,0.000000,0.000000}%
\pgfsetstrokecolor{currentstroke}%
\pgfsetdash{}{0pt}%
\pgfpathmoveto{\pgfqpoint{2.772041in}{1.520985in}}%
\pgfpathlineto{\pgfqpoint{2.909772in}{1.520985in}}%
\pgfpathlineto{\pgfqpoint{2.909772in}{1.656123in}}%
\pgfpathlineto{\pgfqpoint{2.772041in}{1.656123in}}%
\pgfpathlineto{\pgfqpoint{2.772041in}{1.520985in}}%
\pgfpathclose%
\pgfusepath{stroke,fill}%
\end{pgfscope}%
\begin{pgfscope}%
\pgfpathrectangle{\pgfqpoint{2.243603in}{0.862810in}}{\pgfqpoint{1.405419in}{0.918970in}}%
\pgfusepath{clip}%
\pgfsetbuttcap%
\pgfsetmiterjoin%
\definecolor{currentfill}{rgb}{0.341176,0.670588,0.152941}%
\pgfsetfillcolor{currentfill}%
\pgfsetlinewidth{0.752812pt}%
\definecolor{currentstroke}{rgb}{0.000000,0.000000,0.000000}%
\pgfsetstrokecolor{currentstroke}%
\pgfsetdash{}{0pt}%
\pgfpathmoveto{\pgfqpoint{2.982853in}{1.146235in}}%
\pgfpathlineto{\pgfqpoint{3.120584in}{1.146235in}}%
\pgfpathlineto{\pgfqpoint{3.120584in}{1.466762in}}%
\pgfpathlineto{\pgfqpoint{2.982853in}{1.466762in}}%
\pgfpathlineto{\pgfqpoint{2.982853in}{1.146235in}}%
\pgfpathclose%
\pgfusepath{stroke,fill}%
\end{pgfscope}%
\begin{pgfscope}%
\pgfpathrectangle{\pgfqpoint{2.243603in}{0.862810in}}{\pgfqpoint{1.405419in}{0.918970in}}%
\pgfusepath{clip}%
\pgfsetbuttcap%
\pgfsetmiterjoin%
\definecolor{currentfill}{rgb}{0.866667,0.921569,0.807843}%
\pgfsetfillcolor{currentfill}%
\pgfsetlinewidth{0.752812pt}%
\definecolor{currentstroke}{rgb}{0.000000,0.000000,0.000000}%
\pgfsetstrokecolor{currentstroke}%
\pgfsetdash{}{0pt}%
\pgfpathmoveto{\pgfqpoint{3.123395in}{1.183084in}}%
\pgfpathlineto{\pgfqpoint{3.261126in}{1.183084in}}%
\pgfpathlineto{\pgfqpoint{3.261126in}{1.423842in}}%
\pgfpathlineto{\pgfqpoint{3.123395in}{1.423842in}}%
\pgfpathlineto{\pgfqpoint{3.123395in}{1.183084in}}%
\pgfpathclose%
\pgfusepath{stroke,fill}%
\end{pgfscope}%
\begin{pgfscope}%
\pgfpathrectangle{\pgfqpoint{2.243603in}{0.862810in}}{\pgfqpoint{1.405419in}{0.918970in}}%
\pgfusepath{clip}%
\pgfsetbuttcap%
\pgfsetmiterjoin%
\definecolor{currentfill}{rgb}{0.478431,0.435294,0.674510}%
\pgfsetfillcolor{currentfill}%
\pgfsetlinewidth{0.752812pt}%
\definecolor{currentstroke}{rgb}{0.000000,0.000000,0.000000}%
\pgfsetstrokecolor{currentstroke}%
\pgfsetdash{}{0pt}%
\pgfpathmoveto{\pgfqpoint{3.334208in}{1.237065in}}%
\pgfpathlineto{\pgfqpoint{3.471939in}{1.237065in}}%
\pgfpathlineto{\pgfqpoint{3.471939in}{1.479273in}}%
\pgfpathlineto{\pgfqpoint{3.334208in}{1.479273in}}%
\pgfpathlineto{\pgfqpoint{3.334208in}{1.237065in}}%
\pgfpathclose%
\pgfusepath{stroke,fill}%
\end{pgfscope}%
\begin{pgfscope}%
\pgfpathrectangle{\pgfqpoint{2.243603in}{0.862810in}}{\pgfqpoint{1.405419in}{0.918970in}}%
\pgfusepath{clip}%
\pgfsetbuttcap%
\pgfsetmiterjoin%
\definecolor{currentfill}{rgb}{0.870588,0.854902,0.921569}%
\pgfsetfillcolor{currentfill}%
\pgfsetlinewidth{0.752812pt}%
\definecolor{currentstroke}{rgb}{0.000000,0.000000,0.000000}%
\pgfsetstrokecolor{currentstroke}%
\pgfsetdash{}{0pt}%
\pgfpathmoveto{\pgfqpoint{3.474750in}{1.267406in}}%
\pgfpathlineto{\pgfqpoint{3.612481in}{1.267406in}}%
\pgfpathlineto{\pgfqpoint{3.612481in}{1.445547in}}%
\pgfpathlineto{\pgfqpoint{3.474750in}{1.445547in}}%
\pgfpathlineto{\pgfqpoint{3.474750in}{1.267406in}}%
\pgfpathclose%
\pgfusepath{stroke,fill}%
\end{pgfscope}%
\begin{pgfscope}%
\pgfpathrectangle{\pgfqpoint{2.243603in}{0.862810in}}{\pgfqpoint{1.405419in}{0.918970in}}%
\pgfusepath{clip}%
\pgfsetbuttcap%
\pgfsetmiterjoin%
\definecolor{currentfill}{rgb}{0.000000,0.000000,0.000000}%
\pgfsetfillcolor{currentfill}%
\pgfsetlinewidth{0.376406pt}%
\definecolor{currentstroke}{rgb}{0.000000,0.000000,0.000000}%
\pgfsetstrokecolor{currentstroke}%
\pgfsetdash{}{0pt}%
\pgfpathmoveto{\pgfqpoint{2.419280in}{-3729.188002in}}%
\pgfpathlineto{\pgfqpoint{2.419280in}{-3729.188002in}}%
\pgfpathlineto{\pgfqpoint{2.419280in}{-3729.188002in}}%
\pgfpathlineto{\pgfqpoint{2.419280in}{-3729.188002in}}%
\pgfpathclose%
\pgfusepath{stroke,fill}%
\end{pgfscope}%
\begin{pgfscope}%
\pgfpathrectangle{\pgfqpoint{2.243603in}{0.862810in}}{\pgfqpoint{1.405419in}{0.918970in}}%
\pgfusepath{clip}%
\pgfsetbuttcap%
\pgfsetmiterjoin%
\definecolor{currentfill}{rgb}{0.813235,0.819118,0.822059}%
\pgfsetfillcolor{currentfill}%
\pgfsetlinewidth{0.376406pt}%
\definecolor{currentstroke}{rgb}{0.000000,0.000000,0.000000}%
\pgfsetstrokecolor{currentstroke}%
\pgfsetdash{}{0pt}%
\pgfpathmoveto{\pgfqpoint{2.419280in}{-3729.188002in}}%
\pgfpathlineto{\pgfqpoint{2.419280in}{-3729.188002in}}%
\pgfpathlineto{\pgfqpoint{2.419280in}{-3729.188002in}}%
\pgfpathlineto{\pgfqpoint{2.419280in}{-3729.188002in}}%
\pgfpathclose%
\pgfusepath{stroke,fill}%
\end{pgfscope}%
\begin{pgfscope}%
\pgfsetbuttcap%
\pgfsetroundjoin%
\definecolor{currentfill}{rgb}{0.000000,0.000000,0.000000}%
\pgfsetfillcolor{currentfill}%
\pgfsetlinewidth{0.602250pt}%
\definecolor{currentstroke}{rgb}{0.000000,0.000000,0.000000}%
\pgfsetstrokecolor{currentstroke}%
\pgfsetdash{}{0pt}%
\pgfsys@defobject{currentmarker}{\pgfqpoint{-0.027778in}{0.000000in}}{\pgfqpoint{-0.000000in}{0.000000in}}{%
\pgfpathmoveto{\pgfqpoint{-0.000000in}{0.000000in}}%
\pgfpathlineto{\pgfqpoint{-0.027778in}{0.000000in}}%
\pgfusepath{stroke,fill}%
}%
\begin{pgfscope}%
\pgfsys@transformshift{2.243603in}{1.125758in}%
\pgfsys@useobject{currentmarker}{}%
\end{pgfscope}%
\end{pgfscope}%
\begin{pgfscope}%
\pgfsetbuttcap%
\pgfsetroundjoin%
\definecolor{currentfill}{rgb}{0.000000,0.000000,0.000000}%
\pgfsetfillcolor{currentfill}%
\pgfsetlinewidth{0.602250pt}%
\definecolor{currentstroke}{rgb}{0.000000,0.000000,0.000000}%
\pgfsetstrokecolor{currentstroke}%
\pgfsetdash{}{0pt}%
\pgfsys@defobject{currentmarker}{\pgfqpoint{-0.027778in}{0.000000in}}{\pgfqpoint{-0.000000in}{0.000000in}}{%
\pgfpathmoveto{\pgfqpoint{-0.000000in}{0.000000in}}%
\pgfpathlineto{\pgfqpoint{-0.027778in}{0.000000in}}%
\pgfusepath{stroke,fill}%
}%
\begin{pgfscope}%
\pgfsys@transformshift{2.243603in}{1.781780in}%
\pgfsys@useobject{currentmarker}{}%
\end{pgfscope}%
\end{pgfscope}%
\begin{pgfscope}%
\pgfpathrectangle{\pgfqpoint{2.243603in}{0.862810in}}{\pgfqpoint{1.405419in}{0.918970in}}%
\pgfusepath{clip}%
\pgfsetbuttcap%
\pgfsetroundjoin%
\pgfsetlinewidth{0.501875pt}%
\definecolor{currentstroke}{rgb}{0.392157,0.396078,0.403922}%
\pgfsetstrokecolor{currentstroke}%
\pgfsetdash{}{0pt}%
\pgfpathmoveto{\pgfqpoint{2.594958in}{0.852810in}}%
\pgfpathlineto{\pgfqpoint{2.594958in}{1.791780in}}%
\pgfusepath{stroke}%
\end{pgfscope}%
\begin{pgfscope}%
\pgfpathrectangle{\pgfqpoint{2.243603in}{0.862810in}}{\pgfqpoint{1.405419in}{0.918970in}}%
\pgfusepath{clip}%
\pgfsetbuttcap%
\pgfsetroundjoin%
\pgfsetlinewidth{0.501875pt}%
\definecolor{currentstroke}{rgb}{0.392157,0.396078,0.403922}%
\pgfsetstrokecolor{currentstroke}%
\pgfsetdash{}{0pt}%
\pgfpathmoveto{\pgfqpoint{2.946312in}{0.852810in}}%
\pgfpathlineto{\pgfqpoint{2.946312in}{1.791780in}}%
\pgfusepath{stroke}%
\end{pgfscope}%
\begin{pgfscope}%
\pgfpathrectangle{\pgfqpoint{2.243603in}{0.862810in}}{\pgfqpoint{1.405419in}{0.918970in}}%
\pgfusepath{clip}%
\pgfsetbuttcap%
\pgfsetroundjoin%
\pgfsetlinewidth{0.501875pt}%
\definecolor{currentstroke}{rgb}{0.392157,0.396078,0.403922}%
\pgfsetstrokecolor{currentstroke}%
\pgfsetdash{}{0pt}%
\pgfpathmoveto{\pgfqpoint{3.297667in}{0.852810in}}%
\pgfpathlineto{\pgfqpoint{3.297667in}{1.791780in}}%
\pgfusepath{stroke}%
\end{pgfscope}%
\begin{pgfscope}%
\pgfpathrectangle{\pgfqpoint{2.243603in}{0.862810in}}{\pgfqpoint{1.405419in}{0.918970in}}%
\pgfusepath{clip}%
\pgfsetrectcap%
\pgfsetroundjoin%
\pgfsetlinewidth{0.752812pt}%
\definecolor{currentstroke}{rgb}{0.000000,0.000000,0.000000}%
\pgfsetstrokecolor{currentstroke}%
\pgfsetdash{}{0pt}%
\pgfpathmoveto{\pgfqpoint{2.349010in}{1.271781in}}%
\pgfpathlineto{\pgfqpoint{2.349010in}{1.100535in}}%
\pgfusepath{stroke}%
\end{pgfscope}%
\begin{pgfscope}%
\pgfpathrectangle{\pgfqpoint{2.243603in}{0.862810in}}{\pgfqpoint{1.405419in}{0.918970in}}%
\pgfusepath{clip}%
\pgfsetrectcap%
\pgfsetroundjoin%
\pgfsetlinewidth{0.752812pt}%
\definecolor{currentstroke}{rgb}{0.000000,0.000000,0.000000}%
\pgfsetstrokecolor{currentstroke}%
\pgfsetdash{}{0pt}%
\pgfpathmoveto{\pgfqpoint{2.349010in}{1.451084in}}%
\pgfpathlineto{\pgfqpoint{2.349010in}{1.536844in}}%
\pgfusepath{stroke}%
\end{pgfscope}%
\begin{pgfscope}%
\pgfpathrectangle{\pgfqpoint{2.243603in}{0.862810in}}{\pgfqpoint{1.405419in}{0.918970in}}%
\pgfusepath{clip}%
\pgfsetrectcap%
\pgfsetroundjoin%
\pgfsetlinewidth{0.752812pt}%
\definecolor{currentstroke}{rgb}{0.000000,0.000000,0.000000}%
\pgfsetstrokecolor{currentstroke}%
\pgfsetdash{}{0pt}%
\pgfpathmoveto{\pgfqpoint{2.314577in}{1.100535in}}%
\pgfpathlineto{\pgfqpoint{2.383442in}{1.100535in}}%
\pgfusepath{stroke}%
\end{pgfscope}%
\begin{pgfscope}%
\pgfpathrectangle{\pgfqpoint{2.243603in}{0.862810in}}{\pgfqpoint{1.405419in}{0.918970in}}%
\pgfusepath{clip}%
\pgfsetrectcap%
\pgfsetroundjoin%
\pgfsetlinewidth{0.752812pt}%
\definecolor{currentstroke}{rgb}{0.000000,0.000000,0.000000}%
\pgfsetstrokecolor{currentstroke}%
\pgfsetdash{}{0pt}%
\pgfpathmoveto{\pgfqpoint{2.314577in}{1.536844in}}%
\pgfpathlineto{\pgfqpoint{2.383442in}{1.536844in}}%
\pgfusepath{stroke}%
\end{pgfscope}%
\begin{pgfscope}%
\pgfpathrectangle{\pgfqpoint{2.243603in}{0.862810in}}{\pgfqpoint{1.405419in}{0.918970in}}%
\pgfusepath{clip}%
\pgfsetrectcap%
\pgfsetroundjoin%
\pgfsetlinewidth{0.752812pt}%
\definecolor{currentstroke}{rgb}{0.000000,0.000000,0.000000}%
\pgfsetstrokecolor{currentstroke}%
\pgfsetdash{}{0pt}%
\pgfpathmoveto{\pgfqpoint{2.489551in}{1.508771in}}%
\pgfpathlineto{\pgfqpoint{2.489551in}{1.365290in}}%
\pgfusepath{stroke}%
\end{pgfscope}%
\begin{pgfscope}%
\pgfpathrectangle{\pgfqpoint{2.243603in}{0.862810in}}{\pgfqpoint{1.405419in}{0.918970in}}%
\pgfusepath{clip}%
\pgfsetrectcap%
\pgfsetroundjoin%
\pgfsetlinewidth{0.752812pt}%
\definecolor{currentstroke}{rgb}{0.000000,0.000000,0.000000}%
\pgfsetstrokecolor{currentstroke}%
\pgfsetdash{}{0pt}%
\pgfpathmoveto{\pgfqpoint{2.489551in}{1.608299in}}%
\pgfpathlineto{\pgfqpoint{2.489551in}{1.745062in}}%
\pgfusepath{stroke}%
\end{pgfscope}%
\begin{pgfscope}%
\pgfpathrectangle{\pgfqpoint{2.243603in}{0.862810in}}{\pgfqpoint{1.405419in}{0.918970in}}%
\pgfusepath{clip}%
\pgfsetrectcap%
\pgfsetroundjoin%
\pgfsetlinewidth{0.752812pt}%
\definecolor{currentstroke}{rgb}{0.000000,0.000000,0.000000}%
\pgfsetstrokecolor{currentstroke}%
\pgfsetdash{}{0pt}%
\pgfpathmoveto{\pgfqpoint{2.455119in}{1.365290in}}%
\pgfpathlineto{\pgfqpoint{2.523984in}{1.365290in}}%
\pgfusepath{stroke}%
\end{pgfscope}%
\begin{pgfscope}%
\pgfpathrectangle{\pgfqpoint{2.243603in}{0.862810in}}{\pgfqpoint{1.405419in}{0.918970in}}%
\pgfusepath{clip}%
\pgfsetrectcap%
\pgfsetroundjoin%
\pgfsetlinewidth{0.752812pt}%
\definecolor{currentstroke}{rgb}{0.000000,0.000000,0.000000}%
\pgfsetstrokecolor{currentstroke}%
\pgfsetdash{}{0pt}%
\pgfpathmoveto{\pgfqpoint{2.455119in}{1.745062in}}%
\pgfpathlineto{\pgfqpoint{2.523984in}{1.745062in}}%
\pgfusepath{stroke}%
\end{pgfscope}%
\begin{pgfscope}%
\pgfpathrectangle{\pgfqpoint{2.243603in}{0.862810in}}{\pgfqpoint{1.405419in}{0.918970in}}%
\pgfusepath{clip}%
\pgfsetrectcap%
\pgfsetroundjoin%
\pgfsetlinewidth{0.752812pt}%
\definecolor{currentstroke}{rgb}{0.000000,0.000000,0.000000}%
\pgfsetstrokecolor{currentstroke}%
\pgfsetdash{}{0pt}%
\pgfpathmoveto{\pgfqpoint{2.700364in}{1.321786in}}%
\pgfpathlineto{\pgfqpoint{2.700364in}{1.174012in}}%
\pgfusepath{stroke}%
\end{pgfscope}%
\begin{pgfscope}%
\pgfpathrectangle{\pgfqpoint{2.243603in}{0.862810in}}{\pgfqpoint{1.405419in}{0.918970in}}%
\pgfusepath{clip}%
\pgfsetrectcap%
\pgfsetroundjoin%
\pgfsetlinewidth{0.752812pt}%
\definecolor{currentstroke}{rgb}{0.000000,0.000000,0.000000}%
\pgfsetstrokecolor{currentstroke}%
\pgfsetdash{}{0pt}%
\pgfpathmoveto{\pgfqpoint{2.700364in}{1.425620in}}%
\pgfpathlineto{\pgfqpoint{2.700364in}{1.561667in}}%
\pgfusepath{stroke}%
\end{pgfscope}%
\begin{pgfscope}%
\pgfpathrectangle{\pgfqpoint{2.243603in}{0.862810in}}{\pgfqpoint{1.405419in}{0.918970in}}%
\pgfusepath{clip}%
\pgfsetrectcap%
\pgfsetroundjoin%
\pgfsetlinewidth{0.752812pt}%
\definecolor{currentstroke}{rgb}{0.000000,0.000000,0.000000}%
\pgfsetstrokecolor{currentstroke}%
\pgfsetdash{}{0pt}%
\pgfpathmoveto{\pgfqpoint{2.665931in}{1.174012in}}%
\pgfpathlineto{\pgfqpoint{2.734797in}{1.174012in}}%
\pgfusepath{stroke}%
\end{pgfscope}%
\begin{pgfscope}%
\pgfpathrectangle{\pgfqpoint{2.243603in}{0.862810in}}{\pgfqpoint{1.405419in}{0.918970in}}%
\pgfusepath{clip}%
\pgfsetrectcap%
\pgfsetroundjoin%
\pgfsetlinewidth{0.752812pt}%
\definecolor{currentstroke}{rgb}{0.000000,0.000000,0.000000}%
\pgfsetstrokecolor{currentstroke}%
\pgfsetdash{}{0pt}%
\pgfpathmoveto{\pgfqpoint{2.665931in}{1.561667in}}%
\pgfpathlineto{\pgfqpoint{2.734797in}{1.561667in}}%
\pgfusepath{stroke}%
\end{pgfscope}%
\begin{pgfscope}%
\pgfpathrectangle{\pgfqpoint{2.243603in}{0.862810in}}{\pgfqpoint{1.405419in}{0.918970in}}%
\pgfusepath{clip}%
\pgfsetbuttcap%
\pgfsetmiterjoin%
\definecolor{currentfill}{rgb}{0.000000,0.000000,0.000000}%
\pgfsetfillcolor{currentfill}%
\pgfsetlinewidth{1.003750pt}%
\definecolor{currentstroke}{rgb}{0.000000,0.000000,0.000000}%
\pgfsetstrokecolor{currentstroke}%
\pgfsetdash{}{0pt}%
\pgfsys@defobject{currentmarker}{\pgfqpoint{-0.011785in}{-0.019642in}}{\pgfqpoint{0.011785in}{0.019642in}}{%
\pgfpathmoveto{\pgfqpoint{-0.000000in}{-0.019642in}}%
\pgfpathlineto{\pgfqpoint{0.011785in}{0.000000in}}%
\pgfpathlineto{\pgfqpoint{0.000000in}{0.019642in}}%
\pgfpathlineto{\pgfqpoint{-0.011785in}{0.000000in}}%
\pgfpathclose%
\pgfusepath{stroke,fill}%
}%
\begin{pgfscope}%
\pgfsys@transformshift{2.700364in}{1.137358in}%
\pgfsys@useobject{currentmarker}{}%
\end{pgfscope}%
\begin{pgfscope}%
\pgfsys@transformshift{2.700364in}{1.076665in}%
\pgfsys@useobject{currentmarker}{}%
\end{pgfscope}%
\end{pgfscope}%
\begin{pgfscope}%
\pgfpathrectangle{\pgfqpoint{2.243603in}{0.862810in}}{\pgfqpoint{1.405419in}{0.918970in}}%
\pgfusepath{clip}%
\pgfsetrectcap%
\pgfsetroundjoin%
\pgfsetlinewidth{0.752812pt}%
\definecolor{currentstroke}{rgb}{0.000000,0.000000,0.000000}%
\pgfsetstrokecolor{currentstroke}%
\pgfsetdash{}{0pt}%
\pgfpathmoveto{\pgfqpoint{2.840906in}{1.520985in}}%
\pgfpathlineto{\pgfqpoint{2.840906in}{1.382625in}}%
\pgfusepath{stroke}%
\end{pgfscope}%
\begin{pgfscope}%
\pgfpathrectangle{\pgfqpoint{2.243603in}{0.862810in}}{\pgfqpoint{1.405419in}{0.918970in}}%
\pgfusepath{clip}%
\pgfsetrectcap%
\pgfsetroundjoin%
\pgfsetlinewidth{0.752812pt}%
\definecolor{currentstroke}{rgb}{0.000000,0.000000,0.000000}%
\pgfsetstrokecolor{currentstroke}%
\pgfsetdash{}{0pt}%
\pgfpathmoveto{\pgfqpoint{2.840906in}{1.656123in}}%
\pgfpathlineto{\pgfqpoint{2.840906in}{1.762198in}}%
\pgfusepath{stroke}%
\end{pgfscope}%
\begin{pgfscope}%
\pgfpathrectangle{\pgfqpoint{2.243603in}{0.862810in}}{\pgfqpoint{1.405419in}{0.918970in}}%
\pgfusepath{clip}%
\pgfsetrectcap%
\pgfsetroundjoin%
\pgfsetlinewidth{0.752812pt}%
\definecolor{currentstroke}{rgb}{0.000000,0.000000,0.000000}%
\pgfsetstrokecolor{currentstroke}%
\pgfsetdash{}{0pt}%
\pgfpathmoveto{\pgfqpoint{2.806473in}{1.382625in}}%
\pgfpathlineto{\pgfqpoint{2.875339in}{1.382625in}}%
\pgfusepath{stroke}%
\end{pgfscope}%
\begin{pgfscope}%
\pgfpathrectangle{\pgfqpoint{2.243603in}{0.862810in}}{\pgfqpoint{1.405419in}{0.918970in}}%
\pgfusepath{clip}%
\pgfsetrectcap%
\pgfsetroundjoin%
\pgfsetlinewidth{0.752812pt}%
\definecolor{currentstroke}{rgb}{0.000000,0.000000,0.000000}%
\pgfsetstrokecolor{currentstroke}%
\pgfsetdash{}{0pt}%
\pgfpathmoveto{\pgfqpoint{2.806473in}{1.762198in}}%
\pgfpathlineto{\pgfqpoint{2.875339in}{1.762198in}}%
\pgfusepath{stroke}%
\end{pgfscope}%
\begin{pgfscope}%
\pgfpathrectangle{\pgfqpoint{2.243603in}{0.862810in}}{\pgfqpoint{1.405419in}{0.918970in}}%
\pgfusepath{clip}%
\pgfsetrectcap%
\pgfsetroundjoin%
\pgfsetlinewidth{0.752812pt}%
\definecolor{currentstroke}{rgb}{0.000000,0.000000,0.000000}%
\pgfsetstrokecolor{currentstroke}%
\pgfsetdash{}{0pt}%
\pgfpathmoveto{\pgfqpoint{3.051719in}{1.146235in}}%
\pgfpathlineto{\pgfqpoint{3.051719in}{0.852810in}}%
\pgfusepath{stroke}%
\end{pgfscope}%
\begin{pgfscope}%
\pgfpathrectangle{\pgfqpoint{2.243603in}{0.862810in}}{\pgfqpoint{1.405419in}{0.918970in}}%
\pgfusepath{clip}%
\pgfsetrectcap%
\pgfsetroundjoin%
\pgfsetlinewidth{0.752812pt}%
\definecolor{currentstroke}{rgb}{0.000000,0.000000,0.000000}%
\pgfsetstrokecolor{currentstroke}%
\pgfsetdash{}{0pt}%
\pgfpathmoveto{\pgfqpoint{3.051719in}{1.466762in}}%
\pgfpathlineto{\pgfqpoint{3.051719in}{1.570430in}}%
\pgfusepath{stroke}%
\end{pgfscope}%
\begin{pgfscope}%
\pgfpathrectangle{\pgfqpoint{2.243603in}{0.862810in}}{\pgfqpoint{1.405419in}{0.918970in}}%
\pgfusepath{clip}%
\pgfsetrectcap%
\pgfsetroundjoin%
\pgfsetlinewidth{0.752812pt}%
\definecolor{currentstroke}{rgb}{0.000000,0.000000,0.000000}%
\pgfsetstrokecolor{currentstroke}%
\pgfsetdash{}{0pt}%
\pgfusepath{stroke}%
\end{pgfscope}%
\begin{pgfscope}%
\pgfpathrectangle{\pgfqpoint{2.243603in}{0.862810in}}{\pgfqpoint{1.405419in}{0.918970in}}%
\pgfusepath{clip}%
\pgfsetrectcap%
\pgfsetroundjoin%
\pgfsetlinewidth{0.752812pt}%
\definecolor{currentstroke}{rgb}{0.000000,0.000000,0.000000}%
\pgfsetstrokecolor{currentstroke}%
\pgfsetdash{}{0pt}%
\pgfpathmoveto{\pgfqpoint{3.017286in}{1.570430in}}%
\pgfpathlineto{\pgfqpoint{3.086152in}{1.570430in}}%
\pgfusepath{stroke}%
\end{pgfscope}%
\begin{pgfscope}%
\pgfpathrectangle{\pgfqpoint{2.243603in}{0.862810in}}{\pgfqpoint{1.405419in}{0.918970in}}%
\pgfusepath{clip}%
\pgfsetrectcap%
\pgfsetroundjoin%
\pgfsetlinewidth{0.752812pt}%
\definecolor{currentstroke}{rgb}{0.000000,0.000000,0.000000}%
\pgfsetstrokecolor{currentstroke}%
\pgfsetdash{}{0pt}%
\pgfpathmoveto{\pgfqpoint{3.192261in}{1.183084in}}%
\pgfpathlineto{\pgfqpoint{3.192261in}{0.943288in}}%
\pgfusepath{stroke}%
\end{pgfscope}%
\begin{pgfscope}%
\pgfpathrectangle{\pgfqpoint{2.243603in}{0.862810in}}{\pgfqpoint{1.405419in}{0.918970in}}%
\pgfusepath{clip}%
\pgfsetrectcap%
\pgfsetroundjoin%
\pgfsetlinewidth{0.752812pt}%
\definecolor{currentstroke}{rgb}{0.000000,0.000000,0.000000}%
\pgfsetstrokecolor{currentstroke}%
\pgfsetdash{}{0pt}%
\pgfpathmoveto{\pgfqpoint{3.192261in}{1.423842in}}%
\pgfpathlineto{\pgfqpoint{3.192261in}{1.630750in}}%
\pgfusepath{stroke}%
\end{pgfscope}%
\begin{pgfscope}%
\pgfpathrectangle{\pgfqpoint{2.243603in}{0.862810in}}{\pgfqpoint{1.405419in}{0.918970in}}%
\pgfusepath{clip}%
\pgfsetrectcap%
\pgfsetroundjoin%
\pgfsetlinewidth{0.752812pt}%
\definecolor{currentstroke}{rgb}{0.000000,0.000000,0.000000}%
\pgfsetstrokecolor{currentstroke}%
\pgfsetdash{}{0pt}%
\pgfpathmoveto{\pgfqpoint{3.157828in}{0.943288in}}%
\pgfpathlineto{\pgfqpoint{3.226693in}{0.943288in}}%
\pgfusepath{stroke}%
\end{pgfscope}%
\begin{pgfscope}%
\pgfpathrectangle{\pgfqpoint{2.243603in}{0.862810in}}{\pgfqpoint{1.405419in}{0.918970in}}%
\pgfusepath{clip}%
\pgfsetrectcap%
\pgfsetroundjoin%
\pgfsetlinewidth{0.752812pt}%
\definecolor{currentstroke}{rgb}{0.000000,0.000000,0.000000}%
\pgfsetstrokecolor{currentstroke}%
\pgfsetdash{}{0pt}%
\pgfpathmoveto{\pgfqpoint{3.157828in}{1.630750in}}%
\pgfpathlineto{\pgfqpoint{3.226693in}{1.630750in}}%
\pgfusepath{stroke}%
\end{pgfscope}%
\begin{pgfscope}%
\pgfpathrectangle{\pgfqpoint{2.243603in}{0.862810in}}{\pgfqpoint{1.405419in}{0.918970in}}%
\pgfusepath{clip}%
\pgfsetrectcap%
\pgfsetroundjoin%
\pgfsetlinewidth{0.752812pt}%
\definecolor{currentstroke}{rgb}{0.000000,0.000000,0.000000}%
\pgfsetstrokecolor{currentstroke}%
\pgfsetdash{}{0pt}%
\pgfpathmoveto{\pgfqpoint{3.403073in}{1.237065in}}%
\pgfpathlineto{\pgfqpoint{3.403073in}{0.921599in}}%
\pgfusepath{stroke}%
\end{pgfscope}%
\begin{pgfscope}%
\pgfpathrectangle{\pgfqpoint{2.243603in}{0.862810in}}{\pgfqpoint{1.405419in}{0.918970in}}%
\pgfusepath{clip}%
\pgfsetrectcap%
\pgfsetroundjoin%
\pgfsetlinewidth{0.752812pt}%
\definecolor{currentstroke}{rgb}{0.000000,0.000000,0.000000}%
\pgfsetstrokecolor{currentstroke}%
\pgfsetdash{}{0pt}%
\pgfpathmoveto{\pgfqpoint{3.403073in}{1.479273in}}%
\pgfpathlineto{\pgfqpoint{3.403073in}{1.588087in}}%
\pgfusepath{stroke}%
\end{pgfscope}%
\begin{pgfscope}%
\pgfpathrectangle{\pgfqpoint{2.243603in}{0.862810in}}{\pgfqpoint{1.405419in}{0.918970in}}%
\pgfusepath{clip}%
\pgfsetrectcap%
\pgfsetroundjoin%
\pgfsetlinewidth{0.752812pt}%
\definecolor{currentstroke}{rgb}{0.000000,0.000000,0.000000}%
\pgfsetstrokecolor{currentstroke}%
\pgfsetdash{}{0pt}%
\pgfpathmoveto{\pgfqpoint{3.368641in}{0.921599in}}%
\pgfpathlineto{\pgfqpoint{3.437506in}{0.921599in}}%
\pgfusepath{stroke}%
\end{pgfscope}%
\begin{pgfscope}%
\pgfpathrectangle{\pgfqpoint{2.243603in}{0.862810in}}{\pgfqpoint{1.405419in}{0.918970in}}%
\pgfusepath{clip}%
\pgfsetrectcap%
\pgfsetroundjoin%
\pgfsetlinewidth{0.752812pt}%
\definecolor{currentstroke}{rgb}{0.000000,0.000000,0.000000}%
\pgfsetstrokecolor{currentstroke}%
\pgfsetdash{}{0pt}%
\pgfpathmoveto{\pgfqpoint{3.368641in}{1.588087in}}%
\pgfpathlineto{\pgfqpoint{3.437506in}{1.588087in}}%
\pgfusepath{stroke}%
\end{pgfscope}%
\begin{pgfscope}%
\pgfpathrectangle{\pgfqpoint{2.243603in}{0.862810in}}{\pgfqpoint{1.405419in}{0.918970in}}%
\pgfusepath{clip}%
\pgfsetrectcap%
\pgfsetroundjoin%
\pgfsetlinewidth{0.752812pt}%
\definecolor{currentstroke}{rgb}{0.000000,0.000000,0.000000}%
\pgfsetstrokecolor{currentstroke}%
\pgfsetdash{}{0pt}%
\pgfpathmoveto{\pgfqpoint{3.543615in}{1.267406in}}%
\pgfpathlineto{\pgfqpoint{3.543615in}{1.000493in}}%
\pgfusepath{stroke}%
\end{pgfscope}%
\begin{pgfscope}%
\pgfpathrectangle{\pgfqpoint{2.243603in}{0.862810in}}{\pgfqpoint{1.405419in}{0.918970in}}%
\pgfusepath{clip}%
\pgfsetrectcap%
\pgfsetroundjoin%
\pgfsetlinewidth{0.752812pt}%
\definecolor{currentstroke}{rgb}{0.000000,0.000000,0.000000}%
\pgfsetstrokecolor{currentstroke}%
\pgfsetdash{}{0pt}%
\pgfpathmoveto{\pgfqpoint{3.543615in}{1.445547in}}%
\pgfpathlineto{\pgfqpoint{3.543615in}{1.603119in}}%
\pgfusepath{stroke}%
\end{pgfscope}%
\begin{pgfscope}%
\pgfpathrectangle{\pgfqpoint{2.243603in}{0.862810in}}{\pgfqpoint{1.405419in}{0.918970in}}%
\pgfusepath{clip}%
\pgfsetrectcap%
\pgfsetroundjoin%
\pgfsetlinewidth{0.752812pt}%
\definecolor{currentstroke}{rgb}{0.000000,0.000000,0.000000}%
\pgfsetstrokecolor{currentstroke}%
\pgfsetdash{}{0pt}%
\pgfpathmoveto{\pgfqpoint{3.509183in}{1.000493in}}%
\pgfpathlineto{\pgfqpoint{3.578048in}{1.000493in}}%
\pgfusepath{stroke}%
\end{pgfscope}%
\begin{pgfscope}%
\pgfpathrectangle{\pgfqpoint{2.243603in}{0.862810in}}{\pgfqpoint{1.405419in}{0.918970in}}%
\pgfusepath{clip}%
\pgfsetrectcap%
\pgfsetroundjoin%
\pgfsetlinewidth{0.752812pt}%
\definecolor{currentstroke}{rgb}{0.000000,0.000000,0.000000}%
\pgfsetstrokecolor{currentstroke}%
\pgfsetdash{}{0pt}%
\pgfpathmoveto{\pgfqpoint{3.509183in}{1.603119in}}%
\pgfpathlineto{\pgfqpoint{3.578048in}{1.603119in}}%
\pgfusepath{stroke}%
\end{pgfscope}%
\begin{pgfscope}%
\pgfpathrectangle{\pgfqpoint{2.243603in}{0.862810in}}{\pgfqpoint{1.405419in}{0.918970in}}%
\pgfusepath{clip}%
\pgfsetbuttcap%
\pgfsetmiterjoin%
\definecolor{currentfill}{rgb}{0.000000,0.000000,0.000000}%
\pgfsetfillcolor{currentfill}%
\pgfsetlinewidth{1.003750pt}%
\definecolor{currentstroke}{rgb}{0.000000,0.000000,0.000000}%
\pgfsetstrokecolor{currentstroke}%
\pgfsetdash{}{0pt}%
\pgfsys@defobject{currentmarker}{\pgfqpoint{-0.011785in}{-0.019642in}}{\pgfqpoint{0.011785in}{0.019642in}}{%
\pgfpathmoveto{\pgfqpoint{-0.000000in}{-0.019642in}}%
\pgfpathlineto{\pgfqpoint{0.011785in}{0.000000in}}%
\pgfpathlineto{\pgfqpoint{0.000000in}{0.019642in}}%
\pgfpathlineto{\pgfqpoint{-0.011785in}{0.000000in}}%
\pgfpathclose%
\pgfusepath{stroke,fill}%
}%
\begin{pgfscope}%
\pgfsys@transformshift{3.543615in}{1.721218in}%
\pgfsys@useobject{currentmarker}{}%
\end{pgfscope}%
\end{pgfscope}%
\begin{pgfscope}%
\pgfpathrectangle{\pgfqpoint{2.243603in}{0.862810in}}{\pgfqpoint{1.405419in}{0.918970in}}%
\pgfusepath{clip}%
\pgfsetrectcap%
\pgfsetroundjoin%
\pgfsetlinewidth{0.752812pt}%
\definecolor{currentstroke}{rgb}{0.000000,0.000000,0.000000}%
\pgfsetstrokecolor{currentstroke}%
\pgfsetdash{}{0pt}%
\pgfpathmoveto{\pgfqpoint{2.280144in}{1.351045in}}%
\pgfpathlineto{\pgfqpoint{2.417875in}{1.351045in}}%
\pgfusepath{stroke}%
\end{pgfscope}%
\begin{pgfscope}%
\pgfpathrectangle{\pgfqpoint{2.243603in}{0.862810in}}{\pgfqpoint{1.405419in}{0.918970in}}%
\pgfusepath{clip}%
\pgfsetbuttcap%
\pgfsetroundjoin%
\definecolor{currentfill}{rgb}{1.000000,1.000000,1.000000}%
\pgfsetfillcolor{currentfill}%
\pgfsetlinewidth{1.003750pt}%
\definecolor{currentstroke}{rgb}{0.000000,0.000000,0.000000}%
\pgfsetstrokecolor{currentstroke}%
\pgfsetdash{}{0pt}%
\pgfsys@defobject{currentmarker}{\pgfqpoint{-0.027778in}{-0.027778in}}{\pgfqpoint{0.027778in}{0.027778in}}{%
\pgfpathmoveto{\pgfqpoint{0.000000in}{-0.027778in}}%
\pgfpathcurveto{\pgfqpoint{0.007367in}{-0.027778in}}{\pgfqpoint{0.014433in}{-0.024851in}}{\pgfqpoint{0.019642in}{-0.019642in}}%
\pgfpathcurveto{\pgfqpoint{0.024851in}{-0.014433in}}{\pgfqpoint{0.027778in}{-0.007367in}}{\pgfqpoint{0.027778in}{0.000000in}}%
\pgfpathcurveto{\pgfqpoint{0.027778in}{0.007367in}}{\pgfqpoint{0.024851in}{0.014433in}}{\pgfqpoint{0.019642in}{0.019642in}}%
\pgfpathcurveto{\pgfqpoint{0.014433in}{0.024851in}}{\pgfqpoint{0.007367in}{0.027778in}}{\pgfqpoint{0.000000in}{0.027778in}}%
\pgfpathcurveto{\pgfqpoint{-0.007367in}{0.027778in}}{\pgfqpoint{-0.014433in}{0.024851in}}{\pgfqpoint{-0.019642in}{0.019642in}}%
\pgfpathcurveto{\pgfqpoint{-0.024851in}{0.014433in}}{\pgfqpoint{-0.027778in}{0.007367in}}{\pgfqpoint{-0.027778in}{0.000000in}}%
\pgfpathcurveto{\pgfqpoint{-0.027778in}{-0.007367in}}{\pgfqpoint{-0.024851in}{-0.014433in}}{\pgfqpoint{-0.019642in}{-0.019642in}}%
\pgfpathcurveto{\pgfqpoint{-0.014433in}{-0.024851in}}{\pgfqpoint{-0.007367in}{-0.027778in}}{\pgfqpoint{0.000000in}{-0.027778in}}%
\pgfpathclose%
\pgfusepath{stroke,fill}%
}%
\begin{pgfscope}%
\pgfsys@transformshift{2.349010in}{1.357601in}%
\pgfsys@useobject{currentmarker}{}%
\end{pgfscope}%
\end{pgfscope}%
\begin{pgfscope}%
\pgfpathrectangle{\pgfqpoint{2.243603in}{0.862810in}}{\pgfqpoint{1.405419in}{0.918970in}}%
\pgfusepath{clip}%
\pgfsetrectcap%
\pgfsetroundjoin%
\pgfsetlinewidth{0.752812pt}%
\definecolor{currentstroke}{rgb}{0.000000,0.000000,0.000000}%
\pgfsetstrokecolor{currentstroke}%
\pgfsetdash{}{0pt}%
\pgfpathmoveto{\pgfqpoint{2.420686in}{1.552070in}}%
\pgfpathlineto{\pgfqpoint{2.558417in}{1.552070in}}%
\pgfusepath{stroke}%
\end{pgfscope}%
\begin{pgfscope}%
\pgfpathrectangle{\pgfqpoint{2.243603in}{0.862810in}}{\pgfqpoint{1.405419in}{0.918970in}}%
\pgfusepath{clip}%
\pgfsetbuttcap%
\pgfsetroundjoin%
\definecolor{currentfill}{rgb}{1.000000,1.000000,1.000000}%
\pgfsetfillcolor{currentfill}%
\pgfsetlinewidth{1.003750pt}%
\definecolor{currentstroke}{rgb}{0.000000,0.000000,0.000000}%
\pgfsetstrokecolor{currentstroke}%
\pgfsetdash{}{0pt}%
\pgfsys@defobject{currentmarker}{\pgfqpoint{-0.027778in}{-0.027778in}}{\pgfqpoint{0.027778in}{0.027778in}}{%
\pgfpathmoveto{\pgfqpoint{0.000000in}{-0.027778in}}%
\pgfpathcurveto{\pgfqpoint{0.007367in}{-0.027778in}}{\pgfqpoint{0.014433in}{-0.024851in}}{\pgfqpoint{0.019642in}{-0.019642in}}%
\pgfpathcurveto{\pgfqpoint{0.024851in}{-0.014433in}}{\pgfqpoint{0.027778in}{-0.007367in}}{\pgfqpoint{0.027778in}{0.000000in}}%
\pgfpathcurveto{\pgfqpoint{0.027778in}{0.007367in}}{\pgfqpoint{0.024851in}{0.014433in}}{\pgfqpoint{0.019642in}{0.019642in}}%
\pgfpathcurveto{\pgfqpoint{0.014433in}{0.024851in}}{\pgfqpoint{0.007367in}{0.027778in}}{\pgfqpoint{0.000000in}{0.027778in}}%
\pgfpathcurveto{\pgfqpoint{-0.007367in}{0.027778in}}{\pgfqpoint{-0.014433in}{0.024851in}}{\pgfqpoint{-0.019642in}{0.019642in}}%
\pgfpathcurveto{\pgfqpoint{-0.024851in}{0.014433in}}{\pgfqpoint{-0.027778in}{0.007367in}}{\pgfqpoint{-0.027778in}{0.000000in}}%
\pgfpathcurveto{\pgfqpoint{-0.027778in}{-0.007367in}}{\pgfqpoint{-0.024851in}{-0.014433in}}{\pgfqpoint{-0.019642in}{-0.019642in}}%
\pgfpathcurveto{\pgfqpoint{-0.014433in}{-0.024851in}}{\pgfqpoint{-0.007367in}{-0.027778in}}{\pgfqpoint{0.000000in}{-0.027778in}}%
\pgfpathclose%
\pgfusepath{stroke,fill}%
}%
\begin{pgfscope}%
\pgfsys@transformshift{2.489551in}{1.562837in}%
\pgfsys@useobject{currentmarker}{}%
\end{pgfscope}%
\end{pgfscope}%
\begin{pgfscope}%
\pgfpathrectangle{\pgfqpoint{2.243603in}{0.862810in}}{\pgfqpoint{1.405419in}{0.918970in}}%
\pgfusepath{clip}%
\pgfsetrectcap%
\pgfsetroundjoin%
\pgfsetlinewidth{0.752812pt}%
\definecolor{currentstroke}{rgb}{0.000000,0.000000,0.000000}%
\pgfsetstrokecolor{currentstroke}%
\pgfsetdash{}{0pt}%
\pgfpathmoveto{\pgfqpoint{2.631499in}{1.366001in}}%
\pgfpathlineto{\pgfqpoint{2.769230in}{1.366001in}}%
\pgfusepath{stroke}%
\end{pgfscope}%
\begin{pgfscope}%
\pgfpathrectangle{\pgfqpoint{2.243603in}{0.862810in}}{\pgfqpoint{1.405419in}{0.918970in}}%
\pgfusepath{clip}%
\pgfsetbuttcap%
\pgfsetroundjoin%
\definecolor{currentfill}{rgb}{1.000000,1.000000,1.000000}%
\pgfsetfillcolor{currentfill}%
\pgfsetlinewidth{1.003750pt}%
\definecolor{currentstroke}{rgb}{0.000000,0.000000,0.000000}%
\pgfsetstrokecolor{currentstroke}%
\pgfsetdash{}{0pt}%
\pgfsys@defobject{currentmarker}{\pgfqpoint{-0.027778in}{-0.027778in}}{\pgfqpoint{0.027778in}{0.027778in}}{%
\pgfpathmoveto{\pgfqpoint{0.000000in}{-0.027778in}}%
\pgfpathcurveto{\pgfqpoint{0.007367in}{-0.027778in}}{\pgfqpoint{0.014433in}{-0.024851in}}{\pgfqpoint{0.019642in}{-0.019642in}}%
\pgfpathcurveto{\pgfqpoint{0.024851in}{-0.014433in}}{\pgfqpoint{0.027778in}{-0.007367in}}{\pgfqpoint{0.027778in}{0.000000in}}%
\pgfpathcurveto{\pgfqpoint{0.027778in}{0.007367in}}{\pgfqpoint{0.024851in}{0.014433in}}{\pgfqpoint{0.019642in}{0.019642in}}%
\pgfpathcurveto{\pgfqpoint{0.014433in}{0.024851in}}{\pgfqpoint{0.007367in}{0.027778in}}{\pgfqpoint{0.000000in}{0.027778in}}%
\pgfpathcurveto{\pgfqpoint{-0.007367in}{0.027778in}}{\pgfqpoint{-0.014433in}{0.024851in}}{\pgfqpoint{-0.019642in}{0.019642in}}%
\pgfpathcurveto{\pgfqpoint{-0.024851in}{0.014433in}}{\pgfqpoint{-0.027778in}{0.007367in}}{\pgfqpoint{-0.027778in}{0.000000in}}%
\pgfpathcurveto{\pgfqpoint{-0.027778in}{-0.007367in}}{\pgfqpoint{-0.024851in}{-0.014433in}}{\pgfqpoint{-0.019642in}{-0.019642in}}%
\pgfpathcurveto{\pgfqpoint{-0.014433in}{-0.024851in}}{\pgfqpoint{-0.007367in}{-0.027778in}}{\pgfqpoint{0.000000in}{-0.027778in}}%
\pgfpathclose%
\pgfusepath{stroke,fill}%
}%
\begin{pgfscope}%
\pgfsys@transformshift{2.700364in}{1.366421in}%
\pgfsys@useobject{currentmarker}{}%
\end{pgfscope}%
\end{pgfscope}%
\begin{pgfscope}%
\pgfpathrectangle{\pgfqpoint{2.243603in}{0.862810in}}{\pgfqpoint{1.405419in}{0.918970in}}%
\pgfusepath{clip}%
\pgfsetrectcap%
\pgfsetroundjoin%
\pgfsetlinewidth{0.752812pt}%
\definecolor{currentstroke}{rgb}{0.000000,0.000000,0.000000}%
\pgfsetstrokecolor{currentstroke}%
\pgfsetdash{}{0pt}%
\pgfpathmoveto{\pgfqpoint{2.772041in}{1.572332in}}%
\pgfpathlineto{\pgfqpoint{2.909772in}{1.572332in}}%
\pgfusepath{stroke}%
\end{pgfscope}%
\begin{pgfscope}%
\pgfpathrectangle{\pgfqpoint{2.243603in}{0.862810in}}{\pgfqpoint{1.405419in}{0.918970in}}%
\pgfusepath{clip}%
\pgfsetbuttcap%
\pgfsetroundjoin%
\definecolor{currentfill}{rgb}{1.000000,1.000000,1.000000}%
\pgfsetfillcolor{currentfill}%
\pgfsetlinewidth{1.003750pt}%
\definecolor{currentstroke}{rgb}{0.000000,0.000000,0.000000}%
\pgfsetstrokecolor{currentstroke}%
\pgfsetdash{}{0pt}%
\pgfsys@defobject{currentmarker}{\pgfqpoint{-0.027778in}{-0.027778in}}{\pgfqpoint{0.027778in}{0.027778in}}{%
\pgfpathmoveto{\pgfqpoint{0.000000in}{-0.027778in}}%
\pgfpathcurveto{\pgfqpoint{0.007367in}{-0.027778in}}{\pgfqpoint{0.014433in}{-0.024851in}}{\pgfqpoint{0.019642in}{-0.019642in}}%
\pgfpathcurveto{\pgfqpoint{0.024851in}{-0.014433in}}{\pgfqpoint{0.027778in}{-0.007367in}}{\pgfqpoint{0.027778in}{0.000000in}}%
\pgfpathcurveto{\pgfqpoint{0.027778in}{0.007367in}}{\pgfqpoint{0.024851in}{0.014433in}}{\pgfqpoint{0.019642in}{0.019642in}}%
\pgfpathcurveto{\pgfqpoint{0.014433in}{0.024851in}}{\pgfqpoint{0.007367in}{0.027778in}}{\pgfqpoint{0.000000in}{0.027778in}}%
\pgfpathcurveto{\pgfqpoint{-0.007367in}{0.027778in}}{\pgfqpoint{-0.014433in}{0.024851in}}{\pgfqpoint{-0.019642in}{0.019642in}}%
\pgfpathcurveto{\pgfqpoint{-0.024851in}{0.014433in}}{\pgfqpoint{-0.027778in}{0.007367in}}{\pgfqpoint{-0.027778in}{0.000000in}}%
\pgfpathcurveto{\pgfqpoint{-0.027778in}{-0.007367in}}{\pgfqpoint{-0.024851in}{-0.014433in}}{\pgfqpoint{-0.019642in}{-0.019642in}}%
\pgfpathcurveto{\pgfqpoint{-0.014433in}{-0.024851in}}{\pgfqpoint{-0.007367in}{-0.027778in}}{\pgfqpoint{0.000000in}{-0.027778in}}%
\pgfpathclose%
\pgfusepath{stroke,fill}%
}%
\begin{pgfscope}%
\pgfsys@transformshift{2.840906in}{1.586851in}%
\pgfsys@useobject{currentmarker}{}%
\end{pgfscope}%
\end{pgfscope}%
\begin{pgfscope}%
\pgfpathrectangle{\pgfqpoint{2.243603in}{0.862810in}}{\pgfqpoint{1.405419in}{0.918970in}}%
\pgfusepath{clip}%
\pgfsetrectcap%
\pgfsetroundjoin%
\pgfsetlinewidth{0.752812pt}%
\definecolor{currentstroke}{rgb}{0.000000,0.000000,0.000000}%
\pgfsetstrokecolor{currentstroke}%
\pgfsetdash{}{0pt}%
\pgfpathmoveto{\pgfqpoint{2.982853in}{1.373807in}}%
\pgfpathlineto{\pgfqpoint{3.120584in}{1.373807in}}%
\pgfusepath{stroke}%
\end{pgfscope}%
\begin{pgfscope}%
\pgfpathrectangle{\pgfqpoint{2.243603in}{0.862810in}}{\pgfqpoint{1.405419in}{0.918970in}}%
\pgfusepath{clip}%
\pgfsetbuttcap%
\pgfsetroundjoin%
\definecolor{currentfill}{rgb}{1.000000,1.000000,1.000000}%
\pgfsetfillcolor{currentfill}%
\pgfsetlinewidth{1.003750pt}%
\definecolor{currentstroke}{rgb}{0.000000,0.000000,0.000000}%
\pgfsetstrokecolor{currentstroke}%
\pgfsetdash{}{0pt}%
\pgfsys@defobject{currentmarker}{\pgfqpoint{-0.027778in}{-0.027778in}}{\pgfqpoint{0.027778in}{0.027778in}}{%
\pgfpathmoveto{\pgfqpoint{0.000000in}{-0.027778in}}%
\pgfpathcurveto{\pgfqpoint{0.007367in}{-0.027778in}}{\pgfqpoint{0.014433in}{-0.024851in}}{\pgfqpoint{0.019642in}{-0.019642in}}%
\pgfpathcurveto{\pgfqpoint{0.024851in}{-0.014433in}}{\pgfqpoint{0.027778in}{-0.007367in}}{\pgfqpoint{0.027778in}{0.000000in}}%
\pgfpathcurveto{\pgfqpoint{0.027778in}{0.007367in}}{\pgfqpoint{0.024851in}{0.014433in}}{\pgfqpoint{0.019642in}{0.019642in}}%
\pgfpathcurveto{\pgfqpoint{0.014433in}{0.024851in}}{\pgfqpoint{0.007367in}{0.027778in}}{\pgfqpoint{0.000000in}{0.027778in}}%
\pgfpathcurveto{\pgfqpoint{-0.007367in}{0.027778in}}{\pgfqpoint{-0.014433in}{0.024851in}}{\pgfqpoint{-0.019642in}{0.019642in}}%
\pgfpathcurveto{\pgfqpoint{-0.024851in}{0.014433in}}{\pgfqpoint{-0.027778in}{0.007367in}}{\pgfqpoint{-0.027778in}{0.000000in}}%
\pgfpathcurveto{\pgfqpoint{-0.027778in}{-0.007367in}}{\pgfqpoint{-0.024851in}{-0.014433in}}{\pgfqpoint{-0.019642in}{-0.019642in}}%
\pgfpathcurveto{\pgfqpoint{-0.014433in}{-0.024851in}}{\pgfqpoint{-0.007367in}{-0.027778in}}{\pgfqpoint{0.000000in}{-0.027778in}}%
\pgfpathclose%
\pgfusepath{stroke,fill}%
}%
\begin{pgfscope}%
\pgfsys@transformshift{3.051719in}{1.292536in}%
\pgfsys@useobject{currentmarker}{}%
\end{pgfscope}%
\end{pgfscope}%
\begin{pgfscope}%
\pgfpathrectangle{\pgfqpoint{2.243603in}{0.862810in}}{\pgfqpoint{1.405419in}{0.918970in}}%
\pgfusepath{clip}%
\pgfsetrectcap%
\pgfsetroundjoin%
\pgfsetlinewidth{0.752812pt}%
\definecolor{currentstroke}{rgb}{0.000000,0.000000,0.000000}%
\pgfsetstrokecolor{currentstroke}%
\pgfsetdash{}{0pt}%
\pgfpathmoveto{\pgfqpoint{3.123395in}{1.319494in}}%
\pgfpathlineto{\pgfqpoint{3.261126in}{1.319494in}}%
\pgfusepath{stroke}%
\end{pgfscope}%
\begin{pgfscope}%
\pgfpathrectangle{\pgfqpoint{2.243603in}{0.862810in}}{\pgfqpoint{1.405419in}{0.918970in}}%
\pgfusepath{clip}%
\pgfsetbuttcap%
\pgfsetroundjoin%
\definecolor{currentfill}{rgb}{1.000000,1.000000,1.000000}%
\pgfsetfillcolor{currentfill}%
\pgfsetlinewidth{1.003750pt}%
\definecolor{currentstroke}{rgb}{0.000000,0.000000,0.000000}%
\pgfsetstrokecolor{currentstroke}%
\pgfsetdash{}{0pt}%
\pgfsys@defobject{currentmarker}{\pgfqpoint{-0.027778in}{-0.027778in}}{\pgfqpoint{0.027778in}{0.027778in}}{%
\pgfpathmoveto{\pgfqpoint{0.000000in}{-0.027778in}}%
\pgfpathcurveto{\pgfqpoint{0.007367in}{-0.027778in}}{\pgfqpoint{0.014433in}{-0.024851in}}{\pgfqpoint{0.019642in}{-0.019642in}}%
\pgfpathcurveto{\pgfqpoint{0.024851in}{-0.014433in}}{\pgfqpoint{0.027778in}{-0.007367in}}{\pgfqpoint{0.027778in}{0.000000in}}%
\pgfpathcurveto{\pgfqpoint{0.027778in}{0.007367in}}{\pgfqpoint{0.024851in}{0.014433in}}{\pgfqpoint{0.019642in}{0.019642in}}%
\pgfpathcurveto{\pgfqpoint{0.014433in}{0.024851in}}{\pgfqpoint{0.007367in}{0.027778in}}{\pgfqpoint{0.000000in}{0.027778in}}%
\pgfpathcurveto{\pgfqpoint{-0.007367in}{0.027778in}}{\pgfqpoint{-0.014433in}{0.024851in}}{\pgfqpoint{-0.019642in}{0.019642in}}%
\pgfpathcurveto{\pgfqpoint{-0.024851in}{0.014433in}}{\pgfqpoint{-0.027778in}{0.007367in}}{\pgfqpoint{-0.027778in}{0.000000in}}%
\pgfpathcurveto{\pgfqpoint{-0.027778in}{-0.007367in}}{\pgfqpoint{-0.024851in}{-0.014433in}}{\pgfqpoint{-0.019642in}{-0.019642in}}%
\pgfpathcurveto{\pgfqpoint{-0.014433in}{-0.024851in}}{\pgfqpoint{-0.007367in}{-0.027778in}}{\pgfqpoint{0.000000in}{-0.027778in}}%
\pgfpathclose%
\pgfusepath{stroke,fill}%
}%
\begin{pgfscope}%
\pgfsys@transformshift{3.192261in}{1.300417in}%
\pgfsys@useobject{currentmarker}{}%
\end{pgfscope}%
\end{pgfscope}%
\begin{pgfscope}%
\pgfpathrectangle{\pgfqpoint{2.243603in}{0.862810in}}{\pgfqpoint{1.405419in}{0.918970in}}%
\pgfusepath{clip}%
\pgfsetrectcap%
\pgfsetroundjoin%
\pgfsetlinewidth{0.752812pt}%
\definecolor{currentstroke}{rgb}{0.000000,0.000000,0.000000}%
\pgfsetstrokecolor{currentstroke}%
\pgfsetdash{}{0pt}%
\pgfpathmoveto{\pgfqpoint{3.334208in}{1.350660in}}%
\pgfpathlineto{\pgfqpoint{3.471939in}{1.350660in}}%
\pgfusepath{stroke}%
\end{pgfscope}%
\begin{pgfscope}%
\pgfpathrectangle{\pgfqpoint{2.243603in}{0.862810in}}{\pgfqpoint{1.405419in}{0.918970in}}%
\pgfusepath{clip}%
\pgfsetbuttcap%
\pgfsetroundjoin%
\definecolor{currentfill}{rgb}{1.000000,1.000000,1.000000}%
\pgfsetfillcolor{currentfill}%
\pgfsetlinewidth{1.003750pt}%
\definecolor{currentstroke}{rgb}{0.000000,0.000000,0.000000}%
\pgfsetstrokecolor{currentstroke}%
\pgfsetdash{}{0pt}%
\pgfsys@defobject{currentmarker}{\pgfqpoint{-0.027778in}{-0.027778in}}{\pgfqpoint{0.027778in}{0.027778in}}{%
\pgfpathmoveto{\pgfqpoint{0.000000in}{-0.027778in}}%
\pgfpathcurveto{\pgfqpoint{0.007367in}{-0.027778in}}{\pgfqpoint{0.014433in}{-0.024851in}}{\pgfqpoint{0.019642in}{-0.019642in}}%
\pgfpathcurveto{\pgfqpoint{0.024851in}{-0.014433in}}{\pgfqpoint{0.027778in}{-0.007367in}}{\pgfqpoint{0.027778in}{0.000000in}}%
\pgfpathcurveto{\pgfqpoint{0.027778in}{0.007367in}}{\pgfqpoint{0.024851in}{0.014433in}}{\pgfqpoint{0.019642in}{0.019642in}}%
\pgfpathcurveto{\pgfqpoint{0.014433in}{0.024851in}}{\pgfqpoint{0.007367in}{0.027778in}}{\pgfqpoint{0.000000in}{0.027778in}}%
\pgfpathcurveto{\pgfqpoint{-0.007367in}{0.027778in}}{\pgfqpoint{-0.014433in}{0.024851in}}{\pgfqpoint{-0.019642in}{0.019642in}}%
\pgfpathcurveto{\pgfqpoint{-0.024851in}{0.014433in}}{\pgfqpoint{-0.027778in}{0.007367in}}{\pgfqpoint{-0.027778in}{0.000000in}}%
\pgfpathcurveto{\pgfqpoint{-0.027778in}{-0.007367in}}{\pgfqpoint{-0.024851in}{-0.014433in}}{\pgfqpoint{-0.019642in}{-0.019642in}}%
\pgfpathcurveto{\pgfqpoint{-0.014433in}{-0.024851in}}{\pgfqpoint{-0.007367in}{-0.027778in}}{\pgfqpoint{0.000000in}{-0.027778in}}%
\pgfpathclose%
\pgfusepath{stroke,fill}%
}%
\begin{pgfscope}%
\pgfsys@transformshift{3.403073in}{1.349710in}%
\pgfsys@useobject{currentmarker}{}%
\end{pgfscope}%
\end{pgfscope}%
\begin{pgfscope}%
\pgfpathrectangle{\pgfqpoint{2.243603in}{0.862810in}}{\pgfqpoint{1.405419in}{0.918970in}}%
\pgfusepath{clip}%
\pgfsetrectcap%
\pgfsetroundjoin%
\pgfsetlinewidth{0.752812pt}%
\definecolor{currentstroke}{rgb}{0.000000,0.000000,0.000000}%
\pgfsetstrokecolor{currentstroke}%
\pgfsetdash{}{0pt}%
\pgfpathmoveto{\pgfqpoint{3.474750in}{1.369510in}}%
\pgfpathlineto{\pgfqpoint{3.612481in}{1.369510in}}%
\pgfusepath{stroke}%
\end{pgfscope}%
\begin{pgfscope}%
\pgfpathrectangle{\pgfqpoint{2.243603in}{0.862810in}}{\pgfqpoint{1.405419in}{0.918970in}}%
\pgfusepath{clip}%
\pgfsetbuttcap%
\pgfsetroundjoin%
\definecolor{currentfill}{rgb}{1.000000,1.000000,1.000000}%
\pgfsetfillcolor{currentfill}%
\pgfsetlinewidth{1.003750pt}%
\definecolor{currentstroke}{rgb}{0.000000,0.000000,0.000000}%
\pgfsetstrokecolor{currentstroke}%
\pgfsetdash{}{0pt}%
\pgfsys@defobject{currentmarker}{\pgfqpoint{-0.027778in}{-0.027778in}}{\pgfqpoint{0.027778in}{0.027778in}}{%
\pgfpathmoveto{\pgfqpoint{0.000000in}{-0.027778in}}%
\pgfpathcurveto{\pgfqpoint{0.007367in}{-0.027778in}}{\pgfqpoint{0.014433in}{-0.024851in}}{\pgfqpoint{0.019642in}{-0.019642in}}%
\pgfpathcurveto{\pgfqpoint{0.024851in}{-0.014433in}}{\pgfqpoint{0.027778in}{-0.007367in}}{\pgfqpoint{0.027778in}{0.000000in}}%
\pgfpathcurveto{\pgfqpoint{0.027778in}{0.007367in}}{\pgfqpoint{0.024851in}{0.014433in}}{\pgfqpoint{0.019642in}{0.019642in}}%
\pgfpathcurveto{\pgfqpoint{0.014433in}{0.024851in}}{\pgfqpoint{0.007367in}{0.027778in}}{\pgfqpoint{0.000000in}{0.027778in}}%
\pgfpathcurveto{\pgfqpoint{-0.007367in}{0.027778in}}{\pgfqpoint{-0.014433in}{0.024851in}}{\pgfqpoint{-0.019642in}{0.019642in}}%
\pgfpathcurveto{\pgfqpoint{-0.024851in}{0.014433in}}{\pgfqpoint{-0.027778in}{0.007367in}}{\pgfqpoint{-0.027778in}{0.000000in}}%
\pgfpathcurveto{\pgfqpoint{-0.027778in}{-0.007367in}}{\pgfqpoint{-0.024851in}{-0.014433in}}{\pgfqpoint{-0.019642in}{-0.019642in}}%
\pgfpathcurveto{\pgfqpoint{-0.014433in}{-0.024851in}}{\pgfqpoint{-0.007367in}{-0.027778in}}{\pgfqpoint{0.000000in}{-0.027778in}}%
\pgfpathclose%
\pgfusepath{stroke,fill}%
}%
\begin{pgfscope}%
\pgfsys@transformshift{3.543615in}{1.362928in}%
\pgfsys@useobject{currentmarker}{}%
\end{pgfscope}%
\end{pgfscope}%
\begin{pgfscope}%
\pgfsetrectcap%
\pgfsetmiterjoin%
\pgfsetlinewidth{0.803000pt}%
\definecolor{currentstroke}{rgb}{0.000000,0.000000,0.000000}%
\pgfsetstrokecolor{currentstroke}%
\pgfsetdash{}{0pt}%
\pgfpathmoveto{\pgfqpoint{2.243603in}{0.862810in}}%
\pgfpathlineto{\pgfqpoint{2.243603in}{1.781780in}}%
\pgfusepath{stroke}%
\end{pgfscope}%
\begin{pgfscope}%
\pgfsetrectcap%
\pgfsetmiterjoin%
\pgfsetlinewidth{0.803000pt}%
\definecolor{currentstroke}{rgb}{0.000000,0.000000,0.000000}%
\pgfsetstrokecolor{currentstroke}%
\pgfsetdash{}{0pt}%
\pgfpathmoveto{\pgfqpoint{3.649022in}{0.862810in}}%
\pgfpathlineto{\pgfqpoint{3.649022in}{1.781780in}}%
\pgfusepath{stroke}%
\end{pgfscope}%
\begin{pgfscope}%
\pgfsetrectcap%
\pgfsetmiterjoin%
\pgfsetlinewidth{0.803000pt}%
\definecolor{currentstroke}{rgb}{0.000000,0.000000,0.000000}%
\pgfsetstrokecolor{currentstroke}%
\pgfsetdash{}{0pt}%
\pgfpathmoveto{\pgfqpoint{2.243603in}{0.862810in}}%
\pgfpathlineto{\pgfqpoint{3.649022in}{0.862810in}}%
\pgfusepath{stroke}%
\end{pgfscope}%
\begin{pgfscope}%
\pgfsetrectcap%
\pgfsetmiterjoin%
\pgfsetlinewidth{0.803000pt}%
\definecolor{currentstroke}{rgb}{0.000000,0.000000,0.000000}%
\pgfsetstrokecolor{currentstroke}%
\pgfsetdash{}{0pt}%
\pgfpathmoveto{\pgfqpoint{2.243603in}{1.781780in}}%
\pgfpathlineto{\pgfqpoint{3.649022in}{1.781780in}}%
\pgfusepath{stroke}%
\end{pgfscope}%
\begin{pgfscope}%
\definecolor{textcolor}{rgb}{0.000000,0.000000,0.000000}%
\pgfsetstrokecolor{textcolor}%
\pgfsetfillcolor{textcolor}%
\pgftext[x=2.946312in,y=1.815946in,,base]{\color{textcolor}\rmfamily\fontsize{12.000000}{14.400000}\selectfont Zoom with log-scale}%
\end{pgfscope}%
\end{pgfpicture}%
\makeatother%
\endgroup%

    \caption[Results of the one-dimensional within model comparison.]{Results for the one-dimensional within-model comparison. The white circles represent the mean of each variation. The darker shades show the performance with a well-defined mean, while the lighter shades show the performance with an optimistic mean. \gls{ctvbo} significantly reduces the regret and the sensitivity regarding an optimistic prior.}
    \label{fig:WMC_cumulative_regret_1D}
\end{figure}

The dashed line indicates the regret obtained if the minimum of the posterior mean after the initialization had been chosen as the query for the whole time horizon. Figure~\ref{fig:WMC_cumulative_regret_1D} shows that all variations, except SW TV-GP-UCB with an optimistic prior mean, fall below this regret. It indicates the sliding window of $30$ being too small for the chosen forgetting factor. However, it can be stated that regardless of the forgetting method and algorithm, it is worth considering the time-varying nature of the objective function.
The proposed method \gls{ctvbo} reduces the regret for both \gls{b2p} forgetting and \gls{ui} forgetting compared to normal \gls{tvbo}. 

Furthermore, by taking into account the prior knowledge, the variance is reduced. Additionally, the proposed modeling approach \gls{uitvbo} shows only minor differences when changing the prior mean. In contrast, the variations with \gls{b2p} forgetting strongly respond to an optimistic mean with an increased exploratory behavior and thus higher regret. This behavior was expected and stated in  Hypothesis~\ref{hyp:ui_structural_information}. Nevertheless, \gls{ctvbo} restricts the explorative behavior and, therefore, reduces the effect of the optimistic prior compared to standard \gls{tvbo}.

The data selection strategies consistently result in worse regret compared to the variations using all queried data. This behavior is expected as the posterior at each time step is only an approximation. However, it should be noted that the binning approach for \gls{ui} forgetting has only a diminutive impact on regret and is, therefore, a suitable data selection strategy for \gls{ui} forgetting. For the sliding window approach for \gls{b2p} forgetting, a larger window size $W$ could improve the regret because the temporal correlations after $30$ time steps are still significant using a forgetting factor of $\epsilon =0.03$. Combining the proposed methods \gls{uitvbo} and \gls{ctvbo} results in the best performance in terms of cumulative regret, both for a well-defined and optimistic prior mean.

In the results of the two-dimensional within-model comparisons in Figure~\ref{fig:WMC_cumulative_regret_2D}, similar trends can be observed, however, they are not as distinct as in the one-dimensional case. Again, all variations perform better compared to only choosing the minimum of the posterior mean after the initialization. Furthermore, it can be observed that also in the two-dimensional case, the method \gls{ctvbo} reduces the regret as well as its variance compared to standard \gls{tvbo}, which further strengthens Hypothesis~\ref{hyp:ctvbo}. 
\begin{figure}[h]
    \centering
    %% Creator: Matplotlib, PGF backend
%%
%% To include the figure in your LaTeX document, write
%%   \input{<filename>.pgf}
%%
%% Make sure the required packages are loaded in your preamble
%%   \usepackage{pgf}
%%
%% Figures using additional raster images can only be included by \input if
%% they are in the same directory as the main LaTeX file. For loading figures
%% from other directories you can use the `import` package
%%   \usepackage{import}
%%
%% and then include the figures with
%%   \import{<path to file>}{<filename>.pgf}
%%
%% Matplotlib used the following preamble
%%   \usepackage{fontspec}
%%
\begingroup%
\makeatletter%
\begin{pgfpicture}%
\pgfpathrectangle{\pgfpointorigin}{\pgfqpoint{5.507126in}{2.552693in}}%
\pgfusepath{use as bounding box, clip}%
\begin{pgfscope}%
\pgfsetbuttcap%
\pgfsetmiterjoin%
\definecolor{currentfill}{rgb}{1.000000,1.000000,1.000000}%
\pgfsetfillcolor{currentfill}%
\pgfsetlinewidth{0.000000pt}%
\definecolor{currentstroke}{rgb}{1.000000,1.000000,1.000000}%
\pgfsetstrokecolor{currentstroke}%
\pgfsetdash{}{0pt}%
\pgfpathmoveto{\pgfqpoint{0.000000in}{0.000000in}}%
\pgfpathlineto{\pgfqpoint{5.507126in}{0.000000in}}%
\pgfpathlineto{\pgfqpoint{5.507126in}{2.552693in}}%
\pgfpathlineto{\pgfqpoint{0.000000in}{2.552693in}}%
\pgfpathclose%
\pgfusepath{fill}%
\end{pgfscope}%
\begin{pgfscope}%
\pgfsetbuttcap%
\pgfsetmiterjoin%
\definecolor{currentfill}{rgb}{1.000000,1.000000,1.000000}%
\pgfsetfillcolor{currentfill}%
\pgfsetlinewidth{0.000000pt}%
\definecolor{currentstroke}{rgb}{0.000000,0.000000,0.000000}%
\pgfsetstrokecolor{currentstroke}%
\pgfsetstrokeopacity{0.000000}%
\pgfsetdash{}{0pt}%
\pgfpathmoveto{\pgfqpoint{0.550713in}{0.127635in}}%
\pgfpathlineto{\pgfqpoint{3.744846in}{0.127635in}}%
\pgfpathlineto{\pgfqpoint{3.744846in}{2.425059in}}%
\pgfpathlineto{\pgfqpoint{0.550713in}{2.425059in}}%
\pgfpathclose%
\pgfusepath{fill}%
\end{pgfscope}%
\begin{pgfscope}%
\pgfpathrectangle{\pgfqpoint{0.550713in}{0.127635in}}{\pgfqpoint{3.194133in}{2.297424in}}%
\pgfusepath{clip}%
\pgfsetbuttcap%
\pgfsetmiterjoin%
\definecolor{currentfill}{rgb}{0.631373,0.062745,0.207843}%
\pgfsetfillcolor{currentfill}%
\pgfsetlinewidth{0.752812pt}%
\definecolor{currentstroke}{rgb}{0.000000,0.000000,0.000000}%
\pgfsetstrokecolor{currentstroke}%
\pgfsetdash{}{0pt}%
\pgfpathmoveto{\pgfqpoint{0.592236in}{1.259009in}}%
\pgfpathlineto{\pgfqpoint{0.748749in}{1.259009in}}%
\pgfpathlineto{\pgfqpoint{0.748749in}{1.540592in}}%
\pgfpathlineto{\pgfqpoint{0.592236in}{1.540592in}}%
\pgfpathlineto{\pgfqpoint{0.592236in}{1.259009in}}%
\pgfpathclose%
\pgfusepath{stroke,fill}%
\end{pgfscope}%
\begin{pgfscope}%
\pgfpathrectangle{\pgfqpoint{0.550713in}{0.127635in}}{\pgfqpoint{3.194133in}{2.297424in}}%
\pgfusepath{clip}%
\pgfsetbuttcap%
\pgfsetmiterjoin%
\definecolor{currentfill}{rgb}{0.898039,0.772549,0.752941}%
\pgfsetfillcolor{currentfill}%
\pgfsetlinewidth{0.752812pt}%
\definecolor{currentstroke}{rgb}{0.000000,0.000000,0.000000}%
\pgfsetstrokecolor{currentstroke}%
\pgfsetdash{}{0pt}%
\pgfpathmoveto{\pgfqpoint{0.751943in}{1.206496in}}%
\pgfpathlineto{\pgfqpoint{0.908456in}{1.206496in}}%
\pgfpathlineto{\pgfqpoint{0.908456in}{1.457729in}}%
\pgfpathlineto{\pgfqpoint{0.751943in}{1.457729in}}%
\pgfpathlineto{\pgfqpoint{0.751943in}{1.206496in}}%
\pgfpathclose%
\pgfusepath{stroke,fill}%
\end{pgfscope}%
\begin{pgfscope}%
\pgfpathrectangle{\pgfqpoint{0.550713in}{0.127635in}}{\pgfqpoint{3.194133in}{2.297424in}}%
\pgfusepath{clip}%
\pgfsetbuttcap%
\pgfsetmiterjoin%
\definecolor{currentfill}{rgb}{0.890196,0.000000,0.400000}%
\pgfsetfillcolor{currentfill}%
\pgfsetlinewidth{0.752812pt}%
\definecolor{currentstroke}{rgb}{0.000000,0.000000,0.000000}%
\pgfsetstrokecolor{currentstroke}%
\pgfsetdash{}{0pt}%
\pgfpathmoveto{\pgfqpoint{0.991503in}{1.226382in}}%
\pgfpathlineto{\pgfqpoint{1.148015in}{1.226382in}}%
\pgfpathlineto{\pgfqpoint{1.148015in}{1.624032in}}%
\pgfpathlineto{\pgfqpoint{0.991503in}{1.624032in}}%
\pgfpathlineto{\pgfqpoint{0.991503in}{1.226382in}}%
\pgfpathclose%
\pgfusepath{stroke,fill}%
\end{pgfscope}%
\begin{pgfscope}%
\pgfpathrectangle{\pgfqpoint{0.550713in}{0.127635in}}{\pgfqpoint{3.194133in}{2.297424in}}%
\pgfusepath{clip}%
\pgfsetbuttcap%
\pgfsetmiterjoin%
\definecolor{currentfill}{rgb}{0.976471,0.823529,0.854902}%
\pgfsetfillcolor{currentfill}%
\pgfsetlinewidth{0.752812pt}%
\definecolor{currentstroke}{rgb}{0.000000,0.000000,0.000000}%
\pgfsetstrokecolor{currentstroke}%
\pgfsetdash{}{0pt}%
\pgfpathmoveto{\pgfqpoint{1.151210in}{1.189105in}}%
\pgfpathlineto{\pgfqpoint{1.307722in}{1.189105in}}%
\pgfpathlineto{\pgfqpoint{1.307722in}{1.459549in}}%
\pgfpathlineto{\pgfqpoint{1.151210in}{1.459549in}}%
\pgfpathlineto{\pgfqpoint{1.151210in}{1.189105in}}%
\pgfpathclose%
\pgfusepath{stroke,fill}%
\end{pgfscope}%
\begin{pgfscope}%
\pgfpathrectangle{\pgfqpoint{0.550713in}{0.127635in}}{\pgfqpoint{3.194133in}{2.297424in}}%
\pgfusepath{clip}%
\pgfsetbuttcap%
\pgfsetmiterjoin%
\definecolor{currentfill}{rgb}{0.000000,0.329412,0.623529}%
\pgfsetfillcolor{currentfill}%
\pgfsetlinewidth{0.752812pt}%
\definecolor{currentstroke}{rgb}{0.000000,0.000000,0.000000}%
\pgfsetstrokecolor{currentstroke}%
\pgfsetdash{}{0pt}%
\pgfpathmoveto{\pgfqpoint{1.390770in}{1.280029in}}%
\pgfpathlineto{\pgfqpoint{1.547282in}{1.280029in}}%
\pgfpathlineto{\pgfqpoint{1.547282in}{1.563193in}}%
\pgfpathlineto{\pgfqpoint{1.390770in}{1.563193in}}%
\pgfpathlineto{\pgfqpoint{1.390770in}{1.280029in}}%
\pgfpathclose%
\pgfusepath{stroke,fill}%
\end{pgfscope}%
\begin{pgfscope}%
\pgfpathrectangle{\pgfqpoint{0.550713in}{0.127635in}}{\pgfqpoint{3.194133in}{2.297424in}}%
\pgfusepath{clip}%
\pgfsetbuttcap%
\pgfsetmiterjoin%
\definecolor{currentfill}{rgb}{0.780392,0.866667,0.949020}%
\pgfsetfillcolor{currentfill}%
\pgfsetlinewidth{0.752812pt}%
\definecolor{currentstroke}{rgb}{0.000000,0.000000,0.000000}%
\pgfsetstrokecolor{currentstroke}%
\pgfsetdash{}{0pt}%
\pgfpathmoveto{\pgfqpoint{1.550476in}{1.257800in}}%
\pgfpathlineto{\pgfqpoint{1.706989in}{1.257800in}}%
\pgfpathlineto{\pgfqpoint{1.706989in}{1.658680in}}%
\pgfpathlineto{\pgfqpoint{1.550476in}{1.658680in}}%
\pgfpathlineto{\pgfqpoint{1.550476in}{1.257800in}}%
\pgfpathclose%
\pgfusepath{stroke,fill}%
\end{pgfscope}%
\begin{pgfscope}%
\pgfpathrectangle{\pgfqpoint{0.550713in}{0.127635in}}{\pgfqpoint{3.194133in}{2.297424in}}%
\pgfusepath{clip}%
\pgfsetbuttcap%
\pgfsetmiterjoin%
\definecolor{currentfill}{rgb}{0.000000,0.380392,0.396078}%
\pgfsetfillcolor{currentfill}%
\pgfsetlinewidth{0.752812pt}%
\definecolor{currentstroke}{rgb}{0.000000,0.000000,0.000000}%
\pgfsetstrokecolor{currentstroke}%
\pgfsetdash{}{0pt}%
\pgfpathmoveto{\pgfqpoint{1.790036in}{1.244754in}}%
\pgfpathlineto{\pgfqpoint{1.946549in}{1.244754in}}%
\pgfpathlineto{\pgfqpoint{1.946549in}{1.591209in}}%
\pgfpathlineto{\pgfqpoint{1.790036in}{1.591209in}}%
\pgfpathlineto{\pgfqpoint{1.790036in}{1.244754in}}%
\pgfpathclose%
\pgfusepath{stroke,fill}%
\end{pgfscope}%
\begin{pgfscope}%
\pgfpathrectangle{\pgfqpoint{0.550713in}{0.127635in}}{\pgfqpoint{3.194133in}{2.297424in}}%
\pgfusepath{clip}%
\pgfsetbuttcap%
\pgfsetmiterjoin%
\definecolor{currentfill}{rgb}{0.749020,0.815686,0.819608}%
\pgfsetfillcolor{currentfill}%
\pgfsetlinewidth{0.752812pt}%
\definecolor{currentstroke}{rgb}{0.000000,0.000000,0.000000}%
\pgfsetstrokecolor{currentstroke}%
\pgfsetdash{}{0pt}%
\pgfpathmoveto{\pgfqpoint{1.949743in}{1.343208in}}%
\pgfpathlineto{\pgfqpoint{2.106255in}{1.343208in}}%
\pgfpathlineto{\pgfqpoint{2.106255in}{1.617355in}}%
\pgfpathlineto{\pgfqpoint{1.949743in}{1.617355in}}%
\pgfpathlineto{\pgfqpoint{1.949743in}{1.343208in}}%
\pgfpathclose%
\pgfusepath{stroke,fill}%
\end{pgfscope}%
\begin{pgfscope}%
\pgfpathrectangle{\pgfqpoint{0.550713in}{0.127635in}}{\pgfqpoint{3.194133in}{2.297424in}}%
\pgfusepath{clip}%
\pgfsetbuttcap%
\pgfsetmiterjoin%
\definecolor{currentfill}{rgb}{0.380392,0.129412,0.345098}%
\pgfsetfillcolor{currentfill}%
\pgfsetlinewidth{0.752812pt}%
\definecolor{currentstroke}{rgb}{0.000000,0.000000,0.000000}%
\pgfsetstrokecolor{currentstroke}%
\pgfsetdash{}{0pt}%
\pgfpathmoveto{\pgfqpoint{2.189303in}{1.099069in}}%
\pgfpathlineto{\pgfqpoint{2.345815in}{1.099069in}}%
\pgfpathlineto{\pgfqpoint{2.345815in}{1.288640in}}%
\pgfpathlineto{\pgfqpoint{2.189303in}{1.288640in}}%
\pgfpathlineto{\pgfqpoint{2.189303in}{1.099069in}}%
\pgfpathclose%
\pgfusepath{stroke,fill}%
\end{pgfscope}%
\begin{pgfscope}%
\pgfpathrectangle{\pgfqpoint{0.550713in}{0.127635in}}{\pgfqpoint{3.194133in}{2.297424in}}%
\pgfusepath{clip}%
\pgfsetbuttcap%
\pgfsetmiterjoin%
\definecolor{currentfill}{rgb}{0.823529,0.752941,0.803922}%
\pgfsetfillcolor{currentfill}%
\pgfsetlinewidth{0.752812pt}%
\definecolor{currentstroke}{rgb}{0.000000,0.000000,0.000000}%
\pgfsetstrokecolor{currentstroke}%
\pgfsetdash{}{0pt}%
\pgfpathmoveto{\pgfqpoint{2.349010in}{1.238199in}}%
\pgfpathlineto{\pgfqpoint{2.505522in}{1.238199in}}%
\pgfpathlineto{\pgfqpoint{2.505522in}{1.471253in}}%
\pgfpathlineto{\pgfqpoint{2.349010in}{1.471253in}}%
\pgfpathlineto{\pgfqpoint{2.349010in}{1.238199in}}%
\pgfpathclose%
\pgfusepath{stroke,fill}%
\end{pgfscope}%
\begin{pgfscope}%
\pgfpathrectangle{\pgfqpoint{0.550713in}{0.127635in}}{\pgfqpoint{3.194133in}{2.297424in}}%
\pgfusepath{clip}%
\pgfsetbuttcap%
\pgfsetmiterjoin%
\definecolor{currentfill}{rgb}{0.964706,0.658824,0.000000}%
\pgfsetfillcolor{currentfill}%
\pgfsetlinewidth{0.752812pt}%
\definecolor{currentstroke}{rgb}{0.000000,0.000000,0.000000}%
\pgfsetstrokecolor{currentstroke}%
\pgfsetdash{}{0pt}%
\pgfpathmoveto{\pgfqpoint{2.588570in}{1.149610in}}%
\pgfpathlineto{\pgfqpoint{2.745082in}{1.149610in}}%
\pgfpathlineto{\pgfqpoint{2.745082in}{1.248540in}}%
\pgfpathlineto{\pgfqpoint{2.588570in}{1.248540in}}%
\pgfpathlineto{\pgfqpoint{2.588570in}{1.149610in}}%
\pgfpathclose%
\pgfusepath{stroke,fill}%
\end{pgfscope}%
\begin{pgfscope}%
\pgfpathrectangle{\pgfqpoint{0.550713in}{0.127635in}}{\pgfqpoint{3.194133in}{2.297424in}}%
\pgfusepath{clip}%
\pgfsetbuttcap%
\pgfsetmiterjoin%
\definecolor{currentfill}{rgb}{0.996078,0.917647,0.788235}%
\pgfsetfillcolor{currentfill}%
\pgfsetlinewidth{0.752812pt}%
\definecolor{currentstroke}{rgb}{0.000000,0.000000,0.000000}%
\pgfsetstrokecolor{currentstroke}%
\pgfsetdash{}{0pt}%
\pgfpathmoveto{\pgfqpoint{2.748276in}{1.235864in}}%
\pgfpathlineto{\pgfqpoint{2.904789in}{1.235864in}}%
\pgfpathlineto{\pgfqpoint{2.904789in}{1.414695in}}%
\pgfpathlineto{\pgfqpoint{2.748276in}{1.414695in}}%
\pgfpathlineto{\pgfqpoint{2.748276in}{1.235864in}}%
\pgfpathclose%
\pgfusepath{stroke,fill}%
\end{pgfscope}%
\begin{pgfscope}%
\pgfpathrectangle{\pgfqpoint{0.550713in}{0.127635in}}{\pgfqpoint{3.194133in}{2.297424in}}%
\pgfusepath{clip}%
\pgfsetbuttcap%
\pgfsetmiterjoin%
\definecolor{currentfill}{rgb}{0.341176,0.670588,0.152941}%
\pgfsetfillcolor{currentfill}%
\pgfsetlinewidth{0.752812pt}%
\definecolor{currentstroke}{rgb}{0.000000,0.000000,0.000000}%
\pgfsetstrokecolor{currentstroke}%
\pgfsetdash{}{0pt}%
\pgfpathmoveto{\pgfqpoint{2.987836in}{1.129976in}}%
\pgfpathlineto{\pgfqpoint{3.144349in}{1.129976in}}%
\pgfpathlineto{\pgfqpoint{3.144349in}{1.296988in}}%
\pgfpathlineto{\pgfqpoint{2.987836in}{1.296988in}}%
\pgfpathlineto{\pgfqpoint{2.987836in}{1.129976in}}%
\pgfpathclose%
\pgfusepath{stroke,fill}%
\end{pgfscope}%
\begin{pgfscope}%
\pgfpathrectangle{\pgfqpoint{0.550713in}{0.127635in}}{\pgfqpoint{3.194133in}{2.297424in}}%
\pgfusepath{clip}%
\pgfsetbuttcap%
\pgfsetmiterjoin%
\definecolor{currentfill}{rgb}{0.866667,0.921569,0.807843}%
\pgfsetfillcolor{currentfill}%
\pgfsetlinewidth{0.752812pt}%
\definecolor{currentstroke}{rgb}{0.000000,0.000000,0.000000}%
\pgfsetstrokecolor{currentstroke}%
\pgfsetdash{}{0pt}%
\pgfpathmoveto{\pgfqpoint{3.147543in}{1.173293in}}%
\pgfpathlineto{\pgfqpoint{3.304055in}{1.173293in}}%
\pgfpathlineto{\pgfqpoint{3.304055in}{1.289290in}}%
\pgfpathlineto{\pgfqpoint{3.147543in}{1.289290in}}%
\pgfpathlineto{\pgfqpoint{3.147543in}{1.173293in}}%
\pgfpathclose%
\pgfusepath{stroke,fill}%
\end{pgfscope}%
\begin{pgfscope}%
\pgfpathrectangle{\pgfqpoint{0.550713in}{0.127635in}}{\pgfqpoint{3.194133in}{2.297424in}}%
\pgfusepath{clip}%
\pgfsetbuttcap%
\pgfsetmiterjoin%
\definecolor{currentfill}{rgb}{0.478431,0.435294,0.674510}%
\pgfsetfillcolor{currentfill}%
\pgfsetlinewidth{0.752812pt}%
\definecolor{currentstroke}{rgb}{0.000000,0.000000,0.000000}%
\pgfsetstrokecolor{currentstroke}%
\pgfsetdash{}{0pt}%
\pgfpathmoveto{\pgfqpoint{3.387103in}{1.125794in}}%
\pgfpathlineto{\pgfqpoint{3.543615in}{1.125794in}}%
\pgfpathlineto{\pgfqpoint{3.543615in}{1.282275in}}%
\pgfpathlineto{\pgfqpoint{3.387103in}{1.282275in}}%
\pgfpathlineto{\pgfqpoint{3.387103in}{1.125794in}}%
\pgfpathclose%
\pgfusepath{stroke,fill}%
\end{pgfscope}%
\begin{pgfscope}%
\pgfpathrectangle{\pgfqpoint{0.550713in}{0.127635in}}{\pgfqpoint{3.194133in}{2.297424in}}%
\pgfusepath{clip}%
\pgfsetbuttcap%
\pgfsetmiterjoin%
\definecolor{currentfill}{rgb}{0.870588,0.854902,0.921569}%
\pgfsetfillcolor{currentfill}%
\pgfsetlinewidth{0.752812pt}%
\definecolor{currentstroke}{rgb}{0.000000,0.000000,0.000000}%
\pgfsetstrokecolor{currentstroke}%
\pgfsetdash{}{0pt}%
\pgfpathmoveto{\pgfqpoint{3.546809in}{1.199561in}}%
\pgfpathlineto{\pgfqpoint{3.703322in}{1.199561in}}%
\pgfpathlineto{\pgfqpoint{3.703322in}{1.329932in}}%
\pgfpathlineto{\pgfqpoint{3.546809in}{1.329932in}}%
\pgfpathlineto{\pgfqpoint{3.546809in}{1.199561in}}%
\pgfpathclose%
\pgfusepath{stroke,fill}%
\end{pgfscope}%
\begin{pgfscope}%
\pgfpathrectangle{\pgfqpoint{0.550713in}{0.127635in}}{\pgfqpoint{3.194133in}{2.297424in}}%
\pgfusepath{clip}%
\pgfsetbuttcap%
\pgfsetmiterjoin%
\definecolor{currentfill}{rgb}{0.000000,0.000000,0.000000}%
\pgfsetfillcolor{currentfill}%
\pgfsetlinewidth{0.376406pt}%
\definecolor{currentstroke}{rgb}{0.000000,0.000000,0.000000}%
\pgfsetstrokecolor{currentstroke}%
\pgfsetdash{}{0pt}%
\pgfpathmoveto{\pgfqpoint{0.750346in}{0.127635in}}%
\pgfpathlineto{\pgfqpoint{0.750346in}{0.127635in}}%
\pgfpathlineto{\pgfqpoint{0.750346in}{0.127635in}}%
\pgfpathlineto{\pgfqpoint{0.750346in}{0.127635in}}%
\pgfpathclose%
\pgfusepath{stroke,fill}%
\end{pgfscope}%
\begin{pgfscope}%
\pgfpathrectangle{\pgfqpoint{0.550713in}{0.127635in}}{\pgfqpoint{3.194133in}{2.297424in}}%
\pgfusepath{clip}%
\pgfsetbuttcap%
\pgfsetmiterjoin%
\definecolor{currentfill}{rgb}{0.813235,0.819118,0.822059}%
\pgfsetfillcolor{currentfill}%
\pgfsetlinewidth{0.376406pt}%
\definecolor{currentstroke}{rgb}{0.000000,0.000000,0.000000}%
\pgfsetstrokecolor{currentstroke}%
\pgfsetdash{}{0pt}%
\pgfpathmoveto{\pgfqpoint{0.750346in}{0.127635in}}%
\pgfpathlineto{\pgfqpoint{0.750346in}{0.127635in}}%
\pgfpathlineto{\pgfqpoint{0.750346in}{0.127635in}}%
\pgfpathlineto{\pgfqpoint{0.750346in}{0.127635in}}%
\pgfpathclose%
\pgfusepath{stroke,fill}%
\end{pgfscope}%
\begin{pgfscope}%
\pgfsetbuttcap%
\pgfsetroundjoin%
\definecolor{currentfill}{rgb}{0.000000,0.000000,0.000000}%
\pgfsetfillcolor{currentfill}%
\pgfsetlinewidth{0.803000pt}%
\definecolor{currentstroke}{rgb}{0.000000,0.000000,0.000000}%
\pgfsetstrokecolor{currentstroke}%
\pgfsetdash{}{0pt}%
\pgfsys@defobject{currentmarker}{\pgfqpoint{-0.048611in}{0.000000in}}{\pgfqpoint{-0.000000in}{0.000000in}}{%
\pgfpathmoveto{\pgfqpoint{-0.000000in}{0.000000in}}%
\pgfpathlineto{\pgfqpoint{-0.048611in}{0.000000in}}%
\pgfusepath{stroke,fill}%
}%
\begin{pgfscope}%
\pgfsys@transformshift{0.550713in}{0.127635in}%
\pgfsys@useobject{currentmarker}{}%
\end{pgfscope}%
\end{pgfscope}%
\begin{pgfscope}%
\definecolor{textcolor}{rgb}{0.000000,0.000000,0.000000}%
\pgfsetstrokecolor{textcolor}%
\pgfsetfillcolor{textcolor}%
\pgftext[x=0.384046in, y=0.079440in, left, base]{\color{textcolor}\rmfamily\fontsize{10.000000}{12.000000}\selectfont \(\displaystyle {0}\)}%
\end{pgfscope}%
\begin{pgfscope}%
\pgfsetbuttcap%
\pgfsetroundjoin%
\definecolor{currentfill}{rgb}{0.000000,0.000000,0.000000}%
\pgfsetfillcolor{currentfill}%
\pgfsetlinewidth{0.803000pt}%
\definecolor{currentstroke}{rgb}{0.000000,0.000000,0.000000}%
\pgfsetstrokecolor{currentstroke}%
\pgfsetdash{}{0pt}%
\pgfsys@defobject{currentmarker}{\pgfqpoint{-0.048611in}{0.000000in}}{\pgfqpoint{-0.000000in}{0.000000in}}{%
\pgfpathmoveto{\pgfqpoint{-0.000000in}{0.000000in}}%
\pgfpathlineto{\pgfqpoint{-0.048611in}{0.000000in}}%
\pgfusepath{stroke,fill}%
}%
\begin{pgfscope}%
\pgfsys@transformshift{0.550713in}{0.510539in}%
\pgfsys@useobject{currentmarker}{}%
\end{pgfscope}%
\end{pgfscope}%
\begin{pgfscope}%
\definecolor{textcolor}{rgb}{0.000000,0.000000,0.000000}%
\pgfsetstrokecolor{textcolor}%
\pgfsetfillcolor{textcolor}%
\pgftext[x=0.314601in, y=0.462344in, left, base]{\color{textcolor}\rmfamily\fontsize{10.000000}{12.000000}\selectfont \(\displaystyle {20}\)}%
\end{pgfscope}%
\begin{pgfscope}%
\pgfsetbuttcap%
\pgfsetroundjoin%
\definecolor{currentfill}{rgb}{0.000000,0.000000,0.000000}%
\pgfsetfillcolor{currentfill}%
\pgfsetlinewidth{0.803000pt}%
\definecolor{currentstroke}{rgb}{0.000000,0.000000,0.000000}%
\pgfsetstrokecolor{currentstroke}%
\pgfsetdash{}{0pt}%
\pgfsys@defobject{currentmarker}{\pgfqpoint{-0.048611in}{0.000000in}}{\pgfqpoint{-0.000000in}{0.000000in}}{%
\pgfpathmoveto{\pgfqpoint{-0.000000in}{0.000000in}}%
\pgfpathlineto{\pgfqpoint{-0.048611in}{0.000000in}}%
\pgfusepath{stroke,fill}%
}%
\begin{pgfscope}%
\pgfsys@transformshift{0.550713in}{0.893443in}%
\pgfsys@useobject{currentmarker}{}%
\end{pgfscope}%
\end{pgfscope}%
\begin{pgfscope}%
\definecolor{textcolor}{rgb}{0.000000,0.000000,0.000000}%
\pgfsetstrokecolor{textcolor}%
\pgfsetfillcolor{textcolor}%
\pgftext[x=0.314601in, y=0.845248in, left, base]{\color{textcolor}\rmfamily\fontsize{10.000000}{12.000000}\selectfont \(\displaystyle {40}\)}%
\end{pgfscope}%
\begin{pgfscope}%
\pgfsetbuttcap%
\pgfsetroundjoin%
\definecolor{currentfill}{rgb}{0.000000,0.000000,0.000000}%
\pgfsetfillcolor{currentfill}%
\pgfsetlinewidth{0.803000pt}%
\definecolor{currentstroke}{rgb}{0.000000,0.000000,0.000000}%
\pgfsetstrokecolor{currentstroke}%
\pgfsetdash{}{0pt}%
\pgfsys@defobject{currentmarker}{\pgfqpoint{-0.048611in}{0.000000in}}{\pgfqpoint{-0.000000in}{0.000000in}}{%
\pgfpathmoveto{\pgfqpoint{-0.000000in}{0.000000in}}%
\pgfpathlineto{\pgfqpoint{-0.048611in}{0.000000in}}%
\pgfusepath{stroke,fill}%
}%
\begin{pgfscope}%
\pgfsys@transformshift{0.550713in}{1.276347in}%
\pgfsys@useobject{currentmarker}{}%
\end{pgfscope}%
\end{pgfscope}%
\begin{pgfscope}%
\definecolor{textcolor}{rgb}{0.000000,0.000000,0.000000}%
\pgfsetstrokecolor{textcolor}%
\pgfsetfillcolor{textcolor}%
\pgftext[x=0.314601in, y=1.228152in, left, base]{\color{textcolor}\rmfamily\fontsize{10.000000}{12.000000}\selectfont \(\displaystyle {60}\)}%
\end{pgfscope}%
\begin{pgfscope}%
\pgfsetbuttcap%
\pgfsetroundjoin%
\definecolor{currentfill}{rgb}{0.000000,0.000000,0.000000}%
\pgfsetfillcolor{currentfill}%
\pgfsetlinewidth{0.803000pt}%
\definecolor{currentstroke}{rgb}{0.000000,0.000000,0.000000}%
\pgfsetstrokecolor{currentstroke}%
\pgfsetdash{}{0pt}%
\pgfsys@defobject{currentmarker}{\pgfqpoint{-0.048611in}{0.000000in}}{\pgfqpoint{-0.000000in}{0.000000in}}{%
\pgfpathmoveto{\pgfqpoint{-0.000000in}{0.000000in}}%
\pgfpathlineto{\pgfqpoint{-0.048611in}{0.000000in}}%
\pgfusepath{stroke,fill}%
}%
\begin{pgfscope}%
\pgfsys@transformshift{0.550713in}{1.659251in}%
\pgfsys@useobject{currentmarker}{}%
\end{pgfscope}%
\end{pgfscope}%
\begin{pgfscope}%
\definecolor{textcolor}{rgb}{0.000000,0.000000,0.000000}%
\pgfsetstrokecolor{textcolor}%
\pgfsetfillcolor{textcolor}%
\pgftext[x=0.314601in, y=1.611056in, left, base]{\color{textcolor}\rmfamily\fontsize{10.000000}{12.000000}\selectfont \(\displaystyle {80}\)}%
\end{pgfscope}%
\begin{pgfscope}%
\pgfsetbuttcap%
\pgfsetroundjoin%
\definecolor{currentfill}{rgb}{0.000000,0.000000,0.000000}%
\pgfsetfillcolor{currentfill}%
\pgfsetlinewidth{0.803000pt}%
\definecolor{currentstroke}{rgb}{0.000000,0.000000,0.000000}%
\pgfsetstrokecolor{currentstroke}%
\pgfsetdash{}{0pt}%
\pgfsys@defobject{currentmarker}{\pgfqpoint{-0.048611in}{0.000000in}}{\pgfqpoint{-0.000000in}{0.000000in}}{%
\pgfpathmoveto{\pgfqpoint{-0.000000in}{0.000000in}}%
\pgfpathlineto{\pgfqpoint{-0.048611in}{0.000000in}}%
\pgfusepath{stroke,fill}%
}%
\begin{pgfscope}%
\pgfsys@transformshift{0.550713in}{2.042155in}%
\pgfsys@useobject{currentmarker}{}%
\end{pgfscope}%
\end{pgfscope}%
\begin{pgfscope}%
\definecolor{textcolor}{rgb}{0.000000,0.000000,0.000000}%
\pgfsetstrokecolor{textcolor}%
\pgfsetfillcolor{textcolor}%
\pgftext[x=0.245156in, y=1.993960in, left, base]{\color{textcolor}\rmfamily\fontsize{10.000000}{12.000000}\selectfont \(\displaystyle {100}\)}%
\end{pgfscope}%
\begin{pgfscope}%
\pgfsetbuttcap%
\pgfsetroundjoin%
\definecolor{currentfill}{rgb}{0.000000,0.000000,0.000000}%
\pgfsetfillcolor{currentfill}%
\pgfsetlinewidth{0.803000pt}%
\definecolor{currentstroke}{rgb}{0.000000,0.000000,0.000000}%
\pgfsetstrokecolor{currentstroke}%
\pgfsetdash{}{0pt}%
\pgfsys@defobject{currentmarker}{\pgfqpoint{-0.048611in}{0.000000in}}{\pgfqpoint{-0.000000in}{0.000000in}}{%
\pgfpathmoveto{\pgfqpoint{-0.000000in}{0.000000in}}%
\pgfpathlineto{\pgfqpoint{-0.048611in}{0.000000in}}%
\pgfusepath{stroke,fill}%
}%
\begin{pgfscope}%
\pgfsys@transformshift{0.550713in}{2.425059in}%
\pgfsys@useobject{currentmarker}{}%
\end{pgfscope}%
\end{pgfscope}%
\begin{pgfscope}%
\definecolor{textcolor}{rgb}{0.000000,0.000000,0.000000}%
\pgfsetstrokecolor{textcolor}%
\pgfsetfillcolor{textcolor}%
\pgftext[x=0.245156in, y=2.376864in, left, base]{\color{textcolor}\rmfamily\fontsize{10.000000}{12.000000}\selectfont \(\displaystyle {120}\)}%
\end{pgfscope}%
\begin{pgfscope}%
\definecolor{textcolor}{rgb}{0.000000,0.000000,0.000000}%
\pgfsetstrokecolor{textcolor}%
\pgfsetfillcolor{textcolor}%
\pgftext[x=0.189601in,y=1.276347in,,bottom,rotate=90.000000]{\color{textcolor}\rmfamily\fontsize{10.000000}{12.000000}\selectfont \(\displaystyle R_T\)}%
\end{pgfscope}%
\begin{pgfscope}%
\pgfpathrectangle{\pgfqpoint{0.550713in}{0.127635in}}{\pgfqpoint{3.194133in}{2.297424in}}%
\pgfusepath{clip}%
\pgfsetbuttcap%
\pgfsetroundjoin%
\pgfsetlinewidth{0.501875pt}%
\definecolor{currentstroke}{rgb}{0.392157,0.396078,0.403922}%
\pgfsetstrokecolor{currentstroke}%
\pgfsetdash{}{0pt}%
\pgfpathmoveto{\pgfqpoint{0.949979in}{0.127635in}}%
\pgfpathlineto{\pgfqpoint{0.949979in}{2.425059in}}%
\pgfusepath{stroke}%
\end{pgfscope}%
\begin{pgfscope}%
\pgfpathrectangle{\pgfqpoint{0.550713in}{0.127635in}}{\pgfqpoint{3.194133in}{2.297424in}}%
\pgfusepath{clip}%
\pgfsetbuttcap%
\pgfsetroundjoin%
\pgfsetlinewidth{0.501875pt}%
\definecolor{currentstroke}{rgb}{0.392157,0.396078,0.403922}%
\pgfsetstrokecolor{currentstroke}%
\pgfsetdash{}{0pt}%
\pgfpathmoveto{\pgfqpoint{1.349246in}{0.127635in}}%
\pgfpathlineto{\pgfqpoint{1.349246in}{2.425059in}}%
\pgfusepath{stroke}%
\end{pgfscope}%
\begin{pgfscope}%
\pgfpathrectangle{\pgfqpoint{0.550713in}{0.127635in}}{\pgfqpoint{3.194133in}{2.297424in}}%
\pgfusepath{clip}%
\pgfsetbuttcap%
\pgfsetroundjoin%
\pgfsetlinewidth{0.501875pt}%
\definecolor{currentstroke}{rgb}{0.392157,0.396078,0.403922}%
\pgfsetstrokecolor{currentstroke}%
\pgfsetdash{}{0pt}%
\pgfpathmoveto{\pgfqpoint{1.748513in}{0.127635in}}%
\pgfpathlineto{\pgfqpoint{1.748513in}{2.425059in}}%
\pgfusepath{stroke}%
\end{pgfscope}%
\begin{pgfscope}%
\pgfpathrectangle{\pgfqpoint{0.550713in}{0.127635in}}{\pgfqpoint{3.194133in}{2.297424in}}%
\pgfusepath{clip}%
\pgfsetbuttcap%
\pgfsetroundjoin%
\pgfsetlinewidth{0.501875pt}%
\definecolor{currentstroke}{rgb}{0.392157,0.396078,0.403922}%
\pgfsetstrokecolor{currentstroke}%
\pgfsetdash{}{0pt}%
\pgfpathmoveto{\pgfqpoint{2.147779in}{0.127635in}}%
\pgfpathlineto{\pgfqpoint{2.147779in}{2.425059in}}%
\pgfusepath{stroke}%
\end{pgfscope}%
\begin{pgfscope}%
\pgfpathrectangle{\pgfqpoint{0.550713in}{0.127635in}}{\pgfqpoint{3.194133in}{2.297424in}}%
\pgfusepath{clip}%
\pgfsetbuttcap%
\pgfsetroundjoin%
\pgfsetlinewidth{0.501875pt}%
\definecolor{currentstroke}{rgb}{0.392157,0.396078,0.403922}%
\pgfsetstrokecolor{currentstroke}%
\pgfsetdash{}{0pt}%
\pgfpathmoveto{\pgfqpoint{2.547046in}{0.127635in}}%
\pgfpathlineto{\pgfqpoint{2.547046in}{2.425059in}}%
\pgfusepath{stroke}%
\end{pgfscope}%
\begin{pgfscope}%
\pgfpathrectangle{\pgfqpoint{0.550713in}{0.127635in}}{\pgfqpoint{3.194133in}{2.297424in}}%
\pgfusepath{clip}%
\pgfsetbuttcap%
\pgfsetroundjoin%
\pgfsetlinewidth{0.501875pt}%
\definecolor{currentstroke}{rgb}{0.392157,0.396078,0.403922}%
\pgfsetstrokecolor{currentstroke}%
\pgfsetdash{}{0pt}%
\pgfpathmoveto{\pgfqpoint{2.946312in}{0.127635in}}%
\pgfpathlineto{\pgfqpoint{2.946312in}{2.425059in}}%
\pgfusepath{stroke}%
\end{pgfscope}%
\begin{pgfscope}%
\pgfpathrectangle{\pgfqpoint{0.550713in}{0.127635in}}{\pgfqpoint{3.194133in}{2.297424in}}%
\pgfusepath{clip}%
\pgfsetbuttcap%
\pgfsetroundjoin%
\pgfsetlinewidth{0.501875pt}%
\definecolor{currentstroke}{rgb}{0.392157,0.396078,0.403922}%
\pgfsetstrokecolor{currentstroke}%
\pgfsetdash{}{0pt}%
\pgfpathmoveto{\pgfqpoint{3.345579in}{0.127635in}}%
\pgfpathlineto{\pgfqpoint{3.345579in}{2.425059in}}%
\pgfusepath{stroke}%
\end{pgfscope}%
\begin{pgfscope}%
\pgfpathrectangle{\pgfqpoint{0.550713in}{0.127635in}}{\pgfqpoint{3.194133in}{2.297424in}}%
\pgfusepath{clip}%
\pgfsetbuttcap%
\pgfsetroundjoin%
\pgfsetlinewidth{0.853187pt}%
\definecolor{currentstroke}{rgb}{0.392157,0.396078,0.403922}%
\pgfsetstrokecolor{currentstroke}%
\pgfsetdash{{3.145000pt}{1.360000pt}}{0.000000pt}%
\pgfusepath{stroke}%
\end{pgfscope}%
\begin{pgfscope}%
\pgfpathrectangle{\pgfqpoint{0.550713in}{0.127635in}}{\pgfqpoint{3.194133in}{2.297424in}}%
\pgfusepath{clip}%
\pgfsetrectcap%
\pgfsetroundjoin%
\pgfsetlinewidth{0.752812pt}%
\definecolor{currentstroke}{rgb}{0.000000,0.000000,0.000000}%
\pgfsetstrokecolor{currentstroke}%
\pgfsetdash{}{0pt}%
\pgfpathmoveto{\pgfqpoint{0.670493in}{1.259009in}}%
\pgfpathlineto{\pgfqpoint{0.670493in}{0.842703in}}%
\pgfusepath{stroke}%
\end{pgfscope}%
\begin{pgfscope}%
\pgfpathrectangle{\pgfqpoint{0.550713in}{0.127635in}}{\pgfqpoint{3.194133in}{2.297424in}}%
\pgfusepath{clip}%
\pgfsetrectcap%
\pgfsetroundjoin%
\pgfsetlinewidth{0.752812pt}%
\definecolor{currentstroke}{rgb}{0.000000,0.000000,0.000000}%
\pgfsetstrokecolor{currentstroke}%
\pgfsetdash{}{0pt}%
\pgfpathmoveto{\pgfqpoint{0.670493in}{1.540592in}}%
\pgfpathlineto{\pgfqpoint{0.670493in}{1.838025in}}%
\pgfusepath{stroke}%
\end{pgfscope}%
\begin{pgfscope}%
\pgfpathrectangle{\pgfqpoint{0.550713in}{0.127635in}}{\pgfqpoint{3.194133in}{2.297424in}}%
\pgfusepath{clip}%
\pgfsetrectcap%
\pgfsetroundjoin%
\pgfsetlinewidth{0.752812pt}%
\definecolor{currentstroke}{rgb}{0.000000,0.000000,0.000000}%
\pgfsetstrokecolor{currentstroke}%
\pgfsetdash{}{0pt}%
\pgfpathmoveto{\pgfqpoint{0.631364in}{0.842703in}}%
\pgfpathlineto{\pgfqpoint{0.709621in}{0.842703in}}%
\pgfusepath{stroke}%
\end{pgfscope}%
\begin{pgfscope}%
\pgfpathrectangle{\pgfqpoint{0.550713in}{0.127635in}}{\pgfqpoint{3.194133in}{2.297424in}}%
\pgfusepath{clip}%
\pgfsetrectcap%
\pgfsetroundjoin%
\pgfsetlinewidth{0.752812pt}%
\definecolor{currentstroke}{rgb}{0.000000,0.000000,0.000000}%
\pgfsetstrokecolor{currentstroke}%
\pgfsetdash{}{0pt}%
\pgfpathmoveto{\pgfqpoint{0.631364in}{1.838025in}}%
\pgfpathlineto{\pgfqpoint{0.709621in}{1.838025in}}%
\pgfusepath{stroke}%
\end{pgfscope}%
\begin{pgfscope}%
\pgfpathrectangle{\pgfqpoint{0.550713in}{0.127635in}}{\pgfqpoint{3.194133in}{2.297424in}}%
\pgfusepath{clip}%
\pgfsetbuttcap%
\pgfsetmiterjoin%
\definecolor{currentfill}{rgb}{0.000000,0.000000,0.000000}%
\pgfsetfillcolor{currentfill}%
\pgfsetlinewidth{1.003750pt}%
\definecolor{currentstroke}{rgb}{0.000000,0.000000,0.000000}%
\pgfsetstrokecolor{currentstroke}%
\pgfsetdash{}{0pt}%
\pgfsys@defobject{currentmarker}{\pgfqpoint{-0.011785in}{-0.019642in}}{\pgfqpoint{0.011785in}{0.019642in}}{%
\pgfpathmoveto{\pgfqpoint{-0.000000in}{-0.019642in}}%
\pgfpathlineto{\pgfqpoint{0.011785in}{0.000000in}}%
\pgfpathlineto{\pgfqpoint{0.000000in}{0.019642in}}%
\pgfpathlineto{\pgfqpoint{-0.011785in}{0.000000in}}%
\pgfpathclose%
\pgfusepath{stroke,fill}%
}%
\begin{pgfscope}%
\pgfsys@transformshift{0.670493in}{0.800389in}%
\pgfsys@useobject{currentmarker}{}%
\end{pgfscope}%
\begin{pgfscope}%
\pgfsys@transformshift{0.670493in}{0.836213in}%
\pgfsys@useobject{currentmarker}{}%
\end{pgfscope}%
\begin{pgfscope}%
\pgfsys@transformshift{0.670493in}{0.800049in}%
\pgfsys@useobject{currentmarker}{}%
\end{pgfscope}%
\begin{pgfscope}%
\pgfsys@transformshift{0.670493in}{2.208689in}%
\pgfsys@useobject{currentmarker}{}%
\end{pgfscope}%
\begin{pgfscope}%
\pgfsys@transformshift{0.670493in}{2.070154in}%
\pgfsys@useobject{currentmarker}{}%
\end{pgfscope}%
\begin{pgfscope}%
\pgfsys@transformshift{0.670493in}{2.179971in}%
\pgfsys@useobject{currentmarker}{}%
\end{pgfscope}%
\end{pgfscope}%
\begin{pgfscope}%
\pgfpathrectangle{\pgfqpoint{0.550713in}{0.127635in}}{\pgfqpoint{3.194133in}{2.297424in}}%
\pgfusepath{clip}%
\pgfsetrectcap%
\pgfsetroundjoin%
\pgfsetlinewidth{0.752812pt}%
\definecolor{currentstroke}{rgb}{0.000000,0.000000,0.000000}%
\pgfsetstrokecolor{currentstroke}%
\pgfsetdash{}{0pt}%
\pgfpathmoveto{\pgfqpoint{0.830199in}{1.206496in}}%
\pgfpathlineto{\pgfqpoint{0.830199in}{0.837103in}}%
\pgfusepath{stroke}%
\end{pgfscope}%
\begin{pgfscope}%
\pgfpathrectangle{\pgfqpoint{0.550713in}{0.127635in}}{\pgfqpoint{3.194133in}{2.297424in}}%
\pgfusepath{clip}%
\pgfsetrectcap%
\pgfsetroundjoin%
\pgfsetlinewidth{0.752812pt}%
\definecolor{currentstroke}{rgb}{0.000000,0.000000,0.000000}%
\pgfsetstrokecolor{currentstroke}%
\pgfsetdash{}{0pt}%
\pgfpathmoveto{\pgfqpoint{0.830199in}{1.457729in}}%
\pgfpathlineto{\pgfqpoint{0.830199in}{1.601716in}}%
\pgfusepath{stroke}%
\end{pgfscope}%
\begin{pgfscope}%
\pgfpathrectangle{\pgfqpoint{0.550713in}{0.127635in}}{\pgfqpoint{3.194133in}{2.297424in}}%
\pgfusepath{clip}%
\pgfsetrectcap%
\pgfsetroundjoin%
\pgfsetlinewidth{0.752812pt}%
\definecolor{currentstroke}{rgb}{0.000000,0.000000,0.000000}%
\pgfsetstrokecolor{currentstroke}%
\pgfsetdash{}{0pt}%
\pgfpathmoveto{\pgfqpoint{0.791071in}{0.837103in}}%
\pgfpathlineto{\pgfqpoint{0.869327in}{0.837103in}}%
\pgfusepath{stroke}%
\end{pgfscope}%
\begin{pgfscope}%
\pgfpathrectangle{\pgfqpoint{0.550713in}{0.127635in}}{\pgfqpoint{3.194133in}{2.297424in}}%
\pgfusepath{clip}%
\pgfsetrectcap%
\pgfsetroundjoin%
\pgfsetlinewidth{0.752812pt}%
\definecolor{currentstroke}{rgb}{0.000000,0.000000,0.000000}%
\pgfsetstrokecolor{currentstroke}%
\pgfsetdash{}{0pt}%
\pgfpathmoveto{\pgfqpoint{0.791071in}{1.601716in}}%
\pgfpathlineto{\pgfqpoint{0.869327in}{1.601716in}}%
\pgfusepath{stroke}%
\end{pgfscope}%
\begin{pgfscope}%
\pgfpathrectangle{\pgfqpoint{0.550713in}{0.127635in}}{\pgfqpoint{3.194133in}{2.297424in}}%
\pgfusepath{clip}%
\pgfsetbuttcap%
\pgfsetmiterjoin%
\definecolor{currentfill}{rgb}{0.000000,0.000000,0.000000}%
\pgfsetfillcolor{currentfill}%
\pgfsetlinewidth{1.003750pt}%
\definecolor{currentstroke}{rgb}{0.000000,0.000000,0.000000}%
\pgfsetstrokecolor{currentstroke}%
\pgfsetdash{}{0pt}%
\pgfsys@defobject{currentmarker}{\pgfqpoint{-0.011785in}{-0.019642in}}{\pgfqpoint{0.011785in}{0.019642in}}{%
\pgfpathmoveto{\pgfqpoint{-0.000000in}{-0.019642in}}%
\pgfpathlineto{\pgfqpoint{0.011785in}{0.000000in}}%
\pgfpathlineto{\pgfqpoint{0.000000in}{0.019642in}}%
\pgfpathlineto{\pgfqpoint{-0.011785in}{0.000000in}}%
\pgfpathclose%
\pgfusepath{stroke,fill}%
}%
\begin{pgfscope}%
\pgfsys@transformshift{0.830199in}{0.799781in}%
\pgfsys@useobject{currentmarker}{}%
\end{pgfscope}%
\begin{pgfscope}%
\pgfsys@transformshift{0.830199in}{0.824521in}%
\pgfsys@useobject{currentmarker}{}%
\end{pgfscope}%
\end{pgfscope}%
\begin{pgfscope}%
\pgfpathrectangle{\pgfqpoint{0.550713in}{0.127635in}}{\pgfqpoint{3.194133in}{2.297424in}}%
\pgfusepath{clip}%
\pgfsetrectcap%
\pgfsetroundjoin%
\pgfsetlinewidth{0.752812pt}%
\definecolor{currentstroke}{rgb}{0.000000,0.000000,0.000000}%
\pgfsetstrokecolor{currentstroke}%
\pgfsetdash{}{0pt}%
\pgfpathmoveto{\pgfqpoint{1.069759in}{1.226382in}}%
\pgfpathlineto{\pgfqpoint{1.069759in}{0.752437in}}%
\pgfusepath{stroke}%
\end{pgfscope}%
\begin{pgfscope}%
\pgfpathrectangle{\pgfqpoint{0.550713in}{0.127635in}}{\pgfqpoint{3.194133in}{2.297424in}}%
\pgfusepath{clip}%
\pgfsetrectcap%
\pgfsetroundjoin%
\pgfsetlinewidth{0.752812pt}%
\definecolor{currentstroke}{rgb}{0.000000,0.000000,0.000000}%
\pgfsetstrokecolor{currentstroke}%
\pgfsetdash{}{0pt}%
\pgfpathmoveto{\pgfqpoint{1.069759in}{1.624032in}}%
\pgfpathlineto{\pgfqpoint{1.069759in}{2.058006in}}%
\pgfusepath{stroke}%
\end{pgfscope}%
\begin{pgfscope}%
\pgfpathrectangle{\pgfqpoint{0.550713in}{0.127635in}}{\pgfqpoint{3.194133in}{2.297424in}}%
\pgfusepath{clip}%
\pgfsetrectcap%
\pgfsetroundjoin%
\pgfsetlinewidth{0.752812pt}%
\definecolor{currentstroke}{rgb}{0.000000,0.000000,0.000000}%
\pgfsetstrokecolor{currentstroke}%
\pgfsetdash{}{0pt}%
\pgfpathmoveto{\pgfqpoint{1.030631in}{0.752437in}}%
\pgfpathlineto{\pgfqpoint{1.108887in}{0.752437in}}%
\pgfusepath{stroke}%
\end{pgfscope}%
\begin{pgfscope}%
\pgfpathrectangle{\pgfqpoint{0.550713in}{0.127635in}}{\pgfqpoint{3.194133in}{2.297424in}}%
\pgfusepath{clip}%
\pgfsetrectcap%
\pgfsetroundjoin%
\pgfsetlinewidth{0.752812pt}%
\definecolor{currentstroke}{rgb}{0.000000,0.000000,0.000000}%
\pgfsetstrokecolor{currentstroke}%
\pgfsetdash{}{0pt}%
\pgfpathmoveto{\pgfqpoint{1.030631in}{2.058006in}}%
\pgfpathlineto{\pgfqpoint{1.108887in}{2.058006in}}%
\pgfusepath{stroke}%
\end{pgfscope}%
\begin{pgfscope}%
\pgfpathrectangle{\pgfqpoint{0.550713in}{0.127635in}}{\pgfqpoint{3.194133in}{2.297424in}}%
\pgfusepath{clip}%
\pgfsetbuttcap%
\pgfsetmiterjoin%
\definecolor{currentfill}{rgb}{0.000000,0.000000,0.000000}%
\pgfsetfillcolor{currentfill}%
\pgfsetlinewidth{1.003750pt}%
\definecolor{currentstroke}{rgb}{0.000000,0.000000,0.000000}%
\pgfsetstrokecolor{currentstroke}%
\pgfsetdash{}{0pt}%
\pgfsys@defobject{currentmarker}{\pgfqpoint{-0.011785in}{-0.019642in}}{\pgfqpoint{0.011785in}{0.019642in}}{%
\pgfpathmoveto{\pgfqpoint{-0.000000in}{-0.019642in}}%
\pgfpathlineto{\pgfqpoint{0.011785in}{0.000000in}}%
\pgfpathlineto{\pgfqpoint{0.000000in}{0.019642in}}%
\pgfpathlineto{\pgfqpoint{-0.011785in}{0.000000in}}%
\pgfpathclose%
\pgfusepath{stroke,fill}%
}%
\begin{pgfscope}%
\pgfsys@transformshift{1.069759in}{2.232086in}%
\pgfsys@useobject{currentmarker}{}%
\end{pgfscope}%
\end{pgfscope}%
\begin{pgfscope}%
\pgfpathrectangle{\pgfqpoint{0.550713in}{0.127635in}}{\pgfqpoint{3.194133in}{2.297424in}}%
\pgfusepath{clip}%
\pgfsetrectcap%
\pgfsetroundjoin%
\pgfsetlinewidth{0.752812pt}%
\definecolor{currentstroke}{rgb}{0.000000,0.000000,0.000000}%
\pgfsetstrokecolor{currentstroke}%
\pgfsetdash{}{0pt}%
\pgfpathmoveto{\pgfqpoint{1.229466in}{1.189105in}}%
\pgfpathlineto{\pgfqpoint{1.229466in}{0.809275in}}%
\pgfusepath{stroke}%
\end{pgfscope}%
\begin{pgfscope}%
\pgfpathrectangle{\pgfqpoint{0.550713in}{0.127635in}}{\pgfqpoint{3.194133in}{2.297424in}}%
\pgfusepath{clip}%
\pgfsetrectcap%
\pgfsetroundjoin%
\pgfsetlinewidth{0.752812pt}%
\definecolor{currentstroke}{rgb}{0.000000,0.000000,0.000000}%
\pgfsetstrokecolor{currentstroke}%
\pgfsetdash{}{0pt}%
\pgfpathmoveto{\pgfqpoint{1.229466in}{1.459549in}}%
\pgfpathlineto{\pgfqpoint{1.229466in}{1.675710in}}%
\pgfusepath{stroke}%
\end{pgfscope}%
\begin{pgfscope}%
\pgfpathrectangle{\pgfqpoint{0.550713in}{0.127635in}}{\pgfqpoint{3.194133in}{2.297424in}}%
\pgfusepath{clip}%
\pgfsetrectcap%
\pgfsetroundjoin%
\pgfsetlinewidth{0.752812pt}%
\definecolor{currentstroke}{rgb}{0.000000,0.000000,0.000000}%
\pgfsetstrokecolor{currentstroke}%
\pgfsetdash{}{0pt}%
\pgfpathmoveto{\pgfqpoint{1.190338in}{0.809275in}}%
\pgfpathlineto{\pgfqpoint{1.268594in}{0.809275in}}%
\pgfusepath{stroke}%
\end{pgfscope}%
\begin{pgfscope}%
\pgfpathrectangle{\pgfqpoint{0.550713in}{0.127635in}}{\pgfqpoint{3.194133in}{2.297424in}}%
\pgfusepath{clip}%
\pgfsetrectcap%
\pgfsetroundjoin%
\pgfsetlinewidth{0.752812pt}%
\definecolor{currentstroke}{rgb}{0.000000,0.000000,0.000000}%
\pgfsetstrokecolor{currentstroke}%
\pgfsetdash{}{0pt}%
\pgfpathmoveto{\pgfqpoint{1.190338in}{1.675710in}}%
\pgfpathlineto{\pgfqpoint{1.268594in}{1.675710in}}%
\pgfusepath{stroke}%
\end{pgfscope}%
\begin{pgfscope}%
\pgfpathrectangle{\pgfqpoint{0.550713in}{0.127635in}}{\pgfqpoint{3.194133in}{2.297424in}}%
\pgfusepath{clip}%
\pgfsetrectcap%
\pgfsetroundjoin%
\pgfsetlinewidth{0.752812pt}%
\definecolor{currentstroke}{rgb}{0.000000,0.000000,0.000000}%
\pgfsetstrokecolor{currentstroke}%
\pgfsetdash{}{0pt}%
\pgfpathmoveto{\pgfqpoint{1.469026in}{1.280029in}}%
\pgfpathlineto{\pgfqpoint{1.469026in}{0.900997in}}%
\pgfusepath{stroke}%
\end{pgfscope}%
\begin{pgfscope}%
\pgfpathrectangle{\pgfqpoint{0.550713in}{0.127635in}}{\pgfqpoint{3.194133in}{2.297424in}}%
\pgfusepath{clip}%
\pgfsetrectcap%
\pgfsetroundjoin%
\pgfsetlinewidth{0.752812pt}%
\definecolor{currentstroke}{rgb}{0.000000,0.000000,0.000000}%
\pgfsetstrokecolor{currentstroke}%
\pgfsetdash{}{0pt}%
\pgfpathmoveto{\pgfqpoint{1.469026in}{1.563193in}}%
\pgfpathlineto{\pgfqpoint{1.469026in}{1.880263in}}%
\pgfusepath{stroke}%
\end{pgfscope}%
\begin{pgfscope}%
\pgfpathrectangle{\pgfqpoint{0.550713in}{0.127635in}}{\pgfqpoint{3.194133in}{2.297424in}}%
\pgfusepath{clip}%
\pgfsetrectcap%
\pgfsetroundjoin%
\pgfsetlinewidth{0.752812pt}%
\definecolor{currentstroke}{rgb}{0.000000,0.000000,0.000000}%
\pgfsetstrokecolor{currentstroke}%
\pgfsetdash{}{0pt}%
\pgfpathmoveto{\pgfqpoint{1.429898in}{0.900997in}}%
\pgfpathlineto{\pgfqpoint{1.508154in}{0.900997in}}%
\pgfusepath{stroke}%
\end{pgfscope}%
\begin{pgfscope}%
\pgfpathrectangle{\pgfqpoint{0.550713in}{0.127635in}}{\pgfqpoint{3.194133in}{2.297424in}}%
\pgfusepath{clip}%
\pgfsetrectcap%
\pgfsetroundjoin%
\pgfsetlinewidth{0.752812pt}%
\definecolor{currentstroke}{rgb}{0.000000,0.000000,0.000000}%
\pgfsetstrokecolor{currentstroke}%
\pgfsetdash{}{0pt}%
\pgfpathmoveto{\pgfqpoint{1.429898in}{1.880263in}}%
\pgfpathlineto{\pgfqpoint{1.508154in}{1.880263in}}%
\pgfusepath{stroke}%
\end{pgfscope}%
\begin{pgfscope}%
\pgfpathrectangle{\pgfqpoint{0.550713in}{0.127635in}}{\pgfqpoint{3.194133in}{2.297424in}}%
\pgfusepath{clip}%
\pgfsetbuttcap%
\pgfsetmiterjoin%
\definecolor{currentfill}{rgb}{0.000000,0.000000,0.000000}%
\pgfsetfillcolor{currentfill}%
\pgfsetlinewidth{1.003750pt}%
\definecolor{currentstroke}{rgb}{0.000000,0.000000,0.000000}%
\pgfsetstrokecolor{currentstroke}%
\pgfsetdash{}{0pt}%
\pgfsys@defobject{currentmarker}{\pgfqpoint{-0.011785in}{-0.019642in}}{\pgfqpoint{0.011785in}{0.019642in}}{%
\pgfpathmoveto{\pgfqpoint{-0.000000in}{-0.019642in}}%
\pgfpathlineto{\pgfqpoint{0.011785in}{0.000000in}}%
\pgfpathlineto{\pgfqpoint{0.000000in}{0.019642in}}%
\pgfpathlineto{\pgfqpoint{-0.011785in}{0.000000in}}%
\pgfpathclose%
\pgfusepath{stroke,fill}%
}%
\begin{pgfscope}%
\pgfsys@transformshift{1.469026in}{0.795388in}%
\pgfsys@useobject{currentmarker}{}%
\end{pgfscope}%
\begin{pgfscope}%
\pgfsys@transformshift{1.469026in}{2.127945in}%
\pgfsys@useobject{currentmarker}{}%
\end{pgfscope}%
\end{pgfscope}%
\begin{pgfscope}%
\pgfpathrectangle{\pgfqpoint{0.550713in}{0.127635in}}{\pgfqpoint{3.194133in}{2.297424in}}%
\pgfusepath{clip}%
\pgfsetrectcap%
\pgfsetroundjoin%
\pgfsetlinewidth{0.752812pt}%
\definecolor{currentstroke}{rgb}{0.000000,0.000000,0.000000}%
\pgfsetstrokecolor{currentstroke}%
\pgfsetdash{}{0pt}%
\pgfpathmoveto{\pgfqpoint{1.628733in}{1.257800in}}%
\pgfpathlineto{\pgfqpoint{1.628733in}{0.867470in}}%
\pgfusepath{stroke}%
\end{pgfscope}%
\begin{pgfscope}%
\pgfpathrectangle{\pgfqpoint{0.550713in}{0.127635in}}{\pgfqpoint{3.194133in}{2.297424in}}%
\pgfusepath{clip}%
\pgfsetrectcap%
\pgfsetroundjoin%
\pgfsetlinewidth{0.752812pt}%
\definecolor{currentstroke}{rgb}{0.000000,0.000000,0.000000}%
\pgfsetstrokecolor{currentstroke}%
\pgfsetdash{}{0pt}%
\pgfpathmoveto{\pgfqpoint{1.628733in}{1.658680in}}%
\pgfpathlineto{\pgfqpoint{1.628733in}{1.837085in}}%
\pgfusepath{stroke}%
\end{pgfscope}%
\begin{pgfscope}%
\pgfpathrectangle{\pgfqpoint{0.550713in}{0.127635in}}{\pgfqpoint{3.194133in}{2.297424in}}%
\pgfusepath{clip}%
\pgfsetrectcap%
\pgfsetroundjoin%
\pgfsetlinewidth{0.752812pt}%
\definecolor{currentstroke}{rgb}{0.000000,0.000000,0.000000}%
\pgfsetstrokecolor{currentstroke}%
\pgfsetdash{}{0pt}%
\pgfpathmoveto{\pgfqpoint{1.589604in}{0.867470in}}%
\pgfpathlineto{\pgfqpoint{1.667861in}{0.867470in}}%
\pgfusepath{stroke}%
\end{pgfscope}%
\begin{pgfscope}%
\pgfpathrectangle{\pgfqpoint{0.550713in}{0.127635in}}{\pgfqpoint{3.194133in}{2.297424in}}%
\pgfusepath{clip}%
\pgfsetrectcap%
\pgfsetroundjoin%
\pgfsetlinewidth{0.752812pt}%
\definecolor{currentstroke}{rgb}{0.000000,0.000000,0.000000}%
\pgfsetstrokecolor{currentstroke}%
\pgfsetdash{}{0pt}%
\pgfpathmoveto{\pgfqpoint{1.589604in}{1.837085in}}%
\pgfpathlineto{\pgfqpoint{1.667861in}{1.837085in}}%
\pgfusepath{stroke}%
\end{pgfscope}%
\begin{pgfscope}%
\pgfpathrectangle{\pgfqpoint{0.550713in}{0.127635in}}{\pgfqpoint{3.194133in}{2.297424in}}%
\pgfusepath{clip}%
\pgfsetrectcap%
\pgfsetroundjoin%
\pgfsetlinewidth{0.752812pt}%
\definecolor{currentstroke}{rgb}{0.000000,0.000000,0.000000}%
\pgfsetstrokecolor{currentstroke}%
\pgfsetdash{}{0pt}%
\pgfpathmoveto{\pgfqpoint{1.868293in}{1.244754in}}%
\pgfpathlineto{\pgfqpoint{1.868293in}{0.925983in}}%
\pgfusepath{stroke}%
\end{pgfscope}%
\begin{pgfscope}%
\pgfpathrectangle{\pgfqpoint{0.550713in}{0.127635in}}{\pgfqpoint{3.194133in}{2.297424in}}%
\pgfusepath{clip}%
\pgfsetrectcap%
\pgfsetroundjoin%
\pgfsetlinewidth{0.752812pt}%
\definecolor{currentstroke}{rgb}{0.000000,0.000000,0.000000}%
\pgfsetstrokecolor{currentstroke}%
\pgfsetdash{}{0pt}%
\pgfpathmoveto{\pgfqpoint{1.868293in}{1.591209in}}%
\pgfpathlineto{\pgfqpoint{1.868293in}{1.815017in}}%
\pgfusepath{stroke}%
\end{pgfscope}%
\begin{pgfscope}%
\pgfpathrectangle{\pgfqpoint{0.550713in}{0.127635in}}{\pgfqpoint{3.194133in}{2.297424in}}%
\pgfusepath{clip}%
\pgfsetrectcap%
\pgfsetroundjoin%
\pgfsetlinewidth{0.752812pt}%
\definecolor{currentstroke}{rgb}{0.000000,0.000000,0.000000}%
\pgfsetstrokecolor{currentstroke}%
\pgfsetdash{}{0pt}%
\pgfpathmoveto{\pgfqpoint{1.829164in}{0.925983in}}%
\pgfpathlineto{\pgfqpoint{1.907421in}{0.925983in}}%
\pgfusepath{stroke}%
\end{pgfscope}%
\begin{pgfscope}%
\pgfpathrectangle{\pgfqpoint{0.550713in}{0.127635in}}{\pgfqpoint{3.194133in}{2.297424in}}%
\pgfusepath{clip}%
\pgfsetrectcap%
\pgfsetroundjoin%
\pgfsetlinewidth{0.752812pt}%
\definecolor{currentstroke}{rgb}{0.000000,0.000000,0.000000}%
\pgfsetstrokecolor{currentstroke}%
\pgfsetdash{}{0pt}%
\pgfpathmoveto{\pgfqpoint{1.829164in}{1.815017in}}%
\pgfpathlineto{\pgfqpoint{1.907421in}{1.815017in}}%
\pgfusepath{stroke}%
\end{pgfscope}%
\begin{pgfscope}%
\pgfpathrectangle{\pgfqpoint{0.550713in}{0.127635in}}{\pgfqpoint{3.194133in}{2.297424in}}%
\pgfusepath{clip}%
\pgfsetrectcap%
\pgfsetroundjoin%
\pgfsetlinewidth{0.752812pt}%
\definecolor{currentstroke}{rgb}{0.000000,0.000000,0.000000}%
\pgfsetstrokecolor{currentstroke}%
\pgfsetdash{}{0pt}%
\pgfpathmoveto{\pgfqpoint{2.027999in}{1.343208in}}%
\pgfpathlineto{\pgfqpoint{2.027999in}{0.961195in}}%
\pgfusepath{stroke}%
\end{pgfscope}%
\begin{pgfscope}%
\pgfpathrectangle{\pgfqpoint{0.550713in}{0.127635in}}{\pgfqpoint{3.194133in}{2.297424in}}%
\pgfusepath{clip}%
\pgfsetrectcap%
\pgfsetroundjoin%
\pgfsetlinewidth{0.752812pt}%
\definecolor{currentstroke}{rgb}{0.000000,0.000000,0.000000}%
\pgfsetstrokecolor{currentstroke}%
\pgfsetdash{}{0pt}%
\pgfpathmoveto{\pgfqpoint{2.027999in}{1.617355in}}%
\pgfpathlineto{\pgfqpoint{2.027999in}{1.771994in}}%
\pgfusepath{stroke}%
\end{pgfscope}%
\begin{pgfscope}%
\pgfpathrectangle{\pgfqpoint{0.550713in}{0.127635in}}{\pgfqpoint{3.194133in}{2.297424in}}%
\pgfusepath{clip}%
\pgfsetrectcap%
\pgfsetroundjoin%
\pgfsetlinewidth{0.752812pt}%
\definecolor{currentstroke}{rgb}{0.000000,0.000000,0.000000}%
\pgfsetstrokecolor{currentstroke}%
\pgfsetdash{}{0pt}%
\pgfpathmoveto{\pgfqpoint{1.988871in}{0.961195in}}%
\pgfpathlineto{\pgfqpoint{2.067127in}{0.961195in}}%
\pgfusepath{stroke}%
\end{pgfscope}%
\begin{pgfscope}%
\pgfpathrectangle{\pgfqpoint{0.550713in}{0.127635in}}{\pgfqpoint{3.194133in}{2.297424in}}%
\pgfusepath{clip}%
\pgfsetrectcap%
\pgfsetroundjoin%
\pgfsetlinewidth{0.752812pt}%
\definecolor{currentstroke}{rgb}{0.000000,0.000000,0.000000}%
\pgfsetstrokecolor{currentstroke}%
\pgfsetdash{}{0pt}%
\pgfpathmoveto{\pgfqpoint{1.988871in}{1.771994in}}%
\pgfpathlineto{\pgfqpoint{2.067127in}{1.771994in}}%
\pgfusepath{stroke}%
\end{pgfscope}%
\begin{pgfscope}%
\pgfpathrectangle{\pgfqpoint{0.550713in}{0.127635in}}{\pgfqpoint{3.194133in}{2.297424in}}%
\pgfusepath{clip}%
\pgfsetbuttcap%
\pgfsetmiterjoin%
\definecolor{currentfill}{rgb}{0.000000,0.000000,0.000000}%
\pgfsetfillcolor{currentfill}%
\pgfsetlinewidth{1.003750pt}%
\definecolor{currentstroke}{rgb}{0.000000,0.000000,0.000000}%
\pgfsetstrokecolor{currentstroke}%
\pgfsetdash{}{0pt}%
\pgfsys@defobject{currentmarker}{\pgfqpoint{-0.011785in}{-0.019642in}}{\pgfqpoint{0.011785in}{0.019642in}}{%
\pgfpathmoveto{\pgfqpoint{-0.000000in}{-0.019642in}}%
\pgfpathlineto{\pgfqpoint{0.011785in}{0.000000in}}%
\pgfpathlineto{\pgfqpoint{0.000000in}{0.019642in}}%
\pgfpathlineto{\pgfqpoint{-0.011785in}{0.000000in}}%
\pgfpathclose%
\pgfusepath{stroke,fill}%
}%
\begin{pgfscope}%
\pgfsys@transformshift{2.027999in}{0.886786in}%
\pgfsys@useobject{currentmarker}{}%
\end{pgfscope}%
\begin{pgfscope}%
\pgfsys@transformshift{2.027999in}{2.066189in}%
\pgfsys@useobject{currentmarker}{}%
\end{pgfscope}%
\end{pgfscope}%
\begin{pgfscope}%
\pgfpathrectangle{\pgfqpoint{0.550713in}{0.127635in}}{\pgfqpoint{3.194133in}{2.297424in}}%
\pgfusepath{clip}%
\pgfsetrectcap%
\pgfsetroundjoin%
\pgfsetlinewidth{0.752812pt}%
\definecolor{currentstroke}{rgb}{0.000000,0.000000,0.000000}%
\pgfsetstrokecolor{currentstroke}%
\pgfsetdash{}{0pt}%
\pgfpathmoveto{\pgfqpoint{2.267559in}{1.099069in}}%
\pgfpathlineto{\pgfqpoint{2.267559in}{0.838846in}}%
\pgfusepath{stroke}%
\end{pgfscope}%
\begin{pgfscope}%
\pgfpathrectangle{\pgfqpoint{0.550713in}{0.127635in}}{\pgfqpoint{3.194133in}{2.297424in}}%
\pgfusepath{clip}%
\pgfsetrectcap%
\pgfsetroundjoin%
\pgfsetlinewidth{0.752812pt}%
\definecolor{currentstroke}{rgb}{0.000000,0.000000,0.000000}%
\pgfsetstrokecolor{currentstroke}%
\pgfsetdash{}{0pt}%
\pgfpathmoveto{\pgfqpoint{2.267559in}{1.288640in}}%
\pgfpathlineto{\pgfqpoint{2.267559in}{1.413797in}}%
\pgfusepath{stroke}%
\end{pgfscope}%
\begin{pgfscope}%
\pgfpathrectangle{\pgfqpoint{0.550713in}{0.127635in}}{\pgfqpoint{3.194133in}{2.297424in}}%
\pgfusepath{clip}%
\pgfsetrectcap%
\pgfsetroundjoin%
\pgfsetlinewidth{0.752812pt}%
\definecolor{currentstroke}{rgb}{0.000000,0.000000,0.000000}%
\pgfsetstrokecolor{currentstroke}%
\pgfsetdash{}{0pt}%
\pgfpathmoveto{\pgfqpoint{2.228431in}{0.838846in}}%
\pgfpathlineto{\pgfqpoint{2.306687in}{0.838846in}}%
\pgfusepath{stroke}%
\end{pgfscope}%
\begin{pgfscope}%
\pgfpathrectangle{\pgfqpoint{0.550713in}{0.127635in}}{\pgfqpoint{3.194133in}{2.297424in}}%
\pgfusepath{clip}%
\pgfsetrectcap%
\pgfsetroundjoin%
\pgfsetlinewidth{0.752812pt}%
\definecolor{currentstroke}{rgb}{0.000000,0.000000,0.000000}%
\pgfsetstrokecolor{currentstroke}%
\pgfsetdash{}{0pt}%
\pgfpathmoveto{\pgfqpoint{2.228431in}{1.413797in}}%
\pgfpathlineto{\pgfqpoint{2.306687in}{1.413797in}}%
\pgfusepath{stroke}%
\end{pgfscope}%
\begin{pgfscope}%
\pgfpathrectangle{\pgfqpoint{0.550713in}{0.127635in}}{\pgfqpoint{3.194133in}{2.297424in}}%
\pgfusepath{clip}%
\pgfsetrectcap%
\pgfsetroundjoin%
\pgfsetlinewidth{0.752812pt}%
\definecolor{currentstroke}{rgb}{0.000000,0.000000,0.000000}%
\pgfsetstrokecolor{currentstroke}%
\pgfsetdash{}{0pt}%
\pgfpathmoveto{\pgfqpoint{2.427266in}{1.238199in}}%
\pgfpathlineto{\pgfqpoint{2.427266in}{1.096342in}}%
\pgfusepath{stroke}%
\end{pgfscope}%
\begin{pgfscope}%
\pgfpathrectangle{\pgfqpoint{0.550713in}{0.127635in}}{\pgfqpoint{3.194133in}{2.297424in}}%
\pgfusepath{clip}%
\pgfsetrectcap%
\pgfsetroundjoin%
\pgfsetlinewidth{0.752812pt}%
\definecolor{currentstroke}{rgb}{0.000000,0.000000,0.000000}%
\pgfsetstrokecolor{currentstroke}%
\pgfsetdash{}{0pt}%
\pgfpathmoveto{\pgfqpoint{2.427266in}{1.471253in}}%
\pgfpathlineto{\pgfqpoint{2.427266in}{1.537454in}}%
\pgfusepath{stroke}%
\end{pgfscope}%
\begin{pgfscope}%
\pgfpathrectangle{\pgfqpoint{0.550713in}{0.127635in}}{\pgfqpoint{3.194133in}{2.297424in}}%
\pgfusepath{clip}%
\pgfsetrectcap%
\pgfsetroundjoin%
\pgfsetlinewidth{0.752812pt}%
\definecolor{currentstroke}{rgb}{0.000000,0.000000,0.000000}%
\pgfsetstrokecolor{currentstroke}%
\pgfsetdash{}{0pt}%
\pgfpathmoveto{\pgfqpoint{2.388138in}{1.096342in}}%
\pgfpathlineto{\pgfqpoint{2.466394in}{1.096342in}}%
\pgfusepath{stroke}%
\end{pgfscope}%
\begin{pgfscope}%
\pgfpathrectangle{\pgfqpoint{0.550713in}{0.127635in}}{\pgfqpoint{3.194133in}{2.297424in}}%
\pgfusepath{clip}%
\pgfsetrectcap%
\pgfsetroundjoin%
\pgfsetlinewidth{0.752812pt}%
\definecolor{currentstroke}{rgb}{0.000000,0.000000,0.000000}%
\pgfsetstrokecolor{currentstroke}%
\pgfsetdash{}{0pt}%
\pgfpathmoveto{\pgfqpoint{2.388138in}{1.537454in}}%
\pgfpathlineto{\pgfqpoint{2.466394in}{1.537454in}}%
\pgfusepath{stroke}%
\end{pgfscope}%
\begin{pgfscope}%
\pgfpathrectangle{\pgfqpoint{0.550713in}{0.127635in}}{\pgfqpoint{3.194133in}{2.297424in}}%
\pgfusepath{clip}%
\pgfsetrectcap%
\pgfsetroundjoin%
\pgfsetlinewidth{0.752812pt}%
\definecolor{currentstroke}{rgb}{0.000000,0.000000,0.000000}%
\pgfsetstrokecolor{currentstroke}%
\pgfsetdash{}{0pt}%
\pgfpathmoveto{\pgfqpoint{2.666826in}{1.149610in}}%
\pgfpathlineto{\pgfqpoint{2.666826in}{1.109853in}}%
\pgfusepath{stroke}%
\end{pgfscope}%
\begin{pgfscope}%
\pgfpathrectangle{\pgfqpoint{0.550713in}{0.127635in}}{\pgfqpoint{3.194133in}{2.297424in}}%
\pgfusepath{clip}%
\pgfsetrectcap%
\pgfsetroundjoin%
\pgfsetlinewidth{0.752812pt}%
\definecolor{currentstroke}{rgb}{0.000000,0.000000,0.000000}%
\pgfsetstrokecolor{currentstroke}%
\pgfsetdash{}{0pt}%
\pgfpathmoveto{\pgfqpoint{2.666826in}{1.248540in}}%
\pgfpathlineto{\pgfqpoint{2.666826in}{1.370866in}}%
\pgfusepath{stroke}%
\end{pgfscope}%
\begin{pgfscope}%
\pgfpathrectangle{\pgfqpoint{0.550713in}{0.127635in}}{\pgfqpoint{3.194133in}{2.297424in}}%
\pgfusepath{clip}%
\pgfsetrectcap%
\pgfsetroundjoin%
\pgfsetlinewidth{0.752812pt}%
\definecolor{currentstroke}{rgb}{0.000000,0.000000,0.000000}%
\pgfsetstrokecolor{currentstroke}%
\pgfsetdash{}{0pt}%
\pgfpathmoveto{\pgfqpoint{2.627698in}{1.109853in}}%
\pgfpathlineto{\pgfqpoint{2.705954in}{1.109853in}}%
\pgfusepath{stroke}%
\end{pgfscope}%
\begin{pgfscope}%
\pgfpathrectangle{\pgfqpoint{0.550713in}{0.127635in}}{\pgfqpoint{3.194133in}{2.297424in}}%
\pgfusepath{clip}%
\pgfsetrectcap%
\pgfsetroundjoin%
\pgfsetlinewidth{0.752812pt}%
\definecolor{currentstroke}{rgb}{0.000000,0.000000,0.000000}%
\pgfsetstrokecolor{currentstroke}%
\pgfsetdash{}{0pt}%
\pgfpathmoveto{\pgfqpoint{2.627698in}{1.370866in}}%
\pgfpathlineto{\pgfqpoint{2.705954in}{1.370866in}}%
\pgfusepath{stroke}%
\end{pgfscope}%
\begin{pgfscope}%
\pgfpathrectangle{\pgfqpoint{0.550713in}{0.127635in}}{\pgfqpoint{3.194133in}{2.297424in}}%
\pgfusepath{clip}%
\pgfsetbuttcap%
\pgfsetmiterjoin%
\definecolor{currentfill}{rgb}{0.000000,0.000000,0.000000}%
\pgfsetfillcolor{currentfill}%
\pgfsetlinewidth{1.003750pt}%
\definecolor{currentstroke}{rgb}{0.000000,0.000000,0.000000}%
\pgfsetstrokecolor{currentstroke}%
\pgfsetdash{}{0pt}%
\pgfsys@defobject{currentmarker}{\pgfqpoint{-0.011785in}{-0.019642in}}{\pgfqpoint{0.011785in}{0.019642in}}{%
\pgfpathmoveto{\pgfqpoint{-0.000000in}{-0.019642in}}%
\pgfpathlineto{\pgfqpoint{0.011785in}{0.000000in}}%
\pgfpathlineto{\pgfqpoint{0.000000in}{0.019642in}}%
\pgfpathlineto{\pgfqpoint{-0.011785in}{0.000000in}}%
\pgfpathclose%
\pgfusepath{stroke,fill}%
}%
\begin{pgfscope}%
\pgfsys@transformshift{2.666826in}{0.949170in}%
\pgfsys@useobject{currentmarker}{}%
\end{pgfscope}%
\begin{pgfscope}%
\pgfsys@transformshift{2.666826in}{0.920499in}%
\pgfsys@useobject{currentmarker}{}%
\end{pgfscope}%
\begin{pgfscope}%
\pgfsys@transformshift{2.666826in}{0.838374in}%
\pgfsys@useobject{currentmarker}{}%
\end{pgfscope}%
\begin{pgfscope}%
\pgfsys@transformshift{2.666826in}{0.858285in}%
\pgfsys@useobject{currentmarker}{}%
\end{pgfscope}%
\begin{pgfscope}%
\pgfsys@transformshift{2.666826in}{0.877788in}%
\pgfsys@useobject{currentmarker}{}%
\end{pgfscope}%
\end{pgfscope}%
\begin{pgfscope}%
\pgfpathrectangle{\pgfqpoint{0.550713in}{0.127635in}}{\pgfqpoint{3.194133in}{2.297424in}}%
\pgfusepath{clip}%
\pgfsetrectcap%
\pgfsetroundjoin%
\pgfsetlinewidth{0.752812pt}%
\definecolor{currentstroke}{rgb}{0.000000,0.000000,0.000000}%
\pgfsetstrokecolor{currentstroke}%
\pgfsetdash{}{0pt}%
\pgfpathmoveto{\pgfqpoint{2.826532in}{1.235864in}}%
\pgfpathlineto{\pgfqpoint{2.826532in}{1.089612in}}%
\pgfusepath{stroke}%
\end{pgfscope}%
\begin{pgfscope}%
\pgfpathrectangle{\pgfqpoint{0.550713in}{0.127635in}}{\pgfqpoint{3.194133in}{2.297424in}}%
\pgfusepath{clip}%
\pgfsetrectcap%
\pgfsetroundjoin%
\pgfsetlinewidth{0.752812pt}%
\definecolor{currentstroke}{rgb}{0.000000,0.000000,0.000000}%
\pgfsetstrokecolor{currentstroke}%
\pgfsetdash{}{0pt}%
\pgfpathmoveto{\pgfqpoint{2.826532in}{1.414695in}}%
\pgfpathlineto{\pgfqpoint{2.826532in}{1.544697in}}%
\pgfusepath{stroke}%
\end{pgfscope}%
\begin{pgfscope}%
\pgfpathrectangle{\pgfqpoint{0.550713in}{0.127635in}}{\pgfqpoint{3.194133in}{2.297424in}}%
\pgfusepath{clip}%
\pgfsetrectcap%
\pgfsetroundjoin%
\pgfsetlinewidth{0.752812pt}%
\definecolor{currentstroke}{rgb}{0.000000,0.000000,0.000000}%
\pgfsetstrokecolor{currentstroke}%
\pgfsetdash{}{0pt}%
\pgfpathmoveto{\pgfqpoint{2.787404in}{1.089612in}}%
\pgfpathlineto{\pgfqpoint{2.865661in}{1.089612in}}%
\pgfusepath{stroke}%
\end{pgfscope}%
\begin{pgfscope}%
\pgfpathrectangle{\pgfqpoint{0.550713in}{0.127635in}}{\pgfqpoint{3.194133in}{2.297424in}}%
\pgfusepath{clip}%
\pgfsetrectcap%
\pgfsetroundjoin%
\pgfsetlinewidth{0.752812pt}%
\definecolor{currentstroke}{rgb}{0.000000,0.000000,0.000000}%
\pgfsetstrokecolor{currentstroke}%
\pgfsetdash{}{0pt}%
\pgfpathmoveto{\pgfqpoint{2.787404in}{1.544697in}}%
\pgfpathlineto{\pgfqpoint{2.865661in}{1.544697in}}%
\pgfusepath{stroke}%
\end{pgfscope}%
\begin{pgfscope}%
\pgfpathrectangle{\pgfqpoint{0.550713in}{0.127635in}}{\pgfqpoint{3.194133in}{2.297424in}}%
\pgfusepath{clip}%
\pgfsetrectcap%
\pgfsetroundjoin%
\pgfsetlinewidth{0.752812pt}%
\definecolor{currentstroke}{rgb}{0.000000,0.000000,0.000000}%
\pgfsetstrokecolor{currentstroke}%
\pgfsetdash{}{0pt}%
\pgfpathmoveto{\pgfqpoint{3.066092in}{1.129976in}}%
\pgfpathlineto{\pgfqpoint{3.066092in}{0.892772in}}%
\pgfusepath{stroke}%
\end{pgfscope}%
\begin{pgfscope}%
\pgfpathrectangle{\pgfqpoint{0.550713in}{0.127635in}}{\pgfqpoint{3.194133in}{2.297424in}}%
\pgfusepath{clip}%
\pgfsetrectcap%
\pgfsetroundjoin%
\pgfsetlinewidth{0.752812pt}%
\definecolor{currentstroke}{rgb}{0.000000,0.000000,0.000000}%
\pgfsetstrokecolor{currentstroke}%
\pgfsetdash{}{0pt}%
\pgfpathmoveto{\pgfqpoint{3.066092in}{1.296988in}}%
\pgfpathlineto{\pgfqpoint{3.066092in}{1.427275in}}%
\pgfusepath{stroke}%
\end{pgfscope}%
\begin{pgfscope}%
\pgfpathrectangle{\pgfqpoint{0.550713in}{0.127635in}}{\pgfqpoint{3.194133in}{2.297424in}}%
\pgfusepath{clip}%
\pgfsetrectcap%
\pgfsetroundjoin%
\pgfsetlinewidth{0.752812pt}%
\definecolor{currentstroke}{rgb}{0.000000,0.000000,0.000000}%
\pgfsetstrokecolor{currentstroke}%
\pgfsetdash{}{0pt}%
\pgfpathmoveto{\pgfqpoint{3.026964in}{0.892772in}}%
\pgfpathlineto{\pgfqpoint{3.105221in}{0.892772in}}%
\pgfusepath{stroke}%
\end{pgfscope}%
\begin{pgfscope}%
\pgfpathrectangle{\pgfqpoint{0.550713in}{0.127635in}}{\pgfqpoint{3.194133in}{2.297424in}}%
\pgfusepath{clip}%
\pgfsetrectcap%
\pgfsetroundjoin%
\pgfsetlinewidth{0.752812pt}%
\definecolor{currentstroke}{rgb}{0.000000,0.000000,0.000000}%
\pgfsetstrokecolor{currentstroke}%
\pgfsetdash{}{0pt}%
\pgfpathmoveto{\pgfqpoint{3.026964in}{1.427275in}}%
\pgfpathlineto{\pgfqpoint{3.105221in}{1.427275in}}%
\pgfusepath{stroke}%
\end{pgfscope}%
\begin{pgfscope}%
\pgfpathrectangle{\pgfqpoint{0.550713in}{0.127635in}}{\pgfqpoint{3.194133in}{2.297424in}}%
\pgfusepath{clip}%
\pgfsetbuttcap%
\pgfsetmiterjoin%
\definecolor{currentfill}{rgb}{0.000000,0.000000,0.000000}%
\pgfsetfillcolor{currentfill}%
\pgfsetlinewidth{1.003750pt}%
\definecolor{currentstroke}{rgb}{0.000000,0.000000,0.000000}%
\pgfsetstrokecolor{currentstroke}%
\pgfsetdash{}{0pt}%
\pgfsys@defobject{currentmarker}{\pgfqpoint{-0.011785in}{-0.019642in}}{\pgfqpoint{0.011785in}{0.019642in}}{%
\pgfpathmoveto{\pgfqpoint{-0.000000in}{-0.019642in}}%
\pgfpathlineto{\pgfqpoint{0.011785in}{0.000000in}}%
\pgfpathlineto{\pgfqpoint{0.000000in}{0.019642in}}%
\pgfpathlineto{\pgfqpoint{-0.011785in}{0.000000in}}%
\pgfpathclose%
\pgfusepath{stroke,fill}%
}%
\begin{pgfscope}%
\pgfsys@transformshift{3.066092in}{0.808885in}%
\pgfsys@useobject{currentmarker}{}%
\end{pgfscope}%
\begin{pgfscope}%
\pgfsys@transformshift{3.066092in}{0.878123in}%
\pgfsys@useobject{currentmarker}{}%
\end{pgfscope}%
\begin{pgfscope}%
\pgfsys@transformshift{3.066092in}{0.877455in}%
\pgfsys@useobject{currentmarker}{}%
\end{pgfscope}%
\end{pgfscope}%
\begin{pgfscope}%
\pgfpathrectangle{\pgfqpoint{0.550713in}{0.127635in}}{\pgfqpoint{3.194133in}{2.297424in}}%
\pgfusepath{clip}%
\pgfsetrectcap%
\pgfsetroundjoin%
\pgfsetlinewidth{0.752812pt}%
\definecolor{currentstroke}{rgb}{0.000000,0.000000,0.000000}%
\pgfsetstrokecolor{currentstroke}%
\pgfsetdash{}{0pt}%
\pgfpathmoveto{\pgfqpoint{3.225799in}{1.173293in}}%
\pgfpathlineto{\pgfqpoint{3.225799in}{1.163412in}}%
\pgfusepath{stroke}%
\end{pgfscope}%
\begin{pgfscope}%
\pgfpathrectangle{\pgfqpoint{0.550713in}{0.127635in}}{\pgfqpoint{3.194133in}{2.297424in}}%
\pgfusepath{clip}%
\pgfsetrectcap%
\pgfsetroundjoin%
\pgfsetlinewidth{0.752812pt}%
\definecolor{currentstroke}{rgb}{0.000000,0.000000,0.000000}%
\pgfsetstrokecolor{currentstroke}%
\pgfsetdash{}{0pt}%
\pgfpathmoveto{\pgfqpoint{3.225799in}{1.289290in}}%
\pgfpathlineto{\pgfqpoint{3.225799in}{1.448024in}}%
\pgfusepath{stroke}%
\end{pgfscope}%
\begin{pgfscope}%
\pgfpathrectangle{\pgfqpoint{0.550713in}{0.127635in}}{\pgfqpoint{3.194133in}{2.297424in}}%
\pgfusepath{clip}%
\pgfsetrectcap%
\pgfsetroundjoin%
\pgfsetlinewidth{0.752812pt}%
\definecolor{currentstroke}{rgb}{0.000000,0.000000,0.000000}%
\pgfsetstrokecolor{currentstroke}%
\pgfsetdash{}{0pt}%
\pgfpathmoveto{\pgfqpoint{3.186671in}{1.163412in}}%
\pgfpathlineto{\pgfqpoint{3.264927in}{1.163412in}}%
\pgfusepath{stroke}%
\end{pgfscope}%
\begin{pgfscope}%
\pgfpathrectangle{\pgfqpoint{0.550713in}{0.127635in}}{\pgfqpoint{3.194133in}{2.297424in}}%
\pgfusepath{clip}%
\pgfsetrectcap%
\pgfsetroundjoin%
\pgfsetlinewidth{0.752812pt}%
\definecolor{currentstroke}{rgb}{0.000000,0.000000,0.000000}%
\pgfsetstrokecolor{currentstroke}%
\pgfsetdash{}{0pt}%
\pgfpathmoveto{\pgfqpoint{3.186671in}{1.448024in}}%
\pgfpathlineto{\pgfqpoint{3.264927in}{1.448024in}}%
\pgfusepath{stroke}%
\end{pgfscope}%
\begin{pgfscope}%
\pgfpathrectangle{\pgfqpoint{0.550713in}{0.127635in}}{\pgfqpoint{3.194133in}{2.297424in}}%
\pgfusepath{clip}%
\pgfsetbuttcap%
\pgfsetmiterjoin%
\definecolor{currentfill}{rgb}{0.000000,0.000000,0.000000}%
\pgfsetfillcolor{currentfill}%
\pgfsetlinewidth{1.003750pt}%
\definecolor{currentstroke}{rgb}{0.000000,0.000000,0.000000}%
\pgfsetstrokecolor{currentstroke}%
\pgfsetdash{}{0pt}%
\pgfsys@defobject{currentmarker}{\pgfqpoint{-0.011785in}{-0.019642in}}{\pgfqpoint{0.011785in}{0.019642in}}{%
\pgfpathmoveto{\pgfqpoint{-0.000000in}{-0.019642in}}%
\pgfpathlineto{\pgfqpoint{0.011785in}{0.000000in}}%
\pgfpathlineto{\pgfqpoint{0.000000in}{0.019642in}}%
\pgfpathlineto{\pgfqpoint{-0.011785in}{0.000000in}}%
\pgfpathclose%
\pgfusepath{stroke,fill}%
}%
\begin{pgfscope}%
\pgfsys@transformshift{3.225799in}{0.819660in}%
\pgfsys@useobject{currentmarker}{}%
\end{pgfscope}%
\begin{pgfscope}%
\pgfsys@transformshift{3.225799in}{0.860075in}%
\pgfsys@useobject{currentmarker}{}%
\end{pgfscope}%
\begin{pgfscope}%
\pgfsys@transformshift{3.225799in}{0.930805in}%
\pgfsys@useobject{currentmarker}{}%
\end{pgfscope}%
\begin{pgfscope}%
\pgfsys@transformshift{3.225799in}{0.879658in}%
\pgfsys@useobject{currentmarker}{}%
\end{pgfscope}%
\begin{pgfscope}%
\pgfsys@transformshift{3.225799in}{0.885673in}%
\pgfsys@useobject{currentmarker}{}%
\end{pgfscope}%
\end{pgfscope}%
\begin{pgfscope}%
\pgfpathrectangle{\pgfqpoint{0.550713in}{0.127635in}}{\pgfqpoint{3.194133in}{2.297424in}}%
\pgfusepath{clip}%
\pgfsetrectcap%
\pgfsetroundjoin%
\pgfsetlinewidth{0.752812pt}%
\definecolor{currentstroke}{rgb}{0.000000,0.000000,0.000000}%
\pgfsetstrokecolor{currentstroke}%
\pgfsetdash{}{0pt}%
\pgfpathmoveto{\pgfqpoint{3.465359in}{1.125794in}}%
\pgfpathlineto{\pgfqpoint{3.465359in}{0.921982in}}%
\pgfusepath{stroke}%
\end{pgfscope}%
\begin{pgfscope}%
\pgfpathrectangle{\pgfqpoint{0.550713in}{0.127635in}}{\pgfqpoint{3.194133in}{2.297424in}}%
\pgfusepath{clip}%
\pgfsetrectcap%
\pgfsetroundjoin%
\pgfsetlinewidth{0.752812pt}%
\definecolor{currentstroke}{rgb}{0.000000,0.000000,0.000000}%
\pgfsetstrokecolor{currentstroke}%
\pgfsetdash{}{0pt}%
\pgfpathmoveto{\pgfqpoint{3.465359in}{1.282275in}}%
\pgfpathlineto{\pgfqpoint{3.465359in}{1.473101in}}%
\pgfusepath{stroke}%
\end{pgfscope}%
\begin{pgfscope}%
\pgfpathrectangle{\pgfqpoint{0.550713in}{0.127635in}}{\pgfqpoint{3.194133in}{2.297424in}}%
\pgfusepath{clip}%
\pgfsetrectcap%
\pgfsetroundjoin%
\pgfsetlinewidth{0.752812pt}%
\definecolor{currentstroke}{rgb}{0.000000,0.000000,0.000000}%
\pgfsetstrokecolor{currentstroke}%
\pgfsetdash{}{0pt}%
\pgfpathmoveto{\pgfqpoint{3.426231in}{0.921982in}}%
\pgfpathlineto{\pgfqpoint{3.504487in}{0.921982in}}%
\pgfusepath{stroke}%
\end{pgfscope}%
\begin{pgfscope}%
\pgfpathrectangle{\pgfqpoint{0.550713in}{0.127635in}}{\pgfqpoint{3.194133in}{2.297424in}}%
\pgfusepath{clip}%
\pgfsetrectcap%
\pgfsetroundjoin%
\pgfsetlinewidth{0.752812pt}%
\definecolor{currentstroke}{rgb}{0.000000,0.000000,0.000000}%
\pgfsetstrokecolor{currentstroke}%
\pgfsetdash{}{0pt}%
\pgfpathmoveto{\pgfqpoint{3.426231in}{1.473101in}}%
\pgfpathlineto{\pgfqpoint{3.504487in}{1.473101in}}%
\pgfusepath{stroke}%
\end{pgfscope}%
\begin{pgfscope}%
\pgfpathrectangle{\pgfqpoint{0.550713in}{0.127635in}}{\pgfqpoint{3.194133in}{2.297424in}}%
\pgfusepath{clip}%
\pgfsetrectcap%
\pgfsetroundjoin%
\pgfsetlinewidth{0.752812pt}%
\definecolor{currentstroke}{rgb}{0.000000,0.000000,0.000000}%
\pgfsetstrokecolor{currentstroke}%
\pgfsetdash{}{0pt}%
\pgfpathmoveto{\pgfqpoint{3.625066in}{1.199561in}}%
\pgfpathlineto{\pgfqpoint{3.625066in}{1.018878in}}%
\pgfusepath{stroke}%
\end{pgfscope}%
\begin{pgfscope}%
\pgfpathrectangle{\pgfqpoint{0.550713in}{0.127635in}}{\pgfqpoint{3.194133in}{2.297424in}}%
\pgfusepath{clip}%
\pgfsetrectcap%
\pgfsetroundjoin%
\pgfsetlinewidth{0.752812pt}%
\definecolor{currentstroke}{rgb}{0.000000,0.000000,0.000000}%
\pgfsetstrokecolor{currentstroke}%
\pgfsetdash{}{0pt}%
\pgfpathmoveto{\pgfqpoint{3.625066in}{1.329932in}}%
\pgfpathlineto{\pgfqpoint{3.625066in}{1.443408in}}%
\pgfusepath{stroke}%
\end{pgfscope}%
\begin{pgfscope}%
\pgfpathrectangle{\pgfqpoint{0.550713in}{0.127635in}}{\pgfqpoint{3.194133in}{2.297424in}}%
\pgfusepath{clip}%
\pgfsetrectcap%
\pgfsetroundjoin%
\pgfsetlinewidth{0.752812pt}%
\definecolor{currentstroke}{rgb}{0.000000,0.000000,0.000000}%
\pgfsetstrokecolor{currentstroke}%
\pgfsetdash{}{0pt}%
\pgfpathmoveto{\pgfqpoint{3.585938in}{1.018878in}}%
\pgfpathlineto{\pgfqpoint{3.664194in}{1.018878in}}%
\pgfusepath{stroke}%
\end{pgfscope}%
\begin{pgfscope}%
\pgfpathrectangle{\pgfqpoint{0.550713in}{0.127635in}}{\pgfqpoint{3.194133in}{2.297424in}}%
\pgfusepath{clip}%
\pgfsetrectcap%
\pgfsetroundjoin%
\pgfsetlinewidth{0.752812pt}%
\definecolor{currentstroke}{rgb}{0.000000,0.000000,0.000000}%
\pgfsetstrokecolor{currentstroke}%
\pgfsetdash{}{0pt}%
\pgfpathmoveto{\pgfqpoint{3.585938in}{1.443408in}}%
\pgfpathlineto{\pgfqpoint{3.664194in}{1.443408in}}%
\pgfusepath{stroke}%
\end{pgfscope}%
\begin{pgfscope}%
\pgfpathrectangle{\pgfqpoint{0.550713in}{0.127635in}}{\pgfqpoint{3.194133in}{2.297424in}}%
\pgfusepath{clip}%
\pgfsetbuttcap%
\pgfsetmiterjoin%
\definecolor{currentfill}{rgb}{0.000000,0.000000,0.000000}%
\pgfsetfillcolor{currentfill}%
\pgfsetlinewidth{1.003750pt}%
\definecolor{currentstroke}{rgb}{0.000000,0.000000,0.000000}%
\pgfsetstrokecolor{currentstroke}%
\pgfsetdash{}{0pt}%
\pgfsys@defobject{currentmarker}{\pgfqpoint{-0.011785in}{-0.019642in}}{\pgfqpoint{0.011785in}{0.019642in}}{%
\pgfpathmoveto{\pgfqpoint{-0.000000in}{-0.019642in}}%
\pgfpathlineto{\pgfqpoint{0.011785in}{0.000000in}}%
\pgfpathlineto{\pgfqpoint{0.000000in}{0.019642in}}%
\pgfpathlineto{\pgfqpoint{-0.011785in}{0.000000in}}%
\pgfpathclose%
\pgfusepath{stroke,fill}%
}%
\begin{pgfscope}%
\pgfsys@transformshift{3.625066in}{1.553498in}%
\pgfsys@useobject{currentmarker}{}%
\end{pgfscope}%
\end{pgfscope}%
\begin{pgfscope}%
\pgfpathrectangle{\pgfqpoint{0.550713in}{0.127635in}}{\pgfqpoint{3.194133in}{2.297424in}}%
\pgfusepath{clip}%
\pgfsetrectcap%
\pgfsetroundjoin%
\pgfsetlinewidth{0.752812pt}%
\definecolor{currentstroke}{rgb}{0.000000,0.000000,0.000000}%
\pgfsetstrokecolor{currentstroke}%
\pgfsetdash{}{0pt}%
\pgfpathmoveto{\pgfqpoint{0.592236in}{1.422684in}}%
\pgfpathlineto{\pgfqpoint{0.748749in}{1.422684in}}%
\pgfusepath{stroke}%
\end{pgfscope}%
\begin{pgfscope}%
\pgfpathrectangle{\pgfqpoint{0.550713in}{0.127635in}}{\pgfqpoint{3.194133in}{2.297424in}}%
\pgfusepath{clip}%
\pgfsetbuttcap%
\pgfsetroundjoin%
\definecolor{currentfill}{rgb}{1.000000,1.000000,1.000000}%
\pgfsetfillcolor{currentfill}%
\pgfsetlinewidth{1.003750pt}%
\definecolor{currentstroke}{rgb}{0.000000,0.000000,0.000000}%
\pgfsetstrokecolor{currentstroke}%
\pgfsetdash{}{0pt}%
\pgfsys@defobject{currentmarker}{\pgfqpoint{-0.027778in}{-0.027778in}}{\pgfqpoint{0.027778in}{0.027778in}}{%
\pgfpathmoveto{\pgfqpoint{0.000000in}{-0.027778in}}%
\pgfpathcurveto{\pgfqpoint{0.007367in}{-0.027778in}}{\pgfqpoint{0.014433in}{-0.024851in}}{\pgfqpoint{0.019642in}{-0.019642in}}%
\pgfpathcurveto{\pgfqpoint{0.024851in}{-0.014433in}}{\pgfqpoint{0.027778in}{-0.007367in}}{\pgfqpoint{0.027778in}{0.000000in}}%
\pgfpathcurveto{\pgfqpoint{0.027778in}{0.007367in}}{\pgfqpoint{0.024851in}{0.014433in}}{\pgfqpoint{0.019642in}{0.019642in}}%
\pgfpathcurveto{\pgfqpoint{0.014433in}{0.024851in}}{\pgfqpoint{0.007367in}{0.027778in}}{\pgfqpoint{0.000000in}{0.027778in}}%
\pgfpathcurveto{\pgfqpoint{-0.007367in}{0.027778in}}{\pgfqpoint{-0.014433in}{0.024851in}}{\pgfqpoint{-0.019642in}{0.019642in}}%
\pgfpathcurveto{\pgfqpoint{-0.024851in}{0.014433in}}{\pgfqpoint{-0.027778in}{0.007367in}}{\pgfqpoint{-0.027778in}{0.000000in}}%
\pgfpathcurveto{\pgfqpoint{-0.027778in}{-0.007367in}}{\pgfqpoint{-0.024851in}{-0.014433in}}{\pgfqpoint{-0.019642in}{-0.019642in}}%
\pgfpathcurveto{\pgfqpoint{-0.014433in}{-0.024851in}}{\pgfqpoint{-0.007367in}{-0.027778in}}{\pgfqpoint{0.000000in}{-0.027778in}}%
\pgfpathclose%
\pgfusepath{stroke,fill}%
}%
\begin{pgfscope}%
\pgfsys@transformshift{0.670493in}{1.417950in}%
\pgfsys@useobject{currentmarker}{}%
\end{pgfscope}%
\end{pgfscope}%
\begin{pgfscope}%
\pgfpathrectangle{\pgfqpoint{0.550713in}{0.127635in}}{\pgfqpoint{3.194133in}{2.297424in}}%
\pgfusepath{clip}%
\pgfsetrectcap%
\pgfsetroundjoin%
\pgfsetlinewidth{0.752812pt}%
\definecolor{currentstroke}{rgb}{0.000000,0.000000,0.000000}%
\pgfsetstrokecolor{currentstroke}%
\pgfsetdash{}{0pt}%
\pgfpathmoveto{\pgfqpoint{0.751943in}{1.290627in}}%
\pgfpathlineto{\pgfqpoint{0.908456in}{1.290627in}}%
\pgfusepath{stroke}%
\end{pgfscope}%
\begin{pgfscope}%
\pgfpathrectangle{\pgfqpoint{0.550713in}{0.127635in}}{\pgfqpoint{3.194133in}{2.297424in}}%
\pgfusepath{clip}%
\pgfsetbuttcap%
\pgfsetroundjoin%
\definecolor{currentfill}{rgb}{1.000000,1.000000,1.000000}%
\pgfsetfillcolor{currentfill}%
\pgfsetlinewidth{1.003750pt}%
\definecolor{currentstroke}{rgb}{0.000000,0.000000,0.000000}%
\pgfsetstrokecolor{currentstroke}%
\pgfsetdash{}{0pt}%
\pgfsys@defobject{currentmarker}{\pgfqpoint{-0.027778in}{-0.027778in}}{\pgfqpoint{0.027778in}{0.027778in}}{%
\pgfpathmoveto{\pgfqpoint{0.000000in}{-0.027778in}}%
\pgfpathcurveto{\pgfqpoint{0.007367in}{-0.027778in}}{\pgfqpoint{0.014433in}{-0.024851in}}{\pgfqpoint{0.019642in}{-0.019642in}}%
\pgfpathcurveto{\pgfqpoint{0.024851in}{-0.014433in}}{\pgfqpoint{0.027778in}{-0.007367in}}{\pgfqpoint{0.027778in}{0.000000in}}%
\pgfpathcurveto{\pgfqpoint{0.027778in}{0.007367in}}{\pgfqpoint{0.024851in}{0.014433in}}{\pgfqpoint{0.019642in}{0.019642in}}%
\pgfpathcurveto{\pgfqpoint{0.014433in}{0.024851in}}{\pgfqpoint{0.007367in}{0.027778in}}{\pgfqpoint{0.000000in}{0.027778in}}%
\pgfpathcurveto{\pgfqpoint{-0.007367in}{0.027778in}}{\pgfqpoint{-0.014433in}{0.024851in}}{\pgfqpoint{-0.019642in}{0.019642in}}%
\pgfpathcurveto{\pgfqpoint{-0.024851in}{0.014433in}}{\pgfqpoint{-0.027778in}{0.007367in}}{\pgfqpoint{-0.027778in}{0.000000in}}%
\pgfpathcurveto{\pgfqpoint{-0.027778in}{-0.007367in}}{\pgfqpoint{-0.024851in}{-0.014433in}}{\pgfqpoint{-0.019642in}{-0.019642in}}%
\pgfpathcurveto{\pgfqpoint{-0.014433in}{-0.024851in}}{\pgfqpoint{-0.007367in}{-0.027778in}}{\pgfqpoint{0.000000in}{-0.027778in}}%
\pgfpathclose%
\pgfusepath{stroke,fill}%
}%
\begin{pgfscope}%
\pgfsys@transformshift{0.830199in}{1.267873in}%
\pgfsys@useobject{currentmarker}{}%
\end{pgfscope}%
\end{pgfscope}%
\begin{pgfscope}%
\pgfpathrectangle{\pgfqpoint{0.550713in}{0.127635in}}{\pgfqpoint{3.194133in}{2.297424in}}%
\pgfusepath{clip}%
\pgfsetrectcap%
\pgfsetroundjoin%
\pgfsetlinewidth{0.752812pt}%
\definecolor{currentstroke}{rgb}{0.000000,0.000000,0.000000}%
\pgfsetstrokecolor{currentstroke}%
\pgfsetdash{}{0pt}%
\pgfpathmoveto{\pgfqpoint{0.991503in}{1.358507in}}%
\pgfpathlineto{\pgfqpoint{1.148015in}{1.358507in}}%
\pgfusepath{stroke}%
\end{pgfscope}%
\begin{pgfscope}%
\pgfpathrectangle{\pgfqpoint{0.550713in}{0.127635in}}{\pgfqpoint{3.194133in}{2.297424in}}%
\pgfusepath{clip}%
\pgfsetbuttcap%
\pgfsetroundjoin%
\definecolor{currentfill}{rgb}{1.000000,1.000000,1.000000}%
\pgfsetfillcolor{currentfill}%
\pgfsetlinewidth{1.003750pt}%
\definecolor{currentstroke}{rgb}{0.000000,0.000000,0.000000}%
\pgfsetstrokecolor{currentstroke}%
\pgfsetdash{}{0pt}%
\pgfsys@defobject{currentmarker}{\pgfqpoint{-0.027778in}{-0.027778in}}{\pgfqpoint{0.027778in}{0.027778in}}{%
\pgfpathmoveto{\pgfqpoint{0.000000in}{-0.027778in}}%
\pgfpathcurveto{\pgfqpoint{0.007367in}{-0.027778in}}{\pgfqpoint{0.014433in}{-0.024851in}}{\pgfqpoint{0.019642in}{-0.019642in}}%
\pgfpathcurveto{\pgfqpoint{0.024851in}{-0.014433in}}{\pgfqpoint{0.027778in}{-0.007367in}}{\pgfqpoint{0.027778in}{0.000000in}}%
\pgfpathcurveto{\pgfqpoint{0.027778in}{0.007367in}}{\pgfqpoint{0.024851in}{0.014433in}}{\pgfqpoint{0.019642in}{0.019642in}}%
\pgfpathcurveto{\pgfqpoint{0.014433in}{0.024851in}}{\pgfqpoint{0.007367in}{0.027778in}}{\pgfqpoint{0.000000in}{0.027778in}}%
\pgfpathcurveto{\pgfqpoint{-0.007367in}{0.027778in}}{\pgfqpoint{-0.014433in}{0.024851in}}{\pgfqpoint{-0.019642in}{0.019642in}}%
\pgfpathcurveto{\pgfqpoint{-0.024851in}{0.014433in}}{\pgfqpoint{-0.027778in}{0.007367in}}{\pgfqpoint{-0.027778in}{0.000000in}}%
\pgfpathcurveto{\pgfqpoint{-0.027778in}{-0.007367in}}{\pgfqpoint{-0.024851in}{-0.014433in}}{\pgfqpoint{-0.019642in}{-0.019642in}}%
\pgfpathcurveto{\pgfqpoint{-0.014433in}{-0.024851in}}{\pgfqpoint{-0.007367in}{-0.027778in}}{\pgfqpoint{0.000000in}{-0.027778in}}%
\pgfpathclose%
\pgfusepath{stroke,fill}%
}%
\begin{pgfscope}%
\pgfsys@transformshift{1.069759in}{1.429382in}%
\pgfsys@useobject{currentmarker}{}%
\end{pgfscope}%
\end{pgfscope}%
\begin{pgfscope}%
\pgfpathrectangle{\pgfqpoint{0.550713in}{0.127635in}}{\pgfqpoint{3.194133in}{2.297424in}}%
\pgfusepath{clip}%
\pgfsetrectcap%
\pgfsetroundjoin%
\pgfsetlinewidth{0.752812pt}%
\definecolor{currentstroke}{rgb}{0.000000,0.000000,0.000000}%
\pgfsetstrokecolor{currentstroke}%
\pgfsetdash{}{0pt}%
\pgfpathmoveto{\pgfqpoint{1.151210in}{1.232243in}}%
\pgfpathlineto{\pgfqpoint{1.307722in}{1.232243in}}%
\pgfusepath{stroke}%
\end{pgfscope}%
\begin{pgfscope}%
\pgfpathrectangle{\pgfqpoint{0.550713in}{0.127635in}}{\pgfqpoint{3.194133in}{2.297424in}}%
\pgfusepath{clip}%
\pgfsetbuttcap%
\pgfsetroundjoin%
\definecolor{currentfill}{rgb}{1.000000,1.000000,1.000000}%
\pgfsetfillcolor{currentfill}%
\pgfsetlinewidth{1.003750pt}%
\definecolor{currentstroke}{rgb}{0.000000,0.000000,0.000000}%
\pgfsetstrokecolor{currentstroke}%
\pgfsetdash{}{0pt}%
\pgfsys@defobject{currentmarker}{\pgfqpoint{-0.027778in}{-0.027778in}}{\pgfqpoint{0.027778in}{0.027778in}}{%
\pgfpathmoveto{\pgfqpoint{0.000000in}{-0.027778in}}%
\pgfpathcurveto{\pgfqpoint{0.007367in}{-0.027778in}}{\pgfqpoint{0.014433in}{-0.024851in}}{\pgfqpoint{0.019642in}{-0.019642in}}%
\pgfpathcurveto{\pgfqpoint{0.024851in}{-0.014433in}}{\pgfqpoint{0.027778in}{-0.007367in}}{\pgfqpoint{0.027778in}{0.000000in}}%
\pgfpathcurveto{\pgfqpoint{0.027778in}{0.007367in}}{\pgfqpoint{0.024851in}{0.014433in}}{\pgfqpoint{0.019642in}{0.019642in}}%
\pgfpathcurveto{\pgfqpoint{0.014433in}{0.024851in}}{\pgfqpoint{0.007367in}{0.027778in}}{\pgfqpoint{0.000000in}{0.027778in}}%
\pgfpathcurveto{\pgfqpoint{-0.007367in}{0.027778in}}{\pgfqpoint{-0.014433in}{0.024851in}}{\pgfqpoint{-0.019642in}{0.019642in}}%
\pgfpathcurveto{\pgfqpoint{-0.024851in}{0.014433in}}{\pgfqpoint{-0.027778in}{0.007367in}}{\pgfqpoint{-0.027778in}{0.000000in}}%
\pgfpathcurveto{\pgfqpoint{-0.027778in}{-0.007367in}}{\pgfqpoint{-0.024851in}{-0.014433in}}{\pgfqpoint{-0.019642in}{-0.019642in}}%
\pgfpathcurveto{\pgfqpoint{-0.014433in}{-0.024851in}}{\pgfqpoint{-0.007367in}{-0.027778in}}{\pgfqpoint{0.000000in}{-0.027778in}}%
\pgfpathclose%
\pgfusepath{stroke,fill}%
}%
\begin{pgfscope}%
\pgfsys@transformshift{1.229466in}{1.260220in}%
\pgfsys@useobject{currentmarker}{}%
\end{pgfscope}%
\end{pgfscope}%
\begin{pgfscope}%
\pgfpathrectangle{\pgfqpoint{0.550713in}{0.127635in}}{\pgfqpoint{3.194133in}{2.297424in}}%
\pgfusepath{clip}%
\pgfsetrectcap%
\pgfsetroundjoin%
\pgfsetlinewidth{0.752812pt}%
\definecolor{currentstroke}{rgb}{0.000000,0.000000,0.000000}%
\pgfsetstrokecolor{currentstroke}%
\pgfsetdash{}{0pt}%
\pgfpathmoveto{\pgfqpoint{1.390770in}{1.409377in}}%
\pgfpathlineto{\pgfqpoint{1.547282in}{1.409377in}}%
\pgfusepath{stroke}%
\end{pgfscope}%
\begin{pgfscope}%
\pgfpathrectangle{\pgfqpoint{0.550713in}{0.127635in}}{\pgfqpoint{3.194133in}{2.297424in}}%
\pgfusepath{clip}%
\pgfsetbuttcap%
\pgfsetroundjoin%
\definecolor{currentfill}{rgb}{1.000000,1.000000,1.000000}%
\pgfsetfillcolor{currentfill}%
\pgfsetlinewidth{1.003750pt}%
\definecolor{currentstroke}{rgb}{0.000000,0.000000,0.000000}%
\pgfsetstrokecolor{currentstroke}%
\pgfsetdash{}{0pt}%
\pgfsys@defobject{currentmarker}{\pgfqpoint{-0.027778in}{-0.027778in}}{\pgfqpoint{0.027778in}{0.027778in}}{%
\pgfpathmoveto{\pgfqpoint{0.000000in}{-0.027778in}}%
\pgfpathcurveto{\pgfqpoint{0.007367in}{-0.027778in}}{\pgfqpoint{0.014433in}{-0.024851in}}{\pgfqpoint{0.019642in}{-0.019642in}}%
\pgfpathcurveto{\pgfqpoint{0.024851in}{-0.014433in}}{\pgfqpoint{0.027778in}{-0.007367in}}{\pgfqpoint{0.027778in}{0.000000in}}%
\pgfpathcurveto{\pgfqpoint{0.027778in}{0.007367in}}{\pgfqpoint{0.024851in}{0.014433in}}{\pgfqpoint{0.019642in}{0.019642in}}%
\pgfpathcurveto{\pgfqpoint{0.014433in}{0.024851in}}{\pgfqpoint{0.007367in}{0.027778in}}{\pgfqpoint{0.000000in}{0.027778in}}%
\pgfpathcurveto{\pgfqpoint{-0.007367in}{0.027778in}}{\pgfqpoint{-0.014433in}{0.024851in}}{\pgfqpoint{-0.019642in}{0.019642in}}%
\pgfpathcurveto{\pgfqpoint{-0.024851in}{0.014433in}}{\pgfqpoint{-0.027778in}{0.007367in}}{\pgfqpoint{-0.027778in}{0.000000in}}%
\pgfpathcurveto{\pgfqpoint{-0.027778in}{-0.007367in}}{\pgfqpoint{-0.024851in}{-0.014433in}}{\pgfqpoint{-0.019642in}{-0.019642in}}%
\pgfpathcurveto{\pgfqpoint{-0.014433in}{-0.024851in}}{\pgfqpoint{-0.007367in}{-0.027778in}}{\pgfqpoint{0.000000in}{-0.027778in}}%
\pgfpathclose%
\pgfusepath{stroke,fill}%
}%
\begin{pgfscope}%
\pgfsys@transformshift{1.469026in}{1.418950in}%
\pgfsys@useobject{currentmarker}{}%
\end{pgfscope}%
\end{pgfscope}%
\begin{pgfscope}%
\pgfpathrectangle{\pgfqpoint{0.550713in}{0.127635in}}{\pgfqpoint{3.194133in}{2.297424in}}%
\pgfusepath{clip}%
\pgfsetrectcap%
\pgfsetroundjoin%
\pgfsetlinewidth{0.752812pt}%
\definecolor{currentstroke}{rgb}{0.000000,0.000000,0.000000}%
\pgfsetstrokecolor{currentstroke}%
\pgfsetdash{}{0pt}%
\pgfpathmoveto{\pgfqpoint{1.550476in}{1.337231in}}%
\pgfpathlineto{\pgfqpoint{1.706989in}{1.337231in}}%
\pgfusepath{stroke}%
\end{pgfscope}%
\begin{pgfscope}%
\pgfpathrectangle{\pgfqpoint{0.550713in}{0.127635in}}{\pgfqpoint{3.194133in}{2.297424in}}%
\pgfusepath{clip}%
\pgfsetbuttcap%
\pgfsetroundjoin%
\definecolor{currentfill}{rgb}{1.000000,1.000000,1.000000}%
\pgfsetfillcolor{currentfill}%
\pgfsetlinewidth{1.003750pt}%
\definecolor{currentstroke}{rgb}{0.000000,0.000000,0.000000}%
\pgfsetstrokecolor{currentstroke}%
\pgfsetdash{}{0pt}%
\pgfsys@defobject{currentmarker}{\pgfqpoint{-0.027778in}{-0.027778in}}{\pgfqpoint{0.027778in}{0.027778in}}{%
\pgfpathmoveto{\pgfqpoint{0.000000in}{-0.027778in}}%
\pgfpathcurveto{\pgfqpoint{0.007367in}{-0.027778in}}{\pgfqpoint{0.014433in}{-0.024851in}}{\pgfqpoint{0.019642in}{-0.019642in}}%
\pgfpathcurveto{\pgfqpoint{0.024851in}{-0.014433in}}{\pgfqpoint{0.027778in}{-0.007367in}}{\pgfqpoint{0.027778in}{0.000000in}}%
\pgfpathcurveto{\pgfqpoint{0.027778in}{0.007367in}}{\pgfqpoint{0.024851in}{0.014433in}}{\pgfqpoint{0.019642in}{0.019642in}}%
\pgfpathcurveto{\pgfqpoint{0.014433in}{0.024851in}}{\pgfqpoint{0.007367in}{0.027778in}}{\pgfqpoint{0.000000in}{0.027778in}}%
\pgfpathcurveto{\pgfqpoint{-0.007367in}{0.027778in}}{\pgfqpoint{-0.014433in}{0.024851in}}{\pgfqpoint{-0.019642in}{0.019642in}}%
\pgfpathcurveto{\pgfqpoint{-0.024851in}{0.014433in}}{\pgfqpoint{-0.027778in}{0.007367in}}{\pgfqpoint{-0.027778in}{0.000000in}}%
\pgfpathcurveto{\pgfqpoint{-0.027778in}{-0.007367in}}{\pgfqpoint{-0.024851in}{-0.014433in}}{\pgfqpoint{-0.019642in}{-0.019642in}}%
\pgfpathcurveto{\pgfqpoint{-0.014433in}{-0.024851in}}{\pgfqpoint{-0.007367in}{-0.027778in}}{\pgfqpoint{0.000000in}{-0.027778in}}%
\pgfpathclose%
\pgfusepath{stroke,fill}%
}%
\begin{pgfscope}%
\pgfsys@transformshift{1.628733in}{1.399835in}%
\pgfsys@useobject{currentmarker}{}%
\end{pgfscope}%
\end{pgfscope}%
\begin{pgfscope}%
\pgfpathrectangle{\pgfqpoint{0.550713in}{0.127635in}}{\pgfqpoint{3.194133in}{2.297424in}}%
\pgfusepath{clip}%
\pgfsetrectcap%
\pgfsetroundjoin%
\pgfsetlinewidth{0.752812pt}%
\definecolor{currentstroke}{rgb}{0.000000,0.000000,0.000000}%
\pgfsetstrokecolor{currentstroke}%
\pgfsetdash{}{0pt}%
\pgfpathmoveto{\pgfqpoint{1.790036in}{1.422504in}}%
\pgfpathlineto{\pgfqpoint{1.946549in}{1.422504in}}%
\pgfusepath{stroke}%
\end{pgfscope}%
\begin{pgfscope}%
\pgfpathrectangle{\pgfqpoint{0.550713in}{0.127635in}}{\pgfqpoint{3.194133in}{2.297424in}}%
\pgfusepath{clip}%
\pgfsetbuttcap%
\pgfsetroundjoin%
\definecolor{currentfill}{rgb}{1.000000,1.000000,1.000000}%
\pgfsetfillcolor{currentfill}%
\pgfsetlinewidth{1.003750pt}%
\definecolor{currentstroke}{rgb}{0.000000,0.000000,0.000000}%
\pgfsetstrokecolor{currentstroke}%
\pgfsetdash{}{0pt}%
\pgfsys@defobject{currentmarker}{\pgfqpoint{-0.027778in}{-0.027778in}}{\pgfqpoint{0.027778in}{0.027778in}}{%
\pgfpathmoveto{\pgfqpoint{0.000000in}{-0.027778in}}%
\pgfpathcurveto{\pgfqpoint{0.007367in}{-0.027778in}}{\pgfqpoint{0.014433in}{-0.024851in}}{\pgfqpoint{0.019642in}{-0.019642in}}%
\pgfpathcurveto{\pgfqpoint{0.024851in}{-0.014433in}}{\pgfqpoint{0.027778in}{-0.007367in}}{\pgfqpoint{0.027778in}{0.000000in}}%
\pgfpathcurveto{\pgfqpoint{0.027778in}{0.007367in}}{\pgfqpoint{0.024851in}{0.014433in}}{\pgfqpoint{0.019642in}{0.019642in}}%
\pgfpathcurveto{\pgfqpoint{0.014433in}{0.024851in}}{\pgfqpoint{0.007367in}{0.027778in}}{\pgfqpoint{0.000000in}{0.027778in}}%
\pgfpathcurveto{\pgfqpoint{-0.007367in}{0.027778in}}{\pgfqpoint{-0.014433in}{0.024851in}}{\pgfqpoint{-0.019642in}{0.019642in}}%
\pgfpathcurveto{\pgfqpoint{-0.024851in}{0.014433in}}{\pgfqpoint{-0.027778in}{0.007367in}}{\pgfqpoint{-0.027778in}{0.000000in}}%
\pgfpathcurveto{\pgfqpoint{-0.027778in}{-0.007367in}}{\pgfqpoint{-0.024851in}{-0.014433in}}{\pgfqpoint{-0.019642in}{-0.019642in}}%
\pgfpathcurveto{\pgfqpoint{-0.014433in}{-0.024851in}}{\pgfqpoint{-0.007367in}{-0.027778in}}{\pgfqpoint{0.000000in}{-0.027778in}}%
\pgfpathclose%
\pgfusepath{stroke,fill}%
}%
\begin{pgfscope}%
\pgfsys@transformshift{1.868293in}{1.404785in}%
\pgfsys@useobject{currentmarker}{}%
\end{pgfscope}%
\end{pgfscope}%
\begin{pgfscope}%
\pgfpathrectangle{\pgfqpoint{0.550713in}{0.127635in}}{\pgfqpoint{3.194133in}{2.297424in}}%
\pgfusepath{clip}%
\pgfsetrectcap%
\pgfsetroundjoin%
\pgfsetlinewidth{0.752812pt}%
\definecolor{currentstroke}{rgb}{0.000000,0.000000,0.000000}%
\pgfsetstrokecolor{currentstroke}%
\pgfsetdash{}{0pt}%
\pgfpathmoveto{\pgfqpoint{1.949743in}{1.486802in}}%
\pgfpathlineto{\pgfqpoint{2.106255in}{1.486802in}}%
\pgfusepath{stroke}%
\end{pgfscope}%
\begin{pgfscope}%
\pgfpathrectangle{\pgfqpoint{0.550713in}{0.127635in}}{\pgfqpoint{3.194133in}{2.297424in}}%
\pgfusepath{clip}%
\pgfsetbuttcap%
\pgfsetroundjoin%
\definecolor{currentfill}{rgb}{1.000000,1.000000,1.000000}%
\pgfsetfillcolor{currentfill}%
\pgfsetlinewidth{1.003750pt}%
\definecolor{currentstroke}{rgb}{0.000000,0.000000,0.000000}%
\pgfsetstrokecolor{currentstroke}%
\pgfsetdash{}{0pt}%
\pgfsys@defobject{currentmarker}{\pgfqpoint{-0.027778in}{-0.027778in}}{\pgfqpoint{0.027778in}{0.027778in}}{%
\pgfpathmoveto{\pgfqpoint{0.000000in}{-0.027778in}}%
\pgfpathcurveto{\pgfqpoint{0.007367in}{-0.027778in}}{\pgfqpoint{0.014433in}{-0.024851in}}{\pgfqpoint{0.019642in}{-0.019642in}}%
\pgfpathcurveto{\pgfqpoint{0.024851in}{-0.014433in}}{\pgfqpoint{0.027778in}{-0.007367in}}{\pgfqpoint{0.027778in}{0.000000in}}%
\pgfpathcurveto{\pgfqpoint{0.027778in}{0.007367in}}{\pgfqpoint{0.024851in}{0.014433in}}{\pgfqpoint{0.019642in}{0.019642in}}%
\pgfpathcurveto{\pgfqpoint{0.014433in}{0.024851in}}{\pgfqpoint{0.007367in}{0.027778in}}{\pgfqpoint{0.000000in}{0.027778in}}%
\pgfpathcurveto{\pgfqpoint{-0.007367in}{0.027778in}}{\pgfqpoint{-0.014433in}{0.024851in}}{\pgfqpoint{-0.019642in}{0.019642in}}%
\pgfpathcurveto{\pgfqpoint{-0.024851in}{0.014433in}}{\pgfqpoint{-0.027778in}{0.007367in}}{\pgfqpoint{-0.027778in}{0.000000in}}%
\pgfpathcurveto{\pgfqpoint{-0.027778in}{-0.007367in}}{\pgfqpoint{-0.024851in}{-0.014433in}}{\pgfqpoint{-0.019642in}{-0.019642in}}%
\pgfpathcurveto{\pgfqpoint{-0.014433in}{-0.024851in}}{\pgfqpoint{-0.007367in}{-0.027778in}}{\pgfqpoint{0.000000in}{-0.027778in}}%
\pgfpathclose%
\pgfusepath{stroke,fill}%
}%
\begin{pgfscope}%
\pgfsys@transformshift{2.027999in}{1.445038in}%
\pgfsys@useobject{currentmarker}{}%
\end{pgfscope}%
\end{pgfscope}%
\begin{pgfscope}%
\pgfpathrectangle{\pgfqpoint{0.550713in}{0.127635in}}{\pgfqpoint{3.194133in}{2.297424in}}%
\pgfusepath{clip}%
\pgfsetrectcap%
\pgfsetroundjoin%
\pgfsetlinewidth{0.752812pt}%
\definecolor{currentstroke}{rgb}{0.000000,0.000000,0.000000}%
\pgfsetstrokecolor{currentstroke}%
\pgfsetdash{}{0pt}%
\pgfpathmoveto{\pgfqpoint{2.189303in}{1.204150in}}%
\pgfpathlineto{\pgfqpoint{2.345815in}{1.204150in}}%
\pgfusepath{stroke}%
\end{pgfscope}%
\begin{pgfscope}%
\pgfpathrectangle{\pgfqpoint{0.550713in}{0.127635in}}{\pgfqpoint{3.194133in}{2.297424in}}%
\pgfusepath{clip}%
\pgfsetbuttcap%
\pgfsetroundjoin%
\definecolor{currentfill}{rgb}{1.000000,1.000000,1.000000}%
\pgfsetfillcolor{currentfill}%
\pgfsetlinewidth{1.003750pt}%
\definecolor{currentstroke}{rgb}{0.000000,0.000000,0.000000}%
\pgfsetstrokecolor{currentstroke}%
\pgfsetdash{}{0pt}%
\pgfsys@defobject{currentmarker}{\pgfqpoint{-0.027778in}{-0.027778in}}{\pgfqpoint{0.027778in}{0.027778in}}{%
\pgfpathmoveto{\pgfqpoint{0.000000in}{-0.027778in}}%
\pgfpathcurveto{\pgfqpoint{0.007367in}{-0.027778in}}{\pgfqpoint{0.014433in}{-0.024851in}}{\pgfqpoint{0.019642in}{-0.019642in}}%
\pgfpathcurveto{\pgfqpoint{0.024851in}{-0.014433in}}{\pgfqpoint{0.027778in}{-0.007367in}}{\pgfqpoint{0.027778in}{0.000000in}}%
\pgfpathcurveto{\pgfqpoint{0.027778in}{0.007367in}}{\pgfqpoint{0.024851in}{0.014433in}}{\pgfqpoint{0.019642in}{0.019642in}}%
\pgfpathcurveto{\pgfqpoint{0.014433in}{0.024851in}}{\pgfqpoint{0.007367in}{0.027778in}}{\pgfqpoint{0.000000in}{0.027778in}}%
\pgfpathcurveto{\pgfqpoint{-0.007367in}{0.027778in}}{\pgfqpoint{-0.014433in}{0.024851in}}{\pgfqpoint{-0.019642in}{0.019642in}}%
\pgfpathcurveto{\pgfqpoint{-0.024851in}{0.014433in}}{\pgfqpoint{-0.027778in}{0.007367in}}{\pgfqpoint{-0.027778in}{0.000000in}}%
\pgfpathcurveto{\pgfqpoint{-0.027778in}{-0.007367in}}{\pgfqpoint{-0.024851in}{-0.014433in}}{\pgfqpoint{-0.019642in}{-0.019642in}}%
\pgfpathcurveto{\pgfqpoint{-0.014433in}{-0.024851in}}{\pgfqpoint{-0.007367in}{-0.027778in}}{\pgfqpoint{0.000000in}{-0.027778in}}%
\pgfpathclose%
\pgfusepath{stroke,fill}%
}%
\begin{pgfscope}%
\pgfsys@transformshift{2.267559in}{1.168641in}%
\pgfsys@useobject{currentmarker}{}%
\end{pgfscope}%
\end{pgfscope}%
\begin{pgfscope}%
\pgfpathrectangle{\pgfqpoint{0.550713in}{0.127635in}}{\pgfqpoint{3.194133in}{2.297424in}}%
\pgfusepath{clip}%
\pgfsetrectcap%
\pgfsetroundjoin%
\pgfsetlinewidth{0.752812pt}%
\definecolor{currentstroke}{rgb}{0.000000,0.000000,0.000000}%
\pgfsetstrokecolor{currentstroke}%
\pgfsetdash{}{0pt}%
\pgfpathmoveto{\pgfqpoint{2.349010in}{1.296932in}}%
\pgfpathlineto{\pgfqpoint{2.505522in}{1.296932in}}%
\pgfusepath{stroke}%
\end{pgfscope}%
\begin{pgfscope}%
\pgfpathrectangle{\pgfqpoint{0.550713in}{0.127635in}}{\pgfqpoint{3.194133in}{2.297424in}}%
\pgfusepath{clip}%
\pgfsetbuttcap%
\pgfsetroundjoin%
\definecolor{currentfill}{rgb}{1.000000,1.000000,1.000000}%
\pgfsetfillcolor{currentfill}%
\pgfsetlinewidth{1.003750pt}%
\definecolor{currentstroke}{rgb}{0.000000,0.000000,0.000000}%
\pgfsetstrokecolor{currentstroke}%
\pgfsetdash{}{0pt}%
\pgfsys@defobject{currentmarker}{\pgfqpoint{-0.027778in}{-0.027778in}}{\pgfqpoint{0.027778in}{0.027778in}}{%
\pgfpathmoveto{\pgfqpoint{0.000000in}{-0.027778in}}%
\pgfpathcurveto{\pgfqpoint{0.007367in}{-0.027778in}}{\pgfqpoint{0.014433in}{-0.024851in}}{\pgfqpoint{0.019642in}{-0.019642in}}%
\pgfpathcurveto{\pgfqpoint{0.024851in}{-0.014433in}}{\pgfqpoint{0.027778in}{-0.007367in}}{\pgfqpoint{0.027778in}{0.000000in}}%
\pgfpathcurveto{\pgfqpoint{0.027778in}{0.007367in}}{\pgfqpoint{0.024851in}{0.014433in}}{\pgfqpoint{0.019642in}{0.019642in}}%
\pgfpathcurveto{\pgfqpoint{0.014433in}{0.024851in}}{\pgfqpoint{0.007367in}{0.027778in}}{\pgfqpoint{0.000000in}{0.027778in}}%
\pgfpathcurveto{\pgfqpoint{-0.007367in}{0.027778in}}{\pgfqpoint{-0.014433in}{0.024851in}}{\pgfqpoint{-0.019642in}{0.019642in}}%
\pgfpathcurveto{\pgfqpoint{-0.024851in}{0.014433in}}{\pgfqpoint{-0.027778in}{0.007367in}}{\pgfqpoint{-0.027778in}{0.000000in}}%
\pgfpathcurveto{\pgfqpoint{-0.027778in}{-0.007367in}}{\pgfqpoint{-0.024851in}{-0.014433in}}{\pgfqpoint{-0.019642in}{-0.019642in}}%
\pgfpathcurveto{\pgfqpoint{-0.014433in}{-0.024851in}}{\pgfqpoint{-0.007367in}{-0.027778in}}{\pgfqpoint{0.000000in}{-0.027778in}}%
\pgfpathclose%
\pgfusepath{stroke,fill}%
}%
\begin{pgfscope}%
\pgfsys@transformshift{2.427266in}{1.325666in}%
\pgfsys@useobject{currentmarker}{}%
\end{pgfscope}%
\end{pgfscope}%
\begin{pgfscope}%
\pgfpathrectangle{\pgfqpoint{0.550713in}{0.127635in}}{\pgfqpoint{3.194133in}{2.297424in}}%
\pgfusepath{clip}%
\pgfsetrectcap%
\pgfsetroundjoin%
\pgfsetlinewidth{0.752812pt}%
\definecolor{currentstroke}{rgb}{0.000000,0.000000,0.000000}%
\pgfsetstrokecolor{currentstroke}%
\pgfsetdash{}{0pt}%
\pgfpathmoveto{\pgfqpoint{2.588570in}{1.208465in}}%
\pgfpathlineto{\pgfqpoint{2.745082in}{1.208465in}}%
\pgfusepath{stroke}%
\end{pgfscope}%
\begin{pgfscope}%
\pgfpathrectangle{\pgfqpoint{0.550713in}{0.127635in}}{\pgfqpoint{3.194133in}{2.297424in}}%
\pgfusepath{clip}%
\pgfsetbuttcap%
\pgfsetroundjoin%
\definecolor{currentfill}{rgb}{1.000000,1.000000,1.000000}%
\pgfsetfillcolor{currentfill}%
\pgfsetlinewidth{1.003750pt}%
\definecolor{currentstroke}{rgb}{0.000000,0.000000,0.000000}%
\pgfsetstrokecolor{currentstroke}%
\pgfsetdash{}{0pt}%
\pgfsys@defobject{currentmarker}{\pgfqpoint{-0.027778in}{-0.027778in}}{\pgfqpoint{0.027778in}{0.027778in}}{%
\pgfpathmoveto{\pgfqpoint{0.000000in}{-0.027778in}}%
\pgfpathcurveto{\pgfqpoint{0.007367in}{-0.027778in}}{\pgfqpoint{0.014433in}{-0.024851in}}{\pgfqpoint{0.019642in}{-0.019642in}}%
\pgfpathcurveto{\pgfqpoint{0.024851in}{-0.014433in}}{\pgfqpoint{0.027778in}{-0.007367in}}{\pgfqpoint{0.027778in}{0.000000in}}%
\pgfpathcurveto{\pgfqpoint{0.027778in}{0.007367in}}{\pgfqpoint{0.024851in}{0.014433in}}{\pgfqpoint{0.019642in}{0.019642in}}%
\pgfpathcurveto{\pgfqpoint{0.014433in}{0.024851in}}{\pgfqpoint{0.007367in}{0.027778in}}{\pgfqpoint{0.000000in}{0.027778in}}%
\pgfpathcurveto{\pgfqpoint{-0.007367in}{0.027778in}}{\pgfqpoint{-0.014433in}{0.024851in}}{\pgfqpoint{-0.019642in}{0.019642in}}%
\pgfpathcurveto{\pgfqpoint{-0.024851in}{0.014433in}}{\pgfqpoint{-0.027778in}{0.007367in}}{\pgfqpoint{-0.027778in}{0.000000in}}%
\pgfpathcurveto{\pgfqpoint{-0.027778in}{-0.007367in}}{\pgfqpoint{-0.024851in}{-0.014433in}}{\pgfqpoint{-0.019642in}{-0.019642in}}%
\pgfpathcurveto{\pgfqpoint{-0.014433in}{-0.024851in}}{\pgfqpoint{-0.007367in}{-0.027778in}}{\pgfqpoint{0.000000in}{-0.027778in}}%
\pgfpathclose%
\pgfusepath{stroke,fill}%
}%
\begin{pgfscope}%
\pgfsys@transformshift{2.666826in}{1.162693in}%
\pgfsys@useobject{currentmarker}{}%
\end{pgfscope}%
\end{pgfscope}%
\begin{pgfscope}%
\pgfpathrectangle{\pgfqpoint{0.550713in}{0.127635in}}{\pgfqpoint{3.194133in}{2.297424in}}%
\pgfusepath{clip}%
\pgfsetrectcap%
\pgfsetroundjoin%
\pgfsetlinewidth{0.752812pt}%
\definecolor{currentstroke}{rgb}{0.000000,0.000000,0.000000}%
\pgfsetstrokecolor{currentstroke}%
\pgfsetdash{}{0pt}%
\pgfpathmoveto{\pgfqpoint{2.748276in}{1.317410in}}%
\pgfpathlineto{\pgfqpoint{2.904789in}{1.317410in}}%
\pgfusepath{stroke}%
\end{pgfscope}%
\begin{pgfscope}%
\pgfpathrectangle{\pgfqpoint{0.550713in}{0.127635in}}{\pgfqpoint{3.194133in}{2.297424in}}%
\pgfusepath{clip}%
\pgfsetbuttcap%
\pgfsetroundjoin%
\definecolor{currentfill}{rgb}{1.000000,1.000000,1.000000}%
\pgfsetfillcolor{currentfill}%
\pgfsetlinewidth{1.003750pt}%
\definecolor{currentstroke}{rgb}{0.000000,0.000000,0.000000}%
\pgfsetstrokecolor{currentstroke}%
\pgfsetdash{}{0pt}%
\pgfsys@defobject{currentmarker}{\pgfqpoint{-0.027778in}{-0.027778in}}{\pgfqpoint{0.027778in}{0.027778in}}{%
\pgfpathmoveto{\pgfqpoint{0.000000in}{-0.027778in}}%
\pgfpathcurveto{\pgfqpoint{0.007367in}{-0.027778in}}{\pgfqpoint{0.014433in}{-0.024851in}}{\pgfqpoint{0.019642in}{-0.019642in}}%
\pgfpathcurveto{\pgfqpoint{0.024851in}{-0.014433in}}{\pgfqpoint{0.027778in}{-0.007367in}}{\pgfqpoint{0.027778in}{0.000000in}}%
\pgfpathcurveto{\pgfqpoint{0.027778in}{0.007367in}}{\pgfqpoint{0.024851in}{0.014433in}}{\pgfqpoint{0.019642in}{0.019642in}}%
\pgfpathcurveto{\pgfqpoint{0.014433in}{0.024851in}}{\pgfqpoint{0.007367in}{0.027778in}}{\pgfqpoint{0.000000in}{0.027778in}}%
\pgfpathcurveto{\pgfqpoint{-0.007367in}{0.027778in}}{\pgfqpoint{-0.014433in}{0.024851in}}{\pgfqpoint{-0.019642in}{0.019642in}}%
\pgfpathcurveto{\pgfqpoint{-0.024851in}{0.014433in}}{\pgfqpoint{-0.027778in}{0.007367in}}{\pgfqpoint{-0.027778in}{0.000000in}}%
\pgfpathcurveto{\pgfqpoint{-0.027778in}{-0.007367in}}{\pgfqpoint{-0.024851in}{-0.014433in}}{\pgfqpoint{-0.019642in}{-0.019642in}}%
\pgfpathcurveto{\pgfqpoint{-0.014433in}{-0.024851in}}{\pgfqpoint{-0.007367in}{-0.027778in}}{\pgfqpoint{0.000000in}{-0.027778in}}%
\pgfpathclose%
\pgfusepath{stroke,fill}%
}%
\begin{pgfscope}%
\pgfsys@transformshift{2.826532in}{1.308730in}%
\pgfsys@useobject{currentmarker}{}%
\end{pgfscope}%
\end{pgfscope}%
\begin{pgfscope}%
\pgfpathrectangle{\pgfqpoint{0.550713in}{0.127635in}}{\pgfqpoint{3.194133in}{2.297424in}}%
\pgfusepath{clip}%
\pgfsetrectcap%
\pgfsetroundjoin%
\pgfsetlinewidth{0.752812pt}%
\definecolor{currentstroke}{rgb}{0.000000,0.000000,0.000000}%
\pgfsetstrokecolor{currentstroke}%
\pgfsetdash{}{0pt}%
\pgfpathmoveto{\pgfqpoint{2.987836in}{1.196563in}}%
\pgfpathlineto{\pgfqpoint{3.144349in}{1.196563in}}%
\pgfusepath{stroke}%
\end{pgfscope}%
\begin{pgfscope}%
\pgfpathrectangle{\pgfqpoint{0.550713in}{0.127635in}}{\pgfqpoint{3.194133in}{2.297424in}}%
\pgfusepath{clip}%
\pgfsetbuttcap%
\pgfsetroundjoin%
\definecolor{currentfill}{rgb}{1.000000,1.000000,1.000000}%
\pgfsetfillcolor{currentfill}%
\pgfsetlinewidth{1.003750pt}%
\definecolor{currentstroke}{rgb}{0.000000,0.000000,0.000000}%
\pgfsetstrokecolor{currentstroke}%
\pgfsetdash{}{0pt}%
\pgfsys@defobject{currentmarker}{\pgfqpoint{-0.027778in}{-0.027778in}}{\pgfqpoint{0.027778in}{0.027778in}}{%
\pgfpathmoveto{\pgfqpoint{0.000000in}{-0.027778in}}%
\pgfpathcurveto{\pgfqpoint{0.007367in}{-0.027778in}}{\pgfqpoint{0.014433in}{-0.024851in}}{\pgfqpoint{0.019642in}{-0.019642in}}%
\pgfpathcurveto{\pgfqpoint{0.024851in}{-0.014433in}}{\pgfqpoint{0.027778in}{-0.007367in}}{\pgfqpoint{0.027778in}{0.000000in}}%
\pgfpathcurveto{\pgfqpoint{0.027778in}{0.007367in}}{\pgfqpoint{0.024851in}{0.014433in}}{\pgfqpoint{0.019642in}{0.019642in}}%
\pgfpathcurveto{\pgfqpoint{0.014433in}{0.024851in}}{\pgfqpoint{0.007367in}{0.027778in}}{\pgfqpoint{0.000000in}{0.027778in}}%
\pgfpathcurveto{\pgfqpoint{-0.007367in}{0.027778in}}{\pgfqpoint{-0.014433in}{0.024851in}}{\pgfqpoint{-0.019642in}{0.019642in}}%
\pgfpathcurveto{\pgfqpoint{-0.024851in}{0.014433in}}{\pgfqpoint{-0.027778in}{0.007367in}}{\pgfqpoint{-0.027778in}{0.000000in}}%
\pgfpathcurveto{\pgfqpoint{-0.027778in}{-0.007367in}}{\pgfqpoint{-0.024851in}{-0.014433in}}{\pgfqpoint{-0.019642in}{-0.019642in}}%
\pgfpathcurveto{\pgfqpoint{-0.014433in}{-0.024851in}}{\pgfqpoint{-0.007367in}{-0.027778in}}{\pgfqpoint{0.000000in}{-0.027778in}}%
\pgfpathclose%
\pgfusepath{stroke,fill}%
}%
\begin{pgfscope}%
\pgfsys@transformshift{3.066092in}{1.176364in}%
\pgfsys@useobject{currentmarker}{}%
\end{pgfscope}%
\end{pgfscope}%
\begin{pgfscope}%
\pgfpathrectangle{\pgfqpoint{0.550713in}{0.127635in}}{\pgfqpoint{3.194133in}{2.297424in}}%
\pgfusepath{clip}%
\pgfsetrectcap%
\pgfsetroundjoin%
\pgfsetlinewidth{0.752812pt}%
\definecolor{currentstroke}{rgb}{0.000000,0.000000,0.000000}%
\pgfsetstrokecolor{currentstroke}%
\pgfsetdash{}{0pt}%
\pgfpathmoveto{\pgfqpoint{3.147543in}{1.229974in}}%
\pgfpathlineto{\pgfqpoint{3.304055in}{1.229974in}}%
\pgfusepath{stroke}%
\end{pgfscope}%
\begin{pgfscope}%
\pgfpathrectangle{\pgfqpoint{0.550713in}{0.127635in}}{\pgfqpoint{3.194133in}{2.297424in}}%
\pgfusepath{clip}%
\pgfsetbuttcap%
\pgfsetroundjoin%
\definecolor{currentfill}{rgb}{1.000000,1.000000,1.000000}%
\pgfsetfillcolor{currentfill}%
\pgfsetlinewidth{1.003750pt}%
\definecolor{currentstroke}{rgb}{0.000000,0.000000,0.000000}%
\pgfsetstrokecolor{currentstroke}%
\pgfsetdash{}{0pt}%
\pgfsys@defobject{currentmarker}{\pgfqpoint{-0.027778in}{-0.027778in}}{\pgfqpoint{0.027778in}{0.027778in}}{%
\pgfpathmoveto{\pgfqpoint{0.000000in}{-0.027778in}}%
\pgfpathcurveto{\pgfqpoint{0.007367in}{-0.027778in}}{\pgfqpoint{0.014433in}{-0.024851in}}{\pgfqpoint{0.019642in}{-0.019642in}}%
\pgfpathcurveto{\pgfqpoint{0.024851in}{-0.014433in}}{\pgfqpoint{0.027778in}{-0.007367in}}{\pgfqpoint{0.027778in}{0.000000in}}%
\pgfpathcurveto{\pgfqpoint{0.027778in}{0.007367in}}{\pgfqpoint{0.024851in}{0.014433in}}{\pgfqpoint{0.019642in}{0.019642in}}%
\pgfpathcurveto{\pgfqpoint{0.014433in}{0.024851in}}{\pgfqpoint{0.007367in}{0.027778in}}{\pgfqpoint{0.000000in}{0.027778in}}%
\pgfpathcurveto{\pgfqpoint{-0.007367in}{0.027778in}}{\pgfqpoint{-0.014433in}{0.024851in}}{\pgfqpoint{-0.019642in}{0.019642in}}%
\pgfpathcurveto{\pgfqpoint{-0.024851in}{0.014433in}}{\pgfqpoint{-0.027778in}{0.007367in}}{\pgfqpoint{-0.027778in}{0.000000in}}%
\pgfpathcurveto{\pgfqpoint{-0.027778in}{-0.007367in}}{\pgfqpoint{-0.024851in}{-0.014433in}}{\pgfqpoint{-0.019642in}{-0.019642in}}%
\pgfpathcurveto{\pgfqpoint{-0.014433in}{-0.024851in}}{\pgfqpoint{-0.007367in}{-0.027778in}}{\pgfqpoint{0.000000in}{-0.027778in}}%
\pgfpathclose%
\pgfusepath{stroke,fill}%
}%
\begin{pgfscope}%
\pgfsys@transformshift{3.225799in}{1.184736in}%
\pgfsys@useobject{currentmarker}{}%
\end{pgfscope}%
\end{pgfscope}%
\begin{pgfscope}%
\pgfpathrectangle{\pgfqpoint{0.550713in}{0.127635in}}{\pgfqpoint{3.194133in}{2.297424in}}%
\pgfusepath{clip}%
\pgfsetrectcap%
\pgfsetroundjoin%
\pgfsetlinewidth{0.752812pt}%
\definecolor{currentstroke}{rgb}{0.000000,0.000000,0.000000}%
\pgfsetstrokecolor{currentstroke}%
\pgfsetdash{}{0pt}%
\pgfpathmoveto{\pgfqpoint{3.387103in}{1.214798in}}%
\pgfpathlineto{\pgfqpoint{3.543615in}{1.214798in}}%
\pgfusepath{stroke}%
\end{pgfscope}%
\begin{pgfscope}%
\pgfpathrectangle{\pgfqpoint{0.550713in}{0.127635in}}{\pgfqpoint{3.194133in}{2.297424in}}%
\pgfusepath{clip}%
\pgfsetbuttcap%
\pgfsetroundjoin%
\definecolor{currentfill}{rgb}{1.000000,1.000000,1.000000}%
\pgfsetfillcolor{currentfill}%
\pgfsetlinewidth{1.003750pt}%
\definecolor{currentstroke}{rgb}{0.000000,0.000000,0.000000}%
\pgfsetstrokecolor{currentstroke}%
\pgfsetdash{}{0pt}%
\pgfsys@defobject{currentmarker}{\pgfqpoint{-0.027778in}{-0.027778in}}{\pgfqpoint{0.027778in}{0.027778in}}{%
\pgfpathmoveto{\pgfqpoint{0.000000in}{-0.027778in}}%
\pgfpathcurveto{\pgfqpoint{0.007367in}{-0.027778in}}{\pgfqpoint{0.014433in}{-0.024851in}}{\pgfqpoint{0.019642in}{-0.019642in}}%
\pgfpathcurveto{\pgfqpoint{0.024851in}{-0.014433in}}{\pgfqpoint{0.027778in}{-0.007367in}}{\pgfqpoint{0.027778in}{0.000000in}}%
\pgfpathcurveto{\pgfqpoint{0.027778in}{0.007367in}}{\pgfqpoint{0.024851in}{0.014433in}}{\pgfqpoint{0.019642in}{0.019642in}}%
\pgfpathcurveto{\pgfqpoint{0.014433in}{0.024851in}}{\pgfqpoint{0.007367in}{0.027778in}}{\pgfqpoint{0.000000in}{0.027778in}}%
\pgfpathcurveto{\pgfqpoint{-0.007367in}{0.027778in}}{\pgfqpoint{-0.014433in}{0.024851in}}{\pgfqpoint{-0.019642in}{0.019642in}}%
\pgfpathcurveto{\pgfqpoint{-0.024851in}{0.014433in}}{\pgfqpoint{-0.027778in}{0.007367in}}{\pgfqpoint{-0.027778in}{0.000000in}}%
\pgfpathcurveto{\pgfqpoint{-0.027778in}{-0.007367in}}{\pgfqpoint{-0.024851in}{-0.014433in}}{\pgfqpoint{-0.019642in}{-0.019642in}}%
\pgfpathcurveto{\pgfqpoint{-0.014433in}{-0.024851in}}{\pgfqpoint{-0.007367in}{-0.027778in}}{\pgfqpoint{0.000000in}{-0.027778in}}%
\pgfpathclose%
\pgfusepath{stroke,fill}%
}%
\begin{pgfscope}%
\pgfsys@transformshift{3.465359in}{1.188835in}%
\pgfsys@useobject{currentmarker}{}%
\end{pgfscope}%
\end{pgfscope}%
\begin{pgfscope}%
\pgfpathrectangle{\pgfqpoint{0.550713in}{0.127635in}}{\pgfqpoint{3.194133in}{2.297424in}}%
\pgfusepath{clip}%
\pgfsetrectcap%
\pgfsetroundjoin%
\pgfsetlinewidth{0.752812pt}%
\definecolor{currentstroke}{rgb}{0.000000,0.000000,0.000000}%
\pgfsetstrokecolor{currentstroke}%
\pgfsetdash{}{0pt}%
\pgfpathmoveto{\pgfqpoint{3.546809in}{1.277867in}}%
\pgfpathlineto{\pgfqpoint{3.703322in}{1.277867in}}%
\pgfusepath{stroke}%
\end{pgfscope}%
\begin{pgfscope}%
\pgfpathrectangle{\pgfqpoint{0.550713in}{0.127635in}}{\pgfqpoint{3.194133in}{2.297424in}}%
\pgfusepath{clip}%
\pgfsetbuttcap%
\pgfsetroundjoin%
\definecolor{currentfill}{rgb}{1.000000,1.000000,1.000000}%
\pgfsetfillcolor{currentfill}%
\pgfsetlinewidth{1.003750pt}%
\definecolor{currentstroke}{rgb}{0.000000,0.000000,0.000000}%
\pgfsetstrokecolor{currentstroke}%
\pgfsetdash{}{0pt}%
\pgfsys@defobject{currentmarker}{\pgfqpoint{-0.027778in}{-0.027778in}}{\pgfqpoint{0.027778in}{0.027778in}}{%
\pgfpathmoveto{\pgfqpoint{0.000000in}{-0.027778in}}%
\pgfpathcurveto{\pgfqpoint{0.007367in}{-0.027778in}}{\pgfqpoint{0.014433in}{-0.024851in}}{\pgfqpoint{0.019642in}{-0.019642in}}%
\pgfpathcurveto{\pgfqpoint{0.024851in}{-0.014433in}}{\pgfqpoint{0.027778in}{-0.007367in}}{\pgfqpoint{0.027778in}{0.000000in}}%
\pgfpathcurveto{\pgfqpoint{0.027778in}{0.007367in}}{\pgfqpoint{0.024851in}{0.014433in}}{\pgfqpoint{0.019642in}{0.019642in}}%
\pgfpathcurveto{\pgfqpoint{0.014433in}{0.024851in}}{\pgfqpoint{0.007367in}{0.027778in}}{\pgfqpoint{0.000000in}{0.027778in}}%
\pgfpathcurveto{\pgfqpoint{-0.007367in}{0.027778in}}{\pgfqpoint{-0.014433in}{0.024851in}}{\pgfqpoint{-0.019642in}{0.019642in}}%
\pgfpathcurveto{\pgfqpoint{-0.024851in}{0.014433in}}{\pgfqpoint{-0.027778in}{0.007367in}}{\pgfqpoint{-0.027778in}{0.000000in}}%
\pgfpathcurveto{\pgfqpoint{-0.027778in}{-0.007367in}}{\pgfqpoint{-0.024851in}{-0.014433in}}{\pgfqpoint{-0.019642in}{-0.019642in}}%
\pgfpathcurveto{\pgfqpoint{-0.014433in}{-0.024851in}}{\pgfqpoint{-0.007367in}{-0.027778in}}{\pgfqpoint{0.000000in}{-0.027778in}}%
\pgfpathclose%
\pgfusepath{stroke,fill}%
}%
\begin{pgfscope}%
\pgfsys@transformshift{3.625066in}{1.259364in}%
\pgfsys@useobject{currentmarker}{}%
\end{pgfscope}%
\end{pgfscope}%
\begin{pgfscope}%
\pgfsetrectcap%
\pgfsetmiterjoin%
\pgfsetlinewidth{0.752812pt}%
\definecolor{currentstroke}{rgb}{0.000000,0.000000,0.000000}%
\pgfsetstrokecolor{currentstroke}%
\pgfsetdash{}{0pt}%
\pgfpathmoveto{\pgfqpoint{0.550713in}{0.127635in}}%
\pgfpathlineto{\pgfqpoint{0.550713in}{2.425059in}}%
\pgfusepath{stroke}%
\end{pgfscope}%
\begin{pgfscope}%
\pgfsetrectcap%
\pgfsetmiterjoin%
\pgfsetlinewidth{0.752812pt}%
\definecolor{currentstroke}{rgb}{0.000000,0.000000,0.000000}%
\pgfsetstrokecolor{currentstroke}%
\pgfsetdash{}{0pt}%
\pgfpathmoveto{\pgfqpoint{3.744846in}{0.127635in}}%
\pgfpathlineto{\pgfqpoint{3.744846in}{2.425059in}}%
\pgfusepath{stroke}%
\end{pgfscope}%
\begin{pgfscope}%
\pgfsetrectcap%
\pgfsetmiterjoin%
\pgfsetlinewidth{0.752812pt}%
\definecolor{currentstroke}{rgb}{0.000000,0.000000,0.000000}%
\pgfsetstrokecolor{currentstroke}%
\pgfsetdash{}{0pt}%
\pgfpathmoveto{\pgfqpoint{0.550713in}{0.127635in}}%
\pgfpathlineto{\pgfqpoint{3.744846in}{0.127635in}}%
\pgfusepath{stroke}%
\end{pgfscope}%
\begin{pgfscope}%
\pgfsetrectcap%
\pgfsetmiterjoin%
\pgfsetlinewidth{0.752812pt}%
\definecolor{currentstroke}{rgb}{0.000000,0.000000,0.000000}%
\pgfsetstrokecolor{currentstroke}%
\pgfsetdash{}{0pt}%
\pgfpathmoveto{\pgfqpoint{0.550713in}{2.425059in}}%
\pgfpathlineto{\pgfqpoint{3.744846in}{2.425059in}}%
\pgfusepath{stroke}%
\end{pgfscope}%
\begin{pgfscope}%
\pgfsetbuttcap%
\pgfsetroundjoin%
\pgfsetlinewidth{1.003750pt}%
\definecolor{currentstroke}{rgb}{0.392157,0.396078,0.403922}%
\pgfsetstrokecolor{currentstroke}%
\pgfsetdash{{3.700000pt}{1.600000pt}}{0.000000pt}%
\pgfpathmoveto{\pgfqpoint{3.869846in}{2.051207in}}%
\pgfpathlineto{\pgfqpoint{4.147623in}{2.051207in}}%
\pgfusepath{stroke}%
\end{pgfscope}%
\begin{pgfscope}%
\definecolor{textcolor}{rgb}{0.000000,0.000000,0.000000}%
\pgfsetstrokecolor{textcolor}%
\pgfsetfillcolor{textcolor}%
\pgftext[x=4.258735in, y=2.087374in, left, base]{\color{textcolor}\rmfamily\fontsize{10.000000}{12.000000}\selectfont Only Exploitation:}%
\end{pgfscope}%
\begin{pgfscope}%
\definecolor{textcolor}{rgb}{0.000000,0.000000,0.000000}%
\pgfsetstrokecolor{textcolor}%
\pgfsetfillcolor{textcolor}%
\pgftext[x=4.258735in, y=1.943235in, left, base]{\color{textcolor}\rmfamily\fontsize{10.000000}{12.000000}\selectfont \(\displaystyle R_T=154.62\)}%
\end{pgfscope}%
\begin{pgfscope}%
\pgfsetbuttcap%
\pgfsetmiterjoin%
\definecolor{currentfill}{rgb}{0.631373,0.062745,0.207843}%
\pgfsetfillcolor{currentfill}%
\pgfsetlinewidth{0.000000pt}%
\definecolor{currentstroke}{rgb}{0.000000,0.000000,0.000000}%
\pgfsetstrokecolor{currentstroke}%
\pgfsetstrokeopacity{0.000000}%
\pgfsetdash{}{0pt}%
\pgfpathmoveto{\pgfqpoint{3.869846in}{1.749624in}}%
\pgfpathlineto{\pgfqpoint{4.147623in}{1.749624in}}%
\pgfpathlineto{\pgfqpoint{4.147623in}{1.846846in}}%
\pgfpathlineto{\pgfqpoint{3.869846in}{1.846846in}}%
\pgfpathclose%
\pgfusepath{fill}%
\end{pgfscope}%
\begin{pgfscope}%
\definecolor{textcolor}{rgb}{0.000000,0.000000,0.000000}%
\pgfsetstrokecolor{textcolor}%
\pgfsetfillcolor{textcolor}%
\pgftext[x=4.258735in,y=1.749624in,left,base]{\color{textcolor}\rmfamily\fontsize{10.000000}{12.000000}\selectfont TV-GP-UCB}%
\end{pgfscope}%
\begin{pgfscope}%
\pgfsetbuttcap%
\pgfsetmiterjoin%
\definecolor{currentfill}{rgb}{0.890196,0.000000,0.400000}%
\pgfsetfillcolor{currentfill}%
\pgfsetlinewidth{0.000000pt}%
\definecolor{currentstroke}{rgb}{0.000000,0.000000,0.000000}%
\pgfsetstrokecolor{currentstroke}%
\pgfsetstrokeopacity{0.000000}%
\pgfsetdash{}{0pt}%
\pgfpathmoveto{\pgfqpoint{3.869846in}{1.556013in}}%
\pgfpathlineto{\pgfqpoint{4.147623in}{1.556013in}}%
\pgfpathlineto{\pgfqpoint{4.147623in}{1.653235in}}%
\pgfpathlineto{\pgfqpoint{3.869846in}{1.653235in}}%
\pgfpathclose%
\pgfusepath{fill}%
\end{pgfscope}%
\begin{pgfscope}%
\definecolor{textcolor}{rgb}{0.000000,0.000000,0.000000}%
\pgfsetstrokecolor{textcolor}%
\pgfsetfillcolor{textcolor}%
\pgftext[x=4.258735in,y=1.556013in,left,base]{\color{textcolor}\rmfamily\fontsize{10.000000}{12.000000}\selectfont SW TV-GP-UCB}%
\end{pgfscope}%
\begin{pgfscope}%
\pgfsetbuttcap%
\pgfsetmiterjoin%
\definecolor{currentfill}{rgb}{0.000000,0.329412,0.623529}%
\pgfsetfillcolor{currentfill}%
\pgfsetlinewidth{0.000000pt}%
\definecolor{currentstroke}{rgb}{0.000000,0.000000,0.000000}%
\pgfsetstrokecolor{currentstroke}%
\pgfsetstrokeopacity{0.000000}%
\pgfsetdash{}{0pt}%
\pgfpathmoveto{\pgfqpoint{3.869846in}{1.362402in}}%
\pgfpathlineto{\pgfqpoint{4.147623in}{1.362402in}}%
\pgfpathlineto{\pgfqpoint{4.147623in}{1.459624in}}%
\pgfpathlineto{\pgfqpoint{3.869846in}{1.459624in}}%
\pgfpathclose%
\pgfusepath{fill}%
\end{pgfscope}%
\begin{pgfscope}%
\definecolor{textcolor}{rgb}{0.000000,0.000000,0.000000}%
\pgfsetstrokecolor{textcolor}%
\pgfsetfillcolor{textcolor}%
\pgftext[x=4.258735in,y=1.362402in,left,base]{\color{textcolor}\rmfamily\fontsize{10.000000}{12.000000}\selectfont UI-TVBO}%
\end{pgfscope}%
\begin{pgfscope}%
\pgfsetbuttcap%
\pgfsetmiterjoin%
\definecolor{currentfill}{rgb}{0.000000,0.380392,0.396078}%
\pgfsetfillcolor{currentfill}%
\pgfsetlinewidth{0.000000pt}%
\definecolor{currentstroke}{rgb}{0.000000,0.000000,0.000000}%
\pgfsetstrokecolor{currentstroke}%
\pgfsetstrokeopacity{0.000000}%
\pgfsetdash{}{0pt}%
\pgfpathmoveto{\pgfqpoint{3.869846in}{1.168791in}}%
\pgfpathlineto{\pgfqpoint{4.147623in}{1.168791in}}%
\pgfpathlineto{\pgfqpoint{4.147623in}{1.266013in}}%
\pgfpathlineto{\pgfqpoint{3.869846in}{1.266013in}}%
\pgfpathclose%
\pgfusepath{fill}%
\end{pgfscope}%
\begin{pgfscope}%
\definecolor{textcolor}{rgb}{0.000000,0.000000,0.000000}%
\pgfsetstrokecolor{textcolor}%
\pgfsetfillcolor{textcolor}%
\pgftext[x=4.258735in,y=1.168791in,left,base]{\color{textcolor}\rmfamily\fontsize{10.000000}{12.000000}\selectfont B UI-TVBO}%
\end{pgfscope}%
\begin{pgfscope}%
\pgfsetbuttcap%
\pgfsetmiterjoin%
\definecolor{currentfill}{rgb}{0.380392,0.129412,0.345098}%
\pgfsetfillcolor{currentfill}%
\pgfsetlinewidth{0.000000pt}%
\definecolor{currentstroke}{rgb}{0.000000,0.000000,0.000000}%
\pgfsetstrokecolor{currentstroke}%
\pgfsetstrokeopacity{0.000000}%
\pgfsetdash{}{0pt}%
\pgfpathmoveto{\pgfqpoint{3.869846in}{0.975180in}}%
\pgfpathlineto{\pgfqpoint{4.147623in}{0.975180in}}%
\pgfpathlineto{\pgfqpoint{4.147623in}{1.072402in}}%
\pgfpathlineto{\pgfqpoint{3.869846in}{1.072402in}}%
\pgfpathclose%
\pgfusepath{fill}%
\end{pgfscope}%
\begin{pgfscope}%
\definecolor{textcolor}{rgb}{0.000000,0.000000,0.000000}%
\pgfsetstrokecolor{textcolor}%
\pgfsetfillcolor{textcolor}%
\pgftext[x=4.258735in,y=0.975180in,left,base]{\color{textcolor}\rmfamily\fontsize{10.000000}{12.000000}\selectfont C-TV-GP-UCB}%
\end{pgfscope}%
\begin{pgfscope}%
\pgfsetbuttcap%
\pgfsetmiterjoin%
\definecolor{currentfill}{rgb}{0.964706,0.658824,0.000000}%
\pgfsetfillcolor{currentfill}%
\pgfsetlinewidth{0.000000pt}%
\definecolor{currentstroke}{rgb}{0.000000,0.000000,0.000000}%
\pgfsetstrokecolor{currentstroke}%
\pgfsetstrokeopacity{0.000000}%
\pgfsetdash{}{0pt}%
\pgfpathmoveto{\pgfqpoint{3.869846in}{0.781569in}}%
\pgfpathlineto{\pgfqpoint{4.147623in}{0.781569in}}%
\pgfpathlineto{\pgfqpoint{4.147623in}{0.878791in}}%
\pgfpathlineto{\pgfqpoint{3.869846in}{0.878791in}}%
\pgfpathclose%
\pgfusepath{fill}%
\end{pgfscope}%
\begin{pgfscope}%
\definecolor{textcolor}{rgb}{0.000000,0.000000,0.000000}%
\pgfsetstrokecolor{textcolor}%
\pgfsetfillcolor{textcolor}%
\pgftext[x=4.258735in,y=0.781569in,left,base]{\color{textcolor}\rmfamily\fontsize{10.000000}{12.000000}\selectfont SW C-TV-GP-UCB}%
\end{pgfscope}%
\begin{pgfscope}%
\pgfsetbuttcap%
\pgfsetmiterjoin%
\definecolor{currentfill}{rgb}{0.341176,0.670588,0.152941}%
\pgfsetfillcolor{currentfill}%
\pgfsetlinewidth{0.000000pt}%
\definecolor{currentstroke}{rgb}{0.000000,0.000000,0.000000}%
\pgfsetstrokecolor{currentstroke}%
\pgfsetstrokeopacity{0.000000}%
\pgfsetdash{}{0pt}%
\pgfpathmoveto{\pgfqpoint{3.869846in}{0.587958in}}%
\pgfpathlineto{\pgfqpoint{4.147623in}{0.587958in}}%
\pgfpathlineto{\pgfqpoint{4.147623in}{0.685180in}}%
\pgfpathlineto{\pgfqpoint{3.869846in}{0.685180in}}%
\pgfpathclose%
\pgfusepath{fill}%
\end{pgfscope}%
\begin{pgfscope}%
\definecolor{textcolor}{rgb}{0.000000,0.000000,0.000000}%
\pgfsetstrokecolor{textcolor}%
\pgfsetfillcolor{textcolor}%
\pgftext[x=4.258735in,y=0.587958in,left,base]{\color{textcolor}\rmfamily\fontsize{10.000000}{12.000000}\selectfont C-UI-TVBO}%
\end{pgfscope}%
\begin{pgfscope}%
\pgfsetbuttcap%
\pgfsetmiterjoin%
\definecolor{currentfill}{rgb}{0.478431,0.435294,0.674510}%
\pgfsetfillcolor{currentfill}%
\pgfsetlinewidth{0.000000pt}%
\definecolor{currentstroke}{rgb}{0.000000,0.000000,0.000000}%
\pgfsetstrokecolor{currentstroke}%
\pgfsetstrokeopacity{0.000000}%
\pgfsetdash{}{0pt}%
\pgfpathmoveto{\pgfqpoint{3.869846in}{0.394347in}}%
\pgfpathlineto{\pgfqpoint{4.147623in}{0.394347in}}%
\pgfpathlineto{\pgfqpoint{4.147623in}{0.491569in}}%
\pgfpathlineto{\pgfqpoint{3.869846in}{0.491569in}}%
\pgfpathclose%
\pgfusepath{fill}%
\end{pgfscope}%
\begin{pgfscope}%
\definecolor{textcolor}{rgb}{0.000000,0.000000,0.000000}%
\pgfsetstrokecolor{textcolor}%
\pgfsetfillcolor{textcolor}%
\pgftext[x=4.258735in,y=0.394347in,left,base]{\color{textcolor}\rmfamily\fontsize{10.000000}{12.000000}\selectfont B C-UI-TVBO}%
\end{pgfscope}%
\begin{pgfscope}%
\pgfsetbuttcap%
\pgfsetmiterjoin%
\definecolor{currentfill}{rgb}{1.000000,1.000000,1.000000}%
\pgfsetfillcolor{currentfill}%
\pgfsetlinewidth{1.003750pt}%
\definecolor{currentstroke}{rgb}{1.000000,1.000000,1.000000}%
\pgfsetstrokecolor{currentstroke}%
\pgfsetdash{}{0pt}%
\pgfpathmoveto{\pgfqpoint{2.736198in}{1.968269in}}%
\pgfpathlineto{\pgfqpoint{3.689290in}{1.968269in}}%
\pgfpathquadraticcurveto{\pgfqpoint{3.717068in}{1.968269in}}{\pgfqpoint{3.717068in}{1.996046in}}%
\pgfpathlineto{\pgfqpoint{3.717068in}{2.369503in}}%
\pgfpathquadraticcurveto{\pgfqpoint{3.717068in}{2.397281in}}{\pgfqpoint{3.689290in}{2.397281in}}%
\pgfpathlineto{\pgfqpoint{2.736198in}{2.397281in}}%
\pgfpathquadraticcurveto{\pgfqpoint{2.708420in}{2.397281in}}{\pgfqpoint{2.708420in}{2.369503in}}%
\pgfpathlineto{\pgfqpoint{2.708420in}{1.996046in}}%
\pgfpathquadraticcurveto{\pgfqpoint{2.708420in}{1.968269in}}{\pgfqpoint{2.736198in}{1.968269in}}%
\pgfpathclose%
\pgfusepath{stroke,fill}%
\end{pgfscope}%
\begin{pgfscope}%
\pgfsetbuttcap%
\pgfsetmiterjoin%
\definecolor{currentfill}{rgb}{0.000000,0.000000,0.000000}%
\pgfsetfillcolor{currentfill}%
\pgfsetlinewidth{0.000000pt}%
\definecolor{currentstroke}{rgb}{0.000000,0.000000,0.000000}%
\pgfsetstrokecolor{currentstroke}%
\pgfsetstrokeopacity{0.000000}%
\pgfsetdash{}{0pt}%
\pgfpathmoveto{\pgfqpoint{2.763976in}{2.244503in}}%
\pgfpathlineto{\pgfqpoint{3.041753in}{2.244503in}}%
\pgfpathlineto{\pgfqpoint{3.041753in}{2.341725in}}%
\pgfpathlineto{\pgfqpoint{2.763976in}{2.341725in}}%
\pgfpathclose%
\pgfusepath{fill}%
\end{pgfscope}%
\begin{pgfscope}%
\definecolor{textcolor}{rgb}{0.000000,0.000000,0.000000}%
\pgfsetstrokecolor{textcolor}%
\pgfsetfillcolor{textcolor}%
\pgftext[x=3.152864in,y=2.244503in,left,base]{\color{textcolor}\rmfamily\fontsize{10.000000}{12.000000}\selectfont \(\displaystyle \mu_0=0\)}%
\end{pgfscope}%
\begin{pgfscope}%
\pgfsetbuttcap%
\pgfsetmiterjoin%
\definecolor{currentfill}{rgb}{0.811765,0.819608,0.823529}%
\pgfsetfillcolor{currentfill}%
\pgfsetlinewidth{0.000000pt}%
\definecolor{currentstroke}{rgb}{0.000000,0.000000,0.000000}%
\pgfsetstrokecolor{currentstroke}%
\pgfsetstrokeopacity{0.000000}%
\pgfsetdash{}{0pt}%
\pgfpathmoveto{\pgfqpoint{2.763976in}{2.050830in}}%
\pgfpathlineto{\pgfqpoint{3.041753in}{2.050830in}}%
\pgfpathlineto{\pgfqpoint{3.041753in}{2.148053in}}%
\pgfpathlineto{\pgfqpoint{2.763976in}{2.148053in}}%
\pgfpathclose%
\pgfusepath{fill}%
\end{pgfscope}%
\begin{pgfscope}%
\definecolor{textcolor}{rgb}{0.000000,0.000000,0.000000}%
\pgfsetstrokecolor{textcolor}%
\pgfsetfillcolor{textcolor}%
\pgftext[x=3.152864in,y=2.050830in,left,base]{\color{textcolor}\rmfamily\fontsize{10.000000}{12.000000}\selectfont \(\displaystyle \mu_0=-2\)}%
\end{pgfscope}%
\end{pgfpicture}%
\makeatother%
\endgroup%

    \caption[Results of the two-dimensional within model comparison.]{Results for the two-dimensional within-model comparison. It shows lower regret and a smaller regret variance for \gls{ctvbo} and similar regret for \gls{b2p} and \gls{ui} forgetting. The formatting is as in Figure~\ref{fig:WMC_cumulative_regret_1D}.}
    \label{fig:WMC_cumulative_regret_2D}
\end{figure}

The sensitivity of the forgetting strategies to a shifted prior mean are the same for \gls{ctvbo} as in the one-dimensional case. \gls{ui} forgetting reacts only slightly to the shifted mean compared to \gls{b2p} forgetting. Contrary to expectations, the sensitivity to a shifted mean is not evident for \gls{b2p} forgetting with the standard \gls{tvbo} algorithm. Here, the regret even decreases and thus weakens Hypothesis~\ref{hyp:ui_structural_information}. One possible explanation is that the exploration behavior for $\mu_0=0$ was too limited, resulting in the two variations, TV-GP-UCB and SW TV-GP-UCB, exploiting too much and not capturing the change in the objective function sufficiently. This yields in higher regret. The explorative behavior increases again through the optimistic prior in $\mu_0=-2$, and changes in the objective function can be tracked, thus reducing the regret. This assumption is supported by Figure~\ref{fig:WMC_cum_regret_different_mean}. 
\begin{figure}[h!]
    \centering
    %% Creator: Matplotlib, PGF backend
%%
%% To include the figure in your LaTeX document, write
%%   \input{<filename>.pgf}
%%
%% Make sure the required packages are loaded in your preamble
%%   \usepackage{pgf}
%%
%% Figures using additional raster images can only be included by \input if
%% they are in the same directory as the main LaTeX file. For loading figures
%% from other directories you can use the `import` package
%%   \usepackage{import}
%%
%% and then include the figures with
%%   \import{<path to file>}{<filename>.pgf}
%%
%% Matplotlib used the following preamble
%%   \usepackage{fontspec}
%%
\begingroup%
\makeatletter%
\begin{pgfpicture}%
\pgfpathrectangle{\pgfpointorigin}{\pgfqpoint{5.507126in}{1.531616in}}%
\pgfusepath{use as bounding box, clip}%
\begin{pgfscope}%
\pgfsetbuttcap%
\pgfsetmiterjoin%
\definecolor{currentfill}{rgb}{1.000000,1.000000,1.000000}%
\pgfsetfillcolor{currentfill}%
\pgfsetlinewidth{0.000000pt}%
\definecolor{currentstroke}{rgb}{1.000000,1.000000,1.000000}%
\pgfsetstrokecolor{currentstroke}%
\pgfsetdash{}{0pt}%
\pgfpathmoveto{\pgfqpoint{0.000000in}{0.000000in}}%
\pgfpathlineto{\pgfqpoint{5.507126in}{0.000000in}}%
\pgfpathlineto{\pgfqpoint{5.507126in}{1.531616in}}%
\pgfpathlineto{\pgfqpoint{0.000000in}{1.531616in}}%
\pgfpathclose%
\pgfusepath{fill}%
\end{pgfscope}%
\begin{pgfscope}%
\pgfsetbuttcap%
\pgfsetmiterjoin%
\definecolor{currentfill}{rgb}{1.000000,1.000000,1.000000}%
\pgfsetfillcolor{currentfill}%
\pgfsetlinewidth{0.000000pt}%
\definecolor{currentstroke}{rgb}{0.000000,0.000000,0.000000}%
\pgfsetstrokecolor{currentstroke}%
\pgfsetstrokeopacity{0.000000}%
\pgfsetdash{}{0pt}%
\pgfpathmoveto{\pgfqpoint{0.550713in}{0.076581in}}%
\pgfpathlineto{\pgfqpoint{3.854988in}{0.076581in}}%
\pgfpathlineto{\pgfqpoint{3.854988in}{1.455035in}}%
\pgfpathlineto{\pgfqpoint{0.550713in}{1.455035in}}%
\pgfpathclose%
\pgfusepath{fill}%
\end{pgfscope}%
\begin{pgfscope}%
\pgfpathrectangle{\pgfqpoint{0.550713in}{0.076581in}}{\pgfqpoint{3.304276in}{1.378454in}}%
\pgfusepath{clip}%
\pgfsetbuttcap%
\pgfsetmiterjoin%
\definecolor{currentfill}{rgb}{0.631373,0.062745,0.207843}%
\pgfsetfillcolor{currentfill}%
\pgfsetlinewidth{0.752812pt}%
\definecolor{currentstroke}{rgb}{0.000000,0.000000,0.000000}%
\pgfsetstrokecolor{currentstroke}%
\pgfsetdash{}{0pt}%
\pgfpathmoveto{\pgfqpoint{0.720332in}{0.755406in}}%
\pgfpathlineto{\pgfqpoint{1.152091in}{0.755406in}}%
\pgfpathlineto{\pgfqpoint{1.152091in}{0.924355in}}%
\pgfpathlineto{\pgfqpoint{0.720332in}{0.924355in}}%
\pgfpathlineto{\pgfqpoint{0.720332in}{0.755406in}}%
\pgfpathclose%
\pgfusepath{stroke,fill}%
\end{pgfscope}%
\begin{pgfscope}%
\pgfpathrectangle{\pgfqpoint{0.550713in}{0.076581in}}{\pgfqpoint{3.304276in}{1.378454in}}%
\pgfusepath{clip}%
\pgfsetbuttcap%
\pgfsetmiterjoin%
\definecolor{currentfill}{rgb}{0.898039,0.772549,0.752941}%
\pgfsetfillcolor{currentfill}%
\pgfsetlinewidth{0.752812pt}%
\definecolor{currentstroke}{rgb}{0.000000,0.000000,0.000000}%
\pgfsetstrokecolor{currentstroke}%
\pgfsetdash{}{0pt}%
\pgfpathmoveto{\pgfqpoint{1.160902in}{0.684525in}}%
\pgfpathlineto{\pgfqpoint{1.592661in}{0.684525in}}%
\pgfpathlineto{\pgfqpoint{1.592661in}{0.892273in}}%
\pgfpathlineto{\pgfqpoint{1.160902in}{0.892273in}}%
\pgfpathlineto{\pgfqpoint{1.160902in}{0.684525in}}%
\pgfpathclose%
\pgfusepath{stroke,fill}%
\end{pgfscope}%
\begin{pgfscope}%
\pgfpathrectangle{\pgfqpoint{0.550713in}{0.076581in}}{\pgfqpoint{3.304276in}{1.378454in}}%
\pgfusepath{clip}%
\pgfsetbuttcap%
\pgfsetmiterjoin%
\definecolor{currentfill}{rgb}{0.960784,0.909804,0.898039}%
\pgfsetfillcolor{currentfill}%
\pgfsetlinewidth{0.752812pt}%
\definecolor{currentstroke}{rgb}{0.000000,0.000000,0.000000}%
\pgfsetstrokecolor{currentstroke}%
\pgfsetdash{}{0pt}%
\pgfpathmoveto{\pgfqpoint{1.601472in}{0.892264in}}%
\pgfpathlineto{\pgfqpoint{2.033231in}{0.892264in}}%
\pgfpathlineto{\pgfqpoint{2.033231in}{1.102699in}}%
\pgfpathlineto{\pgfqpoint{1.601472in}{1.102699in}}%
\pgfpathlineto{\pgfqpoint{1.601472in}{0.892264in}}%
\pgfpathclose%
\pgfusepath{stroke,fill}%
\end{pgfscope}%
\begin{pgfscope}%
\pgfpathrectangle{\pgfqpoint{0.550713in}{0.076581in}}{\pgfqpoint{3.304276in}{1.378454in}}%
\pgfusepath{clip}%
\pgfsetbuttcap%
\pgfsetmiterjoin%
\definecolor{currentfill}{rgb}{0.890196,0.000000,0.400000}%
\pgfsetfillcolor{currentfill}%
\pgfsetlinewidth{0.752812pt}%
\definecolor{currentstroke}{rgb}{0.000000,0.000000,0.000000}%
\pgfsetstrokecolor{currentstroke}%
\pgfsetdash{}{0pt}%
\pgfpathmoveto{\pgfqpoint{2.372470in}{0.735829in}}%
\pgfpathlineto{\pgfqpoint{2.804229in}{0.735829in}}%
\pgfpathlineto{\pgfqpoint{2.804229in}{0.974419in}}%
\pgfpathlineto{\pgfqpoint{2.372470in}{0.974419in}}%
\pgfpathlineto{\pgfqpoint{2.372470in}{0.735829in}}%
\pgfpathclose%
\pgfusepath{stroke,fill}%
\end{pgfscope}%
\begin{pgfscope}%
\pgfpathrectangle{\pgfqpoint{0.550713in}{0.076581in}}{\pgfqpoint{3.304276in}{1.378454in}}%
\pgfusepath{clip}%
\pgfsetbuttcap%
\pgfsetmiterjoin%
\definecolor{currentfill}{rgb}{0.976471,0.823529,0.854902}%
\pgfsetfillcolor{currentfill}%
\pgfsetlinewidth{0.752812pt}%
\definecolor{currentstroke}{rgb}{0.000000,0.000000,0.000000}%
\pgfsetstrokecolor{currentstroke}%
\pgfsetdash{}{0pt}%
\pgfpathmoveto{\pgfqpoint{2.813040in}{0.700774in}}%
\pgfpathlineto{\pgfqpoint{3.244799in}{0.700774in}}%
\pgfpathlineto{\pgfqpoint{3.244799in}{0.885872in}}%
\pgfpathlineto{\pgfqpoint{2.813040in}{0.885872in}}%
\pgfpathlineto{\pgfqpoint{2.813040in}{0.700774in}}%
\pgfpathclose%
\pgfusepath{stroke,fill}%
\end{pgfscope}%
\begin{pgfscope}%
\pgfpathrectangle{\pgfqpoint{0.550713in}{0.076581in}}{\pgfqpoint{3.304276in}{1.378454in}}%
\pgfusepath{clip}%
\pgfsetbuttcap%
\pgfsetmiterjoin%
\definecolor{currentfill}{rgb}{0.992157,0.933333,0.941176}%
\pgfsetfillcolor{currentfill}%
\pgfsetlinewidth{0.752812pt}%
\definecolor{currentstroke}{rgb}{0.000000,0.000000,0.000000}%
\pgfsetstrokecolor{currentstroke}%
\pgfsetdash{}{0pt}%
\pgfpathmoveto{\pgfqpoint{3.253610in}{0.941807in}}%
\pgfpathlineto{\pgfqpoint{3.685369in}{0.941807in}}%
\pgfpathlineto{\pgfqpoint{3.685369in}{1.264222in}}%
\pgfpathlineto{\pgfqpoint{3.253610in}{1.264222in}}%
\pgfpathlineto{\pgfqpoint{3.253610in}{0.941807in}}%
\pgfpathclose%
\pgfusepath{stroke,fill}%
\end{pgfscope}%
\begin{pgfscope}%
\pgfpathrectangle{\pgfqpoint{0.550713in}{0.076581in}}{\pgfqpoint{3.304276in}{1.378454in}}%
\pgfusepath{clip}%
\pgfsetbuttcap%
\pgfsetmiterjoin%
\definecolor{currentfill}{rgb}{0.000000,0.000000,0.000000}%
\pgfsetfillcolor{currentfill}%
\pgfsetlinewidth{0.376406pt}%
\definecolor{currentstroke}{rgb}{0.000000,0.000000,0.000000}%
\pgfsetstrokecolor{currentstroke}%
\pgfsetdash{}{0pt}%
\pgfpathmoveto{\pgfqpoint{1.376782in}{0.076581in}}%
\pgfpathlineto{\pgfqpoint{1.376782in}{0.076581in}}%
\pgfpathlineto{\pgfqpoint{1.376782in}{0.076581in}}%
\pgfpathlineto{\pgfqpoint{1.376782in}{0.076581in}}%
\pgfpathclose%
\pgfusepath{stroke,fill}%
\end{pgfscope}%
\begin{pgfscope}%
\pgfpathrectangle{\pgfqpoint{0.550713in}{0.076581in}}{\pgfqpoint{3.304276in}{1.378454in}}%
\pgfusepath{clip}%
\pgfsetbuttcap%
\pgfsetmiterjoin%
\definecolor{currentfill}{rgb}{0.813235,0.819118,0.822059}%
\pgfsetfillcolor{currentfill}%
\pgfsetlinewidth{0.376406pt}%
\definecolor{currentstroke}{rgb}{0.000000,0.000000,0.000000}%
\pgfsetstrokecolor{currentstroke}%
\pgfsetdash{}{0pt}%
\pgfpathmoveto{\pgfqpoint{1.376782in}{0.076581in}}%
\pgfpathlineto{\pgfqpoint{1.376782in}{0.076581in}}%
\pgfpathlineto{\pgfqpoint{1.376782in}{0.076581in}}%
\pgfpathlineto{\pgfqpoint{1.376782in}{0.076581in}}%
\pgfpathclose%
\pgfusepath{stroke,fill}%
\end{pgfscope}%
\begin{pgfscope}%
\pgfpathrectangle{\pgfqpoint{0.550713in}{0.076581in}}{\pgfqpoint{3.304276in}{1.378454in}}%
\pgfusepath{clip}%
\pgfsetbuttcap%
\pgfsetmiterjoin%
\definecolor{currentfill}{rgb}{0.925980,0.928922,0.928922}%
\pgfsetfillcolor{currentfill}%
\pgfsetlinewidth{0.376406pt}%
\definecolor{currentstroke}{rgb}{0.000000,0.000000,0.000000}%
\pgfsetstrokecolor{currentstroke}%
\pgfsetdash{}{0pt}%
\pgfpathmoveto{\pgfqpoint{1.376782in}{0.076581in}}%
\pgfpathlineto{\pgfqpoint{1.376782in}{0.076581in}}%
\pgfpathlineto{\pgfqpoint{1.376782in}{0.076581in}}%
\pgfpathlineto{\pgfqpoint{1.376782in}{0.076581in}}%
\pgfpathclose%
\pgfusepath{stroke,fill}%
\end{pgfscope}%
\begin{pgfscope}%
\pgfsetbuttcap%
\pgfsetroundjoin%
\definecolor{currentfill}{rgb}{0.000000,0.000000,0.000000}%
\pgfsetfillcolor{currentfill}%
\pgfsetlinewidth{0.803000pt}%
\definecolor{currentstroke}{rgb}{0.000000,0.000000,0.000000}%
\pgfsetstrokecolor{currentstroke}%
\pgfsetdash{}{0pt}%
\pgfsys@defobject{currentmarker}{\pgfqpoint{-0.048611in}{0.000000in}}{\pgfqpoint{-0.000000in}{0.000000in}}{%
\pgfpathmoveto{\pgfqpoint{-0.000000in}{0.000000in}}%
\pgfpathlineto{\pgfqpoint{-0.048611in}{0.000000in}}%
\pgfusepath{stroke,fill}%
}%
\begin{pgfscope}%
\pgfsys@transformshift{0.550713in}{0.076581in}%
\pgfsys@useobject{currentmarker}{}%
\end{pgfscope}%
\end{pgfscope}%
\begin{pgfscope}%
\definecolor{textcolor}{rgb}{0.000000,0.000000,0.000000}%
\pgfsetstrokecolor{textcolor}%
\pgfsetfillcolor{textcolor}%
\pgftext[x=0.384046in, y=0.028386in, left, base]{\color{textcolor}\rmfamily\fontsize{10.000000}{12.000000}\selectfont \(\displaystyle {0}\)}%
\end{pgfscope}%
\begin{pgfscope}%
\pgfsetbuttcap%
\pgfsetroundjoin%
\definecolor{currentfill}{rgb}{0.000000,0.000000,0.000000}%
\pgfsetfillcolor{currentfill}%
\pgfsetlinewidth{0.803000pt}%
\definecolor{currentstroke}{rgb}{0.000000,0.000000,0.000000}%
\pgfsetstrokecolor{currentstroke}%
\pgfsetdash{}{0pt}%
\pgfsys@defobject{currentmarker}{\pgfqpoint{-0.048611in}{0.000000in}}{\pgfqpoint{-0.000000in}{0.000000in}}{%
\pgfpathmoveto{\pgfqpoint{-0.000000in}{0.000000in}}%
\pgfpathlineto{\pgfqpoint{-0.048611in}{0.000000in}}%
\pgfusepath{stroke,fill}%
}%
\begin{pgfscope}%
\pgfsys@transformshift{0.550713in}{0.650937in}%
\pgfsys@useobject{currentmarker}{}%
\end{pgfscope}%
\end{pgfscope}%
\begin{pgfscope}%
\definecolor{textcolor}{rgb}{0.000000,0.000000,0.000000}%
\pgfsetstrokecolor{textcolor}%
\pgfsetfillcolor{textcolor}%
\pgftext[x=0.314601in, y=0.602742in, left, base]{\color{textcolor}\rmfamily\fontsize{10.000000}{12.000000}\selectfont \(\displaystyle {50}\)}%
\end{pgfscope}%
\begin{pgfscope}%
\pgfsetbuttcap%
\pgfsetroundjoin%
\definecolor{currentfill}{rgb}{0.000000,0.000000,0.000000}%
\pgfsetfillcolor{currentfill}%
\pgfsetlinewidth{0.803000pt}%
\definecolor{currentstroke}{rgb}{0.000000,0.000000,0.000000}%
\pgfsetstrokecolor{currentstroke}%
\pgfsetdash{}{0pt}%
\pgfsys@defobject{currentmarker}{\pgfqpoint{-0.048611in}{0.000000in}}{\pgfqpoint{-0.000000in}{0.000000in}}{%
\pgfpathmoveto{\pgfqpoint{-0.000000in}{0.000000in}}%
\pgfpathlineto{\pgfqpoint{-0.048611in}{0.000000in}}%
\pgfusepath{stroke,fill}%
}%
\begin{pgfscope}%
\pgfsys@transformshift{0.550713in}{1.225293in}%
\pgfsys@useobject{currentmarker}{}%
\end{pgfscope}%
\end{pgfscope}%
\begin{pgfscope}%
\definecolor{textcolor}{rgb}{0.000000,0.000000,0.000000}%
\pgfsetstrokecolor{textcolor}%
\pgfsetfillcolor{textcolor}%
\pgftext[x=0.245156in, y=1.177098in, left, base]{\color{textcolor}\rmfamily\fontsize{10.000000}{12.000000}\selectfont \(\displaystyle {100}\)}%
\end{pgfscope}%
\begin{pgfscope}%
\definecolor{textcolor}{rgb}{0.000000,0.000000,0.000000}%
\pgfsetstrokecolor{textcolor}%
\pgfsetfillcolor{textcolor}%
\pgftext[x=0.189601in,y=0.765808in,,bottom,rotate=90.000000]{\color{textcolor}\rmfamily\fontsize{10.000000}{12.000000}\selectfont \(\displaystyle R_T\)}%
\end{pgfscope}%
\begin{pgfscope}%
\pgfpathrectangle{\pgfqpoint{0.550713in}{0.076581in}}{\pgfqpoint{3.304276in}{1.378454in}}%
\pgfusepath{clip}%
\pgfsetbuttcap%
\pgfsetroundjoin%
\pgfsetlinewidth{0.501875pt}%
\definecolor{currentstroke}{rgb}{0.392157,0.396078,0.403922}%
\pgfsetstrokecolor{currentstroke}%
\pgfsetdash{}{0pt}%
\pgfpathmoveto{\pgfqpoint{2.202850in}{0.076581in}}%
\pgfpathlineto{\pgfqpoint{2.202850in}{1.455035in}}%
\pgfusepath{stroke}%
\end{pgfscope}%
\begin{pgfscope}%
\pgfpathrectangle{\pgfqpoint{0.550713in}{0.076581in}}{\pgfqpoint{3.304276in}{1.378454in}}%
\pgfusepath{clip}%
\pgfsetbuttcap%
\pgfsetroundjoin%
\pgfsetlinewidth{0.853187pt}%
\definecolor{currentstroke}{rgb}{0.392157,0.396078,0.403922}%
\pgfsetstrokecolor{currentstroke}%
\pgfsetdash{{3.145000pt}{1.360000pt}}{0.000000pt}%
\pgfusepath{stroke}%
\end{pgfscope}%
\begin{pgfscope}%
\pgfpathrectangle{\pgfqpoint{0.550713in}{0.076581in}}{\pgfqpoint{3.304276in}{1.378454in}}%
\pgfusepath{clip}%
\pgfsetrectcap%
\pgfsetroundjoin%
\pgfsetlinewidth{0.752812pt}%
\definecolor{currentstroke}{rgb}{0.000000,0.000000,0.000000}%
\pgfsetstrokecolor{currentstroke}%
\pgfsetdash{}{0pt}%
\pgfpathmoveto{\pgfqpoint{0.936211in}{0.755406in}}%
\pgfpathlineto{\pgfqpoint{0.936211in}{0.505622in}}%
\pgfusepath{stroke}%
\end{pgfscope}%
\begin{pgfscope}%
\pgfpathrectangle{\pgfqpoint{0.550713in}{0.076581in}}{\pgfqpoint{3.304276in}{1.378454in}}%
\pgfusepath{clip}%
\pgfsetrectcap%
\pgfsetroundjoin%
\pgfsetlinewidth{0.752812pt}%
\definecolor{currentstroke}{rgb}{0.000000,0.000000,0.000000}%
\pgfsetstrokecolor{currentstroke}%
\pgfsetdash{}{0pt}%
\pgfpathmoveto{\pgfqpoint{0.936211in}{0.924355in}}%
\pgfpathlineto{\pgfqpoint{0.936211in}{1.102815in}}%
\pgfusepath{stroke}%
\end{pgfscope}%
\begin{pgfscope}%
\pgfpathrectangle{\pgfqpoint{0.550713in}{0.076581in}}{\pgfqpoint{3.304276in}{1.378454in}}%
\pgfusepath{clip}%
\pgfsetrectcap%
\pgfsetroundjoin%
\pgfsetlinewidth{0.752812pt}%
\definecolor{currentstroke}{rgb}{0.000000,0.000000,0.000000}%
\pgfsetstrokecolor{currentstroke}%
\pgfsetdash{}{0pt}%
\pgfpathmoveto{\pgfqpoint{0.828272in}{0.505622in}}%
\pgfpathlineto{\pgfqpoint{1.044151in}{0.505622in}}%
\pgfusepath{stroke}%
\end{pgfscope}%
\begin{pgfscope}%
\pgfpathrectangle{\pgfqpoint{0.550713in}{0.076581in}}{\pgfqpoint{3.304276in}{1.378454in}}%
\pgfusepath{clip}%
\pgfsetrectcap%
\pgfsetroundjoin%
\pgfsetlinewidth{0.752812pt}%
\definecolor{currentstroke}{rgb}{0.000000,0.000000,0.000000}%
\pgfsetstrokecolor{currentstroke}%
\pgfsetdash{}{0pt}%
\pgfpathmoveto{\pgfqpoint{0.828272in}{1.102815in}}%
\pgfpathlineto{\pgfqpoint{1.044151in}{1.102815in}}%
\pgfusepath{stroke}%
\end{pgfscope}%
\begin{pgfscope}%
\pgfpathrectangle{\pgfqpoint{0.550713in}{0.076581in}}{\pgfqpoint{3.304276in}{1.378454in}}%
\pgfusepath{clip}%
\pgfsetbuttcap%
\pgfsetmiterjoin%
\definecolor{currentfill}{rgb}{0.000000,0.000000,0.000000}%
\pgfsetfillcolor{currentfill}%
\pgfsetlinewidth{1.003750pt}%
\definecolor{currentstroke}{rgb}{0.000000,0.000000,0.000000}%
\pgfsetstrokecolor{currentstroke}%
\pgfsetdash{}{0pt}%
\pgfsys@defobject{currentmarker}{\pgfqpoint{-0.011785in}{-0.019642in}}{\pgfqpoint{0.011785in}{0.019642in}}{%
\pgfpathmoveto{\pgfqpoint{-0.000000in}{-0.019642in}}%
\pgfpathlineto{\pgfqpoint{0.011785in}{0.000000in}}%
\pgfpathlineto{\pgfqpoint{0.000000in}{0.019642in}}%
\pgfpathlineto{\pgfqpoint{-0.011785in}{0.000000in}}%
\pgfpathclose%
\pgfusepath{stroke,fill}%
}%
\begin{pgfscope}%
\pgfsys@transformshift{0.936211in}{0.480233in}%
\pgfsys@useobject{currentmarker}{}%
\end{pgfscope}%
\begin{pgfscope}%
\pgfsys@transformshift{0.936211in}{0.501728in}%
\pgfsys@useobject{currentmarker}{}%
\end{pgfscope}%
\begin{pgfscope}%
\pgfsys@transformshift{0.936211in}{0.480029in}%
\pgfsys@useobject{currentmarker}{}%
\end{pgfscope}%
\begin{pgfscope}%
\pgfsys@transformshift{0.936211in}{1.325213in}%
\pgfsys@useobject{currentmarker}{}%
\end{pgfscope}%
\begin{pgfscope}%
\pgfsys@transformshift{0.936211in}{1.242092in}%
\pgfsys@useobject{currentmarker}{}%
\end{pgfscope}%
\begin{pgfscope}%
\pgfsys@transformshift{0.936211in}{1.307982in}%
\pgfsys@useobject{currentmarker}{}%
\end{pgfscope}%
\end{pgfscope}%
\begin{pgfscope}%
\pgfpathrectangle{\pgfqpoint{0.550713in}{0.076581in}}{\pgfqpoint{3.304276in}{1.378454in}}%
\pgfusepath{clip}%
\pgfsetrectcap%
\pgfsetroundjoin%
\pgfsetlinewidth{0.752812pt}%
\definecolor{currentstroke}{rgb}{0.000000,0.000000,0.000000}%
\pgfsetstrokecolor{currentstroke}%
\pgfsetdash{}{0pt}%
\pgfpathmoveto{\pgfqpoint{1.376782in}{0.684525in}}%
\pgfpathlineto{\pgfqpoint{1.376782in}{0.501916in}}%
\pgfusepath{stroke}%
\end{pgfscope}%
\begin{pgfscope}%
\pgfpathrectangle{\pgfqpoint{0.550713in}{0.076581in}}{\pgfqpoint{3.304276in}{1.378454in}}%
\pgfusepath{clip}%
\pgfsetrectcap%
\pgfsetroundjoin%
\pgfsetlinewidth{0.752812pt}%
\definecolor{currentstroke}{rgb}{0.000000,0.000000,0.000000}%
\pgfsetstrokecolor{currentstroke}%
\pgfsetdash{}{0pt}%
\pgfpathmoveto{\pgfqpoint{1.376782in}{0.892273in}}%
\pgfpathlineto{\pgfqpoint{1.376782in}{1.109421in}}%
\pgfusepath{stroke}%
\end{pgfscope}%
\begin{pgfscope}%
\pgfpathrectangle{\pgfqpoint{0.550713in}{0.076581in}}{\pgfqpoint{3.304276in}{1.378454in}}%
\pgfusepath{clip}%
\pgfsetrectcap%
\pgfsetroundjoin%
\pgfsetlinewidth{0.752812pt}%
\definecolor{currentstroke}{rgb}{0.000000,0.000000,0.000000}%
\pgfsetstrokecolor{currentstroke}%
\pgfsetdash{}{0pt}%
\pgfpathmoveto{\pgfqpoint{1.268842in}{0.501916in}}%
\pgfpathlineto{\pgfqpoint{1.484721in}{0.501916in}}%
\pgfusepath{stroke}%
\end{pgfscope}%
\begin{pgfscope}%
\pgfpathrectangle{\pgfqpoint{0.550713in}{0.076581in}}{\pgfqpoint{3.304276in}{1.378454in}}%
\pgfusepath{clip}%
\pgfsetrectcap%
\pgfsetroundjoin%
\pgfsetlinewidth{0.752812pt}%
\definecolor{currentstroke}{rgb}{0.000000,0.000000,0.000000}%
\pgfsetstrokecolor{currentstroke}%
\pgfsetdash{}{0pt}%
\pgfpathmoveto{\pgfqpoint{1.268842in}{1.109421in}}%
\pgfpathlineto{\pgfqpoint{1.484721in}{1.109421in}}%
\pgfusepath{stroke}%
\end{pgfscope}%
\begin{pgfscope}%
\pgfpathrectangle{\pgfqpoint{0.550713in}{0.076581in}}{\pgfqpoint{3.304276in}{1.378454in}}%
\pgfusepath{clip}%
\pgfsetrectcap%
\pgfsetroundjoin%
\pgfsetlinewidth{0.752812pt}%
\definecolor{currentstroke}{rgb}{0.000000,0.000000,0.000000}%
\pgfsetstrokecolor{currentstroke}%
\pgfsetdash{}{0pt}%
\pgfpathmoveto{\pgfqpoint{1.817352in}{0.892264in}}%
\pgfpathlineto{\pgfqpoint{1.817352in}{0.664315in}}%
\pgfusepath{stroke}%
\end{pgfscope}%
\begin{pgfscope}%
\pgfpathrectangle{\pgfqpoint{0.550713in}{0.076581in}}{\pgfqpoint{3.304276in}{1.378454in}}%
\pgfusepath{clip}%
\pgfsetrectcap%
\pgfsetroundjoin%
\pgfsetlinewidth{0.752812pt}%
\definecolor{currentstroke}{rgb}{0.000000,0.000000,0.000000}%
\pgfsetstrokecolor{currentstroke}%
\pgfsetdash{}{0pt}%
\pgfpathmoveto{\pgfqpoint{1.817352in}{1.102699in}}%
\pgfpathlineto{\pgfqpoint{1.817352in}{1.111462in}}%
\pgfusepath{stroke}%
\end{pgfscope}%
\begin{pgfscope}%
\pgfpathrectangle{\pgfqpoint{0.550713in}{0.076581in}}{\pgfqpoint{3.304276in}{1.378454in}}%
\pgfusepath{clip}%
\pgfsetrectcap%
\pgfsetroundjoin%
\pgfsetlinewidth{0.752812pt}%
\definecolor{currentstroke}{rgb}{0.000000,0.000000,0.000000}%
\pgfsetstrokecolor{currentstroke}%
\pgfsetdash{}{0pt}%
\pgfpathmoveto{\pgfqpoint{1.709412in}{0.664315in}}%
\pgfpathlineto{\pgfqpoint{1.925291in}{0.664315in}}%
\pgfusepath{stroke}%
\end{pgfscope}%
\begin{pgfscope}%
\pgfpathrectangle{\pgfqpoint{0.550713in}{0.076581in}}{\pgfqpoint{3.304276in}{1.378454in}}%
\pgfusepath{clip}%
\pgfsetrectcap%
\pgfsetroundjoin%
\pgfsetlinewidth{0.752812pt}%
\definecolor{currentstroke}{rgb}{0.000000,0.000000,0.000000}%
\pgfsetstrokecolor{currentstroke}%
\pgfsetdash{}{0pt}%
\pgfpathmoveto{\pgfqpoint{1.709412in}{1.111462in}}%
\pgfpathlineto{\pgfqpoint{1.925291in}{1.111462in}}%
\pgfusepath{stroke}%
\end{pgfscope}%
\begin{pgfscope}%
\pgfpathrectangle{\pgfqpoint{0.550713in}{0.076581in}}{\pgfqpoint{3.304276in}{1.378454in}}%
\pgfusepath{clip}%
\pgfsetbuttcap%
\pgfsetmiterjoin%
\definecolor{currentfill}{rgb}{0.000000,0.000000,0.000000}%
\pgfsetfillcolor{currentfill}%
\pgfsetlinewidth{1.003750pt}%
\definecolor{currentstroke}{rgb}{0.000000,0.000000,0.000000}%
\pgfsetstrokecolor{currentstroke}%
\pgfsetdash{}{0pt}%
\pgfsys@defobject{currentmarker}{\pgfqpoint{-0.011785in}{-0.019642in}}{\pgfqpoint{0.011785in}{0.019642in}}{%
\pgfpathmoveto{\pgfqpoint{-0.000000in}{-0.019642in}}%
\pgfpathlineto{\pgfqpoint{0.011785in}{0.000000in}}%
\pgfpathlineto{\pgfqpoint{0.000000in}{0.019642in}}%
\pgfpathlineto{\pgfqpoint{-0.011785in}{0.000000in}}%
\pgfpathclose%
\pgfusepath{stroke,fill}%
}%
\begin{pgfscope}%
\pgfsys@transformshift{1.817352in}{1.699251in}%
\pgfsys@useobject{currentmarker}{}%
\end{pgfscope}%
\begin{pgfscope}%
\pgfsys@transformshift{1.817352in}{1.475694in}%
\pgfsys@useobject{currentmarker}{}%
\end{pgfscope}%
\begin{pgfscope}%
\pgfsys@transformshift{1.817352in}{1.609162in}%
\pgfsys@useobject{currentmarker}{}%
\end{pgfscope}%
\begin{pgfscope}%
\pgfsys@transformshift{1.817352in}{1.689597in}%
\pgfsys@useobject{currentmarker}{}%
\end{pgfscope}%
\begin{pgfscope}%
\pgfsys@transformshift{1.817352in}{1.644295in}%
\pgfsys@useobject{currentmarker}{}%
\end{pgfscope}%
\end{pgfscope}%
\begin{pgfscope}%
\pgfpathrectangle{\pgfqpoint{0.550713in}{0.076581in}}{\pgfqpoint{3.304276in}{1.378454in}}%
\pgfusepath{clip}%
\pgfsetrectcap%
\pgfsetroundjoin%
\pgfsetlinewidth{0.752812pt}%
\definecolor{currentstroke}{rgb}{0.000000,0.000000,0.000000}%
\pgfsetstrokecolor{currentstroke}%
\pgfsetdash{}{0pt}%
\pgfpathmoveto{\pgfqpoint{2.588349in}{0.735829in}}%
\pgfpathlineto{\pgfqpoint{2.588349in}{0.451462in}}%
\pgfusepath{stroke}%
\end{pgfscope}%
\begin{pgfscope}%
\pgfpathrectangle{\pgfqpoint{0.550713in}{0.076581in}}{\pgfqpoint{3.304276in}{1.378454in}}%
\pgfusepath{clip}%
\pgfsetrectcap%
\pgfsetroundjoin%
\pgfsetlinewidth{0.752812pt}%
\definecolor{currentstroke}{rgb}{0.000000,0.000000,0.000000}%
\pgfsetstrokecolor{currentstroke}%
\pgfsetdash{}{0pt}%
\pgfpathmoveto{\pgfqpoint{2.588349in}{0.974419in}}%
\pgfpathlineto{\pgfqpoint{2.588349in}{1.234804in}}%
\pgfusepath{stroke}%
\end{pgfscope}%
\begin{pgfscope}%
\pgfpathrectangle{\pgfqpoint{0.550713in}{0.076581in}}{\pgfqpoint{3.304276in}{1.378454in}}%
\pgfusepath{clip}%
\pgfsetrectcap%
\pgfsetroundjoin%
\pgfsetlinewidth{0.752812pt}%
\definecolor{currentstroke}{rgb}{0.000000,0.000000,0.000000}%
\pgfsetstrokecolor{currentstroke}%
\pgfsetdash{}{0pt}%
\pgfpathmoveto{\pgfqpoint{2.480410in}{0.451462in}}%
\pgfpathlineto{\pgfqpoint{2.696289in}{0.451462in}}%
\pgfusepath{stroke}%
\end{pgfscope}%
\begin{pgfscope}%
\pgfpathrectangle{\pgfqpoint{0.550713in}{0.076581in}}{\pgfqpoint{3.304276in}{1.378454in}}%
\pgfusepath{clip}%
\pgfsetrectcap%
\pgfsetroundjoin%
\pgfsetlinewidth{0.752812pt}%
\definecolor{currentstroke}{rgb}{0.000000,0.000000,0.000000}%
\pgfsetstrokecolor{currentstroke}%
\pgfsetdash{}{0pt}%
\pgfpathmoveto{\pgfqpoint{2.480410in}{1.234804in}}%
\pgfpathlineto{\pgfqpoint{2.696289in}{1.234804in}}%
\pgfusepath{stroke}%
\end{pgfscope}%
\begin{pgfscope}%
\pgfpathrectangle{\pgfqpoint{0.550713in}{0.076581in}}{\pgfqpoint{3.304276in}{1.378454in}}%
\pgfusepath{clip}%
\pgfsetbuttcap%
\pgfsetmiterjoin%
\definecolor{currentfill}{rgb}{0.000000,0.000000,0.000000}%
\pgfsetfillcolor{currentfill}%
\pgfsetlinewidth{1.003750pt}%
\definecolor{currentstroke}{rgb}{0.000000,0.000000,0.000000}%
\pgfsetstrokecolor{currentstroke}%
\pgfsetdash{}{0pt}%
\pgfsys@defobject{currentmarker}{\pgfqpoint{-0.011785in}{-0.019642in}}{\pgfqpoint{0.011785in}{0.019642in}}{%
\pgfpathmoveto{\pgfqpoint{-0.000000in}{-0.019642in}}%
\pgfpathlineto{\pgfqpoint{0.011785in}{0.000000in}}%
\pgfpathlineto{\pgfqpoint{0.000000in}{0.019642in}}%
\pgfpathlineto{\pgfqpoint{-0.011785in}{0.000000in}}%
\pgfpathclose%
\pgfusepath{stroke,fill}%
}%
\begin{pgfscope}%
\pgfsys@transformshift{2.588349in}{1.339251in}%
\pgfsys@useobject{currentmarker}{}%
\end{pgfscope}%
\end{pgfscope}%
\begin{pgfscope}%
\pgfpathrectangle{\pgfqpoint{0.550713in}{0.076581in}}{\pgfqpoint{3.304276in}{1.378454in}}%
\pgfusepath{clip}%
\pgfsetrectcap%
\pgfsetroundjoin%
\pgfsetlinewidth{0.752812pt}%
\definecolor{currentstroke}{rgb}{0.000000,0.000000,0.000000}%
\pgfsetstrokecolor{currentstroke}%
\pgfsetdash{}{0pt}%
\pgfpathmoveto{\pgfqpoint{3.028919in}{0.700774in}}%
\pgfpathlineto{\pgfqpoint{3.028919in}{0.472358in}}%
\pgfusepath{stroke}%
\end{pgfscope}%
\begin{pgfscope}%
\pgfpathrectangle{\pgfqpoint{0.550713in}{0.076581in}}{\pgfqpoint{3.304276in}{1.378454in}}%
\pgfusepath{clip}%
\pgfsetrectcap%
\pgfsetroundjoin%
\pgfsetlinewidth{0.752812pt}%
\definecolor{currentstroke}{rgb}{0.000000,0.000000,0.000000}%
\pgfsetstrokecolor{currentstroke}%
\pgfsetdash{}{0pt}%
\pgfpathmoveto{\pgfqpoint{3.028919in}{0.885872in}}%
\pgfpathlineto{\pgfqpoint{3.028919in}{0.992354in}}%
\pgfusepath{stroke}%
\end{pgfscope}%
\begin{pgfscope}%
\pgfpathrectangle{\pgfqpoint{0.550713in}{0.076581in}}{\pgfqpoint{3.304276in}{1.378454in}}%
\pgfusepath{clip}%
\pgfsetrectcap%
\pgfsetroundjoin%
\pgfsetlinewidth{0.752812pt}%
\definecolor{currentstroke}{rgb}{0.000000,0.000000,0.000000}%
\pgfsetstrokecolor{currentstroke}%
\pgfsetdash{}{0pt}%
\pgfpathmoveto{\pgfqpoint{2.920980in}{0.472358in}}%
\pgfpathlineto{\pgfqpoint{3.136859in}{0.472358in}}%
\pgfusepath{stroke}%
\end{pgfscope}%
\begin{pgfscope}%
\pgfpathrectangle{\pgfqpoint{0.550713in}{0.076581in}}{\pgfqpoint{3.304276in}{1.378454in}}%
\pgfusepath{clip}%
\pgfsetrectcap%
\pgfsetroundjoin%
\pgfsetlinewidth{0.752812pt}%
\definecolor{currentstroke}{rgb}{0.000000,0.000000,0.000000}%
\pgfsetstrokecolor{currentstroke}%
\pgfsetdash{}{0pt}%
\pgfpathmoveto{\pgfqpoint{2.920980in}{0.992354in}}%
\pgfpathlineto{\pgfqpoint{3.136859in}{0.992354in}}%
\pgfusepath{stroke}%
\end{pgfscope}%
\begin{pgfscope}%
\pgfpathrectangle{\pgfqpoint{0.550713in}{0.076581in}}{\pgfqpoint{3.304276in}{1.378454in}}%
\pgfusepath{clip}%
\pgfsetrectcap%
\pgfsetroundjoin%
\pgfsetlinewidth{0.752812pt}%
\definecolor{currentstroke}{rgb}{0.000000,0.000000,0.000000}%
\pgfsetstrokecolor{currentstroke}%
\pgfsetdash{}{0pt}%
\pgfpathmoveto{\pgfqpoint{3.469489in}{0.941807in}}%
\pgfpathlineto{\pgfqpoint{3.469489in}{0.718473in}}%
\pgfusepath{stroke}%
\end{pgfscope}%
\begin{pgfscope}%
\pgfpathrectangle{\pgfqpoint{0.550713in}{0.076581in}}{\pgfqpoint{3.304276in}{1.378454in}}%
\pgfusepath{clip}%
\pgfsetrectcap%
\pgfsetroundjoin%
\pgfsetlinewidth{0.752812pt}%
\definecolor{currentstroke}{rgb}{0.000000,0.000000,0.000000}%
\pgfsetstrokecolor{currentstroke}%
\pgfsetdash{}{0pt}%
\pgfpathmoveto{\pgfqpoint{3.469489in}{1.264222in}}%
\pgfpathlineto{\pgfqpoint{3.469489in}{1.272579in}}%
\pgfusepath{stroke}%
\end{pgfscope}%
\begin{pgfscope}%
\pgfpathrectangle{\pgfqpoint{0.550713in}{0.076581in}}{\pgfqpoint{3.304276in}{1.378454in}}%
\pgfusepath{clip}%
\pgfsetrectcap%
\pgfsetroundjoin%
\pgfsetlinewidth{0.752812pt}%
\definecolor{currentstroke}{rgb}{0.000000,0.000000,0.000000}%
\pgfsetstrokecolor{currentstroke}%
\pgfsetdash{}{0pt}%
\pgfpathmoveto{\pgfqpoint{3.361550in}{0.718473in}}%
\pgfpathlineto{\pgfqpoint{3.577429in}{0.718473in}}%
\pgfusepath{stroke}%
\end{pgfscope}%
\begin{pgfscope}%
\pgfpathrectangle{\pgfqpoint{0.550713in}{0.076581in}}{\pgfqpoint{3.304276in}{1.378454in}}%
\pgfusepath{clip}%
\pgfsetrectcap%
\pgfsetroundjoin%
\pgfsetlinewidth{0.752812pt}%
\definecolor{currentstroke}{rgb}{0.000000,0.000000,0.000000}%
\pgfsetstrokecolor{currentstroke}%
\pgfsetdash{}{0pt}%
\pgfpathmoveto{\pgfqpoint{3.361550in}{1.272579in}}%
\pgfpathlineto{\pgfqpoint{3.577429in}{1.272579in}}%
\pgfusepath{stroke}%
\end{pgfscope}%
\begin{pgfscope}%
\pgfpathrectangle{\pgfqpoint{0.550713in}{0.076581in}}{\pgfqpoint{3.304276in}{1.378454in}}%
\pgfusepath{clip}%
\pgfsetbuttcap%
\pgfsetmiterjoin%
\definecolor{currentfill}{rgb}{0.000000,0.000000,0.000000}%
\pgfsetfillcolor{currentfill}%
\pgfsetlinewidth{1.003750pt}%
\definecolor{currentstroke}{rgb}{0.000000,0.000000,0.000000}%
\pgfsetstrokecolor{currentstroke}%
\pgfsetdash{}{0pt}%
\pgfsys@defobject{currentmarker}{\pgfqpoint{-0.011785in}{-0.019642in}}{\pgfqpoint{0.011785in}{0.019642in}}{%
\pgfpathmoveto{\pgfqpoint{-0.000000in}{-0.019642in}}%
\pgfpathlineto{\pgfqpoint{0.011785in}{0.000000in}}%
\pgfpathlineto{\pgfqpoint{0.000000in}{0.019642in}}%
\pgfpathlineto{\pgfqpoint{-0.011785in}{0.000000in}}%
\pgfpathclose%
\pgfusepath{stroke,fill}%
}%
\begin{pgfscope}%
\pgfsys@transformshift{3.469489in}{2.026299in}%
\pgfsys@useobject{currentmarker}{}%
\end{pgfscope}%
\begin{pgfscope}%
\pgfsys@transformshift{3.469489in}{2.027381in}%
\pgfsys@useobject{currentmarker}{}%
\end{pgfscope}%
\begin{pgfscope}%
\pgfsys@transformshift{3.469489in}{1.854636in}%
\pgfsys@useobject{currentmarker}{}%
\end{pgfscope}%
\begin{pgfscope}%
\pgfsys@transformshift{3.469489in}{1.834287in}%
\pgfsys@useobject{currentmarker}{}%
\end{pgfscope}%
\begin{pgfscope}%
\pgfsys@transformshift{3.469489in}{1.856289in}%
\pgfsys@useobject{currentmarker}{}%
\end{pgfscope}%
\end{pgfscope}%
\begin{pgfscope}%
\pgfpathrectangle{\pgfqpoint{0.550713in}{0.076581in}}{\pgfqpoint{3.304276in}{1.378454in}}%
\pgfusepath{clip}%
\pgfsetrectcap%
\pgfsetroundjoin%
\pgfsetlinewidth{0.752812pt}%
\definecolor{currentstroke}{rgb}{0.000000,0.000000,0.000000}%
\pgfsetstrokecolor{currentstroke}%
\pgfsetdash{}{0pt}%
\pgfpathmoveto{\pgfqpoint{0.720332in}{0.853610in}}%
\pgfpathlineto{\pgfqpoint{1.152091in}{0.853610in}}%
\pgfusepath{stroke}%
\end{pgfscope}%
\begin{pgfscope}%
\pgfpathrectangle{\pgfqpoint{0.550713in}{0.076581in}}{\pgfqpoint{3.304276in}{1.378454in}}%
\pgfusepath{clip}%
\pgfsetbuttcap%
\pgfsetroundjoin%
\definecolor{currentfill}{rgb}{1.000000,1.000000,1.000000}%
\pgfsetfillcolor{currentfill}%
\pgfsetlinewidth{1.003750pt}%
\definecolor{currentstroke}{rgb}{0.000000,0.000000,0.000000}%
\pgfsetstrokecolor{currentstroke}%
\pgfsetdash{}{0pt}%
\pgfsys@defobject{currentmarker}{\pgfqpoint{-0.027778in}{-0.027778in}}{\pgfqpoint{0.027778in}{0.027778in}}{%
\pgfpathmoveto{\pgfqpoint{0.000000in}{-0.027778in}}%
\pgfpathcurveto{\pgfqpoint{0.007367in}{-0.027778in}}{\pgfqpoint{0.014433in}{-0.024851in}}{\pgfqpoint{0.019642in}{-0.019642in}}%
\pgfpathcurveto{\pgfqpoint{0.024851in}{-0.014433in}}{\pgfqpoint{0.027778in}{-0.007367in}}{\pgfqpoint{0.027778in}{0.000000in}}%
\pgfpathcurveto{\pgfqpoint{0.027778in}{0.007367in}}{\pgfqpoint{0.024851in}{0.014433in}}{\pgfqpoint{0.019642in}{0.019642in}}%
\pgfpathcurveto{\pgfqpoint{0.014433in}{0.024851in}}{\pgfqpoint{0.007367in}{0.027778in}}{\pgfqpoint{0.000000in}{0.027778in}}%
\pgfpathcurveto{\pgfqpoint{-0.007367in}{0.027778in}}{\pgfqpoint{-0.014433in}{0.024851in}}{\pgfqpoint{-0.019642in}{0.019642in}}%
\pgfpathcurveto{\pgfqpoint{-0.024851in}{0.014433in}}{\pgfqpoint{-0.027778in}{0.007367in}}{\pgfqpoint{-0.027778in}{0.000000in}}%
\pgfpathcurveto{\pgfqpoint{-0.027778in}{-0.007367in}}{\pgfqpoint{-0.024851in}{-0.014433in}}{\pgfqpoint{-0.019642in}{-0.019642in}}%
\pgfpathcurveto{\pgfqpoint{-0.014433in}{-0.024851in}}{\pgfqpoint{-0.007367in}{-0.027778in}}{\pgfqpoint{0.000000in}{-0.027778in}}%
\pgfpathclose%
\pgfusepath{stroke,fill}%
}%
\begin{pgfscope}%
\pgfsys@transformshift{0.936211in}{0.850770in}%
\pgfsys@useobject{currentmarker}{}%
\end{pgfscope}%
\end{pgfscope}%
\begin{pgfscope}%
\pgfpathrectangle{\pgfqpoint{0.550713in}{0.076581in}}{\pgfqpoint{3.304276in}{1.378454in}}%
\pgfusepath{clip}%
\pgfsetrectcap%
\pgfsetroundjoin%
\pgfsetlinewidth{0.752812pt}%
\definecolor{currentstroke}{rgb}{0.000000,0.000000,0.000000}%
\pgfsetstrokecolor{currentstroke}%
\pgfsetdash{}{0pt}%
\pgfpathmoveto{\pgfqpoint{1.160902in}{0.777797in}}%
\pgfpathlineto{\pgfqpoint{1.592661in}{0.777797in}}%
\pgfusepath{stroke}%
\end{pgfscope}%
\begin{pgfscope}%
\pgfpathrectangle{\pgfqpoint{0.550713in}{0.076581in}}{\pgfqpoint{3.304276in}{1.378454in}}%
\pgfusepath{clip}%
\pgfsetbuttcap%
\pgfsetroundjoin%
\definecolor{currentfill}{rgb}{1.000000,1.000000,1.000000}%
\pgfsetfillcolor{currentfill}%
\pgfsetlinewidth{1.003750pt}%
\definecolor{currentstroke}{rgb}{0.000000,0.000000,0.000000}%
\pgfsetstrokecolor{currentstroke}%
\pgfsetdash{}{0pt}%
\pgfsys@defobject{currentmarker}{\pgfqpoint{-0.027778in}{-0.027778in}}{\pgfqpoint{0.027778in}{0.027778in}}{%
\pgfpathmoveto{\pgfqpoint{0.000000in}{-0.027778in}}%
\pgfpathcurveto{\pgfqpoint{0.007367in}{-0.027778in}}{\pgfqpoint{0.014433in}{-0.024851in}}{\pgfqpoint{0.019642in}{-0.019642in}}%
\pgfpathcurveto{\pgfqpoint{0.024851in}{-0.014433in}}{\pgfqpoint{0.027778in}{-0.007367in}}{\pgfqpoint{0.027778in}{0.000000in}}%
\pgfpathcurveto{\pgfqpoint{0.027778in}{0.007367in}}{\pgfqpoint{0.024851in}{0.014433in}}{\pgfqpoint{0.019642in}{0.019642in}}%
\pgfpathcurveto{\pgfqpoint{0.014433in}{0.024851in}}{\pgfqpoint{0.007367in}{0.027778in}}{\pgfqpoint{0.000000in}{0.027778in}}%
\pgfpathcurveto{\pgfqpoint{-0.007367in}{0.027778in}}{\pgfqpoint{-0.014433in}{0.024851in}}{\pgfqpoint{-0.019642in}{0.019642in}}%
\pgfpathcurveto{\pgfqpoint{-0.024851in}{0.014433in}}{\pgfqpoint{-0.027778in}{0.007367in}}{\pgfqpoint{-0.027778in}{0.000000in}}%
\pgfpathcurveto{\pgfqpoint{-0.027778in}{-0.007367in}}{\pgfqpoint{-0.024851in}{-0.014433in}}{\pgfqpoint{-0.019642in}{-0.019642in}}%
\pgfpathcurveto{\pgfqpoint{-0.014433in}{-0.024851in}}{\pgfqpoint{-0.007367in}{-0.027778in}}{\pgfqpoint{0.000000in}{-0.027778in}}%
\pgfpathclose%
\pgfusepath{stroke,fill}%
}%
\begin{pgfscope}%
\pgfsys@transformshift{1.376782in}{0.764160in}%
\pgfsys@useobject{currentmarker}{}%
\end{pgfscope}%
\end{pgfscope}%
\begin{pgfscope}%
\pgfpathrectangle{\pgfqpoint{0.550713in}{0.076581in}}{\pgfqpoint{3.304276in}{1.378454in}}%
\pgfusepath{clip}%
\pgfsetrectcap%
\pgfsetroundjoin%
\pgfsetlinewidth{0.752812pt}%
\definecolor{currentstroke}{rgb}{0.000000,0.000000,0.000000}%
\pgfsetstrokecolor{currentstroke}%
\pgfsetdash{}{0pt}%
\pgfpathmoveto{\pgfqpoint{1.601472in}{0.984431in}}%
\pgfpathlineto{\pgfqpoint{2.033231in}{0.984431in}}%
\pgfusepath{stroke}%
\end{pgfscope}%
\begin{pgfscope}%
\pgfpathrectangle{\pgfqpoint{0.550713in}{0.076581in}}{\pgfqpoint{3.304276in}{1.378454in}}%
\pgfusepath{clip}%
\pgfsetbuttcap%
\pgfsetroundjoin%
\definecolor{currentfill}{rgb}{1.000000,1.000000,1.000000}%
\pgfsetfillcolor{currentfill}%
\pgfsetlinewidth{1.003750pt}%
\definecolor{currentstroke}{rgb}{0.000000,0.000000,0.000000}%
\pgfsetstrokecolor{currentstroke}%
\pgfsetdash{}{0pt}%
\pgfsys@defobject{currentmarker}{\pgfqpoint{-0.027778in}{-0.027778in}}{\pgfqpoint{0.027778in}{0.027778in}}{%
\pgfpathmoveto{\pgfqpoint{0.000000in}{-0.027778in}}%
\pgfpathcurveto{\pgfqpoint{0.007367in}{-0.027778in}}{\pgfqpoint{0.014433in}{-0.024851in}}{\pgfqpoint{0.019642in}{-0.019642in}}%
\pgfpathcurveto{\pgfqpoint{0.024851in}{-0.014433in}}{\pgfqpoint{0.027778in}{-0.007367in}}{\pgfqpoint{0.027778in}{0.000000in}}%
\pgfpathcurveto{\pgfqpoint{0.027778in}{0.007367in}}{\pgfqpoint{0.024851in}{0.014433in}}{\pgfqpoint{0.019642in}{0.019642in}}%
\pgfpathcurveto{\pgfqpoint{0.014433in}{0.024851in}}{\pgfqpoint{0.007367in}{0.027778in}}{\pgfqpoint{0.000000in}{0.027778in}}%
\pgfpathcurveto{\pgfqpoint{-0.007367in}{0.027778in}}{\pgfqpoint{-0.014433in}{0.024851in}}{\pgfqpoint{-0.019642in}{0.019642in}}%
\pgfpathcurveto{\pgfqpoint{-0.024851in}{0.014433in}}{\pgfqpoint{-0.027778in}{0.007367in}}{\pgfqpoint{-0.027778in}{0.000000in}}%
\pgfpathcurveto{\pgfqpoint{-0.027778in}{-0.007367in}}{\pgfqpoint{-0.024851in}{-0.014433in}}{\pgfqpoint{-0.019642in}{-0.019642in}}%
\pgfpathcurveto{\pgfqpoint{-0.014433in}{-0.024851in}}{\pgfqpoint{-0.007367in}{-0.027778in}}{\pgfqpoint{0.000000in}{-0.027778in}}%
\pgfpathclose%
\pgfusepath{stroke,fill}%
}%
\begin{pgfscope}%
\pgfsys@transformshift{1.817352in}{1.073864in}%
\pgfsys@useobject{currentmarker}{}%
\end{pgfscope}%
\end{pgfscope}%
\begin{pgfscope}%
\pgfpathrectangle{\pgfqpoint{0.550713in}{0.076581in}}{\pgfqpoint{3.304276in}{1.378454in}}%
\pgfusepath{clip}%
\pgfsetrectcap%
\pgfsetroundjoin%
\pgfsetlinewidth{0.752812pt}%
\definecolor{currentstroke}{rgb}{0.000000,0.000000,0.000000}%
\pgfsetstrokecolor{currentstroke}%
\pgfsetdash{}{0pt}%
\pgfpathmoveto{\pgfqpoint{2.372470in}{0.815104in}}%
\pgfpathlineto{\pgfqpoint{2.804229in}{0.815104in}}%
\pgfusepath{stroke}%
\end{pgfscope}%
\begin{pgfscope}%
\pgfpathrectangle{\pgfqpoint{0.550713in}{0.076581in}}{\pgfqpoint{3.304276in}{1.378454in}}%
\pgfusepath{clip}%
\pgfsetbuttcap%
\pgfsetroundjoin%
\definecolor{currentfill}{rgb}{1.000000,1.000000,1.000000}%
\pgfsetfillcolor{currentfill}%
\pgfsetlinewidth{1.003750pt}%
\definecolor{currentstroke}{rgb}{0.000000,0.000000,0.000000}%
\pgfsetstrokecolor{currentstroke}%
\pgfsetdash{}{0pt}%
\pgfsys@defobject{currentmarker}{\pgfqpoint{-0.027778in}{-0.027778in}}{\pgfqpoint{0.027778in}{0.027778in}}{%
\pgfpathmoveto{\pgfqpoint{0.000000in}{-0.027778in}}%
\pgfpathcurveto{\pgfqpoint{0.007367in}{-0.027778in}}{\pgfqpoint{0.014433in}{-0.024851in}}{\pgfqpoint{0.019642in}{-0.019642in}}%
\pgfpathcurveto{\pgfqpoint{0.024851in}{-0.014433in}}{\pgfqpoint{0.027778in}{-0.007367in}}{\pgfqpoint{0.027778in}{0.000000in}}%
\pgfpathcurveto{\pgfqpoint{0.027778in}{0.007367in}}{\pgfqpoint{0.024851in}{0.014433in}}{\pgfqpoint{0.019642in}{0.019642in}}%
\pgfpathcurveto{\pgfqpoint{0.014433in}{0.024851in}}{\pgfqpoint{0.007367in}{0.027778in}}{\pgfqpoint{0.000000in}{0.027778in}}%
\pgfpathcurveto{\pgfqpoint{-0.007367in}{0.027778in}}{\pgfqpoint{-0.014433in}{0.024851in}}{\pgfqpoint{-0.019642in}{0.019642in}}%
\pgfpathcurveto{\pgfqpoint{-0.024851in}{0.014433in}}{\pgfqpoint{-0.027778in}{0.007367in}}{\pgfqpoint{-0.027778in}{0.000000in}}%
\pgfpathcurveto{\pgfqpoint{-0.027778in}{-0.007367in}}{\pgfqpoint{-0.024851in}{-0.014433in}}{\pgfqpoint{-0.019642in}{-0.019642in}}%
\pgfpathcurveto{\pgfqpoint{-0.014433in}{-0.024851in}}{\pgfqpoint{-0.007367in}{-0.027778in}}{\pgfqpoint{0.000000in}{-0.027778in}}%
\pgfpathclose%
\pgfusepath{stroke,fill}%
}%
\begin{pgfscope}%
\pgfsys@transformshift{2.588349in}{0.857629in}%
\pgfsys@useobject{currentmarker}{}%
\end{pgfscope}%
\end{pgfscope}%
\begin{pgfscope}%
\pgfpathrectangle{\pgfqpoint{0.550713in}{0.076581in}}{\pgfqpoint{3.304276in}{1.378454in}}%
\pgfusepath{clip}%
\pgfsetrectcap%
\pgfsetroundjoin%
\pgfsetlinewidth{0.752812pt}%
\definecolor{currentstroke}{rgb}{0.000000,0.000000,0.000000}%
\pgfsetstrokecolor{currentstroke}%
\pgfsetdash{}{0pt}%
\pgfpathmoveto{\pgfqpoint{2.813040in}{0.783127in}}%
\pgfpathlineto{\pgfqpoint{3.244799in}{0.783127in}}%
\pgfusepath{stroke}%
\end{pgfscope}%
\begin{pgfscope}%
\pgfpathrectangle{\pgfqpoint{0.550713in}{0.076581in}}{\pgfqpoint{3.304276in}{1.378454in}}%
\pgfusepath{clip}%
\pgfsetbuttcap%
\pgfsetroundjoin%
\definecolor{currentfill}{rgb}{1.000000,1.000000,1.000000}%
\pgfsetfillcolor{currentfill}%
\pgfsetlinewidth{1.003750pt}%
\definecolor{currentstroke}{rgb}{0.000000,0.000000,0.000000}%
\pgfsetstrokecolor{currentstroke}%
\pgfsetdash{}{0pt}%
\pgfsys@defobject{currentmarker}{\pgfqpoint{-0.027778in}{-0.027778in}}{\pgfqpoint{0.027778in}{0.027778in}}{%
\pgfpathmoveto{\pgfqpoint{0.000000in}{-0.027778in}}%
\pgfpathcurveto{\pgfqpoint{0.007367in}{-0.027778in}}{\pgfqpoint{0.014433in}{-0.024851in}}{\pgfqpoint{0.019642in}{-0.019642in}}%
\pgfpathcurveto{\pgfqpoint{0.024851in}{-0.014433in}}{\pgfqpoint{0.027778in}{-0.007367in}}{\pgfqpoint{0.027778in}{0.000000in}}%
\pgfpathcurveto{\pgfqpoint{0.027778in}{0.007367in}}{\pgfqpoint{0.024851in}{0.014433in}}{\pgfqpoint{0.019642in}{0.019642in}}%
\pgfpathcurveto{\pgfqpoint{0.014433in}{0.024851in}}{\pgfqpoint{0.007367in}{0.027778in}}{\pgfqpoint{0.000000in}{0.027778in}}%
\pgfpathcurveto{\pgfqpoint{-0.007367in}{0.027778in}}{\pgfqpoint{-0.014433in}{0.024851in}}{\pgfqpoint{-0.019642in}{0.019642in}}%
\pgfpathcurveto{\pgfqpoint{-0.024851in}{0.014433in}}{\pgfqpoint{-0.027778in}{0.007367in}}{\pgfqpoint{-0.027778in}{0.000000in}}%
\pgfpathcurveto{\pgfqpoint{-0.027778in}{-0.007367in}}{\pgfqpoint{-0.024851in}{-0.014433in}}{\pgfqpoint{-0.019642in}{-0.019642in}}%
\pgfpathcurveto{\pgfqpoint{-0.014433in}{-0.024851in}}{\pgfqpoint{-0.007367in}{-0.027778in}}{\pgfqpoint{0.000000in}{-0.027778in}}%
\pgfpathclose%
\pgfusepath{stroke,fill}%
}%
\begin{pgfscope}%
\pgfsys@transformshift{3.028919in}{0.762049in}%
\pgfsys@useobject{currentmarker}{}%
\end{pgfscope}%
\end{pgfscope}%
\begin{pgfscope}%
\pgfpathrectangle{\pgfqpoint{0.550713in}{0.076581in}}{\pgfqpoint{3.304276in}{1.378454in}}%
\pgfusepath{clip}%
\pgfsetrectcap%
\pgfsetroundjoin%
\pgfsetlinewidth{0.752812pt}%
\definecolor{currentstroke}{rgb}{0.000000,0.000000,0.000000}%
\pgfsetstrokecolor{currentstroke}%
\pgfsetdash{}{0pt}%
\pgfpathmoveto{\pgfqpoint{3.253610in}{1.056834in}}%
\pgfpathlineto{\pgfqpoint{3.685369in}{1.056834in}}%
\pgfusepath{stroke}%
\end{pgfscope}%
\begin{pgfscope}%
\pgfpathrectangle{\pgfqpoint{0.550713in}{0.076581in}}{\pgfqpoint{3.304276in}{1.378454in}}%
\pgfusepath{clip}%
\pgfsetbuttcap%
\pgfsetroundjoin%
\definecolor{currentfill}{rgb}{1.000000,1.000000,1.000000}%
\pgfsetfillcolor{currentfill}%
\pgfsetlinewidth{1.003750pt}%
\definecolor{currentstroke}{rgb}{0.000000,0.000000,0.000000}%
\pgfsetstrokecolor{currentstroke}%
\pgfsetdash{}{0pt}%
\pgfsys@defobject{currentmarker}{\pgfqpoint{-0.027778in}{-0.027778in}}{\pgfqpoint{0.027778in}{0.027778in}}{%
\pgfpathmoveto{\pgfqpoint{0.000000in}{-0.027778in}}%
\pgfpathcurveto{\pgfqpoint{0.007367in}{-0.027778in}}{\pgfqpoint{0.014433in}{-0.024851in}}{\pgfqpoint{0.019642in}{-0.019642in}}%
\pgfpathcurveto{\pgfqpoint{0.024851in}{-0.014433in}}{\pgfqpoint{0.027778in}{-0.007367in}}{\pgfqpoint{0.027778in}{0.000000in}}%
\pgfpathcurveto{\pgfqpoint{0.027778in}{0.007367in}}{\pgfqpoint{0.024851in}{0.014433in}}{\pgfqpoint{0.019642in}{0.019642in}}%
\pgfpathcurveto{\pgfqpoint{0.014433in}{0.024851in}}{\pgfqpoint{0.007367in}{0.027778in}}{\pgfqpoint{0.000000in}{0.027778in}}%
\pgfpathcurveto{\pgfqpoint{-0.007367in}{0.027778in}}{\pgfqpoint{-0.014433in}{0.024851in}}{\pgfqpoint{-0.019642in}{0.019642in}}%
\pgfpathcurveto{\pgfqpoint{-0.024851in}{0.014433in}}{\pgfqpoint{-0.027778in}{0.007367in}}{\pgfqpoint{-0.027778in}{0.000000in}}%
\pgfpathcurveto{\pgfqpoint{-0.027778in}{-0.007367in}}{\pgfqpoint{-0.024851in}{-0.014433in}}{\pgfqpoint{-0.019642in}{-0.019642in}}%
\pgfpathcurveto{\pgfqpoint{-0.014433in}{-0.024851in}}{\pgfqpoint{-0.007367in}{-0.027778in}}{\pgfqpoint{0.000000in}{-0.027778in}}%
\pgfpathclose%
\pgfusepath{stroke,fill}%
}%
\begin{pgfscope}%
\pgfsys@transformshift{3.469489in}{1.176478in}%
\pgfsys@useobject{currentmarker}{}%
\end{pgfscope}%
\end{pgfscope}%
\begin{pgfscope}%
\pgfsetrectcap%
\pgfsetmiterjoin%
\pgfsetlinewidth{0.752812pt}%
\definecolor{currentstroke}{rgb}{0.000000,0.000000,0.000000}%
\pgfsetstrokecolor{currentstroke}%
\pgfsetdash{}{0pt}%
\pgfpathmoveto{\pgfqpoint{0.550713in}{0.076581in}}%
\pgfpathlineto{\pgfqpoint{0.550713in}{1.455035in}}%
\pgfusepath{stroke}%
\end{pgfscope}%
\begin{pgfscope}%
\pgfsetrectcap%
\pgfsetmiterjoin%
\pgfsetlinewidth{0.752812pt}%
\definecolor{currentstroke}{rgb}{0.000000,0.000000,0.000000}%
\pgfsetstrokecolor{currentstroke}%
\pgfsetdash{}{0pt}%
\pgfpathmoveto{\pgfqpoint{3.854988in}{0.076581in}}%
\pgfpathlineto{\pgfqpoint{3.854988in}{1.455035in}}%
\pgfusepath{stroke}%
\end{pgfscope}%
\begin{pgfscope}%
\pgfsetrectcap%
\pgfsetmiterjoin%
\pgfsetlinewidth{0.752812pt}%
\definecolor{currentstroke}{rgb}{0.000000,0.000000,0.000000}%
\pgfsetstrokecolor{currentstroke}%
\pgfsetdash{}{0pt}%
\pgfpathmoveto{\pgfqpoint{0.550713in}{0.076581in}}%
\pgfpathlineto{\pgfqpoint{3.854988in}{0.076581in}}%
\pgfusepath{stroke}%
\end{pgfscope}%
\begin{pgfscope}%
\pgfsetrectcap%
\pgfsetmiterjoin%
\pgfsetlinewidth{0.752812pt}%
\definecolor{currentstroke}{rgb}{0.000000,0.000000,0.000000}%
\pgfsetstrokecolor{currentstroke}%
\pgfsetdash{}{0pt}%
\pgfpathmoveto{\pgfqpoint{0.550713in}{1.455035in}}%
\pgfpathlineto{\pgfqpoint{3.854988in}{1.455035in}}%
\pgfusepath{stroke}%
\end{pgfscope}%
\begin{pgfscope}%
\pgfsetbuttcap%
\pgfsetmiterjoin%
\definecolor{currentfill}{rgb}{0.631373,0.062745,0.207843}%
\pgfsetfillcolor{currentfill}%
\pgfsetlinewidth{0.000000pt}%
\definecolor{currentstroke}{rgb}{0.000000,0.000000,0.000000}%
\pgfsetstrokecolor{currentstroke}%
\pgfsetstrokeopacity{0.000000}%
\pgfsetdash{}{0pt}%
\pgfpathmoveto{\pgfqpoint{3.979988in}{0.827475in}}%
\pgfpathlineto{\pgfqpoint{4.257766in}{0.827475in}}%
\pgfpathlineto{\pgfqpoint{4.257766in}{0.924697in}}%
\pgfpathlineto{\pgfqpoint{3.979988in}{0.924697in}}%
\pgfpathclose%
\pgfusepath{fill}%
\end{pgfscope}%
\begin{pgfscope}%
\definecolor{textcolor}{rgb}{0.000000,0.000000,0.000000}%
\pgfsetstrokecolor{textcolor}%
\pgfsetfillcolor{textcolor}%
\pgftext[x=4.368877in,y=0.827475in,left,base]{\color{textcolor}\rmfamily\fontsize{10.000000}{12.000000}\selectfont TV-GP-UCB}%
\end{pgfscope}%
\begin{pgfscope}%
\pgfsetbuttcap%
\pgfsetmiterjoin%
\definecolor{currentfill}{rgb}{0.890196,0.000000,0.400000}%
\pgfsetfillcolor{currentfill}%
\pgfsetlinewidth{0.000000pt}%
\definecolor{currentstroke}{rgb}{0.000000,0.000000,0.000000}%
\pgfsetstrokecolor{currentstroke}%
\pgfsetstrokeopacity{0.000000}%
\pgfsetdash{}{0pt}%
\pgfpathmoveto{\pgfqpoint{3.979988in}{0.633864in}}%
\pgfpathlineto{\pgfqpoint{4.257766in}{0.633864in}}%
\pgfpathlineto{\pgfqpoint{4.257766in}{0.731086in}}%
\pgfpathlineto{\pgfqpoint{3.979988in}{0.731086in}}%
\pgfpathclose%
\pgfusepath{fill}%
\end{pgfscope}%
\begin{pgfscope}%
\definecolor{textcolor}{rgb}{0.000000,0.000000,0.000000}%
\pgfsetstrokecolor{textcolor}%
\pgfsetfillcolor{textcolor}%
\pgftext[x=4.368877in,y=0.633864in,left,base]{\color{textcolor}\rmfamily\fontsize{10.000000}{12.000000}\selectfont SW TV-GP-UCB}%
\end{pgfscope}%
\begin{pgfscope}%
\pgfsetbuttcap%
\pgfsetmiterjoin%
\definecolor{currentfill}{rgb}{1.000000,1.000000,1.000000}%
\pgfsetfillcolor{currentfill}%
\pgfsetlinewidth{1.003750pt}%
\definecolor{currentstroke}{rgb}{1.000000,1.000000,1.000000}%
\pgfsetstrokecolor{currentstroke}%
\pgfsetdash{}{0pt}%
\pgfpathmoveto{\pgfqpoint{0.603736in}{0.104359in}}%
\pgfpathlineto{\pgfqpoint{3.799433in}{0.104359in}}%
\pgfpathquadraticcurveto{\pgfqpoint{3.827210in}{0.104359in}}{\pgfqpoint{3.827210in}{0.132136in}}%
\pgfpathlineto{\pgfqpoint{3.827210in}{0.311920in}}%
\pgfpathquadraticcurveto{\pgfqpoint{3.827210in}{0.339698in}}{\pgfqpoint{3.799433in}{0.339698in}}%
\pgfpathlineto{\pgfqpoint{0.603736in}{0.339698in}}%
\pgfpathquadraticcurveto{\pgfqpoint{0.575958in}{0.339698in}}{\pgfqpoint{0.575958in}{0.311920in}}%
\pgfpathlineto{\pgfqpoint{0.575958in}{0.132136in}}%
\pgfpathquadraticcurveto{\pgfqpoint{0.575958in}{0.104359in}}{\pgfqpoint{0.603736in}{0.104359in}}%
\pgfpathclose%
\pgfusepath{stroke,fill}%
\end{pgfscope}%
\begin{pgfscope}%
\pgfsetbuttcap%
\pgfsetmiterjoin%
\definecolor{currentfill}{rgb}{0.000000,0.000000,0.000000}%
\pgfsetfillcolor{currentfill}%
\pgfsetlinewidth{0.000000pt}%
\definecolor{currentstroke}{rgb}{0.000000,0.000000,0.000000}%
\pgfsetstrokecolor{currentstroke}%
\pgfsetstrokeopacity{0.000000}%
\pgfsetdash{}{0pt}%
\pgfpathmoveto{\pgfqpoint{0.631514in}{0.186920in}}%
\pgfpathlineto{\pgfqpoint{0.909292in}{0.186920in}}%
\pgfpathlineto{\pgfqpoint{0.909292in}{0.284142in}}%
\pgfpathlineto{\pgfqpoint{0.631514in}{0.284142in}}%
\pgfpathclose%
\pgfusepath{fill}%
\end{pgfscope}%
\begin{pgfscope}%
\definecolor{textcolor}{rgb}{0.000000,0.000000,0.000000}%
\pgfsetstrokecolor{textcolor}%
\pgfsetfillcolor{textcolor}%
\pgftext[x=1.020403in,y=0.186920in,left,base]{\color{textcolor}\rmfamily\fontsize{10.000000}{12.000000}\selectfont \(\displaystyle \mu_0=0\)}%
\end{pgfscope}%
\begin{pgfscope}%
\pgfsetbuttcap%
\pgfsetmiterjoin%
\definecolor{currentfill}{rgb}{0.811765,0.819608,0.823529}%
\pgfsetfillcolor{currentfill}%
\pgfsetlinewidth{0.000000pt}%
\definecolor{currentstroke}{rgb}{0.000000,0.000000,0.000000}%
\pgfsetstrokecolor{currentstroke}%
\pgfsetstrokeopacity{0.000000}%
\pgfsetdash{}{0pt}%
\pgfpathmoveto{\pgfqpoint{1.698804in}{0.186920in}}%
\pgfpathlineto{\pgfqpoint{1.976581in}{0.186920in}}%
\pgfpathlineto{\pgfqpoint{1.976581in}{0.284142in}}%
\pgfpathlineto{\pgfqpoint{1.698804in}{0.284142in}}%
\pgfpathclose%
\pgfusepath{fill}%
\end{pgfscope}%
\begin{pgfscope}%
\definecolor{textcolor}{rgb}{0.000000,0.000000,0.000000}%
\pgfsetstrokecolor{textcolor}%
\pgfsetfillcolor{textcolor}%
\pgftext[x=2.087692in,y=0.186920in,left,base]{\color{textcolor}\rmfamily\fontsize{10.000000}{12.000000}\selectfont \(\displaystyle \mu_0=-2\)}%
\end{pgfscope}%
\begin{pgfscope}%
\pgfsetbuttcap%
\pgfsetmiterjoin%
\definecolor{currentfill}{rgb}{0.925490,0.929412,0.929412}%
\pgfsetfillcolor{currentfill}%
\pgfsetlinewidth{0.000000pt}%
\definecolor{currentstroke}{rgb}{0.000000,0.000000,0.000000}%
\pgfsetstrokecolor{currentstroke}%
\pgfsetstrokeopacity{0.000000}%
\pgfsetdash{}{0pt}%
\pgfpathmoveto{\pgfqpoint{2.874118in}{0.186920in}}%
\pgfpathlineto{\pgfqpoint{3.151896in}{0.186920in}}%
\pgfpathlineto{\pgfqpoint{3.151896in}{0.284142in}}%
\pgfpathlineto{\pgfqpoint{2.874118in}{0.284142in}}%
\pgfpathclose%
\pgfusepath{fill}%
\end{pgfscope}%
\begin{pgfscope}%
\definecolor{textcolor}{rgb}{0.000000,0.000000,0.000000}%
\pgfsetstrokecolor{textcolor}%
\pgfsetfillcolor{textcolor}%
\pgftext[x=3.263007in,y=0.186920in,left,base]{\color{textcolor}\rmfamily\fontsize{10.000000}{12.000000}\selectfont \(\displaystyle \mu_0=-4\)}%
\end{pgfscope}%
\end{pgfpicture}%
\makeatother%
\endgroup%

    \caption[Influence of different optimistic means on \gls{b2p} forgetting in the two-dimensional within-model comparison.]{Influence of different optimistic means on \gls{b2p} forgetting in the two-dimensional within-model comparison.}
    \label{fig:WMC_cum_regret_different_mean}
\end{figure}

Here, additional simulations with an optimistic prior mean of $\mu_0=-4$ were performed, showing the high sensitivity of \gls{b2p} forgetting regarding the prior mean as in the one-dimensional within-model comparison.
This high sensitivity can also complicate the tuning of hyperparameters since the exploration behavior depends not only on the forgetting factor and $\beta_{t+1}$ but also on the prior mean. If the mean changes, all parameters have to be readjusted to have a desirable exploration-exploitation trade-off.


\subsection{Out-Of-Model Comparison}
\label{sec:out_of_model}

For the out-of-model comparison, the same models as in the previous \Cref{sec:within-model} are used, however the length scales are no longer known a-priori. Therefore, at each time step a hyperparameter optimization is performed. A prior on the length scales in form of a Gamma distribution $\boldsymbol\Lambda_{ii} \sim \mathcal{G}(\alpha,\beta)$ as by \textcite{Marco_2016} with the probability density function as
\begin{equation}
    p(x|\alpha,\beta) = \begin{cases}
            \frac{\beta^\alpha}{\Gamma(\alpha)} \cdot x^{\alpha-1} \exp{\left(-\beta \cdot x\right)} \, ,& x > 0\\
            0\, ,&x \leq 0
        \end{cases}
        \label{eq:gamma}
\end{equation}
with $\alpha = 11$ and $\beta = \frac{10}{3}$ is chosen. Furthermore, bounds on the length scales are chosen as $\boldsymbol\Lambda_{ii} \in [2,5]$. The other hyperparameter setting are identical to the within-model comparison in \Cref{sec:within-model}. Figure~\ref{fig:OOMC_cumulative_regret_1D} shows the results for the one-dimensional objective functions. All variations except SW TV-GP-UCB with an optimistic prior mean outperform exploiting the mean after the initialization. 

Furthermore, applying \gls{ctvbo} to \gls{ui} forgetting with the proposed modeling approach \gls{uitvbo} results in the lowest regret. The variations using \gls{ui} forgetting are more robust to the change in prior mean compared to the variations using \gls{b2p} forgetting, supporting Hypothesis~\ref{hyp:ui_structural_information}. With a well-defined prior mean, \gls{b2p} and \gls{ui} forgetting show similar mean regret when using standard \gls{tvbo} as stated in Hypothesis~\ref{hyp:ui_good_mean}. However, the main deviation from the within-model comparisons is that \gls{b2p} forgetting with \gls{ctvbo} no longer shows an advantage compared to standard \gls{tvbo} if no data selection strategy is applied. A reason for this could be that the learned length scales result in a posterior, which is very flat, thus increasing the sampling radius around the optimum in the constrained case.
\begin{figure}[h]
    \centering
    %% Creator: Matplotlib, PGF backend
%%
%% To include the figure in your LaTeX document, write
%%   \input{<filename>.pgf}
%%
%% Make sure the required packages are loaded in your preamble
%%   \usepackage{pgf}
%%
%% Figures using additional raster images can only be included by \input if
%% they are in the same directory as the main LaTeX file. For loading figures
%% from other directories you can use the `import` package
%%   \usepackage{import}
%%
%% and then include the figures with
%%   \import{<path to file>}{<filename>.pgf}
%%
%% Matplotlib used the following preamble
%%   \usepackage{fontspec}
%%
\begingroup%
\makeatletter%
\begin{pgfpicture}%
\pgfpathrectangle{\pgfpointorigin}{\pgfqpoint{5.507126in}{1.871975in}}%
\pgfusepath{use as bounding box, clip}%
\begin{pgfscope}%
\pgfsetbuttcap%
\pgfsetmiterjoin%
\definecolor{currentfill}{rgb}{1.000000,1.000000,1.000000}%
\pgfsetfillcolor{currentfill}%
\pgfsetlinewidth{0.000000pt}%
\definecolor{currentstroke}{rgb}{1.000000,1.000000,1.000000}%
\pgfsetstrokecolor{currentstroke}%
\pgfsetdash{}{0pt}%
\pgfpathmoveto{\pgfqpoint{0.000000in}{0.000000in}}%
\pgfpathlineto{\pgfqpoint{5.507126in}{0.000000in}}%
\pgfpathlineto{\pgfqpoint{5.507126in}{1.871975in}}%
\pgfpathlineto{\pgfqpoint{0.000000in}{1.871975in}}%
\pgfpathclose%
\pgfusepath{fill}%
\end{pgfscope}%
\begin{pgfscope}%
\pgfsetbuttcap%
\pgfsetmiterjoin%
\definecolor{currentfill}{rgb}{1.000000,1.000000,1.000000}%
\pgfsetfillcolor{currentfill}%
\pgfsetlinewidth{0.000000pt}%
\definecolor{currentstroke}{rgb}{0.000000,0.000000,0.000000}%
\pgfsetstrokecolor{currentstroke}%
\pgfsetstrokeopacity{0.000000}%
\pgfsetdash{}{0pt}%
\pgfpathmoveto{\pgfqpoint{0.550713in}{0.093599in}}%
\pgfpathlineto{\pgfqpoint{3.579632in}{0.093599in}}%
\pgfpathlineto{\pgfqpoint{3.579632in}{1.778376in}}%
\pgfpathlineto{\pgfqpoint{0.550713in}{1.778376in}}%
\pgfpathclose%
\pgfusepath{fill}%
\end{pgfscope}%
\begin{pgfscope}%
\pgfpathrectangle{\pgfqpoint{0.550713in}{0.093599in}}{\pgfqpoint{3.028919in}{1.684778in}}%
\pgfusepath{clip}%
\pgfsetbuttcap%
\pgfsetmiterjoin%
\definecolor{currentfill}{rgb}{0.631373,0.062745,0.207843}%
\pgfsetfillcolor{currentfill}%
\pgfsetlinewidth{0.752812pt}%
\definecolor{currentstroke}{rgb}{0.000000,0.000000,0.000000}%
\pgfsetstrokecolor{currentstroke}%
\pgfsetdash{}{0pt}%
\pgfpathmoveto{\pgfqpoint{0.590089in}{0.537288in}}%
\pgfpathlineto{\pgfqpoint{0.738506in}{0.537288in}}%
\pgfpathlineto{\pgfqpoint{0.738506in}{0.621337in}}%
\pgfpathlineto{\pgfqpoint{0.590089in}{0.621337in}}%
\pgfpathlineto{\pgfqpoint{0.590089in}{0.537288in}}%
\pgfpathclose%
\pgfusepath{stroke,fill}%
\end{pgfscope}%
\begin{pgfscope}%
\pgfpathrectangle{\pgfqpoint{0.550713in}{0.093599in}}{\pgfqpoint{3.028919in}{1.684778in}}%
\pgfusepath{clip}%
\pgfsetbuttcap%
\pgfsetmiterjoin%
\definecolor{currentfill}{rgb}{0.898039,0.772549,0.752941}%
\pgfsetfillcolor{currentfill}%
\pgfsetlinewidth{0.752812pt}%
\definecolor{currentstroke}{rgb}{0.000000,0.000000,0.000000}%
\pgfsetstrokecolor{currentstroke}%
\pgfsetdash{}{0pt}%
\pgfpathmoveto{\pgfqpoint{0.741535in}{0.810756in}}%
\pgfpathlineto{\pgfqpoint{0.889952in}{0.810756in}}%
\pgfpathlineto{\pgfqpoint{0.889952in}{0.909051in}}%
\pgfpathlineto{\pgfqpoint{0.741535in}{0.909051in}}%
\pgfpathlineto{\pgfqpoint{0.741535in}{0.810756in}}%
\pgfpathclose%
\pgfusepath{stroke,fill}%
\end{pgfscope}%
\begin{pgfscope}%
\pgfpathrectangle{\pgfqpoint{0.550713in}{0.093599in}}{\pgfqpoint{3.028919in}{1.684778in}}%
\pgfusepath{clip}%
\pgfsetbuttcap%
\pgfsetmiterjoin%
\definecolor{currentfill}{rgb}{0.890196,0.000000,0.400000}%
\pgfsetfillcolor{currentfill}%
\pgfsetlinewidth{0.752812pt}%
\definecolor{currentstroke}{rgb}{0.000000,0.000000,0.000000}%
\pgfsetstrokecolor{currentstroke}%
\pgfsetdash{}{0pt}%
\pgfpathmoveto{\pgfqpoint{0.968703in}{0.619040in}}%
\pgfpathlineto{\pgfqpoint{1.117121in}{0.619040in}}%
\pgfpathlineto{\pgfqpoint{1.117121in}{0.828417in}}%
\pgfpathlineto{\pgfqpoint{0.968703in}{0.828417in}}%
\pgfpathlineto{\pgfqpoint{0.968703in}{0.619040in}}%
\pgfpathclose%
\pgfusepath{stroke,fill}%
\end{pgfscope}%
\begin{pgfscope}%
\pgfpathrectangle{\pgfqpoint{0.550713in}{0.093599in}}{\pgfqpoint{3.028919in}{1.684778in}}%
\pgfusepath{clip}%
\pgfsetbuttcap%
\pgfsetmiterjoin%
\definecolor{currentfill}{rgb}{0.976471,0.823529,0.854902}%
\pgfsetfillcolor{currentfill}%
\pgfsetlinewidth{0.752812pt}%
\definecolor{currentstroke}{rgb}{0.000000,0.000000,0.000000}%
\pgfsetstrokecolor{currentstroke}%
\pgfsetdash{}{0pt}%
\pgfpathmoveto{\pgfqpoint{1.120149in}{1.173757in}}%
\pgfpathlineto{\pgfqpoint{1.268566in}{1.173757in}}%
\pgfpathlineto{\pgfqpoint{1.268566in}{1.512378in}}%
\pgfpathlineto{\pgfqpoint{1.120149in}{1.512378in}}%
\pgfpathlineto{\pgfqpoint{1.120149in}{1.173757in}}%
\pgfpathclose%
\pgfusepath{stroke,fill}%
\end{pgfscope}%
\begin{pgfscope}%
\pgfpathrectangle{\pgfqpoint{0.550713in}{0.093599in}}{\pgfqpoint{3.028919in}{1.684778in}}%
\pgfusepath{clip}%
\pgfsetbuttcap%
\pgfsetmiterjoin%
\definecolor{currentfill}{rgb}{0.000000,0.329412,0.623529}%
\pgfsetfillcolor{currentfill}%
\pgfsetlinewidth{0.752812pt}%
\definecolor{currentstroke}{rgb}{0.000000,0.000000,0.000000}%
\pgfsetstrokecolor{currentstroke}%
\pgfsetdash{}{0pt}%
\pgfpathmoveto{\pgfqpoint{1.347318in}{0.548173in}}%
\pgfpathlineto{\pgfqpoint{1.495735in}{0.548173in}}%
\pgfpathlineto{\pgfqpoint{1.495735in}{0.643247in}}%
\pgfpathlineto{\pgfqpoint{1.347318in}{0.643247in}}%
\pgfpathlineto{\pgfqpoint{1.347318in}{0.548173in}}%
\pgfpathclose%
\pgfusepath{stroke,fill}%
\end{pgfscope}%
\begin{pgfscope}%
\pgfpathrectangle{\pgfqpoint{0.550713in}{0.093599in}}{\pgfqpoint{3.028919in}{1.684778in}}%
\pgfusepath{clip}%
\pgfsetbuttcap%
\pgfsetmiterjoin%
\definecolor{currentfill}{rgb}{0.780392,0.866667,0.949020}%
\pgfsetfillcolor{currentfill}%
\pgfsetlinewidth{0.752812pt}%
\definecolor{currentstroke}{rgb}{0.000000,0.000000,0.000000}%
\pgfsetstrokecolor{currentstroke}%
\pgfsetdash{}{0pt}%
\pgfpathmoveto{\pgfqpoint{1.498764in}{0.574583in}}%
\pgfpathlineto{\pgfqpoint{1.647181in}{0.574583in}}%
\pgfpathlineto{\pgfqpoint{1.647181in}{0.718210in}}%
\pgfpathlineto{\pgfqpoint{1.498764in}{0.718210in}}%
\pgfpathlineto{\pgfqpoint{1.498764in}{0.574583in}}%
\pgfpathclose%
\pgfusepath{stroke,fill}%
\end{pgfscope}%
\begin{pgfscope}%
\pgfpathrectangle{\pgfqpoint{0.550713in}{0.093599in}}{\pgfqpoint{3.028919in}{1.684778in}}%
\pgfusepath{clip}%
\pgfsetbuttcap%
\pgfsetmiterjoin%
\definecolor{currentfill}{rgb}{0.000000,0.380392,0.396078}%
\pgfsetfillcolor{currentfill}%
\pgfsetlinewidth{0.752812pt}%
\definecolor{currentstroke}{rgb}{0.000000,0.000000,0.000000}%
\pgfsetstrokecolor{currentstroke}%
\pgfsetdash{}{0pt}%
\pgfpathmoveto{\pgfqpoint{1.725933in}{0.593637in}}%
\pgfpathlineto{\pgfqpoint{1.874350in}{0.593637in}}%
\pgfpathlineto{\pgfqpoint{1.874350in}{0.712756in}}%
\pgfpathlineto{\pgfqpoint{1.725933in}{0.712756in}}%
\pgfpathlineto{\pgfqpoint{1.725933in}{0.593637in}}%
\pgfpathclose%
\pgfusepath{stroke,fill}%
\end{pgfscope}%
\begin{pgfscope}%
\pgfpathrectangle{\pgfqpoint{0.550713in}{0.093599in}}{\pgfqpoint{3.028919in}{1.684778in}}%
\pgfusepath{clip}%
\pgfsetbuttcap%
\pgfsetmiterjoin%
\definecolor{currentfill}{rgb}{0.749020,0.815686,0.819608}%
\pgfsetfillcolor{currentfill}%
\pgfsetlinewidth{0.752812pt}%
\definecolor{currentstroke}{rgb}{0.000000,0.000000,0.000000}%
\pgfsetstrokecolor{currentstroke}%
\pgfsetdash{}{0pt}%
\pgfpathmoveto{\pgfqpoint{1.877379in}{0.621709in}}%
\pgfpathlineto{\pgfqpoint{2.025796in}{0.621709in}}%
\pgfpathlineto{\pgfqpoint{2.025796in}{0.774160in}}%
\pgfpathlineto{\pgfqpoint{1.877379in}{0.774160in}}%
\pgfpathlineto{\pgfqpoint{1.877379in}{0.621709in}}%
\pgfpathclose%
\pgfusepath{stroke,fill}%
\end{pgfscope}%
\begin{pgfscope}%
\pgfpathrectangle{\pgfqpoint{0.550713in}{0.093599in}}{\pgfqpoint{3.028919in}{1.684778in}}%
\pgfusepath{clip}%
\pgfsetbuttcap%
\pgfsetmiterjoin%
\definecolor{currentfill}{rgb}{0.380392,0.129412,0.345098}%
\pgfsetfillcolor{currentfill}%
\pgfsetlinewidth{0.752812pt}%
\definecolor{currentstroke}{rgb}{0.000000,0.000000,0.000000}%
\pgfsetstrokecolor{currentstroke}%
\pgfsetdash{}{0pt}%
\pgfpathmoveto{\pgfqpoint{2.104548in}{0.614619in}}%
\pgfpathlineto{\pgfqpoint{2.252965in}{0.614619in}}%
\pgfpathlineto{\pgfqpoint{2.252965in}{0.678538in}}%
\pgfpathlineto{\pgfqpoint{2.104548in}{0.678538in}}%
\pgfpathlineto{\pgfqpoint{2.104548in}{0.614619in}}%
\pgfpathclose%
\pgfusepath{stroke,fill}%
\end{pgfscope}%
\begin{pgfscope}%
\pgfpathrectangle{\pgfqpoint{0.550713in}{0.093599in}}{\pgfqpoint{3.028919in}{1.684778in}}%
\pgfusepath{clip}%
\pgfsetbuttcap%
\pgfsetmiterjoin%
\definecolor{currentfill}{rgb}{0.823529,0.752941,0.803922}%
\pgfsetfillcolor{currentfill}%
\pgfsetlinewidth{0.752812pt}%
\definecolor{currentstroke}{rgb}{0.000000,0.000000,0.000000}%
\pgfsetstrokecolor{currentstroke}%
\pgfsetdash{}{0pt}%
\pgfpathmoveto{\pgfqpoint{2.255994in}{0.733511in}}%
\pgfpathlineto{\pgfqpoint{2.404411in}{0.733511in}}%
\pgfpathlineto{\pgfqpoint{2.404411in}{0.842090in}}%
\pgfpathlineto{\pgfqpoint{2.255994in}{0.842090in}}%
\pgfpathlineto{\pgfqpoint{2.255994in}{0.733511in}}%
\pgfpathclose%
\pgfusepath{stroke,fill}%
\end{pgfscope}%
\begin{pgfscope}%
\pgfpathrectangle{\pgfqpoint{0.550713in}{0.093599in}}{\pgfqpoint{3.028919in}{1.684778in}}%
\pgfusepath{clip}%
\pgfsetbuttcap%
\pgfsetmiterjoin%
\definecolor{currentfill}{rgb}{0.964706,0.658824,0.000000}%
\pgfsetfillcolor{currentfill}%
\pgfsetlinewidth{0.752812pt}%
\definecolor{currentstroke}{rgb}{0.000000,0.000000,0.000000}%
\pgfsetstrokecolor{currentstroke}%
\pgfsetdash{}{0pt}%
\pgfpathmoveto{\pgfqpoint{2.483163in}{0.581919in}}%
\pgfpathlineto{\pgfqpoint{2.631580in}{0.581919in}}%
\pgfpathlineto{\pgfqpoint{2.631580in}{0.619819in}}%
\pgfpathlineto{\pgfqpoint{2.483163in}{0.619819in}}%
\pgfpathlineto{\pgfqpoint{2.483163in}{0.581919in}}%
\pgfpathclose%
\pgfusepath{stroke,fill}%
\end{pgfscope}%
\begin{pgfscope}%
\pgfpathrectangle{\pgfqpoint{0.550713in}{0.093599in}}{\pgfqpoint{3.028919in}{1.684778in}}%
\pgfusepath{clip}%
\pgfsetbuttcap%
\pgfsetmiterjoin%
\definecolor{currentfill}{rgb}{0.996078,0.917647,0.788235}%
\pgfsetfillcolor{currentfill}%
\pgfsetlinewidth{0.752812pt}%
\definecolor{currentstroke}{rgb}{0.000000,0.000000,0.000000}%
\pgfsetstrokecolor{currentstroke}%
\pgfsetdash{}{0pt}%
\pgfpathmoveto{\pgfqpoint{2.634609in}{0.693261in}}%
\pgfpathlineto{\pgfqpoint{2.783026in}{0.693261in}}%
\pgfpathlineto{\pgfqpoint{2.783026in}{0.783139in}}%
\pgfpathlineto{\pgfqpoint{2.634609in}{0.783139in}}%
\pgfpathlineto{\pgfqpoint{2.634609in}{0.693261in}}%
\pgfpathclose%
\pgfusepath{stroke,fill}%
\end{pgfscope}%
\begin{pgfscope}%
\pgfpathrectangle{\pgfqpoint{0.550713in}{0.093599in}}{\pgfqpoint{3.028919in}{1.684778in}}%
\pgfusepath{clip}%
\pgfsetbuttcap%
\pgfsetmiterjoin%
\definecolor{currentfill}{rgb}{0.341176,0.670588,0.152941}%
\pgfsetfillcolor{currentfill}%
\pgfsetlinewidth{0.752812pt}%
\definecolor{currentstroke}{rgb}{0.000000,0.000000,0.000000}%
\pgfsetstrokecolor{currentstroke}%
\pgfsetdash{}{0pt}%
\pgfpathmoveto{\pgfqpoint{2.861778in}{0.488761in}}%
\pgfpathlineto{\pgfqpoint{3.010195in}{0.488761in}}%
\pgfpathlineto{\pgfqpoint{3.010195in}{0.545758in}}%
\pgfpathlineto{\pgfqpoint{2.861778in}{0.545758in}}%
\pgfpathlineto{\pgfqpoint{2.861778in}{0.488761in}}%
\pgfpathclose%
\pgfusepath{stroke,fill}%
\end{pgfscope}%
\begin{pgfscope}%
\pgfpathrectangle{\pgfqpoint{0.550713in}{0.093599in}}{\pgfqpoint{3.028919in}{1.684778in}}%
\pgfusepath{clip}%
\pgfsetbuttcap%
\pgfsetmiterjoin%
\definecolor{currentfill}{rgb}{0.866667,0.921569,0.807843}%
\pgfsetfillcolor{currentfill}%
\pgfsetlinewidth{0.752812pt}%
\definecolor{currentstroke}{rgb}{0.000000,0.000000,0.000000}%
\pgfsetstrokecolor{currentstroke}%
\pgfsetdash{}{0pt}%
\pgfpathmoveto{\pgfqpoint{3.013224in}{0.491709in}}%
\pgfpathlineto{\pgfqpoint{3.161641in}{0.491709in}}%
\pgfpathlineto{\pgfqpoint{3.161641in}{0.559962in}}%
\pgfpathlineto{\pgfqpoint{3.013224in}{0.559962in}}%
\pgfpathlineto{\pgfqpoint{3.013224in}{0.491709in}}%
\pgfpathclose%
\pgfusepath{stroke,fill}%
\end{pgfscope}%
\begin{pgfscope}%
\pgfpathrectangle{\pgfqpoint{0.550713in}{0.093599in}}{\pgfqpoint{3.028919in}{1.684778in}}%
\pgfusepath{clip}%
\pgfsetbuttcap%
\pgfsetmiterjoin%
\definecolor{currentfill}{rgb}{0.478431,0.435294,0.674510}%
\pgfsetfillcolor{currentfill}%
\pgfsetlinewidth{0.752812pt}%
\definecolor{currentstroke}{rgb}{0.000000,0.000000,0.000000}%
\pgfsetstrokecolor{currentstroke}%
\pgfsetdash{}{0pt}%
\pgfpathmoveto{\pgfqpoint{3.240393in}{0.492839in}}%
\pgfpathlineto{\pgfqpoint{3.388810in}{0.492839in}}%
\pgfpathlineto{\pgfqpoint{3.388810in}{0.526347in}}%
\pgfpathlineto{\pgfqpoint{3.240393in}{0.526347in}}%
\pgfpathlineto{\pgfqpoint{3.240393in}{0.492839in}}%
\pgfpathclose%
\pgfusepath{stroke,fill}%
\end{pgfscope}%
\begin{pgfscope}%
\pgfpathrectangle{\pgfqpoint{0.550713in}{0.093599in}}{\pgfqpoint{3.028919in}{1.684778in}}%
\pgfusepath{clip}%
\pgfsetbuttcap%
\pgfsetmiterjoin%
\definecolor{currentfill}{rgb}{0.870588,0.854902,0.921569}%
\pgfsetfillcolor{currentfill}%
\pgfsetlinewidth{0.752812pt}%
\definecolor{currentstroke}{rgb}{0.000000,0.000000,0.000000}%
\pgfsetstrokecolor{currentstroke}%
\pgfsetdash{}{0pt}%
\pgfpathmoveto{\pgfqpoint{3.391839in}{0.502531in}}%
\pgfpathlineto{\pgfqpoint{3.540256in}{0.502531in}}%
\pgfpathlineto{\pgfqpoint{3.540256in}{0.561235in}}%
\pgfpathlineto{\pgfqpoint{3.391839in}{0.561235in}}%
\pgfpathlineto{\pgfqpoint{3.391839in}{0.502531in}}%
\pgfpathclose%
\pgfusepath{stroke,fill}%
\end{pgfscope}%
\begin{pgfscope}%
\pgfpathrectangle{\pgfqpoint{0.550713in}{0.093599in}}{\pgfqpoint{3.028919in}{1.684778in}}%
\pgfusepath{clip}%
\pgfsetbuttcap%
\pgfsetmiterjoin%
\definecolor{currentfill}{rgb}{0.000000,0.000000,0.000000}%
\pgfsetfillcolor{currentfill}%
\pgfsetlinewidth{0.376406pt}%
\definecolor{currentstroke}{rgb}{0.000000,0.000000,0.000000}%
\pgfsetstrokecolor{currentstroke}%
\pgfsetdash{}{0pt}%
\pgfpathmoveto{\pgfqpoint{0.740020in}{0.093599in}}%
\pgfpathlineto{\pgfqpoint{0.740020in}{0.093599in}}%
\pgfpathlineto{\pgfqpoint{0.740020in}{0.093599in}}%
\pgfpathlineto{\pgfqpoint{0.740020in}{0.093599in}}%
\pgfpathclose%
\pgfusepath{stroke,fill}%
\end{pgfscope}%
\begin{pgfscope}%
\pgfpathrectangle{\pgfqpoint{0.550713in}{0.093599in}}{\pgfqpoint{3.028919in}{1.684778in}}%
\pgfusepath{clip}%
\pgfsetbuttcap%
\pgfsetmiterjoin%
\definecolor{currentfill}{rgb}{0.813235,0.819118,0.822059}%
\pgfsetfillcolor{currentfill}%
\pgfsetlinewidth{0.376406pt}%
\definecolor{currentstroke}{rgb}{0.000000,0.000000,0.000000}%
\pgfsetstrokecolor{currentstroke}%
\pgfsetdash{}{0pt}%
\pgfpathmoveto{\pgfqpoint{0.740020in}{0.093599in}}%
\pgfpathlineto{\pgfqpoint{0.740020in}{0.093599in}}%
\pgfpathlineto{\pgfqpoint{0.740020in}{0.093599in}}%
\pgfpathlineto{\pgfqpoint{0.740020in}{0.093599in}}%
\pgfpathclose%
\pgfusepath{stroke,fill}%
\end{pgfscope}%
\begin{pgfscope}%
\pgfsetbuttcap%
\pgfsetroundjoin%
\definecolor{currentfill}{rgb}{0.000000,0.000000,0.000000}%
\pgfsetfillcolor{currentfill}%
\pgfsetlinewidth{0.803000pt}%
\definecolor{currentstroke}{rgb}{0.000000,0.000000,0.000000}%
\pgfsetstrokecolor{currentstroke}%
\pgfsetdash{}{0pt}%
\pgfsys@defobject{currentmarker}{\pgfqpoint{-0.048611in}{0.000000in}}{\pgfqpoint{-0.000000in}{0.000000in}}{%
\pgfpathmoveto{\pgfqpoint{-0.000000in}{0.000000in}}%
\pgfpathlineto{\pgfqpoint{-0.048611in}{0.000000in}}%
\pgfusepath{stroke,fill}%
}%
\begin{pgfscope}%
\pgfsys@transformshift{0.550713in}{0.093599in}%
\pgfsys@useobject{currentmarker}{}%
\end{pgfscope}%
\end{pgfscope}%
\begin{pgfscope}%
\definecolor{textcolor}{rgb}{0.000000,0.000000,0.000000}%
\pgfsetstrokecolor{textcolor}%
\pgfsetfillcolor{textcolor}%
\pgftext[x=0.384046in, y=0.045404in, left, base]{\color{textcolor}\rmfamily\fontsize{10.000000}{12.000000}\selectfont \(\displaystyle {0}\)}%
\end{pgfscope}%
\begin{pgfscope}%
\pgfsetbuttcap%
\pgfsetroundjoin%
\definecolor{currentfill}{rgb}{0.000000,0.000000,0.000000}%
\pgfsetfillcolor{currentfill}%
\pgfsetlinewidth{0.803000pt}%
\definecolor{currentstroke}{rgb}{0.000000,0.000000,0.000000}%
\pgfsetstrokecolor{currentstroke}%
\pgfsetdash{}{0pt}%
\pgfsys@defobject{currentmarker}{\pgfqpoint{-0.048611in}{0.000000in}}{\pgfqpoint{-0.000000in}{0.000000in}}{%
\pgfpathmoveto{\pgfqpoint{-0.000000in}{0.000000in}}%
\pgfpathlineto{\pgfqpoint{-0.048611in}{0.000000in}}%
\pgfusepath{stroke,fill}%
}%
\begin{pgfscope}%
\pgfsys@transformshift{0.550713in}{0.430554in}%
\pgfsys@useobject{currentmarker}{}%
\end{pgfscope}%
\end{pgfscope}%
\begin{pgfscope}%
\definecolor{textcolor}{rgb}{0.000000,0.000000,0.000000}%
\pgfsetstrokecolor{textcolor}%
\pgfsetfillcolor{textcolor}%
\pgftext[x=0.314601in, y=0.382360in, left, base]{\color{textcolor}\rmfamily\fontsize{10.000000}{12.000000}\selectfont \(\displaystyle {20}\)}%
\end{pgfscope}%
\begin{pgfscope}%
\pgfsetbuttcap%
\pgfsetroundjoin%
\definecolor{currentfill}{rgb}{0.000000,0.000000,0.000000}%
\pgfsetfillcolor{currentfill}%
\pgfsetlinewidth{0.803000pt}%
\definecolor{currentstroke}{rgb}{0.000000,0.000000,0.000000}%
\pgfsetstrokecolor{currentstroke}%
\pgfsetdash{}{0pt}%
\pgfsys@defobject{currentmarker}{\pgfqpoint{-0.048611in}{0.000000in}}{\pgfqpoint{-0.000000in}{0.000000in}}{%
\pgfpathmoveto{\pgfqpoint{-0.000000in}{0.000000in}}%
\pgfpathlineto{\pgfqpoint{-0.048611in}{0.000000in}}%
\pgfusepath{stroke,fill}%
}%
\begin{pgfscope}%
\pgfsys@transformshift{0.550713in}{0.767510in}%
\pgfsys@useobject{currentmarker}{}%
\end{pgfscope}%
\end{pgfscope}%
\begin{pgfscope}%
\definecolor{textcolor}{rgb}{0.000000,0.000000,0.000000}%
\pgfsetstrokecolor{textcolor}%
\pgfsetfillcolor{textcolor}%
\pgftext[x=0.314601in, y=0.719315in, left, base]{\color{textcolor}\rmfamily\fontsize{10.000000}{12.000000}\selectfont \(\displaystyle {40}\)}%
\end{pgfscope}%
\begin{pgfscope}%
\pgfsetbuttcap%
\pgfsetroundjoin%
\definecolor{currentfill}{rgb}{0.000000,0.000000,0.000000}%
\pgfsetfillcolor{currentfill}%
\pgfsetlinewidth{0.803000pt}%
\definecolor{currentstroke}{rgb}{0.000000,0.000000,0.000000}%
\pgfsetstrokecolor{currentstroke}%
\pgfsetdash{}{0pt}%
\pgfsys@defobject{currentmarker}{\pgfqpoint{-0.048611in}{0.000000in}}{\pgfqpoint{-0.000000in}{0.000000in}}{%
\pgfpathmoveto{\pgfqpoint{-0.000000in}{0.000000in}}%
\pgfpathlineto{\pgfqpoint{-0.048611in}{0.000000in}}%
\pgfusepath{stroke,fill}%
}%
\begin{pgfscope}%
\pgfsys@transformshift{0.550713in}{1.104465in}%
\pgfsys@useobject{currentmarker}{}%
\end{pgfscope}%
\end{pgfscope}%
\begin{pgfscope}%
\definecolor{textcolor}{rgb}{0.000000,0.000000,0.000000}%
\pgfsetstrokecolor{textcolor}%
\pgfsetfillcolor{textcolor}%
\pgftext[x=0.314601in, y=1.056271in, left, base]{\color{textcolor}\rmfamily\fontsize{10.000000}{12.000000}\selectfont \(\displaystyle {60}\)}%
\end{pgfscope}%
\begin{pgfscope}%
\pgfsetbuttcap%
\pgfsetroundjoin%
\definecolor{currentfill}{rgb}{0.000000,0.000000,0.000000}%
\pgfsetfillcolor{currentfill}%
\pgfsetlinewidth{0.803000pt}%
\definecolor{currentstroke}{rgb}{0.000000,0.000000,0.000000}%
\pgfsetstrokecolor{currentstroke}%
\pgfsetdash{}{0pt}%
\pgfsys@defobject{currentmarker}{\pgfqpoint{-0.048611in}{0.000000in}}{\pgfqpoint{-0.000000in}{0.000000in}}{%
\pgfpathmoveto{\pgfqpoint{-0.000000in}{0.000000in}}%
\pgfpathlineto{\pgfqpoint{-0.048611in}{0.000000in}}%
\pgfusepath{stroke,fill}%
}%
\begin{pgfscope}%
\pgfsys@transformshift{0.550713in}{1.441421in}%
\pgfsys@useobject{currentmarker}{}%
\end{pgfscope}%
\end{pgfscope}%
\begin{pgfscope}%
\definecolor{textcolor}{rgb}{0.000000,0.000000,0.000000}%
\pgfsetstrokecolor{textcolor}%
\pgfsetfillcolor{textcolor}%
\pgftext[x=0.314601in, y=1.393226in, left, base]{\color{textcolor}\rmfamily\fontsize{10.000000}{12.000000}\selectfont \(\displaystyle {80}\)}%
\end{pgfscope}%
\begin{pgfscope}%
\pgfsetbuttcap%
\pgfsetroundjoin%
\definecolor{currentfill}{rgb}{0.000000,0.000000,0.000000}%
\pgfsetfillcolor{currentfill}%
\pgfsetlinewidth{0.803000pt}%
\definecolor{currentstroke}{rgb}{0.000000,0.000000,0.000000}%
\pgfsetstrokecolor{currentstroke}%
\pgfsetdash{}{0pt}%
\pgfsys@defobject{currentmarker}{\pgfqpoint{-0.048611in}{0.000000in}}{\pgfqpoint{-0.000000in}{0.000000in}}{%
\pgfpathmoveto{\pgfqpoint{-0.000000in}{0.000000in}}%
\pgfpathlineto{\pgfqpoint{-0.048611in}{0.000000in}}%
\pgfusepath{stroke,fill}%
}%
\begin{pgfscope}%
\pgfsys@transformshift{0.550713in}{1.778376in}%
\pgfsys@useobject{currentmarker}{}%
\end{pgfscope}%
\end{pgfscope}%
\begin{pgfscope}%
\definecolor{textcolor}{rgb}{0.000000,0.000000,0.000000}%
\pgfsetstrokecolor{textcolor}%
\pgfsetfillcolor{textcolor}%
\pgftext[x=0.245156in, y=1.730182in, left, base]{\color{textcolor}\rmfamily\fontsize{10.000000}{12.000000}\selectfont \(\displaystyle {100}\)}%
\end{pgfscope}%
\begin{pgfscope}%
\definecolor{textcolor}{rgb}{0.000000,0.000000,0.000000}%
\pgfsetstrokecolor{textcolor}%
\pgfsetfillcolor{textcolor}%
\pgftext[x=0.189601in,y=0.935988in,,bottom,rotate=90.000000]{\color{textcolor}\rmfamily\fontsize{10.000000}{12.000000}\selectfont \(\displaystyle R_T\)}%
\end{pgfscope}%
\begin{pgfscope}%
\pgfpathrectangle{\pgfqpoint{0.550713in}{0.093599in}}{\pgfqpoint{3.028919in}{1.684778in}}%
\pgfusepath{clip}%
\pgfsetbuttcap%
\pgfsetroundjoin%
\pgfsetlinewidth{0.501875pt}%
\definecolor{currentstroke}{rgb}{0.392157,0.396078,0.403922}%
\pgfsetstrokecolor{currentstroke}%
\pgfsetdash{}{0pt}%
\pgfpathmoveto{\pgfqpoint{0.929328in}{0.093599in}}%
\pgfpathlineto{\pgfqpoint{0.929328in}{1.778376in}}%
\pgfusepath{stroke}%
\end{pgfscope}%
\begin{pgfscope}%
\pgfpathrectangle{\pgfqpoint{0.550713in}{0.093599in}}{\pgfqpoint{3.028919in}{1.684778in}}%
\pgfusepath{clip}%
\pgfsetbuttcap%
\pgfsetroundjoin%
\pgfsetlinewidth{0.501875pt}%
\definecolor{currentstroke}{rgb}{0.392157,0.396078,0.403922}%
\pgfsetstrokecolor{currentstroke}%
\pgfsetdash{}{0pt}%
\pgfpathmoveto{\pgfqpoint{1.307942in}{0.093599in}}%
\pgfpathlineto{\pgfqpoint{1.307942in}{1.778376in}}%
\pgfusepath{stroke}%
\end{pgfscope}%
\begin{pgfscope}%
\pgfpathrectangle{\pgfqpoint{0.550713in}{0.093599in}}{\pgfqpoint{3.028919in}{1.684778in}}%
\pgfusepath{clip}%
\pgfsetbuttcap%
\pgfsetroundjoin%
\pgfsetlinewidth{0.501875pt}%
\definecolor{currentstroke}{rgb}{0.392157,0.396078,0.403922}%
\pgfsetstrokecolor{currentstroke}%
\pgfsetdash{}{0pt}%
\pgfpathmoveto{\pgfqpoint{1.686557in}{0.093599in}}%
\pgfpathlineto{\pgfqpoint{1.686557in}{1.778376in}}%
\pgfusepath{stroke}%
\end{pgfscope}%
\begin{pgfscope}%
\pgfpathrectangle{\pgfqpoint{0.550713in}{0.093599in}}{\pgfqpoint{3.028919in}{1.684778in}}%
\pgfusepath{clip}%
\pgfsetbuttcap%
\pgfsetroundjoin%
\pgfsetlinewidth{0.501875pt}%
\definecolor{currentstroke}{rgb}{0.392157,0.396078,0.403922}%
\pgfsetstrokecolor{currentstroke}%
\pgfsetdash{}{0pt}%
\pgfpathmoveto{\pgfqpoint{2.065172in}{0.093599in}}%
\pgfpathlineto{\pgfqpoint{2.065172in}{1.778376in}}%
\pgfusepath{stroke}%
\end{pgfscope}%
\begin{pgfscope}%
\pgfpathrectangle{\pgfqpoint{0.550713in}{0.093599in}}{\pgfqpoint{3.028919in}{1.684778in}}%
\pgfusepath{clip}%
\pgfsetbuttcap%
\pgfsetroundjoin%
\pgfsetlinewidth{0.501875pt}%
\definecolor{currentstroke}{rgb}{0.392157,0.396078,0.403922}%
\pgfsetstrokecolor{currentstroke}%
\pgfsetdash{}{0pt}%
\pgfpathmoveto{\pgfqpoint{2.443787in}{0.093599in}}%
\pgfpathlineto{\pgfqpoint{2.443787in}{1.778376in}}%
\pgfusepath{stroke}%
\end{pgfscope}%
\begin{pgfscope}%
\pgfpathrectangle{\pgfqpoint{0.550713in}{0.093599in}}{\pgfqpoint{3.028919in}{1.684778in}}%
\pgfusepath{clip}%
\pgfsetbuttcap%
\pgfsetroundjoin%
\pgfsetlinewidth{0.501875pt}%
\definecolor{currentstroke}{rgb}{0.392157,0.396078,0.403922}%
\pgfsetstrokecolor{currentstroke}%
\pgfsetdash{}{0pt}%
\pgfpathmoveto{\pgfqpoint{2.822402in}{0.093599in}}%
\pgfpathlineto{\pgfqpoint{2.822402in}{1.778376in}}%
\pgfusepath{stroke}%
\end{pgfscope}%
\begin{pgfscope}%
\pgfpathrectangle{\pgfqpoint{0.550713in}{0.093599in}}{\pgfqpoint{3.028919in}{1.684778in}}%
\pgfusepath{clip}%
\pgfsetbuttcap%
\pgfsetroundjoin%
\pgfsetlinewidth{0.501875pt}%
\definecolor{currentstroke}{rgb}{0.392157,0.396078,0.403922}%
\pgfsetstrokecolor{currentstroke}%
\pgfsetdash{}{0pt}%
\pgfpathmoveto{\pgfqpoint{3.201017in}{0.093599in}}%
\pgfpathlineto{\pgfqpoint{3.201017in}{1.778376in}}%
\pgfusepath{stroke}%
\end{pgfscope}%
\begin{pgfscope}%
\pgfpathrectangle{\pgfqpoint{0.550713in}{0.093599in}}{\pgfqpoint{3.028919in}{1.684778in}}%
\pgfusepath{clip}%
\pgfsetbuttcap%
\pgfsetroundjoin%
\pgfsetlinewidth{0.853187pt}%
\definecolor{currentstroke}{rgb}{0.392157,0.396078,0.403922}%
\pgfsetstrokecolor{currentstroke}%
\pgfsetdash{{3.145000pt}{1.360000pt}}{0.000000pt}%
\pgfpathmoveto{\pgfqpoint{0.540713in}{1.070951in}}%
\pgfpathlineto{\pgfqpoint{3.589632in}{1.070951in}}%
\pgfusepath{stroke}%
\end{pgfscope}%
\begin{pgfscope}%
\pgfpathrectangle{\pgfqpoint{0.550713in}{0.093599in}}{\pgfqpoint{3.028919in}{1.684778in}}%
\pgfusepath{clip}%
\pgfsetrectcap%
\pgfsetroundjoin%
\pgfsetlinewidth{0.752812pt}%
\definecolor{currentstroke}{rgb}{0.000000,0.000000,0.000000}%
\pgfsetstrokecolor{currentstroke}%
\pgfsetdash{}{0pt}%
\pgfpathmoveto{\pgfqpoint{0.664297in}{0.537288in}}%
\pgfpathlineto{\pgfqpoint{0.664297in}{0.435933in}}%
\pgfusepath{stroke}%
\end{pgfscope}%
\begin{pgfscope}%
\pgfpathrectangle{\pgfqpoint{0.550713in}{0.093599in}}{\pgfqpoint{3.028919in}{1.684778in}}%
\pgfusepath{clip}%
\pgfsetrectcap%
\pgfsetroundjoin%
\pgfsetlinewidth{0.752812pt}%
\definecolor{currentstroke}{rgb}{0.000000,0.000000,0.000000}%
\pgfsetstrokecolor{currentstroke}%
\pgfsetdash{}{0pt}%
\pgfpathmoveto{\pgfqpoint{0.664297in}{0.621337in}}%
\pgfpathlineto{\pgfqpoint{0.664297in}{0.637744in}}%
\pgfusepath{stroke}%
\end{pgfscope}%
\begin{pgfscope}%
\pgfpathrectangle{\pgfqpoint{0.550713in}{0.093599in}}{\pgfqpoint{3.028919in}{1.684778in}}%
\pgfusepath{clip}%
\pgfsetrectcap%
\pgfsetroundjoin%
\pgfsetlinewidth{0.752812pt}%
\definecolor{currentstroke}{rgb}{0.000000,0.000000,0.000000}%
\pgfsetstrokecolor{currentstroke}%
\pgfsetdash{}{0pt}%
\pgfpathmoveto{\pgfqpoint{0.627193in}{0.435933in}}%
\pgfpathlineto{\pgfqpoint{0.701401in}{0.435933in}}%
\pgfusepath{stroke}%
\end{pgfscope}%
\begin{pgfscope}%
\pgfpathrectangle{\pgfqpoint{0.550713in}{0.093599in}}{\pgfqpoint{3.028919in}{1.684778in}}%
\pgfusepath{clip}%
\pgfsetrectcap%
\pgfsetroundjoin%
\pgfsetlinewidth{0.752812pt}%
\definecolor{currentstroke}{rgb}{0.000000,0.000000,0.000000}%
\pgfsetstrokecolor{currentstroke}%
\pgfsetdash{}{0pt}%
\pgfpathmoveto{\pgfqpoint{0.627193in}{0.637744in}}%
\pgfpathlineto{\pgfqpoint{0.701401in}{0.637744in}}%
\pgfusepath{stroke}%
\end{pgfscope}%
\begin{pgfscope}%
\pgfpathrectangle{\pgfqpoint{0.550713in}{0.093599in}}{\pgfqpoint{3.028919in}{1.684778in}}%
\pgfusepath{clip}%
\pgfsetbuttcap%
\pgfsetmiterjoin%
\definecolor{currentfill}{rgb}{0.000000,0.000000,0.000000}%
\pgfsetfillcolor{currentfill}%
\pgfsetlinewidth{1.003750pt}%
\definecolor{currentstroke}{rgb}{0.000000,0.000000,0.000000}%
\pgfsetstrokecolor{currentstroke}%
\pgfsetdash{}{0pt}%
\pgfsys@defobject{currentmarker}{\pgfqpoint{-0.011785in}{-0.019642in}}{\pgfqpoint{0.011785in}{0.019642in}}{%
\pgfpathmoveto{\pgfqpoint{-0.000000in}{-0.019642in}}%
\pgfpathlineto{\pgfqpoint{0.011785in}{0.000000in}}%
\pgfpathlineto{\pgfqpoint{0.000000in}{0.019642in}}%
\pgfpathlineto{\pgfqpoint{-0.011785in}{0.000000in}}%
\pgfpathclose%
\pgfusepath{stroke,fill}%
}%
\begin{pgfscope}%
\pgfsys@transformshift{0.664297in}{0.882491in}%
\pgfsys@useobject{currentmarker}{}%
\end{pgfscope}%
\begin{pgfscope}%
\pgfsys@transformshift{0.664297in}{0.859096in}%
\pgfsys@useobject{currentmarker}{}%
\end{pgfscope}%
\begin{pgfscope}%
\pgfsys@transformshift{0.664297in}{0.852502in}%
\pgfsys@useobject{currentmarker}{}%
\end{pgfscope}%
\begin{pgfscope}%
\pgfsys@transformshift{0.664297in}{0.939574in}%
\pgfsys@useobject{currentmarker}{}%
\end{pgfscope}%
\begin{pgfscope}%
\pgfsys@transformshift{0.664297in}{0.916032in}%
\pgfsys@useobject{currentmarker}{}%
\end{pgfscope}%
\end{pgfscope}%
\begin{pgfscope}%
\pgfpathrectangle{\pgfqpoint{0.550713in}{0.093599in}}{\pgfqpoint{3.028919in}{1.684778in}}%
\pgfusepath{clip}%
\pgfsetrectcap%
\pgfsetroundjoin%
\pgfsetlinewidth{0.752812pt}%
\definecolor{currentstroke}{rgb}{0.000000,0.000000,0.000000}%
\pgfsetstrokecolor{currentstroke}%
\pgfsetdash{}{0pt}%
\pgfpathmoveto{\pgfqpoint{0.815743in}{0.810756in}}%
\pgfpathlineto{\pgfqpoint{0.815743in}{0.750928in}}%
\pgfusepath{stroke}%
\end{pgfscope}%
\begin{pgfscope}%
\pgfpathrectangle{\pgfqpoint{0.550713in}{0.093599in}}{\pgfqpoint{3.028919in}{1.684778in}}%
\pgfusepath{clip}%
\pgfsetrectcap%
\pgfsetroundjoin%
\pgfsetlinewidth{0.752812pt}%
\definecolor{currentstroke}{rgb}{0.000000,0.000000,0.000000}%
\pgfsetstrokecolor{currentstroke}%
\pgfsetdash{}{0pt}%
\pgfpathmoveto{\pgfqpoint{0.815743in}{0.909051in}}%
\pgfpathlineto{\pgfqpoint{0.815743in}{0.915796in}}%
\pgfusepath{stroke}%
\end{pgfscope}%
\begin{pgfscope}%
\pgfpathrectangle{\pgfqpoint{0.550713in}{0.093599in}}{\pgfqpoint{3.028919in}{1.684778in}}%
\pgfusepath{clip}%
\pgfsetrectcap%
\pgfsetroundjoin%
\pgfsetlinewidth{0.752812pt}%
\definecolor{currentstroke}{rgb}{0.000000,0.000000,0.000000}%
\pgfsetstrokecolor{currentstroke}%
\pgfsetdash{}{0pt}%
\pgfpathmoveto{\pgfqpoint{0.778639in}{0.750928in}}%
\pgfpathlineto{\pgfqpoint{0.852847in}{0.750928in}}%
\pgfusepath{stroke}%
\end{pgfscope}%
\begin{pgfscope}%
\pgfpathrectangle{\pgfqpoint{0.550713in}{0.093599in}}{\pgfqpoint{3.028919in}{1.684778in}}%
\pgfusepath{clip}%
\pgfsetrectcap%
\pgfsetroundjoin%
\pgfsetlinewidth{0.752812pt}%
\definecolor{currentstroke}{rgb}{0.000000,0.000000,0.000000}%
\pgfsetstrokecolor{currentstroke}%
\pgfsetdash{}{0pt}%
\pgfpathmoveto{\pgfqpoint{0.778639in}{0.915796in}}%
\pgfpathlineto{\pgfqpoint{0.852847in}{0.915796in}}%
\pgfusepath{stroke}%
\end{pgfscope}%
\begin{pgfscope}%
\pgfpathrectangle{\pgfqpoint{0.550713in}{0.093599in}}{\pgfqpoint{3.028919in}{1.684778in}}%
\pgfusepath{clip}%
\pgfsetbuttcap%
\pgfsetmiterjoin%
\definecolor{currentfill}{rgb}{0.000000,0.000000,0.000000}%
\pgfsetfillcolor{currentfill}%
\pgfsetlinewidth{1.003750pt}%
\definecolor{currentstroke}{rgb}{0.000000,0.000000,0.000000}%
\pgfsetstrokecolor{currentstroke}%
\pgfsetdash{}{0pt}%
\pgfsys@defobject{currentmarker}{\pgfqpoint{-0.011785in}{-0.019642in}}{\pgfqpoint{0.011785in}{0.019642in}}{%
\pgfpathmoveto{\pgfqpoint{-0.000000in}{-0.019642in}}%
\pgfpathlineto{\pgfqpoint{0.011785in}{0.000000in}}%
\pgfpathlineto{\pgfqpoint{0.000000in}{0.019642in}}%
\pgfpathlineto{\pgfqpoint{-0.011785in}{0.000000in}}%
\pgfpathclose%
\pgfusepath{stroke,fill}%
}%
\begin{pgfscope}%
\pgfsys@transformshift{0.815743in}{1.145562in}%
\pgfsys@useobject{currentmarker}{}%
\end{pgfscope}%
\begin{pgfscope}%
\pgfsys@transformshift{0.815743in}{1.151191in}%
\pgfsys@useobject{currentmarker}{}%
\end{pgfscope}%
\begin{pgfscope}%
\pgfsys@transformshift{0.815743in}{1.114053in}%
\pgfsys@useobject{currentmarker}{}%
\end{pgfscope}%
\begin{pgfscope}%
\pgfsys@transformshift{0.815743in}{1.178937in}%
\pgfsys@useobject{currentmarker}{}%
\end{pgfscope}%
\begin{pgfscope}%
\pgfsys@transformshift{0.815743in}{1.202074in}%
\pgfsys@useobject{currentmarker}{}%
\end{pgfscope}%
\end{pgfscope}%
\begin{pgfscope}%
\pgfpathrectangle{\pgfqpoint{0.550713in}{0.093599in}}{\pgfqpoint{3.028919in}{1.684778in}}%
\pgfusepath{clip}%
\pgfsetrectcap%
\pgfsetroundjoin%
\pgfsetlinewidth{0.752812pt}%
\definecolor{currentstroke}{rgb}{0.000000,0.000000,0.000000}%
\pgfsetstrokecolor{currentstroke}%
\pgfsetdash{}{0pt}%
\pgfpathmoveto{\pgfqpoint{1.042912in}{0.619040in}}%
\pgfpathlineto{\pgfqpoint{1.042912in}{0.522475in}}%
\pgfusepath{stroke}%
\end{pgfscope}%
\begin{pgfscope}%
\pgfpathrectangle{\pgfqpoint{0.550713in}{0.093599in}}{\pgfqpoint{3.028919in}{1.684778in}}%
\pgfusepath{clip}%
\pgfsetrectcap%
\pgfsetroundjoin%
\pgfsetlinewidth{0.752812pt}%
\definecolor{currentstroke}{rgb}{0.000000,0.000000,0.000000}%
\pgfsetstrokecolor{currentstroke}%
\pgfsetdash{}{0pt}%
\pgfpathmoveto{\pgfqpoint{1.042912in}{0.828417in}}%
\pgfpathlineto{\pgfqpoint{1.042912in}{0.865730in}}%
\pgfusepath{stroke}%
\end{pgfscope}%
\begin{pgfscope}%
\pgfpathrectangle{\pgfqpoint{0.550713in}{0.093599in}}{\pgfqpoint{3.028919in}{1.684778in}}%
\pgfusepath{clip}%
\pgfsetrectcap%
\pgfsetroundjoin%
\pgfsetlinewidth{0.752812pt}%
\definecolor{currentstroke}{rgb}{0.000000,0.000000,0.000000}%
\pgfsetstrokecolor{currentstroke}%
\pgfsetdash{}{0pt}%
\pgfpathmoveto{\pgfqpoint{1.005808in}{0.522475in}}%
\pgfpathlineto{\pgfqpoint{1.080016in}{0.522475in}}%
\pgfusepath{stroke}%
\end{pgfscope}%
\begin{pgfscope}%
\pgfpathrectangle{\pgfqpoint{0.550713in}{0.093599in}}{\pgfqpoint{3.028919in}{1.684778in}}%
\pgfusepath{clip}%
\pgfsetrectcap%
\pgfsetroundjoin%
\pgfsetlinewidth{0.752812pt}%
\definecolor{currentstroke}{rgb}{0.000000,0.000000,0.000000}%
\pgfsetstrokecolor{currentstroke}%
\pgfsetdash{}{0pt}%
\pgfpathmoveto{\pgfqpoint{1.005808in}{0.865730in}}%
\pgfpathlineto{\pgfqpoint{1.080016in}{0.865730in}}%
\pgfusepath{stroke}%
\end{pgfscope}%
\begin{pgfscope}%
\pgfpathrectangle{\pgfqpoint{0.550713in}{0.093599in}}{\pgfqpoint{3.028919in}{1.684778in}}%
\pgfusepath{clip}%
\pgfsetbuttcap%
\pgfsetmiterjoin%
\definecolor{currentfill}{rgb}{0.000000,0.000000,0.000000}%
\pgfsetfillcolor{currentfill}%
\pgfsetlinewidth{1.003750pt}%
\definecolor{currentstroke}{rgb}{0.000000,0.000000,0.000000}%
\pgfsetstrokecolor{currentstroke}%
\pgfsetdash{}{0pt}%
\pgfsys@defobject{currentmarker}{\pgfqpoint{-0.011785in}{-0.019642in}}{\pgfqpoint{0.011785in}{0.019642in}}{%
\pgfpathmoveto{\pgfqpoint{-0.000000in}{-0.019642in}}%
\pgfpathlineto{\pgfqpoint{0.011785in}{0.000000in}}%
\pgfpathlineto{\pgfqpoint{0.000000in}{0.019642in}}%
\pgfpathlineto{\pgfqpoint{-0.011785in}{0.000000in}}%
\pgfpathclose%
\pgfusepath{stroke,fill}%
}%
\begin{pgfscope}%
\pgfsys@transformshift{1.042912in}{1.331643in}%
\pgfsys@useobject{currentmarker}{}%
\end{pgfscope}%
\begin{pgfscope}%
\pgfsys@transformshift{1.042912in}{1.460111in}%
\pgfsys@useobject{currentmarker}{}%
\end{pgfscope}%
\begin{pgfscope}%
\pgfsys@transformshift{1.042912in}{1.377941in}%
\pgfsys@useobject{currentmarker}{}%
\end{pgfscope}%
\begin{pgfscope}%
\pgfsys@transformshift{1.042912in}{1.465722in}%
\pgfsys@useobject{currentmarker}{}%
\end{pgfscope}%
\begin{pgfscope}%
\pgfsys@transformshift{1.042912in}{1.277351in}%
\pgfsys@useobject{currentmarker}{}%
\end{pgfscope}%
\end{pgfscope}%
\begin{pgfscope}%
\pgfpathrectangle{\pgfqpoint{0.550713in}{0.093599in}}{\pgfqpoint{3.028919in}{1.684778in}}%
\pgfusepath{clip}%
\pgfsetrectcap%
\pgfsetroundjoin%
\pgfsetlinewidth{0.752812pt}%
\definecolor{currentstroke}{rgb}{0.000000,0.000000,0.000000}%
\pgfsetstrokecolor{currentstroke}%
\pgfsetdash{}{0pt}%
\pgfpathmoveto{\pgfqpoint{1.194358in}{1.173757in}}%
\pgfpathlineto{\pgfqpoint{1.194358in}{1.082516in}}%
\pgfusepath{stroke}%
\end{pgfscope}%
\begin{pgfscope}%
\pgfpathrectangle{\pgfqpoint{0.550713in}{0.093599in}}{\pgfqpoint{3.028919in}{1.684778in}}%
\pgfusepath{clip}%
\pgfsetrectcap%
\pgfsetroundjoin%
\pgfsetlinewidth{0.752812pt}%
\definecolor{currentstroke}{rgb}{0.000000,0.000000,0.000000}%
\pgfsetstrokecolor{currentstroke}%
\pgfsetdash{}{0pt}%
\pgfpathmoveto{\pgfqpoint{1.194358in}{1.512378in}}%
\pgfpathlineto{\pgfqpoint{1.194358in}{1.619502in}}%
\pgfusepath{stroke}%
\end{pgfscope}%
\begin{pgfscope}%
\pgfpathrectangle{\pgfqpoint{0.550713in}{0.093599in}}{\pgfqpoint{3.028919in}{1.684778in}}%
\pgfusepath{clip}%
\pgfsetrectcap%
\pgfsetroundjoin%
\pgfsetlinewidth{0.752812pt}%
\definecolor{currentstroke}{rgb}{0.000000,0.000000,0.000000}%
\pgfsetstrokecolor{currentstroke}%
\pgfsetdash{}{0pt}%
\pgfpathmoveto{\pgfqpoint{1.157254in}{1.082516in}}%
\pgfpathlineto{\pgfqpoint{1.231462in}{1.082516in}}%
\pgfusepath{stroke}%
\end{pgfscope}%
\begin{pgfscope}%
\pgfpathrectangle{\pgfqpoint{0.550713in}{0.093599in}}{\pgfqpoint{3.028919in}{1.684778in}}%
\pgfusepath{clip}%
\pgfsetrectcap%
\pgfsetroundjoin%
\pgfsetlinewidth{0.752812pt}%
\definecolor{currentstroke}{rgb}{0.000000,0.000000,0.000000}%
\pgfsetstrokecolor{currentstroke}%
\pgfsetdash{}{0pt}%
\pgfpathmoveto{\pgfqpoint{1.157254in}{1.619502in}}%
\pgfpathlineto{\pgfqpoint{1.231462in}{1.619502in}}%
\pgfusepath{stroke}%
\end{pgfscope}%
\begin{pgfscope}%
\pgfpathrectangle{\pgfqpoint{0.550713in}{0.093599in}}{\pgfqpoint{3.028919in}{1.684778in}}%
\pgfusepath{clip}%
\pgfsetrectcap%
\pgfsetroundjoin%
\pgfsetlinewidth{0.752812pt}%
\definecolor{currentstroke}{rgb}{0.000000,0.000000,0.000000}%
\pgfsetstrokecolor{currentstroke}%
\pgfsetdash{}{0pt}%
\pgfpathmoveto{\pgfqpoint{1.421527in}{0.548173in}}%
\pgfpathlineto{\pgfqpoint{1.421527in}{0.468046in}}%
\pgfusepath{stroke}%
\end{pgfscope}%
\begin{pgfscope}%
\pgfpathrectangle{\pgfqpoint{0.550713in}{0.093599in}}{\pgfqpoint{3.028919in}{1.684778in}}%
\pgfusepath{clip}%
\pgfsetrectcap%
\pgfsetroundjoin%
\pgfsetlinewidth{0.752812pt}%
\definecolor{currentstroke}{rgb}{0.000000,0.000000,0.000000}%
\pgfsetstrokecolor{currentstroke}%
\pgfsetdash{}{0pt}%
\pgfpathmoveto{\pgfqpoint{1.421527in}{0.643247in}}%
\pgfpathlineto{\pgfqpoint{1.421527in}{0.735351in}}%
\pgfusepath{stroke}%
\end{pgfscope}%
\begin{pgfscope}%
\pgfpathrectangle{\pgfqpoint{0.550713in}{0.093599in}}{\pgfqpoint{3.028919in}{1.684778in}}%
\pgfusepath{clip}%
\pgfsetrectcap%
\pgfsetroundjoin%
\pgfsetlinewidth{0.752812pt}%
\definecolor{currentstroke}{rgb}{0.000000,0.000000,0.000000}%
\pgfsetstrokecolor{currentstroke}%
\pgfsetdash{}{0pt}%
\pgfpathmoveto{\pgfqpoint{1.384423in}{0.468046in}}%
\pgfpathlineto{\pgfqpoint{1.458631in}{0.468046in}}%
\pgfusepath{stroke}%
\end{pgfscope}%
\begin{pgfscope}%
\pgfpathrectangle{\pgfqpoint{0.550713in}{0.093599in}}{\pgfqpoint{3.028919in}{1.684778in}}%
\pgfusepath{clip}%
\pgfsetrectcap%
\pgfsetroundjoin%
\pgfsetlinewidth{0.752812pt}%
\definecolor{currentstroke}{rgb}{0.000000,0.000000,0.000000}%
\pgfsetstrokecolor{currentstroke}%
\pgfsetdash{}{0pt}%
\pgfpathmoveto{\pgfqpoint{1.384423in}{0.735351in}}%
\pgfpathlineto{\pgfqpoint{1.458631in}{0.735351in}}%
\pgfusepath{stroke}%
\end{pgfscope}%
\begin{pgfscope}%
\pgfpathrectangle{\pgfqpoint{0.550713in}{0.093599in}}{\pgfqpoint{3.028919in}{1.684778in}}%
\pgfusepath{clip}%
\pgfsetbuttcap%
\pgfsetmiterjoin%
\definecolor{currentfill}{rgb}{0.000000,0.000000,0.000000}%
\pgfsetfillcolor{currentfill}%
\pgfsetlinewidth{1.003750pt}%
\definecolor{currentstroke}{rgb}{0.000000,0.000000,0.000000}%
\pgfsetstrokecolor{currentstroke}%
\pgfsetdash{}{0pt}%
\pgfsys@defobject{currentmarker}{\pgfqpoint{-0.011785in}{-0.019642in}}{\pgfqpoint{0.011785in}{0.019642in}}{%
\pgfpathmoveto{\pgfqpoint{-0.000000in}{-0.019642in}}%
\pgfpathlineto{\pgfqpoint{0.011785in}{0.000000in}}%
\pgfpathlineto{\pgfqpoint{0.000000in}{0.019642in}}%
\pgfpathlineto{\pgfqpoint{-0.011785in}{0.000000in}}%
\pgfpathclose%
\pgfusepath{stroke,fill}%
}%
\begin{pgfscope}%
\pgfsys@transformshift{1.421527in}{0.336493in}%
\pgfsys@useobject{currentmarker}{}%
\end{pgfscope}%
\begin{pgfscope}%
\pgfsys@transformshift{1.421527in}{0.798642in}%
\pgfsys@useobject{currentmarker}{}%
\end{pgfscope}%
\begin{pgfscope}%
\pgfsys@transformshift{1.421527in}{0.815949in}%
\pgfsys@useobject{currentmarker}{}%
\end{pgfscope}%
\end{pgfscope}%
\begin{pgfscope}%
\pgfpathrectangle{\pgfqpoint{0.550713in}{0.093599in}}{\pgfqpoint{3.028919in}{1.684778in}}%
\pgfusepath{clip}%
\pgfsetrectcap%
\pgfsetroundjoin%
\pgfsetlinewidth{0.752812pt}%
\definecolor{currentstroke}{rgb}{0.000000,0.000000,0.000000}%
\pgfsetstrokecolor{currentstroke}%
\pgfsetdash{}{0pt}%
\pgfpathmoveto{\pgfqpoint{1.572973in}{0.574583in}}%
\pgfpathlineto{\pgfqpoint{1.572973in}{0.395210in}}%
\pgfusepath{stroke}%
\end{pgfscope}%
\begin{pgfscope}%
\pgfpathrectangle{\pgfqpoint{0.550713in}{0.093599in}}{\pgfqpoint{3.028919in}{1.684778in}}%
\pgfusepath{clip}%
\pgfsetrectcap%
\pgfsetroundjoin%
\pgfsetlinewidth{0.752812pt}%
\definecolor{currentstroke}{rgb}{0.000000,0.000000,0.000000}%
\pgfsetstrokecolor{currentstroke}%
\pgfsetdash{}{0pt}%
\pgfpathmoveto{\pgfqpoint{1.572973in}{0.718210in}}%
\pgfpathlineto{\pgfqpoint{1.572973in}{0.832434in}}%
\pgfusepath{stroke}%
\end{pgfscope}%
\begin{pgfscope}%
\pgfpathrectangle{\pgfqpoint{0.550713in}{0.093599in}}{\pgfqpoint{3.028919in}{1.684778in}}%
\pgfusepath{clip}%
\pgfsetrectcap%
\pgfsetroundjoin%
\pgfsetlinewidth{0.752812pt}%
\definecolor{currentstroke}{rgb}{0.000000,0.000000,0.000000}%
\pgfsetstrokecolor{currentstroke}%
\pgfsetdash{}{0pt}%
\pgfpathmoveto{\pgfqpoint{1.535869in}{0.395210in}}%
\pgfpathlineto{\pgfqpoint{1.610077in}{0.395210in}}%
\pgfusepath{stroke}%
\end{pgfscope}%
\begin{pgfscope}%
\pgfpathrectangle{\pgfqpoint{0.550713in}{0.093599in}}{\pgfqpoint{3.028919in}{1.684778in}}%
\pgfusepath{clip}%
\pgfsetrectcap%
\pgfsetroundjoin%
\pgfsetlinewidth{0.752812pt}%
\definecolor{currentstroke}{rgb}{0.000000,0.000000,0.000000}%
\pgfsetstrokecolor{currentstroke}%
\pgfsetdash{}{0pt}%
\pgfpathmoveto{\pgfqpoint{1.535869in}{0.832434in}}%
\pgfpathlineto{\pgfqpoint{1.610077in}{0.832434in}}%
\pgfusepath{stroke}%
\end{pgfscope}%
\begin{pgfscope}%
\pgfpathrectangle{\pgfqpoint{0.550713in}{0.093599in}}{\pgfqpoint{3.028919in}{1.684778in}}%
\pgfusepath{clip}%
\pgfsetbuttcap%
\pgfsetmiterjoin%
\definecolor{currentfill}{rgb}{0.000000,0.000000,0.000000}%
\pgfsetfillcolor{currentfill}%
\pgfsetlinewidth{1.003750pt}%
\definecolor{currentstroke}{rgb}{0.000000,0.000000,0.000000}%
\pgfsetstrokecolor{currentstroke}%
\pgfsetdash{}{0pt}%
\pgfsys@defobject{currentmarker}{\pgfqpoint{-0.011785in}{-0.019642in}}{\pgfqpoint{0.011785in}{0.019642in}}{%
\pgfpathmoveto{\pgfqpoint{-0.000000in}{-0.019642in}}%
\pgfpathlineto{\pgfqpoint{0.011785in}{0.000000in}}%
\pgfpathlineto{\pgfqpoint{0.000000in}{0.019642in}}%
\pgfpathlineto{\pgfqpoint{-0.011785in}{0.000000in}}%
\pgfpathclose%
\pgfusepath{stroke,fill}%
}%
\begin{pgfscope}%
\pgfsys@transformshift{1.572973in}{0.325818in}%
\pgfsys@useobject{currentmarker}{}%
\end{pgfscope}%
\end{pgfscope}%
\begin{pgfscope}%
\pgfpathrectangle{\pgfqpoint{0.550713in}{0.093599in}}{\pgfqpoint{3.028919in}{1.684778in}}%
\pgfusepath{clip}%
\pgfsetrectcap%
\pgfsetroundjoin%
\pgfsetlinewidth{0.752812pt}%
\definecolor{currentstroke}{rgb}{0.000000,0.000000,0.000000}%
\pgfsetstrokecolor{currentstroke}%
\pgfsetdash{}{0pt}%
\pgfpathmoveto{\pgfqpoint{1.800142in}{0.593637in}}%
\pgfpathlineto{\pgfqpoint{1.800142in}{0.511975in}}%
\pgfusepath{stroke}%
\end{pgfscope}%
\begin{pgfscope}%
\pgfpathrectangle{\pgfqpoint{0.550713in}{0.093599in}}{\pgfqpoint{3.028919in}{1.684778in}}%
\pgfusepath{clip}%
\pgfsetrectcap%
\pgfsetroundjoin%
\pgfsetlinewidth{0.752812pt}%
\definecolor{currentstroke}{rgb}{0.000000,0.000000,0.000000}%
\pgfsetstrokecolor{currentstroke}%
\pgfsetdash{}{0pt}%
\pgfpathmoveto{\pgfqpoint{1.800142in}{0.712756in}}%
\pgfpathlineto{\pgfqpoint{1.800142in}{0.879630in}}%
\pgfusepath{stroke}%
\end{pgfscope}%
\begin{pgfscope}%
\pgfpathrectangle{\pgfqpoint{0.550713in}{0.093599in}}{\pgfqpoint{3.028919in}{1.684778in}}%
\pgfusepath{clip}%
\pgfsetrectcap%
\pgfsetroundjoin%
\pgfsetlinewidth{0.752812pt}%
\definecolor{currentstroke}{rgb}{0.000000,0.000000,0.000000}%
\pgfsetstrokecolor{currentstroke}%
\pgfsetdash{}{0pt}%
\pgfpathmoveto{\pgfqpoint{1.763038in}{0.511975in}}%
\pgfpathlineto{\pgfqpoint{1.837246in}{0.511975in}}%
\pgfusepath{stroke}%
\end{pgfscope}%
\begin{pgfscope}%
\pgfpathrectangle{\pgfqpoint{0.550713in}{0.093599in}}{\pgfqpoint{3.028919in}{1.684778in}}%
\pgfusepath{clip}%
\pgfsetrectcap%
\pgfsetroundjoin%
\pgfsetlinewidth{0.752812pt}%
\definecolor{currentstroke}{rgb}{0.000000,0.000000,0.000000}%
\pgfsetstrokecolor{currentstroke}%
\pgfsetdash{}{0pt}%
\pgfpathmoveto{\pgfqpoint{1.763038in}{0.879630in}}%
\pgfpathlineto{\pgfqpoint{1.837246in}{0.879630in}}%
\pgfusepath{stroke}%
\end{pgfscope}%
\begin{pgfscope}%
\pgfpathrectangle{\pgfqpoint{0.550713in}{0.093599in}}{\pgfqpoint{3.028919in}{1.684778in}}%
\pgfusepath{clip}%
\pgfsetrectcap%
\pgfsetroundjoin%
\pgfsetlinewidth{0.752812pt}%
\definecolor{currentstroke}{rgb}{0.000000,0.000000,0.000000}%
\pgfsetstrokecolor{currentstroke}%
\pgfsetdash{}{0pt}%
\pgfpathmoveto{\pgfqpoint{1.951588in}{0.621709in}}%
\pgfpathlineto{\pgfqpoint{1.951588in}{0.544694in}}%
\pgfusepath{stroke}%
\end{pgfscope}%
\begin{pgfscope}%
\pgfpathrectangle{\pgfqpoint{0.550713in}{0.093599in}}{\pgfqpoint{3.028919in}{1.684778in}}%
\pgfusepath{clip}%
\pgfsetrectcap%
\pgfsetroundjoin%
\pgfsetlinewidth{0.752812pt}%
\definecolor{currentstroke}{rgb}{0.000000,0.000000,0.000000}%
\pgfsetstrokecolor{currentstroke}%
\pgfsetdash{}{0pt}%
\pgfpathmoveto{\pgfqpoint{1.951588in}{0.774160in}}%
\pgfpathlineto{\pgfqpoint{1.951588in}{0.976732in}}%
\pgfusepath{stroke}%
\end{pgfscope}%
\begin{pgfscope}%
\pgfpathrectangle{\pgfqpoint{0.550713in}{0.093599in}}{\pgfqpoint{3.028919in}{1.684778in}}%
\pgfusepath{clip}%
\pgfsetrectcap%
\pgfsetroundjoin%
\pgfsetlinewidth{0.752812pt}%
\definecolor{currentstroke}{rgb}{0.000000,0.000000,0.000000}%
\pgfsetstrokecolor{currentstroke}%
\pgfsetdash{}{0pt}%
\pgfpathmoveto{\pgfqpoint{1.914484in}{0.544694in}}%
\pgfpathlineto{\pgfqpoint{1.988692in}{0.544694in}}%
\pgfusepath{stroke}%
\end{pgfscope}%
\begin{pgfscope}%
\pgfpathrectangle{\pgfqpoint{0.550713in}{0.093599in}}{\pgfqpoint{3.028919in}{1.684778in}}%
\pgfusepath{clip}%
\pgfsetrectcap%
\pgfsetroundjoin%
\pgfsetlinewidth{0.752812pt}%
\definecolor{currentstroke}{rgb}{0.000000,0.000000,0.000000}%
\pgfsetstrokecolor{currentstroke}%
\pgfsetdash{}{0pt}%
\pgfpathmoveto{\pgfqpoint{1.914484in}{0.976732in}}%
\pgfpathlineto{\pgfqpoint{1.988692in}{0.976732in}}%
\pgfusepath{stroke}%
\end{pgfscope}%
\begin{pgfscope}%
\pgfpathrectangle{\pgfqpoint{0.550713in}{0.093599in}}{\pgfqpoint{3.028919in}{1.684778in}}%
\pgfusepath{clip}%
\pgfsetbuttcap%
\pgfsetmiterjoin%
\definecolor{currentfill}{rgb}{0.000000,0.000000,0.000000}%
\pgfsetfillcolor{currentfill}%
\pgfsetlinewidth{1.003750pt}%
\definecolor{currentstroke}{rgb}{0.000000,0.000000,0.000000}%
\pgfsetstrokecolor{currentstroke}%
\pgfsetdash{}{0pt}%
\pgfsys@defobject{currentmarker}{\pgfqpoint{-0.011785in}{-0.019642in}}{\pgfqpoint{0.011785in}{0.019642in}}{%
\pgfpathmoveto{\pgfqpoint{-0.000000in}{-0.019642in}}%
\pgfpathlineto{\pgfqpoint{0.011785in}{0.000000in}}%
\pgfpathlineto{\pgfqpoint{0.000000in}{0.019642in}}%
\pgfpathlineto{\pgfqpoint{-0.011785in}{0.000000in}}%
\pgfpathclose%
\pgfusepath{stroke,fill}%
}%
\begin{pgfscope}%
\pgfsys@transformshift{1.951588in}{1.126856in}%
\pgfsys@useobject{currentmarker}{}%
\end{pgfscope}%
\end{pgfscope}%
\begin{pgfscope}%
\pgfpathrectangle{\pgfqpoint{0.550713in}{0.093599in}}{\pgfqpoint{3.028919in}{1.684778in}}%
\pgfusepath{clip}%
\pgfsetrectcap%
\pgfsetroundjoin%
\pgfsetlinewidth{0.752812pt}%
\definecolor{currentstroke}{rgb}{0.000000,0.000000,0.000000}%
\pgfsetstrokecolor{currentstroke}%
\pgfsetdash{}{0pt}%
\pgfpathmoveto{\pgfqpoint{2.178757in}{0.614619in}}%
\pgfpathlineto{\pgfqpoint{2.178757in}{0.550795in}}%
\pgfusepath{stroke}%
\end{pgfscope}%
\begin{pgfscope}%
\pgfpathrectangle{\pgfqpoint{0.550713in}{0.093599in}}{\pgfqpoint{3.028919in}{1.684778in}}%
\pgfusepath{clip}%
\pgfsetrectcap%
\pgfsetroundjoin%
\pgfsetlinewidth{0.752812pt}%
\definecolor{currentstroke}{rgb}{0.000000,0.000000,0.000000}%
\pgfsetstrokecolor{currentstroke}%
\pgfsetdash{}{0pt}%
\pgfpathmoveto{\pgfqpoint{2.178757in}{0.678538in}}%
\pgfpathlineto{\pgfqpoint{2.178757in}{0.774092in}}%
\pgfusepath{stroke}%
\end{pgfscope}%
\begin{pgfscope}%
\pgfpathrectangle{\pgfqpoint{0.550713in}{0.093599in}}{\pgfqpoint{3.028919in}{1.684778in}}%
\pgfusepath{clip}%
\pgfsetrectcap%
\pgfsetroundjoin%
\pgfsetlinewidth{0.752812pt}%
\definecolor{currentstroke}{rgb}{0.000000,0.000000,0.000000}%
\pgfsetstrokecolor{currentstroke}%
\pgfsetdash{}{0pt}%
\pgfpathmoveto{\pgfqpoint{2.141652in}{0.550795in}}%
\pgfpathlineto{\pgfqpoint{2.215861in}{0.550795in}}%
\pgfusepath{stroke}%
\end{pgfscope}%
\begin{pgfscope}%
\pgfpathrectangle{\pgfqpoint{0.550713in}{0.093599in}}{\pgfqpoint{3.028919in}{1.684778in}}%
\pgfusepath{clip}%
\pgfsetrectcap%
\pgfsetroundjoin%
\pgfsetlinewidth{0.752812pt}%
\definecolor{currentstroke}{rgb}{0.000000,0.000000,0.000000}%
\pgfsetstrokecolor{currentstroke}%
\pgfsetdash{}{0pt}%
\pgfpathmoveto{\pgfqpoint{2.141652in}{0.774092in}}%
\pgfpathlineto{\pgfqpoint{2.215861in}{0.774092in}}%
\pgfusepath{stroke}%
\end{pgfscope}%
\begin{pgfscope}%
\pgfpathrectangle{\pgfqpoint{0.550713in}{0.093599in}}{\pgfqpoint{3.028919in}{1.684778in}}%
\pgfusepath{clip}%
\pgfsetrectcap%
\pgfsetroundjoin%
\pgfsetlinewidth{0.752812pt}%
\definecolor{currentstroke}{rgb}{0.000000,0.000000,0.000000}%
\pgfsetstrokecolor{currentstroke}%
\pgfsetdash{}{0pt}%
\pgfpathmoveto{\pgfqpoint{2.330203in}{0.733511in}}%
\pgfpathlineto{\pgfqpoint{2.330203in}{0.664620in}}%
\pgfusepath{stroke}%
\end{pgfscope}%
\begin{pgfscope}%
\pgfpathrectangle{\pgfqpoint{0.550713in}{0.093599in}}{\pgfqpoint{3.028919in}{1.684778in}}%
\pgfusepath{clip}%
\pgfsetrectcap%
\pgfsetroundjoin%
\pgfsetlinewidth{0.752812pt}%
\definecolor{currentstroke}{rgb}{0.000000,0.000000,0.000000}%
\pgfsetstrokecolor{currentstroke}%
\pgfsetdash{}{0pt}%
\pgfpathmoveto{\pgfqpoint{2.330203in}{0.842090in}}%
\pgfpathlineto{\pgfqpoint{2.330203in}{0.982515in}}%
\pgfusepath{stroke}%
\end{pgfscope}%
\begin{pgfscope}%
\pgfpathrectangle{\pgfqpoint{0.550713in}{0.093599in}}{\pgfqpoint{3.028919in}{1.684778in}}%
\pgfusepath{clip}%
\pgfsetrectcap%
\pgfsetroundjoin%
\pgfsetlinewidth{0.752812pt}%
\definecolor{currentstroke}{rgb}{0.000000,0.000000,0.000000}%
\pgfsetstrokecolor{currentstroke}%
\pgfsetdash{}{0pt}%
\pgfpathmoveto{\pgfqpoint{2.293098in}{0.664620in}}%
\pgfpathlineto{\pgfqpoint{2.367307in}{0.664620in}}%
\pgfusepath{stroke}%
\end{pgfscope}%
\begin{pgfscope}%
\pgfpathrectangle{\pgfqpoint{0.550713in}{0.093599in}}{\pgfqpoint{3.028919in}{1.684778in}}%
\pgfusepath{clip}%
\pgfsetrectcap%
\pgfsetroundjoin%
\pgfsetlinewidth{0.752812pt}%
\definecolor{currentstroke}{rgb}{0.000000,0.000000,0.000000}%
\pgfsetstrokecolor{currentstroke}%
\pgfsetdash{}{0pt}%
\pgfpathmoveto{\pgfqpoint{2.293098in}{0.982515in}}%
\pgfpathlineto{\pgfqpoint{2.367307in}{0.982515in}}%
\pgfusepath{stroke}%
\end{pgfscope}%
\begin{pgfscope}%
\pgfpathrectangle{\pgfqpoint{0.550713in}{0.093599in}}{\pgfqpoint{3.028919in}{1.684778in}}%
\pgfusepath{clip}%
\pgfsetrectcap%
\pgfsetroundjoin%
\pgfsetlinewidth{0.752812pt}%
\definecolor{currentstroke}{rgb}{0.000000,0.000000,0.000000}%
\pgfsetstrokecolor{currentstroke}%
\pgfsetdash{}{0pt}%
\pgfpathmoveto{\pgfqpoint{2.557372in}{0.581919in}}%
\pgfpathlineto{\pgfqpoint{2.557372in}{0.531866in}}%
\pgfusepath{stroke}%
\end{pgfscope}%
\begin{pgfscope}%
\pgfpathrectangle{\pgfqpoint{0.550713in}{0.093599in}}{\pgfqpoint{3.028919in}{1.684778in}}%
\pgfusepath{clip}%
\pgfsetrectcap%
\pgfsetroundjoin%
\pgfsetlinewidth{0.752812pt}%
\definecolor{currentstroke}{rgb}{0.000000,0.000000,0.000000}%
\pgfsetstrokecolor{currentstroke}%
\pgfsetdash{}{0pt}%
\pgfpathmoveto{\pgfqpoint{2.557372in}{0.619819in}}%
\pgfpathlineto{\pgfqpoint{2.557372in}{0.674734in}}%
\pgfusepath{stroke}%
\end{pgfscope}%
\begin{pgfscope}%
\pgfpathrectangle{\pgfqpoint{0.550713in}{0.093599in}}{\pgfqpoint{3.028919in}{1.684778in}}%
\pgfusepath{clip}%
\pgfsetrectcap%
\pgfsetroundjoin%
\pgfsetlinewidth{0.752812pt}%
\definecolor{currentstroke}{rgb}{0.000000,0.000000,0.000000}%
\pgfsetstrokecolor{currentstroke}%
\pgfsetdash{}{0pt}%
\pgfpathmoveto{\pgfqpoint{2.520267in}{0.531866in}}%
\pgfpathlineto{\pgfqpoint{2.594476in}{0.531866in}}%
\pgfusepath{stroke}%
\end{pgfscope}%
\begin{pgfscope}%
\pgfpathrectangle{\pgfqpoint{0.550713in}{0.093599in}}{\pgfqpoint{3.028919in}{1.684778in}}%
\pgfusepath{clip}%
\pgfsetrectcap%
\pgfsetroundjoin%
\pgfsetlinewidth{0.752812pt}%
\definecolor{currentstroke}{rgb}{0.000000,0.000000,0.000000}%
\pgfsetstrokecolor{currentstroke}%
\pgfsetdash{}{0pt}%
\pgfpathmoveto{\pgfqpoint{2.520267in}{0.674734in}}%
\pgfpathlineto{\pgfqpoint{2.594476in}{0.674734in}}%
\pgfusepath{stroke}%
\end{pgfscope}%
\begin{pgfscope}%
\pgfpathrectangle{\pgfqpoint{0.550713in}{0.093599in}}{\pgfqpoint{3.028919in}{1.684778in}}%
\pgfusepath{clip}%
\pgfsetbuttcap%
\pgfsetmiterjoin%
\definecolor{currentfill}{rgb}{0.000000,0.000000,0.000000}%
\pgfsetfillcolor{currentfill}%
\pgfsetlinewidth{1.003750pt}%
\definecolor{currentstroke}{rgb}{0.000000,0.000000,0.000000}%
\pgfsetstrokecolor{currentstroke}%
\pgfsetdash{}{0pt}%
\pgfsys@defobject{currentmarker}{\pgfqpoint{-0.011785in}{-0.019642in}}{\pgfqpoint{0.011785in}{0.019642in}}{%
\pgfpathmoveto{\pgfqpoint{-0.000000in}{-0.019642in}}%
\pgfpathlineto{\pgfqpoint{0.011785in}{0.000000in}}%
\pgfpathlineto{\pgfqpoint{0.000000in}{0.019642in}}%
\pgfpathlineto{\pgfqpoint{-0.011785in}{0.000000in}}%
\pgfpathclose%
\pgfusepath{stroke,fill}%
}%
\begin{pgfscope}%
\pgfsys@transformshift{2.557372in}{0.697951in}%
\pgfsys@useobject{currentmarker}{}%
\end{pgfscope}%
\begin{pgfscope}%
\pgfsys@transformshift{2.557372in}{0.718775in}%
\pgfsys@useobject{currentmarker}{}%
\end{pgfscope}%
\begin{pgfscope}%
\pgfsys@transformshift{2.557372in}{0.704103in}%
\pgfsys@useobject{currentmarker}{}%
\end{pgfscope}%
\begin{pgfscope}%
\pgfsys@transformshift{2.557372in}{0.685960in}%
\pgfsys@useobject{currentmarker}{}%
\end{pgfscope}%
\end{pgfscope}%
\begin{pgfscope}%
\pgfpathrectangle{\pgfqpoint{0.550713in}{0.093599in}}{\pgfqpoint{3.028919in}{1.684778in}}%
\pgfusepath{clip}%
\pgfsetrectcap%
\pgfsetroundjoin%
\pgfsetlinewidth{0.752812pt}%
\definecolor{currentstroke}{rgb}{0.000000,0.000000,0.000000}%
\pgfsetstrokecolor{currentstroke}%
\pgfsetdash{}{0pt}%
\pgfpathmoveto{\pgfqpoint{2.708818in}{0.693261in}}%
\pgfpathlineto{\pgfqpoint{2.708818in}{0.621068in}}%
\pgfusepath{stroke}%
\end{pgfscope}%
\begin{pgfscope}%
\pgfpathrectangle{\pgfqpoint{0.550713in}{0.093599in}}{\pgfqpoint{3.028919in}{1.684778in}}%
\pgfusepath{clip}%
\pgfsetrectcap%
\pgfsetroundjoin%
\pgfsetlinewidth{0.752812pt}%
\definecolor{currentstroke}{rgb}{0.000000,0.000000,0.000000}%
\pgfsetstrokecolor{currentstroke}%
\pgfsetdash{}{0pt}%
\pgfpathmoveto{\pgfqpoint{2.708818in}{0.783139in}}%
\pgfpathlineto{\pgfqpoint{2.708818in}{0.889999in}}%
\pgfusepath{stroke}%
\end{pgfscope}%
\begin{pgfscope}%
\pgfpathrectangle{\pgfqpoint{0.550713in}{0.093599in}}{\pgfqpoint{3.028919in}{1.684778in}}%
\pgfusepath{clip}%
\pgfsetrectcap%
\pgfsetroundjoin%
\pgfsetlinewidth{0.752812pt}%
\definecolor{currentstroke}{rgb}{0.000000,0.000000,0.000000}%
\pgfsetstrokecolor{currentstroke}%
\pgfsetdash{}{0pt}%
\pgfpathmoveto{\pgfqpoint{2.671713in}{0.621068in}}%
\pgfpathlineto{\pgfqpoint{2.745922in}{0.621068in}}%
\pgfusepath{stroke}%
\end{pgfscope}%
\begin{pgfscope}%
\pgfpathrectangle{\pgfqpoint{0.550713in}{0.093599in}}{\pgfqpoint{3.028919in}{1.684778in}}%
\pgfusepath{clip}%
\pgfsetrectcap%
\pgfsetroundjoin%
\pgfsetlinewidth{0.752812pt}%
\definecolor{currentstroke}{rgb}{0.000000,0.000000,0.000000}%
\pgfsetstrokecolor{currentstroke}%
\pgfsetdash{}{0pt}%
\pgfpathmoveto{\pgfqpoint{2.671713in}{0.889999in}}%
\pgfpathlineto{\pgfqpoint{2.745922in}{0.889999in}}%
\pgfusepath{stroke}%
\end{pgfscope}%
\begin{pgfscope}%
\pgfpathrectangle{\pgfqpoint{0.550713in}{0.093599in}}{\pgfqpoint{3.028919in}{1.684778in}}%
\pgfusepath{clip}%
\pgfsetbuttcap%
\pgfsetmiterjoin%
\definecolor{currentfill}{rgb}{0.000000,0.000000,0.000000}%
\pgfsetfillcolor{currentfill}%
\pgfsetlinewidth{1.003750pt}%
\definecolor{currentstroke}{rgb}{0.000000,0.000000,0.000000}%
\pgfsetstrokecolor{currentstroke}%
\pgfsetdash{}{0pt}%
\pgfsys@defobject{currentmarker}{\pgfqpoint{-0.011785in}{-0.019642in}}{\pgfqpoint{0.011785in}{0.019642in}}{%
\pgfpathmoveto{\pgfqpoint{-0.000000in}{-0.019642in}}%
\pgfpathlineto{\pgfqpoint{0.011785in}{0.000000in}}%
\pgfpathlineto{\pgfqpoint{0.000000in}{0.019642in}}%
\pgfpathlineto{\pgfqpoint{-0.011785in}{0.000000in}}%
\pgfpathclose%
\pgfusepath{stroke,fill}%
}%
\begin{pgfscope}%
\pgfsys@transformshift{2.708818in}{0.921122in}%
\pgfsys@useobject{currentmarker}{}%
\end{pgfscope}%
\end{pgfscope}%
\begin{pgfscope}%
\pgfpathrectangle{\pgfqpoint{0.550713in}{0.093599in}}{\pgfqpoint{3.028919in}{1.684778in}}%
\pgfusepath{clip}%
\pgfsetrectcap%
\pgfsetroundjoin%
\pgfsetlinewidth{0.752812pt}%
\definecolor{currentstroke}{rgb}{0.000000,0.000000,0.000000}%
\pgfsetstrokecolor{currentstroke}%
\pgfsetdash{}{0pt}%
\pgfpathmoveto{\pgfqpoint{2.935987in}{0.488761in}}%
\pgfpathlineto{\pgfqpoint{2.935987in}{0.450316in}}%
\pgfusepath{stroke}%
\end{pgfscope}%
\begin{pgfscope}%
\pgfpathrectangle{\pgfqpoint{0.550713in}{0.093599in}}{\pgfqpoint{3.028919in}{1.684778in}}%
\pgfusepath{clip}%
\pgfsetrectcap%
\pgfsetroundjoin%
\pgfsetlinewidth{0.752812pt}%
\definecolor{currentstroke}{rgb}{0.000000,0.000000,0.000000}%
\pgfsetstrokecolor{currentstroke}%
\pgfsetdash{}{0pt}%
\pgfpathmoveto{\pgfqpoint{2.935987in}{0.545758in}}%
\pgfpathlineto{\pgfqpoint{2.935987in}{0.593568in}}%
\pgfusepath{stroke}%
\end{pgfscope}%
\begin{pgfscope}%
\pgfpathrectangle{\pgfqpoint{0.550713in}{0.093599in}}{\pgfqpoint{3.028919in}{1.684778in}}%
\pgfusepath{clip}%
\pgfsetrectcap%
\pgfsetroundjoin%
\pgfsetlinewidth{0.752812pt}%
\definecolor{currentstroke}{rgb}{0.000000,0.000000,0.000000}%
\pgfsetstrokecolor{currentstroke}%
\pgfsetdash{}{0pt}%
\pgfpathmoveto{\pgfqpoint{2.898882in}{0.450316in}}%
\pgfpathlineto{\pgfqpoint{2.973091in}{0.450316in}}%
\pgfusepath{stroke}%
\end{pgfscope}%
\begin{pgfscope}%
\pgfpathrectangle{\pgfqpoint{0.550713in}{0.093599in}}{\pgfqpoint{3.028919in}{1.684778in}}%
\pgfusepath{clip}%
\pgfsetrectcap%
\pgfsetroundjoin%
\pgfsetlinewidth{0.752812pt}%
\definecolor{currentstroke}{rgb}{0.000000,0.000000,0.000000}%
\pgfsetstrokecolor{currentstroke}%
\pgfsetdash{}{0pt}%
\pgfpathmoveto{\pgfqpoint{2.898882in}{0.593568in}}%
\pgfpathlineto{\pgfqpoint{2.973091in}{0.593568in}}%
\pgfusepath{stroke}%
\end{pgfscope}%
\begin{pgfscope}%
\pgfpathrectangle{\pgfqpoint{0.550713in}{0.093599in}}{\pgfqpoint{3.028919in}{1.684778in}}%
\pgfusepath{clip}%
\pgfsetbuttcap%
\pgfsetmiterjoin%
\definecolor{currentfill}{rgb}{0.000000,0.000000,0.000000}%
\pgfsetfillcolor{currentfill}%
\pgfsetlinewidth{1.003750pt}%
\definecolor{currentstroke}{rgb}{0.000000,0.000000,0.000000}%
\pgfsetstrokecolor{currentstroke}%
\pgfsetdash{}{0pt}%
\pgfsys@defobject{currentmarker}{\pgfqpoint{-0.011785in}{-0.019642in}}{\pgfqpoint{0.011785in}{0.019642in}}{%
\pgfpathmoveto{\pgfqpoint{-0.000000in}{-0.019642in}}%
\pgfpathlineto{\pgfqpoint{0.011785in}{0.000000in}}%
\pgfpathlineto{\pgfqpoint{0.000000in}{0.019642in}}%
\pgfpathlineto{\pgfqpoint{-0.011785in}{0.000000in}}%
\pgfpathclose%
\pgfusepath{stroke,fill}%
}%
\begin{pgfscope}%
\pgfsys@transformshift{2.935987in}{0.361978in}%
\pgfsys@useobject{currentmarker}{}%
\end{pgfscope}%
\begin{pgfscope}%
\pgfsys@transformshift{2.935987in}{0.399431in}%
\pgfsys@useobject{currentmarker}{}%
\end{pgfscope}%
\begin{pgfscope}%
\pgfsys@transformshift{2.935987in}{0.370407in}%
\pgfsys@useobject{currentmarker}{}%
\end{pgfscope}%
\end{pgfscope}%
\begin{pgfscope}%
\pgfpathrectangle{\pgfqpoint{0.550713in}{0.093599in}}{\pgfqpoint{3.028919in}{1.684778in}}%
\pgfusepath{clip}%
\pgfsetrectcap%
\pgfsetroundjoin%
\pgfsetlinewidth{0.752812pt}%
\definecolor{currentstroke}{rgb}{0.000000,0.000000,0.000000}%
\pgfsetstrokecolor{currentstroke}%
\pgfsetdash{}{0pt}%
\pgfpathmoveto{\pgfqpoint{3.087433in}{0.491709in}}%
\pgfpathlineto{\pgfqpoint{3.087433in}{0.445702in}}%
\pgfusepath{stroke}%
\end{pgfscope}%
\begin{pgfscope}%
\pgfpathrectangle{\pgfqpoint{0.550713in}{0.093599in}}{\pgfqpoint{3.028919in}{1.684778in}}%
\pgfusepath{clip}%
\pgfsetrectcap%
\pgfsetroundjoin%
\pgfsetlinewidth{0.752812pt}%
\definecolor{currentstroke}{rgb}{0.000000,0.000000,0.000000}%
\pgfsetstrokecolor{currentstroke}%
\pgfsetdash{}{0pt}%
\pgfpathmoveto{\pgfqpoint{3.087433in}{0.559962in}}%
\pgfpathlineto{\pgfqpoint{3.087433in}{0.633855in}}%
\pgfusepath{stroke}%
\end{pgfscope}%
\begin{pgfscope}%
\pgfpathrectangle{\pgfqpoint{0.550713in}{0.093599in}}{\pgfqpoint{3.028919in}{1.684778in}}%
\pgfusepath{clip}%
\pgfsetrectcap%
\pgfsetroundjoin%
\pgfsetlinewidth{0.752812pt}%
\definecolor{currentstroke}{rgb}{0.000000,0.000000,0.000000}%
\pgfsetstrokecolor{currentstroke}%
\pgfsetdash{}{0pt}%
\pgfpathmoveto{\pgfqpoint{3.050328in}{0.445702in}}%
\pgfpathlineto{\pgfqpoint{3.124537in}{0.445702in}}%
\pgfusepath{stroke}%
\end{pgfscope}%
\begin{pgfscope}%
\pgfpathrectangle{\pgfqpoint{0.550713in}{0.093599in}}{\pgfqpoint{3.028919in}{1.684778in}}%
\pgfusepath{clip}%
\pgfsetrectcap%
\pgfsetroundjoin%
\pgfsetlinewidth{0.752812pt}%
\definecolor{currentstroke}{rgb}{0.000000,0.000000,0.000000}%
\pgfsetstrokecolor{currentstroke}%
\pgfsetdash{}{0pt}%
\pgfpathmoveto{\pgfqpoint{3.050328in}{0.633855in}}%
\pgfpathlineto{\pgfqpoint{3.124537in}{0.633855in}}%
\pgfusepath{stroke}%
\end{pgfscope}%
\begin{pgfscope}%
\pgfpathrectangle{\pgfqpoint{0.550713in}{0.093599in}}{\pgfqpoint{3.028919in}{1.684778in}}%
\pgfusepath{clip}%
\pgfsetrectcap%
\pgfsetroundjoin%
\pgfsetlinewidth{0.752812pt}%
\definecolor{currentstroke}{rgb}{0.000000,0.000000,0.000000}%
\pgfsetstrokecolor{currentstroke}%
\pgfsetdash{}{0pt}%
\pgfpathmoveto{\pgfqpoint{3.314601in}{0.492839in}}%
\pgfpathlineto{\pgfqpoint{3.314601in}{0.478037in}}%
\pgfusepath{stroke}%
\end{pgfscope}%
\begin{pgfscope}%
\pgfpathrectangle{\pgfqpoint{0.550713in}{0.093599in}}{\pgfqpoint{3.028919in}{1.684778in}}%
\pgfusepath{clip}%
\pgfsetrectcap%
\pgfsetroundjoin%
\pgfsetlinewidth{0.752812pt}%
\definecolor{currentstroke}{rgb}{0.000000,0.000000,0.000000}%
\pgfsetstrokecolor{currentstroke}%
\pgfsetdash{}{0pt}%
\pgfpathmoveto{\pgfqpoint{3.314601in}{0.526347in}}%
\pgfpathlineto{\pgfqpoint{3.314601in}{0.565158in}}%
\pgfusepath{stroke}%
\end{pgfscope}%
\begin{pgfscope}%
\pgfpathrectangle{\pgfqpoint{0.550713in}{0.093599in}}{\pgfqpoint{3.028919in}{1.684778in}}%
\pgfusepath{clip}%
\pgfsetrectcap%
\pgfsetroundjoin%
\pgfsetlinewidth{0.752812pt}%
\definecolor{currentstroke}{rgb}{0.000000,0.000000,0.000000}%
\pgfsetstrokecolor{currentstroke}%
\pgfsetdash{}{0pt}%
\pgfpathmoveto{\pgfqpoint{3.277497in}{0.478037in}}%
\pgfpathlineto{\pgfqpoint{3.351706in}{0.478037in}}%
\pgfusepath{stroke}%
\end{pgfscope}%
\begin{pgfscope}%
\pgfpathrectangle{\pgfqpoint{0.550713in}{0.093599in}}{\pgfqpoint{3.028919in}{1.684778in}}%
\pgfusepath{clip}%
\pgfsetrectcap%
\pgfsetroundjoin%
\pgfsetlinewidth{0.752812pt}%
\definecolor{currentstroke}{rgb}{0.000000,0.000000,0.000000}%
\pgfsetstrokecolor{currentstroke}%
\pgfsetdash{}{0pt}%
\pgfpathmoveto{\pgfqpoint{3.277497in}{0.565158in}}%
\pgfpathlineto{\pgfqpoint{3.351706in}{0.565158in}}%
\pgfusepath{stroke}%
\end{pgfscope}%
\begin{pgfscope}%
\pgfpathrectangle{\pgfqpoint{0.550713in}{0.093599in}}{\pgfqpoint{3.028919in}{1.684778in}}%
\pgfusepath{clip}%
\pgfsetbuttcap%
\pgfsetmiterjoin%
\definecolor{currentfill}{rgb}{0.000000,0.000000,0.000000}%
\pgfsetfillcolor{currentfill}%
\pgfsetlinewidth{1.003750pt}%
\definecolor{currentstroke}{rgb}{0.000000,0.000000,0.000000}%
\pgfsetstrokecolor{currentstroke}%
\pgfsetdash{}{0pt}%
\pgfsys@defobject{currentmarker}{\pgfqpoint{-0.011785in}{-0.019642in}}{\pgfqpoint{0.011785in}{0.019642in}}{%
\pgfpathmoveto{\pgfqpoint{-0.000000in}{-0.019642in}}%
\pgfpathlineto{\pgfqpoint{0.011785in}{0.000000in}}%
\pgfpathlineto{\pgfqpoint{0.000000in}{0.019642in}}%
\pgfpathlineto{\pgfqpoint{-0.011785in}{0.000000in}}%
\pgfpathclose%
\pgfusepath{stroke,fill}%
}%
\begin{pgfscope}%
\pgfsys@transformshift{3.314601in}{0.439701in}%
\pgfsys@useobject{currentmarker}{}%
\end{pgfscope}%
\begin{pgfscope}%
\pgfsys@transformshift{3.314601in}{0.432622in}%
\pgfsys@useobject{currentmarker}{}%
\end{pgfscope}%
\begin{pgfscope}%
\pgfsys@transformshift{3.314601in}{0.605359in}%
\pgfsys@useobject{currentmarker}{}%
\end{pgfscope}%
\begin{pgfscope}%
\pgfsys@transformshift{3.314601in}{0.595959in}%
\pgfsys@useobject{currentmarker}{}%
\end{pgfscope}%
\end{pgfscope}%
\begin{pgfscope}%
\pgfpathrectangle{\pgfqpoint{0.550713in}{0.093599in}}{\pgfqpoint{3.028919in}{1.684778in}}%
\pgfusepath{clip}%
\pgfsetrectcap%
\pgfsetroundjoin%
\pgfsetlinewidth{0.752812pt}%
\definecolor{currentstroke}{rgb}{0.000000,0.000000,0.000000}%
\pgfsetstrokecolor{currentstroke}%
\pgfsetdash{}{0pt}%
\pgfpathmoveto{\pgfqpoint{3.466047in}{0.502531in}}%
\pgfpathlineto{\pgfqpoint{3.466047in}{0.457209in}}%
\pgfusepath{stroke}%
\end{pgfscope}%
\begin{pgfscope}%
\pgfpathrectangle{\pgfqpoint{0.550713in}{0.093599in}}{\pgfqpoint{3.028919in}{1.684778in}}%
\pgfusepath{clip}%
\pgfsetrectcap%
\pgfsetroundjoin%
\pgfsetlinewidth{0.752812pt}%
\definecolor{currentstroke}{rgb}{0.000000,0.000000,0.000000}%
\pgfsetstrokecolor{currentstroke}%
\pgfsetdash{}{0pt}%
\pgfpathmoveto{\pgfqpoint{3.466047in}{0.561235in}}%
\pgfpathlineto{\pgfqpoint{3.466047in}{0.621996in}}%
\pgfusepath{stroke}%
\end{pgfscope}%
\begin{pgfscope}%
\pgfpathrectangle{\pgfqpoint{0.550713in}{0.093599in}}{\pgfqpoint{3.028919in}{1.684778in}}%
\pgfusepath{clip}%
\pgfsetrectcap%
\pgfsetroundjoin%
\pgfsetlinewidth{0.752812pt}%
\definecolor{currentstroke}{rgb}{0.000000,0.000000,0.000000}%
\pgfsetstrokecolor{currentstroke}%
\pgfsetdash{}{0pt}%
\pgfpathmoveto{\pgfqpoint{3.428943in}{0.457209in}}%
\pgfpathlineto{\pgfqpoint{3.503152in}{0.457209in}}%
\pgfusepath{stroke}%
\end{pgfscope}%
\begin{pgfscope}%
\pgfpathrectangle{\pgfqpoint{0.550713in}{0.093599in}}{\pgfqpoint{3.028919in}{1.684778in}}%
\pgfusepath{clip}%
\pgfsetrectcap%
\pgfsetroundjoin%
\pgfsetlinewidth{0.752812pt}%
\definecolor{currentstroke}{rgb}{0.000000,0.000000,0.000000}%
\pgfsetstrokecolor{currentstroke}%
\pgfsetdash{}{0pt}%
\pgfpathmoveto{\pgfqpoint{3.428943in}{0.621996in}}%
\pgfpathlineto{\pgfqpoint{3.503152in}{0.621996in}}%
\pgfusepath{stroke}%
\end{pgfscope}%
\begin{pgfscope}%
\pgfpathrectangle{\pgfqpoint{0.550713in}{0.093599in}}{\pgfqpoint{3.028919in}{1.684778in}}%
\pgfusepath{clip}%
\pgfsetrectcap%
\pgfsetroundjoin%
\pgfsetlinewidth{0.752812pt}%
\definecolor{currentstroke}{rgb}{0.000000,0.000000,0.000000}%
\pgfsetstrokecolor{currentstroke}%
\pgfsetdash{}{0pt}%
\pgfpathmoveto{\pgfqpoint{0.590089in}{0.585223in}}%
\pgfpathlineto{\pgfqpoint{0.738506in}{0.585223in}}%
\pgfusepath{stroke}%
\end{pgfscope}%
\begin{pgfscope}%
\pgfpathrectangle{\pgfqpoint{0.550713in}{0.093599in}}{\pgfqpoint{3.028919in}{1.684778in}}%
\pgfusepath{clip}%
\pgfsetbuttcap%
\pgfsetroundjoin%
\definecolor{currentfill}{rgb}{1.000000,1.000000,1.000000}%
\pgfsetfillcolor{currentfill}%
\pgfsetlinewidth{1.003750pt}%
\definecolor{currentstroke}{rgb}{0.000000,0.000000,0.000000}%
\pgfsetstrokecolor{currentstroke}%
\pgfsetdash{}{0pt}%
\pgfsys@defobject{currentmarker}{\pgfqpoint{-0.027778in}{-0.027778in}}{\pgfqpoint{0.027778in}{0.027778in}}{%
\pgfpathmoveto{\pgfqpoint{0.000000in}{-0.027778in}}%
\pgfpathcurveto{\pgfqpoint{0.007367in}{-0.027778in}}{\pgfqpoint{0.014433in}{-0.024851in}}{\pgfqpoint{0.019642in}{-0.019642in}}%
\pgfpathcurveto{\pgfqpoint{0.024851in}{-0.014433in}}{\pgfqpoint{0.027778in}{-0.007367in}}{\pgfqpoint{0.027778in}{0.000000in}}%
\pgfpathcurveto{\pgfqpoint{0.027778in}{0.007367in}}{\pgfqpoint{0.024851in}{0.014433in}}{\pgfqpoint{0.019642in}{0.019642in}}%
\pgfpathcurveto{\pgfqpoint{0.014433in}{0.024851in}}{\pgfqpoint{0.007367in}{0.027778in}}{\pgfqpoint{0.000000in}{0.027778in}}%
\pgfpathcurveto{\pgfqpoint{-0.007367in}{0.027778in}}{\pgfqpoint{-0.014433in}{0.024851in}}{\pgfqpoint{-0.019642in}{0.019642in}}%
\pgfpathcurveto{\pgfqpoint{-0.024851in}{0.014433in}}{\pgfqpoint{-0.027778in}{0.007367in}}{\pgfqpoint{-0.027778in}{0.000000in}}%
\pgfpathcurveto{\pgfqpoint{-0.027778in}{-0.007367in}}{\pgfqpoint{-0.024851in}{-0.014433in}}{\pgfqpoint{-0.019642in}{-0.019642in}}%
\pgfpathcurveto{\pgfqpoint{-0.014433in}{-0.024851in}}{\pgfqpoint{-0.007367in}{-0.027778in}}{\pgfqpoint{0.000000in}{-0.027778in}}%
\pgfpathclose%
\pgfusepath{stroke,fill}%
}%
\begin{pgfscope}%
\pgfsys@transformshift{0.664297in}{0.624963in}%
\pgfsys@useobject{currentmarker}{}%
\end{pgfscope}%
\end{pgfscope}%
\begin{pgfscope}%
\pgfpathrectangle{\pgfqpoint{0.550713in}{0.093599in}}{\pgfqpoint{3.028919in}{1.684778in}}%
\pgfusepath{clip}%
\pgfsetrectcap%
\pgfsetroundjoin%
\pgfsetlinewidth{0.752812pt}%
\definecolor{currentstroke}{rgb}{0.000000,0.000000,0.000000}%
\pgfsetstrokecolor{currentstroke}%
\pgfsetdash{}{0pt}%
\pgfpathmoveto{\pgfqpoint{0.741535in}{0.856176in}}%
\pgfpathlineto{\pgfqpoint{0.889952in}{0.856176in}}%
\pgfusepath{stroke}%
\end{pgfscope}%
\begin{pgfscope}%
\pgfpathrectangle{\pgfqpoint{0.550713in}{0.093599in}}{\pgfqpoint{3.028919in}{1.684778in}}%
\pgfusepath{clip}%
\pgfsetbuttcap%
\pgfsetroundjoin%
\definecolor{currentfill}{rgb}{1.000000,1.000000,1.000000}%
\pgfsetfillcolor{currentfill}%
\pgfsetlinewidth{1.003750pt}%
\definecolor{currentstroke}{rgb}{0.000000,0.000000,0.000000}%
\pgfsetstrokecolor{currentstroke}%
\pgfsetdash{}{0pt}%
\pgfsys@defobject{currentmarker}{\pgfqpoint{-0.027778in}{-0.027778in}}{\pgfqpoint{0.027778in}{0.027778in}}{%
\pgfpathmoveto{\pgfqpoint{0.000000in}{-0.027778in}}%
\pgfpathcurveto{\pgfqpoint{0.007367in}{-0.027778in}}{\pgfqpoint{0.014433in}{-0.024851in}}{\pgfqpoint{0.019642in}{-0.019642in}}%
\pgfpathcurveto{\pgfqpoint{0.024851in}{-0.014433in}}{\pgfqpoint{0.027778in}{-0.007367in}}{\pgfqpoint{0.027778in}{0.000000in}}%
\pgfpathcurveto{\pgfqpoint{0.027778in}{0.007367in}}{\pgfqpoint{0.024851in}{0.014433in}}{\pgfqpoint{0.019642in}{0.019642in}}%
\pgfpathcurveto{\pgfqpoint{0.014433in}{0.024851in}}{\pgfqpoint{0.007367in}{0.027778in}}{\pgfqpoint{0.000000in}{0.027778in}}%
\pgfpathcurveto{\pgfqpoint{-0.007367in}{0.027778in}}{\pgfqpoint{-0.014433in}{0.024851in}}{\pgfqpoint{-0.019642in}{0.019642in}}%
\pgfpathcurveto{\pgfqpoint{-0.024851in}{0.014433in}}{\pgfqpoint{-0.027778in}{0.007367in}}{\pgfqpoint{-0.027778in}{0.000000in}}%
\pgfpathcurveto{\pgfqpoint{-0.027778in}{-0.007367in}}{\pgfqpoint{-0.024851in}{-0.014433in}}{\pgfqpoint{-0.019642in}{-0.019642in}}%
\pgfpathcurveto{\pgfqpoint{-0.014433in}{-0.024851in}}{\pgfqpoint{-0.007367in}{-0.027778in}}{\pgfqpoint{0.000000in}{-0.027778in}}%
\pgfpathclose%
\pgfusepath{stroke,fill}%
}%
\begin{pgfscope}%
\pgfsys@transformshift{0.815743in}{0.901055in}%
\pgfsys@useobject{currentmarker}{}%
\end{pgfscope}%
\end{pgfscope}%
\begin{pgfscope}%
\pgfpathrectangle{\pgfqpoint{0.550713in}{0.093599in}}{\pgfqpoint{3.028919in}{1.684778in}}%
\pgfusepath{clip}%
\pgfsetrectcap%
\pgfsetroundjoin%
\pgfsetlinewidth{0.752812pt}%
\definecolor{currentstroke}{rgb}{0.000000,0.000000,0.000000}%
\pgfsetstrokecolor{currentstroke}%
\pgfsetdash{}{0pt}%
\pgfpathmoveto{\pgfqpoint{0.968703in}{0.704832in}}%
\pgfpathlineto{\pgfqpoint{1.117121in}{0.704832in}}%
\pgfusepath{stroke}%
\end{pgfscope}%
\begin{pgfscope}%
\pgfpathrectangle{\pgfqpoint{0.550713in}{0.093599in}}{\pgfqpoint{3.028919in}{1.684778in}}%
\pgfusepath{clip}%
\pgfsetbuttcap%
\pgfsetroundjoin%
\definecolor{currentfill}{rgb}{1.000000,1.000000,1.000000}%
\pgfsetfillcolor{currentfill}%
\pgfsetlinewidth{1.003750pt}%
\definecolor{currentstroke}{rgb}{0.000000,0.000000,0.000000}%
\pgfsetstrokecolor{currentstroke}%
\pgfsetdash{}{0pt}%
\pgfsys@defobject{currentmarker}{\pgfqpoint{-0.027778in}{-0.027778in}}{\pgfqpoint{0.027778in}{0.027778in}}{%
\pgfpathmoveto{\pgfqpoint{0.000000in}{-0.027778in}}%
\pgfpathcurveto{\pgfqpoint{0.007367in}{-0.027778in}}{\pgfqpoint{0.014433in}{-0.024851in}}{\pgfqpoint{0.019642in}{-0.019642in}}%
\pgfpathcurveto{\pgfqpoint{0.024851in}{-0.014433in}}{\pgfqpoint{0.027778in}{-0.007367in}}{\pgfqpoint{0.027778in}{0.000000in}}%
\pgfpathcurveto{\pgfqpoint{0.027778in}{0.007367in}}{\pgfqpoint{0.024851in}{0.014433in}}{\pgfqpoint{0.019642in}{0.019642in}}%
\pgfpathcurveto{\pgfqpoint{0.014433in}{0.024851in}}{\pgfqpoint{0.007367in}{0.027778in}}{\pgfqpoint{0.000000in}{0.027778in}}%
\pgfpathcurveto{\pgfqpoint{-0.007367in}{0.027778in}}{\pgfqpoint{-0.014433in}{0.024851in}}{\pgfqpoint{-0.019642in}{0.019642in}}%
\pgfpathcurveto{\pgfqpoint{-0.024851in}{0.014433in}}{\pgfqpoint{-0.027778in}{0.007367in}}{\pgfqpoint{-0.027778in}{0.000000in}}%
\pgfpathcurveto{\pgfqpoint{-0.027778in}{-0.007367in}}{\pgfqpoint{-0.024851in}{-0.014433in}}{\pgfqpoint{-0.019642in}{-0.019642in}}%
\pgfpathcurveto{\pgfqpoint{-0.014433in}{-0.024851in}}{\pgfqpoint{-0.007367in}{-0.027778in}}{\pgfqpoint{0.000000in}{-0.027778in}}%
\pgfpathclose%
\pgfusepath{stroke,fill}%
}%
\begin{pgfscope}%
\pgfsys@transformshift{1.042912in}{0.813128in}%
\pgfsys@useobject{currentmarker}{}%
\end{pgfscope}%
\end{pgfscope}%
\begin{pgfscope}%
\pgfpathrectangle{\pgfqpoint{0.550713in}{0.093599in}}{\pgfqpoint{3.028919in}{1.684778in}}%
\pgfusepath{clip}%
\pgfsetrectcap%
\pgfsetroundjoin%
\pgfsetlinewidth{0.752812pt}%
\definecolor{currentstroke}{rgb}{0.000000,0.000000,0.000000}%
\pgfsetstrokecolor{currentstroke}%
\pgfsetdash{}{0pt}%
\pgfpathmoveto{\pgfqpoint{1.120149in}{1.368351in}}%
\pgfpathlineto{\pgfqpoint{1.268566in}{1.368351in}}%
\pgfusepath{stroke}%
\end{pgfscope}%
\begin{pgfscope}%
\pgfpathrectangle{\pgfqpoint{0.550713in}{0.093599in}}{\pgfqpoint{3.028919in}{1.684778in}}%
\pgfusepath{clip}%
\pgfsetbuttcap%
\pgfsetroundjoin%
\definecolor{currentfill}{rgb}{1.000000,1.000000,1.000000}%
\pgfsetfillcolor{currentfill}%
\pgfsetlinewidth{1.003750pt}%
\definecolor{currentstroke}{rgb}{0.000000,0.000000,0.000000}%
\pgfsetstrokecolor{currentstroke}%
\pgfsetdash{}{0pt}%
\pgfsys@defobject{currentmarker}{\pgfqpoint{-0.027778in}{-0.027778in}}{\pgfqpoint{0.027778in}{0.027778in}}{%
\pgfpathmoveto{\pgfqpoint{0.000000in}{-0.027778in}}%
\pgfpathcurveto{\pgfqpoint{0.007367in}{-0.027778in}}{\pgfqpoint{0.014433in}{-0.024851in}}{\pgfqpoint{0.019642in}{-0.019642in}}%
\pgfpathcurveto{\pgfqpoint{0.024851in}{-0.014433in}}{\pgfqpoint{0.027778in}{-0.007367in}}{\pgfqpoint{0.027778in}{0.000000in}}%
\pgfpathcurveto{\pgfqpoint{0.027778in}{0.007367in}}{\pgfqpoint{0.024851in}{0.014433in}}{\pgfqpoint{0.019642in}{0.019642in}}%
\pgfpathcurveto{\pgfqpoint{0.014433in}{0.024851in}}{\pgfqpoint{0.007367in}{0.027778in}}{\pgfqpoint{0.000000in}{0.027778in}}%
\pgfpathcurveto{\pgfqpoint{-0.007367in}{0.027778in}}{\pgfqpoint{-0.014433in}{0.024851in}}{\pgfqpoint{-0.019642in}{0.019642in}}%
\pgfpathcurveto{\pgfqpoint{-0.024851in}{0.014433in}}{\pgfqpoint{-0.027778in}{0.007367in}}{\pgfqpoint{-0.027778in}{0.000000in}}%
\pgfpathcurveto{\pgfqpoint{-0.027778in}{-0.007367in}}{\pgfqpoint{-0.024851in}{-0.014433in}}{\pgfqpoint{-0.019642in}{-0.019642in}}%
\pgfpathcurveto{\pgfqpoint{-0.014433in}{-0.024851in}}{\pgfqpoint{-0.007367in}{-0.027778in}}{\pgfqpoint{0.000000in}{-0.027778in}}%
\pgfpathclose%
\pgfusepath{stroke,fill}%
}%
\begin{pgfscope}%
\pgfsys@transformshift{1.194358in}{1.345881in}%
\pgfsys@useobject{currentmarker}{}%
\end{pgfscope}%
\end{pgfscope}%
\begin{pgfscope}%
\pgfpathrectangle{\pgfqpoint{0.550713in}{0.093599in}}{\pgfqpoint{3.028919in}{1.684778in}}%
\pgfusepath{clip}%
\pgfsetrectcap%
\pgfsetroundjoin%
\pgfsetlinewidth{0.752812pt}%
\definecolor{currentstroke}{rgb}{0.000000,0.000000,0.000000}%
\pgfsetstrokecolor{currentstroke}%
\pgfsetdash{}{0pt}%
\pgfpathmoveto{\pgfqpoint{1.347318in}{0.612073in}}%
\pgfpathlineto{\pgfqpoint{1.495735in}{0.612073in}}%
\pgfusepath{stroke}%
\end{pgfscope}%
\begin{pgfscope}%
\pgfpathrectangle{\pgfqpoint{0.550713in}{0.093599in}}{\pgfqpoint{3.028919in}{1.684778in}}%
\pgfusepath{clip}%
\pgfsetbuttcap%
\pgfsetroundjoin%
\definecolor{currentfill}{rgb}{1.000000,1.000000,1.000000}%
\pgfsetfillcolor{currentfill}%
\pgfsetlinewidth{1.003750pt}%
\definecolor{currentstroke}{rgb}{0.000000,0.000000,0.000000}%
\pgfsetstrokecolor{currentstroke}%
\pgfsetdash{}{0pt}%
\pgfsys@defobject{currentmarker}{\pgfqpoint{-0.027778in}{-0.027778in}}{\pgfqpoint{0.027778in}{0.027778in}}{%
\pgfpathmoveto{\pgfqpoint{0.000000in}{-0.027778in}}%
\pgfpathcurveto{\pgfqpoint{0.007367in}{-0.027778in}}{\pgfqpoint{0.014433in}{-0.024851in}}{\pgfqpoint{0.019642in}{-0.019642in}}%
\pgfpathcurveto{\pgfqpoint{0.024851in}{-0.014433in}}{\pgfqpoint{0.027778in}{-0.007367in}}{\pgfqpoint{0.027778in}{0.000000in}}%
\pgfpathcurveto{\pgfqpoint{0.027778in}{0.007367in}}{\pgfqpoint{0.024851in}{0.014433in}}{\pgfqpoint{0.019642in}{0.019642in}}%
\pgfpathcurveto{\pgfqpoint{0.014433in}{0.024851in}}{\pgfqpoint{0.007367in}{0.027778in}}{\pgfqpoint{0.000000in}{0.027778in}}%
\pgfpathcurveto{\pgfqpoint{-0.007367in}{0.027778in}}{\pgfqpoint{-0.014433in}{0.024851in}}{\pgfqpoint{-0.019642in}{0.019642in}}%
\pgfpathcurveto{\pgfqpoint{-0.024851in}{0.014433in}}{\pgfqpoint{-0.027778in}{0.007367in}}{\pgfqpoint{-0.027778in}{0.000000in}}%
\pgfpathcurveto{\pgfqpoint{-0.027778in}{-0.007367in}}{\pgfqpoint{-0.024851in}{-0.014433in}}{\pgfqpoint{-0.019642in}{-0.019642in}}%
\pgfpathcurveto{\pgfqpoint{-0.014433in}{-0.024851in}}{\pgfqpoint{-0.007367in}{-0.027778in}}{\pgfqpoint{0.000000in}{-0.027778in}}%
\pgfpathclose%
\pgfusepath{stroke,fill}%
}%
\begin{pgfscope}%
\pgfsys@transformshift{1.421527in}{0.602391in}%
\pgfsys@useobject{currentmarker}{}%
\end{pgfscope}%
\end{pgfscope}%
\begin{pgfscope}%
\pgfpathrectangle{\pgfqpoint{0.550713in}{0.093599in}}{\pgfqpoint{3.028919in}{1.684778in}}%
\pgfusepath{clip}%
\pgfsetrectcap%
\pgfsetroundjoin%
\pgfsetlinewidth{0.752812pt}%
\definecolor{currentstroke}{rgb}{0.000000,0.000000,0.000000}%
\pgfsetstrokecolor{currentstroke}%
\pgfsetdash{}{0pt}%
\pgfpathmoveto{\pgfqpoint{1.498764in}{0.634908in}}%
\pgfpathlineto{\pgfqpoint{1.647181in}{0.634908in}}%
\pgfusepath{stroke}%
\end{pgfscope}%
\begin{pgfscope}%
\pgfpathrectangle{\pgfqpoint{0.550713in}{0.093599in}}{\pgfqpoint{3.028919in}{1.684778in}}%
\pgfusepath{clip}%
\pgfsetbuttcap%
\pgfsetroundjoin%
\definecolor{currentfill}{rgb}{1.000000,1.000000,1.000000}%
\pgfsetfillcolor{currentfill}%
\pgfsetlinewidth{1.003750pt}%
\definecolor{currentstroke}{rgb}{0.000000,0.000000,0.000000}%
\pgfsetstrokecolor{currentstroke}%
\pgfsetdash{}{0pt}%
\pgfsys@defobject{currentmarker}{\pgfqpoint{-0.027778in}{-0.027778in}}{\pgfqpoint{0.027778in}{0.027778in}}{%
\pgfpathmoveto{\pgfqpoint{0.000000in}{-0.027778in}}%
\pgfpathcurveto{\pgfqpoint{0.007367in}{-0.027778in}}{\pgfqpoint{0.014433in}{-0.024851in}}{\pgfqpoint{0.019642in}{-0.019642in}}%
\pgfpathcurveto{\pgfqpoint{0.024851in}{-0.014433in}}{\pgfqpoint{0.027778in}{-0.007367in}}{\pgfqpoint{0.027778in}{0.000000in}}%
\pgfpathcurveto{\pgfqpoint{0.027778in}{0.007367in}}{\pgfqpoint{0.024851in}{0.014433in}}{\pgfqpoint{0.019642in}{0.019642in}}%
\pgfpathcurveto{\pgfqpoint{0.014433in}{0.024851in}}{\pgfqpoint{0.007367in}{0.027778in}}{\pgfqpoint{0.000000in}{0.027778in}}%
\pgfpathcurveto{\pgfqpoint{-0.007367in}{0.027778in}}{\pgfqpoint{-0.014433in}{0.024851in}}{\pgfqpoint{-0.019642in}{0.019642in}}%
\pgfpathcurveto{\pgfqpoint{-0.024851in}{0.014433in}}{\pgfqpoint{-0.027778in}{0.007367in}}{\pgfqpoint{-0.027778in}{0.000000in}}%
\pgfpathcurveto{\pgfqpoint{-0.027778in}{-0.007367in}}{\pgfqpoint{-0.024851in}{-0.014433in}}{\pgfqpoint{-0.019642in}{-0.019642in}}%
\pgfpathcurveto{\pgfqpoint{-0.014433in}{-0.024851in}}{\pgfqpoint{-0.007367in}{-0.027778in}}{\pgfqpoint{0.000000in}{-0.027778in}}%
\pgfpathclose%
\pgfusepath{stroke,fill}%
}%
\begin{pgfscope}%
\pgfsys@transformshift{1.572973in}{0.637374in}%
\pgfsys@useobject{currentmarker}{}%
\end{pgfscope}%
\end{pgfscope}%
\begin{pgfscope}%
\pgfpathrectangle{\pgfqpoint{0.550713in}{0.093599in}}{\pgfqpoint{3.028919in}{1.684778in}}%
\pgfusepath{clip}%
\pgfsetrectcap%
\pgfsetroundjoin%
\pgfsetlinewidth{0.752812pt}%
\definecolor{currentstroke}{rgb}{0.000000,0.000000,0.000000}%
\pgfsetstrokecolor{currentstroke}%
\pgfsetdash{}{0pt}%
\pgfpathmoveto{\pgfqpoint{1.725933in}{0.673136in}}%
\pgfpathlineto{\pgfqpoint{1.874350in}{0.673136in}}%
\pgfusepath{stroke}%
\end{pgfscope}%
\begin{pgfscope}%
\pgfpathrectangle{\pgfqpoint{0.550713in}{0.093599in}}{\pgfqpoint{3.028919in}{1.684778in}}%
\pgfusepath{clip}%
\pgfsetbuttcap%
\pgfsetroundjoin%
\definecolor{currentfill}{rgb}{1.000000,1.000000,1.000000}%
\pgfsetfillcolor{currentfill}%
\pgfsetlinewidth{1.003750pt}%
\definecolor{currentstroke}{rgb}{0.000000,0.000000,0.000000}%
\pgfsetstrokecolor{currentstroke}%
\pgfsetdash{}{0pt}%
\pgfsys@defobject{currentmarker}{\pgfqpoint{-0.027778in}{-0.027778in}}{\pgfqpoint{0.027778in}{0.027778in}}{%
\pgfpathmoveto{\pgfqpoint{0.000000in}{-0.027778in}}%
\pgfpathcurveto{\pgfqpoint{0.007367in}{-0.027778in}}{\pgfqpoint{0.014433in}{-0.024851in}}{\pgfqpoint{0.019642in}{-0.019642in}}%
\pgfpathcurveto{\pgfqpoint{0.024851in}{-0.014433in}}{\pgfqpoint{0.027778in}{-0.007367in}}{\pgfqpoint{0.027778in}{0.000000in}}%
\pgfpathcurveto{\pgfqpoint{0.027778in}{0.007367in}}{\pgfqpoint{0.024851in}{0.014433in}}{\pgfqpoint{0.019642in}{0.019642in}}%
\pgfpathcurveto{\pgfqpoint{0.014433in}{0.024851in}}{\pgfqpoint{0.007367in}{0.027778in}}{\pgfqpoint{0.000000in}{0.027778in}}%
\pgfpathcurveto{\pgfqpoint{-0.007367in}{0.027778in}}{\pgfqpoint{-0.014433in}{0.024851in}}{\pgfqpoint{-0.019642in}{0.019642in}}%
\pgfpathcurveto{\pgfqpoint{-0.024851in}{0.014433in}}{\pgfqpoint{-0.027778in}{0.007367in}}{\pgfqpoint{-0.027778in}{0.000000in}}%
\pgfpathcurveto{\pgfqpoint{-0.027778in}{-0.007367in}}{\pgfqpoint{-0.024851in}{-0.014433in}}{\pgfqpoint{-0.019642in}{-0.019642in}}%
\pgfpathcurveto{\pgfqpoint{-0.014433in}{-0.024851in}}{\pgfqpoint{-0.007367in}{-0.027778in}}{\pgfqpoint{0.000000in}{-0.027778in}}%
\pgfpathclose%
\pgfusepath{stroke,fill}%
}%
\begin{pgfscope}%
\pgfsys@transformshift{1.800142in}{0.668938in}%
\pgfsys@useobject{currentmarker}{}%
\end{pgfscope}%
\end{pgfscope}%
\begin{pgfscope}%
\pgfpathrectangle{\pgfqpoint{0.550713in}{0.093599in}}{\pgfqpoint{3.028919in}{1.684778in}}%
\pgfusepath{clip}%
\pgfsetrectcap%
\pgfsetroundjoin%
\pgfsetlinewidth{0.752812pt}%
\definecolor{currentstroke}{rgb}{0.000000,0.000000,0.000000}%
\pgfsetstrokecolor{currentstroke}%
\pgfsetdash{}{0pt}%
\pgfpathmoveto{\pgfqpoint{1.877379in}{0.692732in}}%
\pgfpathlineto{\pgfqpoint{2.025796in}{0.692732in}}%
\pgfusepath{stroke}%
\end{pgfscope}%
\begin{pgfscope}%
\pgfpathrectangle{\pgfqpoint{0.550713in}{0.093599in}}{\pgfqpoint{3.028919in}{1.684778in}}%
\pgfusepath{clip}%
\pgfsetbuttcap%
\pgfsetroundjoin%
\definecolor{currentfill}{rgb}{1.000000,1.000000,1.000000}%
\pgfsetfillcolor{currentfill}%
\pgfsetlinewidth{1.003750pt}%
\definecolor{currentstroke}{rgb}{0.000000,0.000000,0.000000}%
\pgfsetstrokecolor{currentstroke}%
\pgfsetdash{}{0pt}%
\pgfsys@defobject{currentmarker}{\pgfqpoint{-0.027778in}{-0.027778in}}{\pgfqpoint{0.027778in}{0.027778in}}{%
\pgfpathmoveto{\pgfqpoint{0.000000in}{-0.027778in}}%
\pgfpathcurveto{\pgfqpoint{0.007367in}{-0.027778in}}{\pgfqpoint{0.014433in}{-0.024851in}}{\pgfqpoint{0.019642in}{-0.019642in}}%
\pgfpathcurveto{\pgfqpoint{0.024851in}{-0.014433in}}{\pgfqpoint{0.027778in}{-0.007367in}}{\pgfqpoint{0.027778in}{0.000000in}}%
\pgfpathcurveto{\pgfqpoint{0.027778in}{0.007367in}}{\pgfqpoint{0.024851in}{0.014433in}}{\pgfqpoint{0.019642in}{0.019642in}}%
\pgfpathcurveto{\pgfqpoint{0.014433in}{0.024851in}}{\pgfqpoint{0.007367in}{0.027778in}}{\pgfqpoint{0.000000in}{0.027778in}}%
\pgfpathcurveto{\pgfqpoint{-0.007367in}{0.027778in}}{\pgfqpoint{-0.014433in}{0.024851in}}{\pgfqpoint{-0.019642in}{0.019642in}}%
\pgfpathcurveto{\pgfqpoint{-0.024851in}{0.014433in}}{\pgfqpoint{-0.027778in}{0.007367in}}{\pgfqpoint{-0.027778in}{0.000000in}}%
\pgfpathcurveto{\pgfqpoint{-0.027778in}{-0.007367in}}{\pgfqpoint{-0.024851in}{-0.014433in}}{\pgfqpoint{-0.019642in}{-0.019642in}}%
\pgfpathcurveto{\pgfqpoint{-0.014433in}{-0.024851in}}{\pgfqpoint{-0.007367in}{-0.027778in}}{\pgfqpoint{0.000000in}{-0.027778in}}%
\pgfpathclose%
\pgfusepath{stroke,fill}%
}%
\begin{pgfscope}%
\pgfsys@transformshift{1.951588in}{0.722062in}%
\pgfsys@useobject{currentmarker}{}%
\end{pgfscope}%
\end{pgfscope}%
\begin{pgfscope}%
\pgfpathrectangle{\pgfqpoint{0.550713in}{0.093599in}}{\pgfqpoint{3.028919in}{1.684778in}}%
\pgfusepath{clip}%
\pgfsetrectcap%
\pgfsetroundjoin%
\pgfsetlinewidth{0.752812pt}%
\definecolor{currentstroke}{rgb}{0.000000,0.000000,0.000000}%
\pgfsetstrokecolor{currentstroke}%
\pgfsetdash{}{0pt}%
\pgfpathmoveto{\pgfqpoint{2.104548in}{0.641992in}}%
\pgfpathlineto{\pgfqpoint{2.252965in}{0.641992in}}%
\pgfusepath{stroke}%
\end{pgfscope}%
\begin{pgfscope}%
\pgfpathrectangle{\pgfqpoint{0.550713in}{0.093599in}}{\pgfqpoint{3.028919in}{1.684778in}}%
\pgfusepath{clip}%
\pgfsetbuttcap%
\pgfsetroundjoin%
\definecolor{currentfill}{rgb}{1.000000,1.000000,1.000000}%
\pgfsetfillcolor{currentfill}%
\pgfsetlinewidth{1.003750pt}%
\definecolor{currentstroke}{rgb}{0.000000,0.000000,0.000000}%
\pgfsetstrokecolor{currentstroke}%
\pgfsetdash{}{0pt}%
\pgfsys@defobject{currentmarker}{\pgfqpoint{-0.027778in}{-0.027778in}}{\pgfqpoint{0.027778in}{0.027778in}}{%
\pgfpathmoveto{\pgfqpoint{0.000000in}{-0.027778in}}%
\pgfpathcurveto{\pgfqpoint{0.007367in}{-0.027778in}}{\pgfqpoint{0.014433in}{-0.024851in}}{\pgfqpoint{0.019642in}{-0.019642in}}%
\pgfpathcurveto{\pgfqpoint{0.024851in}{-0.014433in}}{\pgfqpoint{0.027778in}{-0.007367in}}{\pgfqpoint{0.027778in}{0.000000in}}%
\pgfpathcurveto{\pgfqpoint{0.027778in}{0.007367in}}{\pgfqpoint{0.024851in}{0.014433in}}{\pgfqpoint{0.019642in}{0.019642in}}%
\pgfpathcurveto{\pgfqpoint{0.014433in}{0.024851in}}{\pgfqpoint{0.007367in}{0.027778in}}{\pgfqpoint{0.000000in}{0.027778in}}%
\pgfpathcurveto{\pgfqpoint{-0.007367in}{0.027778in}}{\pgfqpoint{-0.014433in}{0.024851in}}{\pgfqpoint{-0.019642in}{0.019642in}}%
\pgfpathcurveto{\pgfqpoint{-0.024851in}{0.014433in}}{\pgfqpoint{-0.027778in}{0.007367in}}{\pgfqpoint{-0.027778in}{0.000000in}}%
\pgfpathcurveto{\pgfqpoint{-0.027778in}{-0.007367in}}{\pgfqpoint{-0.024851in}{-0.014433in}}{\pgfqpoint{-0.019642in}{-0.019642in}}%
\pgfpathcurveto{\pgfqpoint{-0.014433in}{-0.024851in}}{\pgfqpoint{-0.007367in}{-0.027778in}}{\pgfqpoint{0.000000in}{-0.027778in}}%
\pgfpathclose%
\pgfusepath{stroke,fill}%
}%
\begin{pgfscope}%
\pgfsys@transformshift{2.178757in}{0.647990in}%
\pgfsys@useobject{currentmarker}{}%
\end{pgfscope}%
\end{pgfscope}%
\begin{pgfscope}%
\pgfpathrectangle{\pgfqpoint{0.550713in}{0.093599in}}{\pgfqpoint{3.028919in}{1.684778in}}%
\pgfusepath{clip}%
\pgfsetrectcap%
\pgfsetroundjoin%
\pgfsetlinewidth{0.752812pt}%
\definecolor{currentstroke}{rgb}{0.000000,0.000000,0.000000}%
\pgfsetstrokecolor{currentstroke}%
\pgfsetdash{}{0pt}%
\pgfpathmoveto{\pgfqpoint{2.255994in}{0.802236in}}%
\pgfpathlineto{\pgfqpoint{2.404411in}{0.802236in}}%
\pgfusepath{stroke}%
\end{pgfscope}%
\begin{pgfscope}%
\pgfpathrectangle{\pgfqpoint{0.550713in}{0.093599in}}{\pgfqpoint{3.028919in}{1.684778in}}%
\pgfusepath{clip}%
\pgfsetbuttcap%
\pgfsetroundjoin%
\definecolor{currentfill}{rgb}{1.000000,1.000000,1.000000}%
\pgfsetfillcolor{currentfill}%
\pgfsetlinewidth{1.003750pt}%
\definecolor{currentstroke}{rgb}{0.000000,0.000000,0.000000}%
\pgfsetstrokecolor{currentstroke}%
\pgfsetdash{}{0pt}%
\pgfsys@defobject{currentmarker}{\pgfqpoint{-0.027778in}{-0.027778in}}{\pgfqpoint{0.027778in}{0.027778in}}{%
\pgfpathmoveto{\pgfqpoint{0.000000in}{-0.027778in}}%
\pgfpathcurveto{\pgfqpoint{0.007367in}{-0.027778in}}{\pgfqpoint{0.014433in}{-0.024851in}}{\pgfqpoint{0.019642in}{-0.019642in}}%
\pgfpathcurveto{\pgfqpoint{0.024851in}{-0.014433in}}{\pgfqpoint{0.027778in}{-0.007367in}}{\pgfqpoint{0.027778in}{0.000000in}}%
\pgfpathcurveto{\pgfqpoint{0.027778in}{0.007367in}}{\pgfqpoint{0.024851in}{0.014433in}}{\pgfqpoint{0.019642in}{0.019642in}}%
\pgfpathcurveto{\pgfqpoint{0.014433in}{0.024851in}}{\pgfqpoint{0.007367in}{0.027778in}}{\pgfqpoint{0.000000in}{0.027778in}}%
\pgfpathcurveto{\pgfqpoint{-0.007367in}{0.027778in}}{\pgfqpoint{-0.014433in}{0.024851in}}{\pgfqpoint{-0.019642in}{0.019642in}}%
\pgfpathcurveto{\pgfqpoint{-0.024851in}{0.014433in}}{\pgfqpoint{-0.027778in}{0.007367in}}{\pgfqpoint{-0.027778in}{0.000000in}}%
\pgfpathcurveto{\pgfqpoint{-0.027778in}{-0.007367in}}{\pgfqpoint{-0.024851in}{-0.014433in}}{\pgfqpoint{-0.019642in}{-0.019642in}}%
\pgfpathcurveto{\pgfqpoint{-0.014433in}{-0.024851in}}{\pgfqpoint{-0.007367in}{-0.027778in}}{\pgfqpoint{0.000000in}{-0.027778in}}%
\pgfpathclose%
\pgfusepath{stroke,fill}%
}%
\begin{pgfscope}%
\pgfsys@transformshift{2.330203in}{0.798747in}%
\pgfsys@useobject{currentmarker}{}%
\end{pgfscope}%
\end{pgfscope}%
\begin{pgfscope}%
\pgfpathrectangle{\pgfqpoint{0.550713in}{0.093599in}}{\pgfqpoint{3.028919in}{1.684778in}}%
\pgfusepath{clip}%
\pgfsetrectcap%
\pgfsetroundjoin%
\pgfsetlinewidth{0.752812pt}%
\definecolor{currentstroke}{rgb}{0.000000,0.000000,0.000000}%
\pgfsetstrokecolor{currentstroke}%
\pgfsetdash{}{0pt}%
\pgfpathmoveto{\pgfqpoint{2.483163in}{0.601280in}}%
\pgfpathlineto{\pgfqpoint{2.631580in}{0.601280in}}%
\pgfusepath{stroke}%
\end{pgfscope}%
\begin{pgfscope}%
\pgfpathrectangle{\pgfqpoint{0.550713in}{0.093599in}}{\pgfqpoint{3.028919in}{1.684778in}}%
\pgfusepath{clip}%
\pgfsetbuttcap%
\pgfsetroundjoin%
\definecolor{currentfill}{rgb}{1.000000,1.000000,1.000000}%
\pgfsetfillcolor{currentfill}%
\pgfsetlinewidth{1.003750pt}%
\definecolor{currentstroke}{rgb}{0.000000,0.000000,0.000000}%
\pgfsetstrokecolor{currentstroke}%
\pgfsetdash{}{0pt}%
\pgfsys@defobject{currentmarker}{\pgfqpoint{-0.027778in}{-0.027778in}}{\pgfqpoint{0.027778in}{0.027778in}}{%
\pgfpathmoveto{\pgfqpoint{0.000000in}{-0.027778in}}%
\pgfpathcurveto{\pgfqpoint{0.007367in}{-0.027778in}}{\pgfqpoint{0.014433in}{-0.024851in}}{\pgfqpoint{0.019642in}{-0.019642in}}%
\pgfpathcurveto{\pgfqpoint{0.024851in}{-0.014433in}}{\pgfqpoint{0.027778in}{-0.007367in}}{\pgfqpoint{0.027778in}{0.000000in}}%
\pgfpathcurveto{\pgfqpoint{0.027778in}{0.007367in}}{\pgfqpoint{0.024851in}{0.014433in}}{\pgfqpoint{0.019642in}{0.019642in}}%
\pgfpathcurveto{\pgfqpoint{0.014433in}{0.024851in}}{\pgfqpoint{0.007367in}{0.027778in}}{\pgfqpoint{0.000000in}{0.027778in}}%
\pgfpathcurveto{\pgfqpoint{-0.007367in}{0.027778in}}{\pgfqpoint{-0.014433in}{0.024851in}}{\pgfqpoint{-0.019642in}{0.019642in}}%
\pgfpathcurveto{\pgfqpoint{-0.024851in}{0.014433in}}{\pgfqpoint{-0.027778in}{0.007367in}}{\pgfqpoint{-0.027778in}{0.000000in}}%
\pgfpathcurveto{\pgfqpoint{-0.027778in}{-0.007367in}}{\pgfqpoint{-0.024851in}{-0.014433in}}{\pgfqpoint{-0.019642in}{-0.019642in}}%
\pgfpathcurveto{\pgfqpoint{-0.014433in}{-0.024851in}}{\pgfqpoint{-0.007367in}{-0.027778in}}{\pgfqpoint{0.000000in}{-0.027778in}}%
\pgfpathclose%
\pgfusepath{stroke,fill}%
}%
\begin{pgfscope}%
\pgfsys@transformshift{2.557372in}{0.613167in}%
\pgfsys@useobject{currentmarker}{}%
\end{pgfscope}%
\end{pgfscope}%
\begin{pgfscope}%
\pgfpathrectangle{\pgfqpoint{0.550713in}{0.093599in}}{\pgfqpoint{3.028919in}{1.684778in}}%
\pgfusepath{clip}%
\pgfsetrectcap%
\pgfsetroundjoin%
\pgfsetlinewidth{0.752812pt}%
\definecolor{currentstroke}{rgb}{0.000000,0.000000,0.000000}%
\pgfsetstrokecolor{currentstroke}%
\pgfsetdash{}{0pt}%
\pgfpathmoveto{\pgfqpoint{2.634609in}{0.735088in}}%
\pgfpathlineto{\pgfqpoint{2.783026in}{0.735088in}}%
\pgfusepath{stroke}%
\end{pgfscope}%
\begin{pgfscope}%
\pgfpathrectangle{\pgfqpoint{0.550713in}{0.093599in}}{\pgfqpoint{3.028919in}{1.684778in}}%
\pgfusepath{clip}%
\pgfsetbuttcap%
\pgfsetroundjoin%
\definecolor{currentfill}{rgb}{1.000000,1.000000,1.000000}%
\pgfsetfillcolor{currentfill}%
\pgfsetlinewidth{1.003750pt}%
\definecolor{currentstroke}{rgb}{0.000000,0.000000,0.000000}%
\pgfsetstrokecolor{currentstroke}%
\pgfsetdash{}{0pt}%
\pgfsys@defobject{currentmarker}{\pgfqpoint{-0.027778in}{-0.027778in}}{\pgfqpoint{0.027778in}{0.027778in}}{%
\pgfpathmoveto{\pgfqpoint{0.000000in}{-0.027778in}}%
\pgfpathcurveto{\pgfqpoint{0.007367in}{-0.027778in}}{\pgfqpoint{0.014433in}{-0.024851in}}{\pgfqpoint{0.019642in}{-0.019642in}}%
\pgfpathcurveto{\pgfqpoint{0.024851in}{-0.014433in}}{\pgfqpoint{0.027778in}{-0.007367in}}{\pgfqpoint{0.027778in}{0.000000in}}%
\pgfpathcurveto{\pgfqpoint{0.027778in}{0.007367in}}{\pgfqpoint{0.024851in}{0.014433in}}{\pgfqpoint{0.019642in}{0.019642in}}%
\pgfpathcurveto{\pgfqpoint{0.014433in}{0.024851in}}{\pgfqpoint{0.007367in}{0.027778in}}{\pgfqpoint{0.000000in}{0.027778in}}%
\pgfpathcurveto{\pgfqpoint{-0.007367in}{0.027778in}}{\pgfqpoint{-0.014433in}{0.024851in}}{\pgfqpoint{-0.019642in}{0.019642in}}%
\pgfpathcurveto{\pgfqpoint{-0.024851in}{0.014433in}}{\pgfqpoint{-0.027778in}{0.007367in}}{\pgfqpoint{-0.027778in}{0.000000in}}%
\pgfpathcurveto{\pgfqpoint{-0.027778in}{-0.007367in}}{\pgfqpoint{-0.024851in}{-0.014433in}}{\pgfqpoint{-0.019642in}{-0.019642in}}%
\pgfpathcurveto{\pgfqpoint{-0.014433in}{-0.024851in}}{\pgfqpoint{-0.007367in}{-0.027778in}}{\pgfqpoint{0.000000in}{-0.027778in}}%
\pgfpathclose%
\pgfusepath{stroke,fill}%
}%
\begin{pgfscope}%
\pgfsys@transformshift{2.708818in}{0.750313in}%
\pgfsys@useobject{currentmarker}{}%
\end{pgfscope}%
\end{pgfscope}%
\begin{pgfscope}%
\pgfpathrectangle{\pgfqpoint{0.550713in}{0.093599in}}{\pgfqpoint{3.028919in}{1.684778in}}%
\pgfusepath{clip}%
\pgfsetrectcap%
\pgfsetroundjoin%
\pgfsetlinewidth{0.752812pt}%
\definecolor{currentstroke}{rgb}{0.000000,0.000000,0.000000}%
\pgfsetstrokecolor{currentstroke}%
\pgfsetdash{}{0pt}%
\pgfpathmoveto{\pgfqpoint{2.861778in}{0.516299in}}%
\pgfpathlineto{\pgfqpoint{3.010195in}{0.516299in}}%
\pgfusepath{stroke}%
\end{pgfscope}%
\begin{pgfscope}%
\pgfpathrectangle{\pgfqpoint{0.550713in}{0.093599in}}{\pgfqpoint{3.028919in}{1.684778in}}%
\pgfusepath{clip}%
\pgfsetbuttcap%
\pgfsetroundjoin%
\definecolor{currentfill}{rgb}{1.000000,1.000000,1.000000}%
\pgfsetfillcolor{currentfill}%
\pgfsetlinewidth{1.003750pt}%
\definecolor{currentstroke}{rgb}{0.000000,0.000000,0.000000}%
\pgfsetstrokecolor{currentstroke}%
\pgfsetdash{}{0pt}%
\pgfsys@defobject{currentmarker}{\pgfqpoint{-0.027778in}{-0.027778in}}{\pgfqpoint{0.027778in}{0.027778in}}{%
\pgfpathmoveto{\pgfqpoint{0.000000in}{-0.027778in}}%
\pgfpathcurveto{\pgfqpoint{0.007367in}{-0.027778in}}{\pgfqpoint{0.014433in}{-0.024851in}}{\pgfqpoint{0.019642in}{-0.019642in}}%
\pgfpathcurveto{\pgfqpoint{0.024851in}{-0.014433in}}{\pgfqpoint{0.027778in}{-0.007367in}}{\pgfqpoint{0.027778in}{0.000000in}}%
\pgfpathcurveto{\pgfqpoint{0.027778in}{0.007367in}}{\pgfqpoint{0.024851in}{0.014433in}}{\pgfqpoint{0.019642in}{0.019642in}}%
\pgfpathcurveto{\pgfqpoint{0.014433in}{0.024851in}}{\pgfqpoint{0.007367in}{0.027778in}}{\pgfqpoint{0.000000in}{0.027778in}}%
\pgfpathcurveto{\pgfqpoint{-0.007367in}{0.027778in}}{\pgfqpoint{-0.014433in}{0.024851in}}{\pgfqpoint{-0.019642in}{0.019642in}}%
\pgfpathcurveto{\pgfqpoint{-0.024851in}{0.014433in}}{\pgfqpoint{-0.027778in}{0.007367in}}{\pgfqpoint{-0.027778in}{0.000000in}}%
\pgfpathcurveto{\pgfqpoint{-0.027778in}{-0.007367in}}{\pgfqpoint{-0.024851in}{-0.014433in}}{\pgfqpoint{-0.019642in}{-0.019642in}}%
\pgfpathcurveto{\pgfqpoint{-0.014433in}{-0.024851in}}{\pgfqpoint{-0.007367in}{-0.027778in}}{\pgfqpoint{0.000000in}{-0.027778in}}%
\pgfpathclose%
\pgfusepath{stroke,fill}%
}%
\begin{pgfscope}%
\pgfsys@transformshift{2.935987in}{0.506040in}%
\pgfsys@useobject{currentmarker}{}%
\end{pgfscope}%
\end{pgfscope}%
\begin{pgfscope}%
\pgfpathrectangle{\pgfqpoint{0.550713in}{0.093599in}}{\pgfqpoint{3.028919in}{1.684778in}}%
\pgfusepath{clip}%
\pgfsetrectcap%
\pgfsetroundjoin%
\pgfsetlinewidth{0.752812pt}%
\definecolor{currentstroke}{rgb}{0.000000,0.000000,0.000000}%
\pgfsetstrokecolor{currentstroke}%
\pgfsetdash{}{0pt}%
\pgfpathmoveto{\pgfqpoint{3.013224in}{0.529330in}}%
\pgfpathlineto{\pgfqpoint{3.161641in}{0.529330in}}%
\pgfusepath{stroke}%
\end{pgfscope}%
\begin{pgfscope}%
\pgfpathrectangle{\pgfqpoint{0.550713in}{0.093599in}}{\pgfqpoint{3.028919in}{1.684778in}}%
\pgfusepath{clip}%
\pgfsetbuttcap%
\pgfsetroundjoin%
\definecolor{currentfill}{rgb}{1.000000,1.000000,1.000000}%
\pgfsetfillcolor{currentfill}%
\pgfsetlinewidth{1.003750pt}%
\definecolor{currentstroke}{rgb}{0.000000,0.000000,0.000000}%
\pgfsetstrokecolor{currentstroke}%
\pgfsetdash{}{0pt}%
\pgfsys@defobject{currentmarker}{\pgfqpoint{-0.027778in}{-0.027778in}}{\pgfqpoint{0.027778in}{0.027778in}}{%
\pgfpathmoveto{\pgfqpoint{0.000000in}{-0.027778in}}%
\pgfpathcurveto{\pgfqpoint{0.007367in}{-0.027778in}}{\pgfqpoint{0.014433in}{-0.024851in}}{\pgfqpoint{0.019642in}{-0.019642in}}%
\pgfpathcurveto{\pgfqpoint{0.024851in}{-0.014433in}}{\pgfqpoint{0.027778in}{-0.007367in}}{\pgfqpoint{0.027778in}{0.000000in}}%
\pgfpathcurveto{\pgfqpoint{0.027778in}{0.007367in}}{\pgfqpoint{0.024851in}{0.014433in}}{\pgfqpoint{0.019642in}{0.019642in}}%
\pgfpathcurveto{\pgfqpoint{0.014433in}{0.024851in}}{\pgfqpoint{0.007367in}{0.027778in}}{\pgfqpoint{0.000000in}{0.027778in}}%
\pgfpathcurveto{\pgfqpoint{-0.007367in}{0.027778in}}{\pgfqpoint{-0.014433in}{0.024851in}}{\pgfqpoint{-0.019642in}{0.019642in}}%
\pgfpathcurveto{\pgfqpoint{-0.024851in}{0.014433in}}{\pgfqpoint{-0.027778in}{0.007367in}}{\pgfqpoint{-0.027778in}{0.000000in}}%
\pgfpathcurveto{\pgfqpoint{-0.027778in}{-0.007367in}}{\pgfqpoint{-0.024851in}{-0.014433in}}{\pgfqpoint{-0.019642in}{-0.019642in}}%
\pgfpathcurveto{\pgfqpoint{-0.014433in}{-0.024851in}}{\pgfqpoint{-0.007367in}{-0.027778in}}{\pgfqpoint{0.000000in}{-0.027778in}}%
\pgfpathclose%
\pgfusepath{stroke,fill}%
}%
\begin{pgfscope}%
\pgfsys@transformshift{3.087433in}{0.530128in}%
\pgfsys@useobject{currentmarker}{}%
\end{pgfscope}%
\end{pgfscope}%
\begin{pgfscope}%
\pgfpathrectangle{\pgfqpoint{0.550713in}{0.093599in}}{\pgfqpoint{3.028919in}{1.684778in}}%
\pgfusepath{clip}%
\pgfsetrectcap%
\pgfsetroundjoin%
\pgfsetlinewidth{0.752812pt}%
\definecolor{currentstroke}{rgb}{0.000000,0.000000,0.000000}%
\pgfsetstrokecolor{currentstroke}%
\pgfsetdash{}{0pt}%
\pgfpathmoveto{\pgfqpoint{3.240393in}{0.515447in}}%
\pgfpathlineto{\pgfqpoint{3.388810in}{0.515447in}}%
\pgfusepath{stroke}%
\end{pgfscope}%
\begin{pgfscope}%
\pgfpathrectangle{\pgfqpoint{0.550713in}{0.093599in}}{\pgfqpoint{3.028919in}{1.684778in}}%
\pgfusepath{clip}%
\pgfsetbuttcap%
\pgfsetroundjoin%
\definecolor{currentfill}{rgb}{1.000000,1.000000,1.000000}%
\pgfsetfillcolor{currentfill}%
\pgfsetlinewidth{1.003750pt}%
\definecolor{currentstroke}{rgb}{0.000000,0.000000,0.000000}%
\pgfsetstrokecolor{currentstroke}%
\pgfsetdash{}{0pt}%
\pgfsys@defobject{currentmarker}{\pgfqpoint{-0.027778in}{-0.027778in}}{\pgfqpoint{0.027778in}{0.027778in}}{%
\pgfpathmoveto{\pgfqpoint{0.000000in}{-0.027778in}}%
\pgfpathcurveto{\pgfqpoint{0.007367in}{-0.027778in}}{\pgfqpoint{0.014433in}{-0.024851in}}{\pgfqpoint{0.019642in}{-0.019642in}}%
\pgfpathcurveto{\pgfqpoint{0.024851in}{-0.014433in}}{\pgfqpoint{0.027778in}{-0.007367in}}{\pgfqpoint{0.027778in}{0.000000in}}%
\pgfpathcurveto{\pgfqpoint{0.027778in}{0.007367in}}{\pgfqpoint{0.024851in}{0.014433in}}{\pgfqpoint{0.019642in}{0.019642in}}%
\pgfpathcurveto{\pgfqpoint{0.014433in}{0.024851in}}{\pgfqpoint{0.007367in}{0.027778in}}{\pgfqpoint{0.000000in}{0.027778in}}%
\pgfpathcurveto{\pgfqpoint{-0.007367in}{0.027778in}}{\pgfqpoint{-0.014433in}{0.024851in}}{\pgfqpoint{-0.019642in}{0.019642in}}%
\pgfpathcurveto{\pgfqpoint{-0.024851in}{0.014433in}}{\pgfqpoint{-0.027778in}{0.007367in}}{\pgfqpoint{-0.027778in}{0.000000in}}%
\pgfpathcurveto{\pgfqpoint{-0.027778in}{-0.007367in}}{\pgfqpoint{-0.024851in}{-0.014433in}}{\pgfqpoint{-0.019642in}{-0.019642in}}%
\pgfpathcurveto{\pgfqpoint{-0.014433in}{-0.024851in}}{\pgfqpoint{-0.007367in}{-0.027778in}}{\pgfqpoint{0.000000in}{-0.027778in}}%
\pgfpathclose%
\pgfusepath{stroke,fill}%
}%
\begin{pgfscope}%
\pgfsys@transformshift{3.314601in}{0.514485in}%
\pgfsys@useobject{currentmarker}{}%
\end{pgfscope}%
\end{pgfscope}%
\begin{pgfscope}%
\pgfpathrectangle{\pgfqpoint{0.550713in}{0.093599in}}{\pgfqpoint{3.028919in}{1.684778in}}%
\pgfusepath{clip}%
\pgfsetrectcap%
\pgfsetroundjoin%
\pgfsetlinewidth{0.752812pt}%
\definecolor{currentstroke}{rgb}{0.000000,0.000000,0.000000}%
\pgfsetstrokecolor{currentstroke}%
\pgfsetdash{}{0pt}%
\pgfpathmoveto{\pgfqpoint{3.391839in}{0.527874in}}%
\pgfpathlineto{\pgfqpoint{3.540256in}{0.527874in}}%
\pgfusepath{stroke}%
\end{pgfscope}%
\begin{pgfscope}%
\pgfpathrectangle{\pgfqpoint{0.550713in}{0.093599in}}{\pgfqpoint{3.028919in}{1.684778in}}%
\pgfusepath{clip}%
\pgfsetbuttcap%
\pgfsetroundjoin%
\definecolor{currentfill}{rgb}{1.000000,1.000000,1.000000}%
\pgfsetfillcolor{currentfill}%
\pgfsetlinewidth{1.003750pt}%
\definecolor{currentstroke}{rgb}{0.000000,0.000000,0.000000}%
\pgfsetstrokecolor{currentstroke}%
\pgfsetdash{}{0pt}%
\pgfsys@defobject{currentmarker}{\pgfqpoint{-0.027778in}{-0.027778in}}{\pgfqpoint{0.027778in}{0.027778in}}{%
\pgfpathmoveto{\pgfqpoint{0.000000in}{-0.027778in}}%
\pgfpathcurveto{\pgfqpoint{0.007367in}{-0.027778in}}{\pgfqpoint{0.014433in}{-0.024851in}}{\pgfqpoint{0.019642in}{-0.019642in}}%
\pgfpathcurveto{\pgfqpoint{0.024851in}{-0.014433in}}{\pgfqpoint{0.027778in}{-0.007367in}}{\pgfqpoint{0.027778in}{0.000000in}}%
\pgfpathcurveto{\pgfqpoint{0.027778in}{0.007367in}}{\pgfqpoint{0.024851in}{0.014433in}}{\pgfqpoint{0.019642in}{0.019642in}}%
\pgfpathcurveto{\pgfqpoint{0.014433in}{0.024851in}}{\pgfqpoint{0.007367in}{0.027778in}}{\pgfqpoint{0.000000in}{0.027778in}}%
\pgfpathcurveto{\pgfqpoint{-0.007367in}{0.027778in}}{\pgfqpoint{-0.014433in}{0.024851in}}{\pgfqpoint{-0.019642in}{0.019642in}}%
\pgfpathcurveto{\pgfqpoint{-0.024851in}{0.014433in}}{\pgfqpoint{-0.027778in}{0.007367in}}{\pgfqpoint{-0.027778in}{0.000000in}}%
\pgfpathcurveto{\pgfqpoint{-0.027778in}{-0.007367in}}{\pgfqpoint{-0.024851in}{-0.014433in}}{\pgfqpoint{-0.019642in}{-0.019642in}}%
\pgfpathcurveto{\pgfqpoint{-0.014433in}{-0.024851in}}{\pgfqpoint{-0.007367in}{-0.027778in}}{\pgfqpoint{0.000000in}{-0.027778in}}%
\pgfpathclose%
\pgfusepath{stroke,fill}%
}%
\begin{pgfscope}%
\pgfsys@transformshift{3.466047in}{0.530850in}%
\pgfsys@useobject{currentmarker}{}%
\end{pgfscope}%
\end{pgfscope}%
\begin{pgfscope}%
\pgfsetrectcap%
\pgfsetmiterjoin%
\pgfsetlinewidth{0.803000pt}%
\definecolor{currentstroke}{rgb}{0.000000,0.000000,0.000000}%
\pgfsetstrokecolor{currentstroke}%
\pgfsetdash{}{0pt}%
\pgfpathmoveto{\pgfqpoint{0.550713in}{0.093599in}}%
\pgfpathlineto{\pgfqpoint{0.550713in}{1.778376in}}%
\pgfusepath{stroke}%
\end{pgfscope}%
\begin{pgfscope}%
\pgfsetrectcap%
\pgfsetmiterjoin%
\pgfsetlinewidth{0.803000pt}%
\definecolor{currentstroke}{rgb}{0.000000,0.000000,0.000000}%
\pgfsetstrokecolor{currentstroke}%
\pgfsetdash{}{0pt}%
\pgfpathmoveto{\pgfqpoint{3.579632in}{0.093599in}}%
\pgfpathlineto{\pgfqpoint{3.579632in}{1.778376in}}%
\pgfusepath{stroke}%
\end{pgfscope}%
\begin{pgfscope}%
\pgfsetrectcap%
\pgfsetmiterjoin%
\pgfsetlinewidth{0.803000pt}%
\definecolor{currentstroke}{rgb}{0.000000,0.000000,0.000000}%
\pgfsetstrokecolor{currentstroke}%
\pgfsetdash{}{0pt}%
\pgfpathmoveto{\pgfqpoint{0.550713in}{0.093599in}}%
\pgfpathlineto{\pgfqpoint{3.579632in}{0.093599in}}%
\pgfusepath{stroke}%
\end{pgfscope}%
\begin{pgfscope}%
\pgfsetrectcap%
\pgfsetmiterjoin%
\pgfsetlinewidth{0.803000pt}%
\definecolor{currentstroke}{rgb}{0.000000,0.000000,0.000000}%
\pgfsetstrokecolor{currentstroke}%
\pgfsetdash{}{0pt}%
\pgfpathmoveto{\pgfqpoint{0.550713in}{1.778376in}}%
\pgfpathlineto{\pgfqpoint{3.579632in}{1.778376in}}%
\pgfusepath{stroke}%
\end{pgfscope}%
\begin{pgfscope}%
\pgfsetbuttcap%
\pgfsetroundjoin%
\pgfsetlinewidth{1.003750pt}%
\definecolor{currentstroke}{rgb}{0.392157,0.396078,0.403922}%
\pgfsetstrokecolor{currentstroke}%
\pgfsetdash{{3.700000pt}{1.600000pt}}{0.000000pt}%
\pgfpathmoveto{\pgfqpoint{3.704632in}{1.724320in}}%
\pgfpathlineto{\pgfqpoint{3.982410in}{1.724320in}}%
\pgfusepath{stroke}%
\end{pgfscope}%
\begin{pgfscope}%
\definecolor{textcolor}{rgb}{0.000000,0.000000,0.000000}%
\pgfsetstrokecolor{textcolor}%
\pgfsetfillcolor{textcolor}%
\pgftext[x=4.093521in,y=1.675709in,left,base]{\color{textcolor}\rmfamily\fontsize{10.000000}{12.000000}\selectfont Only Exploitation}%
\end{pgfscope}%
\begin{pgfscope}%
\pgfsetbuttcap%
\pgfsetmiterjoin%
\definecolor{currentfill}{rgb}{0.631373,0.062745,0.207843}%
\pgfsetfillcolor{currentfill}%
\pgfsetlinewidth{0.000000pt}%
\definecolor{currentstroke}{rgb}{0.000000,0.000000,0.000000}%
\pgfsetstrokecolor{currentstroke}%
\pgfsetstrokeopacity{0.000000}%
\pgfsetdash{}{0pt}%
\pgfpathmoveto{\pgfqpoint{3.704632in}{1.480570in}}%
\pgfpathlineto{\pgfqpoint{3.982410in}{1.480570in}}%
\pgfpathlineto{\pgfqpoint{3.982410in}{1.577793in}}%
\pgfpathlineto{\pgfqpoint{3.704632in}{1.577793in}}%
\pgfpathclose%
\pgfusepath{fill}%
\end{pgfscope}%
\begin{pgfscope}%
\definecolor{textcolor}{rgb}{0.000000,0.000000,0.000000}%
\pgfsetstrokecolor{textcolor}%
\pgfsetfillcolor{textcolor}%
\pgftext[x=4.093521in,y=1.480570in,left,base]{\color{textcolor}\rmfamily\fontsize{10.000000}{12.000000}\selectfont TV-GP-UCB}%
\end{pgfscope}%
\begin{pgfscope}%
\pgfsetbuttcap%
\pgfsetmiterjoin%
\definecolor{currentfill}{rgb}{0.890196,0.000000,0.400000}%
\pgfsetfillcolor{currentfill}%
\pgfsetlinewidth{0.000000pt}%
\definecolor{currentstroke}{rgb}{0.000000,0.000000,0.000000}%
\pgfsetstrokecolor{currentstroke}%
\pgfsetstrokeopacity{0.000000}%
\pgfsetdash{}{0pt}%
\pgfpathmoveto{\pgfqpoint{3.704632in}{1.286959in}}%
\pgfpathlineto{\pgfqpoint{3.982410in}{1.286959in}}%
\pgfpathlineto{\pgfqpoint{3.982410in}{1.384182in}}%
\pgfpathlineto{\pgfqpoint{3.704632in}{1.384182in}}%
\pgfpathclose%
\pgfusepath{fill}%
\end{pgfscope}%
\begin{pgfscope}%
\definecolor{textcolor}{rgb}{0.000000,0.000000,0.000000}%
\pgfsetstrokecolor{textcolor}%
\pgfsetfillcolor{textcolor}%
\pgftext[x=4.093521in,y=1.286959in,left,base]{\color{textcolor}\rmfamily\fontsize{10.000000}{12.000000}\selectfont SW TV-GP-UCB}%
\end{pgfscope}%
\begin{pgfscope}%
\pgfsetbuttcap%
\pgfsetmiterjoin%
\definecolor{currentfill}{rgb}{0.000000,0.329412,0.623529}%
\pgfsetfillcolor{currentfill}%
\pgfsetlinewidth{0.000000pt}%
\definecolor{currentstroke}{rgb}{0.000000,0.000000,0.000000}%
\pgfsetstrokecolor{currentstroke}%
\pgfsetstrokeopacity{0.000000}%
\pgfsetdash{}{0pt}%
\pgfpathmoveto{\pgfqpoint{3.704632in}{1.093348in}}%
\pgfpathlineto{\pgfqpoint{3.982410in}{1.093348in}}%
\pgfpathlineto{\pgfqpoint{3.982410in}{1.190571in}}%
\pgfpathlineto{\pgfqpoint{3.704632in}{1.190571in}}%
\pgfpathclose%
\pgfusepath{fill}%
\end{pgfscope}%
\begin{pgfscope}%
\definecolor{textcolor}{rgb}{0.000000,0.000000,0.000000}%
\pgfsetstrokecolor{textcolor}%
\pgfsetfillcolor{textcolor}%
\pgftext[x=4.093521in,y=1.093348in,left,base]{\color{textcolor}\rmfamily\fontsize{10.000000}{12.000000}\selectfont UI-TVBO}%
\end{pgfscope}%
\begin{pgfscope}%
\pgfsetbuttcap%
\pgfsetmiterjoin%
\definecolor{currentfill}{rgb}{0.000000,0.380392,0.396078}%
\pgfsetfillcolor{currentfill}%
\pgfsetlinewidth{0.000000pt}%
\definecolor{currentstroke}{rgb}{0.000000,0.000000,0.000000}%
\pgfsetstrokecolor{currentstroke}%
\pgfsetstrokeopacity{0.000000}%
\pgfsetdash{}{0pt}%
\pgfpathmoveto{\pgfqpoint{3.704632in}{0.899737in}}%
\pgfpathlineto{\pgfqpoint{3.982410in}{0.899737in}}%
\pgfpathlineto{\pgfqpoint{3.982410in}{0.996960in}}%
\pgfpathlineto{\pgfqpoint{3.704632in}{0.996960in}}%
\pgfpathclose%
\pgfusepath{fill}%
\end{pgfscope}%
\begin{pgfscope}%
\definecolor{textcolor}{rgb}{0.000000,0.000000,0.000000}%
\pgfsetstrokecolor{textcolor}%
\pgfsetfillcolor{textcolor}%
\pgftext[x=4.093521in,y=0.899737in,left,base]{\color{textcolor}\rmfamily\fontsize{10.000000}{12.000000}\selectfont B UI-TVBO}%
\end{pgfscope}%
\begin{pgfscope}%
\pgfsetbuttcap%
\pgfsetmiterjoin%
\definecolor{currentfill}{rgb}{0.380392,0.129412,0.345098}%
\pgfsetfillcolor{currentfill}%
\pgfsetlinewidth{0.000000pt}%
\definecolor{currentstroke}{rgb}{0.000000,0.000000,0.000000}%
\pgfsetstrokecolor{currentstroke}%
\pgfsetstrokeopacity{0.000000}%
\pgfsetdash{}{0pt}%
\pgfpathmoveto{\pgfqpoint{3.704632in}{0.706127in}}%
\pgfpathlineto{\pgfqpoint{3.982410in}{0.706127in}}%
\pgfpathlineto{\pgfqpoint{3.982410in}{0.803349in}}%
\pgfpathlineto{\pgfqpoint{3.704632in}{0.803349in}}%
\pgfpathclose%
\pgfusepath{fill}%
\end{pgfscope}%
\begin{pgfscope}%
\definecolor{textcolor}{rgb}{0.000000,0.000000,0.000000}%
\pgfsetstrokecolor{textcolor}%
\pgfsetfillcolor{textcolor}%
\pgftext[x=4.093521in,y=0.706127in,left,base]{\color{textcolor}\rmfamily\fontsize{10.000000}{12.000000}\selectfont C-TV-GP-UCB}%
\end{pgfscope}%
\begin{pgfscope}%
\pgfsetbuttcap%
\pgfsetmiterjoin%
\definecolor{currentfill}{rgb}{0.964706,0.658824,0.000000}%
\pgfsetfillcolor{currentfill}%
\pgfsetlinewidth{0.000000pt}%
\definecolor{currentstroke}{rgb}{0.000000,0.000000,0.000000}%
\pgfsetstrokecolor{currentstroke}%
\pgfsetstrokeopacity{0.000000}%
\pgfsetdash{}{0pt}%
\pgfpathmoveto{\pgfqpoint{3.704632in}{0.512516in}}%
\pgfpathlineto{\pgfqpoint{3.982410in}{0.512516in}}%
\pgfpathlineto{\pgfqpoint{3.982410in}{0.609738in}}%
\pgfpathlineto{\pgfqpoint{3.704632in}{0.609738in}}%
\pgfpathclose%
\pgfusepath{fill}%
\end{pgfscope}%
\begin{pgfscope}%
\definecolor{textcolor}{rgb}{0.000000,0.000000,0.000000}%
\pgfsetstrokecolor{textcolor}%
\pgfsetfillcolor{textcolor}%
\pgftext[x=4.093521in,y=0.512516in,left,base]{\color{textcolor}\rmfamily\fontsize{10.000000}{12.000000}\selectfont SW C-TV-GP-UCB}%
\end{pgfscope}%
\begin{pgfscope}%
\pgfsetbuttcap%
\pgfsetmiterjoin%
\definecolor{currentfill}{rgb}{0.341176,0.670588,0.152941}%
\pgfsetfillcolor{currentfill}%
\pgfsetlinewidth{0.000000pt}%
\definecolor{currentstroke}{rgb}{0.000000,0.000000,0.000000}%
\pgfsetstrokecolor{currentstroke}%
\pgfsetstrokeopacity{0.000000}%
\pgfsetdash{}{0pt}%
\pgfpathmoveto{\pgfqpoint{3.704632in}{0.318905in}}%
\pgfpathlineto{\pgfqpoint{3.982410in}{0.318905in}}%
\pgfpathlineto{\pgfqpoint{3.982410in}{0.416127in}}%
\pgfpathlineto{\pgfqpoint{3.704632in}{0.416127in}}%
\pgfpathclose%
\pgfusepath{fill}%
\end{pgfscope}%
\begin{pgfscope}%
\definecolor{textcolor}{rgb}{0.000000,0.000000,0.000000}%
\pgfsetstrokecolor{textcolor}%
\pgfsetfillcolor{textcolor}%
\pgftext[x=4.093521in,y=0.318905in,left,base]{\color{textcolor}\rmfamily\fontsize{10.000000}{12.000000}\selectfont C-UI-TVBO}%
\end{pgfscope}%
\begin{pgfscope}%
\pgfsetbuttcap%
\pgfsetmiterjoin%
\definecolor{currentfill}{rgb}{0.478431,0.435294,0.674510}%
\pgfsetfillcolor{currentfill}%
\pgfsetlinewidth{0.000000pt}%
\definecolor{currentstroke}{rgb}{0.000000,0.000000,0.000000}%
\pgfsetstrokecolor{currentstroke}%
\pgfsetstrokeopacity{0.000000}%
\pgfsetdash{}{0pt}%
\pgfpathmoveto{\pgfqpoint{3.704632in}{0.125294in}}%
\pgfpathlineto{\pgfqpoint{3.982410in}{0.125294in}}%
\pgfpathlineto{\pgfqpoint{3.982410in}{0.222516in}}%
\pgfpathlineto{\pgfqpoint{3.704632in}{0.222516in}}%
\pgfpathclose%
\pgfusepath{fill}%
\end{pgfscope}%
\begin{pgfscope}%
\definecolor{textcolor}{rgb}{0.000000,0.000000,0.000000}%
\pgfsetstrokecolor{textcolor}%
\pgfsetfillcolor{textcolor}%
\pgftext[x=4.093521in,y=0.125294in,left,base]{\color{textcolor}\rmfamily\fontsize{10.000000}{12.000000}\selectfont B C-UI-TVBO}%
\end{pgfscope}%
\begin{pgfscope}%
\pgfsetbuttcap%
\pgfsetmiterjoin%
\definecolor{currentfill}{rgb}{1.000000,1.000000,1.000000}%
\pgfsetfillcolor{currentfill}%
\pgfsetlinewidth{1.003750pt}%
\definecolor{currentstroke}{rgb}{1.000000,1.000000,1.000000}%
\pgfsetstrokecolor{currentstroke}%
\pgfsetdash{}{0pt}%
\pgfpathmoveto{\pgfqpoint{2.529317in}{1.279920in}}%
\pgfpathlineto{\pgfqpoint{3.482410in}{1.279920in}}%
\pgfpathquadraticcurveto{\pgfqpoint{3.510187in}{1.279920in}}{\pgfqpoint{3.510187in}{1.307697in}}%
\pgfpathlineto{\pgfqpoint{3.510187in}{1.681154in}}%
\pgfpathquadraticcurveto{\pgfqpoint{3.510187in}{1.708932in}}{\pgfqpoint{3.482410in}{1.708932in}}%
\pgfpathlineto{\pgfqpoint{2.529317in}{1.708932in}}%
\pgfpathquadraticcurveto{\pgfqpoint{2.501540in}{1.708932in}}{\pgfqpoint{2.501540in}{1.681154in}}%
\pgfpathlineto{\pgfqpoint{2.501540in}{1.307697in}}%
\pgfpathquadraticcurveto{\pgfqpoint{2.501540in}{1.279920in}}{\pgfqpoint{2.529317in}{1.279920in}}%
\pgfpathclose%
\pgfusepath{stroke,fill}%
\end{pgfscope}%
\begin{pgfscope}%
\pgfsetbuttcap%
\pgfsetmiterjoin%
\definecolor{currentfill}{rgb}{0.000000,0.000000,0.000000}%
\pgfsetfillcolor{currentfill}%
\pgfsetlinewidth{0.000000pt}%
\definecolor{currentstroke}{rgb}{0.000000,0.000000,0.000000}%
\pgfsetstrokecolor{currentstroke}%
\pgfsetstrokeopacity{0.000000}%
\pgfsetdash{}{0pt}%
\pgfpathmoveto{\pgfqpoint{2.557095in}{1.556154in}}%
\pgfpathlineto{\pgfqpoint{2.834873in}{1.556154in}}%
\pgfpathlineto{\pgfqpoint{2.834873in}{1.653376in}}%
\pgfpathlineto{\pgfqpoint{2.557095in}{1.653376in}}%
\pgfpathclose%
\pgfusepath{fill}%
\end{pgfscope}%
\begin{pgfscope}%
\definecolor{textcolor}{rgb}{0.000000,0.000000,0.000000}%
\pgfsetstrokecolor{textcolor}%
\pgfsetfillcolor{textcolor}%
\pgftext[x=2.945984in,y=1.556154in,left,base]{\color{textcolor}\rmfamily\fontsize{10.000000}{12.000000}\selectfont \(\displaystyle \mu_0=0\)}%
\end{pgfscope}%
\begin{pgfscope}%
\pgfsetbuttcap%
\pgfsetmiterjoin%
\definecolor{currentfill}{rgb}{0.811765,0.819608,0.823529}%
\pgfsetfillcolor{currentfill}%
\pgfsetlinewidth{0.000000pt}%
\definecolor{currentstroke}{rgb}{0.000000,0.000000,0.000000}%
\pgfsetstrokecolor{currentstroke}%
\pgfsetstrokeopacity{0.000000}%
\pgfsetdash{}{0pt}%
\pgfpathmoveto{\pgfqpoint{2.557095in}{1.362481in}}%
\pgfpathlineto{\pgfqpoint{2.834873in}{1.362481in}}%
\pgfpathlineto{\pgfqpoint{2.834873in}{1.459704in}}%
\pgfpathlineto{\pgfqpoint{2.557095in}{1.459704in}}%
\pgfpathclose%
\pgfusepath{fill}%
\end{pgfscope}%
\begin{pgfscope}%
\definecolor{textcolor}{rgb}{0.000000,0.000000,0.000000}%
\pgfsetstrokecolor{textcolor}%
\pgfsetfillcolor{textcolor}%
\pgftext[x=2.945984in,y=1.362481in,left,base]{\color{textcolor}\rmfamily\fontsize{10.000000}{12.000000}\selectfont \(\displaystyle \mu_0=-2\)}%
\end{pgfscope}%
\begin{pgfscope}%
\pgfsetbuttcap%
\pgfsetmiterjoin%
\definecolor{currentfill}{rgb}{1.000000,1.000000,1.000000}%
\pgfsetfillcolor{currentfill}%
\pgfsetlinewidth{1.003750pt}%
\definecolor{currentstroke}{rgb}{1.000000,1.000000,1.000000}%
\pgfsetstrokecolor{currentstroke}%
\pgfsetdash{}{0pt}%
\pgfpathmoveto{\pgfqpoint{2.529317in}{1.279920in}}%
\pgfpathlineto{\pgfqpoint{3.482410in}{1.279920in}}%
\pgfpathquadraticcurveto{\pgfqpoint{3.510187in}{1.279920in}}{\pgfqpoint{3.510187in}{1.307697in}}%
\pgfpathlineto{\pgfqpoint{3.510187in}{1.681154in}}%
\pgfpathquadraticcurveto{\pgfqpoint{3.510187in}{1.708932in}}{\pgfqpoint{3.482410in}{1.708932in}}%
\pgfpathlineto{\pgfqpoint{2.529317in}{1.708932in}}%
\pgfpathquadraticcurveto{\pgfqpoint{2.501540in}{1.708932in}}{\pgfqpoint{2.501540in}{1.681154in}}%
\pgfpathlineto{\pgfqpoint{2.501540in}{1.307697in}}%
\pgfpathquadraticcurveto{\pgfqpoint{2.501540in}{1.279920in}}{\pgfqpoint{2.529317in}{1.279920in}}%
\pgfpathclose%
\pgfusepath{stroke,fill}%
\end{pgfscope}%
\begin{pgfscope}%
\pgfsetbuttcap%
\pgfsetmiterjoin%
\definecolor{currentfill}{rgb}{0.000000,0.000000,0.000000}%
\pgfsetfillcolor{currentfill}%
\pgfsetlinewidth{0.000000pt}%
\definecolor{currentstroke}{rgb}{0.000000,0.000000,0.000000}%
\pgfsetstrokecolor{currentstroke}%
\pgfsetstrokeopacity{0.000000}%
\pgfsetdash{}{0pt}%
\pgfpathmoveto{\pgfqpoint{2.557095in}{1.556154in}}%
\pgfpathlineto{\pgfqpoint{2.834873in}{1.556154in}}%
\pgfpathlineto{\pgfqpoint{2.834873in}{1.653376in}}%
\pgfpathlineto{\pgfqpoint{2.557095in}{1.653376in}}%
\pgfpathclose%
\pgfusepath{fill}%
\end{pgfscope}%
\begin{pgfscope}%
\definecolor{textcolor}{rgb}{0.000000,0.000000,0.000000}%
\pgfsetstrokecolor{textcolor}%
\pgfsetfillcolor{textcolor}%
\pgftext[x=2.945984in,y=1.556154in,left,base]{\color{textcolor}\rmfamily\fontsize{10.000000}{12.000000}\selectfont \(\displaystyle \mu_0=0\)}%
\end{pgfscope}%
\begin{pgfscope}%
\pgfsetbuttcap%
\pgfsetmiterjoin%
\definecolor{currentfill}{rgb}{0.811765,0.819608,0.823529}%
\pgfsetfillcolor{currentfill}%
\pgfsetlinewidth{0.000000pt}%
\definecolor{currentstroke}{rgb}{0.000000,0.000000,0.000000}%
\pgfsetstrokecolor{currentstroke}%
\pgfsetstrokeopacity{0.000000}%
\pgfsetdash{}{0pt}%
\pgfpathmoveto{\pgfqpoint{2.557095in}{1.362481in}}%
\pgfpathlineto{\pgfqpoint{2.834873in}{1.362481in}}%
\pgfpathlineto{\pgfqpoint{2.834873in}{1.459704in}}%
\pgfpathlineto{\pgfqpoint{2.557095in}{1.459704in}}%
\pgfpathclose%
\pgfusepath{fill}%
\end{pgfscope}%
\begin{pgfscope}%
\definecolor{textcolor}{rgb}{0.000000,0.000000,0.000000}%
\pgfsetstrokecolor{textcolor}%
\pgfsetfillcolor{textcolor}%
\pgftext[x=2.945984in,y=1.362481in,left,base]{\color{textcolor}\rmfamily\fontsize{10.000000}{12.000000}\selectfont \(\displaystyle \mu_0=-2\)}%
\end{pgfscope}%
\end{pgfpicture}%
\makeatother%
\endgroup%

    \caption[Results of the one-dimensional out-of-model comparison.]{Results for the one-dimensional out-of-model comparison. It shows lower regret and a smaller regret variance for \gls{ctvbo} using \gls{ui} forgetting. The formatting is as in Figure~\ref{fig:WMC_cumulative_regret_1D}.}
    \label{fig:OOMC_cumulative_regret_1D}
\end{figure}

Figure~\ref{fig:OOMC_lengthscales_1D} shows the mean learned length scales over time. All variations learn length scales, which are greater than the true length scale of the objective function. This is expected, as the bounds on the second derivative of the objective functions where $\nicefrac{\partial^2 f_t}{\partial x^2} \in [0, 1]$. The objective functions are therefore very flat, and over-estimating the smoothness of the function is likely.
\begin{figure}[h]
    \centering
    %% Creator: Matplotlib, PGF backend
%%
%% To include the figure in your LaTeX document, write
%%   \input{<filename>.pgf}
%%
%% Make sure the required packages are loaded in your preamble
%%   \usepackage{pgf}
%%
%% Figures using additional raster images can only be included by \input if
%% they are in the same directory as the main LaTeX file. For loading figures
%% from other directories you can use the `import` package
%%   \usepackage{import}
%%
%% and then include the figures with
%%   \import{<path to file>}{<filename>.pgf}
%%
%% Matplotlib used the following preamble
%%   \usepackage{fontspec}
%%
\begingroup%
\makeatletter%
\begin{pgfpicture}%
\pgfpathrectangle{\pgfpointorigin}{\pgfqpoint{5.507126in}{2.042155in}}%
\pgfusepath{use as bounding box, clip}%
\begin{pgfscope}%
\pgfsetbuttcap%
\pgfsetmiterjoin%
\definecolor{currentfill}{rgb}{1.000000,1.000000,1.000000}%
\pgfsetfillcolor{currentfill}%
\pgfsetlinewidth{0.000000pt}%
\definecolor{currentstroke}{rgb}{1.000000,1.000000,1.000000}%
\pgfsetstrokecolor{currentstroke}%
\pgfsetdash{}{0pt}%
\pgfpathmoveto{\pgfqpoint{0.000000in}{0.000000in}}%
\pgfpathlineto{\pgfqpoint{5.507126in}{0.000000in}}%
\pgfpathlineto{\pgfqpoint{5.507126in}{2.042155in}}%
\pgfpathlineto{\pgfqpoint{0.000000in}{2.042155in}}%
\pgfpathclose%
\pgfusepath{fill}%
\end{pgfscope}%
\begin{pgfscope}%
\pgfsetbuttcap%
\pgfsetmiterjoin%
\definecolor{currentfill}{rgb}{1.000000,1.000000,1.000000}%
\pgfsetfillcolor{currentfill}%
\pgfsetlinewidth{0.000000pt}%
\definecolor{currentstroke}{rgb}{0.000000,0.000000,0.000000}%
\pgfsetstrokecolor{currentstroke}%
\pgfsetstrokeopacity{0.000000}%
\pgfsetdash{}{0pt}%
\pgfpathmoveto{\pgfqpoint{0.550713in}{0.408431in}}%
\pgfpathlineto{\pgfqpoint{3.689774in}{0.408431in}}%
\pgfpathlineto{\pgfqpoint{3.689774in}{1.899204in}}%
\pgfpathlineto{\pgfqpoint{0.550713in}{1.899204in}}%
\pgfpathclose%
\pgfusepath{fill}%
\end{pgfscope}%
\begin{pgfscope}%
\pgfsetbuttcap%
\pgfsetroundjoin%
\definecolor{currentfill}{rgb}{0.000000,0.000000,0.000000}%
\pgfsetfillcolor{currentfill}%
\pgfsetlinewidth{0.803000pt}%
\definecolor{currentstroke}{rgb}{0.000000,0.000000,0.000000}%
\pgfsetstrokecolor{currentstroke}%
\pgfsetdash{}{0pt}%
\pgfsys@defobject{currentmarker}{\pgfqpoint{0.000000in}{-0.048611in}}{\pgfqpoint{0.000000in}{0.000000in}}{%
\pgfpathmoveto{\pgfqpoint{0.000000in}{0.000000in}}%
\pgfpathlineto{\pgfqpoint{0.000000in}{-0.048611in}}%
\pgfusepath{stroke,fill}%
}%
\begin{pgfscope}%
\pgfsys@transformshift{0.936211in}{0.408431in}%
\pgfsys@useobject{currentmarker}{}%
\end{pgfscope}%
\end{pgfscope}%
\begin{pgfscope}%
\definecolor{textcolor}{rgb}{0.000000,0.000000,0.000000}%
\pgfsetstrokecolor{textcolor}%
\pgfsetfillcolor{textcolor}%
\pgftext[x=0.936211in,y=0.311209in,,top]{\color{textcolor}\rmfamily\fontsize{10.000000}{12.000000}\selectfont \(\displaystyle {50}\)}%
\end{pgfscope}%
\begin{pgfscope}%
\pgfsetbuttcap%
\pgfsetroundjoin%
\definecolor{currentfill}{rgb}{0.000000,0.000000,0.000000}%
\pgfsetfillcolor{currentfill}%
\pgfsetlinewidth{0.803000pt}%
\definecolor{currentstroke}{rgb}{0.000000,0.000000,0.000000}%
\pgfsetstrokecolor{currentstroke}%
\pgfsetdash{}{0pt}%
\pgfsys@defobject{currentmarker}{\pgfqpoint{0.000000in}{-0.048611in}}{\pgfqpoint{0.000000in}{0.000000in}}{%
\pgfpathmoveto{\pgfqpoint{0.000000in}{0.000000in}}%
\pgfpathlineto{\pgfqpoint{0.000000in}{-0.048611in}}%
\pgfusepath{stroke,fill}%
}%
\begin{pgfscope}%
\pgfsys@transformshift{1.486924in}{0.408431in}%
\pgfsys@useobject{currentmarker}{}%
\end{pgfscope}%
\end{pgfscope}%
\begin{pgfscope}%
\definecolor{textcolor}{rgb}{0.000000,0.000000,0.000000}%
\pgfsetstrokecolor{textcolor}%
\pgfsetfillcolor{textcolor}%
\pgftext[x=1.486924in,y=0.311209in,,top]{\color{textcolor}\rmfamily\fontsize{10.000000}{12.000000}\selectfont \(\displaystyle {100}\)}%
\end{pgfscope}%
\begin{pgfscope}%
\pgfsetbuttcap%
\pgfsetroundjoin%
\definecolor{currentfill}{rgb}{0.000000,0.000000,0.000000}%
\pgfsetfillcolor{currentfill}%
\pgfsetlinewidth{0.803000pt}%
\definecolor{currentstroke}{rgb}{0.000000,0.000000,0.000000}%
\pgfsetstrokecolor{currentstroke}%
\pgfsetdash{}{0pt}%
\pgfsys@defobject{currentmarker}{\pgfqpoint{0.000000in}{-0.048611in}}{\pgfqpoint{0.000000in}{0.000000in}}{%
\pgfpathmoveto{\pgfqpoint{0.000000in}{0.000000in}}%
\pgfpathlineto{\pgfqpoint{0.000000in}{-0.048611in}}%
\pgfusepath{stroke,fill}%
}%
\begin{pgfscope}%
\pgfsys@transformshift{2.037637in}{0.408431in}%
\pgfsys@useobject{currentmarker}{}%
\end{pgfscope}%
\end{pgfscope}%
\begin{pgfscope}%
\definecolor{textcolor}{rgb}{0.000000,0.000000,0.000000}%
\pgfsetstrokecolor{textcolor}%
\pgfsetfillcolor{textcolor}%
\pgftext[x=2.037637in,y=0.311209in,,top]{\color{textcolor}\rmfamily\fontsize{10.000000}{12.000000}\selectfont \(\displaystyle {150}\)}%
\end{pgfscope}%
\begin{pgfscope}%
\pgfsetbuttcap%
\pgfsetroundjoin%
\definecolor{currentfill}{rgb}{0.000000,0.000000,0.000000}%
\pgfsetfillcolor{currentfill}%
\pgfsetlinewidth{0.803000pt}%
\definecolor{currentstroke}{rgb}{0.000000,0.000000,0.000000}%
\pgfsetstrokecolor{currentstroke}%
\pgfsetdash{}{0pt}%
\pgfsys@defobject{currentmarker}{\pgfqpoint{0.000000in}{-0.048611in}}{\pgfqpoint{0.000000in}{0.000000in}}{%
\pgfpathmoveto{\pgfqpoint{0.000000in}{0.000000in}}%
\pgfpathlineto{\pgfqpoint{0.000000in}{-0.048611in}}%
\pgfusepath{stroke,fill}%
}%
\begin{pgfscope}%
\pgfsys@transformshift{2.588349in}{0.408431in}%
\pgfsys@useobject{currentmarker}{}%
\end{pgfscope}%
\end{pgfscope}%
\begin{pgfscope}%
\definecolor{textcolor}{rgb}{0.000000,0.000000,0.000000}%
\pgfsetstrokecolor{textcolor}%
\pgfsetfillcolor{textcolor}%
\pgftext[x=2.588349in,y=0.311209in,,top]{\color{textcolor}\rmfamily\fontsize{10.000000}{12.000000}\selectfont \(\displaystyle {200}\)}%
\end{pgfscope}%
\begin{pgfscope}%
\pgfsetbuttcap%
\pgfsetroundjoin%
\definecolor{currentfill}{rgb}{0.000000,0.000000,0.000000}%
\pgfsetfillcolor{currentfill}%
\pgfsetlinewidth{0.803000pt}%
\definecolor{currentstroke}{rgb}{0.000000,0.000000,0.000000}%
\pgfsetstrokecolor{currentstroke}%
\pgfsetdash{}{0pt}%
\pgfsys@defobject{currentmarker}{\pgfqpoint{0.000000in}{-0.048611in}}{\pgfqpoint{0.000000in}{0.000000in}}{%
\pgfpathmoveto{\pgfqpoint{0.000000in}{0.000000in}}%
\pgfpathlineto{\pgfqpoint{0.000000in}{-0.048611in}}%
\pgfusepath{stroke,fill}%
}%
\begin{pgfscope}%
\pgfsys@transformshift{3.139062in}{0.408431in}%
\pgfsys@useobject{currentmarker}{}%
\end{pgfscope}%
\end{pgfscope}%
\begin{pgfscope}%
\definecolor{textcolor}{rgb}{0.000000,0.000000,0.000000}%
\pgfsetstrokecolor{textcolor}%
\pgfsetfillcolor{textcolor}%
\pgftext[x=3.139062in,y=0.311209in,,top]{\color{textcolor}\rmfamily\fontsize{10.000000}{12.000000}\selectfont \(\displaystyle {250}\)}%
\end{pgfscope}%
\begin{pgfscope}%
\pgfsetbuttcap%
\pgfsetroundjoin%
\definecolor{currentfill}{rgb}{0.000000,0.000000,0.000000}%
\pgfsetfillcolor{currentfill}%
\pgfsetlinewidth{0.803000pt}%
\definecolor{currentstroke}{rgb}{0.000000,0.000000,0.000000}%
\pgfsetstrokecolor{currentstroke}%
\pgfsetdash{}{0pt}%
\pgfsys@defobject{currentmarker}{\pgfqpoint{0.000000in}{-0.048611in}}{\pgfqpoint{0.000000in}{0.000000in}}{%
\pgfpathmoveto{\pgfqpoint{0.000000in}{0.000000in}}%
\pgfpathlineto{\pgfqpoint{0.000000in}{-0.048611in}}%
\pgfusepath{stroke,fill}%
}%
\begin{pgfscope}%
\pgfsys@transformshift{3.689774in}{0.408431in}%
\pgfsys@useobject{currentmarker}{}%
\end{pgfscope}%
\end{pgfscope}%
\begin{pgfscope}%
\definecolor{textcolor}{rgb}{0.000000,0.000000,0.000000}%
\pgfsetstrokecolor{textcolor}%
\pgfsetfillcolor{textcolor}%
\pgftext[x=3.689774in,y=0.311209in,,top]{\color{textcolor}\rmfamily\fontsize{10.000000}{12.000000}\selectfont \(\displaystyle {300}\)}%
\end{pgfscope}%
\begin{pgfscope}%
\definecolor{textcolor}{rgb}{0.000000,0.000000,0.000000}%
\pgfsetstrokecolor{textcolor}%
\pgfsetfillcolor{textcolor}%
\pgftext[x=2.120244in,y=0.132320in,,top]{\color{textcolor}\rmfamily\fontsize{10.000000}{12.000000}\selectfont \(\displaystyle t\)}%
\end{pgfscope}%
\begin{pgfscope}%
\pgfsetbuttcap%
\pgfsetroundjoin%
\definecolor{currentfill}{rgb}{0.000000,0.000000,0.000000}%
\pgfsetfillcolor{currentfill}%
\pgfsetlinewidth{0.803000pt}%
\definecolor{currentstroke}{rgb}{0.000000,0.000000,0.000000}%
\pgfsetstrokecolor{currentstroke}%
\pgfsetdash{}{0pt}%
\pgfsys@defobject{currentmarker}{\pgfqpoint{-0.048611in}{0.000000in}}{\pgfqpoint{-0.000000in}{0.000000in}}{%
\pgfpathmoveto{\pgfqpoint{-0.000000in}{0.000000in}}%
\pgfpathlineto{\pgfqpoint{-0.048611in}{0.000000in}}%
\pgfusepath{stroke,fill}%
}%
\begin{pgfscope}%
\pgfsys@transformshift{0.550713in}{0.587324in}%
\pgfsys@useobject{currentmarker}{}%
\end{pgfscope}%
\end{pgfscope}%
\begin{pgfscope}%
\definecolor{textcolor}{rgb}{0.000000,0.000000,0.000000}%
\pgfsetstrokecolor{textcolor}%
\pgfsetfillcolor{textcolor}%
\pgftext[x=0.276021in, y=0.539129in, left, base]{\color{textcolor}\rmfamily\fontsize{10.000000}{12.000000}\selectfont \(\displaystyle {3.0}\)}%
\end{pgfscope}%
\begin{pgfscope}%
\pgfsetbuttcap%
\pgfsetroundjoin%
\definecolor{currentfill}{rgb}{0.000000,0.000000,0.000000}%
\pgfsetfillcolor{currentfill}%
\pgfsetlinewidth{0.803000pt}%
\definecolor{currentstroke}{rgb}{0.000000,0.000000,0.000000}%
\pgfsetstrokecolor{currentstroke}%
\pgfsetdash{}{0pt}%
\pgfsys@defobject{currentmarker}{\pgfqpoint{-0.048611in}{0.000000in}}{\pgfqpoint{-0.000000in}{0.000000in}}{%
\pgfpathmoveto{\pgfqpoint{-0.000000in}{0.000000in}}%
\pgfpathlineto{\pgfqpoint{-0.048611in}{0.000000in}}%
\pgfusepath{stroke,fill}%
}%
\begin{pgfscope}%
\pgfsys@transformshift{0.550713in}{0.885478in}%
\pgfsys@useobject{currentmarker}{}%
\end{pgfscope}%
\end{pgfscope}%
\begin{pgfscope}%
\definecolor{textcolor}{rgb}{0.000000,0.000000,0.000000}%
\pgfsetstrokecolor{textcolor}%
\pgfsetfillcolor{textcolor}%
\pgftext[x=0.276021in, y=0.837284in, left, base]{\color{textcolor}\rmfamily\fontsize{10.000000}{12.000000}\selectfont \(\displaystyle {3.5}\)}%
\end{pgfscope}%
\begin{pgfscope}%
\pgfsetbuttcap%
\pgfsetroundjoin%
\definecolor{currentfill}{rgb}{0.000000,0.000000,0.000000}%
\pgfsetfillcolor{currentfill}%
\pgfsetlinewidth{0.803000pt}%
\definecolor{currentstroke}{rgb}{0.000000,0.000000,0.000000}%
\pgfsetstrokecolor{currentstroke}%
\pgfsetdash{}{0pt}%
\pgfsys@defobject{currentmarker}{\pgfqpoint{-0.048611in}{0.000000in}}{\pgfqpoint{-0.000000in}{0.000000in}}{%
\pgfpathmoveto{\pgfqpoint{-0.000000in}{0.000000in}}%
\pgfpathlineto{\pgfqpoint{-0.048611in}{0.000000in}}%
\pgfusepath{stroke,fill}%
}%
\begin{pgfscope}%
\pgfsys@transformshift{0.550713in}{1.183633in}%
\pgfsys@useobject{currentmarker}{}%
\end{pgfscope}%
\end{pgfscope}%
\begin{pgfscope}%
\definecolor{textcolor}{rgb}{0.000000,0.000000,0.000000}%
\pgfsetstrokecolor{textcolor}%
\pgfsetfillcolor{textcolor}%
\pgftext[x=0.276021in, y=1.135438in, left, base]{\color{textcolor}\rmfamily\fontsize{10.000000}{12.000000}\selectfont \(\displaystyle {4.0}\)}%
\end{pgfscope}%
\begin{pgfscope}%
\pgfsetbuttcap%
\pgfsetroundjoin%
\definecolor{currentfill}{rgb}{0.000000,0.000000,0.000000}%
\pgfsetfillcolor{currentfill}%
\pgfsetlinewidth{0.803000pt}%
\definecolor{currentstroke}{rgb}{0.000000,0.000000,0.000000}%
\pgfsetstrokecolor{currentstroke}%
\pgfsetdash{}{0pt}%
\pgfsys@defobject{currentmarker}{\pgfqpoint{-0.048611in}{0.000000in}}{\pgfqpoint{-0.000000in}{0.000000in}}{%
\pgfpathmoveto{\pgfqpoint{-0.000000in}{0.000000in}}%
\pgfpathlineto{\pgfqpoint{-0.048611in}{0.000000in}}%
\pgfusepath{stroke,fill}%
}%
\begin{pgfscope}%
\pgfsys@transformshift{0.550713in}{1.481787in}%
\pgfsys@useobject{currentmarker}{}%
\end{pgfscope}%
\end{pgfscope}%
\begin{pgfscope}%
\definecolor{textcolor}{rgb}{0.000000,0.000000,0.000000}%
\pgfsetstrokecolor{textcolor}%
\pgfsetfillcolor{textcolor}%
\pgftext[x=0.276021in, y=1.433593in, left, base]{\color{textcolor}\rmfamily\fontsize{10.000000}{12.000000}\selectfont \(\displaystyle {4.5}\)}%
\end{pgfscope}%
\begin{pgfscope}%
\pgfsetbuttcap%
\pgfsetroundjoin%
\definecolor{currentfill}{rgb}{0.000000,0.000000,0.000000}%
\pgfsetfillcolor{currentfill}%
\pgfsetlinewidth{0.803000pt}%
\definecolor{currentstroke}{rgb}{0.000000,0.000000,0.000000}%
\pgfsetstrokecolor{currentstroke}%
\pgfsetdash{}{0pt}%
\pgfsys@defobject{currentmarker}{\pgfqpoint{-0.048611in}{0.000000in}}{\pgfqpoint{-0.000000in}{0.000000in}}{%
\pgfpathmoveto{\pgfqpoint{-0.000000in}{0.000000in}}%
\pgfpathlineto{\pgfqpoint{-0.048611in}{0.000000in}}%
\pgfusepath{stroke,fill}%
}%
\begin{pgfscope}%
\pgfsys@transformshift{0.550713in}{1.779942in}%
\pgfsys@useobject{currentmarker}{}%
\end{pgfscope}%
\end{pgfscope}%
\begin{pgfscope}%
\definecolor{textcolor}{rgb}{0.000000,0.000000,0.000000}%
\pgfsetstrokecolor{textcolor}%
\pgfsetfillcolor{textcolor}%
\pgftext[x=0.276021in, y=1.731748in, left, base]{\color{textcolor}\rmfamily\fontsize{10.000000}{12.000000}\selectfont \(\displaystyle {5.0}\)}%
\end{pgfscope}%
\begin{pgfscope}%
\definecolor{textcolor}{rgb}{0.000000,0.000000,0.000000}%
\pgfsetstrokecolor{textcolor}%
\pgfsetfillcolor{textcolor}%
\pgftext[x=0.220465in,y=1.153817in,,bottom,rotate=90.000000]{\color{textcolor}\rmfamily\fontsize{10.000000}{12.000000}\selectfont \(\displaystyle \Lambda_{11}\)}%
\end{pgfscope}%
\begin{pgfscope}%
\pgfpathrectangle{\pgfqpoint{0.550713in}{0.408431in}}{\pgfqpoint{3.139062in}{1.490773in}}%
\pgfusepath{clip}%
\pgfsetbuttcap%
\pgfsetroundjoin%
\pgfsetlinewidth{0.853187pt}%
\definecolor{currentstroke}{rgb}{0.392157,0.396078,0.403922}%
\pgfsetstrokecolor{currentstroke}%
\pgfsetdash{}{0pt}%
\pgfpathmoveto{\pgfqpoint{0.540713in}{0.587324in}}%
\pgfpathlineto{\pgfqpoint{3.689774in}{0.587324in}}%
\pgfusepath{stroke}%
\end{pgfscope}%
\begin{pgfscope}%
\pgfpathrectangle{\pgfqpoint{0.550713in}{0.408431in}}{\pgfqpoint{3.139062in}{1.490773in}}%
\pgfusepath{clip}%
\pgfsetbuttcap%
\pgfsetroundjoin%
\pgfsetlinewidth{0.853187pt}%
\definecolor{currentstroke}{rgb}{0.392157,0.396078,0.403922}%
\pgfsetstrokecolor{currentstroke}%
\pgfsetdash{{3.145000pt}{1.360000pt}}{0.000000pt}%
\pgfusepath{stroke}%
\end{pgfscope}%
\begin{pgfscope}%
\pgfpathrectangle{\pgfqpoint{0.550713in}{0.408431in}}{\pgfqpoint{3.139062in}{1.490773in}}%
\pgfusepath{clip}%
\pgfsetbuttcap%
\pgfsetroundjoin%
\pgfsetlinewidth{0.853187pt}%
\definecolor{currentstroke}{rgb}{0.392157,0.396078,0.403922}%
\pgfsetstrokecolor{currentstroke}%
\pgfsetdash{{3.145000pt}{1.360000pt}}{0.000000pt}%
\pgfpathmoveto{\pgfqpoint{0.540713in}{1.779942in}}%
\pgfpathlineto{\pgfqpoint{3.689774in}{1.779942in}}%
\pgfusepath{stroke}%
\end{pgfscope}%
\begin{pgfscope}%
\pgfpathrectangle{\pgfqpoint{0.550713in}{0.408431in}}{\pgfqpoint{3.139062in}{1.490773in}}%
\pgfusepath{clip}%
\pgfsetrectcap%
\pgfsetroundjoin%
\pgfsetlinewidth{0.853187pt}%
\definecolor{currentstroke}{rgb}{0.631373,0.062745,0.207843}%
\pgfsetstrokecolor{currentstroke}%
\pgfsetdash{}{0pt}%
\pgfpathmoveto{\pgfqpoint{0.550713in}{1.264471in}}%
\pgfpathlineto{\pgfqpoint{0.561727in}{1.328869in}}%
\pgfpathlineto{\pgfqpoint{0.572741in}{1.330374in}}%
\pgfpathlineto{\pgfqpoint{0.594770in}{1.338330in}}%
\pgfpathlineto{\pgfqpoint{0.605784in}{1.339970in}}%
\pgfpathlineto{\pgfqpoint{0.627812in}{1.369970in}}%
\pgfpathlineto{\pgfqpoint{0.638827in}{1.391358in}}%
\pgfpathlineto{\pgfqpoint{0.660855in}{1.424908in}}%
\pgfpathlineto{\pgfqpoint{0.671869in}{1.422294in}}%
\pgfpathlineto{\pgfqpoint{0.682884in}{1.432844in}}%
\pgfpathlineto{\pgfqpoint{0.704912in}{1.458491in}}%
\pgfpathlineto{\pgfqpoint{0.726941in}{1.475952in}}%
\pgfpathlineto{\pgfqpoint{0.737955in}{1.488269in}}%
\pgfpathlineto{\pgfqpoint{0.748969in}{1.497562in}}%
\pgfpathlineto{\pgfqpoint{0.759983in}{1.501132in}}%
\pgfpathlineto{\pgfqpoint{0.782012in}{1.534812in}}%
\pgfpathlineto{\pgfqpoint{0.793026in}{1.544784in}}%
\pgfpathlineto{\pgfqpoint{0.804040in}{1.550629in}}%
\pgfpathlineto{\pgfqpoint{0.826069in}{1.556732in}}%
\pgfpathlineto{\pgfqpoint{0.837083in}{1.562225in}}%
\pgfpathlineto{\pgfqpoint{0.848097in}{1.578465in}}%
\pgfpathlineto{\pgfqpoint{0.859112in}{1.592220in}}%
\pgfpathlineto{\pgfqpoint{0.870126in}{1.599783in}}%
\pgfpathlineto{\pgfqpoint{0.881140in}{1.604432in}}%
\pgfpathlineto{\pgfqpoint{0.892154in}{1.615820in}}%
\pgfpathlineto{\pgfqpoint{0.903169in}{1.622873in}}%
\pgfpathlineto{\pgfqpoint{0.925197in}{1.640066in}}%
\pgfpathlineto{\pgfqpoint{0.936211in}{1.651085in}}%
\pgfpathlineto{\pgfqpoint{0.947226in}{1.657405in}}%
\pgfpathlineto{\pgfqpoint{0.958240in}{1.665655in}}%
\pgfpathlineto{\pgfqpoint{0.969254in}{1.677537in}}%
\pgfpathlineto{\pgfqpoint{0.980268in}{1.686105in}}%
\pgfpathlineto{\pgfqpoint{0.991283in}{1.692798in}}%
\pgfpathlineto{\pgfqpoint{1.013311in}{1.696132in}}%
\pgfpathlineto{\pgfqpoint{1.024325in}{1.704598in}}%
\pgfpathlineto{\pgfqpoint{1.035340in}{1.710264in}}%
\pgfpathlineto{\pgfqpoint{1.046354in}{1.722824in}}%
\pgfpathlineto{\pgfqpoint{1.057368in}{1.728532in}}%
\pgfpathlineto{\pgfqpoint{1.068382in}{1.738903in}}%
\pgfpathlineto{\pgfqpoint{1.079397in}{1.739048in}}%
\pgfpathlineto{\pgfqpoint{1.090411in}{1.745091in}}%
\pgfpathlineto{\pgfqpoint{1.101425in}{1.749199in}}%
\pgfpathlineto{\pgfqpoint{1.112439in}{1.751585in}}%
\pgfpathlineto{\pgfqpoint{1.134468in}{1.759210in}}%
\pgfpathlineto{\pgfqpoint{1.145482in}{1.765487in}}%
\pgfpathlineto{\pgfqpoint{1.178525in}{1.770981in}}%
\pgfpathlineto{\pgfqpoint{1.200553in}{1.773129in}}%
\pgfpathlineto{\pgfqpoint{1.222582in}{1.775857in}}%
\pgfpathlineto{\pgfqpoint{1.233596in}{1.773740in}}%
\pgfpathlineto{\pgfqpoint{1.244610in}{1.774304in}}%
\pgfpathlineto{\pgfqpoint{1.255625in}{1.773399in}}%
\pgfpathlineto{\pgfqpoint{1.343739in}{1.777088in}}%
\pgfpathlineto{\pgfqpoint{1.376782in}{1.777837in}}%
\pgfpathlineto{\pgfqpoint{1.398810in}{1.779211in}}%
\pgfpathlineto{\pgfqpoint{1.806337in}{1.779640in}}%
\pgfpathlineto{\pgfqpoint{3.678760in}{1.779758in}}%
\pgfpathlineto{\pgfqpoint{3.678760in}{1.779758in}}%
\pgfusepath{stroke}%
\end{pgfscope}%
\begin{pgfscope}%
\pgfpathrectangle{\pgfqpoint{0.550713in}{0.408431in}}{\pgfqpoint{3.139062in}{1.490773in}}%
\pgfusepath{clip}%
\pgfsetrectcap%
\pgfsetroundjoin%
\pgfsetlinewidth{0.853187pt}%
\definecolor{currentstroke}{rgb}{0.890196,0.000000,0.400000}%
\pgfsetstrokecolor{currentstroke}%
\pgfsetdash{}{0pt}%
\pgfpathmoveto{\pgfqpoint{0.550713in}{1.250637in}}%
\pgfpathlineto{\pgfqpoint{0.561727in}{1.315081in}}%
\pgfpathlineto{\pgfqpoint{0.572741in}{1.317819in}}%
\pgfpathlineto{\pgfqpoint{0.583755in}{1.326278in}}%
\pgfpathlineto{\pgfqpoint{0.594770in}{1.326987in}}%
\pgfpathlineto{\pgfqpoint{0.605784in}{1.333303in}}%
\pgfpathlineto{\pgfqpoint{0.616798in}{1.348645in}}%
\pgfpathlineto{\pgfqpoint{0.627812in}{1.361280in}}%
\pgfpathlineto{\pgfqpoint{0.638827in}{1.385365in}}%
\pgfpathlineto{\pgfqpoint{0.660855in}{1.419995in}}%
\pgfpathlineto{\pgfqpoint{0.682884in}{1.430253in}}%
\pgfpathlineto{\pgfqpoint{0.693898in}{1.446250in}}%
\pgfpathlineto{\pgfqpoint{0.715926in}{1.462276in}}%
\pgfpathlineto{\pgfqpoint{0.737955in}{1.455826in}}%
\pgfpathlineto{\pgfqpoint{0.748969in}{1.437193in}}%
\pgfpathlineto{\pgfqpoint{0.759983in}{1.422330in}}%
\pgfpathlineto{\pgfqpoint{0.770998in}{1.424752in}}%
\pgfpathlineto{\pgfqpoint{0.782012in}{1.410668in}}%
\pgfpathlineto{\pgfqpoint{0.793026in}{1.374657in}}%
\pgfpathlineto{\pgfqpoint{0.804040in}{1.364107in}}%
\pgfpathlineto{\pgfqpoint{0.815055in}{1.350919in}}%
\pgfpathlineto{\pgfqpoint{0.826069in}{1.305097in}}%
\pgfpathlineto{\pgfqpoint{0.837083in}{1.281915in}}%
\pgfpathlineto{\pgfqpoint{0.848097in}{1.267549in}}%
\pgfpathlineto{\pgfqpoint{0.859112in}{1.255494in}}%
\pgfpathlineto{\pgfqpoint{0.870126in}{1.221134in}}%
\pgfpathlineto{\pgfqpoint{0.881140in}{1.177105in}}%
\pgfpathlineto{\pgfqpoint{0.892154in}{1.180853in}}%
\pgfpathlineto{\pgfqpoint{0.903169in}{1.195642in}}%
\pgfpathlineto{\pgfqpoint{0.914183in}{1.200471in}}%
\pgfpathlineto{\pgfqpoint{0.925197in}{1.215523in}}%
\pgfpathlineto{\pgfqpoint{0.936211in}{1.243315in}}%
\pgfpathlineto{\pgfqpoint{0.947226in}{1.250475in}}%
\pgfpathlineto{\pgfqpoint{0.958240in}{1.247299in}}%
\pgfpathlineto{\pgfqpoint{0.969254in}{1.258200in}}%
\pgfpathlineto{\pgfqpoint{0.980268in}{1.240164in}}%
\pgfpathlineto{\pgfqpoint{0.991283in}{1.237974in}}%
\pgfpathlineto{\pgfqpoint{1.002297in}{1.230788in}}%
\pgfpathlineto{\pgfqpoint{1.013311in}{1.235380in}}%
\pgfpathlineto{\pgfqpoint{1.024325in}{1.248564in}}%
\pgfpathlineto{\pgfqpoint{1.035340in}{1.267667in}}%
\pgfpathlineto{\pgfqpoint{1.046354in}{1.259583in}}%
\pgfpathlineto{\pgfqpoint{1.057368in}{1.273204in}}%
\pgfpathlineto{\pgfqpoint{1.068382in}{1.282250in}}%
\pgfpathlineto{\pgfqpoint{1.079397in}{1.268614in}}%
\pgfpathlineto{\pgfqpoint{1.090411in}{1.278652in}}%
\pgfpathlineto{\pgfqpoint{1.101425in}{1.267762in}}%
\pgfpathlineto{\pgfqpoint{1.112439in}{1.269705in}}%
\pgfpathlineto{\pgfqpoint{1.123454in}{1.265385in}}%
\pgfpathlineto{\pgfqpoint{1.134468in}{1.267165in}}%
\pgfpathlineto{\pgfqpoint{1.145482in}{1.282588in}}%
\pgfpathlineto{\pgfqpoint{1.156496in}{1.282401in}}%
\pgfpathlineto{\pgfqpoint{1.167511in}{1.284836in}}%
\pgfpathlineto{\pgfqpoint{1.178525in}{1.269821in}}%
\pgfpathlineto{\pgfqpoint{1.189539in}{1.266290in}}%
\pgfpathlineto{\pgfqpoint{1.200553in}{1.264341in}}%
\pgfpathlineto{\pgfqpoint{1.222582in}{1.251183in}}%
\pgfpathlineto{\pgfqpoint{1.233596in}{1.246452in}}%
\pgfpathlineto{\pgfqpoint{1.244610in}{1.250276in}}%
\pgfpathlineto{\pgfqpoint{1.255625in}{1.240853in}}%
\pgfpathlineto{\pgfqpoint{1.266639in}{1.224243in}}%
\pgfpathlineto{\pgfqpoint{1.288667in}{1.212935in}}%
\pgfpathlineto{\pgfqpoint{1.299682in}{1.220188in}}%
\pgfpathlineto{\pgfqpoint{1.310696in}{1.230541in}}%
\pgfpathlineto{\pgfqpoint{1.321710in}{1.214841in}}%
\pgfpathlineto{\pgfqpoint{1.332725in}{1.218708in}}%
\pgfpathlineto{\pgfqpoint{1.343739in}{1.205502in}}%
\pgfpathlineto{\pgfqpoint{1.354753in}{1.188202in}}%
\pgfpathlineto{\pgfqpoint{1.365767in}{1.197236in}}%
\pgfpathlineto{\pgfqpoint{1.376782in}{1.200377in}}%
\pgfpathlineto{\pgfqpoint{1.398810in}{1.186085in}}%
\pgfpathlineto{\pgfqpoint{1.409824in}{1.189360in}}%
\pgfpathlineto{\pgfqpoint{1.420839in}{1.197880in}}%
\pgfpathlineto{\pgfqpoint{1.431853in}{1.219520in}}%
\pgfpathlineto{\pgfqpoint{1.442867in}{1.229033in}}%
\pgfpathlineto{\pgfqpoint{1.453881in}{1.253012in}}%
\pgfpathlineto{\pgfqpoint{1.464896in}{1.283653in}}%
\pgfpathlineto{\pgfqpoint{1.475910in}{1.305006in}}%
\pgfpathlineto{\pgfqpoint{1.486924in}{1.303222in}}%
\pgfpathlineto{\pgfqpoint{1.497938in}{1.289733in}}%
\pgfpathlineto{\pgfqpoint{1.519967in}{1.313600in}}%
\pgfpathlineto{\pgfqpoint{1.530981in}{1.337637in}}%
\pgfpathlineto{\pgfqpoint{1.541995in}{1.332576in}}%
\pgfpathlineto{\pgfqpoint{1.553010in}{1.337189in}}%
\pgfpathlineto{\pgfqpoint{1.564024in}{1.345711in}}%
\pgfpathlineto{\pgfqpoint{1.575038in}{1.359997in}}%
\pgfpathlineto{\pgfqpoint{1.586052in}{1.394305in}}%
\pgfpathlineto{\pgfqpoint{1.608081in}{1.393818in}}%
\pgfpathlineto{\pgfqpoint{1.619095in}{1.386716in}}%
\pgfpathlineto{\pgfqpoint{1.630109in}{1.381664in}}%
\pgfpathlineto{\pgfqpoint{1.641124in}{1.385884in}}%
\pgfpathlineto{\pgfqpoint{1.652138in}{1.382024in}}%
\pgfpathlineto{\pgfqpoint{1.663152in}{1.390053in}}%
\pgfpathlineto{\pgfqpoint{1.674166in}{1.389461in}}%
\pgfpathlineto{\pgfqpoint{1.685181in}{1.409711in}}%
\pgfpathlineto{\pgfqpoint{1.696195in}{1.404141in}}%
\pgfpathlineto{\pgfqpoint{1.707209in}{1.392225in}}%
\pgfpathlineto{\pgfqpoint{1.718223in}{1.375111in}}%
\pgfpathlineto{\pgfqpoint{1.729238in}{1.378602in}}%
\pgfpathlineto{\pgfqpoint{1.740252in}{1.375508in}}%
\pgfpathlineto{\pgfqpoint{1.751266in}{1.370948in}}%
\pgfpathlineto{\pgfqpoint{1.762280in}{1.349980in}}%
\pgfpathlineto{\pgfqpoint{1.773295in}{1.341886in}}%
\pgfpathlineto{\pgfqpoint{1.784309in}{1.354351in}}%
\pgfpathlineto{\pgfqpoint{1.795323in}{1.326450in}}%
\pgfpathlineto{\pgfqpoint{1.817352in}{1.348070in}}%
\pgfpathlineto{\pgfqpoint{1.828366in}{1.364988in}}%
\pgfpathlineto{\pgfqpoint{1.850394in}{1.346195in}}%
\pgfpathlineto{\pgfqpoint{1.861409in}{1.326413in}}%
\pgfpathlineto{\pgfqpoint{1.872423in}{1.328664in}}%
\pgfpathlineto{\pgfqpoint{1.883437in}{1.339902in}}%
\pgfpathlineto{\pgfqpoint{1.894451in}{1.345396in}}%
\pgfpathlineto{\pgfqpoint{1.905466in}{1.345888in}}%
\pgfpathlineto{\pgfqpoint{1.916480in}{1.324284in}}%
\pgfpathlineto{\pgfqpoint{1.927494in}{1.309103in}}%
\pgfpathlineto{\pgfqpoint{1.938508in}{1.303119in}}%
\pgfpathlineto{\pgfqpoint{1.949523in}{1.304979in}}%
\pgfpathlineto{\pgfqpoint{1.960537in}{1.311598in}}%
\pgfpathlineto{\pgfqpoint{1.971551in}{1.334047in}}%
\pgfpathlineto{\pgfqpoint{1.982565in}{1.341253in}}%
\pgfpathlineto{\pgfqpoint{1.993580in}{1.344190in}}%
\pgfpathlineto{\pgfqpoint{2.004594in}{1.343859in}}%
\pgfpathlineto{\pgfqpoint{2.015608in}{1.337353in}}%
\pgfpathlineto{\pgfqpoint{2.026622in}{1.319956in}}%
\pgfpathlineto{\pgfqpoint{2.037637in}{1.322445in}}%
\pgfpathlineto{\pgfqpoint{2.048651in}{1.339629in}}%
\pgfpathlineto{\pgfqpoint{2.059665in}{1.348147in}}%
\pgfpathlineto{\pgfqpoint{2.070679in}{1.349113in}}%
\pgfpathlineto{\pgfqpoint{2.081694in}{1.327227in}}%
\pgfpathlineto{\pgfqpoint{2.092708in}{1.329845in}}%
\pgfpathlineto{\pgfqpoint{2.103722in}{1.325223in}}%
\pgfpathlineto{\pgfqpoint{2.114736in}{1.311609in}}%
\pgfpathlineto{\pgfqpoint{2.125751in}{1.303928in}}%
\pgfpathlineto{\pgfqpoint{2.136765in}{1.307879in}}%
\pgfpathlineto{\pgfqpoint{2.147779in}{1.294860in}}%
\pgfpathlineto{\pgfqpoint{2.158793in}{1.305183in}}%
\pgfpathlineto{\pgfqpoint{2.169808in}{1.293533in}}%
\pgfpathlineto{\pgfqpoint{2.180822in}{1.295577in}}%
\pgfpathlineto{\pgfqpoint{2.191836in}{1.299628in}}%
\pgfpathlineto{\pgfqpoint{2.202850in}{1.296897in}}%
\pgfpathlineto{\pgfqpoint{2.213865in}{1.284768in}}%
\pgfpathlineto{\pgfqpoint{2.224879in}{1.270537in}}%
\pgfpathlineto{\pgfqpoint{2.235893in}{1.274691in}}%
\pgfpathlineto{\pgfqpoint{2.246907in}{1.273622in}}%
\pgfpathlineto{\pgfqpoint{2.257922in}{1.284549in}}%
\pgfpathlineto{\pgfqpoint{2.268936in}{1.283470in}}%
\pgfpathlineto{\pgfqpoint{2.279950in}{1.288716in}}%
\pgfpathlineto{\pgfqpoint{2.290964in}{1.288548in}}%
\pgfpathlineto{\pgfqpoint{2.301979in}{1.287175in}}%
\pgfpathlineto{\pgfqpoint{2.312993in}{1.296510in}}%
\pgfpathlineto{\pgfqpoint{2.324007in}{1.279000in}}%
\pgfpathlineto{\pgfqpoint{2.335021in}{1.279108in}}%
\pgfpathlineto{\pgfqpoint{2.346036in}{1.283468in}}%
\pgfpathlineto{\pgfqpoint{2.357050in}{1.298118in}}%
\pgfpathlineto{\pgfqpoint{2.368064in}{1.296501in}}%
\pgfpathlineto{\pgfqpoint{2.379078in}{1.286636in}}%
\pgfpathlineto{\pgfqpoint{2.390093in}{1.268731in}}%
\pgfpathlineto{\pgfqpoint{2.401107in}{1.274063in}}%
\pgfpathlineto{\pgfqpoint{2.412121in}{1.284842in}}%
\pgfpathlineto{\pgfqpoint{2.423135in}{1.289946in}}%
\pgfpathlineto{\pgfqpoint{2.434150in}{1.279179in}}%
\pgfpathlineto{\pgfqpoint{2.445164in}{1.289275in}}%
\pgfpathlineto{\pgfqpoint{2.456178in}{1.285928in}}%
\pgfpathlineto{\pgfqpoint{2.467192in}{1.287321in}}%
\pgfpathlineto{\pgfqpoint{2.478207in}{1.280998in}}%
\pgfpathlineto{\pgfqpoint{2.489221in}{1.277530in}}%
\pgfpathlineto{\pgfqpoint{2.500235in}{1.282367in}}%
\pgfpathlineto{\pgfqpoint{2.511249in}{1.272496in}}%
\pgfpathlineto{\pgfqpoint{2.522264in}{1.273735in}}%
\pgfpathlineto{\pgfqpoint{2.533278in}{1.290672in}}%
\pgfpathlineto{\pgfqpoint{2.544292in}{1.299494in}}%
\pgfpathlineto{\pgfqpoint{2.555306in}{1.295613in}}%
\pgfpathlineto{\pgfqpoint{2.566321in}{1.289732in}}%
\pgfpathlineto{\pgfqpoint{2.577335in}{1.289552in}}%
\pgfpathlineto{\pgfqpoint{2.588349in}{1.283538in}}%
\pgfpathlineto{\pgfqpoint{2.599363in}{1.295766in}}%
\pgfpathlineto{\pgfqpoint{2.610378in}{1.294867in}}%
\pgfpathlineto{\pgfqpoint{2.621392in}{1.288995in}}%
\pgfpathlineto{\pgfqpoint{2.632406in}{1.297389in}}%
\pgfpathlineto{\pgfqpoint{2.643421in}{1.309574in}}%
\pgfpathlineto{\pgfqpoint{2.654435in}{1.329777in}}%
\pgfpathlineto{\pgfqpoint{2.665449in}{1.334475in}}%
\pgfpathlineto{\pgfqpoint{2.676463in}{1.345615in}}%
\pgfpathlineto{\pgfqpoint{2.687478in}{1.333001in}}%
\pgfpathlineto{\pgfqpoint{2.698492in}{1.380241in}}%
\pgfpathlineto{\pgfqpoint{2.709506in}{1.401228in}}%
\pgfpathlineto{\pgfqpoint{2.720520in}{1.407952in}}%
\pgfpathlineto{\pgfqpoint{2.731535in}{1.411093in}}%
\pgfpathlineto{\pgfqpoint{2.742549in}{1.423091in}}%
\pgfpathlineto{\pgfqpoint{2.753563in}{1.424519in}}%
\pgfpathlineto{\pgfqpoint{2.764577in}{1.439395in}}%
\pgfpathlineto{\pgfqpoint{2.775592in}{1.464381in}}%
\pgfpathlineto{\pgfqpoint{2.786606in}{1.467908in}}%
\pgfpathlineto{\pgfqpoint{2.797620in}{1.453152in}}%
\pgfpathlineto{\pgfqpoint{2.808634in}{1.456033in}}%
\pgfpathlineto{\pgfqpoint{2.819649in}{1.460918in}}%
\pgfpathlineto{\pgfqpoint{2.830663in}{1.471477in}}%
\pgfpathlineto{\pgfqpoint{2.852691in}{1.478932in}}%
\pgfpathlineto{\pgfqpoint{2.863706in}{1.480043in}}%
\pgfpathlineto{\pgfqpoint{2.896748in}{1.490653in}}%
\pgfpathlineto{\pgfqpoint{2.907763in}{1.498076in}}%
\pgfpathlineto{\pgfqpoint{2.918777in}{1.503349in}}%
\pgfpathlineto{\pgfqpoint{2.929791in}{1.497539in}}%
\pgfpathlineto{\pgfqpoint{2.951820in}{1.494099in}}%
\pgfpathlineto{\pgfqpoint{2.962834in}{1.485411in}}%
\pgfpathlineto{\pgfqpoint{2.973848in}{1.482238in}}%
\pgfpathlineto{\pgfqpoint{2.984862in}{1.488547in}}%
\pgfpathlineto{\pgfqpoint{2.995877in}{1.484963in}}%
\pgfpathlineto{\pgfqpoint{3.006891in}{1.477294in}}%
\pgfpathlineto{\pgfqpoint{3.017905in}{1.474734in}}%
\pgfpathlineto{\pgfqpoint{3.028919in}{1.461350in}}%
\pgfpathlineto{\pgfqpoint{3.039934in}{1.459405in}}%
\pgfpathlineto{\pgfqpoint{3.050948in}{1.440160in}}%
\pgfpathlineto{\pgfqpoint{3.061962in}{1.431275in}}%
\pgfpathlineto{\pgfqpoint{3.072976in}{1.417404in}}%
\pgfpathlineto{\pgfqpoint{3.083991in}{1.407142in}}%
\pgfpathlineto{\pgfqpoint{3.095005in}{1.398653in}}%
\pgfpathlineto{\pgfqpoint{3.106019in}{1.397510in}}%
\pgfpathlineto{\pgfqpoint{3.117033in}{1.394675in}}%
\pgfpathlineto{\pgfqpoint{3.128048in}{1.395325in}}%
\pgfpathlineto{\pgfqpoint{3.139062in}{1.403672in}}%
\pgfpathlineto{\pgfqpoint{3.150076in}{1.402159in}}%
\pgfpathlineto{\pgfqpoint{3.161090in}{1.386673in}}%
\pgfpathlineto{\pgfqpoint{3.172105in}{1.397884in}}%
\pgfpathlineto{\pgfqpoint{3.183119in}{1.388383in}}%
\pgfpathlineto{\pgfqpoint{3.194133in}{1.402182in}}%
\pgfpathlineto{\pgfqpoint{3.205147in}{1.408906in}}%
\pgfpathlineto{\pgfqpoint{3.216162in}{1.433360in}}%
\pgfpathlineto{\pgfqpoint{3.227176in}{1.427170in}}%
\pgfpathlineto{\pgfqpoint{3.238190in}{1.425728in}}%
\pgfpathlineto{\pgfqpoint{3.249204in}{1.413293in}}%
\pgfpathlineto{\pgfqpoint{3.260219in}{1.416087in}}%
\pgfpathlineto{\pgfqpoint{3.271233in}{1.411584in}}%
\pgfpathlineto{\pgfqpoint{3.282247in}{1.414044in}}%
\pgfpathlineto{\pgfqpoint{3.293261in}{1.418551in}}%
\pgfpathlineto{\pgfqpoint{3.304276in}{1.412056in}}%
\pgfpathlineto{\pgfqpoint{3.315290in}{1.411574in}}%
\pgfpathlineto{\pgfqpoint{3.326304in}{1.415225in}}%
\pgfpathlineto{\pgfqpoint{3.337318in}{1.409823in}}%
\pgfpathlineto{\pgfqpoint{3.348333in}{1.393643in}}%
\pgfpathlineto{\pgfqpoint{3.359347in}{1.389156in}}%
\pgfpathlineto{\pgfqpoint{3.370361in}{1.378688in}}%
\pgfpathlineto{\pgfqpoint{3.381375in}{1.376506in}}%
\pgfpathlineto{\pgfqpoint{3.392390in}{1.393932in}}%
\pgfpathlineto{\pgfqpoint{3.403404in}{1.395902in}}%
\pgfpathlineto{\pgfqpoint{3.414418in}{1.379438in}}%
\pgfpathlineto{\pgfqpoint{3.425432in}{1.379763in}}%
\pgfpathlineto{\pgfqpoint{3.436447in}{1.371774in}}%
\pgfpathlineto{\pgfqpoint{3.447461in}{1.385817in}}%
\pgfpathlineto{\pgfqpoint{3.469489in}{1.379014in}}%
\pgfpathlineto{\pgfqpoint{3.480504in}{1.374970in}}%
\pgfpathlineto{\pgfqpoint{3.491518in}{1.367589in}}%
\pgfpathlineto{\pgfqpoint{3.502532in}{1.390726in}}%
\pgfpathlineto{\pgfqpoint{3.524561in}{1.380428in}}%
\pgfpathlineto{\pgfqpoint{3.535575in}{1.378648in}}%
\pgfpathlineto{\pgfqpoint{3.546589in}{1.351337in}}%
\pgfpathlineto{\pgfqpoint{3.557603in}{1.348642in}}%
\pgfpathlineto{\pgfqpoint{3.568618in}{1.349980in}}%
\pgfpathlineto{\pgfqpoint{3.579632in}{1.353698in}}%
\pgfpathlineto{\pgfqpoint{3.590646in}{1.336365in}}%
\pgfpathlineto{\pgfqpoint{3.601660in}{1.336127in}}%
\pgfpathlineto{\pgfqpoint{3.612675in}{1.334264in}}%
\pgfpathlineto{\pgfqpoint{3.623689in}{1.346546in}}%
\pgfpathlineto{\pgfqpoint{3.634703in}{1.354721in}}%
\pgfpathlineto{\pgfqpoint{3.645717in}{1.348297in}}%
\pgfpathlineto{\pgfqpoint{3.656732in}{1.339667in}}%
\pgfpathlineto{\pgfqpoint{3.667746in}{1.328135in}}%
\pgfpathlineto{\pgfqpoint{3.678760in}{1.326870in}}%
\pgfpathlineto{\pgfqpoint{3.678760in}{1.326870in}}%
\pgfusepath{stroke}%
\end{pgfscope}%
\begin{pgfscope}%
\pgfpathrectangle{\pgfqpoint{0.550713in}{0.408431in}}{\pgfqpoint{3.139062in}{1.490773in}}%
\pgfusepath{clip}%
\pgfsetrectcap%
\pgfsetroundjoin%
\pgfsetlinewidth{0.853187pt}%
\definecolor{currentstroke}{rgb}{0.341176,0.670588,0.152941}%
\pgfsetstrokecolor{currentstroke}%
\pgfsetdash{}{0pt}%
\pgfpathmoveto{\pgfqpoint{0.550713in}{1.225317in}}%
\pgfpathlineto{\pgfqpoint{0.561727in}{1.258765in}}%
\pgfpathlineto{\pgfqpoint{0.572741in}{1.260640in}}%
\pgfpathlineto{\pgfqpoint{0.583755in}{1.255459in}}%
\pgfpathlineto{\pgfqpoint{0.594770in}{1.253755in}}%
\pgfpathlineto{\pgfqpoint{0.605784in}{1.260608in}}%
\pgfpathlineto{\pgfqpoint{0.627812in}{1.265840in}}%
\pgfpathlineto{\pgfqpoint{0.638827in}{1.283288in}}%
\pgfpathlineto{\pgfqpoint{0.649841in}{1.289909in}}%
\pgfpathlineto{\pgfqpoint{0.660855in}{1.303110in}}%
\pgfpathlineto{\pgfqpoint{0.671869in}{1.288947in}}%
\pgfpathlineto{\pgfqpoint{0.682884in}{1.291358in}}%
\pgfpathlineto{\pgfqpoint{0.693898in}{1.301937in}}%
\pgfpathlineto{\pgfqpoint{0.704912in}{1.304993in}}%
\pgfpathlineto{\pgfqpoint{0.715926in}{1.301297in}}%
\pgfpathlineto{\pgfqpoint{0.726941in}{1.301498in}}%
\pgfpathlineto{\pgfqpoint{0.737955in}{1.313072in}}%
\pgfpathlineto{\pgfqpoint{0.748969in}{1.309824in}}%
\pgfpathlineto{\pgfqpoint{0.759983in}{1.332985in}}%
\pgfpathlineto{\pgfqpoint{0.770998in}{1.347825in}}%
\pgfpathlineto{\pgfqpoint{0.782012in}{1.351989in}}%
\pgfpathlineto{\pgfqpoint{0.793026in}{1.352916in}}%
\pgfpathlineto{\pgfqpoint{0.804040in}{1.356991in}}%
\pgfpathlineto{\pgfqpoint{0.826069in}{1.351706in}}%
\pgfpathlineto{\pgfqpoint{0.837083in}{1.349444in}}%
\pgfpathlineto{\pgfqpoint{0.848097in}{1.349042in}}%
\pgfpathlineto{\pgfqpoint{0.859112in}{1.357710in}}%
\pgfpathlineto{\pgfqpoint{0.870126in}{1.363375in}}%
\pgfpathlineto{\pgfqpoint{0.881140in}{1.372026in}}%
\pgfpathlineto{\pgfqpoint{0.892154in}{1.374380in}}%
\pgfpathlineto{\pgfqpoint{0.914183in}{1.392358in}}%
\pgfpathlineto{\pgfqpoint{0.925197in}{1.398664in}}%
\pgfpathlineto{\pgfqpoint{0.936211in}{1.412096in}}%
\pgfpathlineto{\pgfqpoint{0.947226in}{1.414620in}}%
\pgfpathlineto{\pgfqpoint{0.969254in}{1.422343in}}%
\pgfpathlineto{\pgfqpoint{0.980268in}{1.424463in}}%
\pgfpathlineto{\pgfqpoint{1.002297in}{1.420770in}}%
\pgfpathlineto{\pgfqpoint{1.024325in}{1.417580in}}%
\pgfpathlineto{\pgfqpoint{1.035340in}{1.421924in}}%
\pgfpathlineto{\pgfqpoint{1.046354in}{1.435600in}}%
\pgfpathlineto{\pgfqpoint{1.057368in}{1.438346in}}%
\pgfpathlineto{\pgfqpoint{1.068382in}{1.450384in}}%
\pgfpathlineto{\pgfqpoint{1.079397in}{1.449735in}}%
\pgfpathlineto{\pgfqpoint{1.090411in}{1.456375in}}%
\pgfpathlineto{\pgfqpoint{1.101425in}{1.458037in}}%
\pgfpathlineto{\pgfqpoint{1.112439in}{1.463860in}}%
\pgfpathlineto{\pgfqpoint{1.123454in}{1.467435in}}%
\pgfpathlineto{\pgfqpoint{1.134468in}{1.469540in}}%
\pgfpathlineto{\pgfqpoint{1.156496in}{1.478606in}}%
\pgfpathlineto{\pgfqpoint{1.167511in}{1.486325in}}%
\pgfpathlineto{\pgfqpoint{1.178525in}{1.486437in}}%
\pgfpathlineto{\pgfqpoint{1.189539in}{1.488712in}}%
\pgfpathlineto{\pgfqpoint{1.200553in}{1.492436in}}%
\pgfpathlineto{\pgfqpoint{1.211568in}{1.494945in}}%
\pgfpathlineto{\pgfqpoint{1.222582in}{1.491798in}}%
\pgfpathlineto{\pgfqpoint{1.244610in}{1.494393in}}%
\pgfpathlineto{\pgfqpoint{1.277653in}{1.495378in}}%
\pgfpathlineto{\pgfqpoint{1.288667in}{1.499953in}}%
\pgfpathlineto{\pgfqpoint{1.299682in}{1.501135in}}%
\pgfpathlineto{\pgfqpoint{1.310696in}{1.499821in}}%
\pgfpathlineto{\pgfqpoint{1.321710in}{1.503131in}}%
\pgfpathlineto{\pgfqpoint{1.354753in}{1.503061in}}%
\pgfpathlineto{\pgfqpoint{1.365767in}{1.504225in}}%
\pgfpathlineto{\pgfqpoint{1.387796in}{1.511772in}}%
\pgfpathlineto{\pgfqpoint{1.398810in}{1.512520in}}%
\pgfpathlineto{\pgfqpoint{1.409824in}{1.517154in}}%
\pgfpathlineto{\pgfqpoint{1.420839in}{1.518975in}}%
\pgfpathlineto{\pgfqpoint{1.453881in}{1.531523in}}%
\pgfpathlineto{\pgfqpoint{1.464896in}{1.530657in}}%
\pgfpathlineto{\pgfqpoint{1.475910in}{1.534988in}}%
\pgfpathlineto{\pgfqpoint{1.486924in}{1.534160in}}%
\pgfpathlineto{\pgfqpoint{1.497938in}{1.530346in}}%
\pgfpathlineto{\pgfqpoint{1.508953in}{1.529257in}}%
\pgfpathlineto{\pgfqpoint{1.519967in}{1.524272in}}%
\pgfpathlineto{\pgfqpoint{1.541995in}{1.522501in}}%
\pgfpathlineto{\pgfqpoint{1.553010in}{1.522359in}}%
\pgfpathlineto{\pgfqpoint{1.564024in}{1.519186in}}%
\pgfpathlineto{\pgfqpoint{1.575038in}{1.526287in}}%
\pgfpathlineto{\pgfqpoint{1.597067in}{1.537174in}}%
\pgfpathlineto{\pgfqpoint{1.608081in}{1.533048in}}%
\pgfpathlineto{\pgfqpoint{1.619095in}{1.536285in}}%
\pgfpathlineto{\pgfqpoint{1.630109in}{1.528410in}}%
\pgfpathlineto{\pgfqpoint{1.641124in}{1.528083in}}%
\pgfpathlineto{\pgfqpoint{1.652138in}{1.522941in}}%
\pgfpathlineto{\pgfqpoint{1.663152in}{1.522011in}}%
\pgfpathlineto{\pgfqpoint{1.685181in}{1.523450in}}%
\pgfpathlineto{\pgfqpoint{1.696195in}{1.518902in}}%
\pgfpathlineto{\pgfqpoint{1.707209in}{1.521173in}}%
\pgfpathlineto{\pgfqpoint{1.729238in}{1.522047in}}%
\pgfpathlineto{\pgfqpoint{1.751266in}{1.511256in}}%
\pgfpathlineto{\pgfqpoint{1.762280in}{1.516747in}}%
\pgfpathlineto{\pgfqpoint{1.773295in}{1.518813in}}%
\pgfpathlineto{\pgfqpoint{1.784309in}{1.519615in}}%
\pgfpathlineto{\pgfqpoint{1.795323in}{1.525388in}}%
\pgfpathlineto{\pgfqpoint{1.806337in}{1.529261in}}%
\pgfpathlineto{\pgfqpoint{1.828366in}{1.533390in}}%
\pgfpathlineto{\pgfqpoint{1.839380in}{1.533575in}}%
\pgfpathlineto{\pgfqpoint{1.850394in}{1.535258in}}%
\pgfpathlineto{\pgfqpoint{1.861409in}{1.538719in}}%
\pgfpathlineto{\pgfqpoint{1.872423in}{1.539679in}}%
\pgfpathlineto{\pgfqpoint{1.883437in}{1.545696in}}%
\pgfpathlineto{\pgfqpoint{1.894451in}{1.544324in}}%
\pgfpathlineto{\pgfqpoint{1.905466in}{1.547153in}}%
\pgfpathlineto{\pgfqpoint{1.916480in}{1.540281in}}%
\pgfpathlineto{\pgfqpoint{1.949523in}{1.529680in}}%
\pgfpathlineto{\pgfqpoint{1.971551in}{1.533858in}}%
\pgfpathlineto{\pgfqpoint{1.982565in}{1.532514in}}%
\pgfpathlineto{\pgfqpoint{1.993580in}{1.532783in}}%
\pgfpathlineto{\pgfqpoint{2.004594in}{1.526267in}}%
\pgfpathlineto{\pgfqpoint{2.015608in}{1.531573in}}%
\pgfpathlineto{\pgfqpoint{2.059665in}{1.542593in}}%
\pgfpathlineto{\pgfqpoint{2.081694in}{1.551860in}}%
\pgfpathlineto{\pgfqpoint{2.092708in}{1.552193in}}%
\pgfpathlineto{\pgfqpoint{2.103722in}{1.544714in}}%
\pgfpathlineto{\pgfqpoint{2.114736in}{1.545085in}}%
\pgfpathlineto{\pgfqpoint{2.136765in}{1.544428in}}%
\pgfpathlineto{\pgfqpoint{2.158793in}{1.547025in}}%
\pgfpathlineto{\pgfqpoint{2.169808in}{1.541606in}}%
\pgfpathlineto{\pgfqpoint{2.180822in}{1.541250in}}%
\pgfpathlineto{\pgfqpoint{2.191836in}{1.535323in}}%
\pgfpathlineto{\pgfqpoint{2.213865in}{1.536758in}}%
\pgfpathlineto{\pgfqpoint{2.257922in}{1.541672in}}%
\pgfpathlineto{\pgfqpoint{2.268936in}{1.542324in}}%
\pgfpathlineto{\pgfqpoint{2.290964in}{1.548850in}}%
\pgfpathlineto{\pgfqpoint{2.301979in}{1.546868in}}%
\pgfpathlineto{\pgfqpoint{2.312993in}{1.549053in}}%
\pgfpathlineto{\pgfqpoint{2.324007in}{1.555097in}}%
\pgfpathlineto{\pgfqpoint{2.335021in}{1.559706in}}%
\pgfpathlineto{\pgfqpoint{2.357050in}{1.562995in}}%
\pgfpathlineto{\pgfqpoint{2.368064in}{1.561996in}}%
\pgfpathlineto{\pgfqpoint{2.390093in}{1.564241in}}%
\pgfpathlineto{\pgfqpoint{2.423135in}{1.564883in}}%
\pgfpathlineto{\pgfqpoint{2.434150in}{1.569837in}}%
\pgfpathlineto{\pgfqpoint{2.456178in}{1.568928in}}%
\pgfpathlineto{\pgfqpoint{2.467192in}{1.572379in}}%
\pgfpathlineto{\pgfqpoint{2.478207in}{1.569628in}}%
\pgfpathlineto{\pgfqpoint{2.489221in}{1.573241in}}%
\pgfpathlineto{\pgfqpoint{2.500235in}{1.571371in}}%
\pgfpathlineto{\pgfqpoint{2.533278in}{1.574663in}}%
\pgfpathlineto{\pgfqpoint{2.544292in}{1.578290in}}%
\pgfpathlineto{\pgfqpoint{2.588349in}{1.578068in}}%
\pgfpathlineto{\pgfqpoint{2.621392in}{1.585345in}}%
\pgfpathlineto{\pgfqpoint{2.676463in}{1.582521in}}%
\pgfpathlineto{\pgfqpoint{2.687478in}{1.586617in}}%
\pgfpathlineto{\pgfqpoint{2.698492in}{1.588157in}}%
\pgfpathlineto{\pgfqpoint{2.720520in}{1.592872in}}%
\pgfpathlineto{\pgfqpoint{2.731535in}{1.593728in}}%
\pgfpathlineto{\pgfqpoint{2.742549in}{1.597242in}}%
\pgfpathlineto{\pgfqpoint{2.753563in}{1.596696in}}%
\pgfpathlineto{\pgfqpoint{2.808634in}{1.599780in}}%
\pgfpathlineto{\pgfqpoint{2.819649in}{1.603663in}}%
\pgfpathlineto{\pgfqpoint{2.830663in}{1.600114in}}%
\pgfpathlineto{\pgfqpoint{2.841677in}{1.601574in}}%
\pgfpathlineto{\pgfqpoint{2.852691in}{1.606864in}}%
\pgfpathlineto{\pgfqpoint{2.863706in}{1.593289in}}%
\pgfpathlineto{\pgfqpoint{2.874720in}{1.590888in}}%
\pgfpathlineto{\pgfqpoint{2.885734in}{1.593629in}}%
\pgfpathlineto{\pgfqpoint{2.896748in}{1.589497in}}%
\pgfpathlineto{\pgfqpoint{2.907763in}{1.588203in}}%
\pgfpathlineto{\pgfqpoint{2.918777in}{1.589642in}}%
\pgfpathlineto{\pgfqpoint{2.940805in}{1.584396in}}%
\pgfpathlineto{\pgfqpoint{2.962834in}{1.584893in}}%
\pgfpathlineto{\pgfqpoint{2.973848in}{1.586227in}}%
\pgfpathlineto{\pgfqpoint{2.984862in}{1.584597in}}%
\pgfpathlineto{\pgfqpoint{3.006891in}{1.589922in}}%
\pgfpathlineto{\pgfqpoint{3.028919in}{1.590112in}}%
\pgfpathlineto{\pgfqpoint{3.039934in}{1.593105in}}%
\pgfpathlineto{\pgfqpoint{3.061962in}{1.594997in}}%
\pgfpathlineto{\pgfqpoint{3.072976in}{1.592942in}}%
\pgfpathlineto{\pgfqpoint{3.083991in}{1.594353in}}%
\pgfpathlineto{\pgfqpoint{3.095005in}{1.599602in}}%
\pgfpathlineto{\pgfqpoint{3.128048in}{1.599854in}}%
\pgfpathlineto{\pgfqpoint{3.139062in}{1.594042in}}%
\pgfpathlineto{\pgfqpoint{3.150076in}{1.593404in}}%
\pgfpathlineto{\pgfqpoint{3.161090in}{1.594921in}}%
\pgfpathlineto{\pgfqpoint{3.194133in}{1.592130in}}%
\pgfpathlineto{\pgfqpoint{3.216162in}{1.596823in}}%
\pgfpathlineto{\pgfqpoint{3.227176in}{1.591071in}}%
\pgfpathlineto{\pgfqpoint{3.238190in}{1.594794in}}%
\pgfpathlineto{\pgfqpoint{3.249204in}{1.593742in}}%
\pgfpathlineto{\pgfqpoint{3.260219in}{1.595226in}}%
\pgfpathlineto{\pgfqpoint{3.271233in}{1.598283in}}%
\pgfpathlineto{\pgfqpoint{3.282247in}{1.597143in}}%
\pgfpathlineto{\pgfqpoint{3.304276in}{1.604360in}}%
\pgfpathlineto{\pgfqpoint{3.315290in}{1.605578in}}%
\pgfpathlineto{\pgfqpoint{3.326304in}{1.604844in}}%
\pgfpathlineto{\pgfqpoint{3.337318in}{1.608167in}}%
\pgfpathlineto{\pgfqpoint{3.359347in}{1.609359in}}%
\pgfpathlineto{\pgfqpoint{3.370361in}{1.608167in}}%
\pgfpathlineto{\pgfqpoint{3.381375in}{1.610869in}}%
\pgfpathlineto{\pgfqpoint{3.414418in}{1.609920in}}%
\pgfpathlineto{\pgfqpoint{3.425432in}{1.611333in}}%
\pgfpathlineto{\pgfqpoint{3.436447in}{1.611467in}}%
\pgfpathlineto{\pgfqpoint{3.458475in}{1.607741in}}%
\pgfpathlineto{\pgfqpoint{3.469489in}{1.607749in}}%
\pgfpathlineto{\pgfqpoint{3.480504in}{1.604495in}}%
\pgfpathlineto{\pgfqpoint{3.524561in}{1.605516in}}%
\pgfpathlineto{\pgfqpoint{3.535575in}{1.608989in}}%
\pgfpathlineto{\pgfqpoint{3.557603in}{1.609446in}}%
\pgfpathlineto{\pgfqpoint{3.579632in}{1.610703in}}%
\pgfpathlineto{\pgfqpoint{3.623689in}{1.618227in}}%
\pgfpathlineto{\pgfqpoint{3.634703in}{1.615710in}}%
\pgfpathlineto{\pgfqpoint{3.656732in}{1.619644in}}%
\pgfpathlineto{\pgfqpoint{3.667746in}{1.616050in}}%
\pgfpathlineto{\pgfqpoint{3.678760in}{1.617561in}}%
\pgfpathlineto{\pgfqpoint{3.678760in}{1.617561in}}%
\pgfusepath{stroke}%
\end{pgfscope}%
\begin{pgfscope}%
\pgfpathrectangle{\pgfqpoint{0.550713in}{0.408431in}}{\pgfqpoint{3.139062in}{1.490773in}}%
\pgfusepath{clip}%
\pgfsetrectcap%
\pgfsetroundjoin%
\pgfsetlinewidth{0.853187pt}%
\definecolor{currentstroke}{rgb}{0.000000,0.380392,0.396078}%
\pgfsetstrokecolor{currentstroke}%
\pgfsetdash{}{0pt}%
\pgfpathmoveto{\pgfqpoint{0.550713in}{1.222259in}}%
\pgfpathlineto{\pgfqpoint{0.561727in}{1.257864in}}%
\pgfpathlineto{\pgfqpoint{0.572741in}{1.255595in}}%
\pgfpathlineto{\pgfqpoint{0.583755in}{1.224322in}}%
\pgfpathlineto{\pgfqpoint{0.594770in}{1.224415in}}%
\pgfpathlineto{\pgfqpoint{0.605784in}{1.221432in}}%
\pgfpathlineto{\pgfqpoint{0.616798in}{1.229659in}}%
\pgfpathlineto{\pgfqpoint{0.627812in}{1.236312in}}%
\pgfpathlineto{\pgfqpoint{0.638827in}{1.235319in}}%
\pgfpathlineto{\pgfqpoint{0.649841in}{1.240637in}}%
\pgfpathlineto{\pgfqpoint{0.660855in}{1.238050in}}%
\pgfpathlineto{\pgfqpoint{0.671869in}{1.238569in}}%
\pgfpathlineto{\pgfqpoint{0.682884in}{1.250638in}}%
\pgfpathlineto{\pgfqpoint{0.693898in}{1.243741in}}%
\pgfpathlineto{\pgfqpoint{0.704912in}{1.254993in}}%
\pgfpathlineto{\pgfqpoint{0.715926in}{1.260626in}}%
\pgfpathlineto{\pgfqpoint{0.726941in}{1.268393in}}%
\pgfpathlineto{\pgfqpoint{0.737955in}{1.256456in}}%
\pgfpathlineto{\pgfqpoint{0.748969in}{1.267883in}}%
\pgfpathlineto{\pgfqpoint{0.759983in}{1.274779in}}%
\pgfpathlineto{\pgfqpoint{0.770998in}{1.279037in}}%
\pgfpathlineto{\pgfqpoint{0.782012in}{1.293234in}}%
\pgfpathlineto{\pgfqpoint{0.793026in}{1.284784in}}%
\pgfpathlineto{\pgfqpoint{0.804040in}{1.266101in}}%
\pgfpathlineto{\pgfqpoint{0.815055in}{1.236242in}}%
\pgfpathlineto{\pgfqpoint{0.826069in}{1.250977in}}%
\pgfpathlineto{\pgfqpoint{0.837083in}{1.239028in}}%
\pgfpathlineto{\pgfqpoint{0.848097in}{1.241653in}}%
\pgfpathlineto{\pgfqpoint{0.859112in}{1.250679in}}%
\pgfpathlineto{\pgfqpoint{0.870126in}{1.266369in}}%
\pgfpathlineto{\pgfqpoint{0.881140in}{1.249761in}}%
\pgfpathlineto{\pgfqpoint{0.892154in}{1.251110in}}%
\pgfpathlineto{\pgfqpoint{0.903169in}{1.235295in}}%
\pgfpathlineto{\pgfqpoint{0.914183in}{1.228329in}}%
\pgfpathlineto{\pgfqpoint{0.925197in}{1.253675in}}%
\pgfpathlineto{\pgfqpoint{0.936211in}{1.267697in}}%
\pgfpathlineto{\pgfqpoint{0.947226in}{1.258699in}}%
\pgfpathlineto{\pgfqpoint{0.958240in}{1.247815in}}%
\pgfpathlineto{\pgfqpoint{0.969254in}{1.254809in}}%
\pgfpathlineto{\pgfqpoint{0.980268in}{1.244557in}}%
\pgfpathlineto{\pgfqpoint{0.991283in}{1.205249in}}%
\pgfpathlineto{\pgfqpoint{1.002297in}{1.194008in}}%
\pgfpathlineto{\pgfqpoint{1.013311in}{1.221815in}}%
\pgfpathlineto{\pgfqpoint{1.024325in}{1.216703in}}%
\pgfpathlineto{\pgfqpoint{1.035340in}{1.213563in}}%
\pgfpathlineto{\pgfqpoint{1.046354in}{1.226554in}}%
\pgfpathlineto{\pgfqpoint{1.057368in}{1.232731in}}%
\pgfpathlineto{\pgfqpoint{1.068382in}{1.242293in}}%
\pgfpathlineto{\pgfqpoint{1.079397in}{1.255716in}}%
\pgfpathlineto{\pgfqpoint{1.090411in}{1.258526in}}%
\pgfpathlineto{\pgfqpoint{1.101425in}{1.270788in}}%
\pgfpathlineto{\pgfqpoint{1.112439in}{1.285860in}}%
\pgfpathlineto{\pgfqpoint{1.123454in}{1.291306in}}%
\pgfpathlineto{\pgfqpoint{1.134468in}{1.286415in}}%
\pgfpathlineto{\pgfqpoint{1.156496in}{1.274723in}}%
\pgfpathlineto{\pgfqpoint{1.167511in}{1.282444in}}%
\pgfpathlineto{\pgfqpoint{1.178525in}{1.275519in}}%
\pgfpathlineto{\pgfqpoint{1.189539in}{1.277253in}}%
\pgfpathlineto{\pgfqpoint{1.200553in}{1.268397in}}%
\pgfpathlineto{\pgfqpoint{1.211568in}{1.297511in}}%
\pgfpathlineto{\pgfqpoint{1.222582in}{1.300426in}}%
\pgfpathlineto{\pgfqpoint{1.233596in}{1.290772in}}%
\pgfpathlineto{\pgfqpoint{1.244610in}{1.315273in}}%
\pgfpathlineto{\pgfqpoint{1.255625in}{1.297684in}}%
\pgfpathlineto{\pgfqpoint{1.266639in}{1.283039in}}%
\pgfpathlineto{\pgfqpoint{1.277653in}{1.283303in}}%
\pgfpathlineto{\pgfqpoint{1.288667in}{1.270001in}}%
\pgfpathlineto{\pgfqpoint{1.299682in}{1.286302in}}%
\pgfpathlineto{\pgfqpoint{1.310696in}{1.289046in}}%
\pgfpathlineto{\pgfqpoint{1.321710in}{1.296725in}}%
\pgfpathlineto{\pgfqpoint{1.332725in}{1.298669in}}%
\pgfpathlineto{\pgfqpoint{1.343739in}{1.290727in}}%
\pgfpathlineto{\pgfqpoint{1.354753in}{1.277476in}}%
\pgfpathlineto{\pgfqpoint{1.365767in}{1.274143in}}%
\pgfpathlineto{\pgfqpoint{1.376782in}{1.290967in}}%
\pgfpathlineto{\pgfqpoint{1.387796in}{1.303870in}}%
\pgfpathlineto{\pgfqpoint{1.398810in}{1.306570in}}%
\pgfpathlineto{\pgfqpoint{1.420839in}{1.278137in}}%
\pgfpathlineto{\pgfqpoint{1.431853in}{1.296040in}}%
\pgfpathlineto{\pgfqpoint{1.442867in}{1.299540in}}%
\pgfpathlineto{\pgfqpoint{1.453881in}{1.316414in}}%
\pgfpathlineto{\pgfqpoint{1.464896in}{1.326556in}}%
\pgfpathlineto{\pgfqpoint{1.475910in}{1.345500in}}%
\pgfpathlineto{\pgfqpoint{1.486924in}{1.347961in}}%
\pgfpathlineto{\pgfqpoint{1.497938in}{1.344209in}}%
\pgfpathlineto{\pgfqpoint{1.508953in}{1.350281in}}%
\pgfpathlineto{\pgfqpoint{1.519967in}{1.346595in}}%
\pgfpathlineto{\pgfqpoint{1.530981in}{1.318764in}}%
\pgfpathlineto{\pgfqpoint{1.541995in}{1.337736in}}%
\pgfpathlineto{\pgfqpoint{1.553010in}{1.330779in}}%
\pgfpathlineto{\pgfqpoint{1.564024in}{1.321028in}}%
\pgfpathlineto{\pgfqpoint{1.586052in}{1.296338in}}%
\pgfpathlineto{\pgfqpoint{1.597067in}{1.298675in}}%
\pgfpathlineto{\pgfqpoint{1.608081in}{1.314927in}}%
\pgfpathlineto{\pgfqpoint{1.619095in}{1.322060in}}%
\pgfpathlineto{\pgfqpoint{1.641124in}{1.285344in}}%
\pgfpathlineto{\pgfqpoint{1.652138in}{1.278339in}}%
\pgfpathlineto{\pgfqpoint{1.663152in}{1.280495in}}%
\pgfpathlineto{\pgfqpoint{1.674166in}{1.265277in}}%
\pgfpathlineto{\pgfqpoint{1.685181in}{1.262587in}}%
\pgfpathlineto{\pgfqpoint{1.696195in}{1.251679in}}%
\pgfpathlineto{\pgfqpoint{1.707209in}{1.261674in}}%
\pgfpathlineto{\pgfqpoint{1.718223in}{1.269622in}}%
\pgfpathlineto{\pgfqpoint{1.729238in}{1.271507in}}%
\pgfpathlineto{\pgfqpoint{1.740252in}{1.271977in}}%
\pgfpathlineto{\pgfqpoint{1.751266in}{1.298509in}}%
\pgfpathlineto{\pgfqpoint{1.762280in}{1.255235in}}%
\pgfpathlineto{\pgfqpoint{1.773295in}{1.235322in}}%
\pgfpathlineto{\pgfqpoint{1.784309in}{1.231984in}}%
\pgfpathlineto{\pgfqpoint{1.795323in}{1.220929in}}%
\pgfpathlineto{\pgfqpoint{1.806337in}{1.244710in}}%
\pgfpathlineto{\pgfqpoint{1.817352in}{1.262600in}}%
\pgfpathlineto{\pgfqpoint{1.828366in}{1.265736in}}%
\pgfpathlineto{\pgfqpoint{1.839380in}{1.266229in}}%
\pgfpathlineto{\pgfqpoint{1.861409in}{1.252551in}}%
\pgfpathlineto{\pgfqpoint{1.872423in}{1.268511in}}%
\pgfpathlineto{\pgfqpoint{1.883437in}{1.288821in}}%
\pgfpathlineto{\pgfqpoint{1.894451in}{1.268942in}}%
\pgfpathlineto{\pgfqpoint{1.905466in}{1.292337in}}%
\pgfpathlineto{\pgfqpoint{1.916480in}{1.258160in}}%
\pgfpathlineto{\pgfqpoint{1.927494in}{1.252913in}}%
\pgfpathlineto{\pgfqpoint{1.938508in}{1.237840in}}%
\pgfpathlineto{\pgfqpoint{1.949523in}{1.243903in}}%
\pgfpathlineto{\pgfqpoint{1.960537in}{1.219540in}}%
\pgfpathlineto{\pgfqpoint{1.971551in}{1.251104in}}%
\pgfpathlineto{\pgfqpoint{1.982565in}{1.263398in}}%
\pgfpathlineto{\pgfqpoint{1.993580in}{1.256645in}}%
\pgfpathlineto{\pgfqpoint{2.004594in}{1.240680in}}%
\pgfpathlineto{\pgfqpoint{2.015608in}{1.214398in}}%
\pgfpathlineto{\pgfqpoint{2.026622in}{1.237992in}}%
\pgfpathlineto{\pgfqpoint{2.037637in}{1.229446in}}%
\pgfpathlineto{\pgfqpoint{2.048651in}{1.227059in}}%
\pgfpathlineto{\pgfqpoint{2.059665in}{1.213681in}}%
\pgfpathlineto{\pgfqpoint{2.070679in}{1.209604in}}%
\pgfpathlineto{\pgfqpoint{2.081694in}{1.200490in}}%
\pgfpathlineto{\pgfqpoint{2.092708in}{1.205907in}}%
\pgfpathlineto{\pgfqpoint{2.103722in}{1.221635in}}%
\pgfpathlineto{\pgfqpoint{2.114736in}{1.212085in}}%
\pgfpathlineto{\pgfqpoint{2.125751in}{1.196739in}}%
\pgfpathlineto{\pgfqpoint{2.136765in}{1.208455in}}%
\pgfpathlineto{\pgfqpoint{2.147779in}{1.198500in}}%
\pgfpathlineto{\pgfqpoint{2.158793in}{1.227282in}}%
\pgfpathlineto{\pgfqpoint{2.169808in}{1.170633in}}%
\pgfpathlineto{\pgfqpoint{2.180822in}{1.156397in}}%
\pgfpathlineto{\pgfqpoint{2.191836in}{1.185392in}}%
\pgfpathlineto{\pgfqpoint{2.202850in}{1.190228in}}%
\pgfpathlineto{\pgfqpoint{2.213865in}{1.186506in}}%
\pgfpathlineto{\pgfqpoint{2.224879in}{1.185969in}}%
\pgfpathlineto{\pgfqpoint{2.235893in}{1.175630in}}%
\pgfpathlineto{\pgfqpoint{2.246907in}{1.176509in}}%
\pgfpathlineto{\pgfqpoint{2.257922in}{1.175330in}}%
\pgfpathlineto{\pgfqpoint{2.268936in}{1.191631in}}%
\pgfpathlineto{\pgfqpoint{2.279950in}{1.230725in}}%
\pgfpathlineto{\pgfqpoint{2.290964in}{1.223634in}}%
\pgfpathlineto{\pgfqpoint{2.301979in}{1.218395in}}%
\pgfpathlineto{\pgfqpoint{2.312993in}{1.200642in}}%
\pgfpathlineto{\pgfqpoint{2.324007in}{1.203188in}}%
\pgfpathlineto{\pgfqpoint{2.335021in}{1.203351in}}%
\pgfpathlineto{\pgfqpoint{2.346036in}{1.197240in}}%
\pgfpathlineto{\pgfqpoint{2.357050in}{1.213033in}}%
\pgfpathlineto{\pgfqpoint{2.368064in}{1.218919in}}%
\pgfpathlineto{\pgfqpoint{2.379078in}{1.239203in}}%
\pgfpathlineto{\pgfqpoint{2.401107in}{1.249812in}}%
\pgfpathlineto{\pgfqpoint{2.412121in}{1.268735in}}%
\pgfpathlineto{\pgfqpoint{2.423135in}{1.266813in}}%
\pgfpathlineto{\pgfqpoint{2.434150in}{1.284282in}}%
\pgfpathlineto{\pgfqpoint{2.445164in}{1.291090in}}%
\pgfpathlineto{\pgfqpoint{2.456178in}{1.288118in}}%
\pgfpathlineto{\pgfqpoint{2.467192in}{1.255811in}}%
\pgfpathlineto{\pgfqpoint{2.478207in}{1.256589in}}%
\pgfpathlineto{\pgfqpoint{2.489221in}{1.252994in}}%
\pgfpathlineto{\pgfqpoint{2.511249in}{1.239926in}}%
\pgfpathlineto{\pgfqpoint{2.522264in}{1.238498in}}%
\pgfpathlineto{\pgfqpoint{2.533278in}{1.234996in}}%
\pgfpathlineto{\pgfqpoint{2.544292in}{1.236095in}}%
\pgfpathlineto{\pgfqpoint{2.555306in}{1.263787in}}%
\pgfpathlineto{\pgfqpoint{2.566321in}{1.248919in}}%
\pgfpathlineto{\pgfqpoint{2.577335in}{1.259315in}}%
\pgfpathlineto{\pgfqpoint{2.588349in}{1.247941in}}%
\pgfpathlineto{\pgfqpoint{2.599363in}{1.277235in}}%
\pgfpathlineto{\pgfqpoint{2.610378in}{1.273534in}}%
\pgfpathlineto{\pgfqpoint{2.621392in}{1.288903in}}%
\pgfpathlineto{\pgfqpoint{2.632406in}{1.291234in}}%
\pgfpathlineto{\pgfqpoint{2.654435in}{1.271283in}}%
\pgfpathlineto{\pgfqpoint{2.665449in}{1.293962in}}%
\pgfpathlineto{\pgfqpoint{2.676463in}{1.305614in}}%
\pgfpathlineto{\pgfqpoint{2.687478in}{1.301328in}}%
\pgfpathlineto{\pgfqpoint{2.698492in}{1.308478in}}%
\pgfpathlineto{\pgfqpoint{2.709506in}{1.302885in}}%
\pgfpathlineto{\pgfqpoint{2.720520in}{1.290435in}}%
\pgfpathlineto{\pgfqpoint{2.731535in}{1.260346in}}%
\pgfpathlineto{\pgfqpoint{2.742549in}{1.300537in}}%
\pgfpathlineto{\pgfqpoint{2.753563in}{1.319542in}}%
\pgfpathlineto{\pgfqpoint{2.764577in}{1.346221in}}%
\pgfpathlineto{\pgfqpoint{2.775592in}{1.342398in}}%
\pgfpathlineto{\pgfqpoint{2.786606in}{1.347043in}}%
\pgfpathlineto{\pgfqpoint{2.797620in}{1.346717in}}%
\pgfpathlineto{\pgfqpoint{2.808634in}{1.343796in}}%
\pgfpathlineto{\pgfqpoint{2.819649in}{1.335005in}}%
\pgfpathlineto{\pgfqpoint{2.830663in}{1.283782in}}%
\pgfpathlineto{\pgfqpoint{2.852691in}{1.284872in}}%
\pgfpathlineto{\pgfqpoint{2.863706in}{1.323447in}}%
\pgfpathlineto{\pgfqpoint{2.874720in}{1.300057in}}%
\pgfpathlineto{\pgfqpoint{2.885734in}{1.312074in}}%
\pgfpathlineto{\pgfqpoint{2.896748in}{1.305096in}}%
\pgfpathlineto{\pgfqpoint{2.907763in}{1.293296in}}%
\pgfpathlineto{\pgfqpoint{2.918777in}{1.292630in}}%
\pgfpathlineto{\pgfqpoint{2.929791in}{1.288122in}}%
\pgfpathlineto{\pgfqpoint{2.940805in}{1.301433in}}%
\pgfpathlineto{\pgfqpoint{2.951820in}{1.312317in}}%
\pgfpathlineto{\pgfqpoint{2.962834in}{1.295332in}}%
\pgfpathlineto{\pgfqpoint{2.973848in}{1.294660in}}%
\pgfpathlineto{\pgfqpoint{2.984862in}{1.279872in}}%
\pgfpathlineto{\pgfqpoint{2.995877in}{1.293850in}}%
\pgfpathlineto{\pgfqpoint{3.006891in}{1.294275in}}%
\pgfpathlineto{\pgfqpoint{3.017905in}{1.285851in}}%
\pgfpathlineto{\pgfqpoint{3.028919in}{1.313518in}}%
\pgfpathlineto{\pgfqpoint{3.039934in}{1.275993in}}%
\pgfpathlineto{\pgfqpoint{3.050948in}{1.264539in}}%
\pgfpathlineto{\pgfqpoint{3.061962in}{1.296007in}}%
\pgfpathlineto{\pgfqpoint{3.072976in}{1.302700in}}%
\pgfpathlineto{\pgfqpoint{3.083991in}{1.318056in}}%
\pgfpathlineto{\pgfqpoint{3.095005in}{1.331195in}}%
\pgfpathlineto{\pgfqpoint{3.106019in}{1.326934in}}%
\pgfpathlineto{\pgfqpoint{3.117033in}{1.337945in}}%
\pgfpathlineto{\pgfqpoint{3.128048in}{1.335550in}}%
\pgfpathlineto{\pgfqpoint{3.139062in}{1.334893in}}%
\pgfpathlineto{\pgfqpoint{3.150076in}{1.296222in}}%
\pgfpathlineto{\pgfqpoint{3.161090in}{1.310405in}}%
\pgfpathlineto{\pgfqpoint{3.172105in}{1.316943in}}%
\pgfpathlineto{\pgfqpoint{3.183119in}{1.299621in}}%
\pgfpathlineto{\pgfqpoint{3.194133in}{1.311168in}}%
\pgfpathlineto{\pgfqpoint{3.205147in}{1.306782in}}%
\pgfpathlineto{\pgfqpoint{3.216162in}{1.315975in}}%
\pgfpathlineto{\pgfqpoint{3.227176in}{1.327097in}}%
\pgfpathlineto{\pgfqpoint{3.238190in}{1.310522in}}%
\pgfpathlineto{\pgfqpoint{3.249204in}{1.310794in}}%
\pgfpathlineto{\pgfqpoint{3.260219in}{1.288012in}}%
\pgfpathlineto{\pgfqpoint{3.271233in}{1.277106in}}%
\pgfpathlineto{\pgfqpoint{3.282247in}{1.308792in}}%
\pgfpathlineto{\pgfqpoint{3.293261in}{1.293015in}}%
\pgfpathlineto{\pgfqpoint{3.304276in}{1.267561in}}%
\pgfpathlineto{\pgfqpoint{3.315290in}{1.272716in}}%
\pgfpathlineto{\pgfqpoint{3.337318in}{1.224321in}}%
\pgfpathlineto{\pgfqpoint{3.348333in}{1.231294in}}%
\pgfpathlineto{\pgfqpoint{3.359347in}{1.290053in}}%
\pgfpathlineto{\pgfqpoint{3.370361in}{1.281995in}}%
\pgfpathlineto{\pgfqpoint{3.381375in}{1.292282in}}%
\pgfpathlineto{\pgfqpoint{3.403404in}{1.284583in}}%
\pgfpathlineto{\pgfqpoint{3.414418in}{1.267738in}}%
\pgfpathlineto{\pgfqpoint{3.425432in}{1.300793in}}%
\pgfpathlineto{\pgfqpoint{3.436447in}{1.295375in}}%
\pgfpathlineto{\pgfqpoint{3.447461in}{1.283046in}}%
\pgfpathlineto{\pgfqpoint{3.458475in}{1.277609in}}%
\pgfpathlineto{\pgfqpoint{3.469489in}{1.302522in}}%
\pgfpathlineto{\pgfqpoint{3.480504in}{1.308594in}}%
\pgfpathlineto{\pgfqpoint{3.491518in}{1.311951in}}%
\pgfpathlineto{\pgfqpoint{3.502532in}{1.297727in}}%
\pgfpathlineto{\pgfqpoint{3.513546in}{1.309131in}}%
\pgfpathlineto{\pgfqpoint{3.524561in}{1.304016in}}%
\pgfpathlineto{\pgfqpoint{3.535575in}{1.311283in}}%
\pgfpathlineto{\pgfqpoint{3.546589in}{1.307833in}}%
\pgfpathlineto{\pgfqpoint{3.557603in}{1.306474in}}%
\pgfpathlineto{\pgfqpoint{3.568618in}{1.310800in}}%
\pgfpathlineto{\pgfqpoint{3.579632in}{1.318372in}}%
\pgfpathlineto{\pgfqpoint{3.590646in}{1.301148in}}%
\pgfpathlineto{\pgfqpoint{3.601660in}{1.306417in}}%
\pgfpathlineto{\pgfqpoint{3.612675in}{1.293710in}}%
\pgfpathlineto{\pgfqpoint{3.623689in}{1.265923in}}%
\pgfpathlineto{\pgfqpoint{3.634703in}{1.272752in}}%
\pgfpathlineto{\pgfqpoint{3.645717in}{1.269702in}}%
\pgfpathlineto{\pgfqpoint{3.656732in}{1.256981in}}%
\pgfpathlineto{\pgfqpoint{3.667746in}{1.250194in}}%
\pgfpathlineto{\pgfqpoint{3.678760in}{1.255116in}}%
\pgfpathlineto{\pgfqpoint{3.678760in}{1.255116in}}%
\pgfusepath{stroke}%
\end{pgfscope}%
\begin{pgfscope}%
\pgfpathrectangle{\pgfqpoint{0.550713in}{0.408431in}}{\pgfqpoint{3.139062in}{1.490773in}}%
\pgfusepath{clip}%
\pgfsetrectcap%
\pgfsetroundjoin%
\pgfsetlinewidth{0.853187pt}%
\definecolor{currentstroke}{rgb}{0.380392,0.129412,0.345098}%
\pgfsetstrokecolor{currentstroke}%
\pgfsetdash{}{0pt}%
\pgfpathmoveto{\pgfqpoint{0.550713in}{1.264917in}}%
\pgfpathlineto{\pgfqpoint{0.561727in}{1.326051in}}%
\pgfpathlineto{\pgfqpoint{0.572741in}{1.318747in}}%
\pgfpathlineto{\pgfqpoint{0.583755in}{1.325303in}}%
\pgfpathlineto{\pgfqpoint{0.594770in}{1.323320in}}%
\pgfpathlineto{\pgfqpoint{0.605784in}{1.334715in}}%
\pgfpathlineto{\pgfqpoint{0.616798in}{1.353551in}}%
\pgfpathlineto{\pgfqpoint{0.627812in}{1.368649in}}%
\pgfpathlineto{\pgfqpoint{0.638827in}{1.400917in}}%
\pgfpathlineto{\pgfqpoint{0.649841in}{1.408404in}}%
\pgfpathlineto{\pgfqpoint{0.660855in}{1.422531in}}%
\pgfpathlineto{\pgfqpoint{0.671869in}{1.418248in}}%
\pgfpathlineto{\pgfqpoint{0.704912in}{1.465298in}}%
\pgfpathlineto{\pgfqpoint{0.715926in}{1.476810in}}%
\pgfpathlineto{\pgfqpoint{0.726941in}{1.483511in}}%
\pgfpathlineto{\pgfqpoint{0.737955in}{1.494825in}}%
\pgfpathlineto{\pgfqpoint{0.748969in}{1.504407in}}%
\pgfpathlineto{\pgfqpoint{0.759983in}{1.515567in}}%
\pgfpathlineto{\pgfqpoint{0.782012in}{1.563542in}}%
\pgfpathlineto{\pgfqpoint{0.793026in}{1.580125in}}%
\pgfpathlineto{\pgfqpoint{0.804040in}{1.592150in}}%
\pgfpathlineto{\pgfqpoint{0.815055in}{1.596188in}}%
\pgfpathlineto{\pgfqpoint{0.826069in}{1.606319in}}%
\pgfpathlineto{\pgfqpoint{0.837083in}{1.626045in}}%
\pgfpathlineto{\pgfqpoint{0.848097in}{1.639709in}}%
\pgfpathlineto{\pgfqpoint{0.859112in}{1.658059in}}%
\pgfpathlineto{\pgfqpoint{0.870126in}{1.674006in}}%
\pgfpathlineto{\pgfqpoint{0.881140in}{1.678244in}}%
\pgfpathlineto{\pgfqpoint{0.892154in}{1.691651in}}%
\pgfpathlineto{\pgfqpoint{0.903169in}{1.699245in}}%
\pgfpathlineto{\pgfqpoint{0.914183in}{1.701895in}}%
\pgfpathlineto{\pgfqpoint{0.925197in}{1.717366in}}%
\pgfpathlineto{\pgfqpoint{0.936211in}{1.717664in}}%
\pgfpathlineto{\pgfqpoint{0.947226in}{1.722895in}}%
\pgfpathlineto{\pgfqpoint{0.958240in}{1.729812in}}%
\pgfpathlineto{\pgfqpoint{0.969254in}{1.738065in}}%
\pgfpathlineto{\pgfqpoint{0.980268in}{1.744108in}}%
\pgfpathlineto{\pgfqpoint{0.991283in}{1.744574in}}%
\pgfpathlineto{\pgfqpoint{1.002297in}{1.740635in}}%
\pgfpathlineto{\pgfqpoint{1.024325in}{1.744527in}}%
\pgfpathlineto{\pgfqpoint{1.035340in}{1.755492in}}%
\pgfpathlineto{\pgfqpoint{1.046354in}{1.763037in}}%
\pgfpathlineto{\pgfqpoint{1.101425in}{1.773624in}}%
\pgfpathlineto{\pgfqpoint{1.112439in}{1.774100in}}%
\pgfpathlineto{\pgfqpoint{1.145482in}{1.778785in}}%
\pgfpathlineto{\pgfqpoint{1.200553in}{1.779262in}}%
\pgfpathlineto{\pgfqpoint{2.555306in}{1.779772in}}%
\pgfpathlineto{\pgfqpoint{3.678760in}{1.779791in}}%
\pgfpathlineto{\pgfqpoint{3.678760in}{1.779791in}}%
\pgfusepath{stroke}%
\end{pgfscope}%
\begin{pgfscope}%
\pgfpathrectangle{\pgfqpoint{0.550713in}{0.408431in}}{\pgfqpoint{3.139062in}{1.490773in}}%
\pgfusepath{clip}%
\pgfsetrectcap%
\pgfsetroundjoin%
\pgfsetlinewidth{0.853187pt}%
\definecolor{currentstroke}{rgb}{0.964706,0.658824,0.000000}%
\pgfsetstrokecolor{currentstroke}%
\pgfsetdash{}{0pt}%
\pgfpathmoveto{\pgfqpoint{0.550713in}{1.270518in}}%
\pgfpathlineto{\pgfqpoint{0.561727in}{1.335531in}}%
\pgfpathlineto{\pgfqpoint{0.572741in}{1.333542in}}%
\pgfpathlineto{\pgfqpoint{0.583755in}{1.338256in}}%
\pgfpathlineto{\pgfqpoint{0.594770in}{1.336535in}}%
\pgfpathlineto{\pgfqpoint{0.605784in}{1.347351in}}%
\pgfpathlineto{\pgfqpoint{0.616798in}{1.368248in}}%
\pgfpathlineto{\pgfqpoint{0.627812in}{1.383995in}}%
\pgfpathlineto{\pgfqpoint{0.638827in}{1.411509in}}%
\pgfpathlineto{\pgfqpoint{0.660855in}{1.435060in}}%
\pgfpathlineto{\pgfqpoint{0.671869in}{1.434413in}}%
\pgfpathlineto{\pgfqpoint{0.682884in}{1.444941in}}%
\pgfpathlineto{\pgfqpoint{0.693898in}{1.463176in}}%
\pgfpathlineto{\pgfqpoint{0.715926in}{1.491128in}}%
\pgfpathlineto{\pgfqpoint{0.726941in}{1.484013in}}%
\pgfpathlineto{\pgfqpoint{0.737955in}{1.483960in}}%
\pgfpathlineto{\pgfqpoint{0.748969in}{1.469655in}}%
\pgfpathlineto{\pgfqpoint{0.759983in}{1.461575in}}%
\pgfpathlineto{\pgfqpoint{0.770998in}{1.467861in}}%
\pgfpathlineto{\pgfqpoint{0.782012in}{1.444732in}}%
\pgfpathlineto{\pgfqpoint{0.793026in}{1.404594in}}%
\pgfpathlineto{\pgfqpoint{0.804040in}{1.401272in}}%
\pgfpathlineto{\pgfqpoint{0.815055in}{1.391869in}}%
\pgfpathlineto{\pgfqpoint{0.826069in}{1.337924in}}%
\pgfpathlineto{\pgfqpoint{0.859112in}{1.318987in}}%
\pgfpathlineto{\pgfqpoint{0.870126in}{1.294999in}}%
\pgfpathlineto{\pgfqpoint{0.881140in}{1.259248in}}%
\pgfpathlineto{\pgfqpoint{0.892154in}{1.270979in}}%
\pgfpathlineto{\pgfqpoint{0.903169in}{1.288912in}}%
\pgfpathlineto{\pgfqpoint{0.925197in}{1.329145in}}%
\pgfpathlineto{\pgfqpoint{0.936211in}{1.343086in}}%
\pgfpathlineto{\pgfqpoint{0.958240in}{1.343421in}}%
\pgfpathlineto{\pgfqpoint{0.969254in}{1.347297in}}%
\pgfpathlineto{\pgfqpoint{0.980268in}{1.337893in}}%
\pgfpathlineto{\pgfqpoint{0.991283in}{1.336595in}}%
\pgfpathlineto{\pgfqpoint{1.002297in}{1.321128in}}%
\pgfpathlineto{\pgfqpoint{1.013311in}{1.327860in}}%
\pgfpathlineto{\pgfqpoint{1.024325in}{1.345384in}}%
\pgfpathlineto{\pgfqpoint{1.035340in}{1.355846in}}%
\pgfpathlineto{\pgfqpoint{1.046354in}{1.327439in}}%
\pgfpathlineto{\pgfqpoint{1.057368in}{1.335166in}}%
\pgfpathlineto{\pgfqpoint{1.068382in}{1.347868in}}%
\pgfpathlineto{\pgfqpoint{1.079397in}{1.354557in}}%
\pgfpathlineto{\pgfqpoint{1.090411in}{1.368869in}}%
\pgfpathlineto{\pgfqpoint{1.101425in}{1.370144in}}%
\pgfpathlineto{\pgfqpoint{1.112439in}{1.366810in}}%
\pgfpathlineto{\pgfqpoint{1.123454in}{1.373346in}}%
\pgfpathlineto{\pgfqpoint{1.134468in}{1.372647in}}%
\pgfpathlineto{\pgfqpoint{1.145482in}{1.403362in}}%
\pgfpathlineto{\pgfqpoint{1.156496in}{1.419806in}}%
\pgfpathlineto{\pgfqpoint{1.167511in}{1.400470in}}%
\pgfpathlineto{\pgfqpoint{1.189539in}{1.377489in}}%
\pgfpathlineto{\pgfqpoint{1.200553in}{1.386093in}}%
\pgfpathlineto{\pgfqpoint{1.211568in}{1.368150in}}%
\pgfpathlineto{\pgfqpoint{1.222582in}{1.382362in}}%
\pgfpathlineto{\pgfqpoint{1.233596in}{1.390627in}}%
\pgfpathlineto{\pgfqpoint{1.244610in}{1.392175in}}%
\pgfpathlineto{\pgfqpoint{1.255625in}{1.380616in}}%
\pgfpathlineto{\pgfqpoint{1.266639in}{1.397137in}}%
\pgfpathlineto{\pgfqpoint{1.277653in}{1.405952in}}%
\pgfpathlineto{\pgfqpoint{1.288667in}{1.410218in}}%
\pgfpathlineto{\pgfqpoint{1.299682in}{1.416144in}}%
\pgfpathlineto{\pgfqpoint{1.310696in}{1.426279in}}%
\pgfpathlineto{\pgfqpoint{1.321710in}{1.417080in}}%
\pgfpathlineto{\pgfqpoint{1.343739in}{1.416456in}}%
\pgfpathlineto{\pgfqpoint{1.354753in}{1.422277in}}%
\pgfpathlineto{\pgfqpoint{1.365767in}{1.419988in}}%
\pgfpathlineto{\pgfqpoint{1.376782in}{1.427546in}}%
\pgfpathlineto{\pgfqpoint{1.398810in}{1.429073in}}%
\pgfpathlineto{\pgfqpoint{1.409824in}{1.431680in}}%
\pgfpathlineto{\pgfqpoint{1.420839in}{1.443592in}}%
\pgfpathlineto{\pgfqpoint{1.431853in}{1.452074in}}%
\pgfpathlineto{\pgfqpoint{1.442867in}{1.453359in}}%
\pgfpathlineto{\pgfqpoint{1.464896in}{1.469163in}}%
\pgfpathlineto{\pgfqpoint{1.475910in}{1.478999in}}%
\pgfpathlineto{\pgfqpoint{1.486924in}{1.465557in}}%
\pgfpathlineto{\pgfqpoint{1.497938in}{1.464189in}}%
\pgfpathlineto{\pgfqpoint{1.519967in}{1.489320in}}%
\pgfpathlineto{\pgfqpoint{1.530981in}{1.489039in}}%
\pgfpathlineto{\pgfqpoint{1.541995in}{1.471288in}}%
\pgfpathlineto{\pgfqpoint{1.553010in}{1.468693in}}%
\pgfpathlineto{\pgfqpoint{1.564024in}{1.456940in}}%
\pgfpathlineto{\pgfqpoint{1.575038in}{1.478765in}}%
\pgfpathlineto{\pgfqpoint{1.586052in}{1.495275in}}%
\pgfpathlineto{\pgfqpoint{1.597067in}{1.503176in}}%
\pgfpathlineto{\pgfqpoint{1.608081in}{1.495373in}}%
\pgfpathlineto{\pgfqpoint{1.619095in}{1.480830in}}%
\pgfpathlineto{\pgfqpoint{1.630109in}{1.481705in}}%
\pgfpathlineto{\pgfqpoint{1.641124in}{1.486126in}}%
\pgfpathlineto{\pgfqpoint{1.652138in}{1.471842in}}%
\pgfpathlineto{\pgfqpoint{1.663152in}{1.466811in}}%
\pgfpathlineto{\pgfqpoint{1.674166in}{1.476621in}}%
\pgfpathlineto{\pgfqpoint{1.685181in}{1.498832in}}%
\pgfpathlineto{\pgfqpoint{1.696195in}{1.486991in}}%
\pgfpathlineto{\pgfqpoint{1.707209in}{1.490658in}}%
\pgfpathlineto{\pgfqpoint{1.718223in}{1.462657in}}%
\pgfpathlineto{\pgfqpoint{1.740252in}{1.472073in}}%
\pgfpathlineto{\pgfqpoint{1.751266in}{1.469699in}}%
\pgfpathlineto{\pgfqpoint{1.762280in}{1.463340in}}%
\pgfpathlineto{\pgfqpoint{1.773295in}{1.444224in}}%
\pgfpathlineto{\pgfqpoint{1.784309in}{1.460552in}}%
\pgfpathlineto{\pgfqpoint{1.795323in}{1.454677in}}%
\pgfpathlineto{\pgfqpoint{1.806337in}{1.462839in}}%
\pgfpathlineto{\pgfqpoint{1.817352in}{1.460378in}}%
\pgfpathlineto{\pgfqpoint{1.828366in}{1.476773in}}%
\pgfpathlineto{\pgfqpoint{1.850394in}{1.455501in}}%
\pgfpathlineto{\pgfqpoint{1.861409in}{1.447925in}}%
\pgfpathlineto{\pgfqpoint{1.872423in}{1.460433in}}%
\pgfpathlineto{\pgfqpoint{1.883437in}{1.462700in}}%
\pgfpathlineto{\pgfqpoint{1.894451in}{1.449803in}}%
\pgfpathlineto{\pgfqpoint{1.905466in}{1.449760in}}%
\pgfpathlineto{\pgfqpoint{1.916480in}{1.451245in}}%
\pgfpathlineto{\pgfqpoint{1.938508in}{1.443396in}}%
\pgfpathlineto{\pgfqpoint{1.949523in}{1.446580in}}%
\pgfpathlineto{\pgfqpoint{1.960537in}{1.439165in}}%
\pgfpathlineto{\pgfqpoint{1.971551in}{1.447477in}}%
\pgfpathlineto{\pgfqpoint{1.993580in}{1.458147in}}%
\pgfpathlineto{\pgfqpoint{2.004594in}{1.473442in}}%
\pgfpathlineto{\pgfqpoint{2.015608in}{1.470434in}}%
\pgfpathlineto{\pgfqpoint{2.037637in}{1.454205in}}%
\pgfpathlineto{\pgfqpoint{2.048651in}{1.464490in}}%
\pgfpathlineto{\pgfqpoint{2.059665in}{1.482483in}}%
\pgfpathlineto{\pgfqpoint{2.070679in}{1.485962in}}%
\pgfpathlineto{\pgfqpoint{2.081694in}{1.453868in}}%
\pgfpathlineto{\pgfqpoint{2.103722in}{1.432680in}}%
\pgfpathlineto{\pgfqpoint{2.114736in}{1.433607in}}%
\pgfpathlineto{\pgfqpoint{2.125751in}{1.439921in}}%
\pgfpathlineto{\pgfqpoint{2.136765in}{1.449678in}}%
\pgfpathlineto{\pgfqpoint{2.147779in}{1.448776in}}%
\pgfpathlineto{\pgfqpoint{2.158793in}{1.459747in}}%
\pgfpathlineto{\pgfqpoint{2.169808in}{1.448197in}}%
\pgfpathlineto{\pgfqpoint{2.180822in}{1.446494in}}%
\pgfpathlineto{\pgfqpoint{2.191836in}{1.451096in}}%
\pgfpathlineto{\pgfqpoint{2.202850in}{1.426650in}}%
\pgfpathlineto{\pgfqpoint{2.213865in}{1.418721in}}%
\pgfpathlineto{\pgfqpoint{2.235893in}{1.414682in}}%
\pgfpathlineto{\pgfqpoint{2.246907in}{1.414898in}}%
\pgfpathlineto{\pgfqpoint{2.268936in}{1.438417in}}%
\pgfpathlineto{\pgfqpoint{2.279950in}{1.459571in}}%
\pgfpathlineto{\pgfqpoint{2.290964in}{1.452398in}}%
\pgfpathlineto{\pgfqpoint{2.301979in}{1.443659in}}%
\pgfpathlineto{\pgfqpoint{2.312993in}{1.444378in}}%
\pgfpathlineto{\pgfqpoint{2.324007in}{1.453897in}}%
\pgfpathlineto{\pgfqpoint{2.335021in}{1.456874in}}%
\pgfpathlineto{\pgfqpoint{2.346036in}{1.456770in}}%
\pgfpathlineto{\pgfqpoint{2.357050in}{1.471418in}}%
\pgfpathlineto{\pgfqpoint{2.368064in}{1.481504in}}%
\pgfpathlineto{\pgfqpoint{2.379078in}{1.476990in}}%
\pgfpathlineto{\pgfqpoint{2.390093in}{1.477132in}}%
\pgfpathlineto{\pgfqpoint{2.401107in}{1.480171in}}%
\pgfpathlineto{\pgfqpoint{2.412121in}{1.492772in}}%
\pgfpathlineto{\pgfqpoint{2.423135in}{1.501633in}}%
\pgfpathlineto{\pgfqpoint{2.445164in}{1.503665in}}%
\pgfpathlineto{\pgfqpoint{2.456178in}{1.507303in}}%
\pgfpathlineto{\pgfqpoint{2.467192in}{1.500882in}}%
\pgfpathlineto{\pgfqpoint{2.489221in}{1.513786in}}%
\pgfpathlineto{\pgfqpoint{2.500235in}{1.525891in}}%
\pgfpathlineto{\pgfqpoint{2.511249in}{1.517162in}}%
\pgfpathlineto{\pgfqpoint{2.522264in}{1.526334in}}%
\pgfpathlineto{\pgfqpoint{2.533278in}{1.532618in}}%
\pgfpathlineto{\pgfqpoint{2.544292in}{1.527513in}}%
\pgfpathlineto{\pgfqpoint{2.555306in}{1.533388in}}%
\pgfpathlineto{\pgfqpoint{2.566321in}{1.543917in}}%
\pgfpathlineto{\pgfqpoint{2.577335in}{1.544054in}}%
\pgfpathlineto{\pgfqpoint{2.588349in}{1.535664in}}%
\pgfpathlineto{\pgfqpoint{2.599363in}{1.545649in}}%
\pgfpathlineto{\pgfqpoint{2.610378in}{1.537410in}}%
\pgfpathlineto{\pgfqpoint{2.621392in}{1.536983in}}%
\pgfpathlineto{\pgfqpoint{2.632406in}{1.550225in}}%
\pgfpathlineto{\pgfqpoint{2.643421in}{1.555626in}}%
\pgfpathlineto{\pgfqpoint{2.654435in}{1.563649in}}%
\pgfpathlineto{\pgfqpoint{2.665449in}{1.547482in}}%
\pgfpathlineto{\pgfqpoint{2.676463in}{1.556464in}}%
\pgfpathlineto{\pgfqpoint{2.687478in}{1.543723in}}%
\pgfpathlineto{\pgfqpoint{2.698492in}{1.548138in}}%
\pgfpathlineto{\pgfqpoint{2.709506in}{1.559417in}}%
\pgfpathlineto{\pgfqpoint{2.720520in}{1.565192in}}%
\pgfpathlineto{\pgfqpoint{2.731535in}{1.564548in}}%
\pgfpathlineto{\pgfqpoint{2.742549in}{1.573898in}}%
\pgfpathlineto{\pgfqpoint{2.753563in}{1.578777in}}%
\pgfpathlineto{\pgfqpoint{2.764577in}{1.566833in}}%
\pgfpathlineto{\pgfqpoint{2.775592in}{1.568285in}}%
\pgfpathlineto{\pgfqpoint{2.786606in}{1.578046in}}%
\pgfpathlineto{\pgfqpoint{2.797620in}{1.561477in}}%
\pgfpathlineto{\pgfqpoint{2.808634in}{1.555171in}}%
\pgfpathlineto{\pgfqpoint{2.819649in}{1.578260in}}%
\pgfpathlineto{\pgfqpoint{2.830663in}{1.585041in}}%
\pgfpathlineto{\pgfqpoint{2.841677in}{1.582726in}}%
\pgfpathlineto{\pgfqpoint{2.852691in}{1.587991in}}%
\pgfpathlineto{\pgfqpoint{2.863706in}{1.601327in}}%
\pgfpathlineto{\pgfqpoint{2.874720in}{1.603716in}}%
\pgfpathlineto{\pgfqpoint{2.885734in}{1.599012in}}%
\pgfpathlineto{\pgfqpoint{2.896748in}{1.598429in}}%
\pgfpathlineto{\pgfqpoint{2.907763in}{1.594761in}}%
\pgfpathlineto{\pgfqpoint{2.918777in}{1.596735in}}%
\pgfpathlineto{\pgfqpoint{2.929791in}{1.596687in}}%
\pgfpathlineto{\pgfqpoint{2.940805in}{1.588571in}}%
\pgfpathlineto{\pgfqpoint{2.951820in}{1.586228in}}%
\pgfpathlineto{\pgfqpoint{2.973848in}{1.596815in}}%
\pgfpathlineto{\pgfqpoint{2.984862in}{1.586801in}}%
\pgfpathlineto{\pgfqpoint{2.995877in}{1.588820in}}%
\pgfpathlineto{\pgfqpoint{3.017905in}{1.575986in}}%
\pgfpathlineto{\pgfqpoint{3.028919in}{1.581047in}}%
\pgfpathlineto{\pgfqpoint{3.050948in}{1.578866in}}%
\pgfpathlineto{\pgfqpoint{3.061962in}{1.580255in}}%
\pgfpathlineto{\pgfqpoint{3.072976in}{1.579694in}}%
\pgfpathlineto{\pgfqpoint{3.083991in}{1.572681in}}%
\pgfpathlineto{\pgfqpoint{3.095005in}{1.576793in}}%
\pgfpathlineto{\pgfqpoint{3.106019in}{1.573654in}}%
\pgfpathlineto{\pgfqpoint{3.117033in}{1.567099in}}%
\pgfpathlineto{\pgfqpoint{3.128048in}{1.563368in}}%
\pgfpathlineto{\pgfqpoint{3.139062in}{1.576690in}}%
\pgfpathlineto{\pgfqpoint{3.150076in}{1.568327in}}%
\pgfpathlineto{\pgfqpoint{3.161090in}{1.552227in}}%
\pgfpathlineto{\pgfqpoint{3.172105in}{1.546214in}}%
\pgfpathlineto{\pgfqpoint{3.183119in}{1.546230in}}%
\pgfpathlineto{\pgfqpoint{3.194133in}{1.539603in}}%
\pgfpathlineto{\pgfqpoint{3.205147in}{1.544735in}}%
\pgfpathlineto{\pgfqpoint{3.216162in}{1.548566in}}%
\pgfpathlineto{\pgfqpoint{3.227176in}{1.546116in}}%
\pgfpathlineto{\pgfqpoint{3.238190in}{1.545791in}}%
\pgfpathlineto{\pgfqpoint{3.249204in}{1.532183in}}%
\pgfpathlineto{\pgfqpoint{3.260219in}{1.525554in}}%
\pgfpathlineto{\pgfqpoint{3.282247in}{1.536066in}}%
\pgfpathlineto{\pgfqpoint{3.293261in}{1.539944in}}%
\pgfpathlineto{\pgfqpoint{3.304276in}{1.529232in}}%
\pgfpathlineto{\pgfqpoint{3.326304in}{1.524879in}}%
\pgfpathlineto{\pgfqpoint{3.337318in}{1.520742in}}%
\pgfpathlineto{\pgfqpoint{3.348333in}{1.512385in}}%
\pgfpathlineto{\pgfqpoint{3.359347in}{1.509574in}}%
\pgfpathlineto{\pgfqpoint{3.381375in}{1.529858in}}%
\pgfpathlineto{\pgfqpoint{3.392390in}{1.545551in}}%
\pgfpathlineto{\pgfqpoint{3.403404in}{1.538201in}}%
\pgfpathlineto{\pgfqpoint{3.414418in}{1.528781in}}%
\pgfpathlineto{\pgfqpoint{3.425432in}{1.523969in}}%
\pgfpathlineto{\pgfqpoint{3.436447in}{1.523130in}}%
\pgfpathlineto{\pgfqpoint{3.447461in}{1.531958in}}%
\pgfpathlineto{\pgfqpoint{3.458475in}{1.531740in}}%
\pgfpathlineto{\pgfqpoint{3.469489in}{1.527139in}}%
\pgfpathlineto{\pgfqpoint{3.502532in}{1.523680in}}%
\pgfpathlineto{\pgfqpoint{3.513546in}{1.518735in}}%
\pgfpathlineto{\pgfqpoint{3.524561in}{1.522389in}}%
\pgfpathlineto{\pgfqpoint{3.546589in}{1.526637in}}%
\pgfpathlineto{\pgfqpoint{3.557603in}{1.523173in}}%
\pgfpathlineto{\pgfqpoint{3.568618in}{1.515924in}}%
\pgfpathlineto{\pgfqpoint{3.579632in}{1.526656in}}%
\pgfpathlineto{\pgfqpoint{3.590646in}{1.523743in}}%
\pgfpathlineto{\pgfqpoint{3.601660in}{1.522071in}}%
\pgfpathlineto{\pgfqpoint{3.612675in}{1.518846in}}%
\pgfpathlineto{\pgfqpoint{3.623689in}{1.520620in}}%
\pgfpathlineto{\pgfqpoint{3.645717in}{1.526487in}}%
\pgfpathlineto{\pgfqpoint{3.656732in}{1.528316in}}%
\pgfpathlineto{\pgfqpoint{3.667746in}{1.503865in}}%
\pgfpathlineto{\pgfqpoint{3.678760in}{1.503755in}}%
\pgfpathlineto{\pgfqpoint{3.678760in}{1.503755in}}%
\pgfusepath{stroke}%
\end{pgfscope}%
\begin{pgfscope}%
\pgfpathrectangle{\pgfqpoint{0.550713in}{0.408431in}}{\pgfqpoint{3.139062in}{1.490773in}}%
\pgfusepath{clip}%
\pgfsetrectcap%
\pgfsetroundjoin%
\pgfsetlinewidth{0.853187pt}%
\definecolor{currentstroke}{rgb}{0.000000,0.329412,0.623529}%
\pgfsetstrokecolor{currentstroke}%
\pgfsetdash{}{0pt}%
\pgfpathmoveto{\pgfqpoint{0.550713in}{1.222163in}}%
\pgfpathlineto{\pgfqpoint{0.561727in}{1.277050in}}%
\pgfpathlineto{\pgfqpoint{0.572741in}{1.271376in}}%
\pgfpathlineto{\pgfqpoint{0.583755in}{1.267681in}}%
\pgfpathlineto{\pgfqpoint{0.594770in}{1.260762in}}%
\pgfpathlineto{\pgfqpoint{0.605784in}{1.271668in}}%
\pgfpathlineto{\pgfqpoint{0.616798in}{1.272383in}}%
\pgfpathlineto{\pgfqpoint{0.627812in}{1.284233in}}%
\pgfpathlineto{\pgfqpoint{0.638827in}{1.302237in}}%
\pgfpathlineto{\pgfqpoint{0.649841in}{1.297124in}}%
\pgfpathlineto{\pgfqpoint{0.660855in}{1.305554in}}%
\pgfpathlineto{\pgfqpoint{0.671869in}{1.308991in}}%
\pgfpathlineto{\pgfqpoint{0.682884in}{1.300484in}}%
\pgfpathlineto{\pgfqpoint{0.693898in}{1.308714in}}%
\pgfpathlineto{\pgfqpoint{0.704912in}{1.308089in}}%
\pgfpathlineto{\pgfqpoint{0.715926in}{1.299052in}}%
\pgfpathlineto{\pgfqpoint{0.726941in}{1.296476in}}%
\pgfpathlineto{\pgfqpoint{0.737955in}{1.300730in}}%
\pgfpathlineto{\pgfqpoint{0.748969in}{1.298812in}}%
\pgfpathlineto{\pgfqpoint{0.770998in}{1.305171in}}%
\pgfpathlineto{\pgfqpoint{0.782012in}{1.322081in}}%
\pgfpathlineto{\pgfqpoint{0.793026in}{1.331083in}}%
\pgfpathlineto{\pgfqpoint{0.804040in}{1.325132in}}%
\pgfpathlineto{\pgfqpoint{0.815055in}{1.329504in}}%
\pgfpathlineto{\pgfqpoint{0.826069in}{1.330171in}}%
\pgfpathlineto{\pgfqpoint{0.837083in}{1.346542in}}%
\pgfpathlineto{\pgfqpoint{0.848097in}{1.331903in}}%
\pgfpathlineto{\pgfqpoint{0.859112in}{1.335948in}}%
\pgfpathlineto{\pgfqpoint{0.870126in}{1.343011in}}%
\pgfpathlineto{\pgfqpoint{0.881140in}{1.339156in}}%
\pgfpathlineto{\pgfqpoint{0.892154in}{1.347219in}}%
\pgfpathlineto{\pgfqpoint{0.903169in}{1.346832in}}%
\pgfpathlineto{\pgfqpoint{0.914183in}{1.350061in}}%
\pgfpathlineto{\pgfqpoint{0.925197in}{1.351070in}}%
\pgfpathlineto{\pgfqpoint{0.936211in}{1.355538in}}%
\pgfpathlineto{\pgfqpoint{0.947226in}{1.348919in}}%
\pgfpathlineto{\pgfqpoint{0.958240in}{1.346175in}}%
\pgfpathlineto{\pgfqpoint{0.969254in}{1.352019in}}%
\pgfpathlineto{\pgfqpoint{0.980268in}{1.343974in}}%
\pgfpathlineto{\pgfqpoint{0.991283in}{1.351622in}}%
\pgfpathlineto{\pgfqpoint{1.002297in}{1.324345in}}%
\pgfpathlineto{\pgfqpoint{1.013311in}{1.325171in}}%
\pgfpathlineto{\pgfqpoint{1.035340in}{1.341822in}}%
\pgfpathlineto{\pgfqpoint{1.046354in}{1.348773in}}%
\pgfpathlineto{\pgfqpoint{1.057368in}{1.351109in}}%
\pgfpathlineto{\pgfqpoint{1.068382in}{1.359662in}}%
\pgfpathlineto{\pgfqpoint{1.079397in}{1.364776in}}%
\pgfpathlineto{\pgfqpoint{1.090411in}{1.365817in}}%
\pgfpathlineto{\pgfqpoint{1.112439in}{1.370884in}}%
\pgfpathlineto{\pgfqpoint{1.123454in}{1.372230in}}%
\pgfpathlineto{\pgfqpoint{1.134468in}{1.369945in}}%
\pgfpathlineto{\pgfqpoint{1.156496in}{1.379642in}}%
\pgfpathlineto{\pgfqpoint{1.167511in}{1.387301in}}%
\pgfpathlineto{\pgfqpoint{1.178525in}{1.376474in}}%
\pgfpathlineto{\pgfqpoint{1.189539in}{1.383327in}}%
\pgfpathlineto{\pgfqpoint{1.200553in}{1.379838in}}%
\pgfpathlineto{\pgfqpoint{1.211568in}{1.362880in}}%
\pgfpathlineto{\pgfqpoint{1.222582in}{1.375408in}}%
\pgfpathlineto{\pgfqpoint{1.233596in}{1.375417in}}%
\pgfpathlineto{\pgfqpoint{1.244610in}{1.370904in}}%
\pgfpathlineto{\pgfqpoint{1.255625in}{1.372036in}}%
\pgfpathlineto{\pgfqpoint{1.277653in}{1.382889in}}%
\pgfpathlineto{\pgfqpoint{1.288667in}{1.395356in}}%
\pgfpathlineto{\pgfqpoint{1.299682in}{1.402405in}}%
\pgfpathlineto{\pgfqpoint{1.310696in}{1.399514in}}%
\pgfpathlineto{\pgfqpoint{1.321710in}{1.401212in}}%
\pgfpathlineto{\pgfqpoint{1.332725in}{1.397988in}}%
\pgfpathlineto{\pgfqpoint{1.343739in}{1.400073in}}%
\pgfpathlineto{\pgfqpoint{1.354753in}{1.400008in}}%
\pgfpathlineto{\pgfqpoint{1.365767in}{1.393410in}}%
\pgfpathlineto{\pgfqpoint{1.376782in}{1.393094in}}%
\pgfpathlineto{\pgfqpoint{1.387796in}{1.393903in}}%
\pgfpathlineto{\pgfqpoint{1.409824in}{1.399054in}}%
\pgfpathlineto{\pgfqpoint{1.420839in}{1.407649in}}%
\pgfpathlineto{\pgfqpoint{1.431853in}{1.411910in}}%
\pgfpathlineto{\pgfqpoint{1.442867in}{1.411715in}}%
\pgfpathlineto{\pgfqpoint{1.453881in}{1.419497in}}%
\pgfpathlineto{\pgfqpoint{1.464896in}{1.424107in}}%
\pgfpathlineto{\pgfqpoint{1.475910in}{1.435884in}}%
\pgfpathlineto{\pgfqpoint{1.486924in}{1.433059in}}%
\pgfpathlineto{\pgfqpoint{1.497938in}{1.433544in}}%
\pgfpathlineto{\pgfqpoint{1.508953in}{1.429214in}}%
\pgfpathlineto{\pgfqpoint{1.519967in}{1.420173in}}%
\pgfpathlineto{\pgfqpoint{1.530981in}{1.424883in}}%
\pgfpathlineto{\pgfqpoint{1.541995in}{1.422521in}}%
\pgfpathlineto{\pgfqpoint{1.553010in}{1.424996in}}%
\pgfpathlineto{\pgfqpoint{1.564024in}{1.416762in}}%
\pgfpathlineto{\pgfqpoint{1.575038in}{1.423104in}}%
\pgfpathlineto{\pgfqpoint{1.586052in}{1.433758in}}%
\pgfpathlineto{\pgfqpoint{1.597067in}{1.437730in}}%
\pgfpathlineto{\pgfqpoint{1.608081in}{1.427683in}}%
\pgfpathlineto{\pgfqpoint{1.652138in}{1.440875in}}%
\pgfpathlineto{\pgfqpoint{1.663152in}{1.435149in}}%
\pgfpathlineto{\pgfqpoint{1.674166in}{1.433093in}}%
\pgfpathlineto{\pgfqpoint{1.685181in}{1.435944in}}%
\pgfpathlineto{\pgfqpoint{1.696195in}{1.442484in}}%
\pgfpathlineto{\pgfqpoint{1.707209in}{1.446022in}}%
\pgfpathlineto{\pgfqpoint{1.718223in}{1.446506in}}%
\pgfpathlineto{\pgfqpoint{1.729238in}{1.450551in}}%
\pgfpathlineto{\pgfqpoint{1.740252in}{1.453273in}}%
\pgfpathlineto{\pgfqpoint{1.751266in}{1.443123in}}%
\pgfpathlineto{\pgfqpoint{1.762280in}{1.452561in}}%
\pgfpathlineto{\pgfqpoint{1.773295in}{1.451509in}}%
\pgfpathlineto{\pgfqpoint{1.784309in}{1.456565in}}%
\pgfpathlineto{\pgfqpoint{1.795323in}{1.455904in}}%
\pgfpathlineto{\pgfqpoint{1.806337in}{1.464648in}}%
\pgfpathlineto{\pgfqpoint{1.817352in}{1.470340in}}%
\pgfpathlineto{\pgfqpoint{1.828366in}{1.474731in}}%
\pgfpathlineto{\pgfqpoint{1.839380in}{1.473280in}}%
\pgfpathlineto{\pgfqpoint{1.850394in}{1.451189in}}%
\pgfpathlineto{\pgfqpoint{1.861409in}{1.460639in}}%
\pgfpathlineto{\pgfqpoint{1.872423in}{1.460312in}}%
\pgfpathlineto{\pgfqpoint{1.883437in}{1.465056in}}%
\pgfpathlineto{\pgfqpoint{1.894451in}{1.462545in}}%
\pgfpathlineto{\pgfqpoint{1.905466in}{1.464730in}}%
\pgfpathlineto{\pgfqpoint{1.916480in}{1.456505in}}%
\pgfpathlineto{\pgfqpoint{1.927494in}{1.455007in}}%
\pgfpathlineto{\pgfqpoint{1.938508in}{1.444547in}}%
\pgfpathlineto{\pgfqpoint{1.949523in}{1.450115in}}%
\pgfpathlineto{\pgfqpoint{1.960537in}{1.445928in}}%
\pgfpathlineto{\pgfqpoint{1.982565in}{1.453523in}}%
\pgfpathlineto{\pgfqpoint{1.993580in}{1.449416in}}%
\pgfpathlineto{\pgfqpoint{2.004594in}{1.449496in}}%
\pgfpathlineto{\pgfqpoint{2.026622in}{1.458241in}}%
\pgfpathlineto{\pgfqpoint{2.059665in}{1.465259in}}%
\pgfpathlineto{\pgfqpoint{2.070679in}{1.469957in}}%
\pgfpathlineto{\pgfqpoint{2.081694in}{1.473400in}}%
\pgfpathlineto{\pgfqpoint{2.092708in}{1.474129in}}%
\pgfpathlineto{\pgfqpoint{2.103722in}{1.476159in}}%
\pgfpathlineto{\pgfqpoint{2.125751in}{1.472414in}}%
\pgfpathlineto{\pgfqpoint{2.136765in}{1.475690in}}%
\pgfpathlineto{\pgfqpoint{2.147779in}{1.475882in}}%
\pgfpathlineto{\pgfqpoint{2.158793in}{1.478415in}}%
\pgfpathlineto{\pgfqpoint{2.169808in}{1.476312in}}%
\pgfpathlineto{\pgfqpoint{2.180822in}{1.470160in}}%
\pgfpathlineto{\pgfqpoint{2.191836in}{1.470024in}}%
\pgfpathlineto{\pgfqpoint{2.202850in}{1.472373in}}%
\pgfpathlineto{\pgfqpoint{2.213865in}{1.472298in}}%
\pgfpathlineto{\pgfqpoint{2.224879in}{1.464217in}}%
\pgfpathlineto{\pgfqpoint{2.235893in}{1.464835in}}%
\pgfpathlineto{\pgfqpoint{2.246907in}{1.468552in}}%
\pgfpathlineto{\pgfqpoint{2.257922in}{1.471083in}}%
\pgfpathlineto{\pgfqpoint{2.268936in}{1.475465in}}%
\pgfpathlineto{\pgfqpoint{2.290964in}{1.481802in}}%
\pgfpathlineto{\pgfqpoint{2.301979in}{1.480580in}}%
\pgfpathlineto{\pgfqpoint{2.312993in}{1.477176in}}%
\pgfpathlineto{\pgfqpoint{2.335021in}{1.491687in}}%
\pgfpathlineto{\pgfqpoint{2.346036in}{1.495356in}}%
\pgfpathlineto{\pgfqpoint{2.357050in}{1.492388in}}%
\pgfpathlineto{\pgfqpoint{2.368064in}{1.496250in}}%
\pgfpathlineto{\pgfqpoint{2.379078in}{1.502049in}}%
\pgfpathlineto{\pgfqpoint{2.390093in}{1.502865in}}%
\pgfpathlineto{\pgfqpoint{2.412121in}{1.509717in}}%
\pgfpathlineto{\pgfqpoint{2.423135in}{1.512550in}}%
\pgfpathlineto{\pgfqpoint{2.434150in}{1.513217in}}%
\pgfpathlineto{\pgfqpoint{2.445164in}{1.515161in}}%
\pgfpathlineto{\pgfqpoint{2.456178in}{1.513564in}}%
\pgfpathlineto{\pgfqpoint{2.467192in}{1.519482in}}%
\pgfpathlineto{\pgfqpoint{2.478207in}{1.523307in}}%
\pgfpathlineto{\pgfqpoint{2.489221in}{1.524582in}}%
\pgfpathlineto{\pgfqpoint{2.500235in}{1.519485in}}%
\pgfpathlineto{\pgfqpoint{2.511249in}{1.521087in}}%
\pgfpathlineto{\pgfqpoint{2.522264in}{1.525455in}}%
\pgfpathlineto{\pgfqpoint{2.533278in}{1.526110in}}%
\pgfpathlineto{\pgfqpoint{2.544292in}{1.528303in}}%
\pgfpathlineto{\pgfqpoint{2.555306in}{1.535528in}}%
\pgfpathlineto{\pgfqpoint{2.566321in}{1.535285in}}%
\pgfpathlineto{\pgfqpoint{2.577335in}{1.537214in}}%
\pgfpathlineto{\pgfqpoint{2.588349in}{1.535677in}}%
\pgfpathlineto{\pgfqpoint{2.621392in}{1.541839in}}%
\pgfpathlineto{\pgfqpoint{2.632406in}{1.541794in}}%
\pgfpathlineto{\pgfqpoint{2.643421in}{1.545366in}}%
\pgfpathlineto{\pgfqpoint{2.665449in}{1.545578in}}%
\pgfpathlineto{\pgfqpoint{2.676463in}{1.546963in}}%
\pgfpathlineto{\pgfqpoint{2.709506in}{1.562919in}}%
\pgfpathlineto{\pgfqpoint{2.720520in}{1.564076in}}%
\pgfpathlineto{\pgfqpoint{2.731535in}{1.567692in}}%
\pgfpathlineto{\pgfqpoint{2.753563in}{1.565236in}}%
\pgfpathlineto{\pgfqpoint{2.764577in}{1.557443in}}%
\pgfpathlineto{\pgfqpoint{2.775592in}{1.555833in}}%
\pgfpathlineto{\pgfqpoint{2.797620in}{1.555666in}}%
\pgfpathlineto{\pgfqpoint{2.808634in}{1.560329in}}%
\pgfpathlineto{\pgfqpoint{2.830663in}{1.557640in}}%
\pgfpathlineto{\pgfqpoint{2.841677in}{1.561735in}}%
\pgfpathlineto{\pgfqpoint{2.874720in}{1.564662in}}%
\pgfpathlineto{\pgfqpoint{2.896748in}{1.565571in}}%
\pgfpathlineto{\pgfqpoint{2.907763in}{1.567957in}}%
\pgfpathlineto{\pgfqpoint{2.918777in}{1.566961in}}%
\pgfpathlineto{\pgfqpoint{2.929791in}{1.563653in}}%
\pgfpathlineto{\pgfqpoint{2.951820in}{1.564723in}}%
\pgfpathlineto{\pgfqpoint{2.962834in}{1.568275in}}%
\pgfpathlineto{\pgfqpoint{2.973848in}{1.568210in}}%
\pgfpathlineto{\pgfqpoint{2.995877in}{1.565792in}}%
\pgfpathlineto{\pgfqpoint{3.006891in}{1.567323in}}%
\pgfpathlineto{\pgfqpoint{3.028919in}{1.572732in}}%
\pgfpathlineto{\pgfqpoint{3.039934in}{1.571561in}}%
\pgfpathlineto{\pgfqpoint{3.061962in}{1.574310in}}%
\pgfpathlineto{\pgfqpoint{3.095005in}{1.579876in}}%
\pgfpathlineto{\pgfqpoint{3.117033in}{1.584274in}}%
\pgfpathlineto{\pgfqpoint{3.128048in}{1.585158in}}%
\pgfpathlineto{\pgfqpoint{3.139062in}{1.584578in}}%
\pgfpathlineto{\pgfqpoint{3.150076in}{1.586875in}}%
\pgfpathlineto{\pgfqpoint{3.161090in}{1.584587in}}%
\pgfpathlineto{\pgfqpoint{3.172105in}{1.583581in}}%
\pgfpathlineto{\pgfqpoint{3.183119in}{1.585368in}}%
\pgfpathlineto{\pgfqpoint{3.194133in}{1.590471in}}%
\pgfpathlineto{\pgfqpoint{3.205147in}{1.590052in}}%
\pgfpathlineto{\pgfqpoint{3.238190in}{1.591983in}}%
\pgfpathlineto{\pgfqpoint{3.260219in}{1.591039in}}%
\pgfpathlineto{\pgfqpoint{3.271233in}{1.593884in}}%
\pgfpathlineto{\pgfqpoint{3.282247in}{1.593021in}}%
\pgfpathlineto{\pgfqpoint{3.293261in}{1.595243in}}%
\pgfpathlineto{\pgfqpoint{3.304276in}{1.594670in}}%
\pgfpathlineto{\pgfqpoint{3.315290in}{1.595970in}}%
\pgfpathlineto{\pgfqpoint{3.326304in}{1.595330in}}%
\pgfpathlineto{\pgfqpoint{3.370361in}{1.598393in}}%
\pgfpathlineto{\pgfqpoint{3.381375in}{1.598835in}}%
\pgfpathlineto{\pgfqpoint{3.392390in}{1.596088in}}%
\pgfpathlineto{\pgfqpoint{3.403404in}{1.603510in}}%
\pgfpathlineto{\pgfqpoint{3.469489in}{1.609430in}}%
\pgfpathlineto{\pgfqpoint{3.480504in}{1.609553in}}%
\pgfpathlineto{\pgfqpoint{3.491518in}{1.602348in}}%
\pgfpathlineto{\pgfqpoint{3.513546in}{1.605813in}}%
\pgfpathlineto{\pgfqpoint{3.524561in}{1.601006in}}%
\pgfpathlineto{\pgfqpoint{3.546589in}{1.598718in}}%
\pgfpathlineto{\pgfqpoint{3.557603in}{1.600085in}}%
\pgfpathlineto{\pgfqpoint{3.568618in}{1.604436in}}%
\pgfpathlineto{\pgfqpoint{3.579632in}{1.607263in}}%
\pgfpathlineto{\pgfqpoint{3.590646in}{1.606255in}}%
\pgfpathlineto{\pgfqpoint{3.601660in}{1.609945in}}%
\pgfpathlineto{\pgfqpoint{3.656732in}{1.613157in}}%
\pgfpathlineto{\pgfqpoint{3.667746in}{1.615451in}}%
\pgfpathlineto{\pgfqpoint{3.678760in}{1.611223in}}%
\pgfpathlineto{\pgfqpoint{3.678760in}{1.611223in}}%
\pgfusepath{stroke}%
\end{pgfscope}%
\begin{pgfscope}%
\pgfpathrectangle{\pgfqpoint{0.550713in}{0.408431in}}{\pgfqpoint{3.139062in}{1.490773in}}%
\pgfusepath{clip}%
\pgfsetrectcap%
\pgfsetroundjoin%
\pgfsetlinewidth{0.853187pt}%
\definecolor{currentstroke}{rgb}{0.478431,0.435294,0.674510}%
\pgfsetstrokecolor{currentstroke}%
\pgfsetdash{}{0pt}%
\pgfpathmoveto{\pgfqpoint{0.550713in}{1.196228in}}%
\pgfpathlineto{\pgfqpoint{0.561727in}{1.255814in}}%
\pgfpathlineto{\pgfqpoint{0.572741in}{1.253701in}}%
\pgfpathlineto{\pgfqpoint{0.583755in}{1.231942in}}%
\pgfpathlineto{\pgfqpoint{0.605784in}{1.232952in}}%
\pgfpathlineto{\pgfqpoint{0.616798in}{1.224335in}}%
\pgfpathlineto{\pgfqpoint{0.627812in}{1.235150in}}%
\pgfpathlineto{\pgfqpoint{0.638827in}{1.239717in}}%
\pgfpathlineto{\pgfqpoint{0.649841in}{1.235286in}}%
\pgfpathlineto{\pgfqpoint{0.660855in}{1.234083in}}%
\pgfpathlineto{\pgfqpoint{0.671869in}{1.210594in}}%
\pgfpathlineto{\pgfqpoint{0.682884in}{1.212094in}}%
\pgfpathlineto{\pgfqpoint{0.693898in}{1.204173in}}%
\pgfpathlineto{\pgfqpoint{0.704912in}{1.185787in}}%
\pgfpathlineto{\pgfqpoint{0.715926in}{1.209169in}}%
\pgfpathlineto{\pgfqpoint{0.726941in}{1.209197in}}%
\pgfpathlineto{\pgfqpoint{0.737955in}{1.231867in}}%
\pgfpathlineto{\pgfqpoint{0.748969in}{1.236484in}}%
\pgfpathlineto{\pgfqpoint{0.759983in}{1.221107in}}%
\pgfpathlineto{\pgfqpoint{0.770998in}{1.228653in}}%
\pgfpathlineto{\pgfqpoint{0.782012in}{1.254381in}}%
\pgfpathlineto{\pgfqpoint{0.793026in}{1.255233in}}%
\pgfpathlineto{\pgfqpoint{0.804040in}{1.215654in}}%
\pgfpathlineto{\pgfqpoint{0.815055in}{1.236860in}}%
\pgfpathlineto{\pgfqpoint{0.826069in}{1.215197in}}%
\pgfpathlineto{\pgfqpoint{0.837083in}{1.233373in}}%
\pgfpathlineto{\pgfqpoint{0.848097in}{1.212101in}}%
\pgfpathlineto{\pgfqpoint{0.859112in}{1.207106in}}%
\pgfpathlineto{\pgfqpoint{0.870126in}{1.231719in}}%
\pgfpathlineto{\pgfqpoint{0.881140in}{1.225505in}}%
\pgfpathlineto{\pgfqpoint{0.892154in}{1.206189in}}%
\pgfpathlineto{\pgfqpoint{0.903169in}{1.242632in}}%
\pgfpathlineto{\pgfqpoint{0.914183in}{1.213548in}}%
\pgfpathlineto{\pgfqpoint{0.925197in}{1.257530in}}%
\pgfpathlineto{\pgfqpoint{0.936211in}{1.264571in}}%
\pgfpathlineto{\pgfqpoint{0.958240in}{1.247152in}}%
\pgfpathlineto{\pgfqpoint{0.980268in}{1.243046in}}%
\pgfpathlineto{\pgfqpoint{0.991283in}{1.251126in}}%
\pgfpathlineto{\pgfqpoint{1.002297in}{1.192686in}}%
\pgfpathlineto{\pgfqpoint{1.013311in}{1.179966in}}%
\pgfpathlineto{\pgfqpoint{1.024325in}{1.221466in}}%
\pgfpathlineto{\pgfqpoint{1.035340in}{1.238344in}}%
\pgfpathlineto{\pgfqpoint{1.046354in}{1.273217in}}%
\pgfpathlineto{\pgfqpoint{1.057368in}{1.277839in}}%
\pgfpathlineto{\pgfqpoint{1.068382in}{1.284097in}}%
\pgfpathlineto{\pgfqpoint{1.079397in}{1.277700in}}%
\pgfpathlineto{\pgfqpoint{1.090411in}{1.256751in}}%
\pgfpathlineto{\pgfqpoint{1.101425in}{1.273311in}}%
\pgfpathlineto{\pgfqpoint{1.112439in}{1.283105in}}%
\pgfpathlineto{\pgfqpoint{1.123454in}{1.294946in}}%
\pgfpathlineto{\pgfqpoint{1.134468in}{1.293381in}}%
\pgfpathlineto{\pgfqpoint{1.145482in}{1.302074in}}%
\pgfpathlineto{\pgfqpoint{1.156496in}{1.302436in}}%
\pgfpathlineto{\pgfqpoint{1.178525in}{1.276673in}}%
\pgfpathlineto{\pgfqpoint{1.189539in}{1.249579in}}%
\pgfpathlineto{\pgfqpoint{1.200553in}{1.234672in}}%
\pgfpathlineto{\pgfqpoint{1.211568in}{1.231050in}}%
\pgfpathlineto{\pgfqpoint{1.222582in}{1.223747in}}%
\pgfpathlineto{\pgfqpoint{1.233596in}{1.232704in}}%
\pgfpathlineto{\pgfqpoint{1.244610in}{1.237239in}}%
\pgfpathlineto{\pgfqpoint{1.255625in}{1.216046in}}%
\pgfpathlineto{\pgfqpoint{1.266639in}{1.227139in}}%
\pgfpathlineto{\pgfqpoint{1.277653in}{1.221883in}}%
\pgfpathlineto{\pgfqpoint{1.288667in}{1.214028in}}%
\pgfpathlineto{\pgfqpoint{1.299682in}{1.236119in}}%
\pgfpathlineto{\pgfqpoint{1.321710in}{1.234076in}}%
\pgfpathlineto{\pgfqpoint{1.332725in}{1.225471in}}%
\pgfpathlineto{\pgfqpoint{1.343739in}{1.226824in}}%
\pgfpathlineto{\pgfqpoint{1.354753in}{1.241475in}}%
\pgfpathlineto{\pgfqpoint{1.365767in}{1.234925in}}%
\pgfpathlineto{\pgfqpoint{1.376782in}{1.272000in}}%
\pgfpathlineto{\pgfqpoint{1.387796in}{1.255202in}}%
\pgfpathlineto{\pgfqpoint{1.398810in}{1.278940in}}%
\pgfpathlineto{\pgfqpoint{1.409824in}{1.280965in}}%
\pgfpathlineto{\pgfqpoint{1.420839in}{1.284901in}}%
\pgfpathlineto{\pgfqpoint{1.431853in}{1.283041in}}%
\pgfpathlineto{\pgfqpoint{1.442867in}{1.265414in}}%
\pgfpathlineto{\pgfqpoint{1.453881in}{1.270973in}}%
\pgfpathlineto{\pgfqpoint{1.464896in}{1.307975in}}%
\pgfpathlineto{\pgfqpoint{1.475910in}{1.310897in}}%
\pgfpathlineto{\pgfqpoint{1.486924in}{1.293260in}}%
\pgfpathlineto{\pgfqpoint{1.497938in}{1.294920in}}%
\pgfpathlineto{\pgfqpoint{1.508953in}{1.272092in}}%
\pgfpathlineto{\pgfqpoint{1.519967in}{1.260689in}}%
\pgfpathlineto{\pgfqpoint{1.530981in}{1.224219in}}%
\pgfpathlineto{\pgfqpoint{1.541995in}{1.156101in}}%
\pgfpathlineto{\pgfqpoint{1.553010in}{1.161297in}}%
\pgfpathlineto{\pgfqpoint{1.564024in}{1.170047in}}%
\pgfpathlineto{\pgfqpoint{1.575038in}{1.174778in}}%
\pgfpathlineto{\pgfqpoint{1.597067in}{1.196092in}}%
\pgfpathlineto{\pgfqpoint{1.608081in}{1.184492in}}%
\pgfpathlineto{\pgfqpoint{1.619095in}{1.124459in}}%
\pgfpathlineto{\pgfqpoint{1.630109in}{1.113979in}}%
\pgfpathlineto{\pgfqpoint{1.641124in}{1.099721in}}%
\pgfpathlineto{\pgfqpoint{1.652138in}{1.090577in}}%
\pgfpathlineto{\pgfqpoint{1.663152in}{1.098883in}}%
\pgfpathlineto{\pgfqpoint{1.674166in}{1.117326in}}%
\pgfpathlineto{\pgfqpoint{1.685181in}{1.151310in}}%
\pgfpathlineto{\pgfqpoint{1.696195in}{1.131924in}}%
\pgfpathlineto{\pgfqpoint{1.707209in}{1.128023in}}%
\pgfpathlineto{\pgfqpoint{1.718223in}{1.143584in}}%
\pgfpathlineto{\pgfqpoint{1.729238in}{1.147600in}}%
\pgfpathlineto{\pgfqpoint{1.740252in}{1.119386in}}%
\pgfpathlineto{\pgfqpoint{1.762280in}{1.129966in}}%
\pgfpathlineto{\pgfqpoint{1.773295in}{1.126870in}}%
\pgfpathlineto{\pgfqpoint{1.784309in}{1.132902in}}%
\pgfpathlineto{\pgfqpoint{1.795323in}{1.166262in}}%
\pgfpathlineto{\pgfqpoint{1.806337in}{1.160753in}}%
\pgfpathlineto{\pgfqpoint{1.817352in}{1.163156in}}%
\pgfpathlineto{\pgfqpoint{1.828366in}{1.138030in}}%
\pgfpathlineto{\pgfqpoint{1.839380in}{1.075158in}}%
\pgfpathlineto{\pgfqpoint{1.850394in}{1.097924in}}%
\pgfpathlineto{\pgfqpoint{1.861409in}{1.075868in}}%
\pgfpathlineto{\pgfqpoint{1.872423in}{1.081825in}}%
\pgfpathlineto{\pgfqpoint{1.883437in}{1.077644in}}%
\pgfpathlineto{\pgfqpoint{1.894451in}{1.093280in}}%
\pgfpathlineto{\pgfqpoint{1.905466in}{1.096219in}}%
\pgfpathlineto{\pgfqpoint{1.916480in}{1.093485in}}%
\pgfpathlineto{\pgfqpoint{1.927494in}{1.109040in}}%
\pgfpathlineto{\pgfqpoint{1.938508in}{1.036836in}}%
\pgfpathlineto{\pgfqpoint{1.949523in}{1.105236in}}%
\pgfpathlineto{\pgfqpoint{1.960537in}{1.114881in}}%
\pgfpathlineto{\pgfqpoint{1.971551in}{1.136353in}}%
\pgfpathlineto{\pgfqpoint{1.982565in}{1.168001in}}%
\pgfpathlineto{\pgfqpoint{1.993580in}{1.154241in}}%
\pgfpathlineto{\pgfqpoint{2.004594in}{1.105482in}}%
\pgfpathlineto{\pgfqpoint{2.015608in}{1.108497in}}%
\pgfpathlineto{\pgfqpoint{2.026622in}{1.119505in}}%
\pgfpathlineto{\pgfqpoint{2.037637in}{1.154354in}}%
\pgfpathlineto{\pgfqpoint{2.048651in}{1.225826in}}%
\pgfpathlineto{\pgfqpoint{2.059665in}{1.243305in}}%
\pgfpathlineto{\pgfqpoint{2.070679in}{1.235786in}}%
\pgfpathlineto{\pgfqpoint{2.081694in}{1.233532in}}%
\pgfpathlineto{\pgfqpoint{2.092708in}{1.218470in}}%
\pgfpathlineto{\pgfqpoint{2.103722in}{1.196235in}}%
\pgfpathlineto{\pgfqpoint{2.114736in}{1.195564in}}%
\pgfpathlineto{\pgfqpoint{2.125751in}{1.199996in}}%
\pgfpathlineto{\pgfqpoint{2.136765in}{1.221649in}}%
\pgfpathlineto{\pgfqpoint{2.147779in}{1.211937in}}%
\pgfpathlineto{\pgfqpoint{2.158793in}{1.209600in}}%
\pgfpathlineto{\pgfqpoint{2.169808in}{1.167802in}}%
\pgfpathlineto{\pgfqpoint{2.180822in}{1.165085in}}%
\pgfpathlineto{\pgfqpoint{2.191836in}{1.192808in}}%
\pgfpathlineto{\pgfqpoint{2.202850in}{1.194359in}}%
\pgfpathlineto{\pgfqpoint{2.213865in}{1.182828in}}%
\pgfpathlineto{\pgfqpoint{2.224879in}{1.185394in}}%
\pgfpathlineto{\pgfqpoint{2.235893in}{1.210362in}}%
\pgfpathlineto{\pgfqpoint{2.257922in}{1.218751in}}%
\pgfpathlineto{\pgfqpoint{2.268936in}{1.256805in}}%
\pgfpathlineto{\pgfqpoint{2.279950in}{1.282998in}}%
\pgfpathlineto{\pgfqpoint{2.290964in}{1.283736in}}%
\pgfpathlineto{\pgfqpoint{2.301979in}{1.279163in}}%
\pgfpathlineto{\pgfqpoint{2.312993in}{1.257972in}}%
\pgfpathlineto{\pgfqpoint{2.324007in}{1.269355in}}%
\pgfpathlineto{\pgfqpoint{2.335021in}{1.247213in}}%
\pgfpathlineto{\pgfqpoint{2.346036in}{1.255516in}}%
\pgfpathlineto{\pgfqpoint{2.357050in}{1.274143in}}%
\pgfpathlineto{\pgfqpoint{2.368064in}{1.262349in}}%
\pgfpathlineto{\pgfqpoint{2.379078in}{1.289045in}}%
\pgfpathlineto{\pgfqpoint{2.390093in}{1.287819in}}%
\pgfpathlineto{\pgfqpoint{2.412121in}{1.261974in}}%
\pgfpathlineto{\pgfqpoint{2.423135in}{1.255855in}}%
\pgfpathlineto{\pgfqpoint{2.434150in}{1.257875in}}%
\pgfpathlineto{\pgfqpoint{2.445164in}{1.257298in}}%
\pgfpathlineto{\pgfqpoint{2.456178in}{1.247168in}}%
\pgfpathlineto{\pgfqpoint{2.467192in}{1.252935in}}%
\pgfpathlineto{\pgfqpoint{2.478207in}{1.244497in}}%
\pgfpathlineto{\pgfqpoint{2.489221in}{1.239710in}}%
\pgfpathlineto{\pgfqpoint{2.500235in}{1.231064in}}%
\pgfpathlineto{\pgfqpoint{2.511249in}{1.250982in}}%
\pgfpathlineto{\pgfqpoint{2.522264in}{1.274305in}}%
\pgfpathlineto{\pgfqpoint{2.533278in}{1.291603in}}%
\pgfpathlineto{\pgfqpoint{2.544292in}{1.306252in}}%
\pgfpathlineto{\pgfqpoint{2.555306in}{1.308952in}}%
\pgfpathlineto{\pgfqpoint{2.566321in}{1.285818in}}%
\pgfpathlineto{\pgfqpoint{2.577335in}{1.279484in}}%
\pgfpathlineto{\pgfqpoint{2.588349in}{1.280168in}}%
\pgfpathlineto{\pgfqpoint{2.599363in}{1.292311in}}%
\pgfpathlineto{\pgfqpoint{2.610378in}{1.310332in}}%
\pgfpathlineto{\pgfqpoint{2.621392in}{1.321964in}}%
\pgfpathlineto{\pgfqpoint{2.632406in}{1.305642in}}%
\pgfpathlineto{\pgfqpoint{2.643421in}{1.292183in}}%
\pgfpathlineto{\pgfqpoint{2.654435in}{1.259733in}}%
\pgfpathlineto{\pgfqpoint{2.665449in}{1.263817in}}%
\pgfpathlineto{\pgfqpoint{2.676463in}{1.243376in}}%
\pgfpathlineto{\pgfqpoint{2.687478in}{1.244164in}}%
\pgfpathlineto{\pgfqpoint{2.698492in}{1.257773in}}%
\pgfpathlineto{\pgfqpoint{2.709506in}{1.266221in}}%
\pgfpathlineto{\pgfqpoint{2.720520in}{1.247926in}}%
\pgfpathlineto{\pgfqpoint{2.731535in}{1.250603in}}%
\pgfpathlineto{\pgfqpoint{2.742549in}{1.286061in}}%
\pgfpathlineto{\pgfqpoint{2.753563in}{1.296102in}}%
\pgfpathlineto{\pgfqpoint{2.764577in}{1.273364in}}%
\pgfpathlineto{\pgfqpoint{2.775592in}{1.285910in}}%
\pgfpathlineto{\pgfqpoint{2.786606in}{1.312139in}}%
\pgfpathlineto{\pgfqpoint{2.797620in}{1.309367in}}%
\pgfpathlineto{\pgfqpoint{2.808634in}{1.312442in}}%
\pgfpathlineto{\pgfqpoint{2.819649in}{1.300948in}}%
\pgfpathlineto{\pgfqpoint{2.830663in}{1.281386in}}%
\pgfpathlineto{\pgfqpoint{2.841677in}{1.282555in}}%
\pgfpathlineto{\pgfqpoint{2.852691in}{1.295704in}}%
\pgfpathlineto{\pgfqpoint{2.863706in}{1.270677in}}%
\pgfpathlineto{\pgfqpoint{2.874720in}{1.280651in}}%
\pgfpathlineto{\pgfqpoint{2.885734in}{1.278321in}}%
\pgfpathlineto{\pgfqpoint{2.896748in}{1.271876in}}%
\pgfpathlineto{\pgfqpoint{2.907763in}{1.269721in}}%
\pgfpathlineto{\pgfqpoint{2.918777in}{1.264008in}}%
\pgfpathlineto{\pgfqpoint{2.929791in}{1.264854in}}%
\pgfpathlineto{\pgfqpoint{2.940805in}{1.240434in}}%
\pgfpathlineto{\pgfqpoint{2.951820in}{1.255395in}}%
\pgfpathlineto{\pgfqpoint{2.962834in}{1.240254in}}%
\pgfpathlineto{\pgfqpoint{2.984862in}{1.219801in}}%
\pgfpathlineto{\pgfqpoint{2.995877in}{1.245440in}}%
\pgfpathlineto{\pgfqpoint{3.006891in}{1.240061in}}%
\pgfpathlineto{\pgfqpoint{3.017905in}{1.217543in}}%
\pgfpathlineto{\pgfqpoint{3.028919in}{1.224880in}}%
\pgfpathlineto{\pgfqpoint{3.039934in}{1.208940in}}%
\pgfpathlineto{\pgfqpoint{3.050948in}{1.204961in}}%
\pgfpathlineto{\pgfqpoint{3.061962in}{1.269648in}}%
\pgfpathlineto{\pgfqpoint{3.072976in}{1.281924in}}%
\pgfpathlineto{\pgfqpoint{3.083991in}{1.273073in}}%
\pgfpathlineto{\pgfqpoint{3.095005in}{1.283870in}}%
\pgfpathlineto{\pgfqpoint{3.106019in}{1.280144in}}%
\pgfpathlineto{\pgfqpoint{3.117033in}{1.268167in}}%
\pgfpathlineto{\pgfqpoint{3.128048in}{1.286544in}}%
\pgfpathlineto{\pgfqpoint{3.139062in}{1.291842in}}%
\pgfpathlineto{\pgfqpoint{3.150076in}{1.240851in}}%
\pgfpathlineto{\pgfqpoint{3.161090in}{1.216321in}}%
\pgfpathlineto{\pgfqpoint{3.172105in}{1.226046in}}%
\pgfpathlineto{\pgfqpoint{3.194133in}{1.270172in}}%
\pgfpathlineto{\pgfqpoint{3.205147in}{1.285397in}}%
\pgfpathlineto{\pgfqpoint{3.216162in}{1.282455in}}%
\pgfpathlineto{\pgfqpoint{3.227176in}{1.286138in}}%
\pgfpathlineto{\pgfqpoint{3.238190in}{1.279543in}}%
\pgfpathlineto{\pgfqpoint{3.249204in}{1.274628in}}%
\pgfpathlineto{\pgfqpoint{3.260219in}{1.231954in}}%
\pgfpathlineto{\pgfqpoint{3.271233in}{1.230761in}}%
\pgfpathlineto{\pgfqpoint{3.282247in}{1.240980in}}%
\pgfpathlineto{\pgfqpoint{3.293261in}{1.232547in}}%
\pgfpathlineto{\pgfqpoint{3.304276in}{1.198784in}}%
\pgfpathlineto{\pgfqpoint{3.315290in}{1.192940in}}%
\pgfpathlineto{\pgfqpoint{3.326304in}{1.174706in}}%
\pgfpathlineto{\pgfqpoint{3.337318in}{1.179603in}}%
\pgfpathlineto{\pgfqpoint{3.348333in}{1.188310in}}%
\pgfpathlineto{\pgfqpoint{3.370361in}{1.260584in}}%
\pgfpathlineto{\pgfqpoint{3.381375in}{1.263801in}}%
\pgfpathlineto{\pgfqpoint{3.392390in}{1.238207in}}%
\pgfpathlineto{\pgfqpoint{3.403404in}{1.246065in}}%
\pgfpathlineto{\pgfqpoint{3.414418in}{1.258276in}}%
\pgfpathlineto{\pgfqpoint{3.425432in}{1.238964in}}%
\pgfpathlineto{\pgfqpoint{3.436447in}{1.235826in}}%
\pgfpathlineto{\pgfqpoint{3.458475in}{1.202271in}}%
\pgfpathlineto{\pgfqpoint{3.469489in}{1.230731in}}%
\pgfpathlineto{\pgfqpoint{3.480504in}{1.233479in}}%
\pgfpathlineto{\pgfqpoint{3.491518in}{1.219378in}}%
\pgfpathlineto{\pgfqpoint{3.502532in}{1.212200in}}%
\pgfpathlineto{\pgfqpoint{3.513546in}{1.228898in}}%
\pgfpathlineto{\pgfqpoint{3.524561in}{1.218070in}}%
\pgfpathlineto{\pgfqpoint{3.535575in}{1.186194in}}%
\pgfpathlineto{\pgfqpoint{3.546589in}{1.184714in}}%
\pgfpathlineto{\pgfqpoint{3.557603in}{1.155777in}}%
\pgfpathlineto{\pgfqpoint{3.568618in}{1.145751in}}%
\pgfpathlineto{\pgfqpoint{3.579632in}{1.155924in}}%
\pgfpathlineto{\pgfqpoint{3.590646in}{1.163536in}}%
\pgfpathlineto{\pgfqpoint{3.601660in}{1.172809in}}%
\pgfpathlineto{\pgfqpoint{3.612675in}{1.168515in}}%
\pgfpathlineto{\pgfqpoint{3.623689in}{1.147675in}}%
\pgfpathlineto{\pgfqpoint{3.634703in}{1.148656in}}%
\pgfpathlineto{\pgfqpoint{3.645717in}{1.192668in}}%
\pgfpathlineto{\pgfqpoint{3.656732in}{1.176876in}}%
\pgfpathlineto{\pgfqpoint{3.667746in}{1.180845in}}%
\pgfpathlineto{\pgfqpoint{3.678760in}{1.161986in}}%
\pgfpathlineto{\pgfqpoint{3.678760in}{1.161986in}}%
\pgfusepath{stroke}%
\end{pgfscope}%
\begin{pgfscope}%
\pgfpathrectangle{\pgfqpoint{0.550713in}{0.408431in}}{\pgfqpoint{3.139062in}{1.490773in}}%
\pgfusepath{clip}%
\pgfsetbuttcap%
\pgfsetroundjoin%
\pgfsetlinewidth{0.853187pt}%
\definecolor{currentstroke}{rgb}{0.631373,0.062745,0.207843}%
\pgfsetstrokecolor{currentstroke}%
\pgfsetdash{{0.850000pt}{1.402500pt}}{0.000000pt}%
\pgfpathmoveto{\pgfqpoint{0.550713in}{1.027282in}}%
\pgfpathlineto{\pgfqpoint{0.561727in}{1.067615in}}%
\pgfpathlineto{\pgfqpoint{0.572741in}{1.067415in}}%
\pgfpathlineto{\pgfqpoint{0.583755in}{1.077573in}}%
\pgfpathlineto{\pgfqpoint{0.594770in}{1.098248in}}%
\pgfpathlineto{\pgfqpoint{0.605784in}{1.121392in}}%
\pgfpathlineto{\pgfqpoint{0.627812in}{1.154147in}}%
\pgfpathlineto{\pgfqpoint{0.638827in}{1.188812in}}%
\pgfpathlineto{\pgfqpoint{0.649841in}{1.210620in}}%
\pgfpathlineto{\pgfqpoint{0.660855in}{1.228243in}}%
\pgfpathlineto{\pgfqpoint{0.671869in}{1.238505in}}%
\pgfpathlineto{\pgfqpoint{0.693898in}{1.268638in}}%
\pgfpathlineto{\pgfqpoint{0.704912in}{1.276982in}}%
\pgfpathlineto{\pgfqpoint{0.715926in}{1.292933in}}%
\pgfpathlineto{\pgfqpoint{0.726941in}{1.311175in}}%
\pgfpathlineto{\pgfqpoint{0.737955in}{1.325901in}}%
\pgfpathlineto{\pgfqpoint{0.748969in}{1.330674in}}%
\pgfpathlineto{\pgfqpoint{0.759983in}{1.346606in}}%
\pgfpathlineto{\pgfqpoint{0.770998in}{1.367402in}}%
\pgfpathlineto{\pgfqpoint{0.782012in}{1.395011in}}%
\pgfpathlineto{\pgfqpoint{0.793026in}{1.409518in}}%
\pgfpathlineto{\pgfqpoint{0.804040in}{1.414663in}}%
\pgfpathlineto{\pgfqpoint{0.826069in}{1.435482in}}%
\pgfpathlineto{\pgfqpoint{0.837083in}{1.455592in}}%
\pgfpathlineto{\pgfqpoint{0.848097in}{1.480386in}}%
\pgfpathlineto{\pgfqpoint{0.870126in}{1.519042in}}%
\pgfpathlineto{\pgfqpoint{0.881140in}{1.527217in}}%
\pgfpathlineto{\pgfqpoint{0.892154in}{1.550464in}}%
\pgfpathlineto{\pgfqpoint{0.925197in}{1.597770in}}%
\pgfpathlineto{\pgfqpoint{0.936211in}{1.612755in}}%
\pgfpathlineto{\pgfqpoint{0.947226in}{1.625631in}}%
\pgfpathlineto{\pgfqpoint{0.958240in}{1.641872in}}%
\pgfpathlineto{\pgfqpoint{0.969254in}{1.650953in}}%
\pgfpathlineto{\pgfqpoint{0.980268in}{1.653673in}}%
\pgfpathlineto{\pgfqpoint{0.991283in}{1.663101in}}%
\pgfpathlineto{\pgfqpoint{1.002297in}{1.658119in}}%
\pgfpathlineto{\pgfqpoint{1.013311in}{1.663897in}}%
\pgfpathlineto{\pgfqpoint{1.024325in}{1.672919in}}%
\pgfpathlineto{\pgfqpoint{1.035340in}{1.679084in}}%
\pgfpathlineto{\pgfqpoint{1.046354in}{1.683498in}}%
\pgfpathlineto{\pgfqpoint{1.068382in}{1.702659in}}%
\pgfpathlineto{\pgfqpoint{1.079397in}{1.707902in}}%
\pgfpathlineto{\pgfqpoint{1.101425in}{1.721741in}}%
\pgfpathlineto{\pgfqpoint{1.112439in}{1.725733in}}%
\pgfpathlineto{\pgfqpoint{1.123454in}{1.732035in}}%
\pgfpathlineto{\pgfqpoint{1.134468in}{1.736538in}}%
\pgfpathlineto{\pgfqpoint{1.145482in}{1.742517in}}%
\pgfpathlineto{\pgfqpoint{1.156496in}{1.745753in}}%
\pgfpathlineto{\pgfqpoint{1.167511in}{1.752875in}}%
\pgfpathlineto{\pgfqpoint{1.178525in}{1.755303in}}%
\pgfpathlineto{\pgfqpoint{1.189539in}{1.759804in}}%
\pgfpathlineto{\pgfqpoint{1.211568in}{1.764258in}}%
\pgfpathlineto{\pgfqpoint{1.222582in}{1.765356in}}%
\pgfpathlineto{\pgfqpoint{1.233596in}{1.768606in}}%
\pgfpathlineto{\pgfqpoint{1.255625in}{1.771475in}}%
\pgfpathlineto{\pgfqpoint{1.310696in}{1.773858in}}%
\pgfpathlineto{\pgfqpoint{1.453881in}{1.779472in}}%
\pgfpathlineto{\pgfqpoint{1.916480in}{1.779706in}}%
\pgfpathlineto{\pgfqpoint{3.678760in}{1.779793in}}%
\pgfpathlineto{\pgfqpoint{3.678760in}{1.779793in}}%
\pgfusepath{stroke}%
\end{pgfscope}%
\begin{pgfscope}%
\pgfpathrectangle{\pgfqpoint{0.550713in}{0.408431in}}{\pgfqpoint{3.139062in}{1.490773in}}%
\pgfusepath{clip}%
\pgfsetbuttcap%
\pgfsetroundjoin%
\pgfsetlinewidth{0.853187pt}%
\definecolor{currentstroke}{rgb}{0.890196,0.000000,0.400000}%
\pgfsetstrokecolor{currentstroke}%
\pgfsetdash{{0.850000pt}{1.402500pt}}{0.000000pt}%
\pgfpathmoveto{\pgfqpoint{0.550713in}{1.027029in}}%
\pgfpathlineto{\pgfqpoint{0.561727in}{1.066736in}}%
\pgfpathlineto{\pgfqpoint{0.572741in}{1.063449in}}%
\pgfpathlineto{\pgfqpoint{0.594770in}{1.087162in}}%
\pgfpathlineto{\pgfqpoint{0.605784in}{1.113194in}}%
\pgfpathlineto{\pgfqpoint{0.616798in}{1.128311in}}%
\pgfpathlineto{\pgfqpoint{0.627812in}{1.140047in}}%
\pgfpathlineto{\pgfqpoint{0.638827in}{1.174286in}}%
\pgfpathlineto{\pgfqpoint{0.649841in}{1.191371in}}%
\pgfpathlineto{\pgfqpoint{0.660855in}{1.212567in}}%
\pgfpathlineto{\pgfqpoint{0.671869in}{1.220868in}}%
\pgfpathlineto{\pgfqpoint{0.682884in}{1.241346in}}%
\pgfpathlineto{\pgfqpoint{0.693898in}{1.249454in}}%
\pgfpathlineto{\pgfqpoint{0.704912in}{1.259526in}}%
\pgfpathlineto{\pgfqpoint{0.715926in}{1.288908in}}%
\pgfpathlineto{\pgfqpoint{0.726941in}{1.286554in}}%
\pgfpathlineto{\pgfqpoint{0.737955in}{1.280207in}}%
\pgfpathlineto{\pgfqpoint{0.748969in}{1.257679in}}%
\pgfpathlineto{\pgfqpoint{0.759983in}{1.251666in}}%
\pgfpathlineto{\pgfqpoint{0.770998in}{1.248454in}}%
\pgfpathlineto{\pgfqpoint{0.782012in}{1.233417in}}%
\pgfpathlineto{\pgfqpoint{0.793026in}{1.199472in}}%
\pgfpathlineto{\pgfqpoint{0.804040in}{1.194291in}}%
\pgfpathlineto{\pgfqpoint{0.815055in}{1.190691in}}%
\pgfpathlineto{\pgfqpoint{0.826069in}{1.146055in}}%
\pgfpathlineto{\pgfqpoint{0.837083in}{1.135678in}}%
\pgfpathlineto{\pgfqpoint{0.848097in}{1.128369in}}%
\pgfpathlineto{\pgfqpoint{0.859112in}{1.136720in}}%
\pgfpathlineto{\pgfqpoint{0.870126in}{1.118785in}}%
\pgfpathlineto{\pgfqpoint{0.881140in}{1.117995in}}%
\pgfpathlineto{\pgfqpoint{0.892154in}{1.154767in}}%
\pgfpathlineto{\pgfqpoint{0.903169in}{1.182784in}}%
\pgfpathlineto{\pgfqpoint{0.914183in}{1.178878in}}%
\pgfpathlineto{\pgfqpoint{0.936211in}{1.203686in}}%
\pgfpathlineto{\pgfqpoint{0.947226in}{1.209346in}}%
\pgfpathlineto{\pgfqpoint{0.958240in}{1.217326in}}%
\pgfpathlineto{\pgfqpoint{0.969254in}{1.231598in}}%
\pgfpathlineto{\pgfqpoint{0.980268in}{1.215057in}}%
\pgfpathlineto{\pgfqpoint{1.002297in}{1.223907in}}%
\pgfpathlineto{\pgfqpoint{1.013311in}{1.224415in}}%
\pgfpathlineto{\pgfqpoint{1.024325in}{1.247372in}}%
\pgfpathlineto{\pgfqpoint{1.035340in}{1.256940in}}%
\pgfpathlineto{\pgfqpoint{1.046354in}{1.252171in}}%
\pgfpathlineto{\pgfqpoint{1.057368in}{1.264299in}}%
\pgfpathlineto{\pgfqpoint{1.068382in}{1.274742in}}%
\pgfpathlineto{\pgfqpoint{1.079397in}{1.268791in}}%
\pgfpathlineto{\pgfqpoint{1.090411in}{1.260700in}}%
\pgfpathlineto{\pgfqpoint{1.101425in}{1.241195in}}%
\pgfpathlineto{\pgfqpoint{1.112439in}{1.239899in}}%
\pgfpathlineto{\pgfqpoint{1.123454in}{1.243882in}}%
\pgfpathlineto{\pgfqpoint{1.134468in}{1.230211in}}%
\pgfpathlineto{\pgfqpoint{1.145482in}{1.261392in}}%
\pgfpathlineto{\pgfqpoint{1.156496in}{1.272498in}}%
\pgfpathlineto{\pgfqpoint{1.167511in}{1.274134in}}%
\pgfpathlineto{\pgfqpoint{1.178525in}{1.260072in}}%
\pgfpathlineto{\pgfqpoint{1.189539in}{1.249097in}}%
\pgfpathlineto{\pgfqpoint{1.200553in}{1.241393in}}%
\pgfpathlineto{\pgfqpoint{1.211568in}{1.228482in}}%
\pgfpathlineto{\pgfqpoint{1.222582in}{1.220100in}}%
\pgfpathlineto{\pgfqpoint{1.233596in}{1.215963in}}%
\pgfpathlineto{\pgfqpoint{1.244610in}{1.242719in}}%
\pgfpathlineto{\pgfqpoint{1.277653in}{1.239306in}}%
\pgfpathlineto{\pgfqpoint{1.288667in}{1.232143in}}%
\pgfpathlineto{\pgfqpoint{1.299682in}{1.242869in}}%
\pgfpathlineto{\pgfqpoint{1.310696in}{1.248283in}}%
\pgfpathlineto{\pgfqpoint{1.321710in}{1.246943in}}%
\pgfpathlineto{\pgfqpoint{1.332725in}{1.241230in}}%
\pgfpathlineto{\pgfqpoint{1.343739in}{1.231654in}}%
\pgfpathlineto{\pgfqpoint{1.354753in}{1.213505in}}%
\pgfpathlineto{\pgfqpoint{1.365767in}{1.212938in}}%
\pgfpathlineto{\pgfqpoint{1.376782in}{1.223858in}}%
\pgfpathlineto{\pgfqpoint{1.387796in}{1.223712in}}%
\pgfpathlineto{\pgfqpoint{1.398810in}{1.225770in}}%
\pgfpathlineto{\pgfqpoint{1.409824in}{1.237272in}}%
\pgfpathlineto{\pgfqpoint{1.420839in}{1.241571in}}%
\pgfpathlineto{\pgfqpoint{1.431853in}{1.258475in}}%
\pgfpathlineto{\pgfqpoint{1.442867in}{1.263556in}}%
\pgfpathlineto{\pgfqpoint{1.453881in}{1.260463in}}%
\pgfpathlineto{\pgfqpoint{1.464896in}{1.281095in}}%
\pgfpathlineto{\pgfqpoint{1.475910in}{1.274339in}}%
\pgfpathlineto{\pgfqpoint{1.486924in}{1.255535in}}%
\pgfpathlineto{\pgfqpoint{1.497938in}{1.245375in}}%
\pgfpathlineto{\pgfqpoint{1.508953in}{1.240568in}}%
\pgfpathlineto{\pgfqpoint{1.519967in}{1.242853in}}%
\pgfpathlineto{\pgfqpoint{1.541995in}{1.261092in}}%
\pgfpathlineto{\pgfqpoint{1.553010in}{1.266760in}}%
\pgfpathlineto{\pgfqpoint{1.564024in}{1.275444in}}%
\pgfpathlineto{\pgfqpoint{1.586052in}{1.318276in}}%
\pgfpathlineto{\pgfqpoint{1.597067in}{1.329271in}}%
\pgfpathlineto{\pgfqpoint{1.608081in}{1.349046in}}%
\pgfpathlineto{\pgfqpoint{1.630109in}{1.363051in}}%
\pgfpathlineto{\pgfqpoint{1.641124in}{1.352289in}}%
\pgfpathlineto{\pgfqpoint{1.652138in}{1.358085in}}%
\pgfpathlineto{\pgfqpoint{1.663152in}{1.367490in}}%
\pgfpathlineto{\pgfqpoint{1.674166in}{1.374731in}}%
\pgfpathlineto{\pgfqpoint{1.685181in}{1.393619in}}%
\pgfpathlineto{\pgfqpoint{1.696195in}{1.391644in}}%
\pgfpathlineto{\pgfqpoint{1.707209in}{1.400861in}}%
\pgfpathlineto{\pgfqpoint{1.718223in}{1.392675in}}%
\pgfpathlineto{\pgfqpoint{1.729238in}{1.393394in}}%
\pgfpathlineto{\pgfqpoint{1.740252in}{1.371487in}}%
\pgfpathlineto{\pgfqpoint{1.751266in}{1.357841in}}%
\pgfpathlineto{\pgfqpoint{1.762280in}{1.361444in}}%
\pgfpathlineto{\pgfqpoint{1.773295in}{1.337624in}}%
\pgfpathlineto{\pgfqpoint{1.784309in}{1.350110in}}%
\pgfpathlineto{\pgfqpoint{1.795323in}{1.342049in}}%
\pgfpathlineto{\pgfqpoint{1.806337in}{1.338256in}}%
\pgfpathlineto{\pgfqpoint{1.817352in}{1.337919in}}%
\pgfpathlineto{\pgfqpoint{1.828366in}{1.345904in}}%
\pgfpathlineto{\pgfqpoint{1.839380in}{1.344905in}}%
\pgfpathlineto{\pgfqpoint{1.850394in}{1.314440in}}%
\pgfpathlineto{\pgfqpoint{1.861409in}{1.311504in}}%
\pgfpathlineto{\pgfqpoint{1.872423in}{1.314325in}}%
\pgfpathlineto{\pgfqpoint{1.883437in}{1.327593in}}%
\pgfpathlineto{\pgfqpoint{1.905466in}{1.345144in}}%
\pgfpathlineto{\pgfqpoint{1.916480in}{1.338787in}}%
\pgfpathlineto{\pgfqpoint{1.927494in}{1.322714in}}%
\pgfpathlineto{\pgfqpoint{1.938508in}{1.311732in}}%
\pgfpathlineto{\pgfqpoint{1.960537in}{1.317184in}}%
\pgfpathlineto{\pgfqpoint{1.971551in}{1.327434in}}%
\pgfpathlineto{\pgfqpoint{1.982565in}{1.331130in}}%
\pgfpathlineto{\pgfqpoint{1.993580in}{1.329261in}}%
\pgfpathlineto{\pgfqpoint{2.004594in}{1.322174in}}%
\pgfpathlineto{\pgfqpoint{2.015608in}{1.321578in}}%
\pgfpathlineto{\pgfqpoint{2.026622in}{1.305322in}}%
\pgfpathlineto{\pgfqpoint{2.037637in}{1.309964in}}%
\pgfpathlineto{\pgfqpoint{2.048651in}{1.327255in}}%
\pgfpathlineto{\pgfqpoint{2.059665in}{1.326654in}}%
\pgfpathlineto{\pgfqpoint{2.070679in}{1.344016in}}%
\pgfpathlineto{\pgfqpoint{2.081694in}{1.315657in}}%
\pgfpathlineto{\pgfqpoint{2.092708in}{1.318974in}}%
\pgfpathlineto{\pgfqpoint{2.103722in}{1.316681in}}%
\pgfpathlineto{\pgfqpoint{2.114736in}{1.299109in}}%
\pgfpathlineto{\pgfqpoint{2.125751in}{1.287019in}}%
\pgfpathlineto{\pgfqpoint{2.136765in}{1.292026in}}%
\pgfpathlineto{\pgfqpoint{2.147779in}{1.300227in}}%
\pgfpathlineto{\pgfqpoint{2.158793in}{1.292035in}}%
\pgfpathlineto{\pgfqpoint{2.169808in}{1.285781in}}%
\pgfpathlineto{\pgfqpoint{2.180822in}{1.287248in}}%
\pgfpathlineto{\pgfqpoint{2.191836in}{1.280499in}}%
\pgfpathlineto{\pgfqpoint{2.202850in}{1.279372in}}%
\pgfpathlineto{\pgfqpoint{2.213865in}{1.276146in}}%
\pgfpathlineto{\pgfqpoint{2.224879in}{1.284832in}}%
\pgfpathlineto{\pgfqpoint{2.246907in}{1.290191in}}%
\pgfpathlineto{\pgfqpoint{2.257922in}{1.300361in}}%
\pgfpathlineto{\pgfqpoint{2.268936in}{1.301418in}}%
\pgfpathlineto{\pgfqpoint{2.279950in}{1.313851in}}%
\pgfpathlineto{\pgfqpoint{2.290964in}{1.344421in}}%
\pgfpathlineto{\pgfqpoint{2.301979in}{1.362733in}}%
\pgfpathlineto{\pgfqpoint{2.312993in}{1.362845in}}%
\pgfpathlineto{\pgfqpoint{2.324007in}{1.369565in}}%
\pgfpathlineto{\pgfqpoint{2.335021in}{1.368060in}}%
\pgfpathlineto{\pgfqpoint{2.357050in}{1.371936in}}%
\pgfpathlineto{\pgfqpoint{2.368064in}{1.381966in}}%
\pgfpathlineto{\pgfqpoint{2.390093in}{1.383030in}}%
\pgfpathlineto{\pgfqpoint{2.412121in}{1.423599in}}%
\pgfpathlineto{\pgfqpoint{2.445164in}{1.432013in}}%
\pgfpathlineto{\pgfqpoint{2.456178in}{1.437175in}}%
\pgfpathlineto{\pgfqpoint{2.467192in}{1.450398in}}%
\pgfpathlineto{\pgfqpoint{2.478207in}{1.454325in}}%
\pgfpathlineto{\pgfqpoint{2.489221in}{1.448921in}}%
\pgfpathlineto{\pgfqpoint{2.500235in}{1.451855in}}%
\pgfpathlineto{\pgfqpoint{2.511249in}{1.452052in}}%
\pgfpathlineto{\pgfqpoint{2.522264in}{1.456057in}}%
\pgfpathlineto{\pgfqpoint{2.533278in}{1.464289in}}%
\pgfpathlineto{\pgfqpoint{2.544292in}{1.467760in}}%
\pgfpathlineto{\pgfqpoint{2.555306in}{1.473209in}}%
\pgfpathlineto{\pgfqpoint{2.566321in}{1.460243in}}%
\pgfpathlineto{\pgfqpoint{2.577335in}{1.451770in}}%
\pgfpathlineto{\pgfqpoint{2.599363in}{1.448726in}}%
\pgfpathlineto{\pgfqpoint{2.610378in}{1.445120in}}%
\pgfpathlineto{\pgfqpoint{2.621392in}{1.422002in}}%
\pgfpathlineto{\pgfqpoint{2.632406in}{1.420863in}}%
\pgfpathlineto{\pgfqpoint{2.643421in}{1.424266in}}%
\pgfpathlineto{\pgfqpoint{2.654435in}{1.438222in}}%
\pgfpathlineto{\pgfqpoint{2.665449in}{1.438737in}}%
\pgfpathlineto{\pgfqpoint{2.676463in}{1.443048in}}%
\pgfpathlineto{\pgfqpoint{2.687478in}{1.423294in}}%
\pgfpathlineto{\pgfqpoint{2.698492in}{1.428917in}}%
\pgfpathlineto{\pgfqpoint{2.709506in}{1.424127in}}%
\pgfpathlineto{\pgfqpoint{2.720520in}{1.430991in}}%
\pgfpathlineto{\pgfqpoint{2.742549in}{1.438759in}}%
\pgfpathlineto{\pgfqpoint{2.753563in}{1.450377in}}%
\pgfpathlineto{\pgfqpoint{2.764577in}{1.435428in}}%
\pgfpathlineto{\pgfqpoint{2.775592in}{1.445919in}}%
\pgfpathlineto{\pgfqpoint{2.786606in}{1.446103in}}%
\pgfpathlineto{\pgfqpoint{2.797620in}{1.434957in}}%
\pgfpathlineto{\pgfqpoint{2.808634in}{1.431857in}}%
\pgfpathlineto{\pgfqpoint{2.819649in}{1.439010in}}%
\pgfpathlineto{\pgfqpoint{2.830663in}{1.450256in}}%
\pgfpathlineto{\pgfqpoint{2.841677in}{1.455631in}}%
\pgfpathlineto{\pgfqpoint{2.852691in}{1.436596in}}%
\pgfpathlineto{\pgfqpoint{2.863706in}{1.445046in}}%
\pgfpathlineto{\pgfqpoint{2.874720in}{1.436075in}}%
\pgfpathlineto{\pgfqpoint{2.885734in}{1.448002in}}%
\pgfpathlineto{\pgfqpoint{2.896748in}{1.444169in}}%
\pgfpathlineto{\pgfqpoint{2.907763in}{1.454264in}}%
\pgfpathlineto{\pgfqpoint{2.918777in}{1.447845in}}%
\pgfpathlineto{\pgfqpoint{2.929791in}{1.447442in}}%
\pgfpathlineto{\pgfqpoint{2.940805in}{1.455520in}}%
\pgfpathlineto{\pgfqpoint{2.951820in}{1.450978in}}%
\pgfpathlineto{\pgfqpoint{2.962834in}{1.448299in}}%
\pgfpathlineto{\pgfqpoint{2.973848in}{1.441248in}}%
\pgfpathlineto{\pgfqpoint{2.984862in}{1.442600in}}%
\pgfpathlineto{\pgfqpoint{3.017905in}{1.442482in}}%
\pgfpathlineto{\pgfqpoint{3.028919in}{1.438569in}}%
\pgfpathlineto{\pgfqpoint{3.039934in}{1.428736in}}%
\pgfpathlineto{\pgfqpoint{3.050948in}{1.413463in}}%
\pgfpathlineto{\pgfqpoint{3.061962in}{1.407330in}}%
\pgfpathlineto{\pgfqpoint{3.072976in}{1.397400in}}%
\pgfpathlineto{\pgfqpoint{3.083991in}{1.389119in}}%
\pgfpathlineto{\pgfqpoint{3.095005in}{1.394916in}}%
\pgfpathlineto{\pgfqpoint{3.106019in}{1.381345in}}%
\pgfpathlineto{\pgfqpoint{3.117033in}{1.372127in}}%
\pgfpathlineto{\pgfqpoint{3.128048in}{1.375392in}}%
\pgfpathlineto{\pgfqpoint{3.139062in}{1.372619in}}%
\pgfpathlineto{\pgfqpoint{3.150076in}{1.368357in}}%
\pgfpathlineto{\pgfqpoint{3.161090in}{1.365397in}}%
\pgfpathlineto{\pgfqpoint{3.172105in}{1.357076in}}%
\pgfpathlineto{\pgfqpoint{3.194133in}{1.320078in}}%
\pgfpathlineto{\pgfqpoint{3.205147in}{1.321998in}}%
\pgfpathlineto{\pgfqpoint{3.216162in}{1.311750in}}%
\pgfpathlineto{\pgfqpoint{3.227176in}{1.316968in}}%
\pgfpathlineto{\pgfqpoint{3.238190in}{1.310174in}}%
\pgfpathlineto{\pgfqpoint{3.249204in}{1.293329in}}%
\pgfpathlineto{\pgfqpoint{3.260219in}{1.282325in}}%
\pgfpathlineto{\pgfqpoint{3.271233in}{1.281543in}}%
\pgfpathlineto{\pgfqpoint{3.282247in}{1.293320in}}%
\pgfpathlineto{\pgfqpoint{3.293261in}{1.294902in}}%
\pgfpathlineto{\pgfqpoint{3.304276in}{1.282827in}}%
\pgfpathlineto{\pgfqpoint{3.315290in}{1.285684in}}%
\pgfpathlineto{\pgfqpoint{3.326304in}{1.296725in}}%
\pgfpathlineto{\pgfqpoint{3.337318in}{1.298395in}}%
\pgfpathlineto{\pgfqpoint{3.348333in}{1.286264in}}%
\pgfpathlineto{\pgfqpoint{3.359347in}{1.288457in}}%
\pgfpathlineto{\pgfqpoint{3.370361in}{1.289329in}}%
\pgfpathlineto{\pgfqpoint{3.381375in}{1.288419in}}%
\pgfpathlineto{\pgfqpoint{3.403404in}{1.312001in}}%
\pgfpathlineto{\pgfqpoint{3.414418in}{1.310039in}}%
\pgfpathlineto{\pgfqpoint{3.425432in}{1.296747in}}%
\pgfpathlineto{\pgfqpoint{3.436447in}{1.294943in}}%
\pgfpathlineto{\pgfqpoint{3.447461in}{1.300574in}}%
\pgfpathlineto{\pgfqpoint{3.458475in}{1.318410in}}%
\pgfpathlineto{\pgfqpoint{3.469489in}{1.313863in}}%
\pgfpathlineto{\pgfqpoint{3.480504in}{1.308103in}}%
\pgfpathlineto{\pgfqpoint{3.491518in}{1.303656in}}%
\pgfpathlineto{\pgfqpoint{3.502532in}{1.314950in}}%
\pgfpathlineto{\pgfqpoint{3.513546in}{1.320676in}}%
\pgfpathlineto{\pgfqpoint{3.524561in}{1.317574in}}%
\pgfpathlineto{\pgfqpoint{3.535575in}{1.321328in}}%
\pgfpathlineto{\pgfqpoint{3.546589in}{1.316863in}}%
\pgfpathlineto{\pgfqpoint{3.557603in}{1.310123in}}%
\pgfpathlineto{\pgfqpoint{3.568618in}{1.324556in}}%
\pgfpathlineto{\pgfqpoint{3.579632in}{1.325643in}}%
\pgfpathlineto{\pgfqpoint{3.590646in}{1.322336in}}%
\pgfpathlineto{\pgfqpoint{3.601660in}{1.320847in}}%
\pgfpathlineto{\pgfqpoint{3.612675in}{1.316274in}}%
\pgfpathlineto{\pgfqpoint{3.623689in}{1.331571in}}%
\pgfpathlineto{\pgfqpoint{3.634703in}{1.324786in}}%
\pgfpathlineto{\pgfqpoint{3.645717in}{1.321228in}}%
\pgfpathlineto{\pgfqpoint{3.667746in}{1.317374in}}%
\pgfpathlineto{\pgfqpoint{3.678760in}{1.329024in}}%
\pgfpathlineto{\pgfqpoint{3.678760in}{1.329024in}}%
\pgfusepath{stroke}%
\end{pgfscope}%
\begin{pgfscope}%
\pgfpathrectangle{\pgfqpoint{0.550713in}{0.408431in}}{\pgfqpoint{3.139062in}{1.490773in}}%
\pgfusepath{clip}%
\pgfsetbuttcap%
\pgfsetroundjoin%
\pgfsetlinewidth{0.853187pt}%
\definecolor{currentstroke}{rgb}{0.341176,0.670588,0.152941}%
\pgfsetstrokecolor{currentstroke}%
\pgfsetdash{{0.850000pt}{1.402500pt}}{0.000000pt}%
\pgfpathmoveto{\pgfqpoint{0.550713in}{0.976599in}}%
\pgfpathlineto{\pgfqpoint{0.561727in}{1.015420in}}%
\pgfpathlineto{\pgfqpoint{0.572741in}{1.003202in}}%
\pgfpathlineto{\pgfqpoint{0.583755in}{1.002315in}}%
\pgfpathlineto{\pgfqpoint{0.594770in}{1.006642in}}%
\pgfpathlineto{\pgfqpoint{0.616798in}{1.029106in}}%
\pgfpathlineto{\pgfqpoint{0.627812in}{1.038560in}}%
\pgfpathlineto{\pgfqpoint{0.638827in}{1.054461in}}%
\pgfpathlineto{\pgfqpoint{0.649841in}{1.056864in}}%
\pgfpathlineto{\pgfqpoint{0.660855in}{1.064150in}}%
\pgfpathlineto{\pgfqpoint{0.671869in}{1.066967in}}%
\pgfpathlineto{\pgfqpoint{0.682884in}{1.073164in}}%
\pgfpathlineto{\pgfqpoint{0.693898in}{1.084350in}}%
\pgfpathlineto{\pgfqpoint{0.704912in}{1.088272in}}%
\pgfpathlineto{\pgfqpoint{0.715926in}{1.096996in}}%
\pgfpathlineto{\pgfqpoint{0.737955in}{1.099566in}}%
\pgfpathlineto{\pgfqpoint{0.748969in}{1.102755in}}%
\pgfpathlineto{\pgfqpoint{0.759983in}{1.091767in}}%
\pgfpathlineto{\pgfqpoint{0.770998in}{1.108259in}}%
\pgfpathlineto{\pgfqpoint{0.782012in}{1.117080in}}%
\pgfpathlineto{\pgfqpoint{0.793026in}{1.123995in}}%
\pgfpathlineto{\pgfqpoint{0.804040in}{1.126367in}}%
\pgfpathlineto{\pgfqpoint{0.815055in}{1.120509in}}%
\pgfpathlineto{\pgfqpoint{0.826069in}{1.121256in}}%
\pgfpathlineto{\pgfqpoint{0.837083in}{1.117535in}}%
\pgfpathlineto{\pgfqpoint{0.848097in}{1.115206in}}%
\pgfpathlineto{\pgfqpoint{0.859112in}{1.124476in}}%
\pgfpathlineto{\pgfqpoint{0.870126in}{1.141182in}}%
\pgfpathlineto{\pgfqpoint{0.892154in}{1.147856in}}%
\pgfpathlineto{\pgfqpoint{0.903169in}{1.155073in}}%
\pgfpathlineto{\pgfqpoint{0.914183in}{1.160707in}}%
\pgfpathlineto{\pgfqpoint{0.925197in}{1.147048in}}%
\pgfpathlineto{\pgfqpoint{0.936211in}{1.145695in}}%
\pgfpathlineto{\pgfqpoint{0.947226in}{1.149666in}}%
\pgfpathlineto{\pgfqpoint{0.969254in}{1.155789in}}%
\pgfpathlineto{\pgfqpoint{0.980268in}{1.153698in}}%
\pgfpathlineto{\pgfqpoint{0.991283in}{1.163949in}}%
\pgfpathlineto{\pgfqpoint{1.002297in}{1.160214in}}%
\pgfpathlineto{\pgfqpoint{1.013311in}{1.163084in}}%
\pgfpathlineto{\pgfqpoint{1.024325in}{1.161050in}}%
\pgfpathlineto{\pgfqpoint{1.035340in}{1.166843in}}%
\pgfpathlineto{\pgfqpoint{1.046354in}{1.167833in}}%
\pgfpathlineto{\pgfqpoint{1.068382in}{1.175108in}}%
\pgfpathlineto{\pgfqpoint{1.079397in}{1.178421in}}%
\pgfpathlineto{\pgfqpoint{1.090411in}{1.186361in}}%
\pgfpathlineto{\pgfqpoint{1.112439in}{1.192820in}}%
\pgfpathlineto{\pgfqpoint{1.123454in}{1.193037in}}%
\pgfpathlineto{\pgfqpoint{1.134468in}{1.188006in}}%
\pgfpathlineto{\pgfqpoint{1.145482in}{1.186519in}}%
\pgfpathlineto{\pgfqpoint{1.167511in}{1.201049in}}%
\pgfpathlineto{\pgfqpoint{1.178525in}{1.202374in}}%
\pgfpathlineto{\pgfqpoint{1.189539in}{1.206229in}}%
\pgfpathlineto{\pgfqpoint{1.200553in}{1.208656in}}%
\pgfpathlineto{\pgfqpoint{1.211568in}{1.202516in}}%
\pgfpathlineto{\pgfqpoint{1.222582in}{1.211463in}}%
\pgfpathlineto{\pgfqpoint{1.244610in}{1.215608in}}%
\pgfpathlineto{\pgfqpoint{1.255625in}{1.215154in}}%
\pgfpathlineto{\pgfqpoint{1.277653in}{1.218223in}}%
\pgfpathlineto{\pgfqpoint{1.288667in}{1.221468in}}%
\pgfpathlineto{\pgfqpoint{1.299682in}{1.222361in}}%
\pgfpathlineto{\pgfqpoint{1.310696in}{1.220165in}}%
\pgfpathlineto{\pgfqpoint{1.321710in}{1.227203in}}%
\pgfpathlineto{\pgfqpoint{1.332725in}{1.223471in}}%
\pgfpathlineto{\pgfqpoint{1.343739in}{1.225051in}}%
\pgfpathlineto{\pgfqpoint{1.354753in}{1.220667in}}%
\pgfpathlineto{\pgfqpoint{1.365767in}{1.226720in}}%
\pgfpathlineto{\pgfqpoint{1.376782in}{1.227480in}}%
\pgfpathlineto{\pgfqpoint{1.398810in}{1.236780in}}%
\pgfpathlineto{\pgfqpoint{1.409824in}{1.242708in}}%
\pgfpathlineto{\pgfqpoint{1.431853in}{1.251975in}}%
\pgfpathlineto{\pgfqpoint{1.442867in}{1.259013in}}%
\pgfpathlineto{\pgfqpoint{1.453881in}{1.257465in}}%
\pgfpathlineto{\pgfqpoint{1.464896in}{1.251415in}}%
\pgfpathlineto{\pgfqpoint{1.486924in}{1.259053in}}%
\pgfpathlineto{\pgfqpoint{1.497938in}{1.260309in}}%
\pgfpathlineto{\pgfqpoint{1.508953in}{1.267705in}}%
\pgfpathlineto{\pgfqpoint{1.519967in}{1.270173in}}%
\pgfpathlineto{\pgfqpoint{1.530981in}{1.276779in}}%
\pgfpathlineto{\pgfqpoint{1.541995in}{1.278306in}}%
\pgfpathlineto{\pgfqpoint{1.553010in}{1.268990in}}%
\pgfpathlineto{\pgfqpoint{1.575038in}{1.288066in}}%
\pgfpathlineto{\pgfqpoint{1.597067in}{1.293867in}}%
\pgfpathlineto{\pgfqpoint{1.608081in}{1.290068in}}%
\pgfpathlineto{\pgfqpoint{1.619095in}{1.294996in}}%
\pgfpathlineto{\pgfqpoint{1.641124in}{1.302274in}}%
\pgfpathlineto{\pgfqpoint{1.652138in}{1.289359in}}%
\pgfpathlineto{\pgfqpoint{1.663152in}{1.288395in}}%
\pgfpathlineto{\pgfqpoint{1.685181in}{1.290547in}}%
\pgfpathlineto{\pgfqpoint{1.696195in}{1.286706in}}%
\pgfpathlineto{\pgfqpoint{1.707209in}{1.289210in}}%
\pgfpathlineto{\pgfqpoint{1.718223in}{1.294312in}}%
\pgfpathlineto{\pgfqpoint{1.729238in}{1.303113in}}%
\pgfpathlineto{\pgfqpoint{1.740252in}{1.307033in}}%
\pgfpathlineto{\pgfqpoint{1.751266in}{1.305201in}}%
\pgfpathlineto{\pgfqpoint{1.762280in}{1.285723in}}%
\pgfpathlineto{\pgfqpoint{1.773295in}{1.270186in}}%
\pgfpathlineto{\pgfqpoint{1.784309in}{1.268139in}}%
\pgfpathlineto{\pgfqpoint{1.795323in}{1.280839in}}%
\pgfpathlineto{\pgfqpoint{1.817352in}{1.297239in}}%
\pgfpathlineto{\pgfqpoint{1.828366in}{1.298711in}}%
\pgfpathlineto{\pgfqpoint{1.839380in}{1.305142in}}%
\pgfpathlineto{\pgfqpoint{1.850394in}{1.302727in}}%
\pgfpathlineto{\pgfqpoint{1.861409in}{1.312197in}}%
\pgfpathlineto{\pgfqpoint{1.872423in}{1.317081in}}%
\pgfpathlineto{\pgfqpoint{1.894451in}{1.323351in}}%
\pgfpathlineto{\pgfqpoint{1.905466in}{1.322595in}}%
\pgfpathlineto{\pgfqpoint{1.916480in}{1.320618in}}%
\pgfpathlineto{\pgfqpoint{1.927494in}{1.308061in}}%
\pgfpathlineto{\pgfqpoint{1.938508in}{1.309893in}}%
\pgfpathlineto{\pgfqpoint{1.949523in}{1.306237in}}%
\pgfpathlineto{\pgfqpoint{1.960537in}{1.305551in}}%
\pgfpathlineto{\pgfqpoint{1.971551in}{1.310236in}}%
\pgfpathlineto{\pgfqpoint{1.982565in}{1.300178in}}%
\pgfpathlineto{\pgfqpoint{1.993580in}{1.306636in}}%
\pgfpathlineto{\pgfqpoint{2.004594in}{1.314955in}}%
\pgfpathlineto{\pgfqpoint{2.015608in}{1.304483in}}%
\pgfpathlineto{\pgfqpoint{2.026622in}{1.308498in}}%
\pgfpathlineto{\pgfqpoint{2.048651in}{1.313256in}}%
\pgfpathlineto{\pgfqpoint{2.059665in}{1.311210in}}%
\pgfpathlineto{\pgfqpoint{2.070679in}{1.317659in}}%
\pgfpathlineto{\pgfqpoint{2.081694in}{1.321703in}}%
\pgfpathlineto{\pgfqpoint{2.125751in}{1.322079in}}%
\pgfpathlineto{\pgfqpoint{2.136765in}{1.325307in}}%
\pgfpathlineto{\pgfqpoint{2.147779in}{1.325069in}}%
\pgfpathlineto{\pgfqpoint{2.158793in}{1.330842in}}%
\pgfpathlineto{\pgfqpoint{2.169808in}{1.332562in}}%
\pgfpathlineto{\pgfqpoint{2.180822in}{1.340996in}}%
\pgfpathlineto{\pgfqpoint{2.191836in}{1.341679in}}%
\pgfpathlineto{\pgfqpoint{2.202850in}{1.337059in}}%
\pgfpathlineto{\pgfqpoint{2.213865in}{1.340908in}}%
\pgfpathlineto{\pgfqpoint{2.224879in}{1.342179in}}%
\pgfpathlineto{\pgfqpoint{2.246907in}{1.341669in}}%
\pgfpathlineto{\pgfqpoint{2.257922in}{1.348115in}}%
\pgfpathlineto{\pgfqpoint{2.279950in}{1.354775in}}%
\pgfpathlineto{\pgfqpoint{2.301979in}{1.363449in}}%
\pgfpathlineto{\pgfqpoint{2.312993in}{1.363832in}}%
\pgfpathlineto{\pgfqpoint{2.324007in}{1.369387in}}%
\pgfpathlineto{\pgfqpoint{2.335021in}{1.369853in}}%
\pgfpathlineto{\pgfqpoint{2.357050in}{1.380232in}}%
\pgfpathlineto{\pgfqpoint{2.379078in}{1.381674in}}%
\pgfpathlineto{\pgfqpoint{2.390093in}{1.385717in}}%
\pgfpathlineto{\pgfqpoint{2.401107in}{1.387295in}}%
\pgfpathlineto{\pgfqpoint{2.412121in}{1.382113in}}%
\pgfpathlineto{\pgfqpoint{2.423135in}{1.385755in}}%
\pgfpathlineto{\pgfqpoint{2.434150in}{1.386345in}}%
\pgfpathlineto{\pgfqpoint{2.445164in}{1.393064in}}%
\pgfpathlineto{\pgfqpoint{2.456178in}{1.392848in}}%
\pgfpathlineto{\pgfqpoint{2.467192in}{1.396201in}}%
\pgfpathlineto{\pgfqpoint{2.478207in}{1.401872in}}%
\pgfpathlineto{\pgfqpoint{2.489221in}{1.402285in}}%
\pgfpathlineto{\pgfqpoint{2.500235in}{1.398974in}}%
\pgfpathlineto{\pgfqpoint{2.522264in}{1.401111in}}%
\pgfpathlineto{\pgfqpoint{2.544292in}{1.407905in}}%
\pgfpathlineto{\pgfqpoint{2.555306in}{1.407889in}}%
\pgfpathlineto{\pgfqpoint{2.566321in}{1.411532in}}%
\pgfpathlineto{\pgfqpoint{2.588349in}{1.411992in}}%
\pgfpathlineto{\pgfqpoint{2.610378in}{1.415109in}}%
\pgfpathlineto{\pgfqpoint{2.621392in}{1.420834in}}%
\pgfpathlineto{\pgfqpoint{2.632406in}{1.419457in}}%
\pgfpathlineto{\pgfqpoint{2.665449in}{1.423129in}}%
\pgfpathlineto{\pgfqpoint{2.676463in}{1.420612in}}%
\pgfpathlineto{\pgfqpoint{2.687478in}{1.424105in}}%
\pgfpathlineto{\pgfqpoint{2.709506in}{1.427500in}}%
\pgfpathlineto{\pgfqpoint{2.720520in}{1.433206in}}%
\pgfpathlineto{\pgfqpoint{2.742549in}{1.436334in}}%
\pgfpathlineto{\pgfqpoint{2.764577in}{1.435642in}}%
\pgfpathlineto{\pgfqpoint{2.775592in}{1.438272in}}%
\pgfpathlineto{\pgfqpoint{2.797620in}{1.438899in}}%
\pgfpathlineto{\pgfqpoint{2.808634in}{1.441701in}}%
\pgfpathlineto{\pgfqpoint{2.819649in}{1.440888in}}%
\pgfpathlineto{\pgfqpoint{2.841677in}{1.425896in}}%
\pgfpathlineto{\pgfqpoint{2.863706in}{1.427808in}}%
\pgfpathlineto{\pgfqpoint{2.874720in}{1.431359in}}%
\pgfpathlineto{\pgfqpoint{2.885734in}{1.433494in}}%
\pgfpathlineto{\pgfqpoint{2.907763in}{1.434466in}}%
\pgfpathlineto{\pgfqpoint{2.929791in}{1.437693in}}%
\pgfpathlineto{\pgfqpoint{2.940805in}{1.440726in}}%
\pgfpathlineto{\pgfqpoint{2.984862in}{1.443996in}}%
\pgfpathlineto{\pgfqpoint{3.006891in}{1.450431in}}%
\pgfpathlineto{\pgfqpoint{3.017905in}{1.451716in}}%
\pgfpathlineto{\pgfqpoint{3.028919in}{1.442726in}}%
\pgfpathlineto{\pgfqpoint{3.039934in}{1.445794in}}%
\pgfpathlineto{\pgfqpoint{3.050948in}{1.444291in}}%
\pgfpathlineto{\pgfqpoint{3.072976in}{1.444067in}}%
\pgfpathlineto{\pgfqpoint{3.095005in}{1.447463in}}%
\pgfpathlineto{\pgfqpoint{3.128048in}{1.448892in}}%
\pgfpathlineto{\pgfqpoint{3.139062in}{1.450852in}}%
\pgfpathlineto{\pgfqpoint{3.161090in}{1.450725in}}%
\pgfpathlineto{\pgfqpoint{3.172105in}{1.453202in}}%
\pgfpathlineto{\pgfqpoint{3.183119in}{1.453377in}}%
\pgfpathlineto{\pgfqpoint{3.194133in}{1.451800in}}%
\pgfpathlineto{\pgfqpoint{3.205147in}{1.458068in}}%
\pgfpathlineto{\pgfqpoint{3.227176in}{1.462630in}}%
\pgfpathlineto{\pgfqpoint{3.260219in}{1.459813in}}%
\pgfpathlineto{\pgfqpoint{3.271233in}{1.462353in}}%
\pgfpathlineto{\pgfqpoint{3.282247in}{1.462234in}}%
\pgfpathlineto{\pgfqpoint{3.293261in}{1.468938in}}%
\pgfpathlineto{\pgfqpoint{3.304276in}{1.471932in}}%
\pgfpathlineto{\pgfqpoint{3.315290in}{1.472095in}}%
\pgfpathlineto{\pgfqpoint{3.326304in}{1.473673in}}%
\pgfpathlineto{\pgfqpoint{3.337318in}{1.473372in}}%
\pgfpathlineto{\pgfqpoint{3.348333in}{1.478815in}}%
\pgfpathlineto{\pgfqpoint{3.359347in}{1.477873in}}%
\pgfpathlineto{\pgfqpoint{3.370361in}{1.480149in}}%
\pgfpathlineto{\pgfqpoint{3.381375in}{1.479236in}}%
\pgfpathlineto{\pgfqpoint{3.392390in}{1.480647in}}%
\pgfpathlineto{\pgfqpoint{3.403404in}{1.475988in}}%
\pgfpathlineto{\pgfqpoint{3.414418in}{1.472843in}}%
\pgfpathlineto{\pgfqpoint{3.425432in}{1.475819in}}%
\pgfpathlineto{\pgfqpoint{3.436447in}{1.481383in}}%
\pgfpathlineto{\pgfqpoint{3.469489in}{1.482598in}}%
\pgfpathlineto{\pgfqpoint{3.524561in}{1.491056in}}%
\pgfpathlineto{\pgfqpoint{3.535575in}{1.489414in}}%
\pgfpathlineto{\pgfqpoint{3.557603in}{1.492106in}}%
\pgfpathlineto{\pgfqpoint{3.568618in}{1.494836in}}%
\pgfpathlineto{\pgfqpoint{3.590646in}{1.495312in}}%
\pgfpathlineto{\pgfqpoint{3.601660in}{1.498710in}}%
\pgfpathlineto{\pgfqpoint{3.612675in}{1.496447in}}%
\pgfpathlineto{\pgfqpoint{3.634703in}{1.502522in}}%
\pgfpathlineto{\pgfqpoint{3.656732in}{1.501392in}}%
\pgfpathlineto{\pgfqpoint{3.667746in}{1.497090in}}%
\pgfpathlineto{\pgfqpoint{3.678760in}{1.501632in}}%
\pgfpathlineto{\pgfqpoint{3.678760in}{1.501632in}}%
\pgfusepath{stroke}%
\end{pgfscope}%
\begin{pgfscope}%
\pgfpathrectangle{\pgfqpoint{0.550713in}{0.408431in}}{\pgfqpoint{3.139062in}{1.490773in}}%
\pgfusepath{clip}%
\pgfsetbuttcap%
\pgfsetroundjoin%
\pgfsetlinewidth{0.853187pt}%
\definecolor{currentstroke}{rgb}{0.000000,0.380392,0.396078}%
\pgfsetstrokecolor{currentstroke}%
\pgfsetdash{{0.850000pt}{1.402500pt}}{0.000000pt}%
\pgfpathmoveto{\pgfqpoint{0.550713in}{0.992222in}}%
\pgfpathlineto{\pgfqpoint{0.561727in}{1.016608in}}%
\pgfpathlineto{\pgfqpoint{0.572741in}{1.028802in}}%
\pgfpathlineto{\pgfqpoint{0.583755in}{1.011340in}}%
\pgfpathlineto{\pgfqpoint{0.594770in}{1.016659in}}%
\pgfpathlineto{\pgfqpoint{0.605784in}{1.028578in}}%
\pgfpathlineto{\pgfqpoint{0.616798in}{1.046263in}}%
\pgfpathlineto{\pgfqpoint{0.627812in}{1.049439in}}%
\pgfpathlineto{\pgfqpoint{0.638827in}{1.048781in}}%
\pgfpathlineto{\pgfqpoint{0.649841in}{1.059203in}}%
\pgfpathlineto{\pgfqpoint{0.660855in}{1.046976in}}%
\pgfpathlineto{\pgfqpoint{0.671869in}{1.015823in}}%
\pgfpathlineto{\pgfqpoint{0.682884in}{1.016323in}}%
\pgfpathlineto{\pgfqpoint{0.693898in}{0.998964in}}%
\pgfpathlineto{\pgfqpoint{0.704912in}{1.002424in}}%
\pgfpathlineto{\pgfqpoint{0.715926in}{0.990143in}}%
\pgfpathlineto{\pgfqpoint{0.726941in}{0.997421in}}%
\pgfpathlineto{\pgfqpoint{0.748969in}{0.982955in}}%
\pgfpathlineto{\pgfqpoint{0.759983in}{0.964411in}}%
\pgfpathlineto{\pgfqpoint{0.770998in}{0.976494in}}%
\pgfpathlineto{\pgfqpoint{0.782012in}{0.991664in}}%
\pgfpathlineto{\pgfqpoint{0.793026in}{1.011704in}}%
\pgfpathlineto{\pgfqpoint{0.804040in}{1.019040in}}%
\pgfpathlineto{\pgfqpoint{0.815055in}{1.016433in}}%
\pgfpathlineto{\pgfqpoint{0.826069in}{1.020087in}}%
\pgfpathlineto{\pgfqpoint{0.837083in}{1.024938in}}%
\pgfpathlineto{\pgfqpoint{0.848097in}{1.016262in}}%
\pgfpathlineto{\pgfqpoint{0.859112in}{1.010972in}}%
\pgfpathlineto{\pgfqpoint{0.870126in}{1.054608in}}%
\pgfpathlineto{\pgfqpoint{0.881140in}{1.026665in}}%
\pgfpathlineto{\pgfqpoint{0.892154in}{1.027479in}}%
\pgfpathlineto{\pgfqpoint{0.903169in}{1.019843in}}%
\pgfpathlineto{\pgfqpoint{0.914183in}{1.003044in}}%
\pgfpathlineto{\pgfqpoint{0.936211in}{1.058014in}}%
\pgfpathlineto{\pgfqpoint{0.947226in}{1.074112in}}%
\pgfpathlineto{\pgfqpoint{0.958240in}{1.048535in}}%
\pgfpathlineto{\pgfqpoint{0.969254in}{1.040511in}}%
\pgfpathlineto{\pgfqpoint{0.980268in}{1.040932in}}%
\pgfpathlineto{\pgfqpoint{0.991283in}{1.040000in}}%
\pgfpathlineto{\pgfqpoint{1.002297in}{1.013149in}}%
\pgfpathlineto{\pgfqpoint{1.013311in}{1.002771in}}%
\pgfpathlineto{\pgfqpoint{1.024325in}{1.004049in}}%
\pgfpathlineto{\pgfqpoint{1.035340in}{1.013173in}}%
\pgfpathlineto{\pgfqpoint{1.046354in}{1.000854in}}%
\pgfpathlineto{\pgfqpoint{1.057368in}{1.021067in}}%
\pgfpathlineto{\pgfqpoint{1.068382in}{1.037515in}}%
\pgfpathlineto{\pgfqpoint{1.079397in}{1.042236in}}%
\pgfpathlineto{\pgfqpoint{1.090411in}{1.030817in}}%
\pgfpathlineto{\pgfqpoint{1.101425in}{1.026450in}}%
\pgfpathlineto{\pgfqpoint{1.112439in}{1.051510in}}%
\pgfpathlineto{\pgfqpoint{1.123454in}{1.063328in}}%
\pgfpathlineto{\pgfqpoint{1.134468in}{1.053912in}}%
\pgfpathlineto{\pgfqpoint{1.145482in}{1.046088in}}%
\pgfpathlineto{\pgfqpoint{1.156496in}{1.052118in}}%
\pgfpathlineto{\pgfqpoint{1.167511in}{1.064969in}}%
\pgfpathlineto{\pgfqpoint{1.178525in}{1.050473in}}%
\pgfpathlineto{\pgfqpoint{1.189539in}{1.042139in}}%
\pgfpathlineto{\pgfqpoint{1.200553in}{1.023213in}}%
\pgfpathlineto{\pgfqpoint{1.211568in}{1.025823in}}%
\pgfpathlineto{\pgfqpoint{1.222582in}{1.000482in}}%
\pgfpathlineto{\pgfqpoint{1.233596in}{1.021673in}}%
\pgfpathlineto{\pgfqpoint{1.244610in}{1.038124in}}%
\pgfpathlineto{\pgfqpoint{1.255625in}{1.044432in}}%
\pgfpathlineto{\pgfqpoint{1.266639in}{1.019444in}}%
\pgfpathlineto{\pgfqpoint{1.277653in}{1.029662in}}%
\pgfpathlineto{\pgfqpoint{1.288667in}{1.026071in}}%
\pgfpathlineto{\pgfqpoint{1.299682in}{1.046459in}}%
\pgfpathlineto{\pgfqpoint{1.310696in}{1.037295in}}%
\pgfpathlineto{\pgfqpoint{1.321710in}{1.034948in}}%
\pgfpathlineto{\pgfqpoint{1.343739in}{1.043405in}}%
\pgfpathlineto{\pgfqpoint{1.354753in}{1.060767in}}%
\pgfpathlineto{\pgfqpoint{1.365767in}{1.014577in}}%
\pgfpathlineto{\pgfqpoint{1.376782in}{1.046228in}}%
\pgfpathlineto{\pgfqpoint{1.387796in}{1.044019in}}%
\pgfpathlineto{\pgfqpoint{1.398810in}{1.040218in}}%
\pgfpathlineto{\pgfqpoint{1.409824in}{1.047566in}}%
\pgfpathlineto{\pgfqpoint{1.420839in}{1.042130in}}%
\pgfpathlineto{\pgfqpoint{1.431853in}{1.064287in}}%
\pgfpathlineto{\pgfqpoint{1.442867in}{1.054673in}}%
\pgfpathlineto{\pgfqpoint{1.453881in}{1.067479in}}%
\pgfpathlineto{\pgfqpoint{1.464896in}{1.043969in}}%
\pgfpathlineto{\pgfqpoint{1.475910in}{1.036707in}}%
\pgfpathlineto{\pgfqpoint{1.486924in}{1.066315in}}%
\pgfpathlineto{\pgfqpoint{1.497938in}{1.057693in}}%
\pgfpathlineto{\pgfqpoint{1.508953in}{1.062873in}}%
\pgfpathlineto{\pgfqpoint{1.519967in}{1.075579in}}%
\pgfpathlineto{\pgfqpoint{1.530981in}{1.081897in}}%
\pgfpathlineto{\pgfqpoint{1.553010in}{1.072400in}}%
\pgfpathlineto{\pgfqpoint{1.564024in}{1.062006in}}%
\pgfpathlineto{\pgfqpoint{1.575038in}{1.066804in}}%
\pgfpathlineto{\pgfqpoint{1.586052in}{1.066918in}}%
\pgfpathlineto{\pgfqpoint{1.608081in}{1.070257in}}%
\pgfpathlineto{\pgfqpoint{1.619095in}{1.069550in}}%
\pgfpathlineto{\pgfqpoint{1.630109in}{1.072835in}}%
\pgfpathlineto{\pgfqpoint{1.641124in}{1.072140in}}%
\pgfpathlineto{\pgfqpoint{1.652138in}{1.053099in}}%
\pgfpathlineto{\pgfqpoint{1.663152in}{1.082844in}}%
\pgfpathlineto{\pgfqpoint{1.674166in}{1.074360in}}%
\pgfpathlineto{\pgfqpoint{1.685181in}{1.063546in}}%
\pgfpathlineto{\pgfqpoint{1.696195in}{1.066852in}}%
\pgfpathlineto{\pgfqpoint{1.707209in}{1.048039in}}%
\pgfpathlineto{\pgfqpoint{1.718223in}{1.065964in}}%
\pgfpathlineto{\pgfqpoint{1.729238in}{1.060788in}}%
\pgfpathlineto{\pgfqpoint{1.751266in}{1.006110in}}%
\pgfpathlineto{\pgfqpoint{1.773295in}{1.001583in}}%
\pgfpathlineto{\pgfqpoint{1.784309in}{0.984680in}}%
\pgfpathlineto{\pgfqpoint{1.795323in}{1.013922in}}%
\pgfpathlineto{\pgfqpoint{1.806337in}{0.974932in}}%
\pgfpathlineto{\pgfqpoint{1.817352in}{1.002741in}}%
\pgfpathlineto{\pgfqpoint{1.828366in}{0.992804in}}%
\pgfpathlineto{\pgfqpoint{1.839380in}{0.987949in}}%
\pgfpathlineto{\pgfqpoint{1.850394in}{1.007204in}}%
\pgfpathlineto{\pgfqpoint{1.861409in}{1.004282in}}%
\pgfpathlineto{\pgfqpoint{1.872423in}{1.027579in}}%
\pgfpathlineto{\pgfqpoint{1.883437in}{1.055655in}}%
\pgfpathlineto{\pgfqpoint{1.894451in}{1.063717in}}%
\pgfpathlineto{\pgfqpoint{1.905466in}{1.092726in}}%
\pgfpathlineto{\pgfqpoint{1.916480in}{1.081811in}}%
\pgfpathlineto{\pgfqpoint{1.927494in}{1.084850in}}%
\pgfpathlineto{\pgfqpoint{1.938508in}{1.046099in}}%
\pgfpathlineto{\pgfqpoint{1.949523in}{1.084547in}}%
\pgfpathlineto{\pgfqpoint{1.960537in}{1.094333in}}%
\pgfpathlineto{\pgfqpoint{1.971551in}{1.117314in}}%
\pgfpathlineto{\pgfqpoint{1.993580in}{1.095406in}}%
\pgfpathlineto{\pgfqpoint{2.004594in}{1.078320in}}%
\pgfpathlineto{\pgfqpoint{2.015608in}{1.087162in}}%
\pgfpathlineto{\pgfqpoint{2.026622in}{1.107451in}}%
\pgfpathlineto{\pgfqpoint{2.037637in}{1.094152in}}%
\pgfpathlineto{\pgfqpoint{2.048651in}{1.113130in}}%
\pgfpathlineto{\pgfqpoint{2.059665in}{1.119585in}}%
\pgfpathlineto{\pgfqpoint{2.070679in}{1.116129in}}%
\pgfpathlineto{\pgfqpoint{2.081694in}{1.114232in}}%
\pgfpathlineto{\pgfqpoint{2.092708in}{1.118077in}}%
\pgfpathlineto{\pgfqpoint{2.103722in}{1.104583in}}%
\pgfpathlineto{\pgfqpoint{2.125751in}{1.092699in}}%
\pgfpathlineto{\pgfqpoint{2.136765in}{1.104147in}}%
\pgfpathlineto{\pgfqpoint{2.147779in}{1.096463in}}%
\pgfpathlineto{\pgfqpoint{2.158793in}{1.148344in}}%
\pgfpathlineto{\pgfqpoint{2.180822in}{1.073300in}}%
\pgfpathlineto{\pgfqpoint{2.191836in}{1.098108in}}%
\pgfpathlineto{\pgfqpoint{2.202850in}{1.077463in}}%
\pgfpathlineto{\pgfqpoint{2.213865in}{1.060110in}}%
\pgfpathlineto{\pgfqpoint{2.224879in}{1.074104in}}%
\pgfpathlineto{\pgfqpoint{2.246907in}{1.075518in}}%
\pgfpathlineto{\pgfqpoint{2.257922in}{1.060017in}}%
\pgfpathlineto{\pgfqpoint{2.268936in}{1.087694in}}%
\pgfpathlineto{\pgfqpoint{2.279950in}{1.102399in}}%
\pgfpathlineto{\pgfqpoint{2.290964in}{1.087277in}}%
\pgfpathlineto{\pgfqpoint{2.301979in}{1.082551in}}%
\pgfpathlineto{\pgfqpoint{2.312993in}{1.120957in}}%
\pgfpathlineto{\pgfqpoint{2.324007in}{1.106824in}}%
\pgfpathlineto{\pgfqpoint{2.335021in}{1.122918in}}%
\pgfpathlineto{\pgfqpoint{2.357050in}{1.150065in}}%
\pgfpathlineto{\pgfqpoint{2.368064in}{1.157014in}}%
\pgfpathlineto{\pgfqpoint{2.390093in}{1.135327in}}%
\pgfpathlineto{\pgfqpoint{2.401107in}{1.144043in}}%
\pgfpathlineto{\pgfqpoint{2.412121in}{1.137851in}}%
\pgfpathlineto{\pgfqpoint{2.423135in}{1.126657in}}%
\pgfpathlineto{\pgfqpoint{2.434150in}{1.131016in}}%
\pgfpathlineto{\pgfqpoint{2.445164in}{1.139150in}}%
\pgfpathlineto{\pgfqpoint{2.456178in}{1.121803in}}%
\pgfpathlineto{\pgfqpoint{2.467192in}{1.115762in}}%
\pgfpathlineto{\pgfqpoint{2.478207in}{1.103044in}}%
\pgfpathlineto{\pgfqpoint{2.489221in}{1.109531in}}%
\pgfpathlineto{\pgfqpoint{2.500235in}{1.087375in}}%
\pgfpathlineto{\pgfqpoint{2.511249in}{1.082159in}}%
\pgfpathlineto{\pgfqpoint{2.522264in}{1.075058in}}%
\pgfpathlineto{\pgfqpoint{2.533278in}{1.071200in}}%
\pgfpathlineto{\pgfqpoint{2.544292in}{1.079389in}}%
\pgfpathlineto{\pgfqpoint{2.555306in}{1.127055in}}%
\pgfpathlineto{\pgfqpoint{2.566321in}{1.141690in}}%
\pgfpathlineto{\pgfqpoint{2.577335in}{1.140983in}}%
\pgfpathlineto{\pgfqpoint{2.588349in}{1.155628in}}%
\pgfpathlineto{\pgfqpoint{2.599363in}{1.164166in}}%
\pgfpathlineto{\pgfqpoint{2.610378in}{1.178038in}}%
\pgfpathlineto{\pgfqpoint{2.632406in}{1.178700in}}%
\pgfpathlineto{\pgfqpoint{2.643421in}{1.145487in}}%
\pgfpathlineto{\pgfqpoint{2.654435in}{1.148253in}}%
\pgfpathlineto{\pgfqpoint{2.665449in}{1.157058in}}%
\pgfpathlineto{\pgfqpoint{2.676463in}{1.168458in}}%
\pgfpathlineto{\pgfqpoint{2.687478in}{1.163353in}}%
\pgfpathlineto{\pgfqpoint{2.698492in}{1.140828in}}%
\pgfpathlineto{\pgfqpoint{2.709506in}{1.134692in}}%
\pgfpathlineto{\pgfqpoint{2.720520in}{1.132011in}}%
\pgfpathlineto{\pgfqpoint{2.731535in}{1.145470in}}%
\pgfpathlineto{\pgfqpoint{2.742549in}{1.146732in}}%
\pgfpathlineto{\pgfqpoint{2.753563in}{1.181107in}}%
\pgfpathlineto{\pgfqpoint{2.764577in}{1.179328in}}%
\pgfpathlineto{\pgfqpoint{2.775592in}{1.194041in}}%
\pgfpathlineto{\pgfqpoint{2.786606in}{1.198932in}}%
\pgfpathlineto{\pgfqpoint{2.797620in}{1.210805in}}%
\pgfpathlineto{\pgfqpoint{2.808634in}{1.199117in}}%
\pgfpathlineto{\pgfqpoint{2.819649in}{1.175910in}}%
\pgfpathlineto{\pgfqpoint{2.830663in}{1.182103in}}%
\pgfpathlineto{\pgfqpoint{2.841677in}{1.194392in}}%
\pgfpathlineto{\pgfqpoint{2.852691in}{1.196253in}}%
\pgfpathlineto{\pgfqpoint{2.863706in}{1.212962in}}%
\pgfpathlineto{\pgfqpoint{2.874720in}{1.205612in}}%
\pgfpathlineto{\pgfqpoint{2.885734in}{1.185647in}}%
\pgfpathlineto{\pgfqpoint{2.896748in}{1.180551in}}%
\pgfpathlineto{\pgfqpoint{2.907763in}{1.153599in}}%
\pgfpathlineto{\pgfqpoint{2.918777in}{1.164085in}}%
\pgfpathlineto{\pgfqpoint{2.940805in}{1.197106in}}%
\pgfpathlineto{\pgfqpoint{2.951820in}{1.208709in}}%
\pgfpathlineto{\pgfqpoint{2.962834in}{1.203194in}}%
\pgfpathlineto{\pgfqpoint{2.973848in}{1.185537in}}%
\pgfpathlineto{\pgfqpoint{2.984862in}{1.181483in}}%
\pgfpathlineto{\pgfqpoint{2.995877in}{1.183585in}}%
\pgfpathlineto{\pgfqpoint{3.006891in}{1.179334in}}%
\pgfpathlineto{\pgfqpoint{3.017905in}{1.169950in}}%
\pgfpathlineto{\pgfqpoint{3.039934in}{1.105725in}}%
\pgfpathlineto{\pgfqpoint{3.050948in}{1.096037in}}%
\pgfpathlineto{\pgfqpoint{3.061962in}{1.135982in}}%
\pgfpathlineto{\pgfqpoint{3.072976in}{1.138858in}}%
\pgfpathlineto{\pgfqpoint{3.083991in}{1.145066in}}%
\pgfpathlineto{\pgfqpoint{3.095005in}{1.155506in}}%
\pgfpathlineto{\pgfqpoint{3.106019in}{1.164247in}}%
\pgfpathlineto{\pgfqpoint{3.117033in}{1.157767in}}%
\pgfpathlineto{\pgfqpoint{3.128048in}{1.156727in}}%
\pgfpathlineto{\pgfqpoint{3.139062in}{1.146738in}}%
\pgfpathlineto{\pgfqpoint{3.150076in}{1.162701in}}%
\pgfpathlineto{\pgfqpoint{3.161090in}{1.137742in}}%
\pgfpathlineto{\pgfqpoint{3.172105in}{1.107484in}}%
\pgfpathlineto{\pgfqpoint{3.183119in}{1.124320in}}%
\pgfpathlineto{\pgfqpoint{3.194133in}{1.110047in}}%
\pgfpathlineto{\pgfqpoint{3.205147in}{1.091905in}}%
\pgfpathlineto{\pgfqpoint{3.216162in}{1.087784in}}%
\pgfpathlineto{\pgfqpoint{3.227176in}{1.088243in}}%
\pgfpathlineto{\pgfqpoint{3.238190in}{1.075433in}}%
\pgfpathlineto{\pgfqpoint{3.249204in}{1.074305in}}%
\pgfpathlineto{\pgfqpoint{3.260219in}{1.068275in}}%
\pgfpathlineto{\pgfqpoint{3.271233in}{1.084059in}}%
\pgfpathlineto{\pgfqpoint{3.282247in}{1.067554in}}%
\pgfpathlineto{\pgfqpoint{3.293261in}{1.093464in}}%
\pgfpathlineto{\pgfqpoint{3.304276in}{1.063970in}}%
\pgfpathlineto{\pgfqpoint{3.315290in}{1.073324in}}%
\pgfpathlineto{\pgfqpoint{3.326304in}{1.067691in}}%
\pgfpathlineto{\pgfqpoint{3.337318in}{1.035655in}}%
\pgfpathlineto{\pgfqpoint{3.348333in}{1.041416in}}%
\pgfpathlineto{\pgfqpoint{3.359347in}{1.059038in}}%
\pgfpathlineto{\pgfqpoint{3.370361in}{1.046354in}}%
\pgfpathlineto{\pgfqpoint{3.381375in}{1.073102in}}%
\pgfpathlineto{\pgfqpoint{3.392390in}{1.073925in}}%
\pgfpathlineto{\pgfqpoint{3.403404in}{1.059481in}}%
\pgfpathlineto{\pgfqpoint{3.425432in}{1.087401in}}%
\pgfpathlineto{\pgfqpoint{3.436447in}{1.092697in}}%
\pgfpathlineto{\pgfqpoint{3.447461in}{1.073845in}}%
\pgfpathlineto{\pgfqpoint{3.458475in}{1.078713in}}%
\pgfpathlineto{\pgfqpoint{3.469489in}{1.073299in}}%
\pgfpathlineto{\pgfqpoint{3.480504in}{1.074094in}}%
\pgfpathlineto{\pgfqpoint{3.491518in}{1.077815in}}%
\pgfpathlineto{\pgfqpoint{3.502532in}{1.077432in}}%
\pgfpathlineto{\pgfqpoint{3.513546in}{1.097589in}}%
\pgfpathlineto{\pgfqpoint{3.524561in}{1.106713in}}%
\pgfpathlineto{\pgfqpoint{3.535575in}{1.094530in}}%
\pgfpathlineto{\pgfqpoint{3.546589in}{1.079727in}}%
\pgfpathlineto{\pgfqpoint{3.557603in}{1.070258in}}%
\pgfpathlineto{\pgfqpoint{3.568618in}{1.068508in}}%
\pgfpathlineto{\pgfqpoint{3.579632in}{1.059867in}}%
\pgfpathlineto{\pgfqpoint{3.590646in}{1.059957in}}%
\pgfpathlineto{\pgfqpoint{3.601660in}{1.066196in}}%
\pgfpathlineto{\pgfqpoint{3.612675in}{1.058298in}}%
\pgfpathlineto{\pgfqpoint{3.623689in}{1.056526in}}%
\pgfpathlineto{\pgfqpoint{3.634703in}{1.080166in}}%
\pgfpathlineto{\pgfqpoint{3.645717in}{1.074474in}}%
\pgfpathlineto{\pgfqpoint{3.656732in}{1.082315in}}%
\pgfpathlineto{\pgfqpoint{3.667746in}{1.096174in}}%
\pgfpathlineto{\pgfqpoint{3.678760in}{1.095097in}}%
\pgfpathlineto{\pgfqpoint{3.678760in}{1.095097in}}%
\pgfusepath{stroke}%
\end{pgfscope}%
\begin{pgfscope}%
\pgfpathrectangle{\pgfqpoint{0.550713in}{0.408431in}}{\pgfqpoint{3.139062in}{1.490773in}}%
\pgfusepath{clip}%
\pgfsetbuttcap%
\pgfsetroundjoin%
\pgfsetlinewidth{0.853187pt}%
\definecolor{currentstroke}{rgb}{0.380392,0.129412,0.345098}%
\pgfsetstrokecolor{currentstroke}%
\pgfsetdash{{0.850000pt}{1.402500pt}}{0.000000pt}%
\pgfpathmoveto{\pgfqpoint{0.550713in}{1.049741in}}%
\pgfpathlineto{\pgfqpoint{0.561727in}{1.121719in}}%
\pgfpathlineto{\pgfqpoint{0.583755in}{1.132033in}}%
\pgfpathlineto{\pgfqpoint{0.594770in}{1.138049in}}%
\pgfpathlineto{\pgfqpoint{0.605784in}{1.156315in}}%
\pgfpathlineto{\pgfqpoint{0.616798in}{1.169923in}}%
\pgfpathlineto{\pgfqpoint{0.627812in}{1.195697in}}%
\pgfpathlineto{\pgfqpoint{0.638827in}{1.210688in}}%
\pgfpathlineto{\pgfqpoint{0.649841in}{1.210930in}}%
\pgfpathlineto{\pgfqpoint{0.660855in}{1.221834in}}%
\pgfpathlineto{\pgfqpoint{0.671869in}{1.228417in}}%
\pgfpathlineto{\pgfqpoint{0.682884in}{1.256600in}}%
\pgfpathlineto{\pgfqpoint{0.693898in}{1.277160in}}%
\pgfpathlineto{\pgfqpoint{0.704912in}{1.289802in}}%
\pgfpathlineto{\pgfqpoint{0.715926in}{1.315273in}}%
\pgfpathlineto{\pgfqpoint{0.737955in}{1.349435in}}%
\pgfpathlineto{\pgfqpoint{0.748969in}{1.340511in}}%
\pgfpathlineto{\pgfqpoint{0.759983in}{1.350275in}}%
\pgfpathlineto{\pgfqpoint{0.770998in}{1.379420in}}%
\pgfpathlineto{\pgfqpoint{0.782012in}{1.403298in}}%
\pgfpathlineto{\pgfqpoint{0.793026in}{1.434708in}}%
\pgfpathlineto{\pgfqpoint{0.804040in}{1.457959in}}%
\pgfpathlineto{\pgfqpoint{0.815055in}{1.464346in}}%
\pgfpathlineto{\pgfqpoint{0.826069in}{1.474488in}}%
\pgfpathlineto{\pgfqpoint{0.837083in}{1.493148in}}%
\pgfpathlineto{\pgfqpoint{0.848097in}{1.500590in}}%
\pgfpathlineto{\pgfqpoint{0.859112in}{1.539336in}}%
\pgfpathlineto{\pgfqpoint{0.870126in}{1.569971in}}%
\pgfpathlineto{\pgfqpoint{0.892154in}{1.602192in}}%
\pgfpathlineto{\pgfqpoint{0.914183in}{1.632047in}}%
\pgfpathlineto{\pgfqpoint{0.925197in}{1.655030in}}%
\pgfpathlineto{\pgfqpoint{0.936211in}{1.673475in}}%
\pgfpathlineto{\pgfqpoint{0.958240in}{1.686667in}}%
\pgfpathlineto{\pgfqpoint{0.969254in}{1.695610in}}%
\pgfpathlineto{\pgfqpoint{0.980268in}{1.693395in}}%
\pgfpathlineto{\pgfqpoint{0.991283in}{1.705332in}}%
\pgfpathlineto{\pgfqpoint{1.002297in}{1.703371in}}%
\pgfpathlineto{\pgfqpoint{1.013311in}{1.707382in}}%
\pgfpathlineto{\pgfqpoint{1.024325in}{1.717956in}}%
\pgfpathlineto{\pgfqpoint{1.035340in}{1.726958in}}%
\pgfpathlineto{\pgfqpoint{1.046354in}{1.731078in}}%
\pgfpathlineto{\pgfqpoint{1.057368in}{1.736502in}}%
\pgfpathlineto{\pgfqpoint{1.090411in}{1.755918in}}%
\pgfpathlineto{\pgfqpoint{1.101425in}{1.762051in}}%
\pgfpathlineto{\pgfqpoint{1.112439in}{1.760422in}}%
\pgfpathlineto{\pgfqpoint{1.123454in}{1.762635in}}%
\pgfpathlineto{\pgfqpoint{1.145482in}{1.770136in}}%
\pgfpathlineto{\pgfqpoint{1.167511in}{1.774762in}}%
\pgfpathlineto{\pgfqpoint{1.178525in}{1.777798in}}%
\pgfpathlineto{\pgfqpoint{1.211568in}{1.779220in}}%
\pgfpathlineto{\pgfqpoint{2.962834in}{1.779785in}}%
\pgfpathlineto{\pgfqpoint{3.678760in}{1.779797in}}%
\pgfpathlineto{\pgfqpoint{3.678760in}{1.779797in}}%
\pgfusepath{stroke}%
\end{pgfscope}%
\begin{pgfscope}%
\pgfpathrectangle{\pgfqpoint{0.550713in}{0.408431in}}{\pgfqpoint{3.139062in}{1.490773in}}%
\pgfusepath{clip}%
\pgfsetbuttcap%
\pgfsetroundjoin%
\pgfsetlinewidth{0.853187pt}%
\definecolor{currentstroke}{rgb}{0.964706,0.658824,0.000000}%
\pgfsetstrokecolor{currentstroke}%
\pgfsetdash{{0.850000pt}{1.402500pt}}{0.000000pt}%
\pgfpathmoveto{\pgfqpoint{0.550713in}{1.031770in}}%
\pgfpathlineto{\pgfqpoint{0.561727in}{1.109567in}}%
\pgfpathlineto{\pgfqpoint{0.583755in}{1.121486in}}%
\pgfpathlineto{\pgfqpoint{0.594770in}{1.130255in}}%
\pgfpathlineto{\pgfqpoint{0.605784in}{1.149726in}}%
\pgfpathlineto{\pgfqpoint{0.616798in}{1.162969in}}%
\pgfpathlineto{\pgfqpoint{0.627812in}{1.186450in}}%
\pgfpathlineto{\pgfqpoint{0.638827in}{1.204825in}}%
\pgfpathlineto{\pgfqpoint{0.649841in}{1.204241in}}%
\pgfpathlineto{\pgfqpoint{0.660855in}{1.223035in}}%
\pgfpathlineto{\pgfqpoint{0.671869in}{1.236507in}}%
\pgfpathlineto{\pgfqpoint{0.693898in}{1.284342in}}%
\pgfpathlineto{\pgfqpoint{0.704912in}{1.301696in}}%
\pgfpathlineto{\pgfqpoint{0.715926in}{1.328419in}}%
\pgfpathlineto{\pgfqpoint{0.737955in}{1.340218in}}%
\pgfpathlineto{\pgfqpoint{0.748969in}{1.311049in}}%
\pgfpathlineto{\pgfqpoint{0.759983in}{1.305110in}}%
\pgfpathlineto{\pgfqpoint{0.770998in}{1.314103in}}%
\pgfpathlineto{\pgfqpoint{0.782012in}{1.309583in}}%
\pgfpathlineto{\pgfqpoint{0.793026in}{1.266784in}}%
\pgfpathlineto{\pgfqpoint{0.804040in}{1.271743in}}%
\pgfpathlineto{\pgfqpoint{0.815055in}{1.271015in}}%
\pgfpathlineto{\pgfqpoint{0.826069in}{1.242138in}}%
\pgfpathlineto{\pgfqpoint{0.837083in}{1.232884in}}%
\pgfpathlineto{\pgfqpoint{0.848097in}{1.227283in}}%
\pgfpathlineto{\pgfqpoint{0.859112in}{1.242965in}}%
\pgfpathlineto{\pgfqpoint{0.870126in}{1.243482in}}%
\pgfpathlineto{\pgfqpoint{0.881140in}{1.231938in}}%
\pgfpathlineto{\pgfqpoint{0.892154in}{1.246646in}}%
\pgfpathlineto{\pgfqpoint{0.903169in}{1.264574in}}%
\pgfpathlineto{\pgfqpoint{0.914183in}{1.288907in}}%
\pgfpathlineto{\pgfqpoint{0.925197in}{1.304310in}}%
\pgfpathlineto{\pgfqpoint{0.936211in}{1.323834in}}%
\pgfpathlineto{\pgfqpoint{0.947226in}{1.332000in}}%
\pgfpathlineto{\pgfqpoint{0.958240in}{1.327303in}}%
\pgfpathlineto{\pgfqpoint{0.969254in}{1.336290in}}%
\pgfpathlineto{\pgfqpoint{0.980268in}{1.324366in}}%
\pgfpathlineto{\pgfqpoint{0.991283in}{1.333340in}}%
\pgfpathlineto{\pgfqpoint{1.002297in}{1.319655in}}%
\pgfpathlineto{\pgfqpoint{1.013311in}{1.321413in}}%
\pgfpathlineto{\pgfqpoint{1.024325in}{1.326437in}}%
\pgfpathlineto{\pgfqpoint{1.035340in}{1.339344in}}%
\pgfpathlineto{\pgfqpoint{1.046354in}{1.321149in}}%
\pgfpathlineto{\pgfqpoint{1.057368in}{1.326160in}}%
\pgfpathlineto{\pgfqpoint{1.068382in}{1.347532in}}%
\pgfpathlineto{\pgfqpoint{1.079397in}{1.345732in}}%
\pgfpathlineto{\pgfqpoint{1.090411in}{1.351654in}}%
\pgfpathlineto{\pgfqpoint{1.101425in}{1.358992in}}%
\pgfpathlineto{\pgfqpoint{1.112439in}{1.359968in}}%
\pgfpathlineto{\pgfqpoint{1.123454in}{1.357364in}}%
\pgfpathlineto{\pgfqpoint{1.134468in}{1.346180in}}%
\pgfpathlineto{\pgfqpoint{1.145482in}{1.377945in}}%
\pgfpathlineto{\pgfqpoint{1.156496in}{1.384760in}}%
\pgfpathlineto{\pgfqpoint{1.167511in}{1.375218in}}%
\pgfpathlineto{\pgfqpoint{1.189539in}{1.372793in}}%
\pgfpathlineto{\pgfqpoint{1.200553in}{1.368383in}}%
\pgfpathlineto{\pgfqpoint{1.211568in}{1.357560in}}%
\pgfpathlineto{\pgfqpoint{1.233596in}{1.351021in}}%
\pgfpathlineto{\pgfqpoint{1.244610in}{1.360627in}}%
\pgfpathlineto{\pgfqpoint{1.255625in}{1.349528in}}%
\pgfpathlineto{\pgfqpoint{1.277653in}{1.381951in}}%
\pgfpathlineto{\pgfqpoint{1.288667in}{1.385758in}}%
\pgfpathlineto{\pgfqpoint{1.299682in}{1.398636in}}%
\pgfpathlineto{\pgfqpoint{1.310696in}{1.387875in}}%
\pgfpathlineto{\pgfqpoint{1.321710in}{1.381374in}}%
\pgfpathlineto{\pgfqpoint{1.332725in}{1.378832in}}%
\pgfpathlineto{\pgfqpoint{1.343739in}{1.380267in}}%
\pgfpathlineto{\pgfqpoint{1.354753in}{1.377182in}}%
\pgfpathlineto{\pgfqpoint{1.365767in}{1.365371in}}%
\pgfpathlineto{\pgfqpoint{1.376782in}{1.366658in}}%
\pgfpathlineto{\pgfqpoint{1.398810in}{1.375265in}}%
\pgfpathlineto{\pgfqpoint{1.409824in}{1.380781in}}%
\pgfpathlineto{\pgfqpoint{1.420839in}{1.395363in}}%
\pgfpathlineto{\pgfqpoint{1.442867in}{1.408644in}}%
\pgfpathlineto{\pgfqpoint{1.453881in}{1.422812in}}%
\pgfpathlineto{\pgfqpoint{1.464896in}{1.416688in}}%
\pgfpathlineto{\pgfqpoint{1.475910in}{1.434971in}}%
\pgfpathlineto{\pgfqpoint{1.486924in}{1.434238in}}%
\pgfpathlineto{\pgfqpoint{1.497938in}{1.428475in}}%
\pgfpathlineto{\pgfqpoint{1.508953in}{1.428777in}}%
\pgfpathlineto{\pgfqpoint{1.519967in}{1.453688in}}%
\pgfpathlineto{\pgfqpoint{1.530981in}{1.457804in}}%
\pgfpathlineto{\pgfqpoint{1.541995in}{1.452101in}}%
\pgfpathlineto{\pgfqpoint{1.553010in}{1.450635in}}%
\pgfpathlineto{\pgfqpoint{1.564024in}{1.453944in}}%
\pgfpathlineto{\pgfqpoint{1.575038in}{1.469647in}}%
\pgfpathlineto{\pgfqpoint{1.586052in}{1.481663in}}%
\pgfpathlineto{\pgfqpoint{1.597067in}{1.491400in}}%
\pgfpathlineto{\pgfqpoint{1.608081in}{1.484886in}}%
\pgfpathlineto{\pgfqpoint{1.630109in}{1.485506in}}%
\pgfpathlineto{\pgfqpoint{1.641124in}{1.484211in}}%
\pgfpathlineto{\pgfqpoint{1.652138in}{1.476892in}}%
\pgfpathlineto{\pgfqpoint{1.663152in}{1.493558in}}%
\pgfpathlineto{\pgfqpoint{1.674166in}{1.519679in}}%
\pgfpathlineto{\pgfqpoint{1.685181in}{1.520973in}}%
\pgfpathlineto{\pgfqpoint{1.696195in}{1.515176in}}%
\pgfpathlineto{\pgfqpoint{1.707209in}{1.522433in}}%
\pgfpathlineto{\pgfqpoint{1.718223in}{1.504983in}}%
\pgfpathlineto{\pgfqpoint{1.729238in}{1.507320in}}%
\pgfpathlineto{\pgfqpoint{1.740252in}{1.503393in}}%
\pgfpathlineto{\pgfqpoint{1.762280in}{1.476229in}}%
\pgfpathlineto{\pgfqpoint{1.773295in}{1.438794in}}%
\pgfpathlineto{\pgfqpoint{1.784309in}{1.450991in}}%
\pgfpathlineto{\pgfqpoint{1.795323in}{1.438934in}}%
\pgfpathlineto{\pgfqpoint{1.806337in}{1.453866in}}%
\pgfpathlineto{\pgfqpoint{1.817352in}{1.453721in}}%
\pgfpathlineto{\pgfqpoint{1.828366in}{1.463834in}}%
\pgfpathlineto{\pgfqpoint{1.839380in}{1.455275in}}%
\pgfpathlineto{\pgfqpoint{1.850394in}{1.444160in}}%
\pgfpathlineto{\pgfqpoint{1.861409in}{1.434750in}}%
\pgfpathlineto{\pgfqpoint{1.872423in}{1.434141in}}%
\pgfpathlineto{\pgfqpoint{1.883437in}{1.436896in}}%
\pgfpathlineto{\pgfqpoint{1.894451in}{1.435953in}}%
\pgfpathlineto{\pgfqpoint{1.905466in}{1.445541in}}%
\pgfpathlineto{\pgfqpoint{1.916480in}{1.451751in}}%
\pgfpathlineto{\pgfqpoint{1.927494in}{1.451127in}}%
\pgfpathlineto{\pgfqpoint{1.938508in}{1.435618in}}%
\pgfpathlineto{\pgfqpoint{1.949523in}{1.426845in}}%
\pgfpathlineto{\pgfqpoint{1.960537in}{1.422660in}}%
\pgfpathlineto{\pgfqpoint{1.982565in}{1.441964in}}%
\pgfpathlineto{\pgfqpoint{1.993580in}{1.432667in}}%
\pgfpathlineto{\pgfqpoint{2.004594in}{1.442509in}}%
\pgfpathlineto{\pgfqpoint{2.026622in}{1.431924in}}%
\pgfpathlineto{\pgfqpoint{2.037637in}{1.425858in}}%
\pgfpathlineto{\pgfqpoint{2.048651in}{1.440847in}}%
\pgfpathlineto{\pgfqpoint{2.059665in}{1.445564in}}%
\pgfpathlineto{\pgfqpoint{2.070679in}{1.456727in}}%
\pgfpathlineto{\pgfqpoint{2.081694in}{1.437196in}}%
\pgfpathlineto{\pgfqpoint{2.114736in}{1.422109in}}%
\pgfpathlineto{\pgfqpoint{2.125751in}{1.412344in}}%
\pgfpathlineto{\pgfqpoint{2.136765in}{1.426206in}}%
\pgfpathlineto{\pgfqpoint{2.147779in}{1.429341in}}%
\pgfpathlineto{\pgfqpoint{2.158793in}{1.439438in}}%
\pgfpathlineto{\pgfqpoint{2.169808in}{1.428625in}}%
\pgfpathlineto{\pgfqpoint{2.180822in}{1.424840in}}%
\pgfpathlineto{\pgfqpoint{2.191836in}{1.428146in}}%
\pgfpathlineto{\pgfqpoint{2.202850in}{1.409961in}}%
\pgfpathlineto{\pgfqpoint{2.213865in}{1.402651in}}%
\pgfpathlineto{\pgfqpoint{2.235893in}{1.410334in}}%
\pgfpathlineto{\pgfqpoint{2.246907in}{1.406227in}}%
\pgfpathlineto{\pgfqpoint{2.257922in}{1.406576in}}%
\pgfpathlineto{\pgfqpoint{2.268936in}{1.405468in}}%
\pgfpathlineto{\pgfqpoint{2.279950in}{1.419647in}}%
\pgfpathlineto{\pgfqpoint{2.290964in}{1.422644in}}%
\pgfpathlineto{\pgfqpoint{2.301979in}{1.418338in}}%
\pgfpathlineto{\pgfqpoint{2.312993in}{1.428517in}}%
\pgfpathlineto{\pgfqpoint{2.324007in}{1.445907in}}%
\pgfpathlineto{\pgfqpoint{2.335021in}{1.435777in}}%
\pgfpathlineto{\pgfqpoint{2.346036in}{1.432816in}}%
\pgfpathlineto{\pgfqpoint{2.357050in}{1.448694in}}%
\pgfpathlineto{\pgfqpoint{2.368064in}{1.456964in}}%
\pgfpathlineto{\pgfqpoint{2.379078in}{1.461747in}}%
\pgfpathlineto{\pgfqpoint{2.390093in}{1.463962in}}%
\pgfpathlineto{\pgfqpoint{2.401107in}{1.480123in}}%
\pgfpathlineto{\pgfqpoint{2.412121in}{1.486049in}}%
\pgfpathlineto{\pgfqpoint{2.434150in}{1.490576in}}%
\pgfpathlineto{\pgfqpoint{2.456178in}{1.508981in}}%
\pgfpathlineto{\pgfqpoint{2.467192in}{1.506511in}}%
\pgfpathlineto{\pgfqpoint{2.478207in}{1.507902in}}%
\pgfpathlineto{\pgfqpoint{2.500235in}{1.523514in}}%
\pgfpathlineto{\pgfqpoint{2.511249in}{1.521238in}}%
\pgfpathlineto{\pgfqpoint{2.522264in}{1.526352in}}%
\pgfpathlineto{\pgfqpoint{2.544292in}{1.532040in}}%
\pgfpathlineto{\pgfqpoint{2.555306in}{1.543089in}}%
\pgfpathlineto{\pgfqpoint{2.566321in}{1.544553in}}%
\pgfpathlineto{\pgfqpoint{2.577335in}{1.544671in}}%
\pgfpathlineto{\pgfqpoint{2.588349in}{1.535752in}}%
\pgfpathlineto{\pgfqpoint{2.599363in}{1.553085in}}%
\pgfpathlineto{\pgfqpoint{2.621392in}{1.546882in}}%
\pgfpathlineto{\pgfqpoint{2.632406in}{1.558140in}}%
\pgfpathlineto{\pgfqpoint{2.643421in}{1.561404in}}%
\pgfpathlineto{\pgfqpoint{2.654435in}{1.570984in}}%
\pgfpathlineto{\pgfqpoint{2.665449in}{1.558636in}}%
\pgfpathlineto{\pgfqpoint{2.676463in}{1.563830in}}%
\pgfpathlineto{\pgfqpoint{2.687478in}{1.558485in}}%
\pgfpathlineto{\pgfqpoint{2.698492in}{1.559945in}}%
\pgfpathlineto{\pgfqpoint{2.709506in}{1.572991in}}%
\pgfpathlineto{\pgfqpoint{2.720520in}{1.570312in}}%
\pgfpathlineto{\pgfqpoint{2.742549in}{1.583661in}}%
\pgfpathlineto{\pgfqpoint{2.753563in}{1.589881in}}%
\pgfpathlineto{\pgfqpoint{2.764577in}{1.582969in}}%
\pgfpathlineto{\pgfqpoint{2.775592in}{1.579343in}}%
\pgfpathlineto{\pgfqpoint{2.786606in}{1.593945in}}%
\pgfpathlineto{\pgfqpoint{2.797620in}{1.576157in}}%
\pgfpathlineto{\pgfqpoint{2.808634in}{1.568193in}}%
\pgfpathlineto{\pgfqpoint{2.819649in}{1.587963in}}%
\pgfpathlineto{\pgfqpoint{2.830663in}{1.589126in}}%
\pgfpathlineto{\pgfqpoint{2.852691in}{1.582647in}}%
\pgfpathlineto{\pgfqpoint{2.863706in}{1.600897in}}%
\pgfpathlineto{\pgfqpoint{2.874720in}{1.599041in}}%
\pgfpathlineto{\pgfqpoint{2.885734in}{1.598600in}}%
\pgfpathlineto{\pgfqpoint{2.896748in}{1.589093in}}%
\pgfpathlineto{\pgfqpoint{2.907763in}{1.589064in}}%
\pgfpathlineto{\pgfqpoint{2.918777in}{1.584895in}}%
\pgfpathlineto{\pgfqpoint{2.940805in}{1.574302in}}%
\pgfpathlineto{\pgfqpoint{2.951820in}{1.572308in}}%
\pgfpathlineto{\pgfqpoint{2.962834in}{1.588280in}}%
\pgfpathlineto{\pgfqpoint{2.973848in}{1.587577in}}%
\pgfpathlineto{\pgfqpoint{2.995877in}{1.567263in}}%
\pgfpathlineto{\pgfqpoint{3.017905in}{1.566158in}}%
\pgfpathlineto{\pgfqpoint{3.028919in}{1.569806in}}%
\pgfpathlineto{\pgfqpoint{3.039934in}{1.564888in}}%
\pgfpathlineto{\pgfqpoint{3.050948in}{1.558502in}}%
\pgfpathlineto{\pgfqpoint{3.061962in}{1.559059in}}%
\pgfpathlineto{\pgfqpoint{3.072976in}{1.552659in}}%
\pgfpathlineto{\pgfqpoint{3.083991in}{1.542874in}}%
\pgfpathlineto{\pgfqpoint{3.095005in}{1.541009in}}%
\pgfpathlineto{\pgfqpoint{3.106019in}{1.537159in}}%
\pgfpathlineto{\pgfqpoint{3.128048in}{1.538947in}}%
\pgfpathlineto{\pgfqpoint{3.139062in}{1.546773in}}%
\pgfpathlineto{\pgfqpoint{3.150076in}{1.538432in}}%
\pgfpathlineto{\pgfqpoint{3.161090in}{1.527442in}}%
\pgfpathlineto{\pgfqpoint{3.172105in}{1.531465in}}%
\pgfpathlineto{\pgfqpoint{3.183119in}{1.531304in}}%
\pgfpathlineto{\pgfqpoint{3.194133in}{1.525272in}}%
\pgfpathlineto{\pgfqpoint{3.205147in}{1.531035in}}%
\pgfpathlineto{\pgfqpoint{3.216162in}{1.529987in}}%
\pgfpathlineto{\pgfqpoint{3.227176in}{1.534180in}}%
\pgfpathlineto{\pgfqpoint{3.249204in}{1.521019in}}%
\pgfpathlineto{\pgfqpoint{3.260219in}{1.516379in}}%
\pgfpathlineto{\pgfqpoint{3.271233in}{1.522566in}}%
\pgfpathlineto{\pgfqpoint{3.282247in}{1.524816in}}%
\pgfpathlineto{\pgfqpoint{3.293261in}{1.530296in}}%
\pgfpathlineto{\pgfqpoint{3.304276in}{1.530734in}}%
\pgfpathlineto{\pgfqpoint{3.315290in}{1.533569in}}%
\pgfpathlineto{\pgfqpoint{3.326304in}{1.528678in}}%
\pgfpathlineto{\pgfqpoint{3.337318in}{1.526823in}}%
\pgfpathlineto{\pgfqpoint{3.348333in}{1.521111in}}%
\pgfpathlineto{\pgfqpoint{3.359347in}{1.518231in}}%
\pgfpathlineto{\pgfqpoint{3.370361in}{1.526547in}}%
\pgfpathlineto{\pgfqpoint{3.381375in}{1.545361in}}%
\pgfpathlineto{\pgfqpoint{3.392390in}{1.551705in}}%
\pgfpathlineto{\pgfqpoint{3.414418in}{1.538704in}}%
\pgfpathlineto{\pgfqpoint{3.425432in}{1.537173in}}%
\pgfpathlineto{\pgfqpoint{3.436447in}{1.540517in}}%
\pgfpathlineto{\pgfqpoint{3.447461in}{1.538357in}}%
\pgfpathlineto{\pgfqpoint{3.458475in}{1.534828in}}%
\pgfpathlineto{\pgfqpoint{3.469489in}{1.535453in}}%
\pgfpathlineto{\pgfqpoint{3.480504in}{1.540286in}}%
\pgfpathlineto{\pgfqpoint{3.491518in}{1.529706in}}%
\pgfpathlineto{\pgfqpoint{3.502532in}{1.525659in}}%
\pgfpathlineto{\pgfqpoint{3.513546in}{1.525490in}}%
\pgfpathlineto{\pgfqpoint{3.524561in}{1.530728in}}%
\pgfpathlineto{\pgfqpoint{3.535575in}{1.539275in}}%
\pgfpathlineto{\pgfqpoint{3.546589in}{1.538615in}}%
\pgfpathlineto{\pgfqpoint{3.568618in}{1.528896in}}%
\pgfpathlineto{\pgfqpoint{3.579632in}{1.535642in}}%
\pgfpathlineto{\pgfqpoint{3.590646in}{1.530732in}}%
\pgfpathlineto{\pgfqpoint{3.601660in}{1.528461in}}%
\pgfpathlineto{\pgfqpoint{3.612675in}{1.528172in}}%
\pgfpathlineto{\pgfqpoint{3.623689in}{1.526396in}}%
\pgfpathlineto{\pgfqpoint{3.634703in}{1.522558in}}%
\pgfpathlineto{\pgfqpoint{3.645717in}{1.532153in}}%
\pgfpathlineto{\pgfqpoint{3.656732in}{1.527993in}}%
\pgfpathlineto{\pgfqpoint{3.667746in}{1.517464in}}%
\pgfpathlineto{\pgfqpoint{3.678760in}{1.511310in}}%
\pgfpathlineto{\pgfqpoint{3.678760in}{1.511310in}}%
\pgfusepath{stroke}%
\end{pgfscope}%
\begin{pgfscope}%
\pgfpathrectangle{\pgfqpoint{0.550713in}{0.408431in}}{\pgfqpoint{3.139062in}{1.490773in}}%
\pgfusepath{clip}%
\pgfsetbuttcap%
\pgfsetroundjoin%
\pgfsetlinewidth{0.853187pt}%
\definecolor{currentstroke}{rgb}{0.000000,0.329412,0.623529}%
\pgfsetstrokecolor{currentstroke}%
\pgfsetdash{{0.850000pt}{1.402500pt}}{0.000000pt}%
\pgfpathmoveto{\pgfqpoint{0.550713in}{0.985522in}}%
\pgfpathlineto{\pgfqpoint{0.561727in}{1.050203in}}%
\pgfpathlineto{\pgfqpoint{0.594770in}{1.041090in}}%
\pgfpathlineto{\pgfqpoint{0.605784in}{1.049315in}}%
\pgfpathlineto{\pgfqpoint{0.616798in}{1.052838in}}%
\pgfpathlineto{\pgfqpoint{0.638827in}{1.077072in}}%
\pgfpathlineto{\pgfqpoint{0.649841in}{1.067545in}}%
\pgfpathlineto{\pgfqpoint{0.660855in}{1.076131in}}%
\pgfpathlineto{\pgfqpoint{0.671869in}{1.069392in}}%
\pgfpathlineto{\pgfqpoint{0.682884in}{1.072380in}}%
\pgfpathlineto{\pgfqpoint{0.693898in}{1.082659in}}%
\pgfpathlineto{\pgfqpoint{0.704912in}{1.074514in}}%
\pgfpathlineto{\pgfqpoint{0.715926in}{1.074749in}}%
\pgfpathlineto{\pgfqpoint{0.726941in}{1.071949in}}%
\pgfpathlineto{\pgfqpoint{0.737955in}{1.080782in}}%
\pgfpathlineto{\pgfqpoint{0.748969in}{1.083149in}}%
\pgfpathlineto{\pgfqpoint{0.759983in}{1.075496in}}%
\pgfpathlineto{\pgfqpoint{0.770998in}{1.093728in}}%
\pgfpathlineto{\pgfqpoint{0.782012in}{1.107845in}}%
\pgfpathlineto{\pgfqpoint{0.793026in}{1.119147in}}%
\pgfpathlineto{\pgfqpoint{0.804040in}{1.124540in}}%
\pgfpathlineto{\pgfqpoint{0.815055in}{1.128074in}}%
\pgfpathlineto{\pgfqpoint{0.826069in}{1.127927in}}%
\pgfpathlineto{\pgfqpoint{0.837083in}{1.138629in}}%
\pgfpathlineto{\pgfqpoint{0.848097in}{1.110752in}}%
\pgfpathlineto{\pgfqpoint{0.870126in}{1.146081in}}%
\pgfpathlineto{\pgfqpoint{0.881140in}{1.148370in}}%
\pgfpathlineto{\pgfqpoint{0.892154in}{1.154143in}}%
\pgfpathlineto{\pgfqpoint{0.914183in}{1.171610in}}%
\pgfpathlineto{\pgfqpoint{0.925197in}{1.174885in}}%
\pgfpathlineto{\pgfqpoint{0.936211in}{1.198096in}}%
\pgfpathlineto{\pgfqpoint{0.947226in}{1.199196in}}%
\pgfpathlineto{\pgfqpoint{0.958240in}{1.198294in}}%
\pgfpathlineto{\pgfqpoint{0.969254in}{1.205813in}}%
\pgfpathlineto{\pgfqpoint{0.980268in}{1.208408in}}%
\pgfpathlineto{\pgfqpoint{0.991283in}{1.219718in}}%
\pgfpathlineto{\pgfqpoint{1.002297in}{1.207396in}}%
\pgfpathlineto{\pgfqpoint{1.013311in}{1.208817in}}%
\pgfpathlineto{\pgfqpoint{1.024325in}{1.221832in}}%
\pgfpathlineto{\pgfqpoint{1.035340in}{1.231021in}}%
\pgfpathlineto{\pgfqpoint{1.057368in}{1.236778in}}%
\pgfpathlineto{\pgfqpoint{1.068382in}{1.245382in}}%
\pgfpathlineto{\pgfqpoint{1.079397in}{1.249403in}}%
\pgfpathlineto{\pgfqpoint{1.090411in}{1.256042in}}%
\pgfpathlineto{\pgfqpoint{1.101425in}{1.259745in}}%
\pgfpathlineto{\pgfqpoint{1.112439in}{1.270436in}}%
\pgfpathlineto{\pgfqpoint{1.123454in}{1.268671in}}%
\pgfpathlineto{\pgfqpoint{1.134468in}{1.270079in}}%
\pgfpathlineto{\pgfqpoint{1.145482in}{1.274041in}}%
\pgfpathlineto{\pgfqpoint{1.156496in}{1.285046in}}%
\pgfpathlineto{\pgfqpoint{1.167511in}{1.301080in}}%
\pgfpathlineto{\pgfqpoint{1.178525in}{1.291638in}}%
\pgfpathlineto{\pgfqpoint{1.189539in}{1.297273in}}%
\pgfpathlineto{\pgfqpoint{1.200553in}{1.296906in}}%
\pgfpathlineto{\pgfqpoint{1.211568in}{1.279161in}}%
\pgfpathlineto{\pgfqpoint{1.233596in}{1.302165in}}%
\pgfpathlineto{\pgfqpoint{1.255625in}{1.297984in}}%
\pgfpathlineto{\pgfqpoint{1.266639in}{1.301302in}}%
\pgfpathlineto{\pgfqpoint{1.277653in}{1.303382in}}%
\pgfpathlineto{\pgfqpoint{1.288667in}{1.310750in}}%
\pgfpathlineto{\pgfqpoint{1.299682in}{1.321901in}}%
\pgfpathlineto{\pgfqpoint{1.310696in}{1.326274in}}%
\pgfpathlineto{\pgfqpoint{1.321710in}{1.326269in}}%
\pgfpathlineto{\pgfqpoint{1.343739in}{1.331259in}}%
\pgfpathlineto{\pgfqpoint{1.365767in}{1.335597in}}%
\pgfpathlineto{\pgfqpoint{1.376782in}{1.340499in}}%
\pgfpathlineto{\pgfqpoint{1.387796in}{1.333285in}}%
\pgfpathlineto{\pgfqpoint{1.398810in}{1.340921in}}%
\pgfpathlineto{\pgfqpoint{1.409824in}{1.345181in}}%
\pgfpathlineto{\pgfqpoint{1.420839in}{1.367264in}}%
\pgfpathlineto{\pgfqpoint{1.453881in}{1.373649in}}%
\pgfpathlineto{\pgfqpoint{1.464896in}{1.379884in}}%
\pgfpathlineto{\pgfqpoint{1.475910in}{1.388848in}}%
\pgfpathlineto{\pgfqpoint{1.486924in}{1.387410in}}%
\pgfpathlineto{\pgfqpoint{1.497938in}{1.383921in}}%
\pgfpathlineto{\pgfqpoint{1.508953in}{1.377513in}}%
\pgfpathlineto{\pgfqpoint{1.519967in}{1.364686in}}%
\pgfpathlineto{\pgfqpoint{1.530981in}{1.368658in}}%
\pgfpathlineto{\pgfqpoint{1.541995in}{1.368683in}}%
\pgfpathlineto{\pgfqpoint{1.553010in}{1.362506in}}%
\pgfpathlineto{\pgfqpoint{1.597067in}{1.401257in}}%
\pgfpathlineto{\pgfqpoint{1.608081in}{1.406968in}}%
\pgfpathlineto{\pgfqpoint{1.619095in}{1.404011in}}%
\pgfpathlineto{\pgfqpoint{1.630109in}{1.404691in}}%
\pgfpathlineto{\pgfqpoint{1.641124in}{1.408432in}}%
\pgfpathlineto{\pgfqpoint{1.652138in}{1.404491in}}%
\pgfpathlineto{\pgfqpoint{1.663152in}{1.406792in}}%
\pgfpathlineto{\pgfqpoint{1.674166in}{1.400701in}}%
\pgfpathlineto{\pgfqpoint{1.685181in}{1.407264in}}%
\pgfpathlineto{\pgfqpoint{1.696195in}{1.411323in}}%
\pgfpathlineto{\pgfqpoint{1.707209in}{1.416915in}}%
\pgfpathlineto{\pgfqpoint{1.729238in}{1.413805in}}%
\pgfpathlineto{\pgfqpoint{1.740252in}{1.411018in}}%
\pgfpathlineto{\pgfqpoint{1.751266in}{1.419841in}}%
\pgfpathlineto{\pgfqpoint{1.762280in}{1.418923in}}%
\pgfpathlineto{\pgfqpoint{1.773295in}{1.411413in}}%
\pgfpathlineto{\pgfqpoint{1.784309in}{1.420071in}}%
\pgfpathlineto{\pgfqpoint{1.795323in}{1.424562in}}%
\pgfpathlineto{\pgfqpoint{1.806337in}{1.436330in}}%
\pgfpathlineto{\pgfqpoint{1.817352in}{1.438388in}}%
\pgfpathlineto{\pgfqpoint{1.828366in}{1.431916in}}%
\pgfpathlineto{\pgfqpoint{1.839380in}{1.432067in}}%
\pgfpathlineto{\pgfqpoint{1.850394in}{1.429232in}}%
\pgfpathlineto{\pgfqpoint{1.861409in}{1.436636in}}%
\pgfpathlineto{\pgfqpoint{1.883437in}{1.442624in}}%
\pgfpathlineto{\pgfqpoint{1.894451in}{1.448755in}}%
\pgfpathlineto{\pgfqpoint{1.905466in}{1.462783in}}%
\pgfpathlineto{\pgfqpoint{1.916480in}{1.460937in}}%
\pgfpathlineto{\pgfqpoint{1.927494in}{1.465487in}}%
\pgfpathlineto{\pgfqpoint{1.938508in}{1.456855in}}%
\pgfpathlineto{\pgfqpoint{1.949523in}{1.458042in}}%
\pgfpathlineto{\pgfqpoint{1.960537in}{1.454086in}}%
\pgfpathlineto{\pgfqpoint{1.971551in}{1.455768in}}%
\pgfpathlineto{\pgfqpoint{1.982565in}{1.453870in}}%
\pgfpathlineto{\pgfqpoint{1.993580in}{1.449290in}}%
\pgfpathlineto{\pgfqpoint{2.015608in}{1.457414in}}%
\pgfpathlineto{\pgfqpoint{2.026622in}{1.458434in}}%
\pgfpathlineto{\pgfqpoint{2.037637in}{1.462058in}}%
\pgfpathlineto{\pgfqpoint{2.048651in}{1.468684in}}%
\pgfpathlineto{\pgfqpoint{2.059665in}{1.470699in}}%
\pgfpathlineto{\pgfqpoint{2.070679in}{1.484024in}}%
\pgfpathlineto{\pgfqpoint{2.081694in}{1.486445in}}%
\pgfpathlineto{\pgfqpoint{2.092708in}{1.486005in}}%
\pgfpathlineto{\pgfqpoint{2.103722in}{1.483807in}}%
\pgfpathlineto{\pgfqpoint{2.125751in}{1.486278in}}%
\pgfpathlineto{\pgfqpoint{2.136765in}{1.491226in}}%
\pgfpathlineto{\pgfqpoint{2.147779in}{1.486415in}}%
\pgfpathlineto{\pgfqpoint{2.180822in}{1.480647in}}%
\pgfpathlineto{\pgfqpoint{2.213865in}{1.484479in}}%
\pgfpathlineto{\pgfqpoint{2.224879in}{1.483560in}}%
\pgfpathlineto{\pgfqpoint{2.246907in}{1.485433in}}%
\pgfpathlineto{\pgfqpoint{2.268936in}{1.497210in}}%
\pgfpathlineto{\pgfqpoint{2.279950in}{1.492863in}}%
\pgfpathlineto{\pgfqpoint{2.290964in}{1.496288in}}%
\pgfpathlineto{\pgfqpoint{2.301979in}{1.497319in}}%
\pgfpathlineto{\pgfqpoint{2.312993in}{1.488049in}}%
\pgfpathlineto{\pgfqpoint{2.324007in}{1.496049in}}%
\pgfpathlineto{\pgfqpoint{2.335021in}{1.509222in}}%
\pgfpathlineto{\pgfqpoint{2.346036in}{1.513190in}}%
\pgfpathlineto{\pgfqpoint{2.368064in}{1.512353in}}%
\pgfpathlineto{\pgfqpoint{2.390093in}{1.515680in}}%
\pgfpathlineto{\pgfqpoint{2.401107in}{1.519627in}}%
\pgfpathlineto{\pgfqpoint{2.412121in}{1.516828in}}%
\pgfpathlineto{\pgfqpoint{2.423135in}{1.517198in}}%
\pgfpathlineto{\pgfqpoint{2.434150in}{1.514420in}}%
\pgfpathlineto{\pgfqpoint{2.445164in}{1.522107in}}%
\pgfpathlineto{\pgfqpoint{2.456178in}{1.524532in}}%
\pgfpathlineto{\pgfqpoint{2.467192in}{1.529755in}}%
\pgfpathlineto{\pgfqpoint{2.478207in}{1.530696in}}%
\pgfpathlineto{\pgfqpoint{2.500235in}{1.525148in}}%
\pgfpathlineto{\pgfqpoint{2.511249in}{1.524456in}}%
\pgfpathlineto{\pgfqpoint{2.522264in}{1.521741in}}%
\pgfpathlineto{\pgfqpoint{2.533278in}{1.526011in}}%
\pgfpathlineto{\pgfqpoint{2.544292in}{1.533131in}}%
\pgfpathlineto{\pgfqpoint{2.566321in}{1.535715in}}%
\pgfpathlineto{\pgfqpoint{2.588349in}{1.536812in}}%
\pgfpathlineto{\pgfqpoint{2.599363in}{1.542186in}}%
\pgfpathlineto{\pgfqpoint{2.610378in}{1.538849in}}%
\pgfpathlineto{\pgfqpoint{2.621392in}{1.540288in}}%
\pgfpathlineto{\pgfqpoint{2.632406in}{1.540108in}}%
\pgfpathlineto{\pgfqpoint{2.643421in}{1.541329in}}%
\pgfpathlineto{\pgfqpoint{2.654435in}{1.537634in}}%
\pgfpathlineto{\pgfqpoint{2.665449in}{1.538130in}}%
\pgfpathlineto{\pgfqpoint{2.676463in}{1.540666in}}%
\pgfpathlineto{\pgfqpoint{2.687478in}{1.546604in}}%
\pgfpathlineto{\pgfqpoint{2.698492in}{1.554319in}}%
\pgfpathlineto{\pgfqpoint{2.709506in}{1.558549in}}%
\pgfpathlineto{\pgfqpoint{2.731535in}{1.561181in}}%
\pgfpathlineto{\pgfqpoint{2.742549in}{1.559657in}}%
\pgfpathlineto{\pgfqpoint{2.753563in}{1.561887in}}%
\pgfpathlineto{\pgfqpoint{2.764577in}{1.558702in}}%
\pgfpathlineto{\pgfqpoint{2.786606in}{1.567864in}}%
\pgfpathlineto{\pgfqpoint{2.797620in}{1.558120in}}%
\pgfpathlineto{\pgfqpoint{2.808634in}{1.553428in}}%
\pgfpathlineto{\pgfqpoint{2.819649in}{1.563885in}}%
\pgfpathlineto{\pgfqpoint{2.830663in}{1.563070in}}%
\pgfpathlineto{\pgfqpoint{2.841677in}{1.571366in}}%
\pgfpathlineto{\pgfqpoint{2.852691in}{1.574222in}}%
\pgfpathlineto{\pgfqpoint{2.863706in}{1.573138in}}%
\pgfpathlineto{\pgfqpoint{2.874720in}{1.569134in}}%
\pgfpathlineto{\pgfqpoint{2.896748in}{1.568646in}}%
\pgfpathlineto{\pgfqpoint{2.918777in}{1.572238in}}%
\pgfpathlineto{\pgfqpoint{2.929791in}{1.571540in}}%
\pgfpathlineto{\pgfqpoint{2.940805in}{1.567577in}}%
\pgfpathlineto{\pgfqpoint{2.951820in}{1.570073in}}%
\pgfpathlineto{\pgfqpoint{2.962834in}{1.575945in}}%
\pgfpathlineto{\pgfqpoint{2.973848in}{1.579622in}}%
\pgfpathlineto{\pgfqpoint{2.984862in}{1.577284in}}%
\pgfpathlineto{\pgfqpoint{2.995877in}{1.580974in}}%
\pgfpathlineto{\pgfqpoint{3.028919in}{1.587822in}}%
\pgfpathlineto{\pgfqpoint{3.061962in}{1.594521in}}%
\pgfpathlineto{\pgfqpoint{3.072976in}{1.597003in}}%
\pgfpathlineto{\pgfqpoint{3.083991in}{1.597242in}}%
\pgfpathlineto{\pgfqpoint{3.095005in}{1.595963in}}%
\pgfpathlineto{\pgfqpoint{3.106019in}{1.600513in}}%
\pgfpathlineto{\pgfqpoint{3.128048in}{1.601181in}}%
\pgfpathlineto{\pgfqpoint{3.139062in}{1.600606in}}%
\pgfpathlineto{\pgfqpoint{3.150076in}{1.593049in}}%
\pgfpathlineto{\pgfqpoint{3.161090in}{1.590068in}}%
\pgfpathlineto{\pgfqpoint{3.172105in}{1.592164in}}%
\pgfpathlineto{\pgfqpoint{3.183119in}{1.592017in}}%
\pgfpathlineto{\pgfqpoint{3.194133in}{1.596258in}}%
\pgfpathlineto{\pgfqpoint{3.205147in}{1.597018in}}%
\pgfpathlineto{\pgfqpoint{3.227176in}{1.600461in}}%
\pgfpathlineto{\pgfqpoint{3.238190in}{1.599985in}}%
\pgfpathlineto{\pgfqpoint{3.249204in}{1.601020in}}%
\pgfpathlineto{\pgfqpoint{3.260219in}{1.600014in}}%
\pgfpathlineto{\pgfqpoint{3.271233in}{1.602241in}}%
\pgfpathlineto{\pgfqpoint{3.293261in}{1.602904in}}%
\pgfpathlineto{\pgfqpoint{3.304276in}{1.600694in}}%
\pgfpathlineto{\pgfqpoint{3.348333in}{1.600702in}}%
\pgfpathlineto{\pgfqpoint{3.359347in}{1.600087in}}%
\pgfpathlineto{\pgfqpoint{3.392390in}{1.602093in}}%
\pgfpathlineto{\pgfqpoint{3.403404in}{1.604952in}}%
\pgfpathlineto{\pgfqpoint{3.414418in}{1.606373in}}%
\pgfpathlineto{\pgfqpoint{3.425432in}{1.590869in}}%
\pgfpathlineto{\pgfqpoint{3.436447in}{1.590368in}}%
\pgfpathlineto{\pgfqpoint{3.491518in}{1.595699in}}%
\pgfpathlineto{\pgfqpoint{3.524561in}{1.600197in}}%
\pgfpathlineto{\pgfqpoint{3.535575in}{1.597675in}}%
\pgfpathlineto{\pgfqpoint{3.568618in}{1.599573in}}%
\pgfpathlineto{\pgfqpoint{3.579632in}{1.596115in}}%
\pgfpathlineto{\pgfqpoint{3.590646in}{1.595386in}}%
\pgfpathlineto{\pgfqpoint{3.601660in}{1.597662in}}%
\pgfpathlineto{\pgfqpoint{3.612675in}{1.602185in}}%
\pgfpathlineto{\pgfqpoint{3.634703in}{1.601196in}}%
\pgfpathlineto{\pgfqpoint{3.656732in}{1.605533in}}%
\pgfpathlineto{\pgfqpoint{3.667746in}{1.605077in}}%
\pgfpathlineto{\pgfqpoint{3.678760in}{1.607798in}}%
\pgfpathlineto{\pgfqpoint{3.678760in}{1.607798in}}%
\pgfusepath{stroke}%
\end{pgfscope}%
\begin{pgfscope}%
\pgfpathrectangle{\pgfqpoint{0.550713in}{0.408431in}}{\pgfqpoint{3.139062in}{1.490773in}}%
\pgfusepath{clip}%
\pgfsetbuttcap%
\pgfsetroundjoin%
\pgfsetlinewidth{0.853187pt}%
\definecolor{currentstroke}{rgb}{0.478431,0.435294,0.674510}%
\pgfsetstrokecolor{currentstroke}%
\pgfsetdash{{0.850000pt}{1.402500pt}}{0.000000pt}%
\pgfpathmoveto{\pgfqpoint{0.550713in}{0.985034in}}%
\pgfpathlineto{\pgfqpoint{0.561727in}{1.049801in}}%
\pgfpathlineto{\pgfqpoint{0.572741in}{1.054487in}}%
\pgfpathlineto{\pgfqpoint{0.583755in}{1.011055in}}%
\pgfpathlineto{\pgfqpoint{0.594770in}{1.000215in}}%
\pgfpathlineto{\pgfqpoint{0.605784in}{0.985334in}}%
\pgfpathlineto{\pgfqpoint{0.616798in}{0.984139in}}%
\pgfpathlineto{\pgfqpoint{0.627812in}{0.993727in}}%
\pgfpathlineto{\pgfqpoint{0.638827in}{0.964121in}}%
\pgfpathlineto{\pgfqpoint{0.649841in}{0.953516in}}%
\pgfpathlineto{\pgfqpoint{0.660855in}{0.957540in}}%
\pgfpathlineto{\pgfqpoint{0.671869in}{0.928908in}}%
\pgfpathlineto{\pgfqpoint{0.682884in}{0.928680in}}%
\pgfpathlineto{\pgfqpoint{0.693898in}{0.918031in}}%
\pgfpathlineto{\pgfqpoint{0.704912in}{0.898973in}}%
\pgfpathlineto{\pgfqpoint{0.715926in}{0.937560in}}%
\pgfpathlineto{\pgfqpoint{0.726941in}{0.938428in}}%
\pgfpathlineto{\pgfqpoint{0.737955in}{0.940652in}}%
\pgfpathlineto{\pgfqpoint{0.748969in}{0.972429in}}%
\pgfpathlineto{\pgfqpoint{0.759983in}{0.972497in}}%
\pgfpathlineto{\pgfqpoint{0.782012in}{0.991946in}}%
\pgfpathlineto{\pgfqpoint{0.804040in}{1.033970in}}%
\pgfpathlineto{\pgfqpoint{0.815055in}{1.025949in}}%
\pgfpathlineto{\pgfqpoint{0.826069in}{1.026120in}}%
\pgfpathlineto{\pgfqpoint{0.837083in}{1.020410in}}%
\pgfpathlineto{\pgfqpoint{0.848097in}{0.998078in}}%
\pgfpathlineto{\pgfqpoint{0.859112in}{0.984180in}}%
\pgfpathlineto{\pgfqpoint{0.870126in}{0.991478in}}%
\pgfpathlineto{\pgfqpoint{0.881140in}{0.989836in}}%
\pgfpathlineto{\pgfqpoint{0.892154in}{0.983162in}}%
\pgfpathlineto{\pgfqpoint{0.914183in}{0.982705in}}%
\pgfpathlineto{\pgfqpoint{0.925197in}{1.016373in}}%
\pgfpathlineto{\pgfqpoint{0.936211in}{1.027030in}}%
\pgfpathlineto{\pgfqpoint{0.947226in}{1.033256in}}%
\pgfpathlineto{\pgfqpoint{0.958240in}{1.009754in}}%
\pgfpathlineto{\pgfqpoint{0.969254in}{1.011141in}}%
\pgfpathlineto{\pgfqpoint{0.980268in}{1.007626in}}%
\pgfpathlineto{\pgfqpoint{0.991283in}{0.969965in}}%
\pgfpathlineto{\pgfqpoint{1.002297in}{0.947848in}}%
\pgfpathlineto{\pgfqpoint{1.013311in}{0.937024in}}%
\pgfpathlineto{\pgfqpoint{1.024325in}{0.943093in}}%
\pgfpathlineto{\pgfqpoint{1.035340in}{0.938637in}}%
\pgfpathlineto{\pgfqpoint{1.046354in}{0.945622in}}%
\pgfpathlineto{\pgfqpoint{1.057368in}{0.974604in}}%
\pgfpathlineto{\pgfqpoint{1.068382in}{0.984334in}}%
\pgfpathlineto{\pgfqpoint{1.090411in}{0.986972in}}%
\pgfpathlineto{\pgfqpoint{1.101425in}{1.013604in}}%
\pgfpathlineto{\pgfqpoint{1.112439in}{1.021157in}}%
\pgfpathlineto{\pgfqpoint{1.123454in}{1.032614in}}%
\pgfpathlineto{\pgfqpoint{1.134468in}{1.031321in}}%
\pgfpathlineto{\pgfqpoint{1.167511in}{1.054671in}}%
\pgfpathlineto{\pgfqpoint{1.178525in}{1.044824in}}%
\pgfpathlineto{\pgfqpoint{1.189539in}{1.018226in}}%
\pgfpathlineto{\pgfqpoint{1.200553in}{1.003171in}}%
\pgfpathlineto{\pgfqpoint{1.211568in}{0.976802in}}%
\pgfpathlineto{\pgfqpoint{1.222582in}{0.970740in}}%
\pgfpathlineto{\pgfqpoint{1.233596in}{1.034458in}}%
\pgfpathlineto{\pgfqpoint{1.244610in}{1.049536in}}%
\pgfpathlineto{\pgfqpoint{1.255625in}{1.040588in}}%
\pgfpathlineto{\pgfqpoint{1.266639in}{1.037673in}}%
\pgfpathlineto{\pgfqpoint{1.288667in}{1.013649in}}%
\pgfpathlineto{\pgfqpoint{1.299682in}{1.023629in}}%
\pgfpathlineto{\pgfqpoint{1.310696in}{0.988777in}}%
\pgfpathlineto{\pgfqpoint{1.321710in}{0.975249in}}%
\pgfpathlineto{\pgfqpoint{1.332725in}{0.953996in}}%
\pgfpathlineto{\pgfqpoint{1.343739in}{0.967049in}}%
\pgfpathlineto{\pgfqpoint{1.354753in}{1.009025in}}%
\pgfpathlineto{\pgfqpoint{1.365767in}{0.989099in}}%
\pgfpathlineto{\pgfqpoint{1.376782in}{1.003759in}}%
\pgfpathlineto{\pgfqpoint{1.398810in}{1.005296in}}%
\pgfpathlineto{\pgfqpoint{1.409824in}{1.030947in}}%
\pgfpathlineto{\pgfqpoint{1.420839in}{1.045006in}}%
\pgfpathlineto{\pgfqpoint{1.431853in}{1.054941in}}%
\pgfpathlineto{\pgfqpoint{1.442867in}{1.038914in}}%
\pgfpathlineto{\pgfqpoint{1.453881in}{1.040869in}}%
\pgfpathlineto{\pgfqpoint{1.464896in}{1.055869in}}%
\pgfpathlineto{\pgfqpoint{1.475910in}{1.072973in}}%
\pgfpathlineto{\pgfqpoint{1.486924in}{1.072112in}}%
\pgfpathlineto{\pgfqpoint{1.497938in}{1.061159in}}%
\pgfpathlineto{\pgfqpoint{1.519967in}{1.020281in}}%
\pgfpathlineto{\pgfqpoint{1.530981in}{1.005907in}}%
\pgfpathlineto{\pgfqpoint{1.541995in}{1.007699in}}%
\pgfpathlineto{\pgfqpoint{1.553010in}{0.996793in}}%
\pgfpathlineto{\pgfqpoint{1.564024in}{0.972631in}}%
\pgfpathlineto{\pgfqpoint{1.575038in}{0.952652in}}%
\pgfpathlineto{\pgfqpoint{1.586052in}{0.971502in}}%
\pgfpathlineto{\pgfqpoint{1.597067in}{0.973255in}}%
\pgfpathlineto{\pgfqpoint{1.608081in}{1.009907in}}%
\pgfpathlineto{\pgfqpoint{1.619095in}{0.990300in}}%
\pgfpathlineto{\pgfqpoint{1.630109in}{1.002009in}}%
\pgfpathlineto{\pgfqpoint{1.641124in}{1.000557in}}%
\pgfpathlineto{\pgfqpoint{1.652138in}{0.977416in}}%
\pgfpathlineto{\pgfqpoint{1.663152in}{0.993282in}}%
\pgfpathlineto{\pgfqpoint{1.674166in}{1.013253in}}%
\pgfpathlineto{\pgfqpoint{1.685181in}{1.018913in}}%
\pgfpathlineto{\pgfqpoint{1.696195in}{1.019218in}}%
\pgfpathlineto{\pgfqpoint{1.707209in}{1.015768in}}%
\pgfpathlineto{\pgfqpoint{1.718223in}{1.021942in}}%
\pgfpathlineto{\pgfqpoint{1.729238in}{1.013969in}}%
\pgfpathlineto{\pgfqpoint{1.740252in}{0.985972in}}%
\pgfpathlineto{\pgfqpoint{1.751266in}{0.924904in}}%
\pgfpathlineto{\pgfqpoint{1.762280in}{0.971783in}}%
\pgfpathlineto{\pgfqpoint{1.773295in}{0.949645in}}%
\pgfpathlineto{\pgfqpoint{1.784309in}{0.948032in}}%
\pgfpathlineto{\pgfqpoint{1.795323in}{0.960955in}}%
\pgfpathlineto{\pgfqpoint{1.806337in}{0.958982in}}%
\pgfpathlineto{\pgfqpoint{1.817352in}{0.950389in}}%
\pgfpathlineto{\pgfqpoint{1.828366in}{0.946851in}}%
\pgfpathlineto{\pgfqpoint{1.839380in}{0.979204in}}%
\pgfpathlineto{\pgfqpoint{1.850394in}{0.946060in}}%
\pgfpathlineto{\pgfqpoint{1.861409in}{0.938410in}}%
\pgfpathlineto{\pgfqpoint{1.872423in}{0.962387in}}%
\pgfpathlineto{\pgfqpoint{1.883437in}{0.960079in}}%
\pgfpathlineto{\pgfqpoint{1.894451in}{0.970836in}}%
\pgfpathlineto{\pgfqpoint{1.905466in}{0.977332in}}%
\pgfpathlineto{\pgfqpoint{1.916480in}{0.980164in}}%
\pgfpathlineto{\pgfqpoint{1.927494in}{0.992111in}}%
\pgfpathlineto{\pgfqpoint{1.938508in}{0.944403in}}%
\pgfpathlineto{\pgfqpoint{1.949523in}{0.969818in}}%
\pgfpathlineto{\pgfqpoint{1.960537in}{0.965867in}}%
\pgfpathlineto{\pgfqpoint{1.971551in}{1.027441in}}%
\pgfpathlineto{\pgfqpoint{1.982565in}{1.017021in}}%
\pgfpathlineto{\pgfqpoint{1.993580in}{1.027119in}}%
\pgfpathlineto{\pgfqpoint{2.004594in}{0.986744in}}%
\pgfpathlineto{\pgfqpoint{2.015608in}{0.956452in}}%
\pgfpathlineto{\pgfqpoint{2.037637in}{0.968330in}}%
\pgfpathlineto{\pgfqpoint{2.048651in}{0.990807in}}%
\pgfpathlineto{\pgfqpoint{2.070679in}{0.984030in}}%
\pgfpathlineto{\pgfqpoint{2.081694in}{0.998955in}}%
\pgfpathlineto{\pgfqpoint{2.092708in}{0.995704in}}%
\pgfpathlineto{\pgfqpoint{2.103722in}{1.016754in}}%
\pgfpathlineto{\pgfqpoint{2.114736in}{1.046818in}}%
\pgfpathlineto{\pgfqpoint{2.125751in}{1.050233in}}%
\pgfpathlineto{\pgfqpoint{2.136765in}{1.056871in}}%
\pgfpathlineto{\pgfqpoint{2.147779in}{1.058661in}}%
\pgfpathlineto{\pgfqpoint{2.158793in}{1.064506in}}%
\pgfpathlineto{\pgfqpoint{2.169808in}{1.022049in}}%
\pgfpathlineto{\pgfqpoint{2.180822in}{1.040523in}}%
\pgfpathlineto{\pgfqpoint{2.191836in}{1.065434in}}%
\pgfpathlineto{\pgfqpoint{2.202850in}{1.036499in}}%
\pgfpathlineto{\pgfqpoint{2.213865in}{1.017739in}}%
\pgfpathlineto{\pgfqpoint{2.224879in}{1.016598in}}%
\pgfpathlineto{\pgfqpoint{2.235893in}{1.020796in}}%
\pgfpathlineto{\pgfqpoint{2.246907in}{1.012521in}}%
\pgfpathlineto{\pgfqpoint{2.257922in}{1.010025in}}%
\pgfpathlineto{\pgfqpoint{2.268936in}{1.058099in}}%
\pgfpathlineto{\pgfqpoint{2.279950in}{1.097971in}}%
\pgfpathlineto{\pgfqpoint{2.301979in}{1.102595in}}%
\pgfpathlineto{\pgfqpoint{2.312993in}{1.096996in}}%
\pgfpathlineto{\pgfqpoint{2.324007in}{1.078867in}}%
\pgfpathlineto{\pgfqpoint{2.335021in}{1.079009in}}%
\pgfpathlineto{\pgfqpoint{2.346036in}{1.093807in}}%
\pgfpathlineto{\pgfqpoint{2.368064in}{1.101586in}}%
\pgfpathlineto{\pgfqpoint{2.379078in}{1.113764in}}%
\pgfpathlineto{\pgfqpoint{2.390093in}{1.111748in}}%
\pgfpathlineto{\pgfqpoint{2.401107in}{1.086635in}}%
\pgfpathlineto{\pgfqpoint{2.412121in}{1.090110in}}%
\pgfpathlineto{\pgfqpoint{2.423135in}{1.108646in}}%
\pgfpathlineto{\pgfqpoint{2.434150in}{1.105935in}}%
\pgfpathlineto{\pgfqpoint{2.445164in}{1.110390in}}%
\pgfpathlineto{\pgfqpoint{2.456178in}{1.087177in}}%
\pgfpathlineto{\pgfqpoint{2.467192in}{1.078141in}}%
\pgfpathlineto{\pgfqpoint{2.478207in}{1.078795in}}%
\pgfpathlineto{\pgfqpoint{2.489221in}{1.088341in}}%
\pgfpathlineto{\pgfqpoint{2.500235in}{1.076954in}}%
\pgfpathlineto{\pgfqpoint{2.511249in}{1.106520in}}%
\pgfpathlineto{\pgfqpoint{2.522264in}{1.107019in}}%
\pgfpathlineto{\pgfqpoint{2.533278in}{1.125628in}}%
\pgfpathlineto{\pgfqpoint{2.544292in}{1.138168in}}%
\pgfpathlineto{\pgfqpoint{2.555306in}{1.156261in}}%
\pgfpathlineto{\pgfqpoint{2.566321in}{1.153128in}}%
\pgfpathlineto{\pgfqpoint{2.577335in}{1.140569in}}%
\pgfpathlineto{\pgfqpoint{2.588349in}{1.132669in}}%
\pgfpathlineto{\pgfqpoint{2.610378in}{1.161191in}}%
\pgfpathlineto{\pgfqpoint{2.621392in}{1.168518in}}%
\pgfpathlineto{\pgfqpoint{2.632406in}{1.183398in}}%
\pgfpathlineto{\pgfqpoint{2.643421in}{1.160066in}}%
\pgfpathlineto{\pgfqpoint{2.654435in}{1.145770in}}%
\pgfpathlineto{\pgfqpoint{2.665449in}{1.138132in}}%
\pgfpathlineto{\pgfqpoint{2.676463in}{1.138656in}}%
\pgfpathlineto{\pgfqpoint{2.687478in}{1.156179in}}%
\pgfpathlineto{\pgfqpoint{2.698492in}{1.143556in}}%
\pgfpathlineto{\pgfqpoint{2.709506in}{1.152663in}}%
\pgfpathlineto{\pgfqpoint{2.720520in}{1.134513in}}%
\pgfpathlineto{\pgfqpoint{2.731535in}{1.139672in}}%
\pgfpathlineto{\pgfqpoint{2.742549in}{1.183597in}}%
\pgfpathlineto{\pgfqpoint{2.753563in}{1.196023in}}%
\pgfpathlineto{\pgfqpoint{2.764577in}{1.184400in}}%
\pgfpathlineto{\pgfqpoint{2.775592in}{1.166294in}}%
\pgfpathlineto{\pgfqpoint{2.786606in}{1.179868in}}%
\pgfpathlineto{\pgfqpoint{2.797620in}{1.157549in}}%
\pgfpathlineto{\pgfqpoint{2.808634in}{1.156601in}}%
\pgfpathlineto{\pgfqpoint{2.819649in}{1.135311in}}%
\pgfpathlineto{\pgfqpoint{2.830663in}{1.140808in}}%
\pgfpathlineto{\pgfqpoint{2.841677in}{1.139895in}}%
\pgfpathlineto{\pgfqpoint{2.852691in}{1.134406in}}%
\pgfpathlineto{\pgfqpoint{2.863706in}{1.133522in}}%
\pgfpathlineto{\pgfqpoint{2.874720in}{1.127391in}}%
\pgfpathlineto{\pgfqpoint{2.896748in}{1.129907in}}%
\pgfpathlineto{\pgfqpoint{2.918777in}{1.157734in}}%
\pgfpathlineto{\pgfqpoint{2.929791in}{1.186016in}}%
\pgfpathlineto{\pgfqpoint{2.940805in}{1.159749in}}%
\pgfpathlineto{\pgfqpoint{2.951820in}{1.179838in}}%
\pgfpathlineto{\pgfqpoint{2.962834in}{1.182772in}}%
\pgfpathlineto{\pgfqpoint{2.973848in}{1.163497in}}%
\pgfpathlineto{\pgfqpoint{2.984862in}{1.140088in}}%
\pgfpathlineto{\pgfqpoint{2.995877in}{1.151642in}}%
\pgfpathlineto{\pgfqpoint{3.006891in}{1.151956in}}%
\pgfpathlineto{\pgfqpoint{3.017905in}{1.137154in}}%
\pgfpathlineto{\pgfqpoint{3.028919in}{1.117013in}}%
\pgfpathlineto{\pgfqpoint{3.039934in}{1.111727in}}%
\pgfpathlineto{\pgfqpoint{3.050948in}{1.146510in}}%
\pgfpathlineto{\pgfqpoint{3.061962in}{1.169094in}}%
\pgfpathlineto{\pgfqpoint{3.072976in}{1.169156in}}%
\pgfpathlineto{\pgfqpoint{3.083991in}{1.160715in}}%
\pgfpathlineto{\pgfqpoint{3.095005in}{1.157507in}}%
\pgfpathlineto{\pgfqpoint{3.117033in}{1.148366in}}%
\pgfpathlineto{\pgfqpoint{3.128048in}{1.150598in}}%
\pgfpathlineto{\pgfqpoint{3.139062in}{1.163275in}}%
\pgfpathlineto{\pgfqpoint{3.150076in}{1.145655in}}%
\pgfpathlineto{\pgfqpoint{3.161090in}{1.137774in}}%
\pgfpathlineto{\pgfqpoint{3.172105in}{1.148566in}}%
\pgfpathlineto{\pgfqpoint{3.183119in}{1.139466in}}%
\pgfpathlineto{\pgfqpoint{3.194133in}{1.152051in}}%
\pgfpathlineto{\pgfqpoint{3.205147in}{1.153090in}}%
\pgfpathlineto{\pgfqpoint{3.216162in}{1.158978in}}%
\pgfpathlineto{\pgfqpoint{3.227176in}{1.141329in}}%
\pgfpathlineto{\pgfqpoint{3.238190in}{1.136137in}}%
\pgfpathlineto{\pgfqpoint{3.249204in}{1.133644in}}%
\pgfpathlineto{\pgfqpoint{3.260219in}{1.118471in}}%
\pgfpathlineto{\pgfqpoint{3.271233in}{1.128154in}}%
\pgfpathlineto{\pgfqpoint{3.282247in}{1.114096in}}%
\pgfpathlineto{\pgfqpoint{3.293261in}{1.138532in}}%
\pgfpathlineto{\pgfqpoint{3.304276in}{1.123918in}}%
\pgfpathlineto{\pgfqpoint{3.315290in}{1.125987in}}%
\pgfpathlineto{\pgfqpoint{3.326304in}{1.085724in}}%
\pgfpathlineto{\pgfqpoint{3.337318in}{1.078693in}}%
\pgfpathlineto{\pgfqpoint{3.348333in}{1.073854in}}%
\pgfpathlineto{\pgfqpoint{3.359347in}{1.081844in}}%
\pgfpathlineto{\pgfqpoint{3.370361in}{1.069019in}}%
\pgfpathlineto{\pgfqpoint{3.381375in}{1.073209in}}%
\pgfpathlineto{\pgfqpoint{3.392390in}{1.060736in}}%
\pgfpathlineto{\pgfqpoint{3.403404in}{1.061881in}}%
\pgfpathlineto{\pgfqpoint{3.414418in}{1.068968in}}%
\pgfpathlineto{\pgfqpoint{3.425432in}{1.047976in}}%
\pgfpathlineto{\pgfqpoint{3.436447in}{1.063743in}}%
\pgfpathlineto{\pgfqpoint{3.447461in}{1.074199in}}%
\pgfpathlineto{\pgfqpoint{3.458475in}{1.077731in}}%
\pgfpathlineto{\pgfqpoint{3.469489in}{1.067784in}}%
\pgfpathlineto{\pgfqpoint{3.480504in}{1.064662in}}%
\pgfpathlineto{\pgfqpoint{3.491518in}{1.066859in}}%
\pgfpathlineto{\pgfqpoint{3.502532in}{1.073409in}}%
\pgfpathlineto{\pgfqpoint{3.513546in}{1.106582in}}%
\pgfpathlineto{\pgfqpoint{3.524561in}{1.086416in}}%
\pgfpathlineto{\pgfqpoint{3.535575in}{1.079720in}}%
\pgfpathlineto{\pgfqpoint{3.546589in}{1.080549in}}%
\pgfpathlineto{\pgfqpoint{3.557603in}{1.049106in}}%
\pgfpathlineto{\pgfqpoint{3.568618in}{1.032230in}}%
\pgfpathlineto{\pgfqpoint{3.579632in}{1.008958in}}%
\pgfpathlineto{\pgfqpoint{3.590646in}{1.013008in}}%
\pgfpathlineto{\pgfqpoint{3.601660in}{1.002967in}}%
\pgfpathlineto{\pgfqpoint{3.623689in}{1.014776in}}%
\pgfpathlineto{\pgfqpoint{3.634703in}{1.032820in}}%
\pgfpathlineto{\pgfqpoint{3.645717in}{1.042462in}}%
\pgfpathlineto{\pgfqpoint{3.656732in}{1.036620in}}%
\pgfpathlineto{\pgfqpoint{3.667746in}{1.037207in}}%
\pgfpathlineto{\pgfqpoint{3.678760in}{1.033948in}}%
\pgfpathlineto{\pgfqpoint{3.678760in}{1.033948in}}%
\pgfusepath{stroke}%
\end{pgfscope}%
\begin{pgfscope}%
\pgfsetrectcap%
\pgfsetmiterjoin%
\pgfsetlinewidth{0.752812pt}%
\definecolor{currentstroke}{rgb}{0.000000,0.000000,0.000000}%
\pgfsetstrokecolor{currentstroke}%
\pgfsetdash{}{0pt}%
\pgfpathmoveto{\pgfqpoint{0.550713in}{0.408431in}}%
\pgfpathlineto{\pgfqpoint{0.550713in}{1.899204in}}%
\pgfusepath{stroke}%
\end{pgfscope}%
\begin{pgfscope}%
\pgfsetrectcap%
\pgfsetmiterjoin%
\pgfsetlinewidth{0.752812pt}%
\definecolor{currentstroke}{rgb}{0.000000,0.000000,0.000000}%
\pgfsetstrokecolor{currentstroke}%
\pgfsetdash{}{0pt}%
\pgfpathmoveto{\pgfqpoint{3.689774in}{0.408431in}}%
\pgfpathlineto{\pgfqpoint{3.689774in}{1.899204in}}%
\pgfusepath{stroke}%
\end{pgfscope}%
\begin{pgfscope}%
\pgfsetrectcap%
\pgfsetmiterjoin%
\pgfsetlinewidth{0.752812pt}%
\definecolor{currentstroke}{rgb}{0.000000,0.000000,0.000000}%
\pgfsetstrokecolor{currentstroke}%
\pgfsetdash{}{0pt}%
\pgfpathmoveto{\pgfqpoint{0.550713in}{0.408431in}}%
\pgfpathlineto{\pgfqpoint{3.689774in}{0.408431in}}%
\pgfusepath{stroke}%
\end{pgfscope}%
\begin{pgfscope}%
\pgfsetrectcap%
\pgfsetmiterjoin%
\pgfsetlinewidth{0.752812pt}%
\definecolor{currentstroke}{rgb}{0.000000,0.000000,0.000000}%
\pgfsetstrokecolor{currentstroke}%
\pgfsetdash{}{0pt}%
\pgfpathmoveto{\pgfqpoint{0.550713in}{1.899204in}}%
\pgfpathlineto{\pgfqpoint{3.689774in}{1.899204in}}%
\pgfusepath{stroke}%
\end{pgfscope}%
\begin{pgfscope}%
\pgfsetbuttcap%
\pgfsetmiterjoin%
\definecolor{currentfill}{rgb}{1.000000,1.000000,1.000000}%
\pgfsetfillcolor{currentfill}%
\pgfsetlinewidth{1.003750pt}%
\definecolor{currentstroke}{rgb}{1.000000,1.000000,1.000000}%
\pgfsetstrokecolor{currentstroke}%
\pgfsetdash{}{0pt}%
\pgfpathmoveto{\pgfqpoint{2.681127in}{0.436209in}}%
\pgfpathlineto{\pgfqpoint{3.634219in}{0.436209in}}%
\pgfpathquadraticcurveto{\pgfqpoint{3.661997in}{0.436209in}}{\pgfqpoint{3.661997in}{0.463986in}}%
\pgfpathlineto{\pgfqpoint{3.661997in}{0.837443in}}%
\pgfpathquadraticcurveto{\pgfqpoint{3.661997in}{0.865221in}}{\pgfqpoint{3.634219in}{0.865221in}}%
\pgfpathlineto{\pgfqpoint{2.681127in}{0.865221in}}%
\pgfpathquadraticcurveto{\pgfqpoint{2.653349in}{0.865221in}}{\pgfqpoint{2.653349in}{0.837443in}}%
\pgfpathlineto{\pgfqpoint{2.653349in}{0.463986in}}%
\pgfpathquadraticcurveto{\pgfqpoint{2.653349in}{0.436209in}}{\pgfqpoint{2.681127in}{0.436209in}}%
\pgfpathclose%
\pgfusepath{stroke,fill}%
\end{pgfscope}%
\begin{pgfscope}%
\pgfsetrectcap%
\pgfsetroundjoin%
\pgfsetlinewidth{0.853187pt}%
\definecolor{currentstroke}{rgb}{0.392157,0.396078,0.403922}%
\pgfsetstrokecolor{currentstroke}%
\pgfsetdash{}{0pt}%
\pgfpathmoveto{\pgfqpoint{2.708904in}{0.761054in}}%
\pgfpathlineto{\pgfqpoint{2.986682in}{0.761054in}}%
\pgfusepath{stroke}%
\end{pgfscope}%
\begin{pgfscope}%
\definecolor{textcolor}{rgb}{0.000000,0.000000,0.000000}%
\pgfsetstrokecolor{textcolor}%
\pgfsetfillcolor{textcolor}%
\pgftext[x=3.097793in,y=0.712443in,left,base]{\color{textcolor}\rmfamily\fontsize{10.000000}{12.000000}\selectfont \(\displaystyle \mu_0=0\)}%
\end{pgfscope}%
\begin{pgfscope}%
\pgfsetbuttcap%
\pgfsetroundjoin%
\pgfsetlinewidth{0.853187pt}%
\definecolor{currentstroke}{rgb}{0.392157,0.396078,0.403922}%
\pgfsetstrokecolor{currentstroke}%
\pgfsetdash{{0.850000pt}{1.402500pt}}{0.000000pt}%
\pgfpathmoveto{\pgfqpoint{2.708904in}{0.567381in}}%
\pgfpathlineto{\pgfqpoint{2.986682in}{0.567381in}}%
\pgfusepath{stroke}%
\end{pgfscope}%
\begin{pgfscope}%
\definecolor{textcolor}{rgb}{0.000000,0.000000,0.000000}%
\pgfsetstrokecolor{textcolor}%
\pgfsetfillcolor{textcolor}%
\pgftext[x=3.097793in,y=0.518770in,left,base]{\color{textcolor}\rmfamily\fontsize{10.000000}{12.000000}\selectfont \(\displaystyle \mu_0=-2\)}%
\end{pgfscope}%
\begin{pgfscope}%
\pgfsetrectcap%
\pgfsetroundjoin%
\pgfsetlinewidth{1.003750pt}%
\definecolor{currentstroke}{rgb}{0.392157,0.396078,0.403922}%
\pgfsetstrokecolor{currentstroke}%
\pgfsetdash{}{0pt}%
\pgfpathmoveto{\pgfqpoint{3.877556in}{1.964834in}}%
\pgfpathlineto{\pgfqpoint{4.155333in}{1.964834in}}%
\pgfusepath{stroke}%
\end{pgfscope}%
\begin{pgfscope}%
\definecolor{textcolor}{rgb}{0.000000,0.000000,0.000000}%
\pgfsetstrokecolor{textcolor}%
\pgfsetfillcolor{textcolor}%
\pgftext[x=4.266445in,y=1.916222in,left,base]{\color{textcolor}\rmfamily\fontsize{10.000000}{12.000000}\selectfont True Lengthscale}%
\end{pgfscope}%
\begin{pgfscope}%
\pgfsetbuttcap%
\pgfsetroundjoin%
\pgfsetlinewidth{1.003750pt}%
\definecolor{currentstroke}{rgb}{0.392157,0.396078,0.403922}%
\pgfsetstrokecolor{currentstroke}%
\pgfsetdash{{3.700000pt}{1.600000pt}}{0.000000pt}%
\pgfpathmoveto{\pgfqpoint{3.877556in}{1.769556in}}%
\pgfpathlineto{\pgfqpoint{4.155333in}{1.769556in}}%
\pgfusepath{stroke}%
\end{pgfscope}%
\begin{pgfscope}%
\definecolor{textcolor}{rgb}{0.000000,0.000000,0.000000}%
\pgfsetstrokecolor{textcolor}%
\pgfsetfillcolor{textcolor}%
\pgftext[x=4.266445in,y=1.720945in,left,base]{\color{textcolor}\rmfamily\fontsize{10.000000}{12.000000}\selectfont Upper Bound}%
\end{pgfscope}%
\begin{pgfscope}%
\pgfsetrectcap%
\pgfsetroundjoin%
\pgfsetlinewidth{1.003750pt}%
\definecolor{currentstroke}{rgb}{0.631373,0.062745,0.207843}%
\pgfsetstrokecolor{currentstroke}%
\pgfsetdash{}{0pt}%
\pgfpathmoveto{\pgfqpoint{3.877556in}{1.575945in}}%
\pgfpathlineto{\pgfqpoint{4.155333in}{1.575945in}}%
\pgfusepath{stroke}%
\end{pgfscope}%
\begin{pgfscope}%
\definecolor{textcolor}{rgb}{0.000000,0.000000,0.000000}%
\pgfsetstrokecolor{textcolor}%
\pgfsetfillcolor{textcolor}%
\pgftext[x=4.266445in,y=1.527334in,left,base]{\color{textcolor}\rmfamily\fontsize{10.000000}{12.000000}\selectfont TV-GP-UCB}%
\end{pgfscope}%
\begin{pgfscope}%
\pgfsetrectcap%
\pgfsetroundjoin%
\pgfsetlinewidth{1.003750pt}%
\definecolor{currentstroke}{rgb}{0.890196,0.000000,0.400000}%
\pgfsetstrokecolor{currentstroke}%
\pgfsetdash{}{0pt}%
\pgfpathmoveto{\pgfqpoint{3.877556in}{1.382334in}}%
\pgfpathlineto{\pgfqpoint{4.155333in}{1.382334in}}%
\pgfusepath{stroke}%
\end{pgfscope}%
\begin{pgfscope}%
\definecolor{textcolor}{rgb}{0.000000,0.000000,0.000000}%
\pgfsetstrokecolor{textcolor}%
\pgfsetfillcolor{textcolor}%
\pgftext[x=4.266445in,y=1.333723in,left,base]{\color{textcolor}\rmfamily\fontsize{10.000000}{12.000000}\selectfont SW TV-GP-UCB}%
\end{pgfscope}%
\begin{pgfscope}%
\pgfsetrectcap%
\pgfsetroundjoin%
\pgfsetlinewidth{1.003750pt}%
\definecolor{currentstroke}{rgb}{0.341176,0.670588,0.152941}%
\pgfsetstrokecolor{currentstroke}%
\pgfsetdash{}{0pt}%
\pgfpathmoveto{\pgfqpoint{3.877556in}{1.188723in}}%
\pgfpathlineto{\pgfqpoint{4.155333in}{1.188723in}}%
\pgfusepath{stroke}%
\end{pgfscope}%
\begin{pgfscope}%
\definecolor{textcolor}{rgb}{0.000000,0.000000,0.000000}%
\pgfsetstrokecolor{textcolor}%
\pgfsetfillcolor{textcolor}%
\pgftext[x=4.266445in,y=1.140112in,left,base]{\color{textcolor}\rmfamily\fontsize{10.000000}{12.000000}\selectfont UI-TVBO}%
\end{pgfscope}%
\begin{pgfscope}%
\pgfsetrectcap%
\pgfsetroundjoin%
\pgfsetlinewidth{1.003750pt}%
\definecolor{currentstroke}{rgb}{0.000000,0.380392,0.396078}%
\pgfsetstrokecolor{currentstroke}%
\pgfsetdash{}{0pt}%
\pgfpathmoveto{\pgfqpoint{3.877556in}{0.995112in}}%
\pgfpathlineto{\pgfqpoint{4.155333in}{0.995112in}}%
\pgfusepath{stroke}%
\end{pgfscope}%
\begin{pgfscope}%
\definecolor{textcolor}{rgb}{0.000000,0.000000,0.000000}%
\pgfsetstrokecolor{textcolor}%
\pgfsetfillcolor{textcolor}%
\pgftext[x=4.266445in,y=0.946501in,left,base]{\color{textcolor}\rmfamily\fontsize{10.000000}{12.000000}\selectfont B UI-TVBO}%
\end{pgfscope}%
\begin{pgfscope}%
\pgfsetrectcap%
\pgfsetroundjoin%
\pgfsetlinewidth{1.003750pt}%
\definecolor{currentstroke}{rgb}{0.380392,0.129412,0.345098}%
\pgfsetstrokecolor{currentstroke}%
\pgfsetdash{}{0pt}%
\pgfpathmoveto{\pgfqpoint{3.877556in}{0.801501in}}%
\pgfpathlineto{\pgfqpoint{4.155333in}{0.801501in}}%
\pgfusepath{stroke}%
\end{pgfscope}%
\begin{pgfscope}%
\definecolor{textcolor}{rgb}{0.000000,0.000000,0.000000}%
\pgfsetstrokecolor{textcolor}%
\pgfsetfillcolor{textcolor}%
\pgftext[x=4.266445in,y=0.752890in,left,base]{\color{textcolor}\rmfamily\fontsize{10.000000}{12.000000}\selectfont C-TV-GP-UCB}%
\end{pgfscope}%
\begin{pgfscope}%
\pgfsetrectcap%
\pgfsetroundjoin%
\pgfsetlinewidth{1.003750pt}%
\definecolor{currentstroke}{rgb}{0.964706,0.658824,0.000000}%
\pgfsetstrokecolor{currentstroke}%
\pgfsetdash{}{0pt}%
\pgfpathmoveto{\pgfqpoint{3.877556in}{0.607890in}}%
\pgfpathlineto{\pgfqpoint{4.155333in}{0.607890in}}%
\pgfusepath{stroke}%
\end{pgfscope}%
\begin{pgfscope}%
\definecolor{textcolor}{rgb}{0.000000,0.000000,0.000000}%
\pgfsetstrokecolor{textcolor}%
\pgfsetfillcolor{textcolor}%
\pgftext[x=4.266445in,y=0.559279in,left,base]{\color{textcolor}\rmfamily\fontsize{10.000000}{12.000000}\selectfont SW C-TV-GP-UCB}%
\end{pgfscope}%
\begin{pgfscope}%
\pgfsetrectcap%
\pgfsetroundjoin%
\pgfsetlinewidth{1.003750pt}%
\definecolor{currentstroke}{rgb}{0.000000,0.329412,0.623529}%
\pgfsetstrokecolor{currentstroke}%
\pgfsetdash{}{0pt}%
\pgfpathmoveto{\pgfqpoint{3.877556in}{0.414279in}}%
\pgfpathlineto{\pgfqpoint{4.155333in}{0.414279in}}%
\pgfusepath{stroke}%
\end{pgfscope}%
\begin{pgfscope}%
\definecolor{textcolor}{rgb}{0.000000,0.000000,0.000000}%
\pgfsetstrokecolor{textcolor}%
\pgfsetfillcolor{textcolor}%
\pgftext[x=4.266445in,y=0.365668in,left,base]{\color{textcolor}\rmfamily\fontsize{10.000000}{12.000000}\selectfont C-UI-TVBO}%
\end{pgfscope}%
\begin{pgfscope}%
\pgfsetrectcap%
\pgfsetroundjoin%
\pgfsetlinewidth{1.003750pt}%
\definecolor{currentstroke}{rgb}{0.478431,0.435294,0.674510}%
\pgfsetstrokecolor{currentstroke}%
\pgfsetdash{}{0pt}%
\pgfpathmoveto{\pgfqpoint{3.877556in}{0.220668in}}%
\pgfpathlineto{\pgfqpoint{4.155333in}{0.220668in}}%
\pgfusepath{stroke}%
\end{pgfscope}%
\begin{pgfscope}%
\definecolor{textcolor}{rgb}{0.000000,0.000000,0.000000}%
\pgfsetstrokecolor{textcolor}%
\pgfsetfillcolor{textcolor}%
\pgftext[x=4.266445in,y=0.172057in,left,base]{\color{textcolor}\rmfamily\fontsize{10.000000}{12.000000}\selectfont B C-UI-TVBO}%
\end{pgfscope}%
\begin{pgfscope}%
\pgfsetrectcap%
\pgfsetroundjoin%
\pgfsetlinewidth{1.003750pt}%
\definecolor{currentstroke}{rgb}{0.392157,0.396078,0.403922}%
\pgfsetstrokecolor{currentstroke}%
\pgfsetdash{}{0pt}%
\pgfpathmoveto{\pgfqpoint{3.877556in}{1.964834in}}%
\pgfpathlineto{\pgfqpoint{4.155333in}{1.964834in}}%
\pgfusepath{stroke}%
\end{pgfscope}%
\begin{pgfscope}%
\definecolor{textcolor}{rgb}{0.000000,0.000000,0.000000}%
\pgfsetstrokecolor{textcolor}%
\pgfsetfillcolor{textcolor}%
\pgftext[x=4.266445in,y=1.916222in,left,base]{\color{textcolor}\rmfamily\fontsize{10.000000}{12.000000}\selectfont True Lengthscale}%
\end{pgfscope}%
\begin{pgfscope}%
\pgfsetbuttcap%
\pgfsetroundjoin%
\pgfsetlinewidth{1.003750pt}%
\definecolor{currentstroke}{rgb}{0.392157,0.396078,0.403922}%
\pgfsetstrokecolor{currentstroke}%
\pgfsetdash{{3.700000pt}{1.600000pt}}{0.000000pt}%
\pgfpathmoveto{\pgfqpoint{3.877556in}{1.769556in}}%
\pgfpathlineto{\pgfqpoint{4.155333in}{1.769556in}}%
\pgfusepath{stroke}%
\end{pgfscope}%
\begin{pgfscope}%
\definecolor{textcolor}{rgb}{0.000000,0.000000,0.000000}%
\pgfsetstrokecolor{textcolor}%
\pgfsetfillcolor{textcolor}%
\pgftext[x=4.266445in,y=1.720945in,left,base]{\color{textcolor}\rmfamily\fontsize{10.000000}{12.000000}\selectfont Upper Bound}%
\end{pgfscope}%
\begin{pgfscope}%
\pgfsetrectcap%
\pgfsetroundjoin%
\pgfsetlinewidth{1.003750pt}%
\definecolor{currentstroke}{rgb}{0.631373,0.062745,0.207843}%
\pgfsetstrokecolor{currentstroke}%
\pgfsetdash{}{0pt}%
\pgfpathmoveto{\pgfqpoint{3.877556in}{1.575945in}}%
\pgfpathlineto{\pgfqpoint{4.155333in}{1.575945in}}%
\pgfusepath{stroke}%
\end{pgfscope}%
\begin{pgfscope}%
\definecolor{textcolor}{rgb}{0.000000,0.000000,0.000000}%
\pgfsetstrokecolor{textcolor}%
\pgfsetfillcolor{textcolor}%
\pgftext[x=4.266445in,y=1.527334in,left,base]{\color{textcolor}\rmfamily\fontsize{10.000000}{12.000000}\selectfont TV-GP-UCB}%
\end{pgfscope}%
\begin{pgfscope}%
\pgfsetrectcap%
\pgfsetroundjoin%
\pgfsetlinewidth{1.003750pt}%
\definecolor{currentstroke}{rgb}{0.890196,0.000000,0.400000}%
\pgfsetstrokecolor{currentstroke}%
\pgfsetdash{}{0pt}%
\pgfpathmoveto{\pgfqpoint{3.877556in}{1.382334in}}%
\pgfpathlineto{\pgfqpoint{4.155333in}{1.382334in}}%
\pgfusepath{stroke}%
\end{pgfscope}%
\begin{pgfscope}%
\definecolor{textcolor}{rgb}{0.000000,0.000000,0.000000}%
\pgfsetstrokecolor{textcolor}%
\pgfsetfillcolor{textcolor}%
\pgftext[x=4.266445in,y=1.333723in,left,base]{\color{textcolor}\rmfamily\fontsize{10.000000}{12.000000}\selectfont SW TV-GP-UCB}%
\end{pgfscope}%
\begin{pgfscope}%
\pgfsetrectcap%
\pgfsetroundjoin%
\pgfsetlinewidth{1.003750pt}%
\definecolor{currentstroke}{rgb}{0.341176,0.670588,0.152941}%
\pgfsetstrokecolor{currentstroke}%
\pgfsetdash{}{0pt}%
\pgfpathmoveto{\pgfqpoint{3.877556in}{1.188723in}}%
\pgfpathlineto{\pgfqpoint{4.155333in}{1.188723in}}%
\pgfusepath{stroke}%
\end{pgfscope}%
\begin{pgfscope}%
\definecolor{textcolor}{rgb}{0.000000,0.000000,0.000000}%
\pgfsetstrokecolor{textcolor}%
\pgfsetfillcolor{textcolor}%
\pgftext[x=4.266445in,y=1.140112in,left,base]{\color{textcolor}\rmfamily\fontsize{10.000000}{12.000000}\selectfont UI-TVBO}%
\end{pgfscope}%
\begin{pgfscope}%
\pgfsetrectcap%
\pgfsetroundjoin%
\pgfsetlinewidth{1.003750pt}%
\definecolor{currentstroke}{rgb}{0.000000,0.380392,0.396078}%
\pgfsetstrokecolor{currentstroke}%
\pgfsetdash{}{0pt}%
\pgfpathmoveto{\pgfqpoint{3.877556in}{0.995112in}}%
\pgfpathlineto{\pgfqpoint{4.155333in}{0.995112in}}%
\pgfusepath{stroke}%
\end{pgfscope}%
\begin{pgfscope}%
\definecolor{textcolor}{rgb}{0.000000,0.000000,0.000000}%
\pgfsetstrokecolor{textcolor}%
\pgfsetfillcolor{textcolor}%
\pgftext[x=4.266445in,y=0.946501in,left,base]{\color{textcolor}\rmfamily\fontsize{10.000000}{12.000000}\selectfont B UI-TVBO}%
\end{pgfscope}%
\begin{pgfscope}%
\pgfsetrectcap%
\pgfsetroundjoin%
\pgfsetlinewidth{1.003750pt}%
\definecolor{currentstroke}{rgb}{0.380392,0.129412,0.345098}%
\pgfsetstrokecolor{currentstroke}%
\pgfsetdash{}{0pt}%
\pgfpathmoveto{\pgfqpoint{3.877556in}{0.801501in}}%
\pgfpathlineto{\pgfqpoint{4.155333in}{0.801501in}}%
\pgfusepath{stroke}%
\end{pgfscope}%
\begin{pgfscope}%
\definecolor{textcolor}{rgb}{0.000000,0.000000,0.000000}%
\pgfsetstrokecolor{textcolor}%
\pgfsetfillcolor{textcolor}%
\pgftext[x=4.266445in,y=0.752890in,left,base]{\color{textcolor}\rmfamily\fontsize{10.000000}{12.000000}\selectfont C-TV-GP-UCB}%
\end{pgfscope}%
\begin{pgfscope}%
\pgfsetrectcap%
\pgfsetroundjoin%
\pgfsetlinewidth{1.003750pt}%
\definecolor{currentstroke}{rgb}{0.964706,0.658824,0.000000}%
\pgfsetstrokecolor{currentstroke}%
\pgfsetdash{}{0pt}%
\pgfpathmoveto{\pgfqpoint{3.877556in}{0.607890in}}%
\pgfpathlineto{\pgfqpoint{4.155333in}{0.607890in}}%
\pgfusepath{stroke}%
\end{pgfscope}%
\begin{pgfscope}%
\definecolor{textcolor}{rgb}{0.000000,0.000000,0.000000}%
\pgfsetstrokecolor{textcolor}%
\pgfsetfillcolor{textcolor}%
\pgftext[x=4.266445in,y=0.559279in,left,base]{\color{textcolor}\rmfamily\fontsize{10.000000}{12.000000}\selectfont SW C-TV-GP-UCB}%
\end{pgfscope}%
\begin{pgfscope}%
\pgfsetrectcap%
\pgfsetroundjoin%
\pgfsetlinewidth{1.003750pt}%
\definecolor{currentstroke}{rgb}{0.000000,0.329412,0.623529}%
\pgfsetstrokecolor{currentstroke}%
\pgfsetdash{}{0pt}%
\pgfpathmoveto{\pgfqpoint{3.877556in}{0.414279in}}%
\pgfpathlineto{\pgfqpoint{4.155333in}{0.414279in}}%
\pgfusepath{stroke}%
\end{pgfscope}%
\begin{pgfscope}%
\definecolor{textcolor}{rgb}{0.000000,0.000000,0.000000}%
\pgfsetstrokecolor{textcolor}%
\pgfsetfillcolor{textcolor}%
\pgftext[x=4.266445in,y=0.365668in,left,base]{\color{textcolor}\rmfamily\fontsize{10.000000}{12.000000}\selectfont C-UI-TVBO}%
\end{pgfscope}%
\begin{pgfscope}%
\pgfsetrectcap%
\pgfsetroundjoin%
\pgfsetlinewidth{1.003750pt}%
\definecolor{currentstroke}{rgb}{0.478431,0.435294,0.674510}%
\pgfsetstrokecolor{currentstroke}%
\pgfsetdash{}{0pt}%
\pgfpathmoveto{\pgfqpoint{3.877556in}{0.220668in}}%
\pgfpathlineto{\pgfqpoint{4.155333in}{0.220668in}}%
\pgfusepath{stroke}%
\end{pgfscope}%
\begin{pgfscope}%
\definecolor{textcolor}{rgb}{0.000000,0.000000,0.000000}%
\pgfsetstrokecolor{textcolor}%
\pgfsetfillcolor{textcolor}%
\pgftext[x=4.266445in,y=0.172057in,left,base]{\color{textcolor}\rmfamily\fontsize{10.000000}{12.000000}\selectfont B C-UI-TVBO}%
\end{pgfscope}%
\end{pgfpicture}%
\makeatother%
\endgroup%

    \caption[Learning the length scales in out-of-model comparison.]{Mean learned length scales of the out-of-model comparison. The dotted lines show the length scales learned with an optimistic prior mean whereas the solid lines show the length scales with a well-defined prior mean.}
    \label{fig:OOMC_lengthscales_1D}
\end{figure}

The length scales of the \gls{b2p} forgetting  variations without data selection strategy quickly reach the upper bound of the length scale, both for \gls{ctvbo} and standard \gls{tvbo}. This is a direct consequence of the forgetting strategy, as the expectation of a measurement in \gls{b2p} forgetting propagates to the prior mean over time. This shows that learning the length scale can be more difficult by using \gls{b2p} forgetting since it tends to overestimate the length scale. For very steep functions, this can result in increased regret. In contrast, the length scale for \gls{ui} forgetting without data selection strategy only increases gradually. 

For both, \gls{b2p} and \gls{ui} forgetting, using a data selection strategy results in a better estimate of the length scale as stale data is discarded and not considered in the hyperparameter optimization. However, this does not directly lead to a decrease in regret (Figure~\ref{fig:OOMC_cumulative_regret_1D}). The upper bound for the length scale could be increased to investigate this further.

The results of the two-dimensional within-model comparisons are shown in Figure~\ref{fig:OOMC_cumulative_regret_2D}. Again, the combination of both proposed variations C-\gls{uitvbo} is the one with the lowest regret, even though the unconstrained \gls{b2p} forgetting methods are very similar.
Here, the trend that constraining the posterior for \gls{b2p} forgetting can become problematic if the objective function is flat is even more evident.
\begin{figure}[h]
    \centering
    %% Creator: Matplotlib, PGF backend
%%
%% To include the figure in your LaTeX document, write
%%   \input{<filename>.pgf}
%%
%% Make sure the required packages are loaded in your preamble
%%   \usepackage{pgf}
%%
%% Figures using additional raster images can only be included by \input if
%% they are in the same directory as the main LaTeX file. For loading figures
%% from other directories you can use the `import` package
%%   \usepackage{import}
%%
%% and then include the figures with
%%   \import{<path to file>}{<filename>.pgf}
%%
%% Matplotlib used the following preamble
%%   \usepackage{fontspec}
%%
\begingroup%
\makeatletter%
\begin{pgfpicture}%
\pgfpathrectangle{\pgfpointorigin}{\pgfqpoint{5.507126in}{2.042155in}}%
\pgfusepath{use as bounding box, clip}%
\begin{pgfscope}%
\pgfsetbuttcap%
\pgfsetmiterjoin%
\definecolor{currentfill}{rgb}{1.000000,1.000000,1.000000}%
\pgfsetfillcolor{currentfill}%
\pgfsetlinewidth{0.000000pt}%
\definecolor{currentstroke}{rgb}{1.000000,1.000000,1.000000}%
\pgfsetstrokecolor{currentstroke}%
\pgfsetdash{}{0pt}%
\pgfpathmoveto{\pgfqpoint{0.000000in}{0.000000in}}%
\pgfpathlineto{\pgfqpoint{5.507126in}{0.000000in}}%
\pgfpathlineto{\pgfqpoint{5.507126in}{2.042155in}}%
\pgfpathlineto{\pgfqpoint{0.000000in}{2.042155in}}%
\pgfpathclose%
\pgfusepath{fill}%
\end{pgfscope}%
\begin{pgfscope}%
\pgfsetbuttcap%
\pgfsetmiterjoin%
\definecolor{currentfill}{rgb}{1.000000,1.000000,1.000000}%
\pgfsetfillcolor{currentfill}%
\pgfsetlinewidth{0.000000pt}%
\definecolor{currentstroke}{rgb}{0.000000,0.000000,0.000000}%
\pgfsetstrokecolor{currentstroke}%
\pgfsetstrokeopacity{0.000000}%
\pgfsetdash{}{0pt}%
\pgfpathmoveto{\pgfqpoint{0.550713in}{0.102108in}}%
\pgfpathlineto{\pgfqpoint{3.744846in}{0.102108in}}%
\pgfpathlineto{\pgfqpoint{3.744846in}{1.940047in}}%
\pgfpathlineto{\pgfqpoint{0.550713in}{1.940047in}}%
\pgfpathclose%
\pgfusepath{fill}%
\end{pgfscope}%
\begin{pgfscope}%
\pgfpathrectangle{\pgfqpoint{0.550713in}{0.102108in}}{\pgfqpoint{3.194133in}{1.837939in}}%
\pgfusepath{clip}%
\pgfsetbuttcap%
\pgfsetmiterjoin%
\definecolor{currentfill}{rgb}{0.631373,0.062745,0.207843}%
\pgfsetfillcolor{currentfill}%
\pgfsetlinewidth{0.752812pt}%
\definecolor{currentstroke}{rgb}{0.000000,0.000000,0.000000}%
\pgfsetstrokecolor{currentstroke}%
\pgfsetdash{}{0pt}%
\pgfpathmoveto{\pgfqpoint{0.592236in}{0.815833in}}%
\pgfpathlineto{\pgfqpoint{0.748749in}{0.815833in}}%
\pgfpathlineto{\pgfqpoint{0.748749in}{0.959334in}}%
\pgfpathlineto{\pgfqpoint{0.592236in}{0.959334in}}%
\pgfpathlineto{\pgfqpoint{0.592236in}{0.815833in}}%
\pgfpathclose%
\pgfusepath{stroke,fill}%
\end{pgfscope}%
\begin{pgfscope}%
\pgfpathrectangle{\pgfqpoint{0.550713in}{0.102108in}}{\pgfqpoint{3.194133in}{1.837939in}}%
\pgfusepath{clip}%
\pgfsetbuttcap%
\pgfsetmiterjoin%
\definecolor{currentfill}{rgb}{0.898039,0.772549,0.752941}%
\pgfsetfillcolor{currentfill}%
\pgfsetlinewidth{0.752812pt}%
\definecolor{currentstroke}{rgb}{0.000000,0.000000,0.000000}%
\pgfsetstrokecolor{currentstroke}%
\pgfsetdash{}{0pt}%
\pgfpathmoveto{\pgfqpoint{0.751943in}{0.824861in}}%
\pgfpathlineto{\pgfqpoint{0.908456in}{0.824861in}}%
\pgfpathlineto{\pgfqpoint{0.908456in}{1.032304in}}%
\pgfpathlineto{\pgfqpoint{0.751943in}{1.032304in}}%
\pgfpathlineto{\pgfqpoint{0.751943in}{0.824861in}}%
\pgfpathclose%
\pgfusepath{stroke,fill}%
\end{pgfscope}%
\begin{pgfscope}%
\pgfpathrectangle{\pgfqpoint{0.550713in}{0.102108in}}{\pgfqpoint{3.194133in}{1.837939in}}%
\pgfusepath{clip}%
\pgfsetbuttcap%
\pgfsetmiterjoin%
\definecolor{currentfill}{rgb}{0.890196,0.000000,0.400000}%
\pgfsetfillcolor{currentfill}%
\pgfsetlinewidth{0.752812pt}%
\definecolor{currentstroke}{rgb}{0.000000,0.000000,0.000000}%
\pgfsetstrokecolor{currentstroke}%
\pgfsetdash{}{0pt}%
\pgfpathmoveto{\pgfqpoint{0.991503in}{0.815620in}}%
\pgfpathlineto{\pgfqpoint{1.148015in}{0.815620in}}%
\pgfpathlineto{\pgfqpoint{1.148015in}{0.984749in}}%
\pgfpathlineto{\pgfqpoint{0.991503in}{0.984749in}}%
\pgfpathlineto{\pgfqpoint{0.991503in}{0.815620in}}%
\pgfpathclose%
\pgfusepath{stroke,fill}%
\end{pgfscope}%
\begin{pgfscope}%
\pgfpathrectangle{\pgfqpoint{0.550713in}{0.102108in}}{\pgfqpoint{3.194133in}{1.837939in}}%
\pgfusepath{clip}%
\pgfsetbuttcap%
\pgfsetmiterjoin%
\definecolor{currentfill}{rgb}{0.976471,0.823529,0.854902}%
\pgfsetfillcolor{currentfill}%
\pgfsetlinewidth{0.752812pt}%
\definecolor{currentstroke}{rgb}{0.000000,0.000000,0.000000}%
\pgfsetstrokecolor{currentstroke}%
\pgfsetdash{}{0pt}%
\pgfpathmoveto{\pgfqpoint{1.151210in}{0.825546in}}%
\pgfpathlineto{\pgfqpoint{1.307722in}{0.825546in}}%
\pgfpathlineto{\pgfqpoint{1.307722in}{0.963364in}}%
\pgfpathlineto{\pgfqpoint{1.151210in}{0.963364in}}%
\pgfpathlineto{\pgfqpoint{1.151210in}{0.825546in}}%
\pgfpathclose%
\pgfusepath{stroke,fill}%
\end{pgfscope}%
\begin{pgfscope}%
\pgfpathrectangle{\pgfqpoint{0.550713in}{0.102108in}}{\pgfqpoint{3.194133in}{1.837939in}}%
\pgfusepath{clip}%
\pgfsetbuttcap%
\pgfsetmiterjoin%
\definecolor{currentfill}{rgb}{0.000000,0.329412,0.623529}%
\pgfsetfillcolor{currentfill}%
\pgfsetlinewidth{0.752812pt}%
\definecolor{currentstroke}{rgb}{0.000000,0.000000,0.000000}%
\pgfsetstrokecolor{currentstroke}%
\pgfsetdash{}{0pt}%
\pgfpathmoveto{\pgfqpoint{1.390770in}{0.904462in}}%
\pgfpathlineto{\pgfqpoint{1.547282in}{0.904462in}}%
\pgfpathlineto{\pgfqpoint{1.547282in}{1.140458in}}%
\pgfpathlineto{\pgfqpoint{1.390770in}{1.140458in}}%
\pgfpathlineto{\pgfqpoint{1.390770in}{0.904462in}}%
\pgfpathclose%
\pgfusepath{stroke,fill}%
\end{pgfscope}%
\begin{pgfscope}%
\pgfpathrectangle{\pgfqpoint{0.550713in}{0.102108in}}{\pgfqpoint{3.194133in}{1.837939in}}%
\pgfusepath{clip}%
\pgfsetbuttcap%
\pgfsetmiterjoin%
\definecolor{currentfill}{rgb}{0.780392,0.866667,0.949020}%
\pgfsetfillcolor{currentfill}%
\pgfsetlinewidth{0.752812pt}%
\definecolor{currentstroke}{rgb}{0.000000,0.000000,0.000000}%
\pgfsetstrokecolor{currentstroke}%
\pgfsetdash{}{0pt}%
\pgfpathmoveto{\pgfqpoint{1.550476in}{0.879996in}}%
\pgfpathlineto{\pgfqpoint{1.706989in}{0.879996in}}%
\pgfpathlineto{\pgfqpoint{1.706989in}{1.185129in}}%
\pgfpathlineto{\pgfqpoint{1.550476in}{1.185129in}}%
\pgfpathlineto{\pgfqpoint{1.550476in}{0.879996in}}%
\pgfpathclose%
\pgfusepath{stroke,fill}%
\end{pgfscope}%
\begin{pgfscope}%
\pgfpathrectangle{\pgfqpoint{0.550713in}{0.102108in}}{\pgfqpoint{3.194133in}{1.837939in}}%
\pgfusepath{clip}%
\pgfsetbuttcap%
\pgfsetmiterjoin%
\definecolor{currentfill}{rgb}{0.000000,0.380392,0.396078}%
\pgfsetfillcolor{currentfill}%
\pgfsetlinewidth{0.752812pt}%
\definecolor{currentstroke}{rgb}{0.000000,0.000000,0.000000}%
\pgfsetstrokecolor{currentstroke}%
\pgfsetdash{}{0pt}%
\pgfpathmoveto{\pgfqpoint{1.790036in}{0.868901in}}%
\pgfpathlineto{\pgfqpoint{1.946549in}{0.868901in}}%
\pgfpathlineto{\pgfqpoint{1.946549in}{1.048535in}}%
\pgfpathlineto{\pgfqpoint{1.790036in}{1.048535in}}%
\pgfpathlineto{\pgfqpoint{1.790036in}{0.868901in}}%
\pgfpathclose%
\pgfusepath{stroke,fill}%
\end{pgfscope}%
\begin{pgfscope}%
\pgfpathrectangle{\pgfqpoint{0.550713in}{0.102108in}}{\pgfqpoint{3.194133in}{1.837939in}}%
\pgfusepath{clip}%
\pgfsetbuttcap%
\pgfsetmiterjoin%
\definecolor{currentfill}{rgb}{0.749020,0.815686,0.819608}%
\pgfsetfillcolor{currentfill}%
\pgfsetlinewidth{0.752812pt}%
\definecolor{currentstroke}{rgb}{0.000000,0.000000,0.000000}%
\pgfsetstrokecolor{currentstroke}%
\pgfsetdash{}{0pt}%
\pgfpathmoveto{\pgfqpoint{1.949743in}{0.898254in}}%
\pgfpathlineto{\pgfqpoint{2.106255in}{0.898254in}}%
\pgfpathlineto{\pgfqpoint{2.106255in}{1.109605in}}%
\pgfpathlineto{\pgfqpoint{1.949743in}{1.109605in}}%
\pgfpathlineto{\pgfqpoint{1.949743in}{0.898254in}}%
\pgfpathclose%
\pgfusepath{stroke,fill}%
\end{pgfscope}%
\begin{pgfscope}%
\pgfpathrectangle{\pgfqpoint{0.550713in}{0.102108in}}{\pgfqpoint{3.194133in}{1.837939in}}%
\pgfusepath{clip}%
\pgfsetbuttcap%
\pgfsetmiterjoin%
\definecolor{currentfill}{rgb}{0.380392,0.129412,0.345098}%
\pgfsetfillcolor{currentfill}%
\pgfsetlinewidth{0.752812pt}%
\definecolor{currentstroke}{rgb}{0.000000,0.000000,0.000000}%
\pgfsetstrokecolor{currentstroke}%
\pgfsetdash{}{0pt}%
\pgfpathmoveto{\pgfqpoint{2.189303in}{1.081128in}}%
\pgfpathlineto{\pgfqpoint{2.345815in}{1.081128in}}%
\pgfpathlineto{\pgfqpoint{2.345815in}{1.302118in}}%
\pgfpathlineto{\pgfqpoint{2.189303in}{1.302118in}}%
\pgfpathlineto{\pgfqpoint{2.189303in}{1.081128in}}%
\pgfpathclose%
\pgfusepath{stroke,fill}%
\end{pgfscope}%
\begin{pgfscope}%
\pgfpathrectangle{\pgfqpoint{0.550713in}{0.102108in}}{\pgfqpoint{3.194133in}{1.837939in}}%
\pgfusepath{clip}%
\pgfsetbuttcap%
\pgfsetmiterjoin%
\definecolor{currentfill}{rgb}{0.823529,0.752941,0.803922}%
\pgfsetfillcolor{currentfill}%
\pgfsetlinewidth{0.752812pt}%
\definecolor{currentstroke}{rgb}{0.000000,0.000000,0.000000}%
\pgfsetstrokecolor{currentstroke}%
\pgfsetdash{}{0pt}%
\pgfpathmoveto{\pgfqpoint{2.349010in}{1.311222in}}%
\pgfpathlineto{\pgfqpoint{2.505522in}{1.311222in}}%
\pgfpathlineto{\pgfqpoint{2.505522in}{1.511157in}}%
\pgfpathlineto{\pgfqpoint{2.349010in}{1.511157in}}%
\pgfpathlineto{\pgfqpoint{2.349010in}{1.311222in}}%
\pgfpathclose%
\pgfusepath{stroke,fill}%
\end{pgfscope}%
\begin{pgfscope}%
\pgfpathrectangle{\pgfqpoint{0.550713in}{0.102108in}}{\pgfqpoint{3.194133in}{1.837939in}}%
\pgfusepath{clip}%
\pgfsetbuttcap%
\pgfsetmiterjoin%
\definecolor{currentfill}{rgb}{0.964706,0.658824,0.000000}%
\pgfsetfillcolor{currentfill}%
\pgfsetlinewidth{0.752812pt}%
\definecolor{currentstroke}{rgb}{0.000000,0.000000,0.000000}%
\pgfsetstrokecolor{currentstroke}%
\pgfsetdash{}{0pt}%
\pgfpathmoveto{\pgfqpoint{2.588570in}{0.964338in}}%
\pgfpathlineto{\pgfqpoint{2.745082in}{0.964338in}}%
\pgfpathlineto{\pgfqpoint{2.745082in}{1.057308in}}%
\pgfpathlineto{\pgfqpoint{2.588570in}{1.057308in}}%
\pgfpathlineto{\pgfqpoint{2.588570in}{0.964338in}}%
\pgfpathclose%
\pgfusepath{stroke,fill}%
\end{pgfscope}%
\begin{pgfscope}%
\pgfpathrectangle{\pgfqpoint{0.550713in}{0.102108in}}{\pgfqpoint{3.194133in}{1.837939in}}%
\pgfusepath{clip}%
\pgfsetbuttcap%
\pgfsetmiterjoin%
\definecolor{currentfill}{rgb}{0.996078,0.917647,0.788235}%
\pgfsetfillcolor{currentfill}%
\pgfsetlinewidth{0.752812pt}%
\definecolor{currentstroke}{rgb}{0.000000,0.000000,0.000000}%
\pgfsetstrokecolor{currentstroke}%
\pgfsetdash{}{0pt}%
\pgfpathmoveto{\pgfqpoint{2.748276in}{1.114258in}}%
\pgfpathlineto{\pgfqpoint{2.904789in}{1.114258in}}%
\pgfpathlineto{\pgfqpoint{2.904789in}{1.302923in}}%
\pgfpathlineto{\pgfqpoint{2.748276in}{1.302923in}}%
\pgfpathlineto{\pgfqpoint{2.748276in}{1.114258in}}%
\pgfpathclose%
\pgfusepath{stroke,fill}%
\end{pgfscope}%
\begin{pgfscope}%
\pgfpathrectangle{\pgfqpoint{0.550713in}{0.102108in}}{\pgfqpoint{3.194133in}{1.837939in}}%
\pgfusepath{clip}%
\pgfsetbuttcap%
\pgfsetmiterjoin%
\definecolor{currentfill}{rgb}{0.341176,0.670588,0.152941}%
\pgfsetfillcolor{currentfill}%
\pgfsetlinewidth{0.752812pt}%
\definecolor{currentstroke}{rgb}{0.000000,0.000000,0.000000}%
\pgfsetstrokecolor{currentstroke}%
\pgfsetdash{}{0pt}%
\pgfpathmoveto{\pgfqpoint{2.987836in}{0.795273in}}%
\pgfpathlineto{\pgfqpoint{3.144349in}{0.795273in}}%
\pgfpathlineto{\pgfqpoint{3.144349in}{0.930978in}}%
\pgfpathlineto{\pgfqpoint{2.987836in}{0.930978in}}%
\pgfpathlineto{\pgfqpoint{2.987836in}{0.795273in}}%
\pgfpathclose%
\pgfusepath{stroke,fill}%
\end{pgfscope}%
\begin{pgfscope}%
\pgfpathrectangle{\pgfqpoint{0.550713in}{0.102108in}}{\pgfqpoint{3.194133in}{1.837939in}}%
\pgfusepath{clip}%
\pgfsetbuttcap%
\pgfsetmiterjoin%
\definecolor{currentfill}{rgb}{0.866667,0.921569,0.807843}%
\pgfsetfillcolor{currentfill}%
\pgfsetlinewidth{0.752812pt}%
\definecolor{currentstroke}{rgb}{0.000000,0.000000,0.000000}%
\pgfsetstrokecolor{currentstroke}%
\pgfsetdash{}{0pt}%
\pgfpathmoveto{\pgfqpoint{3.147543in}{0.848960in}}%
\pgfpathlineto{\pgfqpoint{3.304055in}{0.848960in}}%
\pgfpathlineto{\pgfqpoint{3.304055in}{0.981084in}}%
\pgfpathlineto{\pgfqpoint{3.147543in}{0.981084in}}%
\pgfpathlineto{\pgfqpoint{3.147543in}{0.848960in}}%
\pgfpathclose%
\pgfusepath{stroke,fill}%
\end{pgfscope}%
\begin{pgfscope}%
\pgfpathrectangle{\pgfqpoint{0.550713in}{0.102108in}}{\pgfqpoint{3.194133in}{1.837939in}}%
\pgfusepath{clip}%
\pgfsetbuttcap%
\pgfsetmiterjoin%
\definecolor{currentfill}{rgb}{0.478431,0.435294,0.674510}%
\pgfsetfillcolor{currentfill}%
\pgfsetlinewidth{0.752812pt}%
\definecolor{currentstroke}{rgb}{0.000000,0.000000,0.000000}%
\pgfsetstrokecolor{currentstroke}%
\pgfsetdash{}{0pt}%
\pgfpathmoveto{\pgfqpoint{3.387103in}{0.884310in}}%
\pgfpathlineto{\pgfqpoint{3.543615in}{0.884310in}}%
\pgfpathlineto{\pgfqpoint{3.543615in}{0.988146in}}%
\pgfpathlineto{\pgfqpoint{3.387103in}{0.988146in}}%
\pgfpathlineto{\pgfqpoint{3.387103in}{0.884310in}}%
\pgfpathclose%
\pgfusepath{stroke,fill}%
\end{pgfscope}%
\begin{pgfscope}%
\pgfpathrectangle{\pgfqpoint{0.550713in}{0.102108in}}{\pgfqpoint{3.194133in}{1.837939in}}%
\pgfusepath{clip}%
\pgfsetbuttcap%
\pgfsetmiterjoin%
\definecolor{currentfill}{rgb}{0.870588,0.854902,0.921569}%
\pgfsetfillcolor{currentfill}%
\pgfsetlinewidth{0.752812pt}%
\definecolor{currentstroke}{rgb}{0.000000,0.000000,0.000000}%
\pgfsetstrokecolor{currentstroke}%
\pgfsetdash{}{0pt}%
\pgfpathmoveto{\pgfqpoint{3.546809in}{0.879574in}}%
\pgfpathlineto{\pgfqpoint{3.703322in}{0.879574in}}%
\pgfpathlineto{\pgfqpoint{3.703322in}{0.983152in}}%
\pgfpathlineto{\pgfqpoint{3.546809in}{0.983152in}}%
\pgfpathlineto{\pgfqpoint{3.546809in}{0.879574in}}%
\pgfpathclose%
\pgfusepath{stroke,fill}%
\end{pgfscope}%
\begin{pgfscope}%
\pgfpathrectangle{\pgfqpoint{0.550713in}{0.102108in}}{\pgfqpoint{3.194133in}{1.837939in}}%
\pgfusepath{clip}%
\pgfsetbuttcap%
\pgfsetmiterjoin%
\definecolor{currentfill}{rgb}{0.000000,0.000000,0.000000}%
\pgfsetfillcolor{currentfill}%
\pgfsetlinewidth{0.376406pt}%
\definecolor{currentstroke}{rgb}{0.000000,0.000000,0.000000}%
\pgfsetstrokecolor{currentstroke}%
\pgfsetdash{}{0pt}%
\pgfpathmoveto{\pgfqpoint{0.750346in}{0.102108in}}%
\pgfpathlineto{\pgfqpoint{0.750346in}{0.102108in}}%
\pgfpathlineto{\pgfqpoint{0.750346in}{0.102108in}}%
\pgfpathlineto{\pgfqpoint{0.750346in}{0.102108in}}%
\pgfpathclose%
\pgfusepath{stroke,fill}%
\end{pgfscope}%
\begin{pgfscope}%
\pgfpathrectangle{\pgfqpoint{0.550713in}{0.102108in}}{\pgfqpoint{3.194133in}{1.837939in}}%
\pgfusepath{clip}%
\pgfsetbuttcap%
\pgfsetmiterjoin%
\definecolor{currentfill}{rgb}{0.813235,0.819118,0.822059}%
\pgfsetfillcolor{currentfill}%
\pgfsetlinewidth{0.376406pt}%
\definecolor{currentstroke}{rgb}{0.000000,0.000000,0.000000}%
\pgfsetstrokecolor{currentstroke}%
\pgfsetdash{}{0pt}%
\pgfpathmoveto{\pgfqpoint{0.750346in}{0.102108in}}%
\pgfpathlineto{\pgfqpoint{0.750346in}{0.102108in}}%
\pgfpathlineto{\pgfqpoint{0.750346in}{0.102108in}}%
\pgfpathlineto{\pgfqpoint{0.750346in}{0.102108in}}%
\pgfpathclose%
\pgfusepath{stroke,fill}%
\end{pgfscope}%
\begin{pgfscope}%
\pgfsetbuttcap%
\pgfsetroundjoin%
\definecolor{currentfill}{rgb}{0.000000,0.000000,0.000000}%
\pgfsetfillcolor{currentfill}%
\pgfsetlinewidth{0.803000pt}%
\definecolor{currentstroke}{rgb}{0.000000,0.000000,0.000000}%
\pgfsetstrokecolor{currentstroke}%
\pgfsetdash{}{0pt}%
\pgfsys@defobject{currentmarker}{\pgfqpoint{-0.048611in}{0.000000in}}{\pgfqpoint{-0.000000in}{0.000000in}}{%
\pgfpathmoveto{\pgfqpoint{-0.000000in}{0.000000in}}%
\pgfpathlineto{\pgfqpoint{-0.048611in}{0.000000in}}%
\pgfusepath{stroke,fill}%
}%
\begin{pgfscope}%
\pgfsys@transformshift{0.550713in}{0.102108in}%
\pgfsys@useobject{currentmarker}{}%
\end{pgfscope}%
\end{pgfscope}%
\begin{pgfscope}%
\definecolor{textcolor}{rgb}{0.000000,0.000000,0.000000}%
\pgfsetstrokecolor{textcolor}%
\pgfsetfillcolor{textcolor}%
\pgftext[x=0.384046in, y=0.053913in, left, base]{\color{textcolor}\rmfamily\fontsize{10.000000}{12.000000}\selectfont \(\displaystyle {0}\)}%
\end{pgfscope}%
\begin{pgfscope}%
\pgfsetbuttcap%
\pgfsetroundjoin%
\definecolor{currentfill}{rgb}{0.000000,0.000000,0.000000}%
\pgfsetfillcolor{currentfill}%
\pgfsetlinewidth{0.803000pt}%
\definecolor{currentstroke}{rgb}{0.000000,0.000000,0.000000}%
\pgfsetstrokecolor{currentstroke}%
\pgfsetdash{}{0pt}%
\pgfsys@defobject{currentmarker}{\pgfqpoint{-0.048611in}{0.000000in}}{\pgfqpoint{-0.000000in}{0.000000in}}{%
\pgfpathmoveto{\pgfqpoint{-0.000000in}{0.000000in}}%
\pgfpathlineto{\pgfqpoint{-0.048611in}{0.000000in}}%
\pgfusepath{stroke,fill}%
}%
\begin{pgfscope}%
\pgfsys@transformshift{0.550713in}{0.430311in}%
\pgfsys@useobject{currentmarker}{}%
\end{pgfscope}%
\end{pgfscope}%
\begin{pgfscope}%
\definecolor{textcolor}{rgb}{0.000000,0.000000,0.000000}%
\pgfsetstrokecolor{textcolor}%
\pgfsetfillcolor{textcolor}%
\pgftext[x=0.314601in, y=0.382117in, left, base]{\color{textcolor}\rmfamily\fontsize{10.000000}{12.000000}\selectfont \(\displaystyle {25}\)}%
\end{pgfscope}%
\begin{pgfscope}%
\pgfsetbuttcap%
\pgfsetroundjoin%
\definecolor{currentfill}{rgb}{0.000000,0.000000,0.000000}%
\pgfsetfillcolor{currentfill}%
\pgfsetlinewidth{0.803000pt}%
\definecolor{currentstroke}{rgb}{0.000000,0.000000,0.000000}%
\pgfsetstrokecolor{currentstroke}%
\pgfsetdash{}{0pt}%
\pgfsys@defobject{currentmarker}{\pgfqpoint{-0.048611in}{0.000000in}}{\pgfqpoint{-0.000000in}{0.000000in}}{%
\pgfpathmoveto{\pgfqpoint{-0.000000in}{0.000000in}}%
\pgfpathlineto{\pgfqpoint{-0.048611in}{0.000000in}}%
\pgfusepath{stroke,fill}%
}%
\begin{pgfscope}%
\pgfsys@transformshift{0.550713in}{0.758515in}%
\pgfsys@useobject{currentmarker}{}%
\end{pgfscope}%
\end{pgfscope}%
\begin{pgfscope}%
\definecolor{textcolor}{rgb}{0.000000,0.000000,0.000000}%
\pgfsetstrokecolor{textcolor}%
\pgfsetfillcolor{textcolor}%
\pgftext[x=0.314601in, y=0.710320in, left, base]{\color{textcolor}\rmfamily\fontsize{10.000000}{12.000000}\selectfont \(\displaystyle {50}\)}%
\end{pgfscope}%
\begin{pgfscope}%
\pgfsetbuttcap%
\pgfsetroundjoin%
\definecolor{currentfill}{rgb}{0.000000,0.000000,0.000000}%
\pgfsetfillcolor{currentfill}%
\pgfsetlinewidth{0.803000pt}%
\definecolor{currentstroke}{rgb}{0.000000,0.000000,0.000000}%
\pgfsetstrokecolor{currentstroke}%
\pgfsetdash{}{0pt}%
\pgfsys@defobject{currentmarker}{\pgfqpoint{-0.048611in}{0.000000in}}{\pgfqpoint{-0.000000in}{0.000000in}}{%
\pgfpathmoveto{\pgfqpoint{-0.000000in}{0.000000in}}%
\pgfpathlineto{\pgfqpoint{-0.048611in}{0.000000in}}%
\pgfusepath{stroke,fill}%
}%
\begin{pgfscope}%
\pgfsys@transformshift{0.550713in}{1.086718in}%
\pgfsys@useobject{currentmarker}{}%
\end{pgfscope}%
\end{pgfscope}%
\begin{pgfscope}%
\definecolor{textcolor}{rgb}{0.000000,0.000000,0.000000}%
\pgfsetstrokecolor{textcolor}%
\pgfsetfillcolor{textcolor}%
\pgftext[x=0.314601in, y=1.038524in, left, base]{\color{textcolor}\rmfamily\fontsize{10.000000}{12.000000}\selectfont \(\displaystyle {75}\)}%
\end{pgfscope}%
\begin{pgfscope}%
\pgfsetbuttcap%
\pgfsetroundjoin%
\definecolor{currentfill}{rgb}{0.000000,0.000000,0.000000}%
\pgfsetfillcolor{currentfill}%
\pgfsetlinewidth{0.803000pt}%
\definecolor{currentstroke}{rgb}{0.000000,0.000000,0.000000}%
\pgfsetstrokecolor{currentstroke}%
\pgfsetdash{}{0pt}%
\pgfsys@defobject{currentmarker}{\pgfqpoint{-0.048611in}{0.000000in}}{\pgfqpoint{-0.000000in}{0.000000in}}{%
\pgfpathmoveto{\pgfqpoint{-0.000000in}{0.000000in}}%
\pgfpathlineto{\pgfqpoint{-0.048611in}{0.000000in}}%
\pgfusepath{stroke,fill}%
}%
\begin{pgfscope}%
\pgfsys@transformshift{0.550713in}{1.414921in}%
\pgfsys@useobject{currentmarker}{}%
\end{pgfscope}%
\end{pgfscope}%
\begin{pgfscope}%
\definecolor{textcolor}{rgb}{0.000000,0.000000,0.000000}%
\pgfsetstrokecolor{textcolor}%
\pgfsetfillcolor{textcolor}%
\pgftext[x=0.245156in, y=1.366727in, left, base]{\color{textcolor}\rmfamily\fontsize{10.000000}{12.000000}\selectfont \(\displaystyle {100}\)}%
\end{pgfscope}%
\begin{pgfscope}%
\pgfsetbuttcap%
\pgfsetroundjoin%
\definecolor{currentfill}{rgb}{0.000000,0.000000,0.000000}%
\pgfsetfillcolor{currentfill}%
\pgfsetlinewidth{0.803000pt}%
\definecolor{currentstroke}{rgb}{0.000000,0.000000,0.000000}%
\pgfsetstrokecolor{currentstroke}%
\pgfsetdash{}{0pt}%
\pgfsys@defobject{currentmarker}{\pgfqpoint{-0.048611in}{0.000000in}}{\pgfqpoint{-0.000000in}{0.000000in}}{%
\pgfpathmoveto{\pgfqpoint{-0.000000in}{0.000000in}}%
\pgfpathlineto{\pgfqpoint{-0.048611in}{0.000000in}}%
\pgfusepath{stroke,fill}%
}%
\begin{pgfscope}%
\pgfsys@transformshift{0.550713in}{1.743125in}%
\pgfsys@useobject{currentmarker}{}%
\end{pgfscope}%
\end{pgfscope}%
\begin{pgfscope}%
\definecolor{textcolor}{rgb}{0.000000,0.000000,0.000000}%
\pgfsetstrokecolor{textcolor}%
\pgfsetfillcolor{textcolor}%
\pgftext[x=0.245156in, y=1.694930in, left, base]{\color{textcolor}\rmfamily\fontsize{10.000000}{12.000000}\selectfont \(\displaystyle {125}\)}%
\end{pgfscope}%
\begin{pgfscope}%
\definecolor{textcolor}{rgb}{0.000000,0.000000,0.000000}%
\pgfsetstrokecolor{textcolor}%
\pgfsetfillcolor{textcolor}%
\pgftext[x=0.189601in,y=1.021077in,,bottom,rotate=90.000000]{\color{textcolor}\rmfamily\fontsize{10.000000}{12.000000}\selectfont \(\displaystyle R_T\)}%
\end{pgfscope}%
\begin{pgfscope}%
\pgfpathrectangle{\pgfqpoint{0.550713in}{0.102108in}}{\pgfqpoint{3.194133in}{1.837939in}}%
\pgfusepath{clip}%
\pgfsetbuttcap%
\pgfsetroundjoin%
\pgfsetlinewidth{0.501875pt}%
\definecolor{currentstroke}{rgb}{0.392157,0.396078,0.403922}%
\pgfsetstrokecolor{currentstroke}%
\pgfsetdash{}{0pt}%
\pgfpathmoveto{\pgfqpoint{0.949979in}{0.102108in}}%
\pgfpathlineto{\pgfqpoint{0.949979in}{1.940047in}}%
\pgfusepath{stroke}%
\end{pgfscope}%
\begin{pgfscope}%
\pgfpathrectangle{\pgfqpoint{0.550713in}{0.102108in}}{\pgfqpoint{3.194133in}{1.837939in}}%
\pgfusepath{clip}%
\pgfsetbuttcap%
\pgfsetroundjoin%
\pgfsetlinewidth{0.501875pt}%
\definecolor{currentstroke}{rgb}{0.392157,0.396078,0.403922}%
\pgfsetstrokecolor{currentstroke}%
\pgfsetdash{}{0pt}%
\pgfpathmoveto{\pgfqpoint{1.349246in}{0.102108in}}%
\pgfpathlineto{\pgfqpoint{1.349246in}{1.940047in}}%
\pgfusepath{stroke}%
\end{pgfscope}%
\begin{pgfscope}%
\pgfpathrectangle{\pgfqpoint{0.550713in}{0.102108in}}{\pgfqpoint{3.194133in}{1.837939in}}%
\pgfusepath{clip}%
\pgfsetbuttcap%
\pgfsetroundjoin%
\pgfsetlinewidth{0.501875pt}%
\definecolor{currentstroke}{rgb}{0.392157,0.396078,0.403922}%
\pgfsetstrokecolor{currentstroke}%
\pgfsetdash{}{0pt}%
\pgfpathmoveto{\pgfqpoint{1.748513in}{0.102108in}}%
\pgfpathlineto{\pgfqpoint{1.748513in}{1.940047in}}%
\pgfusepath{stroke}%
\end{pgfscope}%
\begin{pgfscope}%
\pgfpathrectangle{\pgfqpoint{0.550713in}{0.102108in}}{\pgfqpoint{3.194133in}{1.837939in}}%
\pgfusepath{clip}%
\pgfsetbuttcap%
\pgfsetroundjoin%
\pgfsetlinewidth{0.501875pt}%
\definecolor{currentstroke}{rgb}{0.392157,0.396078,0.403922}%
\pgfsetstrokecolor{currentstroke}%
\pgfsetdash{}{0pt}%
\pgfpathmoveto{\pgfqpoint{2.147779in}{0.102108in}}%
\pgfpathlineto{\pgfqpoint{2.147779in}{1.940047in}}%
\pgfusepath{stroke}%
\end{pgfscope}%
\begin{pgfscope}%
\pgfpathrectangle{\pgfqpoint{0.550713in}{0.102108in}}{\pgfqpoint{3.194133in}{1.837939in}}%
\pgfusepath{clip}%
\pgfsetbuttcap%
\pgfsetroundjoin%
\pgfsetlinewidth{0.501875pt}%
\definecolor{currentstroke}{rgb}{0.392157,0.396078,0.403922}%
\pgfsetstrokecolor{currentstroke}%
\pgfsetdash{}{0pt}%
\pgfpathmoveto{\pgfqpoint{2.547046in}{0.102108in}}%
\pgfpathlineto{\pgfqpoint{2.547046in}{1.940047in}}%
\pgfusepath{stroke}%
\end{pgfscope}%
\begin{pgfscope}%
\pgfpathrectangle{\pgfqpoint{0.550713in}{0.102108in}}{\pgfqpoint{3.194133in}{1.837939in}}%
\pgfusepath{clip}%
\pgfsetbuttcap%
\pgfsetroundjoin%
\pgfsetlinewidth{0.501875pt}%
\definecolor{currentstroke}{rgb}{0.392157,0.396078,0.403922}%
\pgfsetstrokecolor{currentstroke}%
\pgfsetdash{}{0pt}%
\pgfpathmoveto{\pgfqpoint{2.946312in}{0.102108in}}%
\pgfpathlineto{\pgfqpoint{2.946312in}{1.940047in}}%
\pgfusepath{stroke}%
\end{pgfscope}%
\begin{pgfscope}%
\pgfpathrectangle{\pgfqpoint{0.550713in}{0.102108in}}{\pgfqpoint{3.194133in}{1.837939in}}%
\pgfusepath{clip}%
\pgfsetbuttcap%
\pgfsetroundjoin%
\pgfsetlinewidth{0.501875pt}%
\definecolor{currentstroke}{rgb}{0.392157,0.396078,0.403922}%
\pgfsetstrokecolor{currentstroke}%
\pgfsetdash{}{0pt}%
\pgfpathmoveto{\pgfqpoint{3.345579in}{0.102108in}}%
\pgfpathlineto{\pgfqpoint{3.345579in}{1.940047in}}%
\pgfusepath{stroke}%
\end{pgfscope}%
\begin{pgfscope}%
\pgfpathrectangle{\pgfqpoint{0.550713in}{0.102108in}}{\pgfqpoint{3.194133in}{1.837939in}}%
\pgfusepath{clip}%
\pgfsetbuttcap%
\pgfsetroundjoin%
\pgfsetlinewidth{0.853187pt}%
\definecolor{currentstroke}{rgb}{0.392157,0.396078,0.403922}%
\pgfsetstrokecolor{currentstroke}%
\pgfsetdash{{3.145000pt}{1.360000pt}}{0.000000pt}%
\pgfpathmoveto{\pgfqpoint{0.540713in}{1.792537in}}%
\pgfpathlineto{\pgfqpoint{3.754846in}{1.792537in}}%
\pgfusepath{stroke}%
\end{pgfscope}%
\begin{pgfscope}%
\pgfpathrectangle{\pgfqpoint{0.550713in}{0.102108in}}{\pgfqpoint{3.194133in}{1.837939in}}%
\pgfusepath{clip}%
\pgfsetrectcap%
\pgfsetroundjoin%
\pgfsetlinewidth{0.752812pt}%
\definecolor{currentstroke}{rgb}{0.000000,0.000000,0.000000}%
\pgfsetstrokecolor{currentstroke}%
\pgfsetdash{}{0pt}%
\pgfpathmoveto{\pgfqpoint{0.670493in}{0.815833in}}%
\pgfpathlineto{\pgfqpoint{0.670493in}{0.615590in}}%
\pgfusepath{stroke}%
\end{pgfscope}%
\begin{pgfscope}%
\pgfpathrectangle{\pgfqpoint{0.550713in}{0.102108in}}{\pgfqpoint{3.194133in}{1.837939in}}%
\pgfusepath{clip}%
\pgfsetrectcap%
\pgfsetroundjoin%
\pgfsetlinewidth{0.752812pt}%
\definecolor{currentstroke}{rgb}{0.000000,0.000000,0.000000}%
\pgfsetstrokecolor{currentstroke}%
\pgfsetdash{}{0pt}%
\pgfpathmoveto{\pgfqpoint{0.670493in}{0.959334in}}%
\pgfpathlineto{\pgfqpoint{0.670493in}{1.089232in}}%
\pgfusepath{stroke}%
\end{pgfscope}%
\begin{pgfscope}%
\pgfpathrectangle{\pgfqpoint{0.550713in}{0.102108in}}{\pgfqpoint{3.194133in}{1.837939in}}%
\pgfusepath{clip}%
\pgfsetrectcap%
\pgfsetroundjoin%
\pgfsetlinewidth{0.752812pt}%
\definecolor{currentstroke}{rgb}{0.000000,0.000000,0.000000}%
\pgfsetstrokecolor{currentstroke}%
\pgfsetdash{}{0pt}%
\pgfpathmoveto{\pgfqpoint{0.631364in}{0.615590in}}%
\pgfpathlineto{\pgfqpoint{0.709621in}{0.615590in}}%
\pgfusepath{stroke}%
\end{pgfscope}%
\begin{pgfscope}%
\pgfpathrectangle{\pgfqpoint{0.550713in}{0.102108in}}{\pgfqpoint{3.194133in}{1.837939in}}%
\pgfusepath{clip}%
\pgfsetrectcap%
\pgfsetroundjoin%
\pgfsetlinewidth{0.752812pt}%
\definecolor{currentstroke}{rgb}{0.000000,0.000000,0.000000}%
\pgfsetstrokecolor{currentstroke}%
\pgfsetdash{}{0pt}%
\pgfpathmoveto{\pgfqpoint{0.631364in}{1.089232in}}%
\pgfpathlineto{\pgfqpoint{0.709621in}{1.089232in}}%
\pgfusepath{stroke}%
\end{pgfscope}%
\begin{pgfscope}%
\pgfpathrectangle{\pgfqpoint{0.550713in}{0.102108in}}{\pgfqpoint{3.194133in}{1.837939in}}%
\pgfusepath{clip}%
\pgfsetbuttcap%
\pgfsetmiterjoin%
\definecolor{currentfill}{rgb}{0.000000,0.000000,0.000000}%
\pgfsetfillcolor{currentfill}%
\pgfsetlinewidth{1.003750pt}%
\definecolor{currentstroke}{rgb}{0.000000,0.000000,0.000000}%
\pgfsetstrokecolor{currentstroke}%
\pgfsetdash{}{0pt}%
\pgfsys@defobject{currentmarker}{\pgfqpoint{-0.011785in}{-0.019642in}}{\pgfqpoint{0.011785in}{0.019642in}}{%
\pgfpathmoveto{\pgfqpoint{-0.000000in}{-0.019642in}}%
\pgfpathlineto{\pgfqpoint{0.011785in}{0.000000in}}%
\pgfpathlineto{\pgfqpoint{0.000000in}{0.019642in}}%
\pgfpathlineto{\pgfqpoint{-0.011785in}{0.000000in}}%
\pgfpathclose%
\pgfusepath{stroke,fill}%
}%
\begin{pgfscope}%
\pgfsys@transformshift{0.670493in}{0.599608in}%
\pgfsys@useobject{currentmarker}{}%
\end{pgfscope}%
\end{pgfscope}%
\begin{pgfscope}%
\pgfpathrectangle{\pgfqpoint{0.550713in}{0.102108in}}{\pgfqpoint{3.194133in}{1.837939in}}%
\pgfusepath{clip}%
\pgfsetrectcap%
\pgfsetroundjoin%
\pgfsetlinewidth{0.752812pt}%
\definecolor{currentstroke}{rgb}{0.000000,0.000000,0.000000}%
\pgfsetstrokecolor{currentstroke}%
\pgfsetdash{}{0pt}%
\pgfpathmoveto{\pgfqpoint{0.830199in}{0.824861in}}%
\pgfpathlineto{\pgfqpoint{0.830199in}{0.697822in}}%
\pgfusepath{stroke}%
\end{pgfscope}%
\begin{pgfscope}%
\pgfpathrectangle{\pgfqpoint{0.550713in}{0.102108in}}{\pgfqpoint{3.194133in}{1.837939in}}%
\pgfusepath{clip}%
\pgfsetrectcap%
\pgfsetroundjoin%
\pgfsetlinewidth{0.752812pt}%
\definecolor{currentstroke}{rgb}{0.000000,0.000000,0.000000}%
\pgfsetstrokecolor{currentstroke}%
\pgfsetdash{}{0pt}%
\pgfpathmoveto{\pgfqpoint{0.830199in}{1.032304in}}%
\pgfpathlineto{\pgfqpoint{0.830199in}{1.211460in}}%
\pgfusepath{stroke}%
\end{pgfscope}%
\begin{pgfscope}%
\pgfpathrectangle{\pgfqpoint{0.550713in}{0.102108in}}{\pgfqpoint{3.194133in}{1.837939in}}%
\pgfusepath{clip}%
\pgfsetrectcap%
\pgfsetroundjoin%
\pgfsetlinewidth{0.752812pt}%
\definecolor{currentstroke}{rgb}{0.000000,0.000000,0.000000}%
\pgfsetstrokecolor{currentstroke}%
\pgfsetdash{}{0pt}%
\pgfpathmoveto{\pgfqpoint{0.791071in}{0.697822in}}%
\pgfpathlineto{\pgfqpoint{0.869327in}{0.697822in}}%
\pgfusepath{stroke}%
\end{pgfscope}%
\begin{pgfscope}%
\pgfpathrectangle{\pgfqpoint{0.550713in}{0.102108in}}{\pgfqpoint{3.194133in}{1.837939in}}%
\pgfusepath{clip}%
\pgfsetrectcap%
\pgfsetroundjoin%
\pgfsetlinewidth{0.752812pt}%
\definecolor{currentstroke}{rgb}{0.000000,0.000000,0.000000}%
\pgfsetstrokecolor{currentstroke}%
\pgfsetdash{}{0pt}%
\pgfpathmoveto{\pgfqpoint{0.791071in}{1.211460in}}%
\pgfpathlineto{\pgfqpoint{0.869327in}{1.211460in}}%
\pgfusepath{stroke}%
\end{pgfscope}%
\begin{pgfscope}%
\pgfpathrectangle{\pgfqpoint{0.550713in}{0.102108in}}{\pgfqpoint{3.194133in}{1.837939in}}%
\pgfusepath{clip}%
\pgfsetrectcap%
\pgfsetroundjoin%
\pgfsetlinewidth{0.752812pt}%
\definecolor{currentstroke}{rgb}{0.000000,0.000000,0.000000}%
\pgfsetstrokecolor{currentstroke}%
\pgfsetdash{}{0pt}%
\pgfpathmoveto{\pgfqpoint{1.069759in}{0.815620in}}%
\pgfpathlineto{\pgfqpoint{1.069759in}{0.576626in}}%
\pgfusepath{stroke}%
\end{pgfscope}%
\begin{pgfscope}%
\pgfpathrectangle{\pgfqpoint{0.550713in}{0.102108in}}{\pgfqpoint{3.194133in}{1.837939in}}%
\pgfusepath{clip}%
\pgfsetrectcap%
\pgfsetroundjoin%
\pgfsetlinewidth{0.752812pt}%
\definecolor{currentstroke}{rgb}{0.000000,0.000000,0.000000}%
\pgfsetstrokecolor{currentstroke}%
\pgfsetdash{}{0pt}%
\pgfpathmoveto{\pgfqpoint{1.069759in}{0.984749in}}%
\pgfpathlineto{\pgfqpoint{1.069759in}{1.219940in}}%
\pgfusepath{stroke}%
\end{pgfscope}%
\begin{pgfscope}%
\pgfpathrectangle{\pgfqpoint{0.550713in}{0.102108in}}{\pgfqpoint{3.194133in}{1.837939in}}%
\pgfusepath{clip}%
\pgfsetrectcap%
\pgfsetroundjoin%
\pgfsetlinewidth{0.752812pt}%
\definecolor{currentstroke}{rgb}{0.000000,0.000000,0.000000}%
\pgfsetstrokecolor{currentstroke}%
\pgfsetdash{}{0pt}%
\pgfpathmoveto{\pgfqpoint{1.030631in}{0.576626in}}%
\pgfpathlineto{\pgfqpoint{1.108887in}{0.576626in}}%
\pgfusepath{stroke}%
\end{pgfscope}%
\begin{pgfscope}%
\pgfpathrectangle{\pgfqpoint{0.550713in}{0.102108in}}{\pgfqpoint{3.194133in}{1.837939in}}%
\pgfusepath{clip}%
\pgfsetrectcap%
\pgfsetroundjoin%
\pgfsetlinewidth{0.752812pt}%
\definecolor{currentstroke}{rgb}{0.000000,0.000000,0.000000}%
\pgfsetstrokecolor{currentstroke}%
\pgfsetdash{}{0pt}%
\pgfpathmoveto{\pgfqpoint{1.030631in}{1.219940in}}%
\pgfpathlineto{\pgfqpoint{1.108887in}{1.219940in}}%
\pgfusepath{stroke}%
\end{pgfscope}%
\begin{pgfscope}%
\pgfpathrectangle{\pgfqpoint{0.550713in}{0.102108in}}{\pgfqpoint{3.194133in}{1.837939in}}%
\pgfusepath{clip}%
\pgfsetbuttcap%
\pgfsetmiterjoin%
\definecolor{currentfill}{rgb}{0.000000,0.000000,0.000000}%
\pgfsetfillcolor{currentfill}%
\pgfsetlinewidth{1.003750pt}%
\definecolor{currentstroke}{rgb}{0.000000,0.000000,0.000000}%
\pgfsetstrokecolor{currentstroke}%
\pgfsetdash{}{0pt}%
\pgfsys@defobject{currentmarker}{\pgfqpoint{-0.011785in}{-0.019642in}}{\pgfqpoint{0.011785in}{0.019642in}}{%
\pgfpathmoveto{\pgfqpoint{-0.000000in}{-0.019642in}}%
\pgfpathlineto{\pgfqpoint{0.011785in}{0.000000in}}%
\pgfpathlineto{\pgfqpoint{0.000000in}{0.019642in}}%
\pgfpathlineto{\pgfqpoint{-0.011785in}{0.000000in}}%
\pgfpathclose%
\pgfusepath{stroke,fill}%
}%
\begin{pgfscope}%
\pgfsys@transformshift{1.069759in}{0.534628in}%
\pgfsys@useobject{currentmarker}{}%
\end{pgfscope}%
\begin{pgfscope}%
\pgfsys@transformshift{1.069759in}{0.558157in}%
\pgfsys@useobject{currentmarker}{}%
\end{pgfscope}%
\end{pgfscope}%
\begin{pgfscope}%
\pgfpathrectangle{\pgfqpoint{0.550713in}{0.102108in}}{\pgfqpoint{3.194133in}{1.837939in}}%
\pgfusepath{clip}%
\pgfsetrectcap%
\pgfsetroundjoin%
\pgfsetlinewidth{0.752812pt}%
\definecolor{currentstroke}{rgb}{0.000000,0.000000,0.000000}%
\pgfsetstrokecolor{currentstroke}%
\pgfsetdash{}{0pt}%
\pgfpathmoveto{\pgfqpoint{1.229466in}{0.825546in}}%
\pgfpathlineto{\pgfqpoint{1.229466in}{0.630255in}}%
\pgfusepath{stroke}%
\end{pgfscope}%
\begin{pgfscope}%
\pgfpathrectangle{\pgfqpoint{0.550713in}{0.102108in}}{\pgfqpoint{3.194133in}{1.837939in}}%
\pgfusepath{clip}%
\pgfsetrectcap%
\pgfsetroundjoin%
\pgfsetlinewidth{0.752812pt}%
\definecolor{currentstroke}{rgb}{0.000000,0.000000,0.000000}%
\pgfsetstrokecolor{currentstroke}%
\pgfsetdash{}{0pt}%
\pgfpathmoveto{\pgfqpoint{1.229466in}{0.963364in}}%
\pgfpathlineto{\pgfqpoint{1.229466in}{1.143672in}}%
\pgfusepath{stroke}%
\end{pgfscope}%
\begin{pgfscope}%
\pgfpathrectangle{\pgfqpoint{0.550713in}{0.102108in}}{\pgfqpoint{3.194133in}{1.837939in}}%
\pgfusepath{clip}%
\pgfsetrectcap%
\pgfsetroundjoin%
\pgfsetlinewidth{0.752812pt}%
\definecolor{currentstroke}{rgb}{0.000000,0.000000,0.000000}%
\pgfsetstrokecolor{currentstroke}%
\pgfsetdash{}{0pt}%
\pgfpathmoveto{\pgfqpoint{1.190338in}{0.630255in}}%
\pgfpathlineto{\pgfqpoint{1.268594in}{0.630255in}}%
\pgfusepath{stroke}%
\end{pgfscope}%
\begin{pgfscope}%
\pgfpathrectangle{\pgfqpoint{0.550713in}{0.102108in}}{\pgfqpoint{3.194133in}{1.837939in}}%
\pgfusepath{clip}%
\pgfsetrectcap%
\pgfsetroundjoin%
\pgfsetlinewidth{0.752812pt}%
\definecolor{currentstroke}{rgb}{0.000000,0.000000,0.000000}%
\pgfsetstrokecolor{currentstroke}%
\pgfsetdash{}{0pt}%
\pgfpathmoveto{\pgfqpoint{1.190338in}{1.143672in}}%
\pgfpathlineto{\pgfqpoint{1.268594in}{1.143672in}}%
\pgfusepath{stroke}%
\end{pgfscope}%
\begin{pgfscope}%
\pgfpathrectangle{\pgfqpoint{0.550713in}{0.102108in}}{\pgfqpoint{3.194133in}{1.837939in}}%
\pgfusepath{clip}%
\pgfsetbuttcap%
\pgfsetmiterjoin%
\definecolor{currentfill}{rgb}{0.000000,0.000000,0.000000}%
\pgfsetfillcolor{currentfill}%
\pgfsetlinewidth{1.003750pt}%
\definecolor{currentstroke}{rgb}{0.000000,0.000000,0.000000}%
\pgfsetstrokecolor{currentstroke}%
\pgfsetdash{}{0pt}%
\pgfsys@defobject{currentmarker}{\pgfqpoint{-0.011785in}{-0.019642in}}{\pgfqpoint{0.011785in}{0.019642in}}{%
\pgfpathmoveto{\pgfqpoint{-0.000000in}{-0.019642in}}%
\pgfpathlineto{\pgfqpoint{0.011785in}{0.000000in}}%
\pgfpathlineto{\pgfqpoint{0.000000in}{0.019642in}}%
\pgfpathlineto{\pgfqpoint{-0.011785in}{0.000000in}}%
\pgfpathclose%
\pgfusepath{stroke,fill}%
}%
\begin{pgfscope}%
\pgfsys@transformshift{1.229466in}{0.595812in}%
\pgfsys@useobject{currentmarker}{}%
\end{pgfscope}%
\begin{pgfscope}%
\pgfsys@transformshift{1.229466in}{0.617978in}%
\pgfsys@useobject{currentmarker}{}%
\end{pgfscope}%
\end{pgfscope}%
\begin{pgfscope}%
\pgfpathrectangle{\pgfqpoint{0.550713in}{0.102108in}}{\pgfqpoint{3.194133in}{1.837939in}}%
\pgfusepath{clip}%
\pgfsetrectcap%
\pgfsetroundjoin%
\pgfsetlinewidth{0.752812pt}%
\definecolor{currentstroke}{rgb}{0.000000,0.000000,0.000000}%
\pgfsetstrokecolor{currentstroke}%
\pgfsetdash{}{0pt}%
\pgfpathmoveto{\pgfqpoint{1.469026in}{0.904462in}}%
\pgfpathlineto{\pgfqpoint{1.469026in}{0.632121in}}%
\pgfusepath{stroke}%
\end{pgfscope}%
\begin{pgfscope}%
\pgfpathrectangle{\pgfqpoint{0.550713in}{0.102108in}}{\pgfqpoint{3.194133in}{1.837939in}}%
\pgfusepath{clip}%
\pgfsetrectcap%
\pgfsetroundjoin%
\pgfsetlinewidth{0.752812pt}%
\definecolor{currentstroke}{rgb}{0.000000,0.000000,0.000000}%
\pgfsetstrokecolor{currentstroke}%
\pgfsetdash{}{0pt}%
\pgfpathmoveto{\pgfqpoint{1.469026in}{1.140458in}}%
\pgfpathlineto{\pgfqpoint{1.469026in}{1.474784in}}%
\pgfusepath{stroke}%
\end{pgfscope}%
\begin{pgfscope}%
\pgfpathrectangle{\pgfqpoint{0.550713in}{0.102108in}}{\pgfqpoint{3.194133in}{1.837939in}}%
\pgfusepath{clip}%
\pgfsetrectcap%
\pgfsetroundjoin%
\pgfsetlinewidth{0.752812pt}%
\definecolor{currentstroke}{rgb}{0.000000,0.000000,0.000000}%
\pgfsetstrokecolor{currentstroke}%
\pgfsetdash{}{0pt}%
\pgfpathmoveto{\pgfqpoint{1.429898in}{0.632121in}}%
\pgfpathlineto{\pgfqpoint{1.508154in}{0.632121in}}%
\pgfusepath{stroke}%
\end{pgfscope}%
\begin{pgfscope}%
\pgfpathrectangle{\pgfqpoint{0.550713in}{0.102108in}}{\pgfqpoint{3.194133in}{1.837939in}}%
\pgfusepath{clip}%
\pgfsetrectcap%
\pgfsetroundjoin%
\pgfsetlinewidth{0.752812pt}%
\definecolor{currentstroke}{rgb}{0.000000,0.000000,0.000000}%
\pgfsetstrokecolor{currentstroke}%
\pgfsetdash{}{0pt}%
\pgfpathmoveto{\pgfqpoint{1.429898in}{1.474784in}}%
\pgfpathlineto{\pgfqpoint{1.508154in}{1.474784in}}%
\pgfusepath{stroke}%
\end{pgfscope}%
\begin{pgfscope}%
\pgfpathrectangle{\pgfqpoint{0.550713in}{0.102108in}}{\pgfqpoint{3.194133in}{1.837939in}}%
\pgfusepath{clip}%
\pgfsetrectcap%
\pgfsetroundjoin%
\pgfsetlinewidth{0.752812pt}%
\definecolor{currentstroke}{rgb}{0.000000,0.000000,0.000000}%
\pgfsetstrokecolor{currentstroke}%
\pgfsetdash{}{0pt}%
\pgfpathmoveto{\pgfqpoint{1.628733in}{0.879996in}}%
\pgfpathlineto{\pgfqpoint{1.628733in}{0.686169in}}%
\pgfusepath{stroke}%
\end{pgfscope}%
\begin{pgfscope}%
\pgfpathrectangle{\pgfqpoint{0.550713in}{0.102108in}}{\pgfqpoint{3.194133in}{1.837939in}}%
\pgfusepath{clip}%
\pgfsetrectcap%
\pgfsetroundjoin%
\pgfsetlinewidth{0.752812pt}%
\definecolor{currentstroke}{rgb}{0.000000,0.000000,0.000000}%
\pgfsetstrokecolor{currentstroke}%
\pgfsetdash{}{0pt}%
\pgfpathmoveto{\pgfqpoint{1.628733in}{1.185129in}}%
\pgfpathlineto{\pgfqpoint{1.628733in}{1.605670in}}%
\pgfusepath{stroke}%
\end{pgfscope}%
\begin{pgfscope}%
\pgfpathrectangle{\pgfqpoint{0.550713in}{0.102108in}}{\pgfqpoint{3.194133in}{1.837939in}}%
\pgfusepath{clip}%
\pgfsetrectcap%
\pgfsetroundjoin%
\pgfsetlinewidth{0.752812pt}%
\definecolor{currentstroke}{rgb}{0.000000,0.000000,0.000000}%
\pgfsetstrokecolor{currentstroke}%
\pgfsetdash{}{0pt}%
\pgfpathmoveto{\pgfqpoint{1.589604in}{0.686169in}}%
\pgfpathlineto{\pgfqpoint{1.667861in}{0.686169in}}%
\pgfusepath{stroke}%
\end{pgfscope}%
\begin{pgfscope}%
\pgfpathrectangle{\pgfqpoint{0.550713in}{0.102108in}}{\pgfqpoint{3.194133in}{1.837939in}}%
\pgfusepath{clip}%
\pgfsetrectcap%
\pgfsetroundjoin%
\pgfsetlinewidth{0.752812pt}%
\definecolor{currentstroke}{rgb}{0.000000,0.000000,0.000000}%
\pgfsetstrokecolor{currentstroke}%
\pgfsetdash{}{0pt}%
\pgfpathmoveto{\pgfqpoint{1.589604in}{1.605670in}}%
\pgfpathlineto{\pgfqpoint{1.667861in}{1.605670in}}%
\pgfusepath{stroke}%
\end{pgfscope}%
\begin{pgfscope}%
\pgfpathrectangle{\pgfqpoint{0.550713in}{0.102108in}}{\pgfqpoint{3.194133in}{1.837939in}}%
\pgfusepath{clip}%
\pgfsetrectcap%
\pgfsetroundjoin%
\pgfsetlinewidth{0.752812pt}%
\definecolor{currentstroke}{rgb}{0.000000,0.000000,0.000000}%
\pgfsetstrokecolor{currentstroke}%
\pgfsetdash{}{0pt}%
\pgfpathmoveto{\pgfqpoint{1.868293in}{0.868901in}}%
\pgfpathlineto{\pgfqpoint{1.868293in}{0.605684in}}%
\pgfusepath{stroke}%
\end{pgfscope}%
\begin{pgfscope}%
\pgfpathrectangle{\pgfqpoint{0.550713in}{0.102108in}}{\pgfqpoint{3.194133in}{1.837939in}}%
\pgfusepath{clip}%
\pgfsetrectcap%
\pgfsetroundjoin%
\pgfsetlinewidth{0.752812pt}%
\definecolor{currentstroke}{rgb}{0.000000,0.000000,0.000000}%
\pgfsetstrokecolor{currentstroke}%
\pgfsetdash{}{0pt}%
\pgfpathmoveto{\pgfqpoint{1.868293in}{1.048535in}}%
\pgfpathlineto{\pgfqpoint{1.868293in}{1.202671in}}%
\pgfusepath{stroke}%
\end{pgfscope}%
\begin{pgfscope}%
\pgfpathrectangle{\pgfqpoint{0.550713in}{0.102108in}}{\pgfqpoint{3.194133in}{1.837939in}}%
\pgfusepath{clip}%
\pgfsetrectcap%
\pgfsetroundjoin%
\pgfsetlinewidth{0.752812pt}%
\definecolor{currentstroke}{rgb}{0.000000,0.000000,0.000000}%
\pgfsetstrokecolor{currentstroke}%
\pgfsetdash{}{0pt}%
\pgfpathmoveto{\pgfqpoint{1.829164in}{0.605684in}}%
\pgfpathlineto{\pgfqpoint{1.907421in}{0.605684in}}%
\pgfusepath{stroke}%
\end{pgfscope}%
\begin{pgfscope}%
\pgfpathrectangle{\pgfqpoint{0.550713in}{0.102108in}}{\pgfqpoint{3.194133in}{1.837939in}}%
\pgfusepath{clip}%
\pgfsetrectcap%
\pgfsetroundjoin%
\pgfsetlinewidth{0.752812pt}%
\definecolor{currentstroke}{rgb}{0.000000,0.000000,0.000000}%
\pgfsetstrokecolor{currentstroke}%
\pgfsetdash{}{0pt}%
\pgfpathmoveto{\pgfqpoint{1.829164in}{1.202671in}}%
\pgfpathlineto{\pgfqpoint{1.907421in}{1.202671in}}%
\pgfusepath{stroke}%
\end{pgfscope}%
\begin{pgfscope}%
\pgfpathrectangle{\pgfqpoint{0.550713in}{0.102108in}}{\pgfqpoint{3.194133in}{1.837939in}}%
\pgfusepath{clip}%
\pgfsetrectcap%
\pgfsetroundjoin%
\pgfsetlinewidth{0.752812pt}%
\definecolor{currentstroke}{rgb}{0.000000,0.000000,0.000000}%
\pgfsetstrokecolor{currentstroke}%
\pgfsetdash{}{0pt}%
\pgfpathmoveto{\pgfqpoint{2.027999in}{0.898254in}}%
\pgfpathlineto{\pgfqpoint{2.027999in}{0.635385in}}%
\pgfusepath{stroke}%
\end{pgfscope}%
\begin{pgfscope}%
\pgfpathrectangle{\pgfqpoint{0.550713in}{0.102108in}}{\pgfqpoint{3.194133in}{1.837939in}}%
\pgfusepath{clip}%
\pgfsetrectcap%
\pgfsetroundjoin%
\pgfsetlinewidth{0.752812pt}%
\definecolor{currentstroke}{rgb}{0.000000,0.000000,0.000000}%
\pgfsetstrokecolor{currentstroke}%
\pgfsetdash{}{0pt}%
\pgfpathmoveto{\pgfqpoint{2.027999in}{1.109605in}}%
\pgfpathlineto{\pgfqpoint{2.027999in}{1.296754in}}%
\pgfusepath{stroke}%
\end{pgfscope}%
\begin{pgfscope}%
\pgfpathrectangle{\pgfqpoint{0.550713in}{0.102108in}}{\pgfqpoint{3.194133in}{1.837939in}}%
\pgfusepath{clip}%
\pgfsetrectcap%
\pgfsetroundjoin%
\pgfsetlinewidth{0.752812pt}%
\definecolor{currentstroke}{rgb}{0.000000,0.000000,0.000000}%
\pgfsetstrokecolor{currentstroke}%
\pgfsetdash{}{0pt}%
\pgfpathmoveto{\pgfqpoint{1.988871in}{0.635385in}}%
\pgfpathlineto{\pgfqpoint{2.067127in}{0.635385in}}%
\pgfusepath{stroke}%
\end{pgfscope}%
\begin{pgfscope}%
\pgfpathrectangle{\pgfqpoint{0.550713in}{0.102108in}}{\pgfqpoint{3.194133in}{1.837939in}}%
\pgfusepath{clip}%
\pgfsetrectcap%
\pgfsetroundjoin%
\pgfsetlinewidth{0.752812pt}%
\definecolor{currentstroke}{rgb}{0.000000,0.000000,0.000000}%
\pgfsetstrokecolor{currentstroke}%
\pgfsetdash{}{0pt}%
\pgfpathmoveto{\pgfqpoint{1.988871in}{1.296754in}}%
\pgfpathlineto{\pgfqpoint{2.067127in}{1.296754in}}%
\pgfusepath{stroke}%
\end{pgfscope}%
\begin{pgfscope}%
\pgfpathrectangle{\pgfqpoint{0.550713in}{0.102108in}}{\pgfqpoint{3.194133in}{1.837939in}}%
\pgfusepath{clip}%
\pgfsetrectcap%
\pgfsetroundjoin%
\pgfsetlinewidth{0.752812pt}%
\definecolor{currentstroke}{rgb}{0.000000,0.000000,0.000000}%
\pgfsetstrokecolor{currentstroke}%
\pgfsetdash{}{0pt}%
\pgfpathmoveto{\pgfqpoint{2.267559in}{1.081128in}}%
\pgfpathlineto{\pgfqpoint{2.267559in}{1.030586in}}%
\pgfusepath{stroke}%
\end{pgfscope}%
\begin{pgfscope}%
\pgfpathrectangle{\pgfqpoint{0.550713in}{0.102108in}}{\pgfqpoint{3.194133in}{1.837939in}}%
\pgfusepath{clip}%
\pgfsetrectcap%
\pgfsetroundjoin%
\pgfsetlinewidth{0.752812pt}%
\definecolor{currentstroke}{rgb}{0.000000,0.000000,0.000000}%
\pgfsetstrokecolor{currentstroke}%
\pgfsetdash{}{0pt}%
\pgfpathmoveto{\pgfqpoint{2.267559in}{1.302118in}}%
\pgfpathlineto{\pgfqpoint{2.267559in}{1.474119in}}%
\pgfusepath{stroke}%
\end{pgfscope}%
\begin{pgfscope}%
\pgfpathrectangle{\pgfqpoint{0.550713in}{0.102108in}}{\pgfqpoint{3.194133in}{1.837939in}}%
\pgfusepath{clip}%
\pgfsetrectcap%
\pgfsetroundjoin%
\pgfsetlinewidth{0.752812pt}%
\definecolor{currentstroke}{rgb}{0.000000,0.000000,0.000000}%
\pgfsetstrokecolor{currentstroke}%
\pgfsetdash{}{0pt}%
\pgfpathmoveto{\pgfqpoint{2.228431in}{1.030586in}}%
\pgfpathlineto{\pgfqpoint{2.306687in}{1.030586in}}%
\pgfusepath{stroke}%
\end{pgfscope}%
\begin{pgfscope}%
\pgfpathrectangle{\pgfqpoint{0.550713in}{0.102108in}}{\pgfqpoint{3.194133in}{1.837939in}}%
\pgfusepath{clip}%
\pgfsetrectcap%
\pgfsetroundjoin%
\pgfsetlinewidth{0.752812pt}%
\definecolor{currentstroke}{rgb}{0.000000,0.000000,0.000000}%
\pgfsetstrokecolor{currentstroke}%
\pgfsetdash{}{0pt}%
\pgfpathmoveto{\pgfqpoint{2.228431in}{1.474119in}}%
\pgfpathlineto{\pgfqpoint{2.306687in}{1.474119in}}%
\pgfusepath{stroke}%
\end{pgfscope}%
\begin{pgfscope}%
\pgfpathrectangle{\pgfqpoint{0.550713in}{0.102108in}}{\pgfqpoint{3.194133in}{1.837939in}}%
\pgfusepath{clip}%
\pgfsetrectcap%
\pgfsetroundjoin%
\pgfsetlinewidth{0.752812pt}%
\definecolor{currentstroke}{rgb}{0.000000,0.000000,0.000000}%
\pgfsetstrokecolor{currentstroke}%
\pgfsetdash{}{0pt}%
\pgfpathmoveto{\pgfqpoint{2.427266in}{1.311222in}}%
\pgfpathlineto{\pgfqpoint{2.427266in}{1.221542in}}%
\pgfusepath{stroke}%
\end{pgfscope}%
\begin{pgfscope}%
\pgfpathrectangle{\pgfqpoint{0.550713in}{0.102108in}}{\pgfqpoint{3.194133in}{1.837939in}}%
\pgfusepath{clip}%
\pgfsetrectcap%
\pgfsetroundjoin%
\pgfsetlinewidth{0.752812pt}%
\definecolor{currentstroke}{rgb}{0.000000,0.000000,0.000000}%
\pgfsetstrokecolor{currentstroke}%
\pgfsetdash{}{0pt}%
\pgfpathmoveto{\pgfqpoint{2.427266in}{1.511157in}}%
\pgfpathlineto{\pgfqpoint{2.427266in}{1.770764in}}%
\pgfusepath{stroke}%
\end{pgfscope}%
\begin{pgfscope}%
\pgfpathrectangle{\pgfqpoint{0.550713in}{0.102108in}}{\pgfqpoint{3.194133in}{1.837939in}}%
\pgfusepath{clip}%
\pgfsetrectcap%
\pgfsetroundjoin%
\pgfsetlinewidth{0.752812pt}%
\definecolor{currentstroke}{rgb}{0.000000,0.000000,0.000000}%
\pgfsetstrokecolor{currentstroke}%
\pgfsetdash{}{0pt}%
\pgfpathmoveto{\pgfqpoint{2.388138in}{1.221542in}}%
\pgfpathlineto{\pgfqpoint{2.466394in}{1.221542in}}%
\pgfusepath{stroke}%
\end{pgfscope}%
\begin{pgfscope}%
\pgfpathrectangle{\pgfqpoint{0.550713in}{0.102108in}}{\pgfqpoint{3.194133in}{1.837939in}}%
\pgfusepath{clip}%
\pgfsetrectcap%
\pgfsetroundjoin%
\pgfsetlinewidth{0.752812pt}%
\definecolor{currentstroke}{rgb}{0.000000,0.000000,0.000000}%
\pgfsetstrokecolor{currentstroke}%
\pgfsetdash{}{0pt}%
\pgfpathmoveto{\pgfqpoint{2.388138in}{1.770764in}}%
\pgfpathlineto{\pgfqpoint{2.466394in}{1.770764in}}%
\pgfusepath{stroke}%
\end{pgfscope}%
\begin{pgfscope}%
\pgfpathrectangle{\pgfqpoint{0.550713in}{0.102108in}}{\pgfqpoint{3.194133in}{1.837939in}}%
\pgfusepath{clip}%
\pgfsetbuttcap%
\pgfsetmiterjoin%
\definecolor{currentfill}{rgb}{0.000000,0.000000,0.000000}%
\pgfsetfillcolor{currentfill}%
\pgfsetlinewidth{1.003750pt}%
\definecolor{currentstroke}{rgb}{0.000000,0.000000,0.000000}%
\pgfsetstrokecolor{currentstroke}%
\pgfsetdash{}{0pt}%
\pgfsys@defobject{currentmarker}{\pgfqpoint{-0.011785in}{-0.019642in}}{\pgfqpoint{0.011785in}{0.019642in}}{%
\pgfpathmoveto{\pgfqpoint{-0.000000in}{-0.019642in}}%
\pgfpathlineto{\pgfqpoint{0.011785in}{0.000000in}}%
\pgfpathlineto{\pgfqpoint{0.000000in}{0.019642in}}%
\pgfpathlineto{\pgfqpoint{-0.011785in}{0.000000in}}%
\pgfpathclose%
\pgfusepath{stroke,fill}%
}%
\begin{pgfscope}%
\pgfsys@transformshift{2.427266in}{1.937401in}%
\pgfsys@useobject{currentmarker}{}%
\end{pgfscope}%
\begin{pgfscope}%
\pgfsys@transformshift{2.427266in}{1.875098in}%
\pgfsys@useobject{currentmarker}{}%
\end{pgfscope}%
\begin{pgfscope}%
\pgfsys@transformshift{2.427266in}{1.829117in}%
\pgfsys@useobject{currentmarker}{}%
\end{pgfscope}%
\end{pgfscope}%
\begin{pgfscope}%
\pgfpathrectangle{\pgfqpoint{0.550713in}{0.102108in}}{\pgfqpoint{3.194133in}{1.837939in}}%
\pgfusepath{clip}%
\pgfsetrectcap%
\pgfsetroundjoin%
\pgfsetlinewidth{0.752812pt}%
\definecolor{currentstroke}{rgb}{0.000000,0.000000,0.000000}%
\pgfsetstrokecolor{currentstroke}%
\pgfsetdash{}{0pt}%
\pgfpathmoveto{\pgfqpoint{2.666826in}{0.964338in}}%
\pgfpathlineto{\pgfqpoint{2.666826in}{0.854642in}}%
\pgfusepath{stroke}%
\end{pgfscope}%
\begin{pgfscope}%
\pgfpathrectangle{\pgfqpoint{0.550713in}{0.102108in}}{\pgfqpoint{3.194133in}{1.837939in}}%
\pgfusepath{clip}%
\pgfsetrectcap%
\pgfsetroundjoin%
\pgfsetlinewidth{0.752812pt}%
\definecolor{currentstroke}{rgb}{0.000000,0.000000,0.000000}%
\pgfsetstrokecolor{currentstroke}%
\pgfsetdash{}{0pt}%
\pgfpathmoveto{\pgfqpoint{2.666826in}{1.057308in}}%
\pgfpathlineto{\pgfqpoint{2.666826in}{1.192992in}}%
\pgfusepath{stroke}%
\end{pgfscope}%
\begin{pgfscope}%
\pgfpathrectangle{\pgfqpoint{0.550713in}{0.102108in}}{\pgfqpoint{3.194133in}{1.837939in}}%
\pgfusepath{clip}%
\pgfsetrectcap%
\pgfsetroundjoin%
\pgfsetlinewidth{0.752812pt}%
\definecolor{currentstroke}{rgb}{0.000000,0.000000,0.000000}%
\pgfsetstrokecolor{currentstroke}%
\pgfsetdash{}{0pt}%
\pgfpathmoveto{\pgfqpoint{2.627698in}{0.854642in}}%
\pgfpathlineto{\pgfqpoint{2.705954in}{0.854642in}}%
\pgfusepath{stroke}%
\end{pgfscope}%
\begin{pgfscope}%
\pgfpathrectangle{\pgfqpoint{0.550713in}{0.102108in}}{\pgfqpoint{3.194133in}{1.837939in}}%
\pgfusepath{clip}%
\pgfsetrectcap%
\pgfsetroundjoin%
\pgfsetlinewidth{0.752812pt}%
\definecolor{currentstroke}{rgb}{0.000000,0.000000,0.000000}%
\pgfsetstrokecolor{currentstroke}%
\pgfsetdash{}{0pt}%
\pgfpathmoveto{\pgfqpoint{2.627698in}{1.192992in}}%
\pgfpathlineto{\pgfqpoint{2.705954in}{1.192992in}}%
\pgfusepath{stroke}%
\end{pgfscope}%
\begin{pgfscope}%
\pgfpathrectangle{\pgfqpoint{0.550713in}{0.102108in}}{\pgfqpoint{3.194133in}{1.837939in}}%
\pgfusepath{clip}%
\pgfsetrectcap%
\pgfsetroundjoin%
\pgfsetlinewidth{0.752812pt}%
\definecolor{currentstroke}{rgb}{0.000000,0.000000,0.000000}%
\pgfsetstrokecolor{currentstroke}%
\pgfsetdash{}{0pt}%
\pgfpathmoveto{\pgfqpoint{2.826532in}{1.114258in}}%
\pgfpathlineto{\pgfqpoint{2.826532in}{1.059716in}}%
\pgfusepath{stroke}%
\end{pgfscope}%
\begin{pgfscope}%
\pgfpathrectangle{\pgfqpoint{0.550713in}{0.102108in}}{\pgfqpoint{3.194133in}{1.837939in}}%
\pgfusepath{clip}%
\pgfsetrectcap%
\pgfsetroundjoin%
\pgfsetlinewidth{0.752812pt}%
\definecolor{currentstroke}{rgb}{0.000000,0.000000,0.000000}%
\pgfsetstrokecolor{currentstroke}%
\pgfsetdash{}{0pt}%
\pgfpathmoveto{\pgfqpoint{2.826532in}{1.302923in}}%
\pgfpathlineto{\pgfqpoint{2.826532in}{1.485038in}}%
\pgfusepath{stroke}%
\end{pgfscope}%
\begin{pgfscope}%
\pgfpathrectangle{\pgfqpoint{0.550713in}{0.102108in}}{\pgfqpoint{3.194133in}{1.837939in}}%
\pgfusepath{clip}%
\pgfsetrectcap%
\pgfsetroundjoin%
\pgfsetlinewidth{0.752812pt}%
\definecolor{currentstroke}{rgb}{0.000000,0.000000,0.000000}%
\pgfsetstrokecolor{currentstroke}%
\pgfsetdash{}{0pt}%
\pgfpathmoveto{\pgfqpoint{2.787404in}{1.059716in}}%
\pgfpathlineto{\pgfqpoint{2.865661in}{1.059716in}}%
\pgfusepath{stroke}%
\end{pgfscope}%
\begin{pgfscope}%
\pgfpathrectangle{\pgfqpoint{0.550713in}{0.102108in}}{\pgfqpoint{3.194133in}{1.837939in}}%
\pgfusepath{clip}%
\pgfsetrectcap%
\pgfsetroundjoin%
\pgfsetlinewidth{0.752812pt}%
\definecolor{currentstroke}{rgb}{0.000000,0.000000,0.000000}%
\pgfsetstrokecolor{currentstroke}%
\pgfsetdash{}{0pt}%
\pgfpathmoveto{\pgfqpoint{2.787404in}{1.485038in}}%
\pgfpathlineto{\pgfqpoint{2.865661in}{1.485038in}}%
\pgfusepath{stroke}%
\end{pgfscope}%
\begin{pgfscope}%
\pgfpathrectangle{\pgfqpoint{0.550713in}{0.102108in}}{\pgfqpoint{3.194133in}{1.837939in}}%
\pgfusepath{clip}%
\pgfsetrectcap%
\pgfsetroundjoin%
\pgfsetlinewidth{0.752812pt}%
\definecolor{currentstroke}{rgb}{0.000000,0.000000,0.000000}%
\pgfsetstrokecolor{currentstroke}%
\pgfsetdash{}{0pt}%
\pgfpathmoveto{\pgfqpoint{3.066092in}{0.795273in}}%
\pgfpathlineto{\pgfqpoint{3.066092in}{0.629893in}}%
\pgfusepath{stroke}%
\end{pgfscope}%
\begin{pgfscope}%
\pgfpathrectangle{\pgfqpoint{0.550713in}{0.102108in}}{\pgfqpoint{3.194133in}{1.837939in}}%
\pgfusepath{clip}%
\pgfsetrectcap%
\pgfsetroundjoin%
\pgfsetlinewidth{0.752812pt}%
\definecolor{currentstroke}{rgb}{0.000000,0.000000,0.000000}%
\pgfsetstrokecolor{currentstroke}%
\pgfsetdash{}{0pt}%
\pgfpathmoveto{\pgfqpoint{3.066092in}{0.930978in}}%
\pgfpathlineto{\pgfqpoint{3.066092in}{1.036146in}}%
\pgfusepath{stroke}%
\end{pgfscope}%
\begin{pgfscope}%
\pgfpathrectangle{\pgfqpoint{0.550713in}{0.102108in}}{\pgfqpoint{3.194133in}{1.837939in}}%
\pgfusepath{clip}%
\pgfsetrectcap%
\pgfsetroundjoin%
\pgfsetlinewidth{0.752812pt}%
\definecolor{currentstroke}{rgb}{0.000000,0.000000,0.000000}%
\pgfsetstrokecolor{currentstroke}%
\pgfsetdash{}{0pt}%
\pgfpathmoveto{\pgfqpoint{3.026964in}{0.629893in}}%
\pgfpathlineto{\pgfqpoint{3.105221in}{0.629893in}}%
\pgfusepath{stroke}%
\end{pgfscope}%
\begin{pgfscope}%
\pgfpathrectangle{\pgfqpoint{0.550713in}{0.102108in}}{\pgfqpoint{3.194133in}{1.837939in}}%
\pgfusepath{clip}%
\pgfsetrectcap%
\pgfsetroundjoin%
\pgfsetlinewidth{0.752812pt}%
\definecolor{currentstroke}{rgb}{0.000000,0.000000,0.000000}%
\pgfsetstrokecolor{currentstroke}%
\pgfsetdash{}{0pt}%
\pgfpathmoveto{\pgfqpoint{3.026964in}{1.036146in}}%
\pgfpathlineto{\pgfqpoint{3.105221in}{1.036146in}}%
\pgfusepath{stroke}%
\end{pgfscope}%
\begin{pgfscope}%
\pgfpathrectangle{\pgfqpoint{0.550713in}{0.102108in}}{\pgfqpoint{3.194133in}{1.837939in}}%
\pgfusepath{clip}%
\pgfsetrectcap%
\pgfsetroundjoin%
\pgfsetlinewidth{0.752812pt}%
\definecolor{currentstroke}{rgb}{0.000000,0.000000,0.000000}%
\pgfsetstrokecolor{currentstroke}%
\pgfsetdash{}{0pt}%
\pgfpathmoveto{\pgfqpoint{3.225799in}{0.848960in}}%
\pgfpathlineto{\pgfqpoint{3.225799in}{0.724339in}}%
\pgfusepath{stroke}%
\end{pgfscope}%
\begin{pgfscope}%
\pgfpathrectangle{\pgfqpoint{0.550713in}{0.102108in}}{\pgfqpoint{3.194133in}{1.837939in}}%
\pgfusepath{clip}%
\pgfsetrectcap%
\pgfsetroundjoin%
\pgfsetlinewidth{0.752812pt}%
\definecolor{currentstroke}{rgb}{0.000000,0.000000,0.000000}%
\pgfsetstrokecolor{currentstroke}%
\pgfsetdash{}{0pt}%
\pgfpathmoveto{\pgfqpoint{3.225799in}{0.981084in}}%
\pgfpathlineto{\pgfqpoint{3.225799in}{1.082594in}}%
\pgfusepath{stroke}%
\end{pgfscope}%
\begin{pgfscope}%
\pgfpathrectangle{\pgfqpoint{0.550713in}{0.102108in}}{\pgfqpoint{3.194133in}{1.837939in}}%
\pgfusepath{clip}%
\pgfsetrectcap%
\pgfsetroundjoin%
\pgfsetlinewidth{0.752812pt}%
\definecolor{currentstroke}{rgb}{0.000000,0.000000,0.000000}%
\pgfsetstrokecolor{currentstroke}%
\pgfsetdash{}{0pt}%
\pgfpathmoveto{\pgfqpoint{3.186671in}{0.724339in}}%
\pgfpathlineto{\pgfqpoint{3.264927in}{0.724339in}}%
\pgfusepath{stroke}%
\end{pgfscope}%
\begin{pgfscope}%
\pgfpathrectangle{\pgfqpoint{0.550713in}{0.102108in}}{\pgfqpoint{3.194133in}{1.837939in}}%
\pgfusepath{clip}%
\pgfsetrectcap%
\pgfsetroundjoin%
\pgfsetlinewidth{0.752812pt}%
\definecolor{currentstroke}{rgb}{0.000000,0.000000,0.000000}%
\pgfsetstrokecolor{currentstroke}%
\pgfsetdash{}{0pt}%
\pgfpathmoveto{\pgfqpoint{3.186671in}{1.082594in}}%
\pgfpathlineto{\pgfqpoint{3.264927in}{1.082594in}}%
\pgfusepath{stroke}%
\end{pgfscope}%
\begin{pgfscope}%
\pgfpathrectangle{\pgfqpoint{0.550713in}{0.102108in}}{\pgfqpoint{3.194133in}{1.837939in}}%
\pgfusepath{clip}%
\pgfsetbuttcap%
\pgfsetmiterjoin%
\definecolor{currentfill}{rgb}{0.000000,0.000000,0.000000}%
\pgfsetfillcolor{currentfill}%
\pgfsetlinewidth{1.003750pt}%
\definecolor{currentstroke}{rgb}{0.000000,0.000000,0.000000}%
\pgfsetstrokecolor{currentstroke}%
\pgfsetdash{}{0pt}%
\pgfsys@defobject{currentmarker}{\pgfqpoint{-0.011785in}{-0.019642in}}{\pgfqpoint{0.011785in}{0.019642in}}{%
\pgfpathmoveto{\pgfqpoint{-0.000000in}{-0.019642in}}%
\pgfpathlineto{\pgfqpoint{0.011785in}{0.000000in}}%
\pgfpathlineto{\pgfqpoint{0.000000in}{0.019642in}}%
\pgfpathlineto{\pgfqpoint{-0.011785in}{0.000000in}}%
\pgfpathclose%
\pgfusepath{stroke,fill}%
}%
\begin{pgfscope}%
\pgfsys@transformshift{3.225799in}{0.645493in}%
\pgfsys@useobject{currentmarker}{}%
\end{pgfscope}%
\begin{pgfscope}%
\pgfsys@transformshift{3.225799in}{0.617521in}%
\pgfsys@useobject{currentmarker}{}%
\end{pgfscope}%
\begin{pgfscope}%
\pgfsys@transformshift{3.225799in}{0.615851in}%
\pgfsys@useobject{currentmarker}{}%
\end{pgfscope}%
\begin{pgfscope}%
\pgfsys@transformshift{3.225799in}{1.275903in}%
\pgfsys@useobject{currentmarker}{}%
\end{pgfscope}%
\end{pgfscope}%
\begin{pgfscope}%
\pgfpathrectangle{\pgfqpoint{0.550713in}{0.102108in}}{\pgfqpoint{3.194133in}{1.837939in}}%
\pgfusepath{clip}%
\pgfsetrectcap%
\pgfsetroundjoin%
\pgfsetlinewidth{0.752812pt}%
\definecolor{currentstroke}{rgb}{0.000000,0.000000,0.000000}%
\pgfsetstrokecolor{currentstroke}%
\pgfsetdash{}{0pt}%
\pgfpathmoveto{\pgfqpoint{3.465359in}{0.884310in}}%
\pgfpathlineto{\pgfqpoint{3.465359in}{0.742991in}}%
\pgfusepath{stroke}%
\end{pgfscope}%
\begin{pgfscope}%
\pgfpathrectangle{\pgfqpoint{0.550713in}{0.102108in}}{\pgfqpoint{3.194133in}{1.837939in}}%
\pgfusepath{clip}%
\pgfsetrectcap%
\pgfsetroundjoin%
\pgfsetlinewidth{0.752812pt}%
\definecolor{currentstroke}{rgb}{0.000000,0.000000,0.000000}%
\pgfsetstrokecolor{currentstroke}%
\pgfsetdash{}{0pt}%
\pgfpathmoveto{\pgfqpoint{3.465359in}{0.988146in}}%
\pgfpathlineto{\pgfqpoint{3.465359in}{1.063072in}}%
\pgfusepath{stroke}%
\end{pgfscope}%
\begin{pgfscope}%
\pgfpathrectangle{\pgfqpoint{0.550713in}{0.102108in}}{\pgfqpoint{3.194133in}{1.837939in}}%
\pgfusepath{clip}%
\pgfsetrectcap%
\pgfsetroundjoin%
\pgfsetlinewidth{0.752812pt}%
\definecolor{currentstroke}{rgb}{0.000000,0.000000,0.000000}%
\pgfsetstrokecolor{currentstroke}%
\pgfsetdash{}{0pt}%
\pgfpathmoveto{\pgfqpoint{3.426231in}{0.742991in}}%
\pgfpathlineto{\pgfqpoint{3.504487in}{0.742991in}}%
\pgfusepath{stroke}%
\end{pgfscope}%
\begin{pgfscope}%
\pgfpathrectangle{\pgfqpoint{0.550713in}{0.102108in}}{\pgfqpoint{3.194133in}{1.837939in}}%
\pgfusepath{clip}%
\pgfsetrectcap%
\pgfsetroundjoin%
\pgfsetlinewidth{0.752812pt}%
\definecolor{currentstroke}{rgb}{0.000000,0.000000,0.000000}%
\pgfsetstrokecolor{currentstroke}%
\pgfsetdash{}{0pt}%
\pgfpathmoveto{\pgfqpoint{3.426231in}{1.063072in}}%
\pgfpathlineto{\pgfqpoint{3.504487in}{1.063072in}}%
\pgfusepath{stroke}%
\end{pgfscope}%
\begin{pgfscope}%
\pgfpathrectangle{\pgfqpoint{0.550713in}{0.102108in}}{\pgfqpoint{3.194133in}{1.837939in}}%
\pgfusepath{clip}%
\pgfsetrectcap%
\pgfsetroundjoin%
\pgfsetlinewidth{0.752812pt}%
\definecolor{currentstroke}{rgb}{0.000000,0.000000,0.000000}%
\pgfsetstrokecolor{currentstroke}%
\pgfsetdash{}{0pt}%
\pgfpathmoveto{\pgfqpoint{3.625066in}{0.879574in}}%
\pgfpathlineto{\pgfqpoint{3.625066in}{0.758148in}}%
\pgfusepath{stroke}%
\end{pgfscope}%
\begin{pgfscope}%
\pgfpathrectangle{\pgfqpoint{0.550713in}{0.102108in}}{\pgfqpoint{3.194133in}{1.837939in}}%
\pgfusepath{clip}%
\pgfsetrectcap%
\pgfsetroundjoin%
\pgfsetlinewidth{0.752812pt}%
\definecolor{currentstroke}{rgb}{0.000000,0.000000,0.000000}%
\pgfsetstrokecolor{currentstroke}%
\pgfsetdash{}{0pt}%
\pgfpathmoveto{\pgfqpoint{3.625066in}{0.983152in}}%
\pgfpathlineto{\pgfqpoint{3.625066in}{1.046735in}}%
\pgfusepath{stroke}%
\end{pgfscope}%
\begin{pgfscope}%
\pgfpathrectangle{\pgfqpoint{0.550713in}{0.102108in}}{\pgfqpoint{3.194133in}{1.837939in}}%
\pgfusepath{clip}%
\pgfsetrectcap%
\pgfsetroundjoin%
\pgfsetlinewidth{0.752812pt}%
\definecolor{currentstroke}{rgb}{0.000000,0.000000,0.000000}%
\pgfsetstrokecolor{currentstroke}%
\pgfsetdash{}{0pt}%
\pgfpathmoveto{\pgfqpoint{3.585938in}{0.758148in}}%
\pgfpathlineto{\pgfqpoint{3.664194in}{0.758148in}}%
\pgfusepath{stroke}%
\end{pgfscope}%
\begin{pgfscope}%
\pgfpathrectangle{\pgfqpoint{0.550713in}{0.102108in}}{\pgfqpoint{3.194133in}{1.837939in}}%
\pgfusepath{clip}%
\pgfsetrectcap%
\pgfsetroundjoin%
\pgfsetlinewidth{0.752812pt}%
\definecolor{currentstroke}{rgb}{0.000000,0.000000,0.000000}%
\pgfsetstrokecolor{currentstroke}%
\pgfsetdash{}{0pt}%
\pgfpathmoveto{\pgfqpoint{3.585938in}{1.046735in}}%
\pgfpathlineto{\pgfqpoint{3.664194in}{1.046735in}}%
\pgfusepath{stroke}%
\end{pgfscope}%
\begin{pgfscope}%
\pgfpathrectangle{\pgfqpoint{0.550713in}{0.102108in}}{\pgfqpoint{3.194133in}{1.837939in}}%
\pgfusepath{clip}%
\pgfsetbuttcap%
\pgfsetmiterjoin%
\definecolor{currentfill}{rgb}{0.000000,0.000000,0.000000}%
\pgfsetfillcolor{currentfill}%
\pgfsetlinewidth{1.003750pt}%
\definecolor{currentstroke}{rgb}{0.000000,0.000000,0.000000}%
\pgfsetstrokecolor{currentstroke}%
\pgfsetdash{}{0pt}%
\pgfsys@defobject{currentmarker}{\pgfqpoint{-0.011785in}{-0.019642in}}{\pgfqpoint{0.011785in}{0.019642in}}{%
\pgfpathmoveto{\pgfqpoint{-0.000000in}{-0.019642in}}%
\pgfpathlineto{\pgfqpoint{0.011785in}{0.000000in}}%
\pgfpathlineto{\pgfqpoint{0.000000in}{0.019642in}}%
\pgfpathlineto{\pgfqpoint{-0.011785in}{0.000000in}}%
\pgfpathclose%
\pgfusepath{stroke,fill}%
}%
\begin{pgfscope}%
\pgfsys@transformshift{3.625066in}{0.714721in}%
\pgfsys@useobject{currentmarker}{}%
\end{pgfscope}%
\end{pgfscope}%
\begin{pgfscope}%
\pgfpathrectangle{\pgfqpoint{0.550713in}{0.102108in}}{\pgfqpoint{3.194133in}{1.837939in}}%
\pgfusepath{clip}%
\pgfsetrectcap%
\pgfsetroundjoin%
\pgfsetlinewidth{0.752812pt}%
\definecolor{currentstroke}{rgb}{0.000000,0.000000,0.000000}%
\pgfsetstrokecolor{currentstroke}%
\pgfsetdash{}{0pt}%
\pgfpathmoveto{\pgfqpoint{0.592236in}{0.921473in}}%
\pgfpathlineto{\pgfqpoint{0.748749in}{0.921473in}}%
\pgfusepath{stroke}%
\end{pgfscope}%
\begin{pgfscope}%
\pgfpathrectangle{\pgfqpoint{0.550713in}{0.102108in}}{\pgfqpoint{3.194133in}{1.837939in}}%
\pgfusepath{clip}%
\pgfsetbuttcap%
\pgfsetroundjoin%
\definecolor{currentfill}{rgb}{1.000000,1.000000,1.000000}%
\pgfsetfillcolor{currentfill}%
\pgfsetlinewidth{1.003750pt}%
\definecolor{currentstroke}{rgb}{0.000000,0.000000,0.000000}%
\pgfsetstrokecolor{currentstroke}%
\pgfsetdash{}{0pt}%
\pgfsys@defobject{currentmarker}{\pgfqpoint{-0.027778in}{-0.027778in}}{\pgfqpoint{0.027778in}{0.027778in}}{%
\pgfpathmoveto{\pgfqpoint{0.000000in}{-0.027778in}}%
\pgfpathcurveto{\pgfqpoint{0.007367in}{-0.027778in}}{\pgfqpoint{0.014433in}{-0.024851in}}{\pgfqpoint{0.019642in}{-0.019642in}}%
\pgfpathcurveto{\pgfqpoint{0.024851in}{-0.014433in}}{\pgfqpoint{0.027778in}{-0.007367in}}{\pgfqpoint{0.027778in}{0.000000in}}%
\pgfpathcurveto{\pgfqpoint{0.027778in}{0.007367in}}{\pgfqpoint{0.024851in}{0.014433in}}{\pgfqpoint{0.019642in}{0.019642in}}%
\pgfpathcurveto{\pgfqpoint{0.014433in}{0.024851in}}{\pgfqpoint{0.007367in}{0.027778in}}{\pgfqpoint{0.000000in}{0.027778in}}%
\pgfpathcurveto{\pgfqpoint{-0.007367in}{0.027778in}}{\pgfqpoint{-0.014433in}{0.024851in}}{\pgfqpoint{-0.019642in}{0.019642in}}%
\pgfpathcurveto{\pgfqpoint{-0.024851in}{0.014433in}}{\pgfqpoint{-0.027778in}{0.007367in}}{\pgfqpoint{-0.027778in}{0.000000in}}%
\pgfpathcurveto{\pgfqpoint{-0.027778in}{-0.007367in}}{\pgfqpoint{-0.024851in}{-0.014433in}}{\pgfqpoint{-0.019642in}{-0.019642in}}%
\pgfpathcurveto{\pgfqpoint{-0.014433in}{-0.024851in}}{\pgfqpoint{-0.007367in}{-0.027778in}}{\pgfqpoint{0.000000in}{-0.027778in}}%
\pgfpathclose%
\pgfusepath{stroke,fill}%
}%
\begin{pgfscope}%
\pgfsys@transformshift{0.670493in}{0.879309in}%
\pgfsys@useobject{currentmarker}{}%
\end{pgfscope}%
\end{pgfscope}%
\begin{pgfscope}%
\pgfpathrectangle{\pgfqpoint{0.550713in}{0.102108in}}{\pgfqpoint{3.194133in}{1.837939in}}%
\pgfusepath{clip}%
\pgfsetrectcap%
\pgfsetroundjoin%
\pgfsetlinewidth{0.752812pt}%
\definecolor{currentstroke}{rgb}{0.000000,0.000000,0.000000}%
\pgfsetstrokecolor{currentstroke}%
\pgfsetdash{}{0pt}%
\pgfpathmoveto{\pgfqpoint{0.751943in}{0.855453in}}%
\pgfpathlineto{\pgfqpoint{0.908456in}{0.855453in}}%
\pgfusepath{stroke}%
\end{pgfscope}%
\begin{pgfscope}%
\pgfpathrectangle{\pgfqpoint{0.550713in}{0.102108in}}{\pgfqpoint{3.194133in}{1.837939in}}%
\pgfusepath{clip}%
\pgfsetbuttcap%
\pgfsetroundjoin%
\definecolor{currentfill}{rgb}{1.000000,1.000000,1.000000}%
\pgfsetfillcolor{currentfill}%
\pgfsetlinewidth{1.003750pt}%
\definecolor{currentstroke}{rgb}{0.000000,0.000000,0.000000}%
\pgfsetstrokecolor{currentstroke}%
\pgfsetdash{}{0pt}%
\pgfsys@defobject{currentmarker}{\pgfqpoint{-0.027778in}{-0.027778in}}{\pgfqpoint{0.027778in}{0.027778in}}{%
\pgfpathmoveto{\pgfqpoint{0.000000in}{-0.027778in}}%
\pgfpathcurveto{\pgfqpoint{0.007367in}{-0.027778in}}{\pgfqpoint{0.014433in}{-0.024851in}}{\pgfqpoint{0.019642in}{-0.019642in}}%
\pgfpathcurveto{\pgfqpoint{0.024851in}{-0.014433in}}{\pgfqpoint{0.027778in}{-0.007367in}}{\pgfqpoint{0.027778in}{0.000000in}}%
\pgfpathcurveto{\pgfqpoint{0.027778in}{0.007367in}}{\pgfqpoint{0.024851in}{0.014433in}}{\pgfqpoint{0.019642in}{0.019642in}}%
\pgfpathcurveto{\pgfqpoint{0.014433in}{0.024851in}}{\pgfqpoint{0.007367in}{0.027778in}}{\pgfqpoint{0.000000in}{0.027778in}}%
\pgfpathcurveto{\pgfqpoint{-0.007367in}{0.027778in}}{\pgfqpoint{-0.014433in}{0.024851in}}{\pgfqpoint{-0.019642in}{0.019642in}}%
\pgfpathcurveto{\pgfqpoint{-0.024851in}{0.014433in}}{\pgfqpoint{-0.027778in}{0.007367in}}{\pgfqpoint{-0.027778in}{0.000000in}}%
\pgfpathcurveto{\pgfqpoint{-0.027778in}{-0.007367in}}{\pgfqpoint{-0.024851in}{-0.014433in}}{\pgfqpoint{-0.019642in}{-0.019642in}}%
\pgfpathcurveto{\pgfqpoint{-0.014433in}{-0.024851in}}{\pgfqpoint{-0.007367in}{-0.027778in}}{\pgfqpoint{0.000000in}{-0.027778in}}%
\pgfpathclose%
\pgfusepath{stroke,fill}%
}%
\begin{pgfscope}%
\pgfsys@transformshift{0.830199in}{0.916810in}%
\pgfsys@useobject{currentmarker}{}%
\end{pgfscope}%
\end{pgfscope}%
\begin{pgfscope}%
\pgfpathrectangle{\pgfqpoint{0.550713in}{0.102108in}}{\pgfqpoint{3.194133in}{1.837939in}}%
\pgfusepath{clip}%
\pgfsetrectcap%
\pgfsetroundjoin%
\pgfsetlinewidth{0.752812pt}%
\definecolor{currentstroke}{rgb}{0.000000,0.000000,0.000000}%
\pgfsetstrokecolor{currentstroke}%
\pgfsetdash{}{0pt}%
\pgfpathmoveto{\pgfqpoint{0.991503in}{0.894525in}}%
\pgfpathlineto{\pgfqpoint{1.148015in}{0.894525in}}%
\pgfusepath{stroke}%
\end{pgfscope}%
\begin{pgfscope}%
\pgfpathrectangle{\pgfqpoint{0.550713in}{0.102108in}}{\pgfqpoint{3.194133in}{1.837939in}}%
\pgfusepath{clip}%
\pgfsetbuttcap%
\pgfsetroundjoin%
\definecolor{currentfill}{rgb}{1.000000,1.000000,1.000000}%
\pgfsetfillcolor{currentfill}%
\pgfsetlinewidth{1.003750pt}%
\definecolor{currentstroke}{rgb}{0.000000,0.000000,0.000000}%
\pgfsetstrokecolor{currentstroke}%
\pgfsetdash{}{0pt}%
\pgfsys@defobject{currentmarker}{\pgfqpoint{-0.027778in}{-0.027778in}}{\pgfqpoint{0.027778in}{0.027778in}}{%
\pgfpathmoveto{\pgfqpoint{0.000000in}{-0.027778in}}%
\pgfpathcurveto{\pgfqpoint{0.007367in}{-0.027778in}}{\pgfqpoint{0.014433in}{-0.024851in}}{\pgfqpoint{0.019642in}{-0.019642in}}%
\pgfpathcurveto{\pgfqpoint{0.024851in}{-0.014433in}}{\pgfqpoint{0.027778in}{-0.007367in}}{\pgfqpoint{0.027778in}{0.000000in}}%
\pgfpathcurveto{\pgfqpoint{0.027778in}{0.007367in}}{\pgfqpoint{0.024851in}{0.014433in}}{\pgfqpoint{0.019642in}{0.019642in}}%
\pgfpathcurveto{\pgfqpoint{0.014433in}{0.024851in}}{\pgfqpoint{0.007367in}{0.027778in}}{\pgfqpoint{0.000000in}{0.027778in}}%
\pgfpathcurveto{\pgfqpoint{-0.007367in}{0.027778in}}{\pgfqpoint{-0.014433in}{0.024851in}}{\pgfqpoint{-0.019642in}{0.019642in}}%
\pgfpathcurveto{\pgfqpoint{-0.024851in}{0.014433in}}{\pgfqpoint{-0.027778in}{0.007367in}}{\pgfqpoint{-0.027778in}{0.000000in}}%
\pgfpathcurveto{\pgfqpoint{-0.027778in}{-0.007367in}}{\pgfqpoint{-0.024851in}{-0.014433in}}{\pgfqpoint{-0.019642in}{-0.019642in}}%
\pgfpathcurveto{\pgfqpoint{-0.014433in}{-0.024851in}}{\pgfqpoint{-0.007367in}{-0.027778in}}{\pgfqpoint{0.000000in}{-0.027778in}}%
\pgfpathclose%
\pgfusepath{stroke,fill}%
}%
\begin{pgfscope}%
\pgfsys@transformshift{1.069759in}{0.877903in}%
\pgfsys@useobject{currentmarker}{}%
\end{pgfscope}%
\end{pgfscope}%
\begin{pgfscope}%
\pgfpathrectangle{\pgfqpoint{0.550713in}{0.102108in}}{\pgfqpoint{3.194133in}{1.837939in}}%
\pgfusepath{clip}%
\pgfsetrectcap%
\pgfsetroundjoin%
\pgfsetlinewidth{0.752812pt}%
\definecolor{currentstroke}{rgb}{0.000000,0.000000,0.000000}%
\pgfsetstrokecolor{currentstroke}%
\pgfsetdash{}{0pt}%
\pgfpathmoveto{\pgfqpoint{1.151210in}{0.898314in}}%
\pgfpathlineto{\pgfqpoint{1.307722in}{0.898314in}}%
\pgfusepath{stroke}%
\end{pgfscope}%
\begin{pgfscope}%
\pgfpathrectangle{\pgfqpoint{0.550713in}{0.102108in}}{\pgfqpoint{3.194133in}{1.837939in}}%
\pgfusepath{clip}%
\pgfsetbuttcap%
\pgfsetroundjoin%
\definecolor{currentfill}{rgb}{1.000000,1.000000,1.000000}%
\pgfsetfillcolor{currentfill}%
\pgfsetlinewidth{1.003750pt}%
\definecolor{currentstroke}{rgb}{0.000000,0.000000,0.000000}%
\pgfsetstrokecolor{currentstroke}%
\pgfsetdash{}{0pt}%
\pgfsys@defobject{currentmarker}{\pgfqpoint{-0.027778in}{-0.027778in}}{\pgfqpoint{0.027778in}{0.027778in}}{%
\pgfpathmoveto{\pgfqpoint{0.000000in}{-0.027778in}}%
\pgfpathcurveto{\pgfqpoint{0.007367in}{-0.027778in}}{\pgfqpoint{0.014433in}{-0.024851in}}{\pgfqpoint{0.019642in}{-0.019642in}}%
\pgfpathcurveto{\pgfqpoint{0.024851in}{-0.014433in}}{\pgfqpoint{0.027778in}{-0.007367in}}{\pgfqpoint{0.027778in}{0.000000in}}%
\pgfpathcurveto{\pgfqpoint{0.027778in}{0.007367in}}{\pgfqpoint{0.024851in}{0.014433in}}{\pgfqpoint{0.019642in}{0.019642in}}%
\pgfpathcurveto{\pgfqpoint{0.014433in}{0.024851in}}{\pgfqpoint{0.007367in}{0.027778in}}{\pgfqpoint{0.000000in}{0.027778in}}%
\pgfpathcurveto{\pgfqpoint{-0.007367in}{0.027778in}}{\pgfqpoint{-0.014433in}{0.024851in}}{\pgfqpoint{-0.019642in}{0.019642in}}%
\pgfpathcurveto{\pgfqpoint{-0.024851in}{0.014433in}}{\pgfqpoint{-0.027778in}{0.007367in}}{\pgfqpoint{-0.027778in}{0.000000in}}%
\pgfpathcurveto{\pgfqpoint{-0.027778in}{-0.007367in}}{\pgfqpoint{-0.024851in}{-0.014433in}}{\pgfqpoint{-0.019642in}{-0.019642in}}%
\pgfpathcurveto{\pgfqpoint{-0.014433in}{-0.024851in}}{\pgfqpoint{-0.007367in}{-0.027778in}}{\pgfqpoint{0.000000in}{-0.027778in}}%
\pgfpathclose%
\pgfusepath{stroke,fill}%
}%
\begin{pgfscope}%
\pgfsys@transformshift{1.229466in}{0.880981in}%
\pgfsys@useobject{currentmarker}{}%
\end{pgfscope}%
\end{pgfscope}%
\begin{pgfscope}%
\pgfpathrectangle{\pgfqpoint{0.550713in}{0.102108in}}{\pgfqpoint{3.194133in}{1.837939in}}%
\pgfusepath{clip}%
\pgfsetrectcap%
\pgfsetroundjoin%
\pgfsetlinewidth{0.752812pt}%
\definecolor{currentstroke}{rgb}{0.000000,0.000000,0.000000}%
\pgfsetstrokecolor{currentstroke}%
\pgfsetdash{}{0pt}%
\pgfpathmoveto{\pgfqpoint{1.390770in}{1.005228in}}%
\pgfpathlineto{\pgfqpoint{1.547282in}{1.005228in}}%
\pgfusepath{stroke}%
\end{pgfscope}%
\begin{pgfscope}%
\pgfpathrectangle{\pgfqpoint{0.550713in}{0.102108in}}{\pgfqpoint{3.194133in}{1.837939in}}%
\pgfusepath{clip}%
\pgfsetbuttcap%
\pgfsetroundjoin%
\definecolor{currentfill}{rgb}{1.000000,1.000000,1.000000}%
\pgfsetfillcolor{currentfill}%
\pgfsetlinewidth{1.003750pt}%
\definecolor{currentstroke}{rgb}{0.000000,0.000000,0.000000}%
\pgfsetstrokecolor{currentstroke}%
\pgfsetdash{}{0pt}%
\pgfsys@defobject{currentmarker}{\pgfqpoint{-0.027778in}{-0.027778in}}{\pgfqpoint{0.027778in}{0.027778in}}{%
\pgfpathmoveto{\pgfqpoint{0.000000in}{-0.027778in}}%
\pgfpathcurveto{\pgfqpoint{0.007367in}{-0.027778in}}{\pgfqpoint{0.014433in}{-0.024851in}}{\pgfqpoint{0.019642in}{-0.019642in}}%
\pgfpathcurveto{\pgfqpoint{0.024851in}{-0.014433in}}{\pgfqpoint{0.027778in}{-0.007367in}}{\pgfqpoint{0.027778in}{0.000000in}}%
\pgfpathcurveto{\pgfqpoint{0.027778in}{0.007367in}}{\pgfqpoint{0.024851in}{0.014433in}}{\pgfqpoint{0.019642in}{0.019642in}}%
\pgfpathcurveto{\pgfqpoint{0.014433in}{0.024851in}}{\pgfqpoint{0.007367in}{0.027778in}}{\pgfqpoint{0.000000in}{0.027778in}}%
\pgfpathcurveto{\pgfqpoint{-0.007367in}{0.027778in}}{\pgfqpoint{-0.014433in}{0.024851in}}{\pgfqpoint{-0.019642in}{0.019642in}}%
\pgfpathcurveto{\pgfqpoint{-0.024851in}{0.014433in}}{\pgfqpoint{-0.027778in}{0.007367in}}{\pgfqpoint{-0.027778in}{0.000000in}}%
\pgfpathcurveto{\pgfqpoint{-0.027778in}{-0.007367in}}{\pgfqpoint{-0.024851in}{-0.014433in}}{\pgfqpoint{-0.019642in}{-0.019642in}}%
\pgfpathcurveto{\pgfqpoint{-0.014433in}{-0.024851in}}{\pgfqpoint{-0.007367in}{-0.027778in}}{\pgfqpoint{0.000000in}{-0.027778in}}%
\pgfpathclose%
\pgfusepath{stroke,fill}%
}%
\begin{pgfscope}%
\pgfsys@transformshift{1.469026in}{1.014224in}%
\pgfsys@useobject{currentmarker}{}%
\end{pgfscope}%
\end{pgfscope}%
\begin{pgfscope}%
\pgfpathrectangle{\pgfqpoint{0.550713in}{0.102108in}}{\pgfqpoint{3.194133in}{1.837939in}}%
\pgfusepath{clip}%
\pgfsetrectcap%
\pgfsetroundjoin%
\pgfsetlinewidth{0.752812pt}%
\definecolor{currentstroke}{rgb}{0.000000,0.000000,0.000000}%
\pgfsetstrokecolor{currentstroke}%
\pgfsetdash{}{0pt}%
\pgfpathmoveto{\pgfqpoint{1.550476in}{1.072687in}}%
\pgfpathlineto{\pgfqpoint{1.706989in}{1.072687in}}%
\pgfusepath{stroke}%
\end{pgfscope}%
\begin{pgfscope}%
\pgfpathrectangle{\pgfqpoint{0.550713in}{0.102108in}}{\pgfqpoint{3.194133in}{1.837939in}}%
\pgfusepath{clip}%
\pgfsetbuttcap%
\pgfsetroundjoin%
\definecolor{currentfill}{rgb}{1.000000,1.000000,1.000000}%
\pgfsetfillcolor{currentfill}%
\pgfsetlinewidth{1.003750pt}%
\definecolor{currentstroke}{rgb}{0.000000,0.000000,0.000000}%
\pgfsetstrokecolor{currentstroke}%
\pgfsetdash{}{0pt}%
\pgfsys@defobject{currentmarker}{\pgfqpoint{-0.027778in}{-0.027778in}}{\pgfqpoint{0.027778in}{0.027778in}}{%
\pgfpathmoveto{\pgfqpoint{0.000000in}{-0.027778in}}%
\pgfpathcurveto{\pgfqpoint{0.007367in}{-0.027778in}}{\pgfqpoint{0.014433in}{-0.024851in}}{\pgfqpoint{0.019642in}{-0.019642in}}%
\pgfpathcurveto{\pgfqpoint{0.024851in}{-0.014433in}}{\pgfqpoint{0.027778in}{-0.007367in}}{\pgfqpoint{0.027778in}{0.000000in}}%
\pgfpathcurveto{\pgfqpoint{0.027778in}{0.007367in}}{\pgfqpoint{0.024851in}{0.014433in}}{\pgfqpoint{0.019642in}{0.019642in}}%
\pgfpathcurveto{\pgfqpoint{0.014433in}{0.024851in}}{\pgfqpoint{0.007367in}{0.027778in}}{\pgfqpoint{0.000000in}{0.027778in}}%
\pgfpathcurveto{\pgfqpoint{-0.007367in}{0.027778in}}{\pgfqpoint{-0.014433in}{0.024851in}}{\pgfqpoint{-0.019642in}{0.019642in}}%
\pgfpathcurveto{\pgfqpoint{-0.024851in}{0.014433in}}{\pgfqpoint{-0.027778in}{0.007367in}}{\pgfqpoint{-0.027778in}{0.000000in}}%
\pgfpathcurveto{\pgfqpoint{-0.027778in}{-0.007367in}}{\pgfqpoint{-0.024851in}{-0.014433in}}{\pgfqpoint{-0.019642in}{-0.019642in}}%
\pgfpathcurveto{\pgfqpoint{-0.014433in}{-0.024851in}}{\pgfqpoint{-0.007367in}{-0.027778in}}{\pgfqpoint{0.000000in}{-0.027778in}}%
\pgfpathclose%
\pgfusepath{stroke,fill}%
}%
\begin{pgfscope}%
\pgfsys@transformshift{1.628733in}{1.062018in}%
\pgfsys@useobject{currentmarker}{}%
\end{pgfscope}%
\end{pgfscope}%
\begin{pgfscope}%
\pgfpathrectangle{\pgfqpoint{0.550713in}{0.102108in}}{\pgfqpoint{3.194133in}{1.837939in}}%
\pgfusepath{clip}%
\pgfsetrectcap%
\pgfsetroundjoin%
\pgfsetlinewidth{0.752812pt}%
\definecolor{currentstroke}{rgb}{0.000000,0.000000,0.000000}%
\pgfsetstrokecolor{currentstroke}%
\pgfsetdash{}{0pt}%
\pgfpathmoveto{\pgfqpoint{1.790036in}{0.984084in}}%
\pgfpathlineto{\pgfqpoint{1.946549in}{0.984084in}}%
\pgfusepath{stroke}%
\end{pgfscope}%
\begin{pgfscope}%
\pgfpathrectangle{\pgfqpoint{0.550713in}{0.102108in}}{\pgfqpoint{3.194133in}{1.837939in}}%
\pgfusepath{clip}%
\pgfsetbuttcap%
\pgfsetroundjoin%
\definecolor{currentfill}{rgb}{1.000000,1.000000,1.000000}%
\pgfsetfillcolor{currentfill}%
\pgfsetlinewidth{1.003750pt}%
\definecolor{currentstroke}{rgb}{0.000000,0.000000,0.000000}%
\pgfsetstrokecolor{currentstroke}%
\pgfsetdash{}{0pt}%
\pgfsys@defobject{currentmarker}{\pgfqpoint{-0.027778in}{-0.027778in}}{\pgfqpoint{0.027778in}{0.027778in}}{%
\pgfpathmoveto{\pgfqpoint{0.000000in}{-0.027778in}}%
\pgfpathcurveto{\pgfqpoint{0.007367in}{-0.027778in}}{\pgfqpoint{0.014433in}{-0.024851in}}{\pgfqpoint{0.019642in}{-0.019642in}}%
\pgfpathcurveto{\pgfqpoint{0.024851in}{-0.014433in}}{\pgfqpoint{0.027778in}{-0.007367in}}{\pgfqpoint{0.027778in}{0.000000in}}%
\pgfpathcurveto{\pgfqpoint{0.027778in}{0.007367in}}{\pgfqpoint{0.024851in}{0.014433in}}{\pgfqpoint{0.019642in}{0.019642in}}%
\pgfpathcurveto{\pgfqpoint{0.014433in}{0.024851in}}{\pgfqpoint{0.007367in}{0.027778in}}{\pgfqpoint{0.000000in}{0.027778in}}%
\pgfpathcurveto{\pgfqpoint{-0.007367in}{0.027778in}}{\pgfqpoint{-0.014433in}{0.024851in}}{\pgfqpoint{-0.019642in}{0.019642in}}%
\pgfpathcurveto{\pgfqpoint{-0.024851in}{0.014433in}}{\pgfqpoint{-0.027778in}{0.007367in}}{\pgfqpoint{-0.027778in}{0.000000in}}%
\pgfpathcurveto{\pgfqpoint{-0.027778in}{-0.007367in}}{\pgfqpoint{-0.024851in}{-0.014433in}}{\pgfqpoint{-0.019642in}{-0.019642in}}%
\pgfpathcurveto{\pgfqpoint{-0.014433in}{-0.024851in}}{\pgfqpoint{-0.007367in}{-0.027778in}}{\pgfqpoint{0.000000in}{-0.027778in}}%
\pgfpathclose%
\pgfusepath{stroke,fill}%
}%
\begin{pgfscope}%
\pgfsys@transformshift{1.868293in}{0.941783in}%
\pgfsys@useobject{currentmarker}{}%
\end{pgfscope}%
\end{pgfscope}%
\begin{pgfscope}%
\pgfpathrectangle{\pgfqpoint{0.550713in}{0.102108in}}{\pgfqpoint{3.194133in}{1.837939in}}%
\pgfusepath{clip}%
\pgfsetrectcap%
\pgfsetroundjoin%
\pgfsetlinewidth{0.752812pt}%
\definecolor{currentstroke}{rgb}{0.000000,0.000000,0.000000}%
\pgfsetstrokecolor{currentstroke}%
\pgfsetdash{}{0pt}%
\pgfpathmoveto{\pgfqpoint{1.949743in}{0.967480in}}%
\pgfpathlineto{\pgfqpoint{2.106255in}{0.967480in}}%
\pgfusepath{stroke}%
\end{pgfscope}%
\begin{pgfscope}%
\pgfpathrectangle{\pgfqpoint{0.550713in}{0.102108in}}{\pgfqpoint{3.194133in}{1.837939in}}%
\pgfusepath{clip}%
\pgfsetbuttcap%
\pgfsetroundjoin%
\definecolor{currentfill}{rgb}{1.000000,1.000000,1.000000}%
\pgfsetfillcolor{currentfill}%
\pgfsetlinewidth{1.003750pt}%
\definecolor{currentstroke}{rgb}{0.000000,0.000000,0.000000}%
\pgfsetstrokecolor{currentstroke}%
\pgfsetdash{}{0pt}%
\pgfsys@defobject{currentmarker}{\pgfqpoint{-0.027778in}{-0.027778in}}{\pgfqpoint{0.027778in}{0.027778in}}{%
\pgfpathmoveto{\pgfqpoint{0.000000in}{-0.027778in}}%
\pgfpathcurveto{\pgfqpoint{0.007367in}{-0.027778in}}{\pgfqpoint{0.014433in}{-0.024851in}}{\pgfqpoint{0.019642in}{-0.019642in}}%
\pgfpathcurveto{\pgfqpoint{0.024851in}{-0.014433in}}{\pgfqpoint{0.027778in}{-0.007367in}}{\pgfqpoint{0.027778in}{0.000000in}}%
\pgfpathcurveto{\pgfqpoint{0.027778in}{0.007367in}}{\pgfqpoint{0.024851in}{0.014433in}}{\pgfqpoint{0.019642in}{0.019642in}}%
\pgfpathcurveto{\pgfqpoint{0.014433in}{0.024851in}}{\pgfqpoint{0.007367in}{0.027778in}}{\pgfqpoint{0.000000in}{0.027778in}}%
\pgfpathcurveto{\pgfqpoint{-0.007367in}{0.027778in}}{\pgfqpoint{-0.014433in}{0.024851in}}{\pgfqpoint{-0.019642in}{0.019642in}}%
\pgfpathcurveto{\pgfqpoint{-0.024851in}{0.014433in}}{\pgfqpoint{-0.027778in}{0.007367in}}{\pgfqpoint{-0.027778in}{0.000000in}}%
\pgfpathcurveto{\pgfqpoint{-0.027778in}{-0.007367in}}{\pgfqpoint{-0.024851in}{-0.014433in}}{\pgfqpoint{-0.019642in}{-0.019642in}}%
\pgfpathcurveto{\pgfqpoint{-0.014433in}{-0.024851in}}{\pgfqpoint{-0.007367in}{-0.027778in}}{\pgfqpoint{0.000000in}{-0.027778in}}%
\pgfpathclose%
\pgfusepath{stroke,fill}%
}%
\begin{pgfscope}%
\pgfsys@transformshift{2.027999in}{0.979186in}%
\pgfsys@useobject{currentmarker}{}%
\end{pgfscope}%
\end{pgfscope}%
\begin{pgfscope}%
\pgfpathrectangle{\pgfqpoint{0.550713in}{0.102108in}}{\pgfqpoint{3.194133in}{1.837939in}}%
\pgfusepath{clip}%
\pgfsetrectcap%
\pgfsetroundjoin%
\pgfsetlinewidth{0.752812pt}%
\definecolor{currentstroke}{rgb}{0.000000,0.000000,0.000000}%
\pgfsetstrokecolor{currentstroke}%
\pgfsetdash{}{0pt}%
\pgfpathmoveto{\pgfqpoint{2.189303in}{1.163995in}}%
\pgfpathlineto{\pgfqpoint{2.345815in}{1.163995in}}%
\pgfusepath{stroke}%
\end{pgfscope}%
\begin{pgfscope}%
\pgfpathrectangle{\pgfqpoint{0.550713in}{0.102108in}}{\pgfqpoint{3.194133in}{1.837939in}}%
\pgfusepath{clip}%
\pgfsetbuttcap%
\pgfsetroundjoin%
\definecolor{currentfill}{rgb}{1.000000,1.000000,1.000000}%
\pgfsetfillcolor{currentfill}%
\pgfsetlinewidth{1.003750pt}%
\definecolor{currentstroke}{rgb}{0.000000,0.000000,0.000000}%
\pgfsetstrokecolor{currentstroke}%
\pgfsetdash{}{0pt}%
\pgfsys@defobject{currentmarker}{\pgfqpoint{-0.027778in}{-0.027778in}}{\pgfqpoint{0.027778in}{0.027778in}}{%
\pgfpathmoveto{\pgfqpoint{0.000000in}{-0.027778in}}%
\pgfpathcurveto{\pgfqpoint{0.007367in}{-0.027778in}}{\pgfqpoint{0.014433in}{-0.024851in}}{\pgfqpoint{0.019642in}{-0.019642in}}%
\pgfpathcurveto{\pgfqpoint{0.024851in}{-0.014433in}}{\pgfqpoint{0.027778in}{-0.007367in}}{\pgfqpoint{0.027778in}{0.000000in}}%
\pgfpathcurveto{\pgfqpoint{0.027778in}{0.007367in}}{\pgfqpoint{0.024851in}{0.014433in}}{\pgfqpoint{0.019642in}{0.019642in}}%
\pgfpathcurveto{\pgfqpoint{0.014433in}{0.024851in}}{\pgfqpoint{0.007367in}{0.027778in}}{\pgfqpoint{0.000000in}{0.027778in}}%
\pgfpathcurveto{\pgfqpoint{-0.007367in}{0.027778in}}{\pgfqpoint{-0.014433in}{0.024851in}}{\pgfqpoint{-0.019642in}{0.019642in}}%
\pgfpathcurveto{\pgfqpoint{-0.024851in}{0.014433in}}{\pgfqpoint{-0.027778in}{0.007367in}}{\pgfqpoint{-0.027778in}{0.000000in}}%
\pgfpathcurveto{\pgfqpoint{-0.027778in}{-0.007367in}}{\pgfqpoint{-0.024851in}{-0.014433in}}{\pgfqpoint{-0.019642in}{-0.019642in}}%
\pgfpathcurveto{\pgfqpoint{-0.014433in}{-0.024851in}}{\pgfqpoint{-0.007367in}{-0.027778in}}{\pgfqpoint{0.000000in}{-0.027778in}}%
\pgfpathclose%
\pgfusepath{stroke,fill}%
}%
\begin{pgfscope}%
\pgfsys@transformshift{2.267559in}{1.207910in}%
\pgfsys@useobject{currentmarker}{}%
\end{pgfscope}%
\end{pgfscope}%
\begin{pgfscope}%
\pgfpathrectangle{\pgfqpoint{0.550713in}{0.102108in}}{\pgfqpoint{3.194133in}{1.837939in}}%
\pgfusepath{clip}%
\pgfsetrectcap%
\pgfsetroundjoin%
\pgfsetlinewidth{0.752812pt}%
\definecolor{currentstroke}{rgb}{0.000000,0.000000,0.000000}%
\pgfsetstrokecolor{currentstroke}%
\pgfsetdash{}{0pt}%
\pgfpathmoveto{\pgfqpoint{2.349010in}{1.421513in}}%
\pgfpathlineto{\pgfqpoint{2.505522in}{1.421513in}}%
\pgfusepath{stroke}%
\end{pgfscope}%
\begin{pgfscope}%
\pgfpathrectangle{\pgfqpoint{0.550713in}{0.102108in}}{\pgfqpoint{3.194133in}{1.837939in}}%
\pgfusepath{clip}%
\pgfsetbuttcap%
\pgfsetroundjoin%
\definecolor{currentfill}{rgb}{1.000000,1.000000,1.000000}%
\pgfsetfillcolor{currentfill}%
\pgfsetlinewidth{1.003750pt}%
\definecolor{currentstroke}{rgb}{0.000000,0.000000,0.000000}%
\pgfsetstrokecolor{currentstroke}%
\pgfsetdash{}{0pt}%
\pgfsys@defobject{currentmarker}{\pgfqpoint{-0.027778in}{-0.027778in}}{\pgfqpoint{0.027778in}{0.027778in}}{%
\pgfpathmoveto{\pgfqpoint{0.000000in}{-0.027778in}}%
\pgfpathcurveto{\pgfqpoint{0.007367in}{-0.027778in}}{\pgfqpoint{0.014433in}{-0.024851in}}{\pgfqpoint{0.019642in}{-0.019642in}}%
\pgfpathcurveto{\pgfqpoint{0.024851in}{-0.014433in}}{\pgfqpoint{0.027778in}{-0.007367in}}{\pgfqpoint{0.027778in}{0.000000in}}%
\pgfpathcurveto{\pgfqpoint{0.027778in}{0.007367in}}{\pgfqpoint{0.024851in}{0.014433in}}{\pgfqpoint{0.019642in}{0.019642in}}%
\pgfpathcurveto{\pgfqpoint{0.014433in}{0.024851in}}{\pgfqpoint{0.007367in}{0.027778in}}{\pgfqpoint{0.000000in}{0.027778in}}%
\pgfpathcurveto{\pgfqpoint{-0.007367in}{0.027778in}}{\pgfqpoint{-0.014433in}{0.024851in}}{\pgfqpoint{-0.019642in}{0.019642in}}%
\pgfpathcurveto{\pgfqpoint{-0.024851in}{0.014433in}}{\pgfqpoint{-0.027778in}{0.007367in}}{\pgfqpoint{-0.027778in}{0.000000in}}%
\pgfpathcurveto{\pgfqpoint{-0.027778in}{-0.007367in}}{\pgfqpoint{-0.024851in}{-0.014433in}}{\pgfqpoint{-0.019642in}{-0.019642in}}%
\pgfpathcurveto{\pgfqpoint{-0.014433in}{-0.024851in}}{\pgfqpoint{-0.007367in}{-0.027778in}}{\pgfqpoint{0.000000in}{-0.027778in}}%
\pgfpathclose%
\pgfusepath{stroke,fill}%
}%
\begin{pgfscope}%
\pgfsys@transformshift{2.427266in}{1.467285in}%
\pgfsys@useobject{currentmarker}{}%
\end{pgfscope}%
\end{pgfscope}%
\begin{pgfscope}%
\pgfpathrectangle{\pgfqpoint{0.550713in}{0.102108in}}{\pgfqpoint{3.194133in}{1.837939in}}%
\pgfusepath{clip}%
\pgfsetrectcap%
\pgfsetroundjoin%
\pgfsetlinewidth{0.752812pt}%
\definecolor{currentstroke}{rgb}{0.000000,0.000000,0.000000}%
\pgfsetstrokecolor{currentstroke}%
\pgfsetdash{}{0pt}%
\pgfpathmoveto{\pgfqpoint{2.588570in}{1.015135in}}%
\pgfpathlineto{\pgfqpoint{2.745082in}{1.015135in}}%
\pgfusepath{stroke}%
\end{pgfscope}%
\begin{pgfscope}%
\pgfpathrectangle{\pgfqpoint{0.550713in}{0.102108in}}{\pgfqpoint{3.194133in}{1.837939in}}%
\pgfusepath{clip}%
\pgfsetbuttcap%
\pgfsetroundjoin%
\definecolor{currentfill}{rgb}{1.000000,1.000000,1.000000}%
\pgfsetfillcolor{currentfill}%
\pgfsetlinewidth{1.003750pt}%
\definecolor{currentstroke}{rgb}{0.000000,0.000000,0.000000}%
\pgfsetstrokecolor{currentstroke}%
\pgfsetdash{}{0pt}%
\pgfsys@defobject{currentmarker}{\pgfqpoint{-0.027778in}{-0.027778in}}{\pgfqpoint{0.027778in}{0.027778in}}{%
\pgfpathmoveto{\pgfqpoint{0.000000in}{-0.027778in}}%
\pgfpathcurveto{\pgfqpoint{0.007367in}{-0.027778in}}{\pgfqpoint{0.014433in}{-0.024851in}}{\pgfqpoint{0.019642in}{-0.019642in}}%
\pgfpathcurveto{\pgfqpoint{0.024851in}{-0.014433in}}{\pgfqpoint{0.027778in}{-0.007367in}}{\pgfqpoint{0.027778in}{0.000000in}}%
\pgfpathcurveto{\pgfqpoint{0.027778in}{0.007367in}}{\pgfqpoint{0.024851in}{0.014433in}}{\pgfqpoint{0.019642in}{0.019642in}}%
\pgfpathcurveto{\pgfqpoint{0.014433in}{0.024851in}}{\pgfqpoint{0.007367in}{0.027778in}}{\pgfqpoint{0.000000in}{0.027778in}}%
\pgfpathcurveto{\pgfqpoint{-0.007367in}{0.027778in}}{\pgfqpoint{-0.014433in}{0.024851in}}{\pgfqpoint{-0.019642in}{0.019642in}}%
\pgfpathcurveto{\pgfqpoint{-0.024851in}{0.014433in}}{\pgfqpoint{-0.027778in}{0.007367in}}{\pgfqpoint{-0.027778in}{0.000000in}}%
\pgfpathcurveto{\pgfqpoint{-0.027778in}{-0.007367in}}{\pgfqpoint{-0.024851in}{-0.014433in}}{\pgfqpoint{-0.019642in}{-0.019642in}}%
\pgfpathcurveto{\pgfqpoint{-0.014433in}{-0.024851in}}{\pgfqpoint{-0.007367in}{-0.027778in}}{\pgfqpoint{0.000000in}{-0.027778in}}%
\pgfpathclose%
\pgfusepath{stroke,fill}%
}%
\begin{pgfscope}%
\pgfsys@transformshift{2.666826in}{1.017774in}%
\pgfsys@useobject{currentmarker}{}%
\end{pgfscope}%
\end{pgfscope}%
\begin{pgfscope}%
\pgfpathrectangle{\pgfqpoint{0.550713in}{0.102108in}}{\pgfqpoint{3.194133in}{1.837939in}}%
\pgfusepath{clip}%
\pgfsetrectcap%
\pgfsetroundjoin%
\pgfsetlinewidth{0.752812pt}%
\definecolor{currentstroke}{rgb}{0.000000,0.000000,0.000000}%
\pgfsetstrokecolor{currentstroke}%
\pgfsetdash{}{0pt}%
\pgfpathmoveto{\pgfqpoint{2.748276in}{1.159124in}}%
\pgfpathlineto{\pgfqpoint{2.904789in}{1.159124in}}%
\pgfusepath{stroke}%
\end{pgfscope}%
\begin{pgfscope}%
\pgfpathrectangle{\pgfqpoint{0.550713in}{0.102108in}}{\pgfqpoint{3.194133in}{1.837939in}}%
\pgfusepath{clip}%
\pgfsetbuttcap%
\pgfsetroundjoin%
\definecolor{currentfill}{rgb}{1.000000,1.000000,1.000000}%
\pgfsetfillcolor{currentfill}%
\pgfsetlinewidth{1.003750pt}%
\definecolor{currentstroke}{rgb}{0.000000,0.000000,0.000000}%
\pgfsetstrokecolor{currentstroke}%
\pgfsetdash{}{0pt}%
\pgfsys@defobject{currentmarker}{\pgfqpoint{-0.027778in}{-0.027778in}}{\pgfqpoint{0.027778in}{0.027778in}}{%
\pgfpathmoveto{\pgfqpoint{0.000000in}{-0.027778in}}%
\pgfpathcurveto{\pgfqpoint{0.007367in}{-0.027778in}}{\pgfqpoint{0.014433in}{-0.024851in}}{\pgfqpoint{0.019642in}{-0.019642in}}%
\pgfpathcurveto{\pgfqpoint{0.024851in}{-0.014433in}}{\pgfqpoint{0.027778in}{-0.007367in}}{\pgfqpoint{0.027778in}{0.000000in}}%
\pgfpathcurveto{\pgfqpoint{0.027778in}{0.007367in}}{\pgfqpoint{0.024851in}{0.014433in}}{\pgfqpoint{0.019642in}{0.019642in}}%
\pgfpathcurveto{\pgfqpoint{0.014433in}{0.024851in}}{\pgfqpoint{0.007367in}{0.027778in}}{\pgfqpoint{0.000000in}{0.027778in}}%
\pgfpathcurveto{\pgfqpoint{-0.007367in}{0.027778in}}{\pgfqpoint{-0.014433in}{0.024851in}}{\pgfqpoint{-0.019642in}{0.019642in}}%
\pgfpathcurveto{\pgfqpoint{-0.024851in}{0.014433in}}{\pgfqpoint{-0.027778in}{0.007367in}}{\pgfqpoint{-0.027778in}{0.000000in}}%
\pgfpathcurveto{\pgfqpoint{-0.027778in}{-0.007367in}}{\pgfqpoint{-0.024851in}{-0.014433in}}{\pgfqpoint{-0.019642in}{-0.019642in}}%
\pgfpathcurveto{\pgfqpoint{-0.014433in}{-0.024851in}}{\pgfqpoint{-0.007367in}{-0.027778in}}{\pgfqpoint{0.000000in}{-0.027778in}}%
\pgfpathclose%
\pgfusepath{stroke,fill}%
}%
\begin{pgfscope}%
\pgfsys@transformshift{2.826532in}{1.222369in}%
\pgfsys@useobject{currentmarker}{}%
\end{pgfscope}%
\end{pgfscope}%
\begin{pgfscope}%
\pgfpathrectangle{\pgfqpoint{0.550713in}{0.102108in}}{\pgfqpoint{3.194133in}{1.837939in}}%
\pgfusepath{clip}%
\pgfsetrectcap%
\pgfsetroundjoin%
\pgfsetlinewidth{0.752812pt}%
\definecolor{currentstroke}{rgb}{0.000000,0.000000,0.000000}%
\pgfsetstrokecolor{currentstroke}%
\pgfsetdash{}{0pt}%
\pgfpathmoveto{\pgfqpoint{2.987836in}{0.876386in}}%
\pgfpathlineto{\pgfqpoint{3.144349in}{0.876386in}}%
\pgfusepath{stroke}%
\end{pgfscope}%
\begin{pgfscope}%
\pgfpathrectangle{\pgfqpoint{0.550713in}{0.102108in}}{\pgfqpoint{3.194133in}{1.837939in}}%
\pgfusepath{clip}%
\pgfsetbuttcap%
\pgfsetroundjoin%
\definecolor{currentfill}{rgb}{1.000000,1.000000,1.000000}%
\pgfsetfillcolor{currentfill}%
\pgfsetlinewidth{1.003750pt}%
\definecolor{currentstroke}{rgb}{0.000000,0.000000,0.000000}%
\pgfsetstrokecolor{currentstroke}%
\pgfsetdash{}{0pt}%
\pgfsys@defobject{currentmarker}{\pgfqpoint{-0.027778in}{-0.027778in}}{\pgfqpoint{0.027778in}{0.027778in}}{%
\pgfpathmoveto{\pgfqpoint{0.000000in}{-0.027778in}}%
\pgfpathcurveto{\pgfqpoint{0.007367in}{-0.027778in}}{\pgfqpoint{0.014433in}{-0.024851in}}{\pgfqpoint{0.019642in}{-0.019642in}}%
\pgfpathcurveto{\pgfqpoint{0.024851in}{-0.014433in}}{\pgfqpoint{0.027778in}{-0.007367in}}{\pgfqpoint{0.027778in}{0.000000in}}%
\pgfpathcurveto{\pgfqpoint{0.027778in}{0.007367in}}{\pgfqpoint{0.024851in}{0.014433in}}{\pgfqpoint{0.019642in}{0.019642in}}%
\pgfpathcurveto{\pgfqpoint{0.014433in}{0.024851in}}{\pgfqpoint{0.007367in}{0.027778in}}{\pgfqpoint{0.000000in}{0.027778in}}%
\pgfpathcurveto{\pgfqpoint{-0.007367in}{0.027778in}}{\pgfqpoint{-0.014433in}{0.024851in}}{\pgfqpoint{-0.019642in}{0.019642in}}%
\pgfpathcurveto{\pgfqpoint{-0.024851in}{0.014433in}}{\pgfqpoint{-0.027778in}{0.007367in}}{\pgfqpoint{-0.027778in}{0.000000in}}%
\pgfpathcurveto{\pgfqpoint{-0.027778in}{-0.007367in}}{\pgfqpoint{-0.024851in}{-0.014433in}}{\pgfqpoint{-0.019642in}{-0.019642in}}%
\pgfpathcurveto{\pgfqpoint{-0.014433in}{-0.024851in}}{\pgfqpoint{-0.007367in}{-0.027778in}}{\pgfqpoint{0.000000in}{-0.027778in}}%
\pgfpathclose%
\pgfusepath{stroke,fill}%
}%
\begin{pgfscope}%
\pgfsys@transformshift{3.066092in}{0.854167in}%
\pgfsys@useobject{currentmarker}{}%
\end{pgfscope}%
\end{pgfscope}%
\begin{pgfscope}%
\pgfpathrectangle{\pgfqpoint{0.550713in}{0.102108in}}{\pgfqpoint{3.194133in}{1.837939in}}%
\pgfusepath{clip}%
\pgfsetrectcap%
\pgfsetroundjoin%
\pgfsetlinewidth{0.752812pt}%
\definecolor{currentstroke}{rgb}{0.000000,0.000000,0.000000}%
\pgfsetstrokecolor{currentstroke}%
\pgfsetdash{}{0pt}%
\pgfpathmoveto{\pgfqpoint{3.147543in}{0.897924in}}%
\pgfpathlineto{\pgfqpoint{3.304055in}{0.897924in}}%
\pgfusepath{stroke}%
\end{pgfscope}%
\begin{pgfscope}%
\pgfpathrectangle{\pgfqpoint{0.550713in}{0.102108in}}{\pgfqpoint{3.194133in}{1.837939in}}%
\pgfusepath{clip}%
\pgfsetbuttcap%
\pgfsetroundjoin%
\definecolor{currentfill}{rgb}{1.000000,1.000000,1.000000}%
\pgfsetfillcolor{currentfill}%
\pgfsetlinewidth{1.003750pt}%
\definecolor{currentstroke}{rgb}{0.000000,0.000000,0.000000}%
\pgfsetstrokecolor{currentstroke}%
\pgfsetdash{}{0pt}%
\pgfsys@defobject{currentmarker}{\pgfqpoint{-0.027778in}{-0.027778in}}{\pgfqpoint{0.027778in}{0.027778in}}{%
\pgfpathmoveto{\pgfqpoint{0.000000in}{-0.027778in}}%
\pgfpathcurveto{\pgfqpoint{0.007367in}{-0.027778in}}{\pgfqpoint{0.014433in}{-0.024851in}}{\pgfqpoint{0.019642in}{-0.019642in}}%
\pgfpathcurveto{\pgfqpoint{0.024851in}{-0.014433in}}{\pgfqpoint{0.027778in}{-0.007367in}}{\pgfqpoint{0.027778in}{0.000000in}}%
\pgfpathcurveto{\pgfqpoint{0.027778in}{0.007367in}}{\pgfqpoint{0.024851in}{0.014433in}}{\pgfqpoint{0.019642in}{0.019642in}}%
\pgfpathcurveto{\pgfqpoint{0.014433in}{0.024851in}}{\pgfqpoint{0.007367in}{0.027778in}}{\pgfqpoint{0.000000in}{0.027778in}}%
\pgfpathcurveto{\pgfqpoint{-0.007367in}{0.027778in}}{\pgfqpoint{-0.014433in}{0.024851in}}{\pgfqpoint{-0.019642in}{0.019642in}}%
\pgfpathcurveto{\pgfqpoint{-0.024851in}{0.014433in}}{\pgfqpoint{-0.027778in}{0.007367in}}{\pgfqpoint{-0.027778in}{0.000000in}}%
\pgfpathcurveto{\pgfqpoint{-0.027778in}{-0.007367in}}{\pgfqpoint{-0.024851in}{-0.014433in}}{\pgfqpoint{-0.019642in}{-0.019642in}}%
\pgfpathcurveto{\pgfqpoint{-0.014433in}{-0.024851in}}{\pgfqpoint{-0.007367in}{-0.027778in}}{\pgfqpoint{0.000000in}{-0.027778in}}%
\pgfpathclose%
\pgfusepath{stroke,fill}%
}%
\begin{pgfscope}%
\pgfsys@transformshift{3.225799in}{0.897590in}%
\pgfsys@useobject{currentmarker}{}%
\end{pgfscope}%
\end{pgfscope}%
\begin{pgfscope}%
\pgfpathrectangle{\pgfqpoint{0.550713in}{0.102108in}}{\pgfqpoint{3.194133in}{1.837939in}}%
\pgfusepath{clip}%
\pgfsetrectcap%
\pgfsetroundjoin%
\pgfsetlinewidth{0.752812pt}%
\definecolor{currentstroke}{rgb}{0.000000,0.000000,0.000000}%
\pgfsetstrokecolor{currentstroke}%
\pgfsetdash{}{0pt}%
\pgfpathmoveto{\pgfqpoint{3.387103in}{0.962400in}}%
\pgfpathlineto{\pgfqpoint{3.543615in}{0.962400in}}%
\pgfusepath{stroke}%
\end{pgfscope}%
\begin{pgfscope}%
\pgfpathrectangle{\pgfqpoint{0.550713in}{0.102108in}}{\pgfqpoint{3.194133in}{1.837939in}}%
\pgfusepath{clip}%
\pgfsetbuttcap%
\pgfsetroundjoin%
\definecolor{currentfill}{rgb}{1.000000,1.000000,1.000000}%
\pgfsetfillcolor{currentfill}%
\pgfsetlinewidth{1.003750pt}%
\definecolor{currentstroke}{rgb}{0.000000,0.000000,0.000000}%
\pgfsetstrokecolor{currentstroke}%
\pgfsetdash{}{0pt}%
\pgfsys@defobject{currentmarker}{\pgfqpoint{-0.027778in}{-0.027778in}}{\pgfqpoint{0.027778in}{0.027778in}}{%
\pgfpathmoveto{\pgfqpoint{0.000000in}{-0.027778in}}%
\pgfpathcurveto{\pgfqpoint{0.007367in}{-0.027778in}}{\pgfqpoint{0.014433in}{-0.024851in}}{\pgfqpoint{0.019642in}{-0.019642in}}%
\pgfpathcurveto{\pgfqpoint{0.024851in}{-0.014433in}}{\pgfqpoint{0.027778in}{-0.007367in}}{\pgfqpoint{0.027778in}{0.000000in}}%
\pgfpathcurveto{\pgfqpoint{0.027778in}{0.007367in}}{\pgfqpoint{0.024851in}{0.014433in}}{\pgfqpoint{0.019642in}{0.019642in}}%
\pgfpathcurveto{\pgfqpoint{0.014433in}{0.024851in}}{\pgfqpoint{0.007367in}{0.027778in}}{\pgfqpoint{0.000000in}{0.027778in}}%
\pgfpathcurveto{\pgfqpoint{-0.007367in}{0.027778in}}{\pgfqpoint{-0.014433in}{0.024851in}}{\pgfqpoint{-0.019642in}{0.019642in}}%
\pgfpathcurveto{\pgfqpoint{-0.024851in}{0.014433in}}{\pgfqpoint{-0.027778in}{0.007367in}}{\pgfqpoint{-0.027778in}{0.000000in}}%
\pgfpathcurveto{\pgfqpoint{-0.027778in}{-0.007367in}}{\pgfqpoint{-0.024851in}{-0.014433in}}{\pgfqpoint{-0.019642in}{-0.019642in}}%
\pgfpathcurveto{\pgfqpoint{-0.014433in}{-0.024851in}}{\pgfqpoint{-0.007367in}{-0.027778in}}{\pgfqpoint{0.000000in}{-0.027778in}}%
\pgfpathclose%
\pgfusepath{stroke,fill}%
}%
\begin{pgfscope}%
\pgfsys@transformshift{3.465359in}{0.928765in}%
\pgfsys@useobject{currentmarker}{}%
\end{pgfscope}%
\end{pgfscope}%
\begin{pgfscope}%
\pgfpathrectangle{\pgfqpoint{0.550713in}{0.102108in}}{\pgfqpoint{3.194133in}{1.837939in}}%
\pgfusepath{clip}%
\pgfsetrectcap%
\pgfsetroundjoin%
\pgfsetlinewidth{0.752812pt}%
\definecolor{currentstroke}{rgb}{0.000000,0.000000,0.000000}%
\pgfsetstrokecolor{currentstroke}%
\pgfsetdash{}{0pt}%
\pgfpathmoveto{\pgfqpoint{3.546809in}{0.948236in}}%
\pgfpathlineto{\pgfqpoint{3.703322in}{0.948236in}}%
\pgfusepath{stroke}%
\end{pgfscope}%
\begin{pgfscope}%
\pgfpathrectangle{\pgfqpoint{0.550713in}{0.102108in}}{\pgfqpoint{3.194133in}{1.837939in}}%
\pgfusepath{clip}%
\pgfsetbuttcap%
\pgfsetroundjoin%
\definecolor{currentfill}{rgb}{1.000000,1.000000,1.000000}%
\pgfsetfillcolor{currentfill}%
\pgfsetlinewidth{1.003750pt}%
\definecolor{currentstroke}{rgb}{0.000000,0.000000,0.000000}%
\pgfsetstrokecolor{currentstroke}%
\pgfsetdash{}{0pt}%
\pgfsys@defobject{currentmarker}{\pgfqpoint{-0.027778in}{-0.027778in}}{\pgfqpoint{0.027778in}{0.027778in}}{%
\pgfpathmoveto{\pgfqpoint{0.000000in}{-0.027778in}}%
\pgfpathcurveto{\pgfqpoint{0.007367in}{-0.027778in}}{\pgfqpoint{0.014433in}{-0.024851in}}{\pgfqpoint{0.019642in}{-0.019642in}}%
\pgfpathcurveto{\pgfqpoint{0.024851in}{-0.014433in}}{\pgfqpoint{0.027778in}{-0.007367in}}{\pgfqpoint{0.027778in}{0.000000in}}%
\pgfpathcurveto{\pgfqpoint{0.027778in}{0.007367in}}{\pgfqpoint{0.024851in}{0.014433in}}{\pgfqpoint{0.019642in}{0.019642in}}%
\pgfpathcurveto{\pgfqpoint{0.014433in}{0.024851in}}{\pgfqpoint{0.007367in}{0.027778in}}{\pgfqpoint{0.000000in}{0.027778in}}%
\pgfpathcurveto{\pgfqpoint{-0.007367in}{0.027778in}}{\pgfqpoint{-0.014433in}{0.024851in}}{\pgfqpoint{-0.019642in}{0.019642in}}%
\pgfpathcurveto{\pgfqpoint{-0.024851in}{0.014433in}}{\pgfqpoint{-0.027778in}{0.007367in}}{\pgfqpoint{-0.027778in}{0.000000in}}%
\pgfpathcurveto{\pgfqpoint{-0.027778in}{-0.007367in}}{\pgfqpoint{-0.024851in}{-0.014433in}}{\pgfqpoint{-0.019642in}{-0.019642in}}%
\pgfpathcurveto{\pgfqpoint{-0.014433in}{-0.024851in}}{\pgfqpoint{-0.007367in}{-0.027778in}}{\pgfqpoint{0.000000in}{-0.027778in}}%
\pgfpathclose%
\pgfusepath{stroke,fill}%
}%
\begin{pgfscope}%
\pgfsys@transformshift{3.625066in}{0.920720in}%
\pgfsys@useobject{currentmarker}{}%
\end{pgfscope}%
\end{pgfscope}%
\begin{pgfscope}%
\pgfsetrectcap%
\pgfsetmiterjoin%
\pgfsetlinewidth{0.752812pt}%
\definecolor{currentstroke}{rgb}{0.000000,0.000000,0.000000}%
\pgfsetstrokecolor{currentstroke}%
\pgfsetdash{}{0pt}%
\pgfpathmoveto{\pgfqpoint{0.550713in}{0.102108in}}%
\pgfpathlineto{\pgfqpoint{0.550713in}{1.940047in}}%
\pgfusepath{stroke}%
\end{pgfscope}%
\begin{pgfscope}%
\pgfsetrectcap%
\pgfsetmiterjoin%
\pgfsetlinewidth{0.752812pt}%
\definecolor{currentstroke}{rgb}{0.000000,0.000000,0.000000}%
\pgfsetstrokecolor{currentstroke}%
\pgfsetdash{}{0pt}%
\pgfpathmoveto{\pgfqpoint{3.744846in}{0.102108in}}%
\pgfpathlineto{\pgfqpoint{3.744846in}{1.940047in}}%
\pgfusepath{stroke}%
\end{pgfscope}%
\begin{pgfscope}%
\pgfsetrectcap%
\pgfsetmiterjoin%
\pgfsetlinewidth{0.752812pt}%
\definecolor{currentstroke}{rgb}{0.000000,0.000000,0.000000}%
\pgfsetstrokecolor{currentstroke}%
\pgfsetdash{}{0pt}%
\pgfpathmoveto{\pgfqpoint{0.550713in}{0.102108in}}%
\pgfpathlineto{\pgfqpoint{3.744846in}{0.102108in}}%
\pgfusepath{stroke}%
\end{pgfscope}%
\begin{pgfscope}%
\pgfsetrectcap%
\pgfsetmiterjoin%
\pgfsetlinewidth{0.752812pt}%
\definecolor{currentstroke}{rgb}{0.000000,0.000000,0.000000}%
\pgfsetstrokecolor{currentstroke}%
\pgfsetdash{}{0pt}%
\pgfpathmoveto{\pgfqpoint{0.550713in}{1.940047in}}%
\pgfpathlineto{\pgfqpoint{3.744846in}{1.940047in}}%
\pgfusepath{stroke}%
\end{pgfscope}%
\begin{pgfscope}%
\pgfsetbuttcap%
\pgfsetroundjoin%
\pgfsetlinewidth{1.003750pt}%
\definecolor{currentstroke}{rgb}{0.392157,0.396078,0.403922}%
\pgfsetstrokecolor{currentstroke}%
\pgfsetdash{{3.700000pt}{1.600000pt}}{0.000000pt}%
\pgfpathmoveto{\pgfqpoint{3.869846in}{1.795938in}}%
\pgfpathlineto{\pgfqpoint{4.147623in}{1.795938in}}%
\pgfusepath{stroke}%
\end{pgfscope}%
\begin{pgfscope}%
\definecolor{textcolor}{rgb}{0.000000,0.000000,0.000000}%
\pgfsetstrokecolor{textcolor}%
\pgfsetfillcolor{textcolor}%
\pgftext[x=4.258735in, y=1.832104in, left, base]{\color{textcolor}\rmfamily\fontsize{10.000000}{12.000000}\selectfont Only Exploitation:}%
\end{pgfscope}%
\begin{pgfscope}%
\definecolor{textcolor}{rgb}{0.000000,0.000000,0.000000}%
\pgfsetstrokecolor{textcolor}%
\pgfsetfillcolor{textcolor}%
\pgftext[x=4.258735in, y=1.687966in, left, base]{\color{textcolor}\rmfamily\fontsize{10.000000}{12.000000}\selectfont \(\displaystyle R_T=128.76\)}%
\end{pgfscope}%
\begin{pgfscope}%
\pgfsetbuttcap%
\pgfsetmiterjoin%
\definecolor{currentfill}{rgb}{0.631373,0.062745,0.207843}%
\pgfsetfillcolor{currentfill}%
\pgfsetlinewidth{0.000000pt}%
\definecolor{currentstroke}{rgb}{0.000000,0.000000,0.000000}%
\pgfsetstrokecolor{currentstroke}%
\pgfsetstrokeopacity{0.000000}%
\pgfsetdash{}{0pt}%
\pgfpathmoveto{\pgfqpoint{3.869846in}{1.494355in}}%
\pgfpathlineto{\pgfqpoint{4.147623in}{1.494355in}}%
\pgfpathlineto{\pgfqpoint{4.147623in}{1.591577in}}%
\pgfpathlineto{\pgfqpoint{3.869846in}{1.591577in}}%
\pgfpathclose%
\pgfusepath{fill}%
\end{pgfscope}%
\begin{pgfscope}%
\definecolor{textcolor}{rgb}{0.000000,0.000000,0.000000}%
\pgfsetstrokecolor{textcolor}%
\pgfsetfillcolor{textcolor}%
\pgftext[x=4.258735in,y=1.494355in,left,base]{\color{textcolor}\rmfamily\fontsize{10.000000}{12.000000}\selectfont TV-GP-UCB}%
\end{pgfscope}%
\begin{pgfscope}%
\pgfsetbuttcap%
\pgfsetmiterjoin%
\definecolor{currentfill}{rgb}{0.890196,0.000000,0.400000}%
\pgfsetfillcolor{currentfill}%
\pgfsetlinewidth{0.000000pt}%
\definecolor{currentstroke}{rgb}{0.000000,0.000000,0.000000}%
\pgfsetstrokecolor{currentstroke}%
\pgfsetstrokeopacity{0.000000}%
\pgfsetdash{}{0pt}%
\pgfpathmoveto{\pgfqpoint{3.869846in}{1.300744in}}%
\pgfpathlineto{\pgfqpoint{4.147623in}{1.300744in}}%
\pgfpathlineto{\pgfqpoint{4.147623in}{1.397966in}}%
\pgfpathlineto{\pgfqpoint{3.869846in}{1.397966in}}%
\pgfpathclose%
\pgfusepath{fill}%
\end{pgfscope}%
\begin{pgfscope}%
\definecolor{textcolor}{rgb}{0.000000,0.000000,0.000000}%
\pgfsetstrokecolor{textcolor}%
\pgfsetfillcolor{textcolor}%
\pgftext[x=4.258735in,y=1.300744in,left,base]{\color{textcolor}\rmfamily\fontsize{10.000000}{12.000000}\selectfont SW TV-GP-UCB}%
\end{pgfscope}%
\begin{pgfscope}%
\pgfsetbuttcap%
\pgfsetmiterjoin%
\definecolor{currentfill}{rgb}{0.000000,0.329412,0.623529}%
\pgfsetfillcolor{currentfill}%
\pgfsetlinewidth{0.000000pt}%
\definecolor{currentstroke}{rgb}{0.000000,0.000000,0.000000}%
\pgfsetstrokecolor{currentstroke}%
\pgfsetstrokeopacity{0.000000}%
\pgfsetdash{}{0pt}%
\pgfpathmoveto{\pgfqpoint{3.869846in}{1.107133in}}%
\pgfpathlineto{\pgfqpoint{4.147623in}{1.107133in}}%
\pgfpathlineto{\pgfqpoint{4.147623in}{1.204355in}}%
\pgfpathlineto{\pgfqpoint{3.869846in}{1.204355in}}%
\pgfpathclose%
\pgfusepath{fill}%
\end{pgfscope}%
\begin{pgfscope}%
\definecolor{textcolor}{rgb}{0.000000,0.000000,0.000000}%
\pgfsetstrokecolor{textcolor}%
\pgfsetfillcolor{textcolor}%
\pgftext[x=4.258735in,y=1.107133in,left,base]{\color{textcolor}\rmfamily\fontsize{10.000000}{12.000000}\selectfont UI-TVBO}%
\end{pgfscope}%
\begin{pgfscope}%
\pgfsetbuttcap%
\pgfsetmiterjoin%
\definecolor{currentfill}{rgb}{0.000000,0.380392,0.396078}%
\pgfsetfillcolor{currentfill}%
\pgfsetlinewidth{0.000000pt}%
\definecolor{currentstroke}{rgb}{0.000000,0.000000,0.000000}%
\pgfsetstrokecolor{currentstroke}%
\pgfsetstrokeopacity{0.000000}%
\pgfsetdash{}{0pt}%
\pgfpathmoveto{\pgfqpoint{3.869846in}{0.913522in}}%
\pgfpathlineto{\pgfqpoint{4.147623in}{0.913522in}}%
\pgfpathlineto{\pgfqpoint{4.147623in}{1.010744in}}%
\pgfpathlineto{\pgfqpoint{3.869846in}{1.010744in}}%
\pgfpathclose%
\pgfusepath{fill}%
\end{pgfscope}%
\begin{pgfscope}%
\definecolor{textcolor}{rgb}{0.000000,0.000000,0.000000}%
\pgfsetstrokecolor{textcolor}%
\pgfsetfillcolor{textcolor}%
\pgftext[x=4.258735in,y=0.913522in,left,base]{\color{textcolor}\rmfamily\fontsize{10.000000}{12.000000}\selectfont B UI-TVBO}%
\end{pgfscope}%
\begin{pgfscope}%
\pgfsetbuttcap%
\pgfsetmiterjoin%
\definecolor{currentfill}{rgb}{0.380392,0.129412,0.345098}%
\pgfsetfillcolor{currentfill}%
\pgfsetlinewidth{0.000000pt}%
\definecolor{currentstroke}{rgb}{0.000000,0.000000,0.000000}%
\pgfsetstrokecolor{currentstroke}%
\pgfsetstrokeopacity{0.000000}%
\pgfsetdash{}{0pt}%
\pgfpathmoveto{\pgfqpoint{3.869846in}{0.719911in}}%
\pgfpathlineto{\pgfqpoint{4.147623in}{0.719911in}}%
\pgfpathlineto{\pgfqpoint{4.147623in}{0.817133in}}%
\pgfpathlineto{\pgfqpoint{3.869846in}{0.817133in}}%
\pgfpathclose%
\pgfusepath{fill}%
\end{pgfscope}%
\begin{pgfscope}%
\definecolor{textcolor}{rgb}{0.000000,0.000000,0.000000}%
\pgfsetstrokecolor{textcolor}%
\pgfsetfillcolor{textcolor}%
\pgftext[x=4.258735in,y=0.719911in,left,base]{\color{textcolor}\rmfamily\fontsize{10.000000}{12.000000}\selectfont C-TV-GP-UCB}%
\end{pgfscope}%
\begin{pgfscope}%
\pgfsetbuttcap%
\pgfsetmiterjoin%
\definecolor{currentfill}{rgb}{0.964706,0.658824,0.000000}%
\pgfsetfillcolor{currentfill}%
\pgfsetlinewidth{0.000000pt}%
\definecolor{currentstroke}{rgb}{0.000000,0.000000,0.000000}%
\pgfsetstrokecolor{currentstroke}%
\pgfsetstrokeopacity{0.000000}%
\pgfsetdash{}{0pt}%
\pgfpathmoveto{\pgfqpoint{3.869846in}{0.526300in}}%
\pgfpathlineto{\pgfqpoint{4.147623in}{0.526300in}}%
\pgfpathlineto{\pgfqpoint{4.147623in}{0.623522in}}%
\pgfpathlineto{\pgfqpoint{3.869846in}{0.623522in}}%
\pgfpathclose%
\pgfusepath{fill}%
\end{pgfscope}%
\begin{pgfscope}%
\definecolor{textcolor}{rgb}{0.000000,0.000000,0.000000}%
\pgfsetstrokecolor{textcolor}%
\pgfsetfillcolor{textcolor}%
\pgftext[x=4.258735in,y=0.526300in,left,base]{\color{textcolor}\rmfamily\fontsize{10.000000}{12.000000}\selectfont SW C-TV-GP-UCB}%
\end{pgfscope}%
\begin{pgfscope}%
\pgfsetbuttcap%
\pgfsetmiterjoin%
\definecolor{currentfill}{rgb}{0.341176,0.670588,0.152941}%
\pgfsetfillcolor{currentfill}%
\pgfsetlinewidth{0.000000pt}%
\definecolor{currentstroke}{rgb}{0.000000,0.000000,0.000000}%
\pgfsetstrokecolor{currentstroke}%
\pgfsetstrokeopacity{0.000000}%
\pgfsetdash{}{0pt}%
\pgfpathmoveto{\pgfqpoint{3.869846in}{0.332689in}}%
\pgfpathlineto{\pgfqpoint{4.147623in}{0.332689in}}%
\pgfpathlineto{\pgfqpoint{4.147623in}{0.429911in}}%
\pgfpathlineto{\pgfqpoint{3.869846in}{0.429911in}}%
\pgfpathclose%
\pgfusepath{fill}%
\end{pgfscope}%
\begin{pgfscope}%
\definecolor{textcolor}{rgb}{0.000000,0.000000,0.000000}%
\pgfsetstrokecolor{textcolor}%
\pgfsetfillcolor{textcolor}%
\pgftext[x=4.258735in,y=0.332689in,left,base]{\color{textcolor}\rmfamily\fontsize{10.000000}{12.000000}\selectfont C-UI-TVBO}%
\end{pgfscope}%
\begin{pgfscope}%
\pgfsetbuttcap%
\pgfsetmiterjoin%
\definecolor{currentfill}{rgb}{0.478431,0.435294,0.674510}%
\pgfsetfillcolor{currentfill}%
\pgfsetlinewidth{0.000000pt}%
\definecolor{currentstroke}{rgb}{0.000000,0.000000,0.000000}%
\pgfsetstrokecolor{currentstroke}%
\pgfsetstrokeopacity{0.000000}%
\pgfsetdash{}{0pt}%
\pgfpathmoveto{\pgfqpoint{3.869846in}{0.139078in}}%
\pgfpathlineto{\pgfqpoint{4.147623in}{0.139078in}}%
\pgfpathlineto{\pgfqpoint{4.147623in}{0.236300in}}%
\pgfpathlineto{\pgfqpoint{3.869846in}{0.236300in}}%
\pgfpathclose%
\pgfusepath{fill}%
\end{pgfscope}%
\begin{pgfscope}%
\definecolor{textcolor}{rgb}{0.000000,0.000000,0.000000}%
\pgfsetstrokecolor{textcolor}%
\pgfsetfillcolor{textcolor}%
\pgftext[x=4.258735in,y=0.139078in,left,base]{\color{textcolor}\rmfamily\fontsize{10.000000}{12.000000}\selectfont B C-UI-TVBO}%
\end{pgfscope}%
\begin{pgfscope}%
\pgfsetbuttcap%
\pgfsetmiterjoin%
\definecolor{currentfill}{rgb}{1.000000,1.000000,1.000000}%
\pgfsetfillcolor{currentfill}%
\pgfsetlinewidth{1.003750pt}%
\definecolor{currentstroke}{rgb}{1.000000,1.000000,1.000000}%
\pgfsetstrokecolor{currentstroke}%
\pgfsetdash{}{0pt}%
\pgfpathmoveto{\pgfqpoint{1.671233in}{0.129886in}}%
\pgfpathlineto{\pgfqpoint{2.624325in}{0.129886in}}%
\pgfpathquadraticcurveto{\pgfqpoint{2.652103in}{0.129886in}}{\pgfqpoint{2.652103in}{0.157663in}}%
\pgfpathlineto{\pgfqpoint{2.652103in}{0.531120in}}%
\pgfpathquadraticcurveto{\pgfqpoint{2.652103in}{0.558898in}}{\pgfqpoint{2.624325in}{0.558898in}}%
\pgfpathlineto{\pgfqpoint{1.671233in}{0.558898in}}%
\pgfpathquadraticcurveto{\pgfqpoint{1.643455in}{0.558898in}}{\pgfqpoint{1.643455in}{0.531120in}}%
\pgfpathlineto{\pgfqpoint{1.643455in}{0.157663in}}%
\pgfpathquadraticcurveto{\pgfqpoint{1.643455in}{0.129886in}}{\pgfqpoint{1.671233in}{0.129886in}}%
\pgfpathclose%
\pgfusepath{stroke,fill}%
\end{pgfscope}%
\begin{pgfscope}%
\pgfsetbuttcap%
\pgfsetmiterjoin%
\definecolor{currentfill}{rgb}{0.000000,0.000000,0.000000}%
\pgfsetfillcolor{currentfill}%
\pgfsetlinewidth{0.000000pt}%
\definecolor{currentstroke}{rgb}{0.000000,0.000000,0.000000}%
\pgfsetstrokecolor{currentstroke}%
\pgfsetstrokeopacity{0.000000}%
\pgfsetdash{}{0pt}%
\pgfpathmoveto{\pgfqpoint{1.699011in}{0.406120in}}%
\pgfpathlineto{\pgfqpoint{1.976789in}{0.406120in}}%
\pgfpathlineto{\pgfqpoint{1.976789in}{0.503342in}}%
\pgfpathlineto{\pgfqpoint{1.699011in}{0.503342in}}%
\pgfpathclose%
\pgfusepath{fill}%
\end{pgfscope}%
\begin{pgfscope}%
\definecolor{textcolor}{rgb}{0.000000,0.000000,0.000000}%
\pgfsetstrokecolor{textcolor}%
\pgfsetfillcolor{textcolor}%
\pgftext[x=2.087900in,y=0.406120in,left,base]{\color{textcolor}\rmfamily\fontsize{10.000000}{12.000000}\selectfont \(\displaystyle \mu_0=0\)}%
\end{pgfscope}%
\begin{pgfscope}%
\pgfsetbuttcap%
\pgfsetmiterjoin%
\definecolor{currentfill}{rgb}{0.811765,0.819608,0.823529}%
\pgfsetfillcolor{currentfill}%
\pgfsetlinewidth{0.000000pt}%
\definecolor{currentstroke}{rgb}{0.000000,0.000000,0.000000}%
\pgfsetstrokecolor{currentstroke}%
\pgfsetstrokeopacity{0.000000}%
\pgfsetdash{}{0pt}%
\pgfpathmoveto{\pgfqpoint{1.699011in}{0.212447in}}%
\pgfpathlineto{\pgfqpoint{1.976789in}{0.212447in}}%
\pgfpathlineto{\pgfqpoint{1.976789in}{0.309669in}}%
\pgfpathlineto{\pgfqpoint{1.699011in}{0.309669in}}%
\pgfpathclose%
\pgfusepath{fill}%
\end{pgfscope}%
\begin{pgfscope}%
\definecolor{textcolor}{rgb}{0.000000,0.000000,0.000000}%
\pgfsetstrokecolor{textcolor}%
\pgfsetfillcolor{textcolor}%
\pgftext[x=2.087900in,y=0.212447in,left,base]{\color{textcolor}\rmfamily\fontsize{10.000000}{12.000000}\selectfont \(\displaystyle \mu_0=-2\)}%
\end{pgfscope}%
\end{pgfpicture}%
\makeatother%
\endgroup%

    \caption[Results of the two-dimensional out-of-model comparison.]{Results for the two-dimensional out-of-model comparison. \gls{ctvbo} using \gls{b2p} forgetting can result in higher regret for flat objective functions. The formatting is as in Figure~\ref{fig:WMC_cumulative_regret_1D}.}
    \label{fig:OOMC_cumulative_regret_2D}
\end{figure}

The assumption of an increased sampling radius due to \gls{b2p} forgetting and \gls{ctvbo} is confirmed when considering the distribution of the queries taken in the simulations in Figure~\ref{fig:OOMC_sample_dsitribution}. It shows that \gls{ctvbo} almost always prevents sampling at the bounds, which was one of the main motivations. Especially, it avoids sampling at the corners of the feasible set as observed for the unconstrained variations in the top row. For a convex function with the optimum within the feasible set, sampling at the corners will likely yield in high regret.
\begin{figure}[h]
    \centering
    \import{thesis/figures/pgf\string_figures/}{OOMC\string_sample\string_distribution.pgf}
    \caption[Sample distribution of the two-dimensional out-of-model comparison.]{Sample distribution of the two-dimensional out-of-model comparison for the optimistic mean $\mu_0=-2$. \gls{ctvbo} reduces global exploration.}
    \label{fig:OOMC_sample_dsitribution}
\end{figure}

However, for \gls{b2p} forgetting, constraining the posterior comes at the cost of an increased sampling radius around the predicted optimum. In contrast, unconstrained \gls{b2p} forgetting samples more often at the boundaries of the feasible set. However, since the objective functions are very flat due to the bounding functions, this behavior does not increase the regret significantly. 

In other scenarios, with a larger $b(\mathbf{X}_v)$, this frequent sampling at the bounds can lead to significant increases in the regret. Also, in practical applications, this explorative behavior to the bounds should be limited. Unconstrained \gls{ui} forgetting also chooses queries at the bounds of the feasible set due to the ever-increasing variance. The constraints of \gls{ctvbo} limit the increase in variance and thus this explorative behavior. In contrast to \gls{b2p} forgetting, the sampling radius is not significantly increased resulting in the low regret shown in Figure~\ref{fig:OOMC_cumulative_regret_2D}.

\subsection{1-D Moving Parabola}
\label{sec:1D}

The objective functions for the within-model and out-of-model comparison were generated according to the model assumption of temporal change according to a Wiener process and allowed only for a qualitative comparison. For a quantitative comparison between the variations in Table \ref{tab:models}, benchmarks with one and two dimensions inspired by the test functions in \textcite{Renganathan_2020} where designed. They satisfy Assumption~\ref{ass:prior_knowledge_convex}, and therefore, \gls{ctvbo} with convexity constraints can be applied.

The one dimensional benchmark is a moving parabola with the objective function as
\begin{equation}
    f_t(x) = \begin{cases}
        g_{\text{1D}}(x,t) \, ,& t < 140\\
        g_{\text{1D}}(x,t=50) \, ,&140 \leq t \leq 225 \\
        g_{\text{1D}}(x,t=-50)\, ,&t > 225
        \end{cases}
        \label{eq:1d_parabola}
\end{equation}
with
\begin{align}
    g_{\text{1D}}(x,t) =& a_1 (a_2 \cdot x + a_3 + a_4 \cdot t)^2 \nonumber\\&
    + 2(a_2 \cdot x \sin(a_5\cdot t)) - \cos(a_5\cdot t)^2 + b.
\end{align}
The coefficients $a_1$ to $a_5$ as well as $b$ are displayed in Table \ref{tab:coefficients_1D}.
\bgroup
\def\arraystretch{1.2}
\begin{table}[h]
    \centering
    \begin{tabular}{c||c c c c c c}
        \textbf{Coefficient} & $a_1$ &$a_2$&$a_3$&$a_4$&$a_5$&$b$ \\\hline\hline
        \textbf{Value} & $4$ &$0.25$&$-0.5$&$-0.01$&$0.1$&$5$
    \end{tabular}
    \caption{Coefficients for the 1-D moving parabola.}
    \label{tab:coefficients_1D}
\end{table}
\egroup

The feasible set for optimizing the acquisition function is defined as $\mathcal{X}=[-5, 9]$ and the time horizon is $T=300$. The resulting objective function with the trajectory of the optimizer is displayed in Figure~\ref{fig:Parabola1D}.
It consists of a part with gradual change for $t<140$ and two sudden changes at $t=140$ and $t=225$.

The variations in Table~\ref{tab:models} were evaluated on five different runs with different, but for each variant consistent, initializations of $N=15$ data points. The initial training data was normalized to zero mean and a standard deviation of one. Therefore, the prior mean was set to $\mu_0=0$, and the output variance was fixed to $\sigma_k^2=1$. Each subsequent data point was normalized using the mean and standard deviation of the initial data set. For the length scale a hyper prior of $\boldsymbol\Lambda_{11} \sim \mathcal{G}(15, \nicefrac{10}{3})$ \eqref{eq:gamma} was chosen. Furthermore, an interval for the length scale was set to $\boldsymbol\Lambda_{11} \in [2, 7]$. The bounding functions were defined as $a(\mathbf{X}_v)=0$ and $b(\mathbf{X}_v)=4$.
\begin{figure}[h]
    \centering
    %% Creator: Matplotlib, PGF backend
%%
%% To include the figure in your LaTeX document, write
%%   \input{<filename>.pgf}
%%
%% Make sure the required packages are loaded in your preamble
%%   \usepackage{pgf}
%%
%% Figures using additional raster images can only be included by \input if
%% they are in the same directory as the main LaTeX file. For loading figures
%% from other directories you can use the `import` package
%%   \usepackage{import}
%%
%% and then include the figures with
%%   \import{<path to file>}{<filename>.pgf}
%%
%% Matplotlib used the following preamble
%%   \usepackage{fontspec}
%%
\begingroup%
\makeatletter%
\begin{pgfpicture}%
\pgfpathrectangle{\pgfpointorigin}{\pgfqpoint{5.507126in}{2.552693in}}%
\pgfusepath{use as bounding box, clip}%
\begin{pgfscope}%
\pgfsetbuttcap%
\pgfsetmiterjoin%
\definecolor{currentfill}{rgb}{1.000000,1.000000,1.000000}%
\pgfsetfillcolor{currentfill}%
\pgfsetlinewidth{0.000000pt}%
\definecolor{currentstroke}{rgb}{1.000000,1.000000,1.000000}%
\pgfsetstrokecolor{currentstroke}%
\pgfsetdash{}{0pt}%
\pgfpathmoveto{\pgfqpoint{0.000000in}{0.000000in}}%
\pgfpathlineto{\pgfqpoint{5.507126in}{0.000000in}}%
\pgfpathlineto{\pgfqpoint{5.507126in}{2.552693in}}%
\pgfpathlineto{\pgfqpoint{0.000000in}{2.552693in}}%
\pgfpathclose%
\pgfusepath{fill}%
\end{pgfscope}%
\begin{pgfscope}%
\pgfsetbuttcap%
\pgfsetmiterjoin%
\definecolor{currentfill}{rgb}{1.000000,1.000000,1.000000}%
\pgfsetfillcolor{currentfill}%
\pgfsetlinewidth{0.000000pt}%
\definecolor{currentstroke}{rgb}{0.000000,0.000000,0.000000}%
\pgfsetstrokecolor{currentstroke}%
\pgfsetstrokeopacity{0.000000}%
\pgfsetdash{}{0pt}%
\pgfpathmoveto{\pgfqpoint{0.605784in}{0.382904in}}%
\pgfpathlineto{\pgfqpoint{4.669272in}{0.382904in}}%
\pgfpathlineto{\pgfqpoint{4.669272in}{2.425059in}}%
\pgfpathlineto{\pgfqpoint{0.605784in}{2.425059in}}%
\pgfpathclose%
\pgfusepath{fill}%
\end{pgfscope}%
\begin{pgfscope}%
\pgfpathrectangle{\pgfqpoint{0.605784in}{0.382904in}}{\pgfqpoint{4.063488in}{2.042155in}}%
\pgfusepath{clip}%
\pgfsetbuttcap%
\pgfsetroundjoin%
\definecolor{currentfill}{rgb}{0.267004,0.004874,0.329415}%
\pgfsetfillcolor{currentfill}%
\pgfsetlinewidth{1.003750pt}%
\definecolor{currentstroke}{rgb}{0.267004,0.004874,0.329415}%
\pgfsetstrokecolor{currentstroke}%
\pgfsetdash{}{0pt}%
\pgfsys@defobject{currentmarker}{\pgfqpoint{1.981347in}{1.748620in}}{\pgfqpoint{2.236483in}{2.066156in}}{%
\pgfpathmoveto{\pgfqpoint{1.997736in}{1.764968in}}%
\pgfpathlineto{\pgfqpoint{1.987498in}{1.785596in}}%
\pgfpathlineto{\pgfqpoint{1.982648in}{1.806224in}}%
\pgfpathlineto{\pgfqpoint{1.981347in}{1.826852in}}%
\pgfpathlineto{\pgfqpoint{1.982509in}{1.847480in}}%
\pgfpathlineto{\pgfqpoint{1.985449in}{1.868107in}}%
\pgfpathlineto{\pgfqpoint{1.989715in}{1.888735in}}%
\pgfpathlineto{\pgfqpoint{1.994996in}{1.909363in}}%
\pgfpathlineto{\pgfqpoint{2.001071in}{1.929991in}}%
\pgfpathlineto{\pgfqpoint{2.001325in}{1.930723in}}%
\pgfpathlineto{\pgfqpoint{2.011779in}{1.950619in}}%
\pgfpathlineto{\pgfqpoint{2.023073in}{1.971247in}}%
\pgfpathlineto{\pgfqpoint{2.034694in}{1.991874in}}%
\pgfpathlineto{\pgfqpoint{2.042371in}{2.004864in}}%
\pgfpathlineto{\pgfqpoint{2.049487in}{2.012502in}}%
\pgfpathlineto{\pgfqpoint{2.069643in}{2.033130in}}%
\pgfpathlineto{\pgfqpoint{2.083416in}{2.046973in}}%
\pgfpathlineto{\pgfqpoint{2.097367in}{2.053758in}}%
\pgfpathlineto{\pgfqpoint{2.124461in}{2.066156in}}%
\pgfpathlineto{\pgfqpoint{2.165507in}{2.060743in}}%
\pgfpathlineto{\pgfqpoint{2.173373in}{2.053758in}}%
\pgfpathlineto{\pgfqpoint{2.194269in}{2.033130in}}%
\pgfpathlineto{\pgfqpoint{2.206552in}{2.019521in}}%
\pgfpathlineto{\pgfqpoint{2.209704in}{2.012502in}}%
\pgfpathlineto{\pgfqpoint{2.217669in}{1.991874in}}%
\pgfpathlineto{\pgfqpoint{2.224397in}{1.971247in}}%
\pgfpathlineto{\pgfqpoint{2.229777in}{1.950619in}}%
\pgfpathlineto{\pgfqpoint{2.233683in}{1.929991in}}%
\pgfpathlineto{\pgfqpoint{2.235972in}{1.909363in}}%
\pgfpathlineto{\pgfqpoint{2.236483in}{1.888735in}}%
\pgfpathlineto{\pgfqpoint{2.235033in}{1.868107in}}%
\pgfpathlineto{\pgfqpoint{2.231412in}{1.847480in}}%
\pgfpathlineto{\pgfqpoint{2.225381in}{1.826852in}}%
\pgfpathlineto{\pgfqpoint{2.216661in}{1.806224in}}%
\pgfpathlineto{\pgfqpoint{2.206552in}{1.788350in}}%
\pgfpathlineto{\pgfqpoint{2.203094in}{1.785596in}}%
\pgfpathlineto{\pgfqpoint{2.170667in}{1.764968in}}%
\pgfpathlineto{\pgfqpoint{2.165507in}{1.762261in}}%
\pgfpathlineto{\pgfqpoint{2.124461in}{1.754160in}}%
\pgfpathlineto{\pgfqpoint{2.083416in}{1.750254in}}%
\pgfpathlineto{\pgfqpoint{2.042371in}{1.748620in}}%
\pgfpathlineto{\pgfqpoint{2.001325in}{1.760437in}}%
\pgfpathclose%
\pgfusepath{stroke,fill}%
}%
\begin{pgfscope}%
\pgfsys@transformshift{0.000000in}{0.000000in}%
\pgfsys@useobject{currentmarker}{}%
\end{pgfscope}%
\end{pgfscope}%
\begin{pgfscope}%
\pgfpathrectangle{\pgfqpoint{0.605784in}{0.382904in}}{\pgfqpoint{4.063488in}{2.042155in}}%
\pgfusepath{clip}%
\pgfsetbuttcap%
\pgfsetroundjoin%
\definecolor{currentfill}{rgb}{0.277941,0.056324,0.381191}%
\pgfsetfillcolor{currentfill}%
\pgfsetlinewidth{1.003750pt}%
\definecolor{currentstroke}{rgb}{0.277941,0.056324,0.381191}%
\pgfsetstrokecolor{currentstroke}%
\pgfsetdash{}{0pt}%
\pgfpathmoveto{\pgfqpoint{0.625055in}{1.042994in}}%
\pgfpathlineto{\pgfqpoint{0.605784in}{1.048119in}}%
\pgfpathlineto{\pgfqpoint{0.605784in}{1.063622in}}%
\pgfpathlineto{\pgfqpoint{0.605784in}{1.084250in}}%
\pgfpathlineto{\pgfqpoint{0.605784in}{1.104878in}}%
\pgfpathlineto{\pgfqpoint{0.605784in}{1.125506in}}%
\pgfpathlineto{\pgfqpoint{0.605784in}{1.146134in}}%
\pgfpathlineto{\pgfqpoint{0.605784in}{1.166761in}}%
\pgfpathlineto{\pgfqpoint{0.605784in}{1.187389in}}%
\pgfpathlineto{\pgfqpoint{0.605784in}{1.208017in}}%
\pgfpathlineto{\pgfqpoint{0.605784in}{1.228645in}}%
\pgfpathlineto{\pgfqpoint{0.605784in}{1.249273in}}%
\pgfpathlineto{\pgfqpoint{0.605784in}{1.269900in}}%
\pgfpathlineto{\pgfqpoint{0.605784in}{1.290528in}}%
\pgfpathlineto{\pgfqpoint{0.605784in}{1.311156in}}%
\pgfpathlineto{\pgfqpoint{0.605784in}{1.331784in}}%
\pgfpathlineto{\pgfqpoint{0.605784in}{1.352412in}}%
\pgfpathlineto{\pgfqpoint{0.605784in}{1.373040in}}%
\pgfpathlineto{\pgfqpoint{0.605784in}{1.393667in}}%
\pgfpathlineto{\pgfqpoint{0.605784in}{1.414295in}}%
\pgfpathlineto{\pgfqpoint{0.605784in}{1.434923in}}%
\pgfpathlineto{\pgfqpoint{0.605784in}{1.455551in}}%
\pgfpathlineto{\pgfqpoint{0.605784in}{1.476179in}}%
\pgfpathlineto{\pgfqpoint{0.605784in}{1.496807in}}%
\pgfpathlineto{\pgfqpoint{0.605784in}{1.517434in}}%
\pgfpathlineto{\pgfqpoint{0.605784in}{1.532937in}}%
\pgfpathlineto{\pgfqpoint{0.627507in}{1.517434in}}%
\pgfpathlineto{\pgfqpoint{0.646829in}{1.503614in}}%
\pgfpathlineto{\pgfqpoint{0.653540in}{1.496807in}}%
\pgfpathlineto{\pgfqpoint{0.673037in}{1.476179in}}%
\pgfpathlineto{\pgfqpoint{0.687875in}{1.460008in}}%
\pgfpathlineto{\pgfqpoint{0.691769in}{1.455551in}}%
\pgfpathlineto{\pgfqpoint{0.708519in}{1.434923in}}%
\pgfpathlineto{\pgfqpoint{0.724243in}{1.414295in}}%
\pgfpathlineto{\pgfqpoint{0.728920in}{1.407586in}}%
\pgfpathlineto{\pgfqpoint{0.741211in}{1.393667in}}%
\pgfpathlineto{\pgfqpoint{0.757335in}{1.373040in}}%
\pgfpathlineto{\pgfqpoint{0.769965in}{1.354454in}}%
\pgfpathlineto{\pgfqpoint{0.772898in}{1.352412in}}%
\pgfpathlineto{\pgfqpoint{0.796522in}{1.331784in}}%
\pgfpathlineto{\pgfqpoint{0.811011in}{1.314631in}}%
\pgfpathlineto{\pgfqpoint{0.852056in}{1.328910in}}%
\pgfpathlineto{\pgfqpoint{0.853335in}{1.331784in}}%
\pgfpathlineto{\pgfqpoint{0.864476in}{1.352412in}}%
\pgfpathlineto{\pgfqpoint{0.876391in}{1.373040in}}%
\pgfpathlineto{\pgfqpoint{0.888896in}{1.393667in}}%
\pgfpathlineto{\pgfqpoint{0.893101in}{1.400019in}}%
\pgfpathlineto{\pgfqpoint{0.899399in}{1.414295in}}%
\pgfpathlineto{\pgfqpoint{0.909035in}{1.434923in}}%
\pgfpathlineto{\pgfqpoint{0.918937in}{1.455551in}}%
\pgfpathlineto{\pgfqpoint{0.929058in}{1.476179in}}%
\pgfpathlineto{\pgfqpoint{0.934147in}{1.486076in}}%
\pgfpathlineto{\pgfqpoint{0.939145in}{1.496807in}}%
\pgfpathlineto{\pgfqpoint{0.949060in}{1.517434in}}%
\pgfpathlineto{\pgfqpoint{0.958976in}{1.538062in}}%
\pgfpathlineto{\pgfqpoint{0.968892in}{1.558690in}}%
\pgfpathlineto{\pgfqpoint{0.975192in}{1.571547in}}%
\pgfpathlineto{\pgfqpoint{0.979294in}{1.579318in}}%
\pgfpathlineto{\pgfqpoint{0.990364in}{1.599946in}}%
\pgfpathlineto{\pgfqpoint{1.001210in}{1.620573in}}%
\pgfpathlineto{\pgfqpoint{1.011867in}{1.641201in}}%
\pgfpathlineto{\pgfqpoint{1.016237in}{1.649535in}}%
\pgfpathlineto{\pgfqpoint{1.024069in}{1.661829in}}%
\pgfpathlineto{\pgfqpoint{1.037008in}{1.682457in}}%
\pgfpathlineto{\pgfqpoint{1.049473in}{1.703085in}}%
\pgfpathlineto{\pgfqpoint{1.057283in}{1.716168in}}%
\pgfpathlineto{\pgfqpoint{1.063291in}{1.723713in}}%
\pgfpathlineto{\pgfqpoint{1.079443in}{1.744340in}}%
\pgfpathlineto{\pgfqpoint{1.094702in}{1.764968in}}%
\pgfpathlineto{\pgfqpoint{1.098328in}{1.769879in}}%
\pgfpathlineto{\pgfqpoint{1.114518in}{1.785596in}}%
\pgfpathlineto{\pgfqpoint{1.134520in}{1.806224in}}%
\pgfpathlineto{\pgfqpoint{1.139373in}{1.811309in}}%
\pgfpathlineto{\pgfqpoint{1.159959in}{1.826852in}}%
\pgfpathlineto{\pgfqpoint{1.180419in}{1.843097in}}%
\pgfpathlineto{\pgfqpoint{1.187617in}{1.847480in}}%
\pgfpathlineto{\pgfqpoint{1.221023in}{1.868107in}}%
\pgfpathlineto{\pgfqpoint{1.221464in}{1.868379in}}%
\pgfpathlineto{\pgfqpoint{1.262509in}{1.888279in}}%
\pgfpathlineto{\pgfqpoint{1.263886in}{1.888735in}}%
\pgfpathlineto{\pgfqpoint{1.303555in}{1.901076in}}%
\pgfpathlineto{\pgfqpoint{1.344600in}{1.903054in}}%
\pgfpathlineto{\pgfqpoint{1.385645in}{1.889543in}}%
\pgfpathlineto{\pgfqpoint{1.386679in}{1.888735in}}%
\pgfpathlineto{\pgfqpoint{1.412287in}{1.868107in}}%
\pgfpathlineto{\pgfqpoint{1.426691in}{1.856377in}}%
\pgfpathlineto{\pgfqpoint{1.433583in}{1.847480in}}%
\pgfpathlineto{\pgfqpoint{1.448993in}{1.826852in}}%
\pgfpathlineto{\pgfqpoint{1.464066in}{1.806224in}}%
\pgfpathlineto{\pgfqpoint{1.467736in}{1.800949in}}%
\pgfpathlineto{\pgfqpoint{1.476007in}{1.785596in}}%
\pgfpathlineto{\pgfqpoint{1.486546in}{1.764968in}}%
\pgfpathlineto{\pgfqpoint{1.496539in}{1.744340in}}%
\pgfpathlineto{\pgfqpoint{1.505944in}{1.723713in}}%
\pgfpathlineto{\pgfqpoint{1.508781in}{1.716925in}}%
\pgfpathlineto{\pgfqpoint{1.514073in}{1.703085in}}%
\pgfpathlineto{\pgfqpoint{1.521190in}{1.682457in}}%
\pgfpathlineto{\pgfqpoint{1.527543in}{1.661829in}}%
\pgfpathlineto{\pgfqpoint{1.533088in}{1.641201in}}%
\pgfpathlineto{\pgfqpoint{1.537775in}{1.620573in}}%
\pgfpathlineto{\pgfqpoint{1.541553in}{1.599946in}}%
\pgfpathlineto{\pgfqpoint{1.544363in}{1.579318in}}%
\pgfpathlineto{\pgfqpoint{1.546145in}{1.558690in}}%
\pgfpathlineto{\pgfqpoint{1.546831in}{1.538062in}}%
\pgfpathlineto{\pgfqpoint{1.546348in}{1.517434in}}%
\pgfpathlineto{\pgfqpoint{1.544616in}{1.496807in}}%
\pgfpathlineto{\pgfqpoint{1.541547in}{1.476179in}}%
\pgfpathlineto{\pgfqpoint{1.537047in}{1.455551in}}%
\pgfpathlineto{\pgfqpoint{1.531010in}{1.434923in}}%
\pgfpathlineto{\pgfqpoint{1.523321in}{1.414295in}}%
\pgfpathlineto{\pgfqpoint{1.513852in}{1.393667in}}%
\pgfpathlineto{\pgfqpoint{1.508781in}{1.384351in}}%
\pgfpathlineto{\pgfqpoint{1.500265in}{1.373040in}}%
\pgfpathlineto{\pgfqpoint{1.481616in}{1.352412in}}%
\pgfpathlineto{\pgfqpoint{1.467736in}{1.339492in}}%
\pgfpathlineto{\pgfqpoint{1.447860in}{1.331784in}}%
\pgfpathlineto{\pgfqpoint{1.426691in}{1.324826in}}%
\pgfpathlineto{\pgfqpoint{1.385645in}{1.328018in}}%
\pgfpathlineto{\pgfqpoint{1.372849in}{1.331784in}}%
\pgfpathlineto{\pgfqpoint{1.344600in}{1.340326in}}%
\pgfpathlineto{\pgfqpoint{1.304609in}{1.352412in}}%
\pgfpathlineto{\pgfqpoint{1.303555in}{1.352749in}}%
\pgfpathlineto{\pgfqpoint{1.262509in}{1.357783in}}%
\pgfpathlineto{\pgfqpoint{1.237395in}{1.352412in}}%
\pgfpathlineto{\pgfqpoint{1.221464in}{1.349086in}}%
\pgfpathlineto{\pgfqpoint{1.191419in}{1.331784in}}%
\pgfpathlineto{\pgfqpoint{1.180419in}{1.325357in}}%
\pgfpathlineto{\pgfqpoint{1.163647in}{1.311156in}}%
\pgfpathlineto{\pgfqpoint{1.140417in}{1.290528in}}%
\pgfpathlineto{\pgfqpoint{1.139373in}{1.289604in}}%
\pgfpathlineto{\pgfqpoint{1.118747in}{1.269900in}}%
\pgfpathlineto{\pgfqpoint{1.098754in}{1.249273in}}%
\pgfpathlineto{\pgfqpoint{1.098328in}{1.248833in}}%
\pgfpathlineto{\pgfqpoint{1.075669in}{1.228645in}}%
\pgfpathlineto{\pgfqpoint{1.057283in}{1.210567in}}%
\pgfpathlineto{\pgfqpoint{1.053372in}{1.208017in}}%
\pgfpathlineto{\pgfqpoint{1.024090in}{1.187389in}}%
\pgfpathlineto{\pgfqpoint{1.016237in}{1.181328in}}%
\pgfpathlineto{\pgfqpoint{0.978103in}{1.166761in}}%
\pgfpathlineto{\pgfqpoint{0.975192in}{1.165512in}}%
\pgfpathlineto{\pgfqpoint{0.934147in}{1.165585in}}%
\pgfpathlineto{\pgfqpoint{0.929382in}{1.166761in}}%
\pgfpathlineto{\pgfqpoint{0.893101in}{1.180197in}}%
\pgfpathlineto{\pgfqpoint{0.870551in}{1.187389in}}%
\pgfpathlineto{\pgfqpoint{0.852056in}{1.197592in}}%
\pgfpathlineto{\pgfqpoint{0.831523in}{1.187389in}}%
\pgfpathlineto{\pgfqpoint{0.811011in}{1.177728in}}%
\pgfpathlineto{\pgfqpoint{0.804511in}{1.166761in}}%
\pgfpathlineto{\pgfqpoint{0.789256in}{1.146134in}}%
\pgfpathlineto{\pgfqpoint{0.771380in}{1.125506in}}%
\pgfpathlineto{\pgfqpoint{0.769965in}{1.124103in}}%
\pgfpathlineto{\pgfqpoint{0.755254in}{1.104878in}}%
\pgfpathlineto{\pgfqpoint{0.736539in}{1.084250in}}%
\pgfpathlineto{\pgfqpoint{0.728920in}{1.076977in}}%
\pgfpathlineto{\pgfqpoint{0.712494in}{1.063622in}}%
\pgfpathlineto{\pgfqpoint{0.687875in}{1.046783in}}%
\pgfpathlineto{\pgfqpoint{0.673279in}{1.042994in}}%
\pgfpathlineto{\pgfqpoint{0.646829in}{1.037216in}}%
\pgfpathclose%
\pgfusepath{stroke,fill}%
\end{pgfscope}%
\begin{pgfscope}%
\pgfpathrectangle{\pgfqpoint{0.605784in}{0.382904in}}{\pgfqpoint{4.063488in}{2.042155in}}%
\pgfusepath{clip}%
\pgfsetbuttcap%
\pgfsetroundjoin%
\definecolor{currentfill}{rgb}{0.277941,0.056324,0.381191}%
\pgfsetfillcolor{currentfill}%
\pgfsetlinewidth{1.003750pt}%
\definecolor{currentstroke}{rgb}{0.277941,0.056324,0.381191}%
\pgfsetstrokecolor{currentstroke}%
\pgfsetdash{}{0pt}%
\pgfpathmoveto{\pgfqpoint{1.833377in}{1.517434in}}%
\pgfpathlineto{\pgfqpoint{1.823641in}{1.538062in}}%
\pgfpathlineto{\pgfqpoint{1.816702in}{1.558690in}}%
\pgfpathlineto{\pgfqpoint{1.811994in}{1.579318in}}%
\pgfpathlineto{\pgfqpoint{1.809094in}{1.599946in}}%
\pgfpathlineto{\pgfqpoint{1.807680in}{1.620573in}}%
\pgfpathlineto{\pgfqpoint{1.807502in}{1.641201in}}%
\pgfpathlineto{\pgfqpoint{1.808364in}{1.661829in}}%
\pgfpathlineto{\pgfqpoint{1.810107in}{1.682457in}}%
\pgfpathlineto{\pgfqpoint{1.812606in}{1.703085in}}%
\pgfpathlineto{\pgfqpoint{1.815757in}{1.723713in}}%
\pgfpathlineto{\pgfqpoint{1.819473in}{1.744340in}}%
\pgfpathlineto{\pgfqpoint{1.823683in}{1.764968in}}%
\pgfpathlineto{\pgfqpoint{1.828329in}{1.785596in}}%
\pgfpathlineto{\pgfqpoint{1.833359in}{1.806224in}}%
\pgfpathlineto{\pgfqpoint{1.837144in}{1.820594in}}%
\pgfpathlineto{\pgfqpoint{1.838870in}{1.826852in}}%
\pgfpathlineto{\pgfqpoint{1.844997in}{1.847480in}}%
\pgfpathlineto{\pgfqpoint{1.851349in}{1.868107in}}%
\pgfpathlineto{\pgfqpoint{1.857900in}{1.888735in}}%
\pgfpathlineto{\pgfqpoint{1.864627in}{1.909363in}}%
\pgfpathlineto{\pgfqpoint{1.871511in}{1.929991in}}%
\pgfpathlineto{\pgfqpoint{1.878189in}{1.949562in}}%
\pgfpathlineto{\pgfqpoint{1.878604in}{1.950619in}}%
\pgfpathlineto{\pgfqpoint{1.887071in}{1.971247in}}%
\pgfpathlineto{\pgfqpoint{1.895568in}{1.991874in}}%
\pgfpathlineto{\pgfqpoint{1.904092in}{2.012502in}}%
\pgfpathlineto{\pgfqpoint{1.912641in}{2.033130in}}%
\pgfpathlineto{\pgfqpoint{1.919235in}{2.048871in}}%
\pgfpathlineto{\pgfqpoint{1.921761in}{2.053758in}}%
\pgfpathlineto{\pgfqpoint{1.932648in}{2.074386in}}%
\pgfpathlineto{\pgfqpoint{1.943423in}{2.095013in}}%
\pgfpathlineto{\pgfqpoint{1.954097in}{2.115641in}}%
\pgfpathlineto{\pgfqpoint{1.960280in}{2.127509in}}%
\pgfpathlineto{\pgfqpoint{1.966236in}{2.136269in}}%
\pgfpathlineto{\pgfqpoint{1.980322in}{2.156897in}}%
\pgfpathlineto{\pgfqpoint{1.994150in}{2.177525in}}%
\pgfpathlineto{\pgfqpoint{2.001325in}{2.188213in}}%
\pgfpathlineto{\pgfqpoint{2.010526in}{2.198153in}}%
\pgfpathlineto{\pgfqpoint{2.029567in}{2.218780in}}%
\pgfpathlineto{\pgfqpoint{2.042371in}{2.232798in}}%
\pgfpathlineto{\pgfqpoint{2.051494in}{2.239408in}}%
\pgfpathlineto{\pgfqpoint{2.080014in}{2.260036in}}%
\pgfpathlineto{\pgfqpoint{2.083416in}{2.262472in}}%
\pgfpathlineto{\pgfqpoint{2.124461in}{2.277177in}}%
\pgfpathlineto{\pgfqpoint{2.165507in}{2.275716in}}%
\pgfpathlineto{\pgfqpoint{2.197966in}{2.260036in}}%
\pgfpathlineto{\pgfqpoint{2.206552in}{2.255731in}}%
\pgfpathlineto{\pgfqpoint{2.222906in}{2.239408in}}%
\pgfpathlineto{\pgfqpoint{2.243116in}{2.218780in}}%
\pgfpathlineto{\pgfqpoint{2.247597in}{2.214036in}}%
\pgfpathlineto{\pgfqpoint{2.257657in}{2.198153in}}%
\pgfpathlineto{\pgfqpoint{2.270307in}{2.177525in}}%
\pgfpathlineto{\pgfqpoint{2.282579in}{2.156897in}}%
\pgfpathlineto{\pgfqpoint{2.288643in}{2.146257in}}%
\pgfpathlineto{\pgfqpoint{2.293113in}{2.136269in}}%
\pgfpathlineto{\pgfqpoint{2.301876in}{2.115641in}}%
\pgfpathlineto{\pgfqpoint{2.310229in}{2.095013in}}%
\pgfpathlineto{\pgfqpoint{2.318150in}{2.074386in}}%
\pgfpathlineto{\pgfqpoint{2.325614in}{2.053758in}}%
\pgfpathlineto{\pgfqpoint{2.329688in}{2.041637in}}%
\pgfpathlineto{\pgfqpoint{2.332236in}{2.033130in}}%
\pgfpathlineto{\pgfqpoint{2.337875in}{2.012502in}}%
\pgfpathlineto{\pgfqpoint{2.343006in}{1.991874in}}%
\pgfpathlineto{\pgfqpoint{2.347606in}{1.971247in}}%
\pgfpathlineto{\pgfqpoint{2.351649in}{1.950619in}}%
\pgfpathlineto{\pgfqpoint{2.355109in}{1.929991in}}%
\pgfpathlineto{\pgfqpoint{2.357959in}{1.909363in}}%
\pgfpathlineto{\pgfqpoint{2.360168in}{1.888735in}}%
\pgfpathlineto{\pgfqpoint{2.361704in}{1.868107in}}%
\pgfpathlineto{\pgfqpoint{2.362534in}{1.847480in}}%
\pgfpathlineto{\pgfqpoint{2.362620in}{1.826852in}}%
\pgfpathlineto{\pgfqpoint{2.361925in}{1.806224in}}%
\pgfpathlineto{\pgfqpoint{2.360405in}{1.785596in}}%
\pgfpathlineto{\pgfqpoint{2.358017in}{1.764968in}}%
\pgfpathlineto{\pgfqpoint{2.354713in}{1.744340in}}%
\pgfpathlineto{\pgfqpoint{2.350440in}{1.723713in}}%
\pgfpathlineto{\pgfqpoint{2.345143in}{1.703085in}}%
\pgfpathlineto{\pgfqpoint{2.338763in}{1.682457in}}%
\pgfpathlineto{\pgfqpoint{2.331235in}{1.661829in}}%
\pgfpathlineto{\pgfqpoint{2.329688in}{1.658121in}}%
\pgfpathlineto{\pgfqpoint{2.320610in}{1.641201in}}%
\pgfpathlineto{\pgfqpoint{2.307783in}{1.620573in}}%
\pgfpathlineto{\pgfqpoint{2.292980in}{1.599946in}}%
\pgfpathlineto{\pgfqpoint{2.288643in}{1.594552in}}%
\pgfpathlineto{\pgfqpoint{2.268911in}{1.579318in}}%
\pgfpathlineto{\pgfqpoint{2.247597in}{1.564899in}}%
\pgfpathlineto{\pgfqpoint{2.228328in}{1.558690in}}%
\pgfpathlineto{\pgfqpoint{2.206552in}{1.552368in}}%
\pgfpathlineto{\pgfqpoint{2.165507in}{1.547483in}}%
\pgfpathlineto{\pgfqpoint{2.124461in}{1.543152in}}%
\pgfpathlineto{\pgfqpoint{2.099705in}{1.538062in}}%
\pgfpathlineto{\pgfqpoint{2.083416in}{1.534795in}}%
\pgfpathlineto{\pgfqpoint{2.042371in}{1.520751in}}%
\pgfpathlineto{\pgfqpoint{2.034688in}{1.517434in}}%
\pgfpathlineto{\pgfqpoint{2.001325in}{1.502574in}}%
\pgfpathlineto{\pgfqpoint{1.987631in}{1.496807in}}%
\pgfpathlineto{\pgfqpoint{1.960280in}{1.484612in}}%
\pgfpathlineto{\pgfqpoint{1.929168in}{1.476179in}}%
\pgfpathlineto{\pgfqpoint{1.919235in}{1.473264in}}%
\pgfpathlineto{\pgfqpoint{1.886280in}{1.476179in}}%
\pgfpathlineto{\pgfqpoint{1.878189in}{1.477110in}}%
\pgfpathlineto{\pgfqpoint{1.851045in}{1.496807in}}%
\pgfpathlineto{\pgfqpoint{1.837144in}{1.511217in}}%
\pgfpathclose%
\pgfpathmoveto{\pgfqpoint{2.001325in}{1.760437in}}%
\pgfpathlineto{\pgfqpoint{2.042371in}{1.748620in}}%
\pgfpathlineto{\pgfqpoint{2.083416in}{1.750254in}}%
\pgfpathlineto{\pgfqpoint{2.124461in}{1.754160in}}%
\pgfpathlineto{\pgfqpoint{2.165507in}{1.762261in}}%
\pgfpathlineto{\pgfqpoint{2.170667in}{1.764968in}}%
\pgfpathlineto{\pgfqpoint{2.203094in}{1.785596in}}%
\pgfpathlineto{\pgfqpoint{2.206552in}{1.788350in}}%
\pgfpathlineto{\pgfqpoint{2.216661in}{1.806224in}}%
\pgfpathlineto{\pgfqpoint{2.225381in}{1.826852in}}%
\pgfpathlineto{\pgfqpoint{2.231412in}{1.847480in}}%
\pgfpathlineto{\pgfqpoint{2.235033in}{1.868107in}}%
\pgfpathlineto{\pgfqpoint{2.236483in}{1.888735in}}%
\pgfpathlineto{\pgfqpoint{2.235972in}{1.909363in}}%
\pgfpathlineto{\pgfqpoint{2.233683in}{1.929991in}}%
\pgfpathlineto{\pgfqpoint{2.229777in}{1.950619in}}%
\pgfpathlineto{\pgfqpoint{2.224397in}{1.971247in}}%
\pgfpathlineto{\pgfqpoint{2.217669in}{1.991874in}}%
\pgfpathlineto{\pgfqpoint{2.209704in}{2.012502in}}%
\pgfpathlineto{\pgfqpoint{2.206552in}{2.019521in}}%
\pgfpathlineto{\pgfqpoint{2.194269in}{2.033130in}}%
\pgfpathlineto{\pgfqpoint{2.173373in}{2.053758in}}%
\pgfpathlineto{\pgfqpoint{2.165507in}{2.060743in}}%
\pgfpathlineto{\pgfqpoint{2.124461in}{2.066156in}}%
\pgfpathlineto{\pgfqpoint{2.097367in}{2.053758in}}%
\pgfpathlineto{\pgfqpoint{2.083416in}{2.046973in}}%
\pgfpathlineto{\pgfqpoint{2.069643in}{2.033130in}}%
\pgfpathlineto{\pgfqpoint{2.049487in}{2.012502in}}%
\pgfpathlineto{\pgfqpoint{2.042371in}{2.004864in}}%
\pgfpathlineto{\pgfqpoint{2.034694in}{1.991874in}}%
\pgfpathlineto{\pgfqpoint{2.023073in}{1.971247in}}%
\pgfpathlineto{\pgfqpoint{2.011779in}{1.950619in}}%
\pgfpathlineto{\pgfqpoint{2.001325in}{1.930723in}}%
\pgfpathlineto{\pgfqpoint{2.001071in}{1.929991in}}%
\pgfpathlineto{\pgfqpoint{1.994996in}{1.909363in}}%
\pgfpathlineto{\pgfqpoint{1.989715in}{1.888735in}}%
\pgfpathlineto{\pgfqpoint{1.985449in}{1.868107in}}%
\pgfpathlineto{\pgfqpoint{1.982509in}{1.847480in}}%
\pgfpathlineto{\pgfqpoint{1.981347in}{1.826852in}}%
\pgfpathlineto{\pgfqpoint{1.982648in}{1.806224in}}%
\pgfpathlineto{\pgfqpoint{1.987498in}{1.785596in}}%
\pgfpathlineto{\pgfqpoint{1.997736in}{1.764968in}}%
\pgfpathclose%
\pgfusepath{stroke,fill}%
\end{pgfscope}%
\begin{pgfscope}%
\pgfpathrectangle{\pgfqpoint{0.605784in}{0.382904in}}{\pgfqpoint{4.063488in}{2.042155in}}%
\pgfusepath{clip}%
\pgfsetbuttcap%
\pgfsetroundjoin%
\definecolor{currentfill}{rgb}{0.277941,0.056324,0.381191}%
\pgfsetfillcolor{currentfill}%
\pgfsetlinewidth{1.003750pt}%
\definecolor{currentstroke}{rgb}{0.277941,0.056324,0.381191}%
\pgfsetstrokecolor{currentstroke}%
\pgfsetdash{}{0pt}%
\pgfpathmoveto{\pgfqpoint{2.534837in}{1.249273in}}%
\pgfpathlineto{\pgfqpoint{2.534270in}{1.269900in}}%
\pgfpathlineto{\pgfqpoint{2.533742in}{1.290528in}}%
\pgfpathlineto{\pgfqpoint{2.533257in}{1.311156in}}%
\pgfpathlineto{\pgfqpoint{2.532822in}{1.331784in}}%
\pgfpathlineto{\pgfqpoint{2.532445in}{1.352412in}}%
\pgfpathlineto{\pgfqpoint{2.532134in}{1.373040in}}%
\pgfpathlineto{\pgfqpoint{2.531899in}{1.393667in}}%
\pgfpathlineto{\pgfqpoint{2.531755in}{1.414295in}}%
\pgfpathlineto{\pgfqpoint{2.531716in}{1.434923in}}%
\pgfpathlineto{\pgfqpoint{2.531804in}{1.455551in}}%
\pgfpathlineto{\pgfqpoint{2.532044in}{1.476179in}}%
\pgfpathlineto{\pgfqpoint{2.532469in}{1.496807in}}%
\pgfpathlineto{\pgfqpoint{2.533124in}{1.517434in}}%
\pgfpathlineto{\pgfqpoint{2.534068in}{1.538062in}}%
\pgfpathlineto{\pgfqpoint{2.534915in}{1.551765in}}%
\pgfpathlineto{\pgfqpoint{2.575960in}{1.551765in}}%
\pgfpathlineto{\pgfqpoint{2.617005in}{1.551765in}}%
\pgfpathlineto{\pgfqpoint{2.658051in}{1.551765in}}%
\pgfpathlineto{\pgfqpoint{2.699096in}{1.551765in}}%
\pgfpathlineto{\pgfqpoint{2.740141in}{1.551765in}}%
\pgfpathlineto{\pgfqpoint{2.781187in}{1.551765in}}%
\pgfpathlineto{\pgfqpoint{2.822232in}{1.551765in}}%
\pgfpathlineto{\pgfqpoint{2.863277in}{1.551765in}}%
\pgfpathlineto{\pgfqpoint{2.904323in}{1.551765in}}%
\pgfpathlineto{\pgfqpoint{2.945368in}{1.551765in}}%
\pgfpathlineto{\pgfqpoint{2.986413in}{1.551765in}}%
\pgfpathlineto{\pgfqpoint{3.027459in}{1.551765in}}%
\pgfpathlineto{\pgfqpoint{3.068504in}{1.551765in}}%
\pgfpathlineto{\pgfqpoint{3.109549in}{1.551765in}}%
\pgfpathlineto{\pgfqpoint{3.150595in}{1.551765in}}%
\pgfpathlineto{\pgfqpoint{3.191640in}{1.551765in}}%
\pgfpathlineto{\pgfqpoint{3.232685in}{1.551765in}}%
\pgfpathlineto{\pgfqpoint{3.273731in}{1.551765in}}%
\pgfpathlineto{\pgfqpoint{3.314776in}{1.551765in}}%
\pgfpathlineto{\pgfqpoint{3.355821in}{1.551765in}}%
\pgfpathlineto{\pgfqpoint{3.396867in}{1.551765in}}%
\pgfpathlineto{\pgfqpoint{3.437912in}{1.551765in}}%
\pgfpathlineto{\pgfqpoint{3.478957in}{1.551765in}}%
\pgfpathlineto{\pgfqpoint{3.520003in}{1.551765in}}%
\pgfpathlineto{\pgfqpoint{3.561048in}{1.551765in}}%
\pgfpathlineto{\pgfqpoint{3.602093in}{1.551765in}}%
\pgfpathlineto{\pgfqpoint{3.643139in}{1.551765in}}%
\pgfpathlineto{\pgfqpoint{3.644695in}{1.538062in}}%
\pgfpathlineto{\pgfqpoint{3.647039in}{1.517434in}}%
\pgfpathlineto{\pgfqpoint{3.649462in}{1.496807in}}%
\pgfpathlineto{\pgfqpoint{3.651992in}{1.476179in}}%
\pgfpathlineto{\pgfqpoint{3.654668in}{1.455551in}}%
\pgfpathlineto{\pgfqpoint{3.657553in}{1.434923in}}%
\pgfpathlineto{\pgfqpoint{3.660750in}{1.414295in}}%
\pgfpathlineto{\pgfqpoint{3.664448in}{1.393667in}}%
\pgfpathlineto{\pgfqpoint{3.669022in}{1.373040in}}%
\pgfpathlineto{\pgfqpoint{3.675349in}{1.352412in}}%
\pgfpathlineto{\pgfqpoint{3.684184in}{1.334339in}}%
\pgfpathlineto{\pgfqpoint{3.725229in}{1.334339in}}%
\pgfpathlineto{\pgfqpoint{3.766275in}{1.334339in}}%
\pgfpathlineto{\pgfqpoint{3.807320in}{1.334339in}}%
\pgfpathlineto{\pgfqpoint{3.848365in}{1.334339in}}%
\pgfpathlineto{\pgfqpoint{3.889411in}{1.334339in}}%
\pgfpathlineto{\pgfqpoint{3.930456in}{1.334339in}}%
\pgfpathlineto{\pgfqpoint{3.971501in}{1.334339in}}%
\pgfpathlineto{\pgfqpoint{4.012547in}{1.334339in}}%
\pgfpathlineto{\pgfqpoint{4.053592in}{1.334339in}}%
\pgfpathlineto{\pgfqpoint{4.094637in}{1.334339in}}%
\pgfpathlineto{\pgfqpoint{4.135683in}{1.334339in}}%
\pgfpathlineto{\pgfqpoint{4.176728in}{1.334339in}}%
\pgfpathlineto{\pgfqpoint{4.217773in}{1.334339in}}%
\pgfpathlineto{\pgfqpoint{4.258819in}{1.334339in}}%
\pgfpathlineto{\pgfqpoint{4.299864in}{1.334339in}}%
\pgfpathlineto{\pgfqpoint{4.340909in}{1.334339in}}%
\pgfpathlineto{\pgfqpoint{4.381955in}{1.334339in}}%
\pgfpathlineto{\pgfqpoint{4.423000in}{1.334339in}}%
\pgfpathlineto{\pgfqpoint{4.464045in}{1.334339in}}%
\pgfpathlineto{\pgfqpoint{4.505091in}{1.334339in}}%
\pgfpathlineto{\pgfqpoint{4.546136in}{1.334339in}}%
\pgfpathlineto{\pgfqpoint{4.587181in}{1.334339in}}%
\pgfpathlineto{\pgfqpoint{4.628227in}{1.334339in}}%
\pgfpathlineto{\pgfqpoint{4.669272in}{1.334339in}}%
\pgfpathlineto{\pgfqpoint{4.669272in}{1.331784in}}%
\pgfpathlineto{\pgfqpoint{4.669272in}{1.311156in}}%
\pgfpathlineto{\pgfqpoint{4.669272in}{1.290528in}}%
\pgfpathlineto{\pgfqpoint{4.669272in}{1.269900in}}%
\pgfpathlineto{\pgfqpoint{4.669272in}{1.249273in}}%
\pgfpathlineto{\pgfqpoint{4.669272in}{1.228645in}}%
\pgfpathlineto{\pgfqpoint{4.669272in}{1.208017in}}%
\pgfpathlineto{\pgfqpoint{4.669272in}{1.187389in}}%
\pgfpathlineto{\pgfqpoint{4.669272in}{1.166761in}}%
\pgfpathlineto{\pgfqpoint{4.669272in}{1.146134in}}%
\pgfpathlineto{\pgfqpoint{4.669272in}{1.125506in}}%
\pgfpathlineto{\pgfqpoint{4.669272in}{1.104878in}}%
\pgfpathlineto{\pgfqpoint{4.669272in}{1.084250in}}%
\pgfpathlineto{\pgfqpoint{4.669272in}{1.063622in}}%
\pgfpathlineto{\pgfqpoint{4.669272in}{1.042994in}}%
\pgfpathlineto{\pgfqpoint{4.669272in}{1.029292in}}%
\pgfpathlineto{\pgfqpoint{4.628227in}{1.029292in}}%
\pgfpathlineto{\pgfqpoint{4.587181in}{1.029292in}}%
\pgfpathlineto{\pgfqpoint{4.546136in}{1.029292in}}%
\pgfpathlineto{\pgfqpoint{4.505091in}{1.029292in}}%
\pgfpathlineto{\pgfqpoint{4.464045in}{1.029292in}}%
\pgfpathlineto{\pgfqpoint{4.423000in}{1.029292in}}%
\pgfpathlineto{\pgfqpoint{4.381955in}{1.029292in}}%
\pgfpathlineto{\pgfqpoint{4.340909in}{1.029292in}}%
\pgfpathlineto{\pgfqpoint{4.299864in}{1.029292in}}%
\pgfpathlineto{\pgfqpoint{4.258819in}{1.029292in}}%
\pgfpathlineto{\pgfqpoint{4.217773in}{1.029292in}}%
\pgfpathlineto{\pgfqpoint{4.176728in}{1.029292in}}%
\pgfpathlineto{\pgfqpoint{4.135683in}{1.029292in}}%
\pgfpathlineto{\pgfqpoint{4.094637in}{1.029292in}}%
\pgfpathlineto{\pgfqpoint{4.053592in}{1.029292in}}%
\pgfpathlineto{\pgfqpoint{4.012547in}{1.029292in}}%
\pgfpathlineto{\pgfqpoint{3.971501in}{1.029292in}}%
\pgfpathlineto{\pgfqpoint{3.930456in}{1.029292in}}%
\pgfpathlineto{\pgfqpoint{3.889411in}{1.029292in}}%
\pgfpathlineto{\pgfqpoint{3.848365in}{1.029292in}}%
\pgfpathlineto{\pgfqpoint{3.807320in}{1.029292in}}%
\pgfpathlineto{\pgfqpoint{3.766275in}{1.029292in}}%
\pgfpathlineto{\pgfqpoint{3.725229in}{1.029292in}}%
\pgfpathlineto{\pgfqpoint{3.684184in}{1.029292in}}%
\pgfpathlineto{\pgfqpoint{3.682627in}{1.042994in}}%
\pgfpathlineto{\pgfqpoint{3.680284in}{1.063622in}}%
\pgfpathlineto{\pgfqpoint{3.677860in}{1.084250in}}%
\pgfpathlineto{\pgfqpoint{3.675330in}{1.104878in}}%
\pgfpathlineto{\pgfqpoint{3.672654in}{1.125506in}}%
\pgfpathlineto{\pgfqpoint{3.669770in}{1.146134in}}%
\pgfpathlineto{\pgfqpoint{3.666573in}{1.166761in}}%
\pgfpathlineto{\pgfqpoint{3.662874in}{1.187389in}}%
\pgfpathlineto{\pgfqpoint{3.658300in}{1.208017in}}%
\pgfpathlineto{\pgfqpoint{3.651973in}{1.228645in}}%
\pgfpathlineto{\pgfqpoint{3.643139in}{1.246718in}}%
\pgfpathlineto{\pgfqpoint{3.602093in}{1.246718in}}%
\pgfpathlineto{\pgfqpoint{3.561048in}{1.246718in}}%
\pgfpathlineto{\pgfqpoint{3.520003in}{1.246718in}}%
\pgfpathlineto{\pgfqpoint{3.478957in}{1.246718in}}%
\pgfpathlineto{\pgfqpoint{3.437912in}{1.246718in}}%
\pgfpathlineto{\pgfqpoint{3.396867in}{1.246718in}}%
\pgfpathlineto{\pgfqpoint{3.355821in}{1.246718in}}%
\pgfpathlineto{\pgfqpoint{3.314776in}{1.246718in}}%
\pgfpathlineto{\pgfqpoint{3.273731in}{1.246718in}}%
\pgfpathlineto{\pgfqpoint{3.232685in}{1.246718in}}%
\pgfpathlineto{\pgfqpoint{3.191640in}{1.246718in}}%
\pgfpathlineto{\pgfqpoint{3.150595in}{1.246718in}}%
\pgfpathlineto{\pgfqpoint{3.109549in}{1.246718in}}%
\pgfpathlineto{\pgfqpoint{3.068504in}{1.246718in}}%
\pgfpathlineto{\pgfqpoint{3.027459in}{1.246718in}}%
\pgfpathlineto{\pgfqpoint{2.986413in}{1.246718in}}%
\pgfpathlineto{\pgfqpoint{2.945368in}{1.246718in}}%
\pgfpathlineto{\pgfqpoint{2.904323in}{1.246718in}}%
\pgfpathlineto{\pgfqpoint{2.863277in}{1.246718in}}%
\pgfpathlineto{\pgfqpoint{2.822232in}{1.246718in}}%
\pgfpathlineto{\pgfqpoint{2.781187in}{1.246718in}}%
\pgfpathlineto{\pgfqpoint{2.740141in}{1.246718in}}%
\pgfpathlineto{\pgfqpoint{2.699096in}{1.246718in}}%
\pgfpathlineto{\pgfqpoint{2.658051in}{1.246718in}}%
\pgfpathlineto{\pgfqpoint{2.617005in}{1.246718in}}%
\pgfpathlineto{\pgfqpoint{2.575960in}{1.246718in}}%
\pgfpathlineto{\pgfqpoint{2.534915in}{1.246718in}}%
\pgfpathclose%
\pgfusepath{stroke,fill}%
\end{pgfscope}%
\begin{pgfscope}%
\pgfpathrectangle{\pgfqpoint{0.605784in}{0.382904in}}{\pgfqpoint{4.063488in}{2.042155in}}%
\pgfusepath{clip}%
\pgfsetbuttcap%
\pgfsetroundjoin%
\definecolor{currentfill}{rgb}{0.281887,0.150881,0.465405}%
\pgfsetfillcolor{currentfill}%
\pgfsetlinewidth{1.003750pt}%
\definecolor{currentstroke}{rgb}{0.281887,0.150881,0.465405}%
\pgfsetstrokecolor{currentstroke}%
\pgfsetdash{}{0pt}%
\pgfpathmoveto{\pgfqpoint{0.609374in}{0.877972in}}%
\pgfpathlineto{\pgfqpoint{0.605784in}{0.879256in}}%
\pgfpathlineto{\pgfqpoint{0.605784in}{0.898600in}}%
\pgfpathlineto{\pgfqpoint{0.605784in}{0.919227in}}%
\pgfpathlineto{\pgfqpoint{0.605784in}{0.939855in}}%
\pgfpathlineto{\pgfqpoint{0.605784in}{0.960483in}}%
\pgfpathlineto{\pgfqpoint{0.605784in}{0.981111in}}%
\pgfpathlineto{\pgfqpoint{0.605784in}{1.001739in}}%
\pgfpathlineto{\pgfqpoint{0.605784in}{1.022367in}}%
\pgfpathlineto{\pgfqpoint{0.605784in}{1.042994in}}%
\pgfpathlineto{\pgfqpoint{0.605784in}{1.048119in}}%
\pgfpathlineto{\pgfqpoint{0.625055in}{1.042994in}}%
\pgfpathlineto{\pgfqpoint{0.646829in}{1.037216in}}%
\pgfpathlineto{\pgfqpoint{0.673279in}{1.042994in}}%
\pgfpathlineto{\pgfqpoint{0.687875in}{1.046783in}}%
\pgfpathlineto{\pgfqpoint{0.712494in}{1.063622in}}%
\pgfpathlineto{\pgfqpoint{0.728920in}{1.076977in}}%
\pgfpathlineto{\pgfqpoint{0.736539in}{1.084250in}}%
\pgfpathlineto{\pgfqpoint{0.755254in}{1.104878in}}%
\pgfpathlineto{\pgfqpoint{0.769965in}{1.124103in}}%
\pgfpathlineto{\pgfqpoint{0.771380in}{1.125506in}}%
\pgfpathlineto{\pgfqpoint{0.789256in}{1.146134in}}%
\pgfpathlineto{\pgfqpoint{0.804511in}{1.166761in}}%
\pgfpathlineto{\pgfqpoint{0.811011in}{1.177728in}}%
\pgfpathlineto{\pgfqpoint{0.831523in}{1.187389in}}%
\pgfpathlineto{\pgfqpoint{0.852056in}{1.197592in}}%
\pgfpathlineto{\pgfqpoint{0.870551in}{1.187389in}}%
\pgfpathlineto{\pgfqpoint{0.893101in}{1.180197in}}%
\pgfpathlineto{\pgfqpoint{0.929382in}{1.166761in}}%
\pgfpathlineto{\pgfqpoint{0.934147in}{1.165585in}}%
\pgfpathlineto{\pgfqpoint{0.975192in}{1.165512in}}%
\pgfpathlineto{\pgfqpoint{0.978103in}{1.166761in}}%
\pgfpathlineto{\pgfqpoint{1.016237in}{1.181328in}}%
\pgfpathlineto{\pgfqpoint{1.024090in}{1.187389in}}%
\pgfpathlineto{\pgfqpoint{1.053372in}{1.208017in}}%
\pgfpathlineto{\pgfqpoint{1.057283in}{1.210567in}}%
\pgfpathlineto{\pgfqpoint{1.075669in}{1.228645in}}%
\pgfpathlineto{\pgfqpoint{1.098328in}{1.248833in}}%
\pgfpathlineto{\pgfqpoint{1.098754in}{1.249273in}}%
\pgfpathlineto{\pgfqpoint{1.118747in}{1.269900in}}%
\pgfpathlineto{\pgfqpoint{1.139373in}{1.289604in}}%
\pgfpathlineto{\pgfqpoint{1.140417in}{1.290528in}}%
\pgfpathlineto{\pgfqpoint{1.163647in}{1.311156in}}%
\pgfpathlineto{\pgfqpoint{1.180419in}{1.325357in}}%
\pgfpathlineto{\pgfqpoint{1.191419in}{1.331784in}}%
\pgfpathlineto{\pgfqpoint{1.221464in}{1.349086in}}%
\pgfpathlineto{\pgfqpoint{1.237395in}{1.352412in}}%
\pgfpathlineto{\pgfqpoint{1.262509in}{1.357783in}}%
\pgfpathlineto{\pgfqpoint{1.303555in}{1.352749in}}%
\pgfpathlineto{\pgfqpoint{1.304609in}{1.352412in}}%
\pgfpathlineto{\pgfqpoint{1.344600in}{1.340326in}}%
\pgfpathlineto{\pgfqpoint{1.372849in}{1.331784in}}%
\pgfpathlineto{\pgfqpoint{1.385645in}{1.328018in}}%
\pgfpathlineto{\pgfqpoint{1.426691in}{1.324826in}}%
\pgfpathlineto{\pgfqpoint{1.447860in}{1.331784in}}%
\pgfpathlineto{\pgfqpoint{1.467736in}{1.339492in}}%
\pgfpathlineto{\pgfqpoint{1.481616in}{1.352412in}}%
\pgfpathlineto{\pgfqpoint{1.500265in}{1.373040in}}%
\pgfpathlineto{\pgfqpoint{1.508781in}{1.384351in}}%
\pgfpathlineto{\pgfqpoint{1.513852in}{1.393667in}}%
\pgfpathlineto{\pgfqpoint{1.523321in}{1.414295in}}%
\pgfpathlineto{\pgfqpoint{1.531010in}{1.434923in}}%
\pgfpathlineto{\pgfqpoint{1.537047in}{1.455551in}}%
\pgfpathlineto{\pgfqpoint{1.541547in}{1.476179in}}%
\pgfpathlineto{\pgfqpoint{1.544616in}{1.496807in}}%
\pgfpathlineto{\pgfqpoint{1.546348in}{1.517434in}}%
\pgfpathlineto{\pgfqpoint{1.546831in}{1.538062in}}%
\pgfpathlineto{\pgfqpoint{1.546145in}{1.558690in}}%
\pgfpathlineto{\pgfqpoint{1.544363in}{1.579318in}}%
\pgfpathlineto{\pgfqpoint{1.541553in}{1.599946in}}%
\pgfpathlineto{\pgfqpoint{1.537775in}{1.620573in}}%
\pgfpathlineto{\pgfqpoint{1.533088in}{1.641201in}}%
\pgfpathlineto{\pgfqpoint{1.527543in}{1.661829in}}%
\pgfpathlineto{\pgfqpoint{1.521190in}{1.682457in}}%
\pgfpathlineto{\pgfqpoint{1.514073in}{1.703085in}}%
\pgfpathlineto{\pgfqpoint{1.508781in}{1.716925in}}%
\pgfpathlineto{\pgfqpoint{1.505944in}{1.723713in}}%
\pgfpathlineto{\pgfqpoint{1.496539in}{1.744340in}}%
\pgfpathlineto{\pgfqpoint{1.486546in}{1.764968in}}%
\pgfpathlineto{\pgfqpoint{1.476007in}{1.785596in}}%
\pgfpathlineto{\pgfqpoint{1.467736in}{1.800949in}}%
\pgfpathlineto{\pgfqpoint{1.464066in}{1.806224in}}%
\pgfpathlineto{\pgfqpoint{1.448993in}{1.826852in}}%
\pgfpathlineto{\pgfqpoint{1.433583in}{1.847480in}}%
\pgfpathlineto{\pgfqpoint{1.426691in}{1.856377in}}%
\pgfpathlineto{\pgfqpoint{1.412287in}{1.868107in}}%
\pgfpathlineto{\pgfqpoint{1.386679in}{1.888735in}}%
\pgfpathlineto{\pgfqpoint{1.385645in}{1.889543in}}%
\pgfpathlineto{\pgfqpoint{1.344600in}{1.903054in}}%
\pgfpathlineto{\pgfqpoint{1.303555in}{1.901076in}}%
\pgfpathlineto{\pgfqpoint{1.263886in}{1.888735in}}%
\pgfpathlineto{\pgfqpoint{1.262509in}{1.888279in}}%
\pgfpathlineto{\pgfqpoint{1.221464in}{1.868379in}}%
\pgfpathlineto{\pgfqpoint{1.221023in}{1.868107in}}%
\pgfpathlineto{\pgfqpoint{1.187617in}{1.847480in}}%
\pgfpathlineto{\pgfqpoint{1.180419in}{1.843097in}}%
\pgfpathlineto{\pgfqpoint{1.159959in}{1.826852in}}%
\pgfpathlineto{\pgfqpoint{1.139373in}{1.811309in}}%
\pgfpathlineto{\pgfqpoint{1.134520in}{1.806224in}}%
\pgfpathlineto{\pgfqpoint{1.114518in}{1.785596in}}%
\pgfpathlineto{\pgfqpoint{1.098328in}{1.769879in}}%
\pgfpathlineto{\pgfqpoint{1.094702in}{1.764968in}}%
\pgfpathlineto{\pgfqpoint{1.079443in}{1.744340in}}%
\pgfpathlineto{\pgfqpoint{1.063291in}{1.723713in}}%
\pgfpathlineto{\pgfqpoint{1.057283in}{1.716168in}}%
\pgfpathlineto{\pgfqpoint{1.049473in}{1.703085in}}%
\pgfpathlineto{\pgfqpoint{1.037008in}{1.682457in}}%
\pgfpathlineto{\pgfqpoint{1.024069in}{1.661829in}}%
\pgfpathlineto{\pgfqpoint{1.016237in}{1.649535in}}%
\pgfpathlineto{\pgfqpoint{1.011867in}{1.641201in}}%
\pgfpathlineto{\pgfqpoint{1.001210in}{1.620573in}}%
\pgfpathlineto{\pgfqpoint{0.990364in}{1.599946in}}%
\pgfpathlineto{\pgfqpoint{0.979294in}{1.579318in}}%
\pgfpathlineto{\pgfqpoint{0.975192in}{1.571547in}}%
\pgfpathlineto{\pgfqpoint{0.968892in}{1.558690in}}%
\pgfpathlineto{\pgfqpoint{0.958976in}{1.538062in}}%
\pgfpathlineto{\pgfqpoint{0.949060in}{1.517434in}}%
\pgfpathlineto{\pgfqpoint{0.939145in}{1.496807in}}%
\pgfpathlineto{\pgfqpoint{0.934147in}{1.486076in}}%
\pgfpathlineto{\pgfqpoint{0.929058in}{1.476179in}}%
\pgfpathlineto{\pgfqpoint{0.918937in}{1.455551in}}%
\pgfpathlineto{\pgfqpoint{0.909035in}{1.434923in}}%
\pgfpathlineto{\pgfqpoint{0.899399in}{1.414295in}}%
\pgfpathlineto{\pgfqpoint{0.893101in}{1.400019in}}%
\pgfpathlineto{\pgfqpoint{0.888896in}{1.393667in}}%
\pgfpathlineto{\pgfqpoint{0.876391in}{1.373040in}}%
\pgfpathlineto{\pgfqpoint{0.864476in}{1.352412in}}%
\pgfpathlineto{\pgfqpoint{0.853335in}{1.331784in}}%
\pgfpathlineto{\pgfqpoint{0.852056in}{1.328910in}}%
\pgfpathlineto{\pgfqpoint{0.811011in}{1.314631in}}%
\pgfpathlineto{\pgfqpoint{0.796522in}{1.331784in}}%
\pgfpathlineto{\pgfqpoint{0.772898in}{1.352412in}}%
\pgfpathlineto{\pgfqpoint{0.769965in}{1.354454in}}%
\pgfpathlineto{\pgfqpoint{0.757335in}{1.373040in}}%
\pgfpathlineto{\pgfqpoint{0.741211in}{1.393667in}}%
\pgfpathlineto{\pgfqpoint{0.728920in}{1.407586in}}%
\pgfpathlineto{\pgfqpoint{0.724243in}{1.414295in}}%
\pgfpathlineto{\pgfqpoint{0.708519in}{1.434923in}}%
\pgfpathlineto{\pgfqpoint{0.691769in}{1.455551in}}%
\pgfpathlineto{\pgfqpoint{0.687875in}{1.460008in}}%
\pgfpathlineto{\pgfqpoint{0.673037in}{1.476179in}}%
\pgfpathlineto{\pgfqpoint{0.653540in}{1.496807in}}%
\pgfpathlineto{\pgfqpoint{0.646829in}{1.503614in}}%
\pgfpathlineto{\pgfqpoint{0.627507in}{1.517434in}}%
\pgfpathlineto{\pgfqpoint{0.605784in}{1.532937in}}%
\pgfpathlineto{\pgfqpoint{0.605784in}{1.538062in}}%
\pgfpathlineto{\pgfqpoint{0.605784in}{1.558690in}}%
\pgfpathlineto{\pgfqpoint{0.605784in}{1.579318in}}%
\pgfpathlineto{\pgfqpoint{0.605784in}{1.599946in}}%
\pgfpathlineto{\pgfqpoint{0.605784in}{1.620573in}}%
\pgfpathlineto{\pgfqpoint{0.605784in}{1.641201in}}%
\pgfpathlineto{\pgfqpoint{0.605784in}{1.661829in}}%
\pgfpathlineto{\pgfqpoint{0.605784in}{1.682457in}}%
\pgfpathlineto{\pgfqpoint{0.605784in}{1.701800in}}%
\pgfpathlineto{\pgfqpoint{0.636891in}{1.682457in}}%
\pgfpathlineto{\pgfqpoint{0.646829in}{1.676269in}}%
\pgfpathlineto{\pgfqpoint{0.665631in}{1.661829in}}%
\pgfpathlineto{\pgfqpoint{0.687875in}{1.644591in}}%
\pgfpathlineto{\pgfqpoint{0.692473in}{1.641201in}}%
\pgfpathlineto{\pgfqpoint{0.719610in}{1.620573in}}%
\pgfpathlineto{\pgfqpoint{0.728920in}{1.613290in}}%
\pgfpathlineto{\pgfqpoint{0.753713in}{1.599946in}}%
\pgfpathlineto{\pgfqpoint{0.769965in}{1.590758in}}%
\pgfpathlineto{\pgfqpoint{0.811011in}{1.585142in}}%
\pgfpathlineto{\pgfqpoint{0.847684in}{1.599946in}}%
\pgfpathlineto{\pgfqpoint{0.852056in}{1.601692in}}%
\pgfpathlineto{\pgfqpoint{0.872444in}{1.620573in}}%
\pgfpathlineto{\pgfqpoint{0.893101in}{1.640066in}}%
\pgfpathlineto{\pgfqpoint{0.893949in}{1.641201in}}%
\pgfpathlineto{\pgfqpoint{0.909630in}{1.661829in}}%
\pgfpathlineto{\pgfqpoint{0.925028in}{1.682457in}}%
\pgfpathlineto{\pgfqpoint{0.934147in}{1.694700in}}%
\pgfpathlineto{\pgfqpoint{0.939648in}{1.703085in}}%
\pgfpathlineto{\pgfqpoint{0.953213in}{1.723713in}}%
\pgfpathlineto{\pgfqpoint{0.966518in}{1.744340in}}%
\pgfpathlineto{\pgfqpoint{0.975192in}{1.757866in}}%
\pgfpathlineto{\pgfqpoint{0.979828in}{1.764968in}}%
\pgfpathlineto{\pgfqpoint{0.993293in}{1.785596in}}%
\pgfpathlineto{\pgfqpoint{1.006431in}{1.806224in}}%
\pgfpathlineto{\pgfqpoint{1.016237in}{1.821833in}}%
\pgfpathlineto{\pgfqpoint{1.019810in}{1.826852in}}%
\pgfpathlineto{\pgfqpoint{1.034470in}{1.847480in}}%
\pgfpathlineto{\pgfqpoint{1.048670in}{1.868107in}}%
\pgfpathlineto{\pgfqpoint{1.057283in}{1.880801in}}%
\pgfpathlineto{\pgfqpoint{1.063883in}{1.888735in}}%
\pgfpathlineto{\pgfqpoint{1.080807in}{1.909363in}}%
\pgfpathlineto{\pgfqpoint{1.097076in}{1.929991in}}%
\pgfpathlineto{\pgfqpoint{1.098328in}{1.931575in}}%
\pgfpathlineto{\pgfqpoint{1.117713in}{1.950619in}}%
\pgfpathlineto{\pgfqpoint{1.137790in}{1.971247in}}%
\pgfpathlineto{\pgfqpoint{1.139373in}{1.972877in}}%
\pgfpathlineto{\pgfqpoint{1.163722in}{1.991874in}}%
\pgfpathlineto{\pgfqpoint{1.180419in}{2.005309in}}%
\pgfpathlineto{\pgfqpoint{1.192331in}{2.012502in}}%
\pgfpathlineto{\pgfqpoint{1.221464in}{2.030257in}}%
\pgfpathlineto{\pgfqpoint{1.228032in}{2.033130in}}%
\pgfpathlineto{\pgfqpoint{1.262509in}{2.047985in}}%
\pgfpathlineto{\pgfqpoint{1.286575in}{2.053758in}}%
\pgfpathlineto{\pgfqpoint{1.303555in}{2.057676in}}%
\pgfpathlineto{\pgfqpoint{1.344600in}{2.057033in}}%
\pgfpathlineto{\pgfqpoint{1.354726in}{2.053758in}}%
\pgfpathlineto{\pgfqpoint{1.385645in}{2.043582in}}%
\pgfpathlineto{\pgfqpoint{1.401286in}{2.033130in}}%
\pgfpathlineto{\pgfqpoint{1.426691in}{2.016055in}}%
\pgfpathlineto{\pgfqpoint{1.430287in}{2.012502in}}%
\pgfpathlineto{\pgfqpoint{1.450751in}{1.991874in}}%
\pgfpathlineto{\pgfqpoint{1.467736in}{1.974669in}}%
\pgfpathlineto{\pgfqpoint{1.470510in}{1.971247in}}%
\pgfpathlineto{\pgfqpoint{1.486760in}{1.950619in}}%
\pgfpathlineto{\pgfqpoint{1.502793in}{1.929991in}}%
\pgfpathlineto{\pgfqpoint{1.508781in}{1.922075in}}%
\pgfpathlineto{\pgfqpoint{1.518077in}{1.909363in}}%
\pgfpathlineto{\pgfqpoint{1.532691in}{1.888735in}}%
\pgfpathlineto{\pgfqpoint{1.546907in}{1.868107in}}%
\pgfpathlineto{\pgfqpoint{1.549827in}{1.863687in}}%
\pgfpathlineto{\pgfqpoint{1.562326in}{1.847480in}}%
\pgfpathlineto{\pgfqpoint{1.577522in}{1.826852in}}%
\pgfpathlineto{\pgfqpoint{1.590872in}{1.807843in}}%
\pgfpathlineto{\pgfqpoint{1.592674in}{1.806224in}}%
\pgfpathlineto{\pgfqpoint{1.614036in}{1.785596in}}%
\pgfpathlineto{\pgfqpoint{1.631917in}{1.766939in}}%
\pgfpathlineto{\pgfqpoint{1.640801in}{1.764968in}}%
\pgfpathlineto{\pgfqpoint{1.672963in}{1.756857in}}%
\pgfpathlineto{\pgfqpoint{1.683402in}{1.764968in}}%
\pgfpathlineto{\pgfqpoint{1.710271in}{1.785596in}}%
\pgfpathlineto{\pgfqpoint{1.714008in}{1.788395in}}%
\pgfpathlineto{\pgfqpoint{1.724834in}{1.806224in}}%
\pgfpathlineto{\pgfqpoint{1.737550in}{1.826852in}}%
\pgfpathlineto{\pgfqpoint{1.750378in}{1.847480in}}%
\pgfpathlineto{\pgfqpoint{1.755053in}{1.854777in}}%
\pgfpathlineto{\pgfqpoint{1.761249in}{1.868107in}}%
\pgfpathlineto{\pgfqpoint{1.771043in}{1.888735in}}%
\pgfpathlineto{\pgfqpoint{1.780920in}{1.909363in}}%
\pgfpathlineto{\pgfqpoint{1.790874in}{1.929991in}}%
\pgfpathlineto{\pgfqpoint{1.796099in}{1.940594in}}%
\pgfpathlineto{\pgfqpoint{1.800490in}{1.950619in}}%
\pgfpathlineto{\pgfqpoint{1.809699in}{1.971247in}}%
\pgfpathlineto{\pgfqpoint{1.818930in}{1.991874in}}%
\pgfpathlineto{\pgfqpoint{1.828183in}{2.012502in}}%
\pgfpathlineto{\pgfqpoint{1.837144in}{2.032419in}}%
\pgfpathlineto{\pgfqpoint{1.837467in}{2.033130in}}%
\pgfpathlineto{\pgfqpoint{1.847093in}{2.053758in}}%
\pgfpathlineto{\pgfqpoint{1.856679in}{2.074386in}}%
\pgfpathlineto{\pgfqpoint{1.866228in}{2.095013in}}%
\pgfpathlineto{\pgfqpoint{1.875744in}{2.115641in}}%
\pgfpathlineto{\pgfqpoint{1.878189in}{2.120855in}}%
\pgfpathlineto{\pgfqpoint{1.886199in}{2.136269in}}%
\pgfpathlineto{\pgfqpoint{1.896889in}{2.156897in}}%
\pgfpathlineto{\pgfqpoint{1.907474in}{2.177525in}}%
\pgfpathlineto{\pgfqpoint{1.917965in}{2.198153in}}%
\pgfpathlineto{\pgfqpoint{1.919235in}{2.200610in}}%
\pgfpathlineto{\pgfqpoint{1.930416in}{2.218780in}}%
\pgfpathlineto{\pgfqpoint{1.942957in}{2.239408in}}%
\pgfpathlineto{\pgfqpoint{1.955315in}{2.260036in}}%
\pgfpathlineto{\pgfqpoint{1.960280in}{2.268293in}}%
\pgfpathlineto{\pgfqpoint{1.969712in}{2.280664in}}%
\pgfpathlineto{\pgfqpoint{1.985325in}{2.301292in}}%
\pgfpathlineto{\pgfqpoint{2.000642in}{2.321920in}}%
\pgfpathlineto{\pgfqpoint{2.001325in}{2.322830in}}%
\pgfpathlineto{\pgfqpoint{2.021309in}{2.342547in}}%
\pgfpathlineto{\pgfqpoint{2.041772in}{2.363175in}}%
\pgfpathlineto{\pgfqpoint{2.042371in}{2.363773in}}%
\pgfpathlineto{\pgfqpoint{2.072334in}{2.383803in}}%
\pgfpathlineto{\pgfqpoint{2.083416in}{2.391229in}}%
\pgfpathlineto{\pgfqpoint{2.122143in}{2.404431in}}%
\pgfpathlineto{\pgfqpoint{2.124461in}{2.405207in}}%
\pgfpathlineto{\pgfqpoint{2.165507in}{2.404529in}}%
\pgfpathlineto{\pgfqpoint{2.165754in}{2.404431in}}%
\pgfpathlineto{\pgfqpoint{2.206552in}{2.387793in}}%
\pgfpathlineto{\pgfqpoint{2.211486in}{2.383803in}}%
\pgfpathlineto{\pgfqpoint{2.236474in}{2.363175in}}%
\pgfpathlineto{\pgfqpoint{2.247597in}{2.353838in}}%
\pgfpathlineto{\pgfqpoint{2.256766in}{2.342547in}}%
\pgfpathlineto{\pgfqpoint{2.273200in}{2.321920in}}%
\pgfpathlineto{\pgfqpoint{2.288643in}{2.302280in}}%
\pgfpathlineto{\pgfqpoint{2.289266in}{2.301292in}}%
\pgfpathlineto{\pgfqpoint{2.301864in}{2.280664in}}%
\pgfpathlineto{\pgfqpoint{2.314217in}{2.260036in}}%
\pgfpathlineto{\pgfqpoint{2.326315in}{2.239408in}}%
\pgfpathlineto{\pgfqpoint{2.329688in}{2.233460in}}%
\pgfpathlineto{\pgfqpoint{2.337299in}{2.218780in}}%
\pgfpathlineto{\pgfqpoint{2.347644in}{2.198153in}}%
\pgfpathlineto{\pgfqpoint{2.357667in}{2.177525in}}%
\pgfpathlineto{\pgfqpoint{2.367354in}{2.156897in}}%
\pgfpathlineto{\pgfqpoint{2.370733in}{2.149368in}}%
\pgfpathlineto{\pgfqpoint{2.376904in}{2.136269in}}%
\pgfpathlineto{\pgfqpoint{2.386150in}{2.115641in}}%
\pgfpathlineto{\pgfqpoint{2.394949in}{2.095013in}}%
\pgfpathlineto{\pgfqpoint{2.403285in}{2.074386in}}%
\pgfpathlineto{\pgfqpoint{2.411143in}{2.053758in}}%
\pgfpathlineto{\pgfqpoint{2.411779in}{2.051960in}}%
\pgfpathlineto{\pgfqpoint{2.420096in}{2.033130in}}%
\pgfpathlineto{\pgfqpoint{2.428493in}{2.012502in}}%
\pgfpathlineto{\pgfqpoint{2.436166in}{1.991874in}}%
\pgfpathlineto{\pgfqpoint{2.443100in}{1.971247in}}%
\pgfpathlineto{\pgfqpoint{2.449280in}{1.950619in}}%
\pgfpathlineto{\pgfqpoint{2.452824in}{1.937070in}}%
\pgfpathlineto{\pgfqpoint{2.455919in}{1.929991in}}%
\pgfpathlineto{\pgfqpoint{2.463476in}{1.909363in}}%
\pgfpathlineto{\pgfqpoint{2.469566in}{1.888735in}}%
\pgfpathlineto{\pgfqpoint{2.474202in}{1.868107in}}%
\pgfpathlineto{\pgfqpoint{2.477400in}{1.847480in}}%
\pgfpathlineto{\pgfqpoint{2.479175in}{1.826852in}}%
\pgfpathlineto{\pgfqpoint{2.479540in}{1.806224in}}%
\pgfpathlineto{\pgfqpoint{2.478510in}{1.785596in}}%
\pgfpathlineto{\pgfqpoint{2.476099in}{1.764968in}}%
\pgfpathlineto{\pgfqpoint{2.472320in}{1.744340in}}%
\pgfpathlineto{\pgfqpoint{2.467187in}{1.723713in}}%
\pgfpathlineto{\pgfqpoint{2.460714in}{1.703085in}}%
\pgfpathlineto{\pgfqpoint{2.452913in}{1.682457in}}%
\pgfpathlineto{\pgfqpoint{2.452824in}{1.682255in}}%
\pgfpathlineto{\pgfqpoint{2.446563in}{1.661829in}}%
\pgfpathlineto{\pgfqpoint{2.439187in}{1.641201in}}%
\pgfpathlineto{\pgfqpoint{2.430730in}{1.620573in}}%
\pgfpathlineto{\pgfqpoint{2.421166in}{1.599946in}}%
\pgfpathlineto{\pgfqpoint{2.411779in}{1.581825in}}%
\pgfpathlineto{\pgfqpoint{2.410584in}{1.579318in}}%
\pgfpathlineto{\pgfqpoint{2.399680in}{1.558690in}}%
\pgfpathlineto{\pgfqpoint{2.387520in}{1.538062in}}%
\pgfpathlineto{\pgfqpoint{2.374045in}{1.517434in}}%
\pgfpathlineto{\pgfqpoint{2.370733in}{1.512779in}}%
\pgfpathlineto{\pgfqpoint{2.357796in}{1.496807in}}%
\pgfpathlineto{\pgfqpoint{2.339279in}{1.476179in}}%
\pgfpathlineto{\pgfqpoint{2.329688in}{1.466426in}}%
\pgfpathlineto{\pgfqpoint{2.314615in}{1.455551in}}%
\pgfpathlineto{\pgfqpoint{2.288643in}{1.438623in}}%
\pgfpathlineto{\pgfqpoint{2.278144in}{1.434923in}}%
\pgfpathlineto{\pgfqpoint{2.247597in}{1.425076in}}%
\pgfpathlineto{\pgfqpoint{2.206552in}{1.420299in}}%
\pgfpathlineto{\pgfqpoint{2.165507in}{1.418702in}}%
\pgfpathlineto{\pgfqpoint{2.124461in}{1.415096in}}%
\pgfpathlineto{\pgfqpoint{2.120897in}{1.414295in}}%
\pgfpathlineto{\pgfqpoint{2.083416in}{1.406025in}}%
\pgfpathlineto{\pgfqpoint{2.051985in}{1.393667in}}%
\pgfpathlineto{\pgfqpoint{2.042371in}{1.389878in}}%
\pgfpathlineto{\pgfqpoint{2.010549in}{1.373040in}}%
\pgfpathlineto{\pgfqpoint{2.001325in}{1.368050in}}%
\pgfpathlineto{\pgfqpoint{1.974453in}{1.352412in}}%
\pgfpathlineto{\pgfqpoint{1.960280in}{1.343831in}}%
\pgfpathlineto{\pgfqpoint{1.937451in}{1.331784in}}%
\pgfpathlineto{\pgfqpoint{1.919235in}{1.321639in}}%
\pgfpathlineto{\pgfqpoint{1.891451in}{1.311156in}}%
\pgfpathlineto{\pgfqpoint{1.878189in}{1.305823in}}%
\pgfpathlineto{\pgfqpoint{1.837144in}{1.299516in}}%
\pgfpathlineto{\pgfqpoint{1.796099in}{1.303121in}}%
\pgfpathlineto{\pgfqpoint{1.761656in}{1.311156in}}%
\pgfpathlineto{\pgfqpoint{1.755053in}{1.313040in}}%
\pgfpathlineto{\pgfqpoint{1.714008in}{1.320561in}}%
\pgfpathlineto{\pgfqpoint{1.672963in}{1.312282in}}%
\pgfpathlineto{\pgfqpoint{1.671459in}{1.311156in}}%
\pgfpathlineto{\pgfqpoint{1.642167in}{1.290528in}}%
\pgfpathlineto{\pgfqpoint{1.631917in}{1.283651in}}%
\pgfpathlineto{\pgfqpoint{1.618674in}{1.269900in}}%
\pgfpathlineto{\pgfqpoint{1.597098in}{1.249273in}}%
\pgfpathlineto{\pgfqpoint{1.590872in}{1.243755in}}%
\pgfpathlineto{\pgfqpoint{1.575428in}{1.228645in}}%
\pgfpathlineto{\pgfqpoint{1.552193in}{1.208017in}}%
\pgfpathlineto{\pgfqpoint{1.549827in}{1.206080in}}%
\pgfpathlineto{\pgfqpoint{1.522228in}{1.187389in}}%
\pgfpathlineto{\pgfqpoint{1.508781in}{1.179169in}}%
\pgfpathlineto{\pgfqpoint{1.471335in}{1.166761in}}%
\pgfpathlineto{\pgfqpoint{1.467736in}{1.165685in}}%
\pgfpathlineto{\pgfqpoint{1.426691in}{1.165130in}}%
\pgfpathlineto{\pgfqpoint{1.419006in}{1.166761in}}%
\pgfpathlineto{\pgfqpoint{1.385645in}{1.173883in}}%
\pgfpathlineto{\pgfqpoint{1.344600in}{1.186284in}}%
\pgfpathlineto{\pgfqpoint{1.339958in}{1.187389in}}%
\pgfpathlineto{\pgfqpoint{1.303555in}{1.196368in}}%
\pgfpathlineto{\pgfqpoint{1.262509in}{1.197964in}}%
\pgfpathlineto{\pgfqpoint{1.222737in}{1.187389in}}%
\pgfpathlineto{\pgfqpoint{1.221464in}{1.187056in}}%
\pgfpathlineto{\pgfqpoint{1.186694in}{1.166761in}}%
\pgfpathlineto{\pgfqpoint{1.180419in}{1.163066in}}%
\pgfpathlineto{\pgfqpoint{1.160182in}{1.146134in}}%
\pgfpathlineto{\pgfqpoint{1.139373in}{1.128173in}}%
\pgfpathlineto{\pgfqpoint{1.136627in}{1.125506in}}%
\pgfpathlineto{\pgfqpoint{1.115553in}{1.104878in}}%
\pgfpathlineto{\pgfqpoint{1.098328in}{1.087297in}}%
\pgfpathlineto{\pgfqpoint{1.095190in}{1.084250in}}%
\pgfpathlineto{\pgfqpoint{1.074269in}{1.063622in}}%
\pgfpathlineto{\pgfqpoint{1.057283in}{1.045980in}}%
\pgfpathlineto{\pgfqpoint{1.053811in}{1.042994in}}%
\pgfpathlineto{\pgfqpoint{1.030455in}{1.022367in}}%
\pgfpathlineto{\pgfqpoint{1.016237in}{1.009085in}}%
\pgfpathlineto{\pgfqpoint{1.005628in}{1.001739in}}%
\pgfpathlineto{\pgfqpoint{0.977570in}{0.981111in}}%
\pgfpathlineto{\pgfqpoint{0.975192in}{0.979300in}}%
\pgfpathlineto{\pgfqpoint{0.940014in}{0.960483in}}%
\pgfpathlineto{\pgfqpoint{0.934147in}{0.957160in}}%
\pgfpathlineto{\pgfqpoint{0.893101in}{0.940099in}}%
\pgfpathlineto{\pgfqpoint{0.892450in}{0.939855in}}%
\pgfpathlineto{\pgfqpoint{0.852056in}{0.924459in}}%
\pgfpathlineto{\pgfqpoint{0.839916in}{0.919227in}}%
\pgfpathlineto{\pgfqpoint{0.811011in}{0.906903in}}%
\pgfpathlineto{\pgfqpoint{0.793531in}{0.898600in}}%
\pgfpathlineto{\pgfqpoint{0.769965in}{0.887849in}}%
\pgfpathlineto{\pgfqpoint{0.746179in}{0.877972in}}%
\pgfpathlineto{\pgfqpoint{0.728920in}{0.871256in}}%
\pgfpathlineto{\pgfqpoint{0.687875in}{0.862243in}}%
\pgfpathlineto{\pgfqpoint{0.646829in}{0.864588in}}%
\pgfpathclose%
\pgfpathmoveto{\pgfqpoint{1.837144in}{1.511217in}}%
\pgfpathlineto{\pgfqpoint{1.851045in}{1.496807in}}%
\pgfpathlineto{\pgfqpoint{1.878189in}{1.477110in}}%
\pgfpathlineto{\pgfqpoint{1.886280in}{1.476179in}}%
\pgfpathlineto{\pgfqpoint{1.919235in}{1.473264in}}%
\pgfpathlineto{\pgfqpoint{1.929168in}{1.476179in}}%
\pgfpathlineto{\pgfqpoint{1.960280in}{1.484612in}}%
\pgfpathlineto{\pgfqpoint{1.987631in}{1.496807in}}%
\pgfpathlineto{\pgfqpoint{2.001325in}{1.502574in}}%
\pgfpathlineto{\pgfqpoint{2.034688in}{1.517434in}}%
\pgfpathlineto{\pgfqpoint{2.042371in}{1.520751in}}%
\pgfpathlineto{\pgfqpoint{2.083416in}{1.534795in}}%
\pgfpathlineto{\pgfqpoint{2.099705in}{1.538062in}}%
\pgfpathlineto{\pgfqpoint{2.124461in}{1.543152in}}%
\pgfpathlineto{\pgfqpoint{2.165507in}{1.547483in}}%
\pgfpathlineto{\pgfqpoint{2.206552in}{1.552368in}}%
\pgfpathlineto{\pgfqpoint{2.228328in}{1.558690in}}%
\pgfpathlineto{\pgfqpoint{2.247597in}{1.564899in}}%
\pgfpathlineto{\pgfqpoint{2.268911in}{1.579318in}}%
\pgfpathlineto{\pgfqpoint{2.288643in}{1.594552in}}%
\pgfpathlineto{\pgfqpoint{2.292980in}{1.599946in}}%
\pgfpathlineto{\pgfqpoint{2.307783in}{1.620573in}}%
\pgfpathlineto{\pgfqpoint{2.320610in}{1.641201in}}%
\pgfpathlineto{\pgfqpoint{2.329688in}{1.658121in}}%
\pgfpathlineto{\pgfqpoint{2.331235in}{1.661829in}}%
\pgfpathlineto{\pgfqpoint{2.338763in}{1.682457in}}%
\pgfpathlineto{\pgfqpoint{2.345143in}{1.703085in}}%
\pgfpathlineto{\pgfqpoint{2.350440in}{1.723713in}}%
\pgfpathlineto{\pgfqpoint{2.354713in}{1.744340in}}%
\pgfpathlineto{\pgfqpoint{2.358017in}{1.764968in}}%
\pgfpathlineto{\pgfqpoint{2.360405in}{1.785596in}}%
\pgfpathlineto{\pgfqpoint{2.361925in}{1.806224in}}%
\pgfpathlineto{\pgfqpoint{2.362620in}{1.826852in}}%
\pgfpathlineto{\pgfqpoint{2.362534in}{1.847480in}}%
\pgfpathlineto{\pgfqpoint{2.361704in}{1.868107in}}%
\pgfpathlineto{\pgfqpoint{2.360168in}{1.888735in}}%
\pgfpathlineto{\pgfqpoint{2.357959in}{1.909363in}}%
\pgfpathlineto{\pgfqpoint{2.355109in}{1.929991in}}%
\pgfpathlineto{\pgfqpoint{2.351649in}{1.950619in}}%
\pgfpathlineto{\pgfqpoint{2.347606in}{1.971247in}}%
\pgfpathlineto{\pgfqpoint{2.343006in}{1.991874in}}%
\pgfpathlineto{\pgfqpoint{2.337875in}{2.012502in}}%
\pgfpathlineto{\pgfqpoint{2.332236in}{2.033130in}}%
\pgfpathlineto{\pgfqpoint{2.329688in}{2.041637in}}%
\pgfpathlineto{\pgfqpoint{2.325614in}{2.053758in}}%
\pgfpathlineto{\pgfqpoint{2.318150in}{2.074386in}}%
\pgfpathlineto{\pgfqpoint{2.310229in}{2.095013in}}%
\pgfpathlineto{\pgfqpoint{2.301876in}{2.115641in}}%
\pgfpathlineto{\pgfqpoint{2.293113in}{2.136269in}}%
\pgfpathlineto{\pgfqpoint{2.288643in}{2.146257in}}%
\pgfpathlineto{\pgfqpoint{2.282579in}{2.156897in}}%
\pgfpathlineto{\pgfqpoint{2.270307in}{2.177525in}}%
\pgfpathlineto{\pgfqpoint{2.257657in}{2.198153in}}%
\pgfpathlineto{\pgfqpoint{2.247597in}{2.214036in}}%
\pgfpathlineto{\pgfqpoint{2.243116in}{2.218780in}}%
\pgfpathlineto{\pgfqpoint{2.222906in}{2.239408in}}%
\pgfpathlineto{\pgfqpoint{2.206552in}{2.255731in}}%
\pgfpathlineto{\pgfqpoint{2.197966in}{2.260036in}}%
\pgfpathlineto{\pgfqpoint{2.165507in}{2.275716in}}%
\pgfpathlineto{\pgfqpoint{2.124461in}{2.277177in}}%
\pgfpathlineto{\pgfqpoint{2.083416in}{2.262472in}}%
\pgfpathlineto{\pgfqpoint{2.080014in}{2.260036in}}%
\pgfpathlineto{\pgfqpoint{2.051494in}{2.239408in}}%
\pgfpathlineto{\pgfqpoint{2.042371in}{2.232798in}}%
\pgfpathlineto{\pgfqpoint{2.029567in}{2.218780in}}%
\pgfpathlineto{\pgfqpoint{2.010526in}{2.198153in}}%
\pgfpathlineto{\pgfqpoint{2.001325in}{2.188213in}}%
\pgfpathlineto{\pgfqpoint{1.994150in}{2.177525in}}%
\pgfpathlineto{\pgfqpoint{1.980322in}{2.156897in}}%
\pgfpathlineto{\pgfqpoint{1.966236in}{2.136269in}}%
\pgfpathlineto{\pgfqpoint{1.960280in}{2.127509in}}%
\pgfpathlineto{\pgfqpoint{1.954097in}{2.115641in}}%
\pgfpathlineto{\pgfqpoint{1.943423in}{2.095013in}}%
\pgfpathlineto{\pgfqpoint{1.932648in}{2.074386in}}%
\pgfpathlineto{\pgfqpoint{1.921761in}{2.053758in}}%
\pgfpathlineto{\pgfqpoint{1.919235in}{2.048871in}}%
\pgfpathlineto{\pgfqpoint{1.912641in}{2.033130in}}%
\pgfpathlineto{\pgfqpoint{1.904092in}{2.012502in}}%
\pgfpathlineto{\pgfqpoint{1.895568in}{1.991874in}}%
\pgfpathlineto{\pgfqpoint{1.887071in}{1.971247in}}%
\pgfpathlineto{\pgfqpoint{1.878604in}{1.950619in}}%
\pgfpathlineto{\pgfqpoint{1.878189in}{1.949562in}}%
\pgfpathlineto{\pgfqpoint{1.871511in}{1.929991in}}%
\pgfpathlineto{\pgfqpoint{1.864627in}{1.909363in}}%
\pgfpathlineto{\pgfqpoint{1.857900in}{1.888735in}}%
\pgfpathlineto{\pgfqpoint{1.851349in}{1.868107in}}%
\pgfpathlineto{\pgfqpoint{1.844997in}{1.847480in}}%
\pgfpathlineto{\pgfqpoint{1.838870in}{1.826852in}}%
\pgfpathlineto{\pgfqpoint{1.837144in}{1.820594in}}%
\pgfpathlineto{\pgfqpoint{1.833359in}{1.806224in}}%
\pgfpathlineto{\pgfqpoint{1.828329in}{1.785596in}}%
\pgfpathlineto{\pgfqpoint{1.823683in}{1.764968in}}%
\pgfpathlineto{\pgfqpoint{1.819473in}{1.744340in}}%
\pgfpathlineto{\pgfqpoint{1.815757in}{1.723713in}}%
\pgfpathlineto{\pgfqpoint{1.812606in}{1.703085in}}%
\pgfpathlineto{\pgfqpoint{1.810107in}{1.682457in}}%
\pgfpathlineto{\pgfqpoint{1.808364in}{1.661829in}}%
\pgfpathlineto{\pgfqpoint{1.807502in}{1.641201in}}%
\pgfpathlineto{\pgfqpoint{1.807680in}{1.620573in}}%
\pgfpathlineto{\pgfqpoint{1.809094in}{1.599946in}}%
\pgfpathlineto{\pgfqpoint{1.811994in}{1.579318in}}%
\pgfpathlineto{\pgfqpoint{1.816702in}{1.558690in}}%
\pgfpathlineto{\pgfqpoint{1.823641in}{1.538062in}}%
\pgfpathlineto{\pgfqpoint{1.833377in}{1.517434in}}%
\pgfpathclose%
\pgfusepath{stroke,fill}%
\end{pgfscope}%
\begin{pgfscope}%
\pgfpathrectangle{\pgfqpoint{0.605784in}{0.382904in}}{\pgfqpoint{4.063488in}{2.042155in}}%
\pgfusepath{clip}%
\pgfsetbuttcap%
\pgfsetroundjoin%
\definecolor{currentfill}{rgb}{0.281887,0.150881,0.465405}%
\pgfsetfillcolor{currentfill}%
\pgfsetlinewidth{1.003750pt}%
\definecolor{currentstroke}{rgb}{0.281887,0.150881,0.465405}%
\pgfsetstrokecolor{currentstroke}%
\pgfsetdash{}{0pt}%
\pgfpathmoveto{\pgfqpoint{2.534462in}{1.042994in}}%
\pgfpathlineto{\pgfqpoint{2.533455in}{1.063622in}}%
\pgfpathlineto{\pgfqpoint{2.532449in}{1.084250in}}%
\pgfpathlineto{\pgfqpoint{2.531444in}{1.104878in}}%
\pgfpathlineto{\pgfqpoint{2.530441in}{1.125506in}}%
\pgfpathlineto{\pgfqpoint{2.529440in}{1.146134in}}%
\pgfpathlineto{\pgfqpoint{2.528439in}{1.166761in}}%
\pgfpathlineto{\pgfqpoint{2.527441in}{1.187389in}}%
\pgfpathlineto{\pgfqpoint{2.526445in}{1.208017in}}%
\pgfpathlineto{\pgfqpoint{2.525451in}{1.228645in}}%
\pgfpathlineto{\pgfqpoint{2.524460in}{1.249273in}}%
\pgfpathlineto{\pgfqpoint{2.523471in}{1.269900in}}%
\pgfpathlineto{\pgfqpoint{2.522486in}{1.290528in}}%
\pgfpathlineto{\pgfqpoint{2.521504in}{1.311156in}}%
\pgfpathlineto{\pgfqpoint{2.520526in}{1.331784in}}%
\pgfpathlineto{\pgfqpoint{2.519553in}{1.352412in}}%
\pgfpathlineto{\pgfqpoint{2.518585in}{1.373040in}}%
\pgfpathlineto{\pgfqpoint{2.517624in}{1.393667in}}%
\pgfpathlineto{\pgfqpoint{2.516670in}{1.414295in}}%
\pgfpathlineto{\pgfqpoint{2.515725in}{1.434923in}}%
\pgfpathlineto{\pgfqpoint{2.514790in}{1.455551in}}%
\pgfpathlineto{\pgfqpoint{2.513867in}{1.476179in}}%
\pgfpathlineto{\pgfqpoint{2.512960in}{1.496807in}}%
\pgfpathlineto{\pgfqpoint{2.512071in}{1.517434in}}%
\pgfpathlineto{\pgfqpoint{2.511206in}{1.538062in}}%
\pgfpathlineto{\pgfqpoint{2.510372in}{1.558690in}}%
\pgfpathlineto{\pgfqpoint{2.509577in}{1.579318in}}%
\pgfpathlineto{\pgfqpoint{2.508836in}{1.599946in}}%
\pgfpathlineto{\pgfqpoint{2.508169in}{1.620573in}}%
\pgfpathlineto{\pgfqpoint{2.507610in}{1.641201in}}%
\pgfpathlineto{\pgfqpoint{2.507218in}{1.661829in}}%
\pgfpathlineto{\pgfqpoint{2.507099in}{1.682457in}}%
\pgfpathlineto{\pgfqpoint{2.507478in}{1.703085in}}%
\pgfpathlineto{\pgfqpoint{2.508923in}{1.723713in}}%
\pgfpathlineto{\pgfqpoint{2.513384in}{1.744340in}}%
\pgfpathlineto{\pgfqpoint{2.534915in}{1.764918in}}%
\pgfpathlineto{\pgfqpoint{2.575960in}{1.764918in}}%
\pgfpathlineto{\pgfqpoint{2.617005in}{1.764918in}}%
\pgfpathlineto{\pgfqpoint{2.658051in}{1.764918in}}%
\pgfpathlineto{\pgfqpoint{2.699096in}{1.764918in}}%
\pgfpathlineto{\pgfqpoint{2.740141in}{1.764918in}}%
\pgfpathlineto{\pgfqpoint{2.781187in}{1.764918in}}%
\pgfpathlineto{\pgfqpoint{2.822232in}{1.764918in}}%
\pgfpathlineto{\pgfqpoint{2.863277in}{1.764918in}}%
\pgfpathlineto{\pgfqpoint{2.904323in}{1.764918in}}%
\pgfpathlineto{\pgfqpoint{2.945368in}{1.764918in}}%
\pgfpathlineto{\pgfqpoint{2.986413in}{1.764918in}}%
\pgfpathlineto{\pgfqpoint{3.027459in}{1.764918in}}%
\pgfpathlineto{\pgfqpoint{3.068504in}{1.764918in}}%
\pgfpathlineto{\pgfqpoint{3.109549in}{1.764918in}}%
\pgfpathlineto{\pgfqpoint{3.150595in}{1.764918in}}%
\pgfpathlineto{\pgfqpoint{3.191640in}{1.764918in}}%
\pgfpathlineto{\pgfqpoint{3.232685in}{1.764918in}}%
\pgfpathlineto{\pgfqpoint{3.273731in}{1.764918in}}%
\pgfpathlineto{\pgfqpoint{3.314776in}{1.764918in}}%
\pgfpathlineto{\pgfqpoint{3.355821in}{1.764918in}}%
\pgfpathlineto{\pgfqpoint{3.396867in}{1.764918in}}%
\pgfpathlineto{\pgfqpoint{3.437912in}{1.764918in}}%
\pgfpathlineto{\pgfqpoint{3.478957in}{1.764918in}}%
\pgfpathlineto{\pgfqpoint{3.520003in}{1.764918in}}%
\pgfpathlineto{\pgfqpoint{3.561048in}{1.764918in}}%
\pgfpathlineto{\pgfqpoint{3.602093in}{1.764918in}}%
\pgfpathlineto{\pgfqpoint{3.643139in}{1.764918in}}%
\pgfpathlineto{\pgfqpoint{3.646178in}{1.744340in}}%
\pgfpathlineto{\pgfqpoint{3.649329in}{1.723713in}}%
\pgfpathlineto{\pgfqpoint{3.652601in}{1.703085in}}%
\pgfpathlineto{\pgfqpoint{3.656013in}{1.682457in}}%
\pgfpathlineto{\pgfqpoint{3.659587in}{1.661829in}}%
\pgfpathlineto{\pgfqpoint{3.663353in}{1.641201in}}%
\pgfpathlineto{\pgfqpoint{3.667347in}{1.620573in}}%
\pgfpathlineto{\pgfqpoint{3.671613in}{1.599946in}}%
\pgfpathlineto{\pgfqpoint{3.676211in}{1.579318in}}%
\pgfpathlineto{\pgfqpoint{3.681218in}{1.558690in}}%
\pgfpathlineto{\pgfqpoint{3.684184in}{1.547190in}}%
\pgfpathlineto{\pgfqpoint{3.725229in}{1.547190in}}%
\pgfpathlineto{\pgfqpoint{3.766275in}{1.547190in}}%
\pgfpathlineto{\pgfqpoint{3.807320in}{1.547190in}}%
\pgfpathlineto{\pgfqpoint{3.848365in}{1.547190in}}%
\pgfpathlineto{\pgfqpoint{3.889411in}{1.547190in}}%
\pgfpathlineto{\pgfqpoint{3.930456in}{1.547190in}}%
\pgfpathlineto{\pgfqpoint{3.971501in}{1.547190in}}%
\pgfpathlineto{\pgfqpoint{4.012547in}{1.547190in}}%
\pgfpathlineto{\pgfqpoint{4.053592in}{1.547190in}}%
\pgfpathlineto{\pgfqpoint{4.094637in}{1.547190in}}%
\pgfpathlineto{\pgfqpoint{4.135683in}{1.547190in}}%
\pgfpathlineto{\pgfqpoint{4.176728in}{1.547190in}}%
\pgfpathlineto{\pgfqpoint{4.217773in}{1.547190in}}%
\pgfpathlineto{\pgfqpoint{4.258819in}{1.547190in}}%
\pgfpathlineto{\pgfqpoint{4.299864in}{1.547190in}}%
\pgfpathlineto{\pgfqpoint{4.340909in}{1.547190in}}%
\pgfpathlineto{\pgfqpoint{4.381955in}{1.547190in}}%
\pgfpathlineto{\pgfqpoint{4.423000in}{1.547190in}}%
\pgfpathlineto{\pgfqpoint{4.464045in}{1.547190in}}%
\pgfpathlineto{\pgfqpoint{4.505091in}{1.547190in}}%
\pgfpathlineto{\pgfqpoint{4.546136in}{1.547190in}}%
\pgfpathlineto{\pgfqpoint{4.587181in}{1.547190in}}%
\pgfpathlineto{\pgfqpoint{4.628227in}{1.547190in}}%
\pgfpathlineto{\pgfqpoint{4.669272in}{1.547190in}}%
\pgfpathlineto{\pgfqpoint{4.669272in}{1.538062in}}%
\pgfpathlineto{\pgfqpoint{4.669272in}{1.517434in}}%
\pgfpathlineto{\pgfqpoint{4.669272in}{1.496807in}}%
\pgfpathlineto{\pgfqpoint{4.669272in}{1.476179in}}%
\pgfpathlineto{\pgfqpoint{4.669272in}{1.455551in}}%
\pgfpathlineto{\pgfqpoint{4.669272in}{1.434923in}}%
\pgfpathlineto{\pgfqpoint{4.669272in}{1.414295in}}%
\pgfpathlineto{\pgfqpoint{4.669272in}{1.393667in}}%
\pgfpathlineto{\pgfqpoint{4.669272in}{1.373040in}}%
\pgfpathlineto{\pgfqpoint{4.669272in}{1.352412in}}%
\pgfpathlineto{\pgfqpoint{4.669272in}{1.334339in}}%
\pgfpathlineto{\pgfqpoint{4.628227in}{1.334339in}}%
\pgfpathlineto{\pgfqpoint{4.587181in}{1.334339in}}%
\pgfpathlineto{\pgfqpoint{4.546136in}{1.334339in}}%
\pgfpathlineto{\pgfqpoint{4.505091in}{1.334339in}}%
\pgfpathlineto{\pgfqpoint{4.464045in}{1.334339in}}%
\pgfpathlineto{\pgfqpoint{4.423000in}{1.334339in}}%
\pgfpathlineto{\pgfqpoint{4.381955in}{1.334339in}}%
\pgfpathlineto{\pgfqpoint{4.340909in}{1.334339in}}%
\pgfpathlineto{\pgfqpoint{4.299864in}{1.334339in}}%
\pgfpathlineto{\pgfqpoint{4.258819in}{1.334339in}}%
\pgfpathlineto{\pgfqpoint{4.217773in}{1.334339in}}%
\pgfpathlineto{\pgfqpoint{4.176728in}{1.334339in}}%
\pgfpathlineto{\pgfqpoint{4.135683in}{1.334339in}}%
\pgfpathlineto{\pgfqpoint{4.094637in}{1.334339in}}%
\pgfpathlineto{\pgfqpoint{4.053592in}{1.334339in}}%
\pgfpathlineto{\pgfqpoint{4.012547in}{1.334339in}}%
\pgfpathlineto{\pgfqpoint{3.971501in}{1.334339in}}%
\pgfpathlineto{\pgfqpoint{3.930456in}{1.334339in}}%
\pgfpathlineto{\pgfqpoint{3.889411in}{1.334339in}}%
\pgfpathlineto{\pgfqpoint{3.848365in}{1.334339in}}%
\pgfpathlineto{\pgfqpoint{3.807320in}{1.334339in}}%
\pgfpathlineto{\pgfqpoint{3.766275in}{1.334339in}}%
\pgfpathlineto{\pgfqpoint{3.725229in}{1.334339in}}%
\pgfpathlineto{\pgfqpoint{3.684184in}{1.334339in}}%
\pgfpathlineto{\pgfqpoint{3.675349in}{1.352412in}}%
\pgfpathlineto{\pgfqpoint{3.669022in}{1.373040in}}%
\pgfpathlineto{\pgfqpoint{3.664448in}{1.393667in}}%
\pgfpathlineto{\pgfqpoint{3.660750in}{1.414295in}}%
\pgfpathlineto{\pgfqpoint{3.657553in}{1.434923in}}%
\pgfpathlineto{\pgfqpoint{3.654668in}{1.455551in}}%
\pgfpathlineto{\pgfqpoint{3.651992in}{1.476179in}}%
\pgfpathlineto{\pgfqpoint{3.649462in}{1.496807in}}%
\pgfpathlineto{\pgfqpoint{3.647039in}{1.517434in}}%
\pgfpathlineto{\pgfqpoint{3.644695in}{1.538062in}}%
\pgfpathlineto{\pgfqpoint{3.643139in}{1.551765in}}%
\pgfpathlineto{\pgfqpoint{3.602093in}{1.551765in}}%
\pgfpathlineto{\pgfqpoint{3.561048in}{1.551765in}}%
\pgfpathlineto{\pgfqpoint{3.520003in}{1.551765in}}%
\pgfpathlineto{\pgfqpoint{3.478957in}{1.551765in}}%
\pgfpathlineto{\pgfqpoint{3.437912in}{1.551765in}}%
\pgfpathlineto{\pgfqpoint{3.396867in}{1.551765in}}%
\pgfpathlineto{\pgfqpoint{3.355821in}{1.551765in}}%
\pgfpathlineto{\pgfqpoint{3.314776in}{1.551765in}}%
\pgfpathlineto{\pgfqpoint{3.273731in}{1.551765in}}%
\pgfpathlineto{\pgfqpoint{3.232685in}{1.551765in}}%
\pgfpathlineto{\pgfqpoint{3.191640in}{1.551765in}}%
\pgfpathlineto{\pgfqpoint{3.150595in}{1.551765in}}%
\pgfpathlineto{\pgfqpoint{3.109549in}{1.551765in}}%
\pgfpathlineto{\pgfqpoint{3.068504in}{1.551765in}}%
\pgfpathlineto{\pgfqpoint{3.027459in}{1.551765in}}%
\pgfpathlineto{\pgfqpoint{2.986413in}{1.551765in}}%
\pgfpathlineto{\pgfqpoint{2.945368in}{1.551765in}}%
\pgfpathlineto{\pgfqpoint{2.904323in}{1.551765in}}%
\pgfpathlineto{\pgfqpoint{2.863277in}{1.551765in}}%
\pgfpathlineto{\pgfqpoint{2.822232in}{1.551765in}}%
\pgfpathlineto{\pgfqpoint{2.781187in}{1.551765in}}%
\pgfpathlineto{\pgfqpoint{2.740141in}{1.551765in}}%
\pgfpathlineto{\pgfqpoint{2.699096in}{1.551765in}}%
\pgfpathlineto{\pgfqpoint{2.658051in}{1.551765in}}%
\pgfpathlineto{\pgfqpoint{2.617005in}{1.551765in}}%
\pgfpathlineto{\pgfqpoint{2.575960in}{1.551765in}}%
\pgfpathlineto{\pgfqpoint{2.534915in}{1.551765in}}%
\pgfpathlineto{\pgfqpoint{2.534068in}{1.538062in}}%
\pgfpathlineto{\pgfqpoint{2.533124in}{1.517434in}}%
\pgfpathlineto{\pgfqpoint{2.532469in}{1.496807in}}%
\pgfpathlineto{\pgfqpoint{2.532044in}{1.476179in}}%
\pgfpathlineto{\pgfqpoint{2.531804in}{1.455551in}}%
\pgfpathlineto{\pgfqpoint{2.531716in}{1.434923in}}%
\pgfpathlineto{\pgfqpoint{2.531755in}{1.414295in}}%
\pgfpathlineto{\pgfqpoint{2.531899in}{1.393667in}}%
\pgfpathlineto{\pgfqpoint{2.532134in}{1.373040in}}%
\pgfpathlineto{\pgfqpoint{2.532445in}{1.352412in}}%
\pgfpathlineto{\pgfqpoint{2.532822in}{1.331784in}}%
\pgfpathlineto{\pgfqpoint{2.533257in}{1.311156in}}%
\pgfpathlineto{\pgfqpoint{2.533742in}{1.290528in}}%
\pgfpathlineto{\pgfqpoint{2.534270in}{1.269900in}}%
\pgfpathlineto{\pgfqpoint{2.534837in}{1.249273in}}%
\pgfpathlineto{\pgfqpoint{2.534915in}{1.246718in}}%
\pgfpathlineto{\pgfqpoint{2.575960in}{1.246718in}}%
\pgfpathlineto{\pgfqpoint{2.617005in}{1.246718in}}%
\pgfpathlineto{\pgfqpoint{2.658051in}{1.246718in}}%
\pgfpathlineto{\pgfqpoint{2.699096in}{1.246718in}}%
\pgfpathlineto{\pgfqpoint{2.740141in}{1.246718in}}%
\pgfpathlineto{\pgfqpoint{2.781187in}{1.246718in}}%
\pgfpathlineto{\pgfqpoint{2.822232in}{1.246718in}}%
\pgfpathlineto{\pgfqpoint{2.863277in}{1.246718in}}%
\pgfpathlineto{\pgfqpoint{2.904323in}{1.246718in}}%
\pgfpathlineto{\pgfqpoint{2.945368in}{1.246718in}}%
\pgfpathlineto{\pgfqpoint{2.986413in}{1.246718in}}%
\pgfpathlineto{\pgfqpoint{3.027459in}{1.246718in}}%
\pgfpathlineto{\pgfqpoint{3.068504in}{1.246718in}}%
\pgfpathlineto{\pgfqpoint{3.109549in}{1.246718in}}%
\pgfpathlineto{\pgfqpoint{3.150595in}{1.246718in}}%
\pgfpathlineto{\pgfqpoint{3.191640in}{1.246718in}}%
\pgfpathlineto{\pgfqpoint{3.232685in}{1.246718in}}%
\pgfpathlineto{\pgfqpoint{3.273731in}{1.246718in}}%
\pgfpathlineto{\pgfqpoint{3.314776in}{1.246718in}}%
\pgfpathlineto{\pgfqpoint{3.355821in}{1.246718in}}%
\pgfpathlineto{\pgfqpoint{3.396867in}{1.246718in}}%
\pgfpathlineto{\pgfqpoint{3.437912in}{1.246718in}}%
\pgfpathlineto{\pgfqpoint{3.478957in}{1.246718in}}%
\pgfpathlineto{\pgfqpoint{3.520003in}{1.246718in}}%
\pgfpathlineto{\pgfqpoint{3.561048in}{1.246718in}}%
\pgfpathlineto{\pgfqpoint{3.602093in}{1.246718in}}%
\pgfpathlineto{\pgfqpoint{3.643139in}{1.246718in}}%
\pgfpathlineto{\pgfqpoint{3.651973in}{1.228645in}}%
\pgfpathlineto{\pgfqpoint{3.658300in}{1.208017in}}%
\pgfpathlineto{\pgfqpoint{3.662874in}{1.187389in}}%
\pgfpathlineto{\pgfqpoint{3.666573in}{1.166761in}}%
\pgfpathlineto{\pgfqpoint{3.669770in}{1.146134in}}%
\pgfpathlineto{\pgfqpoint{3.672654in}{1.125506in}}%
\pgfpathlineto{\pgfqpoint{3.675330in}{1.104878in}}%
\pgfpathlineto{\pgfqpoint{3.677860in}{1.084250in}}%
\pgfpathlineto{\pgfqpoint{3.680284in}{1.063622in}}%
\pgfpathlineto{\pgfqpoint{3.682627in}{1.042994in}}%
\pgfpathlineto{\pgfqpoint{3.684184in}{1.029292in}}%
\pgfpathlineto{\pgfqpoint{3.725229in}{1.029292in}}%
\pgfpathlineto{\pgfqpoint{3.766275in}{1.029292in}}%
\pgfpathlineto{\pgfqpoint{3.807320in}{1.029292in}}%
\pgfpathlineto{\pgfqpoint{3.848365in}{1.029292in}}%
\pgfpathlineto{\pgfqpoint{3.889411in}{1.029292in}}%
\pgfpathlineto{\pgfqpoint{3.930456in}{1.029292in}}%
\pgfpathlineto{\pgfqpoint{3.971501in}{1.029292in}}%
\pgfpathlineto{\pgfqpoint{4.012547in}{1.029292in}}%
\pgfpathlineto{\pgfqpoint{4.053592in}{1.029292in}}%
\pgfpathlineto{\pgfqpoint{4.094637in}{1.029292in}}%
\pgfpathlineto{\pgfqpoint{4.135683in}{1.029292in}}%
\pgfpathlineto{\pgfqpoint{4.176728in}{1.029292in}}%
\pgfpathlineto{\pgfqpoint{4.217773in}{1.029292in}}%
\pgfpathlineto{\pgfqpoint{4.258819in}{1.029292in}}%
\pgfpathlineto{\pgfqpoint{4.299864in}{1.029292in}}%
\pgfpathlineto{\pgfqpoint{4.340909in}{1.029292in}}%
\pgfpathlineto{\pgfqpoint{4.381955in}{1.029292in}}%
\pgfpathlineto{\pgfqpoint{4.423000in}{1.029292in}}%
\pgfpathlineto{\pgfqpoint{4.464045in}{1.029292in}}%
\pgfpathlineto{\pgfqpoint{4.505091in}{1.029292in}}%
\pgfpathlineto{\pgfqpoint{4.546136in}{1.029292in}}%
\pgfpathlineto{\pgfqpoint{4.587181in}{1.029292in}}%
\pgfpathlineto{\pgfqpoint{4.628227in}{1.029292in}}%
\pgfpathlineto{\pgfqpoint{4.669272in}{1.029292in}}%
\pgfpathlineto{\pgfqpoint{4.669272in}{1.022367in}}%
\pgfpathlineto{\pgfqpoint{4.669272in}{1.001739in}}%
\pgfpathlineto{\pgfqpoint{4.669272in}{0.981111in}}%
\pgfpathlineto{\pgfqpoint{4.669272in}{0.960483in}}%
\pgfpathlineto{\pgfqpoint{4.669272in}{0.939855in}}%
\pgfpathlineto{\pgfqpoint{4.669272in}{0.919227in}}%
\pgfpathlineto{\pgfqpoint{4.669272in}{0.898600in}}%
\pgfpathlineto{\pgfqpoint{4.669272in}{0.877972in}}%
\pgfpathlineto{\pgfqpoint{4.669272in}{0.857344in}}%
\pgfpathlineto{\pgfqpoint{4.669272in}{0.836716in}}%
\pgfpathlineto{\pgfqpoint{4.669272in}{0.816139in}}%
\pgfpathlineto{\pgfqpoint{4.628227in}{0.816139in}}%
\pgfpathlineto{\pgfqpoint{4.587181in}{0.816139in}}%
\pgfpathlineto{\pgfqpoint{4.546136in}{0.816139in}}%
\pgfpathlineto{\pgfqpoint{4.505091in}{0.816139in}}%
\pgfpathlineto{\pgfqpoint{4.464045in}{0.816139in}}%
\pgfpathlineto{\pgfqpoint{4.423000in}{0.816139in}}%
\pgfpathlineto{\pgfqpoint{4.381955in}{0.816139in}}%
\pgfpathlineto{\pgfqpoint{4.340909in}{0.816139in}}%
\pgfpathlineto{\pgfqpoint{4.299864in}{0.816139in}}%
\pgfpathlineto{\pgfqpoint{4.258819in}{0.816139in}}%
\pgfpathlineto{\pgfqpoint{4.217773in}{0.816139in}}%
\pgfpathlineto{\pgfqpoint{4.176728in}{0.816139in}}%
\pgfpathlineto{\pgfqpoint{4.135683in}{0.816139in}}%
\pgfpathlineto{\pgfqpoint{4.094637in}{0.816139in}}%
\pgfpathlineto{\pgfqpoint{4.053592in}{0.816139in}}%
\pgfpathlineto{\pgfqpoint{4.012547in}{0.816139in}}%
\pgfpathlineto{\pgfqpoint{3.971501in}{0.816139in}}%
\pgfpathlineto{\pgfqpoint{3.930456in}{0.816139in}}%
\pgfpathlineto{\pgfqpoint{3.889411in}{0.816139in}}%
\pgfpathlineto{\pgfqpoint{3.848365in}{0.816139in}}%
\pgfpathlineto{\pgfqpoint{3.807320in}{0.816139in}}%
\pgfpathlineto{\pgfqpoint{3.766275in}{0.816139in}}%
\pgfpathlineto{\pgfqpoint{3.725229in}{0.816139in}}%
\pgfpathlineto{\pgfqpoint{3.684184in}{0.816139in}}%
\pgfpathlineto{\pgfqpoint{3.681145in}{0.836716in}}%
\pgfpathlineto{\pgfqpoint{3.677993in}{0.857344in}}%
\pgfpathlineto{\pgfqpoint{3.674721in}{0.877972in}}%
\pgfpathlineto{\pgfqpoint{3.671310in}{0.898600in}}%
\pgfpathlineto{\pgfqpoint{3.667735in}{0.919227in}}%
\pgfpathlineto{\pgfqpoint{3.663969in}{0.939855in}}%
\pgfpathlineto{\pgfqpoint{3.659976in}{0.960483in}}%
\pgfpathlineto{\pgfqpoint{3.655709in}{0.981111in}}%
\pgfpathlineto{\pgfqpoint{3.651111in}{1.001739in}}%
\pgfpathlineto{\pgfqpoint{3.646105in}{1.022367in}}%
\pgfpathlineto{\pgfqpoint{3.643139in}{1.033867in}}%
\pgfpathlineto{\pgfqpoint{3.602093in}{1.033867in}}%
\pgfpathlineto{\pgfqpoint{3.561048in}{1.033867in}}%
\pgfpathlineto{\pgfqpoint{3.520003in}{1.033867in}}%
\pgfpathlineto{\pgfqpoint{3.478957in}{1.033867in}}%
\pgfpathlineto{\pgfqpoint{3.437912in}{1.033867in}}%
\pgfpathlineto{\pgfqpoint{3.396867in}{1.033867in}}%
\pgfpathlineto{\pgfqpoint{3.355821in}{1.033867in}}%
\pgfpathlineto{\pgfqpoint{3.314776in}{1.033867in}}%
\pgfpathlineto{\pgfqpoint{3.273731in}{1.033867in}}%
\pgfpathlineto{\pgfqpoint{3.232685in}{1.033867in}}%
\pgfpathlineto{\pgfqpoint{3.191640in}{1.033867in}}%
\pgfpathlineto{\pgfqpoint{3.150595in}{1.033867in}}%
\pgfpathlineto{\pgfqpoint{3.109549in}{1.033867in}}%
\pgfpathlineto{\pgfqpoint{3.068504in}{1.033867in}}%
\pgfpathlineto{\pgfqpoint{3.027459in}{1.033867in}}%
\pgfpathlineto{\pgfqpoint{2.986413in}{1.033867in}}%
\pgfpathlineto{\pgfqpoint{2.945368in}{1.033867in}}%
\pgfpathlineto{\pgfqpoint{2.904323in}{1.033867in}}%
\pgfpathlineto{\pgfqpoint{2.863277in}{1.033867in}}%
\pgfpathlineto{\pgfqpoint{2.822232in}{1.033867in}}%
\pgfpathlineto{\pgfqpoint{2.781187in}{1.033867in}}%
\pgfpathlineto{\pgfqpoint{2.740141in}{1.033867in}}%
\pgfpathlineto{\pgfqpoint{2.699096in}{1.033867in}}%
\pgfpathlineto{\pgfqpoint{2.658051in}{1.033867in}}%
\pgfpathlineto{\pgfqpoint{2.617005in}{1.033867in}}%
\pgfpathlineto{\pgfqpoint{2.575960in}{1.033867in}}%
\pgfpathlineto{\pgfqpoint{2.534915in}{1.033867in}}%
\pgfpathclose%
\pgfusepath{stroke,fill}%
\end{pgfscope}%
\begin{pgfscope}%
\pgfpathrectangle{\pgfqpoint{0.605784in}{0.382904in}}{\pgfqpoint{4.063488in}{2.042155in}}%
\pgfusepath{clip}%
\pgfsetbuttcap%
\pgfsetroundjoin%
\definecolor{currentfill}{rgb}{0.263663,0.237631,0.518762}%
\pgfsetfillcolor{currentfill}%
\pgfsetlinewidth{1.003750pt}%
\definecolor{currentstroke}{rgb}{0.263663,0.237631,0.518762}%
\pgfsetstrokecolor{currentstroke}%
\pgfsetdash{}{0pt}%
\pgfsys@defobject{currentmarker}{\pgfqpoint{0.605784in}{0.687855in}}{\pgfqpoint{4.669272in}{2.425059in}}{%
\pgfpathmoveto{\pgfqpoint{0.625696in}{0.754205in}}%
\pgfpathlineto{\pgfqpoint{0.605784in}{0.761947in}}%
\pgfpathlineto{\pgfqpoint{0.605784in}{0.774833in}}%
\pgfpathlineto{\pgfqpoint{0.605784in}{0.795460in}}%
\pgfpathlineto{\pgfqpoint{0.605784in}{0.816088in}}%
\pgfpathlineto{\pgfqpoint{0.605784in}{0.836716in}}%
\pgfpathlineto{\pgfqpoint{0.605784in}{0.857344in}}%
\pgfpathlineto{\pgfqpoint{0.605784in}{0.877972in}}%
\pgfpathlineto{\pgfqpoint{0.605784in}{0.879256in}}%
\pgfpathlineto{\pgfqpoint{0.609374in}{0.877972in}}%
\pgfpathlineto{\pgfqpoint{0.646829in}{0.864588in}}%
\pgfpathlineto{\pgfqpoint{0.687875in}{0.862243in}}%
\pgfpathlineto{\pgfqpoint{0.728920in}{0.871256in}}%
\pgfpathlineto{\pgfqpoint{0.746179in}{0.877972in}}%
\pgfpathlineto{\pgfqpoint{0.769965in}{0.887849in}}%
\pgfpathlineto{\pgfqpoint{0.793531in}{0.898600in}}%
\pgfpathlineto{\pgfqpoint{0.811011in}{0.906903in}}%
\pgfpathlineto{\pgfqpoint{0.839916in}{0.919227in}}%
\pgfpathlineto{\pgfqpoint{0.852056in}{0.924459in}}%
\pgfpathlineto{\pgfqpoint{0.892450in}{0.939855in}}%
\pgfpathlineto{\pgfqpoint{0.893101in}{0.940099in}}%
\pgfpathlineto{\pgfqpoint{0.934147in}{0.957160in}}%
\pgfpathlineto{\pgfqpoint{0.940014in}{0.960483in}}%
\pgfpathlineto{\pgfqpoint{0.975192in}{0.979300in}}%
\pgfpathlineto{\pgfqpoint{0.977570in}{0.981111in}}%
\pgfpathlineto{\pgfqpoint{1.005628in}{1.001739in}}%
\pgfpathlineto{\pgfqpoint{1.016237in}{1.009085in}}%
\pgfpathlineto{\pgfqpoint{1.030455in}{1.022367in}}%
\pgfpathlineto{\pgfqpoint{1.053811in}{1.042994in}}%
\pgfpathlineto{\pgfqpoint{1.057283in}{1.045980in}}%
\pgfpathlineto{\pgfqpoint{1.074269in}{1.063622in}}%
\pgfpathlineto{\pgfqpoint{1.095190in}{1.084250in}}%
\pgfpathlineto{\pgfqpoint{1.098328in}{1.087297in}}%
\pgfpathlineto{\pgfqpoint{1.115553in}{1.104878in}}%
\pgfpathlineto{\pgfqpoint{1.136627in}{1.125506in}}%
\pgfpathlineto{\pgfqpoint{1.139373in}{1.128173in}}%
\pgfpathlineto{\pgfqpoint{1.160182in}{1.146134in}}%
\pgfpathlineto{\pgfqpoint{1.180419in}{1.163066in}}%
\pgfpathlineto{\pgfqpoint{1.186694in}{1.166761in}}%
\pgfpathlineto{\pgfqpoint{1.221464in}{1.187056in}}%
\pgfpathlineto{\pgfqpoint{1.222737in}{1.187389in}}%
\pgfpathlineto{\pgfqpoint{1.262509in}{1.197964in}}%
\pgfpathlineto{\pgfqpoint{1.303555in}{1.196368in}}%
\pgfpathlineto{\pgfqpoint{1.339958in}{1.187389in}}%
\pgfpathlineto{\pgfqpoint{1.344600in}{1.186284in}}%
\pgfpathlineto{\pgfqpoint{1.385645in}{1.173883in}}%
\pgfpathlineto{\pgfqpoint{1.419006in}{1.166761in}}%
\pgfpathlineto{\pgfqpoint{1.426691in}{1.165130in}}%
\pgfpathlineto{\pgfqpoint{1.467736in}{1.165685in}}%
\pgfpathlineto{\pgfqpoint{1.471335in}{1.166761in}}%
\pgfpathlineto{\pgfqpoint{1.508781in}{1.179169in}}%
\pgfpathlineto{\pgfqpoint{1.522228in}{1.187389in}}%
\pgfpathlineto{\pgfqpoint{1.549827in}{1.206080in}}%
\pgfpathlineto{\pgfqpoint{1.552193in}{1.208017in}}%
\pgfpathlineto{\pgfqpoint{1.575428in}{1.228645in}}%
\pgfpathlineto{\pgfqpoint{1.590872in}{1.243755in}}%
\pgfpathlineto{\pgfqpoint{1.597098in}{1.249273in}}%
\pgfpathlineto{\pgfqpoint{1.618674in}{1.269900in}}%
\pgfpathlineto{\pgfqpoint{1.631917in}{1.283651in}}%
\pgfpathlineto{\pgfqpoint{1.642167in}{1.290528in}}%
\pgfpathlineto{\pgfqpoint{1.671459in}{1.311156in}}%
\pgfpathlineto{\pgfqpoint{1.672963in}{1.312282in}}%
\pgfpathlineto{\pgfqpoint{1.714008in}{1.320561in}}%
\pgfpathlineto{\pgfqpoint{1.755053in}{1.313040in}}%
\pgfpathlineto{\pgfqpoint{1.761656in}{1.311156in}}%
\pgfpathlineto{\pgfqpoint{1.796099in}{1.303121in}}%
\pgfpathlineto{\pgfqpoint{1.837144in}{1.299516in}}%
\pgfpathlineto{\pgfqpoint{1.878189in}{1.305823in}}%
\pgfpathlineto{\pgfqpoint{1.891451in}{1.311156in}}%
\pgfpathlineto{\pgfqpoint{1.919235in}{1.321639in}}%
\pgfpathlineto{\pgfqpoint{1.937451in}{1.331784in}}%
\pgfpathlineto{\pgfqpoint{1.960280in}{1.343831in}}%
\pgfpathlineto{\pgfqpoint{1.974453in}{1.352412in}}%
\pgfpathlineto{\pgfqpoint{2.001325in}{1.368050in}}%
\pgfpathlineto{\pgfqpoint{2.010549in}{1.373040in}}%
\pgfpathlineto{\pgfqpoint{2.042371in}{1.389878in}}%
\pgfpathlineto{\pgfqpoint{2.051985in}{1.393667in}}%
\pgfpathlineto{\pgfqpoint{2.083416in}{1.406025in}}%
\pgfpathlineto{\pgfqpoint{2.120897in}{1.414295in}}%
\pgfpathlineto{\pgfqpoint{2.124461in}{1.415096in}}%
\pgfpathlineto{\pgfqpoint{2.165507in}{1.418702in}}%
\pgfpathlineto{\pgfqpoint{2.206552in}{1.420299in}}%
\pgfpathlineto{\pgfqpoint{2.247597in}{1.425076in}}%
\pgfpathlineto{\pgfqpoint{2.278144in}{1.434923in}}%
\pgfpathlineto{\pgfqpoint{2.288643in}{1.438623in}}%
\pgfpathlineto{\pgfqpoint{2.314615in}{1.455551in}}%
\pgfpathlineto{\pgfqpoint{2.329688in}{1.466426in}}%
\pgfpathlineto{\pgfqpoint{2.339279in}{1.476179in}}%
\pgfpathlineto{\pgfqpoint{2.357796in}{1.496807in}}%
\pgfpathlineto{\pgfqpoint{2.370733in}{1.512779in}}%
\pgfpathlineto{\pgfqpoint{2.374045in}{1.517434in}}%
\pgfpathlineto{\pgfqpoint{2.387520in}{1.538062in}}%
\pgfpathlineto{\pgfqpoint{2.399680in}{1.558690in}}%
\pgfpathlineto{\pgfqpoint{2.410584in}{1.579318in}}%
\pgfpathlineto{\pgfqpoint{2.411779in}{1.581825in}}%
\pgfpathlineto{\pgfqpoint{2.421166in}{1.599946in}}%
\pgfpathlineto{\pgfqpoint{2.430730in}{1.620573in}}%
\pgfpathlineto{\pgfqpoint{2.439187in}{1.641201in}}%
\pgfpathlineto{\pgfqpoint{2.446563in}{1.661829in}}%
\pgfpathlineto{\pgfqpoint{2.452824in}{1.682255in}}%
\pgfpathlineto{\pgfqpoint{2.452913in}{1.682457in}}%
\pgfpathlineto{\pgfqpoint{2.460714in}{1.703085in}}%
\pgfpathlineto{\pgfqpoint{2.467187in}{1.723713in}}%
\pgfpathlineto{\pgfqpoint{2.472320in}{1.744340in}}%
\pgfpathlineto{\pgfqpoint{2.476099in}{1.764968in}}%
\pgfpathlineto{\pgfqpoint{2.478510in}{1.785596in}}%
\pgfpathlineto{\pgfqpoint{2.479540in}{1.806224in}}%
\pgfpathlineto{\pgfqpoint{2.479175in}{1.826852in}}%
\pgfpathlineto{\pgfqpoint{2.477400in}{1.847480in}}%
\pgfpathlineto{\pgfqpoint{2.474202in}{1.868107in}}%
\pgfpathlineto{\pgfqpoint{2.469566in}{1.888735in}}%
\pgfpathlineto{\pgfqpoint{2.463476in}{1.909363in}}%
\pgfpathlineto{\pgfqpoint{2.455919in}{1.929991in}}%
\pgfpathlineto{\pgfqpoint{2.452824in}{1.937070in}}%
\pgfpathlineto{\pgfqpoint{2.449280in}{1.950619in}}%
\pgfpathlineto{\pgfqpoint{2.443100in}{1.971247in}}%
\pgfpathlineto{\pgfqpoint{2.436166in}{1.991874in}}%
\pgfpathlineto{\pgfqpoint{2.428493in}{2.012502in}}%
\pgfpathlineto{\pgfqpoint{2.420096in}{2.033130in}}%
\pgfpathlineto{\pgfqpoint{2.411779in}{2.051960in}}%
\pgfpathlineto{\pgfqpoint{2.411143in}{2.053758in}}%
\pgfpathlineto{\pgfqpoint{2.403285in}{2.074386in}}%
\pgfpathlineto{\pgfqpoint{2.394949in}{2.095013in}}%
\pgfpathlineto{\pgfqpoint{2.386150in}{2.115641in}}%
\pgfpathlineto{\pgfqpoint{2.376904in}{2.136269in}}%
\pgfpathlineto{\pgfqpoint{2.370733in}{2.149368in}}%
\pgfpathlineto{\pgfqpoint{2.367354in}{2.156897in}}%
\pgfpathlineto{\pgfqpoint{2.357667in}{2.177525in}}%
\pgfpathlineto{\pgfqpoint{2.347644in}{2.198153in}}%
\pgfpathlineto{\pgfqpoint{2.337299in}{2.218780in}}%
\pgfpathlineto{\pgfqpoint{2.329688in}{2.233460in}}%
\pgfpathlineto{\pgfqpoint{2.326315in}{2.239408in}}%
\pgfpathlineto{\pgfqpoint{2.314217in}{2.260036in}}%
\pgfpathlineto{\pgfqpoint{2.301864in}{2.280664in}}%
\pgfpathlineto{\pgfqpoint{2.289266in}{2.301292in}}%
\pgfpathlineto{\pgfqpoint{2.288643in}{2.302280in}}%
\pgfpathlineto{\pgfqpoint{2.273200in}{2.321920in}}%
\pgfpathlineto{\pgfqpoint{2.256766in}{2.342547in}}%
\pgfpathlineto{\pgfqpoint{2.247597in}{2.353838in}}%
\pgfpathlineto{\pgfqpoint{2.236474in}{2.363175in}}%
\pgfpathlineto{\pgfqpoint{2.211486in}{2.383803in}}%
\pgfpathlineto{\pgfqpoint{2.206552in}{2.387793in}}%
\pgfpathlineto{\pgfqpoint{2.165754in}{2.404431in}}%
\pgfpathlineto{\pgfqpoint{2.165507in}{2.404529in}}%
\pgfpathlineto{\pgfqpoint{2.124461in}{2.405207in}}%
\pgfpathlineto{\pgfqpoint{2.122143in}{2.404431in}}%
\pgfpathlineto{\pgfqpoint{2.083416in}{2.391229in}}%
\pgfpathlineto{\pgfqpoint{2.072334in}{2.383803in}}%
\pgfpathlineto{\pgfqpoint{2.042371in}{2.363773in}}%
\pgfpathlineto{\pgfqpoint{2.041772in}{2.363175in}}%
\pgfpathlineto{\pgfqpoint{2.021309in}{2.342547in}}%
\pgfpathlineto{\pgfqpoint{2.001325in}{2.322830in}}%
\pgfpathlineto{\pgfqpoint{2.000642in}{2.321920in}}%
\pgfpathlineto{\pgfqpoint{1.985325in}{2.301292in}}%
\pgfpathlineto{\pgfqpoint{1.969712in}{2.280664in}}%
\pgfpathlineto{\pgfqpoint{1.960280in}{2.268293in}}%
\pgfpathlineto{\pgfqpoint{1.955315in}{2.260036in}}%
\pgfpathlineto{\pgfqpoint{1.942957in}{2.239408in}}%
\pgfpathlineto{\pgfqpoint{1.930416in}{2.218780in}}%
\pgfpathlineto{\pgfqpoint{1.919235in}{2.200610in}}%
\pgfpathlineto{\pgfqpoint{1.917965in}{2.198153in}}%
\pgfpathlineto{\pgfqpoint{1.907474in}{2.177525in}}%
\pgfpathlineto{\pgfqpoint{1.896889in}{2.156897in}}%
\pgfpathlineto{\pgfqpoint{1.886199in}{2.136269in}}%
\pgfpathlineto{\pgfqpoint{1.878189in}{2.120855in}}%
\pgfpathlineto{\pgfqpoint{1.875744in}{2.115641in}}%
\pgfpathlineto{\pgfqpoint{1.866228in}{2.095013in}}%
\pgfpathlineto{\pgfqpoint{1.856679in}{2.074386in}}%
\pgfpathlineto{\pgfqpoint{1.847093in}{2.053758in}}%
\pgfpathlineto{\pgfqpoint{1.837467in}{2.033130in}}%
\pgfpathlineto{\pgfqpoint{1.837144in}{2.032419in}}%
\pgfpathlineto{\pgfqpoint{1.828183in}{2.012502in}}%
\pgfpathlineto{\pgfqpoint{1.818930in}{1.991874in}}%
\pgfpathlineto{\pgfqpoint{1.809699in}{1.971247in}}%
\pgfpathlineto{\pgfqpoint{1.800490in}{1.950619in}}%
\pgfpathlineto{\pgfqpoint{1.796099in}{1.940594in}}%
\pgfpathlineto{\pgfqpoint{1.790874in}{1.929991in}}%
\pgfpathlineto{\pgfqpoint{1.780920in}{1.909363in}}%
\pgfpathlineto{\pgfqpoint{1.771043in}{1.888735in}}%
\pgfpathlineto{\pgfqpoint{1.761249in}{1.868107in}}%
\pgfpathlineto{\pgfqpoint{1.755053in}{1.854777in}}%
\pgfpathlineto{\pgfqpoint{1.750378in}{1.847480in}}%
\pgfpathlineto{\pgfqpoint{1.737550in}{1.826852in}}%
\pgfpathlineto{\pgfqpoint{1.724834in}{1.806224in}}%
\pgfpathlineto{\pgfqpoint{1.714008in}{1.788395in}}%
\pgfpathlineto{\pgfqpoint{1.710271in}{1.785596in}}%
\pgfpathlineto{\pgfqpoint{1.683402in}{1.764968in}}%
\pgfpathlineto{\pgfqpoint{1.672963in}{1.756857in}}%
\pgfpathlineto{\pgfqpoint{1.640801in}{1.764968in}}%
\pgfpathlineto{\pgfqpoint{1.631917in}{1.766939in}}%
\pgfpathlineto{\pgfqpoint{1.614036in}{1.785596in}}%
\pgfpathlineto{\pgfqpoint{1.592674in}{1.806224in}}%
\pgfpathlineto{\pgfqpoint{1.590872in}{1.807843in}}%
\pgfpathlineto{\pgfqpoint{1.577522in}{1.826852in}}%
\pgfpathlineto{\pgfqpoint{1.562326in}{1.847480in}}%
\pgfpathlineto{\pgfqpoint{1.549827in}{1.863687in}}%
\pgfpathlineto{\pgfqpoint{1.546907in}{1.868107in}}%
\pgfpathlineto{\pgfqpoint{1.532691in}{1.888735in}}%
\pgfpathlineto{\pgfqpoint{1.518077in}{1.909363in}}%
\pgfpathlineto{\pgfqpoint{1.508781in}{1.922075in}}%
\pgfpathlineto{\pgfqpoint{1.502793in}{1.929991in}}%
\pgfpathlineto{\pgfqpoint{1.486760in}{1.950619in}}%
\pgfpathlineto{\pgfqpoint{1.470510in}{1.971247in}}%
\pgfpathlineto{\pgfqpoint{1.467736in}{1.974669in}}%
\pgfpathlineto{\pgfqpoint{1.450751in}{1.991874in}}%
\pgfpathlineto{\pgfqpoint{1.430287in}{2.012502in}}%
\pgfpathlineto{\pgfqpoint{1.426691in}{2.016055in}}%
\pgfpathlineto{\pgfqpoint{1.401286in}{2.033130in}}%
\pgfpathlineto{\pgfqpoint{1.385645in}{2.043582in}}%
\pgfpathlineto{\pgfqpoint{1.354726in}{2.053758in}}%
\pgfpathlineto{\pgfqpoint{1.344600in}{2.057033in}}%
\pgfpathlineto{\pgfqpoint{1.303555in}{2.057676in}}%
\pgfpathlineto{\pgfqpoint{1.286575in}{2.053758in}}%
\pgfpathlineto{\pgfqpoint{1.262509in}{2.047985in}}%
\pgfpathlineto{\pgfqpoint{1.228032in}{2.033130in}}%
\pgfpathlineto{\pgfqpoint{1.221464in}{2.030257in}}%
\pgfpathlineto{\pgfqpoint{1.192331in}{2.012502in}}%
\pgfpathlineto{\pgfqpoint{1.180419in}{2.005309in}}%
\pgfpathlineto{\pgfqpoint{1.163722in}{1.991874in}}%
\pgfpathlineto{\pgfqpoint{1.139373in}{1.972877in}}%
\pgfpathlineto{\pgfqpoint{1.137790in}{1.971247in}}%
\pgfpathlineto{\pgfqpoint{1.117713in}{1.950619in}}%
\pgfpathlineto{\pgfqpoint{1.098328in}{1.931575in}}%
\pgfpathlineto{\pgfqpoint{1.097076in}{1.929991in}}%
\pgfpathlineto{\pgfqpoint{1.080807in}{1.909363in}}%
\pgfpathlineto{\pgfqpoint{1.063883in}{1.888735in}}%
\pgfpathlineto{\pgfqpoint{1.057283in}{1.880801in}}%
\pgfpathlineto{\pgfqpoint{1.048670in}{1.868107in}}%
\pgfpathlineto{\pgfqpoint{1.034470in}{1.847480in}}%
\pgfpathlineto{\pgfqpoint{1.019810in}{1.826852in}}%
\pgfpathlineto{\pgfqpoint{1.016237in}{1.821833in}}%
\pgfpathlineto{\pgfqpoint{1.006431in}{1.806224in}}%
\pgfpathlineto{\pgfqpoint{0.993293in}{1.785596in}}%
\pgfpathlineto{\pgfqpoint{0.979828in}{1.764968in}}%
\pgfpathlineto{\pgfqpoint{0.975192in}{1.757866in}}%
\pgfpathlineto{\pgfqpoint{0.966518in}{1.744340in}}%
\pgfpathlineto{\pgfqpoint{0.953213in}{1.723713in}}%
\pgfpathlineto{\pgfqpoint{0.939648in}{1.703085in}}%
\pgfpathlineto{\pgfqpoint{0.934147in}{1.694700in}}%
\pgfpathlineto{\pgfqpoint{0.925028in}{1.682457in}}%
\pgfpathlineto{\pgfqpoint{0.909630in}{1.661829in}}%
\pgfpathlineto{\pgfqpoint{0.893949in}{1.641201in}}%
\pgfpathlineto{\pgfqpoint{0.893101in}{1.640066in}}%
\pgfpathlineto{\pgfqpoint{0.872444in}{1.620573in}}%
\pgfpathlineto{\pgfqpoint{0.852056in}{1.601692in}}%
\pgfpathlineto{\pgfqpoint{0.847684in}{1.599946in}}%
\pgfpathlineto{\pgfqpoint{0.811011in}{1.585142in}}%
\pgfpathlineto{\pgfqpoint{0.769965in}{1.590758in}}%
\pgfpathlineto{\pgfqpoint{0.753713in}{1.599946in}}%
\pgfpathlineto{\pgfqpoint{0.728920in}{1.613290in}}%
\pgfpathlineto{\pgfqpoint{0.719610in}{1.620573in}}%
\pgfpathlineto{\pgfqpoint{0.692473in}{1.641201in}}%
\pgfpathlineto{\pgfqpoint{0.687875in}{1.644591in}}%
\pgfpathlineto{\pgfqpoint{0.665631in}{1.661829in}}%
\pgfpathlineto{\pgfqpoint{0.646829in}{1.676269in}}%
\pgfpathlineto{\pgfqpoint{0.636891in}{1.682457in}}%
\pgfpathlineto{\pgfqpoint{0.605784in}{1.701800in}}%
\pgfpathlineto{\pgfqpoint{0.605784in}{1.703085in}}%
\pgfpathlineto{\pgfqpoint{0.605784in}{1.723713in}}%
\pgfpathlineto{\pgfqpoint{0.605784in}{1.744340in}}%
\pgfpathlineto{\pgfqpoint{0.605784in}{1.764968in}}%
\pgfpathlineto{\pgfqpoint{0.605784in}{1.785596in}}%
\pgfpathlineto{\pgfqpoint{0.605784in}{1.806224in}}%
\pgfpathlineto{\pgfqpoint{0.605784in}{1.819109in}}%
\pgfpathlineto{\pgfqpoint{0.627595in}{1.806224in}}%
\pgfpathlineto{\pgfqpoint{0.646829in}{1.794850in}}%
\pgfpathlineto{\pgfqpoint{0.660343in}{1.785596in}}%
\pgfpathlineto{\pgfqpoint{0.687875in}{1.766612in}}%
\pgfpathlineto{\pgfqpoint{0.690498in}{1.764968in}}%
\pgfpathlineto{\pgfqpoint{0.722710in}{1.744340in}}%
\pgfpathlineto{\pgfqpoint{0.728920in}{1.740281in}}%
\pgfpathlineto{\pgfqpoint{0.768276in}{1.723713in}}%
\pgfpathlineto{\pgfqpoint{0.769965in}{1.722975in}}%
\pgfpathlineto{\pgfqpoint{0.811011in}{1.720744in}}%
\pgfpathlineto{\pgfqpoint{0.818311in}{1.723713in}}%
\pgfpathlineto{\pgfqpoint{0.852056in}{1.737330in}}%
\pgfpathlineto{\pgfqpoint{0.860246in}{1.744340in}}%
\pgfpathlineto{\pgfqpoint{0.884313in}{1.764968in}}%
\pgfpathlineto{\pgfqpoint{0.893101in}{1.772517in}}%
\pgfpathlineto{\pgfqpoint{0.903994in}{1.785596in}}%
\pgfpathlineto{\pgfqpoint{0.921038in}{1.806224in}}%
\pgfpathlineto{\pgfqpoint{0.934147in}{1.822272in}}%
\pgfpathlineto{\pgfqpoint{0.937413in}{1.826852in}}%
\pgfpathlineto{\pgfqpoint{0.952185in}{1.847480in}}%
\pgfpathlineto{\pgfqpoint{0.966672in}{1.868107in}}%
\pgfpathlineto{\pgfqpoint{0.975192in}{1.880317in}}%
\pgfpathlineto{\pgfqpoint{0.981050in}{1.888735in}}%
\pgfpathlineto{\pgfqpoint{0.995339in}{1.909363in}}%
\pgfpathlineto{\pgfqpoint{1.009311in}{1.929991in}}%
\pgfpathlineto{\pgfqpoint{1.016237in}{1.940280in}}%
\pgfpathlineto{\pgfqpoint{1.023901in}{1.950619in}}%
\pgfpathlineto{\pgfqpoint{1.039017in}{1.971247in}}%
\pgfpathlineto{\pgfqpoint{1.053731in}{1.991874in}}%
\pgfpathlineto{\pgfqpoint{1.057283in}{1.996859in}}%
\pgfpathlineto{\pgfqpoint{1.070502in}{2.012502in}}%
\pgfpathlineto{\pgfqpoint{1.087505in}{2.033130in}}%
\pgfpathlineto{\pgfqpoint{1.098328in}{2.046520in}}%
\pgfpathlineto{\pgfqpoint{1.105669in}{2.053758in}}%
\pgfpathlineto{\pgfqpoint{1.126300in}{2.074386in}}%
\pgfpathlineto{\pgfqpoint{1.139373in}{2.087778in}}%
\pgfpathlineto{\pgfqpoint{1.148554in}{2.095013in}}%
\pgfpathlineto{\pgfqpoint{1.174334in}{2.115641in}}%
\pgfpathlineto{\pgfqpoint{1.180419in}{2.120560in}}%
\pgfpathlineto{\pgfqpoint{1.206509in}{2.136269in}}%
\pgfpathlineto{\pgfqpoint{1.221464in}{2.145338in}}%
\pgfpathlineto{\pgfqpoint{1.249193in}{2.156897in}}%
\pgfpathlineto{\pgfqpoint{1.262509in}{2.162383in}}%
\pgfpathlineto{\pgfqpoint{1.303555in}{2.170859in}}%
\pgfpathlineto{\pgfqpoint{1.344600in}{2.169233in}}%
\pgfpathlineto{\pgfqpoint{1.382926in}{2.156897in}}%
\pgfpathlineto{\pgfqpoint{1.385645in}{2.156009in}}%
\pgfpathlineto{\pgfqpoint{1.417240in}{2.136269in}}%
\pgfpathlineto{\pgfqpoint{1.426691in}{2.130337in}}%
\pgfpathlineto{\pgfqpoint{1.443141in}{2.115641in}}%
\pgfpathlineto{\pgfqpoint{1.466190in}{2.095013in}}%
\pgfpathlineto{\pgfqpoint{1.467736in}{2.093607in}}%
\pgfpathlineto{\pgfqpoint{1.485500in}{2.074386in}}%
\pgfpathlineto{\pgfqpoint{1.504448in}{2.053758in}}%
\pgfpathlineto{\pgfqpoint{1.508781in}{2.048949in}}%
\pgfpathlineto{\pgfqpoint{1.522902in}{2.033130in}}%
\pgfpathlineto{\pgfqpoint{1.541006in}{2.012502in}}%
\pgfpathlineto{\pgfqpoint{1.549827in}{2.002229in}}%
\pgfpathlineto{\pgfqpoint{1.560558in}{1.991874in}}%
\pgfpathlineto{\pgfqpoint{1.581314in}{1.971247in}}%
\pgfpathlineto{\pgfqpoint{1.590872in}{1.961444in}}%
\pgfpathlineto{\pgfqpoint{1.608787in}{1.950619in}}%
\pgfpathlineto{\pgfqpoint{1.631917in}{1.935964in}}%
\pgfpathlineto{\pgfqpoint{1.672963in}{1.934322in}}%
\pgfpathlineto{\pgfqpoint{1.698357in}{1.950619in}}%
\pgfpathlineto{\pgfqpoint{1.714008in}{1.960641in}}%
\pgfpathlineto{\pgfqpoint{1.722355in}{1.971247in}}%
\pgfpathlineto{\pgfqpoint{1.738704in}{1.991874in}}%
\pgfpathlineto{\pgfqpoint{1.754943in}{2.012502in}}%
\pgfpathlineto{\pgfqpoint{1.755053in}{2.012640in}}%
\pgfpathlineto{\pgfqpoint{1.767181in}{2.033130in}}%
\pgfpathlineto{\pgfqpoint{1.779340in}{2.053758in}}%
\pgfpathlineto{\pgfqpoint{1.791450in}{2.074386in}}%
\pgfpathlineto{\pgfqpoint{1.796099in}{2.082214in}}%
\pgfpathlineto{\pgfqpoint{1.802816in}{2.095013in}}%
\pgfpathlineto{\pgfqpoint{1.813684in}{2.115641in}}%
\pgfpathlineto{\pgfqpoint{1.824492in}{2.136269in}}%
\pgfpathlineto{\pgfqpoint{1.835244in}{2.156897in}}%
\pgfpathlineto{\pgfqpoint{1.837144in}{2.160490in}}%
\pgfpathlineto{\pgfqpoint{1.846121in}{2.177525in}}%
\pgfpathlineto{\pgfqpoint{1.856943in}{2.198153in}}%
\pgfpathlineto{\pgfqpoint{1.867675in}{2.218780in}}%
\pgfpathlineto{\pgfqpoint{1.878189in}{2.239145in}}%
\pgfpathlineto{\pgfqpoint{1.878338in}{2.239408in}}%
\pgfpathlineto{\pgfqpoint{1.890129in}{2.260036in}}%
\pgfpathlineto{\pgfqpoint{1.901775in}{2.280664in}}%
\pgfpathlineto{\pgfqpoint{1.913285in}{2.301292in}}%
\pgfpathlineto{\pgfqpoint{1.919235in}{2.311948in}}%
\pgfpathlineto{\pgfqpoint{1.925770in}{2.321920in}}%
\pgfpathlineto{\pgfqpoint{1.939256in}{2.342547in}}%
\pgfpathlineto{\pgfqpoint{1.952535in}{2.363175in}}%
\pgfpathlineto{\pgfqpoint{1.960280in}{2.375258in}}%
\pgfpathlineto{\pgfqpoint{1.967130in}{2.383803in}}%
\pgfpathlineto{\pgfqpoint{1.983616in}{2.404431in}}%
\pgfpathlineto{\pgfqpoint{1.999798in}{2.425059in}}%
\pgfpathlineto{\pgfqpoint{2.001325in}{2.425059in}}%
\pgfpathlineto{\pgfqpoint{2.042371in}{2.425059in}}%
\pgfpathlineto{\pgfqpoint{2.083416in}{2.425059in}}%
\pgfpathlineto{\pgfqpoint{2.124461in}{2.425059in}}%
\pgfpathlineto{\pgfqpoint{2.165507in}{2.425059in}}%
\pgfpathlineto{\pgfqpoint{2.206552in}{2.425059in}}%
\pgfpathlineto{\pgfqpoint{2.247597in}{2.425059in}}%
\pgfpathlineto{\pgfqpoint{2.279797in}{2.425059in}}%
\pgfpathlineto{\pgfqpoint{2.288643in}{2.415163in}}%
\pgfpathlineto{\pgfqpoint{2.296422in}{2.404431in}}%
\pgfpathlineto{\pgfqpoint{2.311116in}{2.383803in}}%
\pgfpathlineto{\pgfqpoint{2.325642in}{2.363175in}}%
\pgfpathlineto{\pgfqpoint{2.329688in}{2.357300in}}%
\pgfpathlineto{\pgfqpoint{2.339102in}{2.342547in}}%
\pgfpathlineto{\pgfqpoint{2.351993in}{2.321920in}}%
\pgfpathlineto{\pgfqpoint{2.364648in}{2.301292in}}%
\pgfpathlineto{\pgfqpoint{2.370733in}{2.291114in}}%
\pgfpathlineto{\pgfqpoint{2.377410in}{2.280664in}}%
\pgfpathlineto{\pgfqpoint{2.390193in}{2.260036in}}%
\pgfpathlineto{\pgfqpoint{2.402631in}{2.239408in}}%
\pgfpathlineto{\pgfqpoint{2.411779in}{2.223754in}}%
\pgfpathlineto{\pgfqpoint{2.415542in}{2.218780in}}%
\pgfpathlineto{\pgfqpoint{2.430491in}{2.198153in}}%
\pgfpathlineto{\pgfqpoint{2.444846in}{2.177525in}}%
\pgfpathlineto{\pgfqpoint{2.452824in}{2.165521in}}%
\pgfpathlineto{\pgfqpoint{2.463557in}{2.156897in}}%
\pgfpathlineto{\pgfqpoint{2.487625in}{2.136269in}}%
\pgfpathlineto{\pgfqpoint{2.493869in}{2.130540in}}%
\pgfpathlineto{\pgfqpoint{2.495249in}{2.115641in}}%
\pgfpathlineto{\pgfqpoint{2.497239in}{2.095013in}}%
\pgfpathlineto{\pgfqpoint{2.499364in}{2.074386in}}%
\pgfpathlineto{\pgfqpoint{2.501653in}{2.053758in}}%
\pgfpathlineto{\pgfqpoint{2.504146in}{2.033130in}}%
\pgfpathlineto{\pgfqpoint{2.506897in}{2.012502in}}%
\pgfpathlineto{\pgfqpoint{2.509982in}{1.991874in}}%
\pgfpathlineto{\pgfqpoint{2.513506in}{1.971247in}}%
\pgfpathlineto{\pgfqpoint{2.517628in}{1.950619in}}%
\pgfpathlineto{\pgfqpoint{2.522591in}{1.929991in}}%
\pgfpathlineto{\pgfqpoint{2.528792in}{1.909363in}}%
\pgfpathlineto{\pgfqpoint{2.534915in}{1.893201in}}%
\pgfpathlineto{\pgfqpoint{2.575960in}{1.893201in}}%
\pgfpathlineto{\pgfqpoint{2.617005in}{1.893201in}}%
\pgfpathlineto{\pgfqpoint{2.658051in}{1.893201in}}%
\pgfpathlineto{\pgfqpoint{2.699096in}{1.893201in}}%
\pgfpathlineto{\pgfqpoint{2.740141in}{1.893201in}}%
\pgfpathlineto{\pgfqpoint{2.781187in}{1.893201in}}%
\pgfpathlineto{\pgfqpoint{2.822232in}{1.893201in}}%
\pgfpathlineto{\pgfqpoint{2.863277in}{1.893201in}}%
\pgfpathlineto{\pgfqpoint{2.904323in}{1.893201in}}%
\pgfpathlineto{\pgfqpoint{2.945368in}{1.893201in}}%
\pgfpathlineto{\pgfqpoint{2.986413in}{1.893201in}}%
\pgfpathlineto{\pgfqpoint{3.027459in}{1.893201in}}%
\pgfpathlineto{\pgfqpoint{3.068504in}{1.893201in}}%
\pgfpathlineto{\pgfqpoint{3.109549in}{1.893201in}}%
\pgfpathlineto{\pgfqpoint{3.150595in}{1.893201in}}%
\pgfpathlineto{\pgfqpoint{3.191640in}{1.893201in}}%
\pgfpathlineto{\pgfqpoint{3.232685in}{1.893201in}}%
\pgfpathlineto{\pgfqpoint{3.273731in}{1.893201in}}%
\pgfpathlineto{\pgfqpoint{3.314776in}{1.893201in}}%
\pgfpathlineto{\pgfqpoint{3.355821in}{1.893201in}}%
\pgfpathlineto{\pgfqpoint{3.396867in}{1.893201in}}%
\pgfpathlineto{\pgfqpoint{3.437912in}{1.893201in}}%
\pgfpathlineto{\pgfqpoint{3.478957in}{1.893201in}}%
\pgfpathlineto{\pgfqpoint{3.520003in}{1.893201in}}%
\pgfpathlineto{\pgfqpoint{3.561048in}{1.893201in}}%
\pgfpathlineto{\pgfqpoint{3.602093in}{1.893201in}}%
\pgfpathlineto{\pgfqpoint{3.643139in}{1.893201in}}%
\pgfpathlineto{\pgfqpoint{3.643842in}{1.888735in}}%
\pgfpathlineto{\pgfqpoint{3.647095in}{1.868107in}}%
\pgfpathlineto{\pgfqpoint{3.650445in}{1.847480in}}%
\pgfpathlineto{\pgfqpoint{3.653903in}{1.826852in}}%
\pgfpathlineto{\pgfqpoint{3.657481in}{1.806224in}}%
\pgfpathlineto{\pgfqpoint{3.661196in}{1.785596in}}%
\pgfpathlineto{\pgfqpoint{3.665065in}{1.764968in}}%
\pgfpathlineto{\pgfqpoint{3.669108in}{1.744340in}}%
\pgfpathlineto{\pgfqpoint{3.673352in}{1.723713in}}%
\pgfpathlineto{\pgfqpoint{3.677825in}{1.703085in}}%
\pgfpathlineto{\pgfqpoint{3.682564in}{1.682457in}}%
\pgfpathlineto{\pgfqpoint{3.684184in}{1.675593in}}%
\pgfpathlineto{\pgfqpoint{3.725229in}{1.675593in}}%
\pgfpathlineto{\pgfqpoint{3.766275in}{1.675593in}}%
\pgfpathlineto{\pgfqpoint{3.807320in}{1.675593in}}%
\pgfpathlineto{\pgfqpoint{3.848365in}{1.675593in}}%
\pgfpathlineto{\pgfqpoint{3.889411in}{1.675593in}}%
\pgfpathlineto{\pgfqpoint{3.930456in}{1.675593in}}%
\pgfpathlineto{\pgfqpoint{3.971501in}{1.675593in}}%
\pgfpathlineto{\pgfqpoint{4.012547in}{1.675593in}}%
\pgfpathlineto{\pgfqpoint{4.053592in}{1.675593in}}%
\pgfpathlineto{\pgfqpoint{4.094637in}{1.675593in}}%
\pgfpathlineto{\pgfqpoint{4.135683in}{1.675593in}}%
\pgfpathlineto{\pgfqpoint{4.176728in}{1.675593in}}%
\pgfpathlineto{\pgfqpoint{4.217773in}{1.675593in}}%
\pgfpathlineto{\pgfqpoint{4.258819in}{1.675593in}}%
\pgfpathlineto{\pgfqpoint{4.299864in}{1.675593in}}%
\pgfpathlineto{\pgfqpoint{4.340909in}{1.675593in}}%
\pgfpathlineto{\pgfqpoint{4.381955in}{1.675593in}}%
\pgfpathlineto{\pgfqpoint{4.423000in}{1.675593in}}%
\pgfpathlineto{\pgfqpoint{4.464045in}{1.675593in}}%
\pgfpathlineto{\pgfqpoint{4.505091in}{1.675593in}}%
\pgfpathlineto{\pgfqpoint{4.546136in}{1.675593in}}%
\pgfpathlineto{\pgfqpoint{4.587181in}{1.675593in}}%
\pgfpathlineto{\pgfqpoint{4.628227in}{1.675593in}}%
\pgfpathlineto{\pgfqpoint{4.669272in}{1.675593in}}%
\pgfpathlineto{\pgfqpoint{4.669272in}{1.661829in}}%
\pgfpathlineto{\pgfqpoint{4.669272in}{1.641201in}}%
\pgfpathlineto{\pgfqpoint{4.669272in}{1.620573in}}%
\pgfpathlineto{\pgfqpoint{4.669272in}{1.599946in}}%
\pgfpathlineto{\pgfqpoint{4.669272in}{1.579318in}}%
\pgfpathlineto{\pgfqpoint{4.669272in}{1.558690in}}%
\pgfpathlineto{\pgfqpoint{4.669272in}{1.547190in}}%
\pgfpathlineto{\pgfqpoint{4.628227in}{1.547190in}}%
\pgfpathlineto{\pgfqpoint{4.587181in}{1.547190in}}%
\pgfpathlineto{\pgfqpoint{4.546136in}{1.547190in}}%
\pgfpathlineto{\pgfqpoint{4.505091in}{1.547190in}}%
\pgfpathlineto{\pgfqpoint{4.464045in}{1.547190in}}%
\pgfpathlineto{\pgfqpoint{4.423000in}{1.547190in}}%
\pgfpathlineto{\pgfqpoint{4.381955in}{1.547190in}}%
\pgfpathlineto{\pgfqpoint{4.340909in}{1.547190in}}%
\pgfpathlineto{\pgfqpoint{4.299864in}{1.547190in}}%
\pgfpathlineto{\pgfqpoint{4.258819in}{1.547190in}}%
\pgfpathlineto{\pgfqpoint{4.217773in}{1.547190in}}%
\pgfpathlineto{\pgfqpoint{4.176728in}{1.547190in}}%
\pgfpathlineto{\pgfqpoint{4.135683in}{1.547190in}}%
\pgfpathlineto{\pgfqpoint{4.094637in}{1.547190in}}%
\pgfpathlineto{\pgfqpoint{4.053592in}{1.547190in}}%
\pgfpathlineto{\pgfqpoint{4.012547in}{1.547190in}}%
\pgfpathlineto{\pgfqpoint{3.971501in}{1.547190in}}%
\pgfpathlineto{\pgfqpoint{3.930456in}{1.547190in}}%
\pgfpathlineto{\pgfqpoint{3.889411in}{1.547190in}}%
\pgfpathlineto{\pgfqpoint{3.848365in}{1.547190in}}%
\pgfpathlineto{\pgfqpoint{3.807320in}{1.547190in}}%
\pgfpathlineto{\pgfqpoint{3.766275in}{1.547190in}}%
\pgfpathlineto{\pgfqpoint{3.725229in}{1.547190in}}%
\pgfpathlineto{\pgfqpoint{3.684184in}{1.547190in}}%
\pgfpathlineto{\pgfqpoint{3.681218in}{1.558690in}}%
\pgfpathlineto{\pgfqpoint{3.676211in}{1.579318in}}%
\pgfpathlineto{\pgfqpoint{3.671613in}{1.599946in}}%
\pgfpathlineto{\pgfqpoint{3.667347in}{1.620573in}}%
\pgfpathlineto{\pgfqpoint{3.663353in}{1.641201in}}%
\pgfpathlineto{\pgfqpoint{3.659587in}{1.661829in}}%
\pgfpathlineto{\pgfqpoint{3.656013in}{1.682457in}}%
\pgfpathlineto{\pgfqpoint{3.652601in}{1.703085in}}%
\pgfpathlineto{\pgfqpoint{3.649329in}{1.723713in}}%
\pgfpathlineto{\pgfqpoint{3.646178in}{1.744340in}}%
\pgfpathlineto{\pgfqpoint{3.643139in}{1.764918in}}%
\pgfpathlineto{\pgfqpoint{3.602093in}{1.764918in}}%
\pgfpathlineto{\pgfqpoint{3.561048in}{1.764918in}}%
\pgfpathlineto{\pgfqpoint{3.520003in}{1.764918in}}%
\pgfpathlineto{\pgfqpoint{3.478957in}{1.764918in}}%
\pgfpathlineto{\pgfqpoint{3.437912in}{1.764918in}}%
\pgfpathlineto{\pgfqpoint{3.396867in}{1.764918in}}%
\pgfpathlineto{\pgfqpoint{3.355821in}{1.764918in}}%
\pgfpathlineto{\pgfqpoint{3.314776in}{1.764918in}}%
\pgfpathlineto{\pgfqpoint{3.273731in}{1.764918in}}%
\pgfpathlineto{\pgfqpoint{3.232685in}{1.764918in}}%
\pgfpathlineto{\pgfqpoint{3.191640in}{1.764918in}}%
\pgfpathlineto{\pgfqpoint{3.150595in}{1.764918in}}%
\pgfpathlineto{\pgfqpoint{3.109549in}{1.764918in}}%
\pgfpathlineto{\pgfqpoint{3.068504in}{1.764918in}}%
\pgfpathlineto{\pgfqpoint{3.027459in}{1.764918in}}%
\pgfpathlineto{\pgfqpoint{2.986413in}{1.764918in}}%
\pgfpathlineto{\pgfqpoint{2.945368in}{1.764918in}}%
\pgfpathlineto{\pgfqpoint{2.904323in}{1.764918in}}%
\pgfpathlineto{\pgfqpoint{2.863277in}{1.764918in}}%
\pgfpathlineto{\pgfqpoint{2.822232in}{1.764918in}}%
\pgfpathlineto{\pgfqpoint{2.781187in}{1.764918in}}%
\pgfpathlineto{\pgfqpoint{2.740141in}{1.764918in}}%
\pgfpathlineto{\pgfqpoint{2.699096in}{1.764918in}}%
\pgfpathlineto{\pgfqpoint{2.658051in}{1.764918in}}%
\pgfpathlineto{\pgfqpoint{2.617005in}{1.764918in}}%
\pgfpathlineto{\pgfqpoint{2.575960in}{1.764918in}}%
\pgfpathlineto{\pgfqpoint{2.534915in}{1.764918in}}%
\pgfpathlineto{\pgfqpoint{2.513384in}{1.744340in}}%
\pgfpathlineto{\pgfqpoint{2.508923in}{1.723713in}}%
\pgfpathlineto{\pgfqpoint{2.507478in}{1.703085in}}%
\pgfpathlineto{\pgfqpoint{2.507099in}{1.682457in}}%
\pgfpathlineto{\pgfqpoint{2.507218in}{1.661829in}}%
\pgfpathlineto{\pgfqpoint{2.507610in}{1.641201in}}%
\pgfpathlineto{\pgfqpoint{2.508169in}{1.620573in}}%
\pgfpathlineto{\pgfqpoint{2.508836in}{1.599946in}}%
\pgfpathlineto{\pgfqpoint{2.509577in}{1.579318in}}%
\pgfpathlineto{\pgfqpoint{2.510372in}{1.558690in}}%
\pgfpathlineto{\pgfqpoint{2.511206in}{1.538062in}}%
\pgfpathlineto{\pgfqpoint{2.512071in}{1.517434in}}%
\pgfpathlineto{\pgfqpoint{2.512960in}{1.496807in}}%
\pgfpathlineto{\pgfqpoint{2.513867in}{1.476179in}}%
\pgfpathlineto{\pgfqpoint{2.514790in}{1.455551in}}%
\pgfpathlineto{\pgfqpoint{2.515725in}{1.434923in}}%
\pgfpathlineto{\pgfqpoint{2.516670in}{1.414295in}}%
\pgfpathlineto{\pgfqpoint{2.517624in}{1.393667in}}%
\pgfpathlineto{\pgfqpoint{2.518585in}{1.373040in}}%
\pgfpathlineto{\pgfqpoint{2.519553in}{1.352412in}}%
\pgfpathlineto{\pgfqpoint{2.520526in}{1.331784in}}%
\pgfpathlineto{\pgfqpoint{2.521504in}{1.311156in}}%
\pgfpathlineto{\pgfqpoint{2.522486in}{1.290528in}}%
\pgfpathlineto{\pgfqpoint{2.523471in}{1.269900in}}%
\pgfpathlineto{\pgfqpoint{2.524460in}{1.249273in}}%
\pgfpathlineto{\pgfqpoint{2.525451in}{1.228645in}}%
\pgfpathlineto{\pgfqpoint{2.526445in}{1.208017in}}%
\pgfpathlineto{\pgfqpoint{2.527441in}{1.187389in}}%
\pgfpathlineto{\pgfqpoint{2.528439in}{1.166761in}}%
\pgfpathlineto{\pgfqpoint{2.529440in}{1.146134in}}%
\pgfpathlineto{\pgfqpoint{2.530441in}{1.125506in}}%
\pgfpathlineto{\pgfqpoint{2.531444in}{1.104878in}}%
\pgfpathlineto{\pgfqpoint{2.532449in}{1.084250in}}%
\pgfpathlineto{\pgfqpoint{2.533455in}{1.063622in}}%
\pgfpathlineto{\pgfqpoint{2.534462in}{1.042994in}}%
\pgfpathlineto{\pgfqpoint{2.534915in}{1.033867in}}%
\pgfpathlineto{\pgfqpoint{2.575960in}{1.033867in}}%
\pgfpathlineto{\pgfqpoint{2.617005in}{1.033867in}}%
\pgfpathlineto{\pgfqpoint{2.658051in}{1.033867in}}%
\pgfpathlineto{\pgfqpoint{2.699096in}{1.033867in}}%
\pgfpathlineto{\pgfqpoint{2.740141in}{1.033867in}}%
\pgfpathlineto{\pgfqpoint{2.781187in}{1.033867in}}%
\pgfpathlineto{\pgfqpoint{2.822232in}{1.033867in}}%
\pgfpathlineto{\pgfqpoint{2.863277in}{1.033867in}}%
\pgfpathlineto{\pgfqpoint{2.904323in}{1.033867in}}%
\pgfpathlineto{\pgfqpoint{2.945368in}{1.033867in}}%
\pgfpathlineto{\pgfqpoint{2.986413in}{1.033867in}}%
\pgfpathlineto{\pgfqpoint{3.027459in}{1.033867in}}%
\pgfpathlineto{\pgfqpoint{3.068504in}{1.033867in}}%
\pgfpathlineto{\pgfqpoint{3.109549in}{1.033867in}}%
\pgfpathlineto{\pgfqpoint{3.150595in}{1.033867in}}%
\pgfpathlineto{\pgfqpoint{3.191640in}{1.033867in}}%
\pgfpathlineto{\pgfqpoint{3.232685in}{1.033867in}}%
\pgfpathlineto{\pgfqpoint{3.273731in}{1.033867in}}%
\pgfpathlineto{\pgfqpoint{3.314776in}{1.033867in}}%
\pgfpathlineto{\pgfqpoint{3.355821in}{1.033867in}}%
\pgfpathlineto{\pgfqpoint{3.396867in}{1.033867in}}%
\pgfpathlineto{\pgfqpoint{3.437912in}{1.033867in}}%
\pgfpathlineto{\pgfqpoint{3.478957in}{1.033867in}}%
\pgfpathlineto{\pgfqpoint{3.520003in}{1.033867in}}%
\pgfpathlineto{\pgfqpoint{3.561048in}{1.033867in}}%
\pgfpathlineto{\pgfqpoint{3.602093in}{1.033867in}}%
\pgfpathlineto{\pgfqpoint{3.643139in}{1.033867in}}%
\pgfpathlineto{\pgfqpoint{3.646105in}{1.022367in}}%
\pgfpathlineto{\pgfqpoint{3.651111in}{1.001739in}}%
\pgfpathlineto{\pgfqpoint{3.655709in}{0.981111in}}%
\pgfpathlineto{\pgfqpoint{3.659976in}{0.960483in}}%
\pgfpathlineto{\pgfqpoint{3.663969in}{0.939855in}}%
\pgfpathlineto{\pgfqpoint{3.667735in}{0.919227in}}%
\pgfpathlineto{\pgfqpoint{3.671310in}{0.898600in}}%
\pgfpathlineto{\pgfqpoint{3.674721in}{0.877972in}}%
\pgfpathlineto{\pgfqpoint{3.677993in}{0.857344in}}%
\pgfpathlineto{\pgfqpoint{3.681145in}{0.836716in}}%
\pgfpathlineto{\pgfqpoint{3.684184in}{0.816139in}}%
\pgfpathlineto{\pgfqpoint{3.725229in}{0.816139in}}%
\pgfpathlineto{\pgfqpoint{3.766275in}{0.816139in}}%
\pgfpathlineto{\pgfqpoint{3.807320in}{0.816139in}}%
\pgfpathlineto{\pgfqpoint{3.848365in}{0.816139in}}%
\pgfpathlineto{\pgfqpoint{3.889411in}{0.816139in}}%
\pgfpathlineto{\pgfqpoint{3.930456in}{0.816139in}}%
\pgfpathlineto{\pgfqpoint{3.971501in}{0.816139in}}%
\pgfpathlineto{\pgfqpoint{4.012547in}{0.816139in}}%
\pgfpathlineto{\pgfqpoint{4.053592in}{0.816139in}}%
\pgfpathlineto{\pgfqpoint{4.094637in}{0.816139in}}%
\pgfpathlineto{\pgfqpoint{4.135683in}{0.816139in}}%
\pgfpathlineto{\pgfqpoint{4.176728in}{0.816139in}}%
\pgfpathlineto{\pgfqpoint{4.217773in}{0.816139in}}%
\pgfpathlineto{\pgfqpoint{4.258819in}{0.816139in}}%
\pgfpathlineto{\pgfqpoint{4.299864in}{0.816139in}}%
\pgfpathlineto{\pgfqpoint{4.340909in}{0.816139in}}%
\pgfpathlineto{\pgfqpoint{4.381955in}{0.816139in}}%
\pgfpathlineto{\pgfqpoint{4.423000in}{0.816139in}}%
\pgfpathlineto{\pgfqpoint{4.464045in}{0.816139in}}%
\pgfpathlineto{\pgfqpoint{4.505091in}{0.816139in}}%
\pgfpathlineto{\pgfqpoint{4.546136in}{0.816139in}}%
\pgfpathlineto{\pgfqpoint{4.587181in}{0.816139in}}%
\pgfpathlineto{\pgfqpoint{4.628227in}{0.816139in}}%
\pgfpathlineto{\pgfqpoint{4.669272in}{0.816139in}}%
\pgfpathlineto{\pgfqpoint{4.669272in}{0.816088in}}%
\pgfpathlineto{\pgfqpoint{4.669272in}{0.795460in}}%
\pgfpathlineto{\pgfqpoint{4.669272in}{0.774833in}}%
\pgfpathlineto{\pgfqpoint{4.669272in}{0.754205in}}%
\pgfpathlineto{\pgfqpoint{4.669272in}{0.733577in}}%
\pgfpathlineto{\pgfqpoint{4.669272in}{0.712949in}}%
\pgfpathlineto{\pgfqpoint{4.669272in}{0.692321in}}%
\pgfpathlineto{\pgfqpoint{4.669272in}{0.687855in}}%
\pgfpathlineto{\pgfqpoint{4.628227in}{0.687855in}}%
\pgfpathlineto{\pgfqpoint{4.587181in}{0.687855in}}%
\pgfpathlineto{\pgfqpoint{4.546136in}{0.687855in}}%
\pgfpathlineto{\pgfqpoint{4.505091in}{0.687855in}}%
\pgfpathlineto{\pgfqpoint{4.464045in}{0.687855in}}%
\pgfpathlineto{\pgfqpoint{4.423000in}{0.687855in}}%
\pgfpathlineto{\pgfqpoint{4.381955in}{0.687855in}}%
\pgfpathlineto{\pgfqpoint{4.340909in}{0.687855in}}%
\pgfpathlineto{\pgfqpoint{4.299864in}{0.687855in}}%
\pgfpathlineto{\pgfqpoint{4.258819in}{0.687855in}}%
\pgfpathlineto{\pgfqpoint{4.217773in}{0.687855in}}%
\pgfpathlineto{\pgfqpoint{4.176728in}{0.687855in}}%
\pgfpathlineto{\pgfqpoint{4.135683in}{0.687855in}}%
\pgfpathlineto{\pgfqpoint{4.094637in}{0.687855in}}%
\pgfpathlineto{\pgfqpoint{4.053592in}{0.687855in}}%
\pgfpathlineto{\pgfqpoint{4.012547in}{0.687855in}}%
\pgfpathlineto{\pgfqpoint{3.971501in}{0.687855in}}%
\pgfpathlineto{\pgfqpoint{3.930456in}{0.687855in}}%
\pgfpathlineto{\pgfqpoint{3.889411in}{0.687855in}}%
\pgfpathlineto{\pgfqpoint{3.848365in}{0.687855in}}%
\pgfpathlineto{\pgfqpoint{3.807320in}{0.687855in}}%
\pgfpathlineto{\pgfqpoint{3.766275in}{0.687855in}}%
\pgfpathlineto{\pgfqpoint{3.725229in}{0.687855in}}%
\pgfpathlineto{\pgfqpoint{3.684184in}{0.687855in}}%
\pgfpathlineto{\pgfqpoint{3.683480in}{0.692321in}}%
\pgfpathlineto{\pgfqpoint{3.680227in}{0.712949in}}%
\pgfpathlineto{\pgfqpoint{3.676877in}{0.733577in}}%
\pgfpathlineto{\pgfqpoint{3.673420in}{0.754205in}}%
\pgfpathlineto{\pgfqpoint{3.669841in}{0.774833in}}%
\pgfpathlineto{\pgfqpoint{3.666126in}{0.795460in}}%
\pgfpathlineto{\pgfqpoint{3.662258in}{0.816088in}}%
\pgfpathlineto{\pgfqpoint{3.658214in}{0.836716in}}%
\pgfpathlineto{\pgfqpoint{3.653971in}{0.857344in}}%
\pgfpathlineto{\pgfqpoint{3.649498in}{0.877972in}}%
\pgfpathlineto{\pgfqpoint{3.644759in}{0.898600in}}%
\pgfpathlineto{\pgfqpoint{3.643139in}{0.905464in}}%
\pgfpathlineto{\pgfqpoint{3.602093in}{0.905464in}}%
\pgfpathlineto{\pgfqpoint{3.561048in}{0.905464in}}%
\pgfpathlineto{\pgfqpoint{3.520003in}{0.905464in}}%
\pgfpathlineto{\pgfqpoint{3.478957in}{0.905464in}}%
\pgfpathlineto{\pgfqpoint{3.437912in}{0.905464in}}%
\pgfpathlineto{\pgfqpoint{3.396867in}{0.905464in}}%
\pgfpathlineto{\pgfqpoint{3.355821in}{0.905464in}}%
\pgfpathlineto{\pgfqpoint{3.314776in}{0.905464in}}%
\pgfpathlineto{\pgfqpoint{3.273731in}{0.905464in}}%
\pgfpathlineto{\pgfqpoint{3.232685in}{0.905464in}}%
\pgfpathlineto{\pgfqpoint{3.191640in}{0.905464in}}%
\pgfpathlineto{\pgfqpoint{3.150595in}{0.905464in}}%
\pgfpathlineto{\pgfqpoint{3.109549in}{0.905464in}}%
\pgfpathlineto{\pgfqpoint{3.068504in}{0.905464in}}%
\pgfpathlineto{\pgfqpoint{3.027459in}{0.905464in}}%
\pgfpathlineto{\pgfqpoint{2.986413in}{0.905464in}}%
\pgfpathlineto{\pgfqpoint{2.945368in}{0.905464in}}%
\pgfpathlineto{\pgfqpoint{2.904323in}{0.905464in}}%
\pgfpathlineto{\pgfqpoint{2.863277in}{0.905464in}}%
\pgfpathlineto{\pgfqpoint{2.822232in}{0.905464in}}%
\pgfpathlineto{\pgfqpoint{2.781187in}{0.905464in}}%
\pgfpathlineto{\pgfqpoint{2.740141in}{0.905464in}}%
\pgfpathlineto{\pgfqpoint{2.699096in}{0.905464in}}%
\pgfpathlineto{\pgfqpoint{2.658051in}{0.905464in}}%
\pgfpathlineto{\pgfqpoint{2.617005in}{0.905464in}}%
\pgfpathlineto{\pgfqpoint{2.575960in}{0.905464in}}%
\pgfpathlineto{\pgfqpoint{2.534915in}{0.905464in}}%
\pgfpathlineto{\pgfqpoint{2.534133in}{0.919227in}}%
\pgfpathlineto{\pgfqpoint{2.532963in}{0.939855in}}%
\pgfpathlineto{\pgfqpoint{2.531787in}{0.960483in}}%
\pgfpathlineto{\pgfqpoint{2.530602in}{0.981111in}}%
\pgfpathlineto{\pgfqpoint{2.529410in}{1.001739in}}%
\pgfpathlineto{\pgfqpoint{2.528208in}{1.022367in}}%
\pgfpathlineto{\pgfqpoint{2.526996in}{1.042994in}}%
\pgfpathlineto{\pgfqpoint{2.525773in}{1.063622in}}%
\pgfpathlineto{\pgfqpoint{2.524539in}{1.084250in}}%
\pgfpathlineto{\pgfqpoint{2.523292in}{1.104878in}}%
\pgfpathlineto{\pgfqpoint{2.522032in}{1.125506in}}%
\pgfpathlineto{\pgfqpoint{2.520755in}{1.146134in}}%
\pgfpathlineto{\pgfqpoint{2.519462in}{1.166761in}}%
\pgfpathlineto{\pgfqpoint{2.518151in}{1.187389in}}%
\pgfpathlineto{\pgfqpoint{2.516818in}{1.208017in}}%
\pgfpathlineto{\pgfqpoint{2.515463in}{1.228645in}}%
\pgfpathlineto{\pgfqpoint{2.514082in}{1.249273in}}%
\pgfpathlineto{\pgfqpoint{2.512672in}{1.269900in}}%
\pgfpathlineto{\pgfqpoint{2.511230in}{1.290528in}}%
\pgfpathlineto{\pgfqpoint{2.509751in}{1.311156in}}%
\pgfpathlineto{\pgfqpoint{2.508230in}{1.331784in}}%
\pgfpathlineto{\pgfqpoint{2.506661in}{1.352412in}}%
\pgfpathlineto{\pgfqpoint{2.505037in}{1.373040in}}%
\pgfpathlineto{\pgfqpoint{2.503349in}{1.393667in}}%
\pgfpathlineto{\pgfqpoint{2.501585in}{1.414295in}}%
\pgfpathlineto{\pgfqpoint{2.499733in}{1.434923in}}%
\pgfpathlineto{\pgfqpoint{2.497776in}{1.455551in}}%
\pgfpathlineto{\pgfqpoint{2.495691in}{1.476179in}}%
\pgfpathlineto{\pgfqpoint{2.493869in}{1.493170in}}%
\pgfpathlineto{\pgfqpoint{2.476636in}{1.476179in}}%
\pgfpathlineto{\pgfqpoint{2.454514in}{1.455551in}}%
\pgfpathlineto{\pgfqpoint{2.452824in}{1.454059in}}%
\pgfpathlineto{\pgfqpoint{2.435822in}{1.434923in}}%
\pgfpathlineto{\pgfqpoint{2.416141in}{1.414295in}}%
\pgfpathlineto{\pgfqpoint{2.411779in}{1.409997in}}%
\pgfpathlineto{\pgfqpoint{2.395437in}{1.393667in}}%
\pgfpathlineto{\pgfqpoint{2.373093in}{1.373040in}}%
\pgfpathlineto{\pgfqpoint{2.370733in}{1.370997in}}%
\pgfpathlineto{\pgfqpoint{2.344648in}{1.352412in}}%
\pgfpathlineto{\pgfqpoint{2.329688in}{1.342581in}}%
\pgfpathlineto{\pgfqpoint{2.304222in}{1.331784in}}%
\pgfpathlineto{\pgfqpoint{2.288643in}{1.325673in}}%
\pgfpathlineto{\pgfqpoint{2.247597in}{1.318505in}}%
\pgfpathlineto{\pgfqpoint{2.206552in}{1.317241in}}%
\pgfpathlineto{\pgfqpoint{2.165507in}{1.317240in}}%
\pgfpathlineto{\pgfqpoint{2.124461in}{1.314020in}}%
\pgfpathlineto{\pgfqpoint{2.112253in}{1.311156in}}%
\pgfpathlineto{\pgfqpoint{2.083416in}{1.304494in}}%
\pgfpathlineto{\pgfqpoint{2.049932in}{1.290528in}}%
\pgfpathlineto{\pgfqpoint{2.042371in}{1.287368in}}%
\pgfpathlineto{\pgfqpoint{2.011719in}{1.269900in}}%
\pgfpathlineto{\pgfqpoint{2.001325in}{1.263868in}}%
\pgfpathlineto{\pgfqpoint{1.978810in}{1.249273in}}%
\pgfpathlineto{\pgfqpoint{1.960280in}{1.236863in}}%
\pgfpathlineto{\pgfqpoint{1.947255in}{1.228645in}}%
\pgfpathlineto{\pgfqpoint{1.919235in}{1.210171in}}%
\pgfpathlineto{\pgfqpoint{1.915145in}{1.208017in}}%
\pgfpathlineto{\pgfqpoint{1.878189in}{1.187529in}}%
\pgfpathlineto{\pgfqpoint{1.877815in}{1.187389in}}%
\pgfpathlineto{\pgfqpoint{1.837144in}{1.171335in}}%
\pgfpathlineto{\pgfqpoint{1.817577in}{1.166761in}}%
\pgfpathlineto{\pgfqpoint{1.796099in}{1.161488in}}%
\pgfpathlineto{\pgfqpoint{1.755053in}{1.155412in}}%
\pgfpathlineto{\pgfqpoint{1.714008in}{1.148111in}}%
\pgfpathlineto{\pgfqpoint{1.707811in}{1.146134in}}%
\pgfpathlineto{\pgfqpoint{1.672963in}{1.135040in}}%
\pgfpathlineto{\pgfqpoint{1.654166in}{1.125506in}}%
\pgfpathlineto{\pgfqpoint{1.631917in}{1.114529in}}%
\pgfpathlineto{\pgfqpoint{1.616233in}{1.104878in}}%
\pgfpathlineto{\pgfqpoint{1.590872in}{1.090037in}}%
\pgfpathlineto{\pgfqpoint{1.580826in}{1.084250in}}%
\pgfpathlineto{\pgfqpoint{1.549827in}{1.067547in}}%
\pgfpathlineto{\pgfqpoint{1.539829in}{1.063622in}}%
\pgfpathlineto{\pgfqpoint{1.508781in}{1.052323in}}%
\pgfpathlineto{\pgfqpoint{1.467736in}{1.046829in}}%
\pgfpathlineto{\pgfqpoint{1.426691in}{1.050896in}}%
\pgfpathlineto{\pgfqpoint{1.385645in}{1.061487in}}%
\pgfpathlineto{\pgfqpoint{1.378631in}{1.063622in}}%
\pgfpathlineto{\pgfqpoint{1.344600in}{1.074128in}}%
\pgfpathlineto{\pgfqpoint{1.303555in}{1.083075in}}%
\pgfpathlineto{\pgfqpoint{1.262509in}{1.083479in}}%
\pgfpathlineto{\pgfqpoint{1.221464in}{1.072025in}}%
\pgfpathlineto{\pgfqpoint{1.207127in}{1.063622in}}%
\pgfpathlineto{\pgfqpoint{1.180419in}{1.047855in}}%
\pgfpathlineto{\pgfqpoint{1.174588in}{1.042994in}}%
\pgfpathlineto{\pgfqpoint{1.150045in}{1.022367in}}%
\pgfpathlineto{\pgfqpoint{1.139373in}{1.013259in}}%
\pgfpathlineto{\pgfqpoint{1.127595in}{1.001739in}}%
\pgfpathlineto{\pgfqpoint{1.106968in}{0.981111in}}%
\pgfpathlineto{\pgfqpoint{1.098328in}{0.972331in}}%
\pgfpathlineto{\pgfqpoint{1.086544in}{0.960483in}}%
\pgfpathlineto{\pgfqpoint{1.066595in}{0.939855in}}%
\pgfpathlineto{\pgfqpoint{1.057283in}{0.930006in}}%
\pgfpathlineto{\pgfqpoint{1.045713in}{0.919227in}}%
\pgfpathlineto{\pgfqpoint{1.024314in}{0.898600in}}%
\pgfpathlineto{\pgfqpoint{1.016237in}{0.890609in}}%
\pgfpathlineto{\pgfqpoint{1.000394in}{0.877972in}}%
\pgfpathlineto{\pgfqpoint{0.975786in}{0.857344in}}%
\pgfpathlineto{\pgfqpoint{0.975192in}{0.856838in}}%
\pgfpathlineto{\pgfqpoint{0.944454in}{0.836716in}}%
\pgfpathlineto{\pgfqpoint{0.934147in}{0.829671in}}%
\pgfpathlineto{\pgfqpoint{0.908500in}{0.816088in}}%
\pgfpathlineto{\pgfqpoint{0.893101in}{0.807677in}}%
\pgfpathlineto{\pgfqpoint{0.866485in}{0.795460in}}%
\pgfpathlineto{\pgfqpoint{0.852056in}{0.788749in}}%
\pgfpathlineto{\pgfqpoint{0.819398in}{0.774833in}}%
\pgfpathlineto{\pgfqpoint{0.811011in}{0.771285in}}%
\pgfpathlineto{\pgfqpoint{0.769965in}{0.755644in}}%
\pgfpathlineto{\pgfqpoint{0.764972in}{0.754205in}}%
\pgfpathlineto{\pgfqpoint{0.728920in}{0.744309in}}%
\pgfpathlineto{\pgfqpoint{0.687875in}{0.740259in}}%
\pgfpathlineto{\pgfqpoint{0.646829in}{0.745995in}}%
\pgfpathclose%
\pgfusepath{stroke,fill}%
}%
\begin{pgfscope}%
\pgfsys@transformshift{0.000000in}{0.000000in}%
\pgfsys@useobject{currentmarker}{}%
\end{pgfscope}%
\end{pgfscope}%
\begin{pgfscope}%
\pgfpathrectangle{\pgfqpoint{0.605784in}{0.382904in}}{\pgfqpoint{4.063488in}{2.042155in}}%
\pgfusepath{clip}%
\pgfsetbuttcap%
\pgfsetroundjoin%
\definecolor{currentfill}{rgb}{0.229739,0.322361,0.545706}%
\pgfsetfillcolor{currentfill}%
\pgfsetlinewidth{1.003750pt}%
\definecolor{currentstroke}{rgb}{0.229739,0.322361,0.545706}%
\pgfsetstrokecolor{currentstroke}%
\pgfsetdash{}{0pt}%
\pgfpathmoveto{\pgfqpoint{0.643201in}{0.651066in}}%
\pgfpathlineto{\pgfqpoint{0.605784in}{0.666235in}}%
\pgfpathlineto{\pgfqpoint{0.605784in}{0.671694in}}%
\pgfpathlineto{\pgfqpoint{0.605784in}{0.692321in}}%
\pgfpathlineto{\pgfqpoint{0.605784in}{0.712949in}}%
\pgfpathlineto{\pgfqpoint{0.605784in}{0.733577in}}%
\pgfpathlineto{\pgfqpoint{0.605784in}{0.754205in}}%
\pgfpathlineto{\pgfqpoint{0.605784in}{0.761947in}}%
\pgfpathlineto{\pgfqpoint{0.625696in}{0.754205in}}%
\pgfpathlineto{\pgfqpoint{0.646829in}{0.745995in}}%
\pgfpathlineto{\pgfqpoint{0.687875in}{0.740259in}}%
\pgfpathlineto{\pgfqpoint{0.728920in}{0.744309in}}%
\pgfpathlineto{\pgfqpoint{0.764972in}{0.754205in}}%
\pgfpathlineto{\pgfqpoint{0.769965in}{0.755644in}}%
\pgfpathlineto{\pgfqpoint{0.811011in}{0.771285in}}%
\pgfpathlineto{\pgfqpoint{0.819398in}{0.774833in}}%
\pgfpathlineto{\pgfqpoint{0.852056in}{0.788749in}}%
\pgfpathlineto{\pgfqpoint{0.866485in}{0.795460in}}%
\pgfpathlineto{\pgfqpoint{0.893101in}{0.807677in}}%
\pgfpathlineto{\pgfqpoint{0.908500in}{0.816088in}}%
\pgfpathlineto{\pgfqpoint{0.934147in}{0.829671in}}%
\pgfpathlineto{\pgfqpoint{0.944454in}{0.836716in}}%
\pgfpathlineto{\pgfqpoint{0.975192in}{0.856838in}}%
\pgfpathlineto{\pgfqpoint{0.975786in}{0.857344in}}%
\pgfpathlineto{\pgfqpoint{1.000394in}{0.877972in}}%
\pgfpathlineto{\pgfqpoint{1.016237in}{0.890609in}}%
\pgfpathlineto{\pgfqpoint{1.024314in}{0.898600in}}%
\pgfpathlineto{\pgfqpoint{1.045713in}{0.919227in}}%
\pgfpathlineto{\pgfqpoint{1.057283in}{0.930006in}}%
\pgfpathlineto{\pgfqpoint{1.066595in}{0.939855in}}%
\pgfpathlineto{\pgfqpoint{1.086544in}{0.960483in}}%
\pgfpathlineto{\pgfqpoint{1.098328in}{0.972331in}}%
\pgfpathlineto{\pgfqpoint{1.106968in}{0.981111in}}%
\pgfpathlineto{\pgfqpoint{1.127595in}{1.001739in}}%
\pgfpathlineto{\pgfqpoint{1.139373in}{1.013259in}}%
\pgfpathlineto{\pgfqpoint{1.150045in}{1.022367in}}%
\pgfpathlineto{\pgfqpoint{1.174588in}{1.042994in}}%
\pgfpathlineto{\pgfqpoint{1.180419in}{1.047855in}}%
\pgfpathlineto{\pgfqpoint{1.207127in}{1.063622in}}%
\pgfpathlineto{\pgfqpoint{1.221464in}{1.072025in}}%
\pgfpathlineto{\pgfqpoint{1.262509in}{1.083479in}}%
\pgfpathlineto{\pgfqpoint{1.303555in}{1.083075in}}%
\pgfpathlineto{\pgfqpoint{1.344600in}{1.074128in}}%
\pgfpathlineto{\pgfqpoint{1.378631in}{1.063622in}}%
\pgfpathlineto{\pgfqpoint{1.385645in}{1.061487in}}%
\pgfpathlineto{\pgfqpoint{1.426691in}{1.050896in}}%
\pgfpathlineto{\pgfqpoint{1.467736in}{1.046829in}}%
\pgfpathlineto{\pgfqpoint{1.508781in}{1.052323in}}%
\pgfpathlineto{\pgfqpoint{1.539829in}{1.063622in}}%
\pgfpathlineto{\pgfqpoint{1.549827in}{1.067547in}}%
\pgfpathlineto{\pgfqpoint{1.580826in}{1.084250in}}%
\pgfpathlineto{\pgfqpoint{1.590872in}{1.090037in}}%
\pgfpathlineto{\pgfqpoint{1.616233in}{1.104878in}}%
\pgfpathlineto{\pgfqpoint{1.631917in}{1.114529in}}%
\pgfpathlineto{\pgfqpoint{1.654166in}{1.125506in}}%
\pgfpathlineto{\pgfqpoint{1.672963in}{1.135040in}}%
\pgfpathlineto{\pgfqpoint{1.707811in}{1.146134in}}%
\pgfpathlineto{\pgfqpoint{1.714008in}{1.148111in}}%
\pgfpathlineto{\pgfqpoint{1.755053in}{1.155412in}}%
\pgfpathlineto{\pgfqpoint{1.796099in}{1.161488in}}%
\pgfpathlineto{\pgfqpoint{1.817577in}{1.166761in}}%
\pgfpathlineto{\pgfqpoint{1.837144in}{1.171335in}}%
\pgfpathlineto{\pgfqpoint{1.877815in}{1.187389in}}%
\pgfpathlineto{\pgfqpoint{1.878189in}{1.187529in}}%
\pgfpathlineto{\pgfqpoint{1.915145in}{1.208017in}}%
\pgfpathlineto{\pgfqpoint{1.919235in}{1.210171in}}%
\pgfpathlineto{\pgfqpoint{1.947255in}{1.228645in}}%
\pgfpathlineto{\pgfqpoint{1.960280in}{1.236863in}}%
\pgfpathlineto{\pgfqpoint{1.978810in}{1.249273in}}%
\pgfpathlineto{\pgfqpoint{2.001325in}{1.263868in}}%
\pgfpathlineto{\pgfqpoint{2.011719in}{1.269900in}}%
\pgfpathlineto{\pgfqpoint{2.042371in}{1.287368in}}%
\pgfpathlineto{\pgfqpoint{2.049932in}{1.290528in}}%
\pgfpathlineto{\pgfqpoint{2.083416in}{1.304494in}}%
\pgfpathlineto{\pgfqpoint{2.112253in}{1.311156in}}%
\pgfpathlineto{\pgfqpoint{2.124461in}{1.314020in}}%
\pgfpathlineto{\pgfqpoint{2.165507in}{1.317240in}}%
\pgfpathlineto{\pgfqpoint{2.206552in}{1.317241in}}%
\pgfpathlineto{\pgfqpoint{2.247597in}{1.318505in}}%
\pgfpathlineto{\pgfqpoint{2.288643in}{1.325673in}}%
\pgfpathlineto{\pgfqpoint{2.304222in}{1.331784in}}%
\pgfpathlineto{\pgfqpoint{2.329688in}{1.342581in}}%
\pgfpathlineto{\pgfqpoint{2.344648in}{1.352412in}}%
\pgfpathlineto{\pgfqpoint{2.370733in}{1.370997in}}%
\pgfpathlineto{\pgfqpoint{2.373093in}{1.373040in}}%
\pgfpathlineto{\pgfqpoint{2.395437in}{1.393667in}}%
\pgfpathlineto{\pgfqpoint{2.411779in}{1.409997in}}%
\pgfpathlineto{\pgfqpoint{2.416141in}{1.414295in}}%
\pgfpathlineto{\pgfqpoint{2.435822in}{1.434923in}}%
\pgfpathlineto{\pgfqpoint{2.452824in}{1.454059in}}%
\pgfpathlineto{\pgfqpoint{2.454514in}{1.455551in}}%
\pgfpathlineto{\pgfqpoint{2.476636in}{1.476179in}}%
\pgfpathlineto{\pgfqpoint{2.493869in}{1.493170in}}%
\pgfpathlineto{\pgfqpoint{2.495691in}{1.476179in}}%
\pgfpathlineto{\pgfqpoint{2.497776in}{1.455551in}}%
\pgfpathlineto{\pgfqpoint{2.499733in}{1.434923in}}%
\pgfpathlineto{\pgfqpoint{2.501585in}{1.414295in}}%
\pgfpathlineto{\pgfqpoint{2.503349in}{1.393667in}}%
\pgfpathlineto{\pgfqpoint{2.505037in}{1.373040in}}%
\pgfpathlineto{\pgfqpoint{2.506661in}{1.352412in}}%
\pgfpathlineto{\pgfqpoint{2.508230in}{1.331784in}}%
\pgfpathlineto{\pgfqpoint{2.509751in}{1.311156in}}%
\pgfpathlineto{\pgfqpoint{2.511230in}{1.290528in}}%
\pgfpathlineto{\pgfqpoint{2.512672in}{1.269900in}}%
\pgfpathlineto{\pgfqpoint{2.514082in}{1.249273in}}%
\pgfpathlineto{\pgfqpoint{2.515463in}{1.228645in}}%
\pgfpathlineto{\pgfqpoint{2.516818in}{1.208017in}}%
\pgfpathlineto{\pgfqpoint{2.518151in}{1.187389in}}%
\pgfpathlineto{\pgfqpoint{2.519462in}{1.166761in}}%
\pgfpathlineto{\pgfqpoint{2.520755in}{1.146134in}}%
\pgfpathlineto{\pgfqpoint{2.522032in}{1.125506in}}%
\pgfpathlineto{\pgfqpoint{2.523292in}{1.104878in}}%
\pgfpathlineto{\pgfqpoint{2.524539in}{1.084250in}}%
\pgfpathlineto{\pgfqpoint{2.525773in}{1.063622in}}%
\pgfpathlineto{\pgfqpoint{2.526996in}{1.042994in}}%
\pgfpathlineto{\pgfqpoint{2.528208in}{1.022367in}}%
\pgfpathlineto{\pgfqpoint{2.529410in}{1.001739in}}%
\pgfpathlineto{\pgfqpoint{2.530602in}{0.981111in}}%
\pgfpathlineto{\pgfqpoint{2.531787in}{0.960483in}}%
\pgfpathlineto{\pgfqpoint{2.532963in}{0.939855in}}%
\pgfpathlineto{\pgfqpoint{2.534133in}{0.919227in}}%
\pgfpathlineto{\pgfqpoint{2.534915in}{0.905464in}}%
\pgfpathlineto{\pgfqpoint{2.575960in}{0.905464in}}%
\pgfpathlineto{\pgfqpoint{2.617005in}{0.905464in}}%
\pgfpathlineto{\pgfqpoint{2.658051in}{0.905464in}}%
\pgfpathlineto{\pgfqpoint{2.699096in}{0.905464in}}%
\pgfpathlineto{\pgfqpoint{2.740141in}{0.905464in}}%
\pgfpathlineto{\pgfqpoint{2.781187in}{0.905464in}}%
\pgfpathlineto{\pgfqpoint{2.822232in}{0.905464in}}%
\pgfpathlineto{\pgfqpoint{2.863277in}{0.905464in}}%
\pgfpathlineto{\pgfqpoint{2.904323in}{0.905464in}}%
\pgfpathlineto{\pgfqpoint{2.945368in}{0.905464in}}%
\pgfpathlineto{\pgfqpoint{2.986413in}{0.905464in}}%
\pgfpathlineto{\pgfqpoint{3.027459in}{0.905464in}}%
\pgfpathlineto{\pgfqpoint{3.068504in}{0.905464in}}%
\pgfpathlineto{\pgfqpoint{3.109549in}{0.905464in}}%
\pgfpathlineto{\pgfqpoint{3.150595in}{0.905464in}}%
\pgfpathlineto{\pgfqpoint{3.191640in}{0.905464in}}%
\pgfpathlineto{\pgfqpoint{3.232685in}{0.905464in}}%
\pgfpathlineto{\pgfqpoint{3.273731in}{0.905464in}}%
\pgfpathlineto{\pgfqpoint{3.314776in}{0.905464in}}%
\pgfpathlineto{\pgfqpoint{3.355821in}{0.905464in}}%
\pgfpathlineto{\pgfqpoint{3.396867in}{0.905464in}}%
\pgfpathlineto{\pgfqpoint{3.437912in}{0.905464in}}%
\pgfpathlineto{\pgfqpoint{3.478957in}{0.905464in}}%
\pgfpathlineto{\pgfqpoint{3.520003in}{0.905464in}}%
\pgfpathlineto{\pgfqpoint{3.561048in}{0.905464in}}%
\pgfpathlineto{\pgfqpoint{3.602093in}{0.905464in}}%
\pgfpathlineto{\pgfqpoint{3.643139in}{0.905464in}}%
\pgfpathlineto{\pgfqpoint{3.644759in}{0.898600in}}%
\pgfpathlineto{\pgfqpoint{3.649498in}{0.877972in}}%
\pgfpathlineto{\pgfqpoint{3.653971in}{0.857344in}}%
\pgfpathlineto{\pgfqpoint{3.658214in}{0.836716in}}%
\pgfpathlineto{\pgfqpoint{3.662258in}{0.816088in}}%
\pgfpathlineto{\pgfqpoint{3.666126in}{0.795460in}}%
\pgfpathlineto{\pgfqpoint{3.669841in}{0.774833in}}%
\pgfpathlineto{\pgfqpoint{3.673420in}{0.754205in}}%
\pgfpathlineto{\pgfqpoint{3.676877in}{0.733577in}}%
\pgfpathlineto{\pgfqpoint{3.680227in}{0.712949in}}%
\pgfpathlineto{\pgfqpoint{3.683480in}{0.692321in}}%
\pgfpathlineto{\pgfqpoint{3.684184in}{0.687855in}}%
\pgfpathlineto{\pgfqpoint{3.725229in}{0.687855in}}%
\pgfpathlineto{\pgfqpoint{3.766275in}{0.687855in}}%
\pgfpathlineto{\pgfqpoint{3.807320in}{0.687855in}}%
\pgfpathlineto{\pgfqpoint{3.848365in}{0.687855in}}%
\pgfpathlineto{\pgfqpoint{3.889411in}{0.687855in}}%
\pgfpathlineto{\pgfqpoint{3.930456in}{0.687855in}}%
\pgfpathlineto{\pgfqpoint{3.971501in}{0.687855in}}%
\pgfpathlineto{\pgfqpoint{4.012547in}{0.687855in}}%
\pgfpathlineto{\pgfqpoint{4.053592in}{0.687855in}}%
\pgfpathlineto{\pgfqpoint{4.094637in}{0.687855in}}%
\pgfpathlineto{\pgfqpoint{4.135683in}{0.687855in}}%
\pgfpathlineto{\pgfqpoint{4.176728in}{0.687855in}}%
\pgfpathlineto{\pgfqpoint{4.217773in}{0.687855in}}%
\pgfpathlineto{\pgfqpoint{4.258819in}{0.687855in}}%
\pgfpathlineto{\pgfqpoint{4.299864in}{0.687855in}}%
\pgfpathlineto{\pgfqpoint{4.340909in}{0.687855in}}%
\pgfpathlineto{\pgfqpoint{4.381955in}{0.687855in}}%
\pgfpathlineto{\pgfqpoint{4.423000in}{0.687855in}}%
\pgfpathlineto{\pgfqpoint{4.464045in}{0.687855in}}%
\pgfpathlineto{\pgfqpoint{4.505091in}{0.687855in}}%
\pgfpathlineto{\pgfqpoint{4.546136in}{0.687855in}}%
\pgfpathlineto{\pgfqpoint{4.587181in}{0.687855in}}%
\pgfpathlineto{\pgfqpoint{4.628227in}{0.687855in}}%
\pgfpathlineto{\pgfqpoint{4.669272in}{0.687855in}}%
\pgfpathlineto{\pgfqpoint{4.669272in}{0.671694in}}%
\pgfpathlineto{\pgfqpoint{4.669272in}{0.651066in}}%
\pgfpathlineto{\pgfqpoint{4.669272in}{0.630438in}}%
\pgfpathlineto{\pgfqpoint{4.669272in}{0.609810in}}%
\pgfpathlineto{\pgfqpoint{4.669272in}{0.589182in}}%
\pgfpathlineto{\pgfqpoint{4.669272in}{0.586530in}}%
\pgfpathlineto{\pgfqpoint{4.628227in}{0.586530in}}%
\pgfpathlineto{\pgfqpoint{4.587181in}{0.586530in}}%
\pgfpathlineto{\pgfqpoint{4.546136in}{0.586530in}}%
\pgfpathlineto{\pgfqpoint{4.505091in}{0.586530in}}%
\pgfpathlineto{\pgfqpoint{4.464045in}{0.586530in}}%
\pgfpathlineto{\pgfqpoint{4.423000in}{0.586530in}}%
\pgfpathlineto{\pgfqpoint{4.381955in}{0.586530in}}%
\pgfpathlineto{\pgfqpoint{4.340909in}{0.586530in}}%
\pgfpathlineto{\pgfqpoint{4.299864in}{0.586530in}}%
\pgfpathlineto{\pgfqpoint{4.258819in}{0.586530in}}%
\pgfpathlineto{\pgfqpoint{4.217773in}{0.586530in}}%
\pgfpathlineto{\pgfqpoint{4.176728in}{0.586530in}}%
\pgfpathlineto{\pgfqpoint{4.135683in}{0.586530in}}%
\pgfpathlineto{\pgfqpoint{4.094637in}{0.586530in}}%
\pgfpathlineto{\pgfqpoint{4.053592in}{0.586530in}}%
\pgfpathlineto{\pgfqpoint{4.012547in}{0.586530in}}%
\pgfpathlineto{\pgfqpoint{3.971501in}{0.586530in}}%
\pgfpathlineto{\pgfqpoint{3.930456in}{0.586530in}}%
\pgfpathlineto{\pgfqpoint{3.889411in}{0.586530in}}%
\pgfpathlineto{\pgfqpoint{3.848365in}{0.586530in}}%
\pgfpathlineto{\pgfqpoint{3.807320in}{0.586530in}}%
\pgfpathlineto{\pgfqpoint{3.766275in}{0.586530in}}%
\pgfpathlineto{\pgfqpoint{3.725229in}{0.586530in}}%
\pgfpathlineto{\pgfqpoint{3.684184in}{0.586530in}}%
\pgfpathlineto{\pgfqpoint{3.683754in}{0.589182in}}%
\pgfpathlineto{\pgfqpoint{3.680412in}{0.609810in}}%
\pgfpathlineto{\pgfqpoint{3.676984in}{0.630438in}}%
\pgfpathlineto{\pgfqpoint{3.673460in}{0.651066in}}%
\pgfpathlineto{\pgfqpoint{3.669830in}{0.671694in}}%
\pgfpathlineto{\pgfqpoint{3.666085in}{0.692321in}}%
\pgfpathlineto{\pgfqpoint{3.662210in}{0.712949in}}%
\pgfpathlineto{\pgfqpoint{3.658193in}{0.733577in}}%
\pgfpathlineto{\pgfqpoint{3.654017in}{0.754205in}}%
\pgfpathlineto{\pgfqpoint{3.649662in}{0.774833in}}%
\pgfpathlineto{\pgfqpoint{3.645107in}{0.795460in}}%
\pgfpathlineto{\pgfqpoint{3.643139in}{0.804162in}}%
\pgfpathlineto{\pgfqpoint{3.602093in}{0.804162in}}%
\pgfpathlineto{\pgfqpoint{3.561048in}{0.804162in}}%
\pgfpathlineto{\pgfqpoint{3.520003in}{0.804162in}}%
\pgfpathlineto{\pgfqpoint{3.478957in}{0.804162in}}%
\pgfpathlineto{\pgfqpoint{3.437912in}{0.804162in}}%
\pgfpathlineto{\pgfqpoint{3.396867in}{0.804162in}}%
\pgfpathlineto{\pgfqpoint{3.355821in}{0.804162in}}%
\pgfpathlineto{\pgfqpoint{3.314776in}{0.804162in}}%
\pgfpathlineto{\pgfqpoint{3.273731in}{0.804162in}}%
\pgfpathlineto{\pgfqpoint{3.232685in}{0.804162in}}%
\pgfpathlineto{\pgfqpoint{3.191640in}{0.804162in}}%
\pgfpathlineto{\pgfqpoint{3.150595in}{0.804162in}}%
\pgfpathlineto{\pgfqpoint{3.109549in}{0.804162in}}%
\pgfpathlineto{\pgfqpoint{3.068504in}{0.804162in}}%
\pgfpathlineto{\pgfqpoint{3.027459in}{0.804162in}}%
\pgfpathlineto{\pgfqpoint{2.986413in}{0.804162in}}%
\pgfpathlineto{\pgfqpoint{2.945368in}{0.804162in}}%
\pgfpathlineto{\pgfqpoint{2.904323in}{0.804162in}}%
\pgfpathlineto{\pgfqpoint{2.863277in}{0.804162in}}%
\pgfpathlineto{\pgfqpoint{2.822232in}{0.804162in}}%
\pgfpathlineto{\pgfqpoint{2.781187in}{0.804162in}}%
\pgfpathlineto{\pgfqpoint{2.740141in}{0.804162in}}%
\pgfpathlineto{\pgfqpoint{2.699096in}{0.804162in}}%
\pgfpathlineto{\pgfqpoint{2.658051in}{0.804162in}}%
\pgfpathlineto{\pgfqpoint{2.617005in}{0.804162in}}%
\pgfpathlineto{\pgfqpoint{2.575960in}{0.804162in}}%
\pgfpathlineto{\pgfqpoint{2.534915in}{0.804162in}}%
\pgfpathlineto{\pgfqpoint{2.534183in}{0.816088in}}%
\pgfpathlineto{\pgfqpoint{2.532917in}{0.836716in}}%
\pgfpathlineto{\pgfqpoint{2.531642in}{0.857344in}}%
\pgfpathlineto{\pgfqpoint{2.530355in}{0.877972in}}%
\pgfpathlineto{\pgfqpoint{2.529055in}{0.898600in}}%
\pgfpathlineto{\pgfqpoint{2.527743in}{0.919227in}}%
\pgfpathlineto{\pgfqpoint{2.526416in}{0.939855in}}%
\pgfpathlineto{\pgfqpoint{2.525074in}{0.960483in}}%
\pgfpathlineto{\pgfqpoint{2.523716in}{0.981111in}}%
\pgfpathlineto{\pgfqpoint{2.522340in}{1.001739in}}%
\pgfpathlineto{\pgfqpoint{2.520946in}{1.022367in}}%
\pgfpathlineto{\pgfqpoint{2.519530in}{1.042994in}}%
\pgfpathlineto{\pgfqpoint{2.518092in}{1.063622in}}%
\pgfpathlineto{\pgfqpoint{2.516630in}{1.084250in}}%
\pgfpathlineto{\pgfqpoint{2.515140in}{1.104878in}}%
\pgfpathlineto{\pgfqpoint{2.513622in}{1.125506in}}%
\pgfpathlineto{\pgfqpoint{2.512071in}{1.146134in}}%
\pgfpathlineto{\pgfqpoint{2.510485in}{1.166761in}}%
\pgfpathlineto{\pgfqpoint{2.508860in}{1.187389in}}%
\pgfpathlineto{\pgfqpoint{2.507192in}{1.208017in}}%
\pgfpathlineto{\pgfqpoint{2.505475in}{1.228645in}}%
\pgfpathlineto{\pgfqpoint{2.503704in}{1.249273in}}%
\pgfpathlineto{\pgfqpoint{2.501873in}{1.269900in}}%
\pgfpathlineto{\pgfqpoint{2.499974in}{1.290528in}}%
\pgfpathlineto{\pgfqpoint{2.497998in}{1.311156in}}%
\pgfpathlineto{\pgfqpoint{2.495934in}{1.331784in}}%
\pgfpathlineto{\pgfqpoint{2.493869in}{1.351500in}}%
\pgfpathlineto{\pgfqpoint{2.465628in}{1.331784in}}%
\pgfpathlineto{\pgfqpoint{2.452824in}{1.323185in}}%
\pgfpathlineto{\pgfqpoint{2.437727in}{1.311156in}}%
\pgfpathlineto{\pgfqpoint{2.411779in}{1.291588in}}%
\pgfpathlineto{\pgfqpoint{2.410327in}{1.290528in}}%
\pgfpathlineto{\pgfqpoint{2.380358in}{1.269900in}}%
\pgfpathlineto{\pgfqpoint{2.370733in}{1.263674in}}%
\pgfpathlineto{\pgfqpoint{2.341992in}{1.249273in}}%
\pgfpathlineto{\pgfqpoint{2.329688in}{1.243505in}}%
\pgfpathlineto{\pgfqpoint{2.288643in}{1.232375in}}%
\pgfpathlineto{\pgfqpoint{2.247597in}{1.228845in}}%
\pgfpathlineto{\pgfqpoint{2.206552in}{1.229703in}}%
\pgfpathlineto{\pgfqpoint{2.165507in}{1.230685in}}%
\pgfpathlineto{\pgfqpoint{2.137653in}{1.228645in}}%
\pgfpathlineto{\pgfqpoint{2.124461in}{1.227705in}}%
\pgfpathlineto{\pgfqpoint{2.083416in}{1.217959in}}%
\pgfpathlineto{\pgfqpoint{2.060348in}{1.208017in}}%
\pgfpathlineto{\pgfqpoint{2.042371in}{1.200255in}}%
\pgfpathlineto{\pgfqpoint{2.020695in}{1.187389in}}%
\pgfpathlineto{\pgfqpoint{2.001325in}{1.175707in}}%
\pgfpathlineto{\pgfqpoint{1.988296in}{1.166761in}}%
\pgfpathlineto{\pgfqpoint{1.960280in}{1.146971in}}%
\pgfpathlineto{\pgfqpoint{1.959066in}{1.146134in}}%
\pgfpathlineto{\pgfqpoint{1.929760in}{1.125506in}}%
\pgfpathlineto{\pgfqpoint{1.919235in}{1.117833in}}%
\pgfpathlineto{\pgfqpoint{1.898618in}{1.104878in}}%
\pgfpathlineto{\pgfqpoint{1.878189in}{1.091479in}}%
\pgfpathlineto{\pgfqpoint{1.863859in}{1.084250in}}%
\pgfpathlineto{\pgfqpoint{1.837144in}{1.070163in}}%
\pgfpathlineto{\pgfqpoint{1.819930in}{1.063622in}}%
\pgfpathlineto{\pgfqpoint{1.796099in}{1.054194in}}%
\pgfpathlineto{\pgfqpoint{1.759228in}{1.042994in}}%
\pgfpathlineto{\pgfqpoint{1.755053in}{1.041687in}}%
\pgfpathlineto{\pgfqpoint{1.714008in}{1.029677in}}%
\pgfpathlineto{\pgfqpoint{1.693562in}{1.022367in}}%
\pgfpathlineto{\pgfqpoint{1.672963in}{1.015013in}}%
\pgfpathlineto{\pgfqpoint{1.642894in}{1.001739in}}%
\pgfpathlineto{\pgfqpoint{1.631917in}{0.996995in}}%
\pgfpathlineto{\pgfqpoint{1.598105in}{0.981111in}}%
\pgfpathlineto{\pgfqpoint{1.590872in}{0.977843in}}%
\pgfpathlineto{\pgfqpoint{1.549827in}{0.961498in}}%
\pgfpathlineto{\pgfqpoint{1.545781in}{0.960483in}}%
\pgfpathlineto{\pgfqpoint{1.508781in}{0.951763in}}%
\pgfpathlineto{\pgfqpoint{1.467736in}{0.950359in}}%
\pgfpathlineto{\pgfqpoint{1.426691in}{0.956845in}}%
\pgfpathlineto{\pgfqpoint{1.413976in}{0.960483in}}%
\pgfpathlineto{\pgfqpoint{1.385645in}{0.968621in}}%
\pgfpathlineto{\pgfqpoint{1.344807in}{0.981111in}}%
\pgfpathlineto{\pgfqpoint{1.344600in}{0.981175in}}%
\pgfpathlineto{\pgfqpoint{1.303555in}{0.989739in}}%
\pgfpathlineto{\pgfqpoint{1.262509in}{0.989493in}}%
\pgfpathlineto{\pgfqpoint{1.233869in}{0.981111in}}%
\pgfpathlineto{\pgfqpoint{1.221464in}{0.977516in}}%
\pgfpathlineto{\pgfqpoint{1.192483in}{0.960483in}}%
\pgfpathlineto{\pgfqpoint{1.180419in}{0.953349in}}%
\pgfpathlineto{\pgfqpoint{1.164192in}{0.939855in}}%
\pgfpathlineto{\pgfqpoint{1.139881in}{0.919227in}}%
\pgfpathlineto{\pgfqpoint{1.139373in}{0.918797in}}%
\pgfpathlineto{\pgfqpoint{1.118802in}{0.898600in}}%
\pgfpathlineto{\pgfqpoint{1.098458in}{0.877972in}}%
\pgfpathlineto{\pgfqpoint{1.098328in}{0.877839in}}%
\pgfpathlineto{\pgfqpoint{1.078311in}{0.857344in}}%
\pgfpathlineto{\pgfqpoint{1.058905in}{0.836716in}}%
\pgfpathlineto{\pgfqpoint{1.057283in}{0.834984in}}%
\pgfpathlineto{\pgfqpoint{1.037803in}{0.816088in}}%
\pgfpathlineto{\pgfqpoint{1.017406in}{0.795460in}}%
\pgfpathlineto{\pgfqpoint{1.016237in}{0.794270in}}%
\pgfpathlineto{\pgfqpoint{0.993472in}{0.774833in}}%
\pgfpathlineto{\pgfqpoint{0.975192in}{0.758548in}}%
\pgfpathlineto{\pgfqpoint{0.969050in}{0.754205in}}%
\pgfpathlineto{\pgfqpoint{0.940620in}{0.733577in}}%
\pgfpathlineto{\pgfqpoint{0.934147in}{0.728758in}}%
\pgfpathlineto{\pgfqpoint{0.907284in}{0.712949in}}%
\pgfpathlineto{\pgfqpoint{0.893101in}{0.704386in}}%
\pgfpathlineto{\pgfqpoint{0.868684in}{0.692321in}}%
\pgfpathlineto{\pgfqpoint{0.852056in}{0.684014in}}%
\pgfpathlineto{\pgfqpoint{0.823007in}{0.671694in}}%
\pgfpathlineto{\pgfqpoint{0.811011in}{0.666637in}}%
\pgfpathlineto{\pgfqpoint{0.769965in}{0.652572in}}%
\pgfpathlineto{\pgfqpoint{0.763234in}{0.651066in}}%
\pgfpathlineto{\pgfqpoint{0.728920in}{0.643688in}}%
\pgfpathlineto{\pgfqpoint{0.687875in}{0.642148in}}%
\pgfpathlineto{\pgfqpoint{0.646829in}{0.649596in}}%
\pgfpathclose%
\pgfusepath{stroke,fill}%
\end{pgfscope}%
\begin{pgfscope}%
\pgfpathrectangle{\pgfqpoint{0.605784in}{0.382904in}}{\pgfqpoint{4.063488in}{2.042155in}}%
\pgfusepath{clip}%
\pgfsetbuttcap%
\pgfsetroundjoin%
\definecolor{currentfill}{rgb}{0.229739,0.322361,0.545706}%
\pgfsetfillcolor{currentfill}%
\pgfsetlinewidth{1.003750pt}%
\definecolor{currentstroke}{rgb}{0.229739,0.322361,0.545706}%
\pgfsetstrokecolor{currentstroke}%
\pgfsetdash{}{0pt}%
\pgfpathmoveto{\pgfqpoint{0.627595in}{1.806224in}}%
\pgfpathlineto{\pgfqpoint{0.605784in}{1.819109in}}%
\pgfpathlineto{\pgfqpoint{0.605784in}{1.826852in}}%
\pgfpathlineto{\pgfqpoint{0.605784in}{1.847480in}}%
\pgfpathlineto{\pgfqpoint{0.605784in}{1.868107in}}%
\pgfpathlineto{\pgfqpoint{0.605784in}{1.888735in}}%
\pgfpathlineto{\pgfqpoint{0.605784in}{1.909363in}}%
\pgfpathlineto{\pgfqpoint{0.605784in}{1.914822in}}%
\pgfpathlineto{\pgfqpoint{0.615291in}{1.909363in}}%
\pgfpathlineto{\pgfqpoint{0.646829in}{1.891238in}}%
\pgfpathlineto{\pgfqpoint{0.650737in}{1.888735in}}%
\pgfpathlineto{\pgfqpoint{0.682649in}{1.868107in}}%
\pgfpathlineto{\pgfqpoint{0.687875in}{1.864701in}}%
\pgfpathlineto{\pgfqpoint{0.717662in}{1.847480in}}%
\pgfpathlineto{\pgfqpoint{0.728920in}{1.840867in}}%
\pgfpathlineto{\pgfqpoint{0.767792in}{1.826852in}}%
\pgfpathlineto{\pgfqpoint{0.769965in}{1.826044in}}%
\pgfpathlineto{\pgfqpoint{0.811011in}{1.825428in}}%
\pgfpathlineto{\pgfqpoint{0.814499in}{1.826852in}}%
\pgfpathlineto{\pgfqpoint{0.852056in}{1.842084in}}%
\pgfpathlineto{\pgfqpoint{0.858631in}{1.847480in}}%
\pgfpathlineto{\pgfqpoint{0.883757in}{1.868107in}}%
\pgfpathlineto{\pgfqpoint{0.893101in}{1.875806in}}%
\pgfpathlineto{\pgfqpoint{0.904408in}{1.888735in}}%
\pgfpathlineto{\pgfqpoint{0.922298in}{1.909363in}}%
\pgfpathlineto{\pgfqpoint{0.934147in}{1.923148in}}%
\pgfpathlineto{\pgfqpoint{0.939258in}{1.929991in}}%
\pgfpathlineto{\pgfqpoint{0.954647in}{1.950619in}}%
\pgfpathlineto{\pgfqpoint{0.969757in}{1.971247in}}%
\pgfpathlineto{\pgfqpoint{0.975192in}{1.978675in}}%
\pgfpathlineto{\pgfqpoint{0.984707in}{1.991874in}}%
\pgfpathlineto{\pgfqpoint{0.999408in}{2.012502in}}%
\pgfpathlineto{\pgfqpoint{1.013813in}{2.033130in}}%
\pgfpathlineto{\pgfqpoint{1.016237in}{2.036594in}}%
\pgfpathlineto{\pgfqpoint{1.029245in}{2.053758in}}%
\pgfpathlineto{\pgfqpoint{1.044568in}{2.074386in}}%
\pgfpathlineto{\pgfqpoint{1.057283in}{2.091837in}}%
\pgfpathlineto{\pgfqpoint{1.059986in}{2.095013in}}%
\pgfpathlineto{\pgfqpoint{1.077484in}{2.115641in}}%
\pgfpathlineto{\pgfqpoint{1.094488in}{2.136269in}}%
\pgfpathlineto{\pgfqpoint{1.098328in}{2.140947in}}%
\pgfpathlineto{\pgfqpoint{1.114463in}{2.156897in}}%
\pgfpathlineto{\pgfqpoint{1.134811in}{2.177525in}}%
\pgfpathlineto{\pgfqpoint{1.139373in}{2.182181in}}%
\pgfpathlineto{\pgfqpoint{1.159504in}{2.198153in}}%
\pgfpathlineto{\pgfqpoint{1.180419in}{2.215093in}}%
\pgfpathlineto{\pgfqpoint{1.186551in}{2.218780in}}%
\pgfpathlineto{\pgfqpoint{1.220668in}{2.239408in}}%
\pgfpathlineto{\pgfqpoint{1.221464in}{2.239889in}}%
\pgfpathlineto{\pgfqpoint{1.262509in}{2.256465in}}%
\pgfpathlineto{\pgfqpoint{1.280986in}{2.260036in}}%
\pgfpathlineto{\pgfqpoint{1.303555in}{2.264286in}}%
\pgfpathlineto{\pgfqpoint{1.344600in}{2.262152in}}%
\pgfpathlineto{\pgfqpoint{1.351197in}{2.260036in}}%
\pgfpathlineto{\pgfqpoint{1.385645in}{2.248852in}}%
\pgfpathlineto{\pgfqpoint{1.401508in}{2.239408in}}%
\pgfpathlineto{\pgfqpoint{1.426691in}{2.224358in}}%
\pgfpathlineto{\pgfqpoint{1.433416in}{2.218780in}}%
\pgfpathlineto{\pgfqpoint{1.458093in}{2.198153in}}%
\pgfpathlineto{\pgfqpoint{1.467736in}{2.190046in}}%
\pgfpathlineto{\pgfqpoint{1.480499in}{2.177525in}}%
\pgfpathlineto{\pgfqpoint{1.501370in}{2.156897in}}%
\pgfpathlineto{\pgfqpoint{1.508781in}{2.149484in}}%
\pgfpathlineto{\pgfqpoint{1.522094in}{2.136269in}}%
\pgfpathlineto{\pgfqpoint{1.542607in}{2.115641in}}%
\pgfpathlineto{\pgfqpoint{1.549827in}{2.108254in}}%
\pgfpathlineto{\pgfqpoint{1.565764in}{2.095013in}}%
\pgfpathlineto{\pgfqpoint{1.590087in}{2.074386in}}%
\pgfpathlineto{\pgfqpoint{1.590872in}{2.073701in}}%
\pgfpathlineto{\pgfqpoint{1.631496in}{2.053758in}}%
\pgfpathlineto{\pgfqpoint{1.631917in}{2.053543in}}%
\pgfpathlineto{\pgfqpoint{1.642200in}{2.053758in}}%
\pgfpathlineto{\pgfqpoint{1.672963in}{2.054387in}}%
\pgfpathlineto{\pgfqpoint{1.706060in}{2.074386in}}%
\pgfpathlineto{\pgfqpoint{1.714008in}{2.079180in}}%
\pgfpathlineto{\pgfqpoint{1.727872in}{2.095013in}}%
\pgfpathlineto{\pgfqpoint{1.745856in}{2.115641in}}%
\pgfpathlineto{\pgfqpoint{1.755053in}{2.126170in}}%
\pgfpathlineto{\pgfqpoint{1.761595in}{2.136269in}}%
\pgfpathlineto{\pgfqpoint{1.775011in}{2.156897in}}%
\pgfpathlineto{\pgfqpoint{1.788330in}{2.177525in}}%
\pgfpathlineto{\pgfqpoint{1.796099in}{2.189532in}}%
\pgfpathlineto{\pgfqpoint{1.801015in}{2.198153in}}%
\pgfpathlineto{\pgfqpoint{1.812833in}{2.218780in}}%
\pgfpathlineto{\pgfqpoint{1.824558in}{2.239408in}}%
\pgfpathlineto{\pgfqpoint{1.836194in}{2.260036in}}%
\pgfpathlineto{\pgfqpoint{1.837144in}{2.261700in}}%
\pgfpathlineto{\pgfqpoint{1.847852in}{2.280664in}}%
\pgfpathlineto{\pgfqpoint{1.859402in}{2.301292in}}%
\pgfpathlineto{\pgfqpoint{1.870840in}{2.321920in}}%
\pgfpathlineto{\pgfqpoint{1.878189in}{2.335197in}}%
\pgfpathlineto{\pgfqpoint{1.882588in}{2.342547in}}%
\pgfpathlineto{\pgfqpoint{1.894957in}{2.363175in}}%
\pgfpathlineto{\pgfqpoint{1.907170in}{2.383803in}}%
\pgfpathlineto{\pgfqpoint{1.919235in}{2.404424in}}%
\pgfpathlineto{\pgfqpoint{1.919239in}{2.404431in}}%
\pgfpathlineto{\pgfqpoint{1.933387in}{2.425059in}}%
\pgfpathlineto{\pgfqpoint{1.960280in}{2.425059in}}%
\pgfpathlineto{\pgfqpoint{1.999798in}{2.425059in}}%
\pgfpathlineto{\pgfqpoint{1.983616in}{2.404431in}}%
\pgfpathlineto{\pgfqpoint{1.967130in}{2.383803in}}%
\pgfpathlineto{\pgfqpoint{1.960280in}{2.375258in}}%
\pgfpathlineto{\pgfqpoint{1.952535in}{2.363175in}}%
\pgfpathlineto{\pgfqpoint{1.939256in}{2.342547in}}%
\pgfpathlineto{\pgfqpoint{1.925770in}{2.321920in}}%
\pgfpathlineto{\pgfqpoint{1.919235in}{2.311948in}}%
\pgfpathlineto{\pgfqpoint{1.913285in}{2.301292in}}%
\pgfpathlineto{\pgfqpoint{1.901775in}{2.280664in}}%
\pgfpathlineto{\pgfqpoint{1.890129in}{2.260036in}}%
\pgfpathlineto{\pgfqpoint{1.878338in}{2.239408in}}%
\pgfpathlineto{\pgfqpoint{1.878189in}{2.239145in}}%
\pgfpathlineto{\pgfqpoint{1.867675in}{2.218780in}}%
\pgfpathlineto{\pgfqpoint{1.856943in}{2.198153in}}%
\pgfpathlineto{\pgfqpoint{1.846121in}{2.177525in}}%
\pgfpathlineto{\pgfqpoint{1.837144in}{2.160490in}}%
\pgfpathlineto{\pgfqpoint{1.835244in}{2.156897in}}%
\pgfpathlineto{\pgfqpoint{1.824492in}{2.136269in}}%
\pgfpathlineto{\pgfqpoint{1.813684in}{2.115641in}}%
\pgfpathlineto{\pgfqpoint{1.802816in}{2.095013in}}%
\pgfpathlineto{\pgfqpoint{1.796099in}{2.082214in}}%
\pgfpathlineto{\pgfqpoint{1.791450in}{2.074386in}}%
\pgfpathlineto{\pgfqpoint{1.779340in}{2.053758in}}%
\pgfpathlineto{\pgfqpoint{1.767181in}{2.033130in}}%
\pgfpathlineto{\pgfqpoint{1.755053in}{2.012640in}}%
\pgfpathlineto{\pgfqpoint{1.754943in}{2.012502in}}%
\pgfpathlineto{\pgfqpoint{1.738704in}{1.991874in}}%
\pgfpathlineto{\pgfqpoint{1.722355in}{1.971247in}}%
\pgfpathlineto{\pgfqpoint{1.714008in}{1.960641in}}%
\pgfpathlineto{\pgfqpoint{1.698357in}{1.950619in}}%
\pgfpathlineto{\pgfqpoint{1.672963in}{1.934322in}}%
\pgfpathlineto{\pgfqpoint{1.631917in}{1.935964in}}%
\pgfpathlineto{\pgfqpoint{1.608787in}{1.950619in}}%
\pgfpathlineto{\pgfqpoint{1.590872in}{1.961444in}}%
\pgfpathlineto{\pgfqpoint{1.581314in}{1.971247in}}%
\pgfpathlineto{\pgfqpoint{1.560558in}{1.991874in}}%
\pgfpathlineto{\pgfqpoint{1.549827in}{2.002229in}}%
\pgfpathlineto{\pgfqpoint{1.541006in}{2.012502in}}%
\pgfpathlineto{\pgfqpoint{1.522902in}{2.033130in}}%
\pgfpathlineto{\pgfqpoint{1.508781in}{2.048949in}}%
\pgfpathlineto{\pgfqpoint{1.504448in}{2.053758in}}%
\pgfpathlineto{\pgfqpoint{1.485500in}{2.074386in}}%
\pgfpathlineto{\pgfqpoint{1.467736in}{2.093607in}}%
\pgfpathlineto{\pgfqpoint{1.466190in}{2.095013in}}%
\pgfpathlineto{\pgfqpoint{1.443141in}{2.115641in}}%
\pgfpathlineto{\pgfqpoint{1.426691in}{2.130337in}}%
\pgfpathlineto{\pgfqpoint{1.417240in}{2.136269in}}%
\pgfpathlineto{\pgfqpoint{1.385645in}{2.156009in}}%
\pgfpathlineto{\pgfqpoint{1.382926in}{2.156897in}}%
\pgfpathlineto{\pgfqpoint{1.344600in}{2.169233in}}%
\pgfpathlineto{\pgfqpoint{1.303555in}{2.170859in}}%
\pgfpathlineto{\pgfqpoint{1.262509in}{2.162383in}}%
\pgfpathlineto{\pgfqpoint{1.249193in}{2.156897in}}%
\pgfpathlineto{\pgfqpoint{1.221464in}{2.145338in}}%
\pgfpathlineto{\pgfqpoint{1.206509in}{2.136269in}}%
\pgfpathlineto{\pgfqpoint{1.180419in}{2.120560in}}%
\pgfpathlineto{\pgfqpoint{1.174334in}{2.115641in}}%
\pgfpathlineto{\pgfqpoint{1.148554in}{2.095013in}}%
\pgfpathlineto{\pgfqpoint{1.139373in}{2.087778in}}%
\pgfpathlineto{\pgfqpoint{1.126300in}{2.074386in}}%
\pgfpathlineto{\pgfqpoint{1.105669in}{2.053758in}}%
\pgfpathlineto{\pgfqpoint{1.098328in}{2.046520in}}%
\pgfpathlineto{\pgfqpoint{1.087505in}{2.033130in}}%
\pgfpathlineto{\pgfqpoint{1.070502in}{2.012502in}}%
\pgfpathlineto{\pgfqpoint{1.057283in}{1.996859in}}%
\pgfpathlineto{\pgfqpoint{1.053731in}{1.991874in}}%
\pgfpathlineto{\pgfqpoint{1.039017in}{1.971247in}}%
\pgfpathlineto{\pgfqpoint{1.023901in}{1.950619in}}%
\pgfpathlineto{\pgfqpoint{1.016237in}{1.940280in}}%
\pgfpathlineto{\pgfqpoint{1.009311in}{1.929991in}}%
\pgfpathlineto{\pgfqpoint{0.995339in}{1.909363in}}%
\pgfpathlineto{\pgfqpoint{0.981050in}{1.888735in}}%
\pgfpathlineto{\pgfqpoint{0.975192in}{1.880317in}}%
\pgfpathlineto{\pgfqpoint{0.966672in}{1.868107in}}%
\pgfpathlineto{\pgfqpoint{0.952185in}{1.847480in}}%
\pgfpathlineto{\pgfqpoint{0.937413in}{1.826852in}}%
\pgfpathlineto{\pgfqpoint{0.934147in}{1.822272in}}%
\pgfpathlineto{\pgfqpoint{0.921038in}{1.806224in}}%
\pgfpathlineto{\pgfqpoint{0.903994in}{1.785596in}}%
\pgfpathlineto{\pgfqpoint{0.893101in}{1.772517in}}%
\pgfpathlineto{\pgfqpoint{0.884313in}{1.764968in}}%
\pgfpathlineto{\pgfqpoint{0.860246in}{1.744340in}}%
\pgfpathlineto{\pgfqpoint{0.852056in}{1.737330in}}%
\pgfpathlineto{\pgfqpoint{0.818311in}{1.723713in}}%
\pgfpathlineto{\pgfqpoint{0.811011in}{1.720744in}}%
\pgfpathlineto{\pgfqpoint{0.769965in}{1.722975in}}%
\pgfpathlineto{\pgfqpoint{0.768276in}{1.723713in}}%
\pgfpathlineto{\pgfqpoint{0.728920in}{1.740281in}}%
\pgfpathlineto{\pgfqpoint{0.722710in}{1.744340in}}%
\pgfpathlineto{\pgfqpoint{0.690498in}{1.764968in}}%
\pgfpathlineto{\pgfqpoint{0.687875in}{1.766612in}}%
\pgfpathlineto{\pgfqpoint{0.660343in}{1.785596in}}%
\pgfpathlineto{\pgfqpoint{0.646829in}{1.794850in}}%
\pgfpathclose%
\pgfusepath{stroke,fill}%
\end{pgfscope}%
\begin{pgfscope}%
\pgfpathrectangle{\pgfqpoint{0.605784in}{0.382904in}}{\pgfqpoint{4.063488in}{2.042155in}}%
\pgfusepath{clip}%
\pgfsetbuttcap%
\pgfsetroundjoin%
\definecolor{currentfill}{rgb}{0.229739,0.322361,0.545706}%
\pgfsetfillcolor{currentfill}%
\pgfsetlinewidth{1.003750pt}%
\definecolor{currentstroke}{rgb}{0.229739,0.322361,0.545706}%
\pgfsetstrokecolor{currentstroke}%
\pgfsetdash{}{0pt}%
\pgfpathmoveto{\pgfqpoint{2.279797in}{2.425059in}}%
\pgfpathlineto{\pgfqpoint{2.288643in}{2.425059in}}%
\pgfpathlineto{\pgfqpoint{2.329688in}{2.425059in}}%
\pgfpathlineto{\pgfqpoint{2.352125in}{2.425059in}}%
\pgfpathlineto{\pgfqpoint{2.366653in}{2.404431in}}%
\pgfpathlineto{\pgfqpoint{2.370733in}{2.398514in}}%
\pgfpathlineto{\pgfqpoint{2.381697in}{2.383803in}}%
\pgfpathlineto{\pgfqpoint{2.396754in}{2.363175in}}%
\pgfpathlineto{\pgfqpoint{2.411526in}{2.342547in}}%
\pgfpathlineto{\pgfqpoint{2.411779in}{2.342185in}}%
\pgfpathlineto{\pgfqpoint{2.430347in}{2.321920in}}%
\pgfpathlineto{\pgfqpoint{2.448723in}{2.301292in}}%
\pgfpathlineto{\pgfqpoint{2.452824in}{2.296528in}}%
\pgfpathlineto{\pgfqpoint{2.479986in}{2.280664in}}%
\pgfpathlineto{\pgfqpoint{2.493869in}{2.272154in}}%
\pgfpathlineto{\pgfqpoint{2.495016in}{2.260036in}}%
\pgfpathlineto{\pgfqpoint{2.497016in}{2.239408in}}%
\pgfpathlineto{\pgfqpoint{2.499106in}{2.218780in}}%
\pgfpathlineto{\pgfqpoint{2.501301in}{2.198153in}}%
\pgfpathlineto{\pgfqpoint{2.503617in}{2.177525in}}%
\pgfpathlineto{\pgfqpoint{2.506073in}{2.156897in}}%
\pgfpathlineto{\pgfqpoint{2.508694in}{2.136269in}}%
\pgfpathlineto{\pgfqpoint{2.511510in}{2.115641in}}%
\pgfpathlineto{\pgfqpoint{2.514559in}{2.095013in}}%
\pgfpathlineto{\pgfqpoint{2.517889in}{2.074386in}}%
\pgfpathlineto{\pgfqpoint{2.521565in}{2.053758in}}%
\pgfpathlineto{\pgfqpoint{2.525669in}{2.033130in}}%
\pgfpathlineto{\pgfqpoint{2.530314in}{2.012502in}}%
\pgfpathlineto{\pgfqpoint{2.534915in}{1.994527in}}%
\pgfpathlineto{\pgfqpoint{2.575960in}{1.994527in}}%
\pgfpathlineto{\pgfqpoint{2.617005in}{1.994527in}}%
\pgfpathlineto{\pgfqpoint{2.658051in}{1.994527in}}%
\pgfpathlineto{\pgfqpoint{2.699096in}{1.994527in}}%
\pgfpathlineto{\pgfqpoint{2.740141in}{1.994527in}}%
\pgfpathlineto{\pgfqpoint{2.781187in}{1.994527in}}%
\pgfpathlineto{\pgfqpoint{2.822232in}{1.994527in}}%
\pgfpathlineto{\pgfqpoint{2.863277in}{1.994527in}}%
\pgfpathlineto{\pgfqpoint{2.904323in}{1.994527in}}%
\pgfpathlineto{\pgfqpoint{2.945368in}{1.994527in}}%
\pgfpathlineto{\pgfqpoint{2.986413in}{1.994527in}}%
\pgfpathlineto{\pgfqpoint{3.027459in}{1.994527in}}%
\pgfpathlineto{\pgfqpoint{3.068504in}{1.994527in}}%
\pgfpathlineto{\pgfqpoint{3.109549in}{1.994527in}}%
\pgfpathlineto{\pgfqpoint{3.150595in}{1.994527in}}%
\pgfpathlineto{\pgfqpoint{3.191640in}{1.994527in}}%
\pgfpathlineto{\pgfqpoint{3.232685in}{1.994527in}}%
\pgfpathlineto{\pgfqpoint{3.273731in}{1.994527in}}%
\pgfpathlineto{\pgfqpoint{3.314776in}{1.994527in}}%
\pgfpathlineto{\pgfqpoint{3.355821in}{1.994527in}}%
\pgfpathlineto{\pgfqpoint{3.396867in}{1.994527in}}%
\pgfpathlineto{\pgfqpoint{3.437912in}{1.994527in}}%
\pgfpathlineto{\pgfqpoint{3.478957in}{1.994527in}}%
\pgfpathlineto{\pgfqpoint{3.520003in}{1.994527in}}%
\pgfpathlineto{\pgfqpoint{3.561048in}{1.994527in}}%
\pgfpathlineto{\pgfqpoint{3.602093in}{1.994527in}}%
\pgfpathlineto{\pgfqpoint{3.643139in}{1.994527in}}%
\pgfpathlineto{\pgfqpoint{3.643569in}{1.991874in}}%
\pgfpathlineto{\pgfqpoint{3.646910in}{1.971247in}}%
\pgfpathlineto{\pgfqpoint{3.650339in}{1.950619in}}%
\pgfpathlineto{\pgfqpoint{3.653863in}{1.929991in}}%
\pgfpathlineto{\pgfqpoint{3.657492in}{1.909363in}}%
\pgfpathlineto{\pgfqpoint{3.661238in}{1.888735in}}%
\pgfpathlineto{\pgfqpoint{3.665112in}{1.868107in}}%
\pgfpathlineto{\pgfqpoint{3.669129in}{1.847480in}}%
\pgfpathlineto{\pgfqpoint{3.673305in}{1.826852in}}%
\pgfpathlineto{\pgfqpoint{3.677660in}{1.806224in}}%
\pgfpathlineto{\pgfqpoint{3.682216in}{1.785596in}}%
\pgfpathlineto{\pgfqpoint{3.684184in}{1.776894in}}%
\pgfpathlineto{\pgfqpoint{3.725229in}{1.776894in}}%
\pgfpathlineto{\pgfqpoint{3.766275in}{1.776894in}}%
\pgfpathlineto{\pgfqpoint{3.807320in}{1.776894in}}%
\pgfpathlineto{\pgfqpoint{3.848365in}{1.776894in}}%
\pgfpathlineto{\pgfqpoint{3.889411in}{1.776894in}}%
\pgfpathlineto{\pgfqpoint{3.930456in}{1.776894in}}%
\pgfpathlineto{\pgfqpoint{3.971501in}{1.776894in}}%
\pgfpathlineto{\pgfqpoint{4.012547in}{1.776894in}}%
\pgfpathlineto{\pgfqpoint{4.053592in}{1.776894in}}%
\pgfpathlineto{\pgfqpoint{4.094637in}{1.776894in}}%
\pgfpathlineto{\pgfqpoint{4.135683in}{1.776894in}}%
\pgfpathlineto{\pgfqpoint{4.176728in}{1.776894in}}%
\pgfpathlineto{\pgfqpoint{4.217773in}{1.776894in}}%
\pgfpathlineto{\pgfqpoint{4.258819in}{1.776894in}}%
\pgfpathlineto{\pgfqpoint{4.299864in}{1.776894in}}%
\pgfpathlineto{\pgfqpoint{4.340909in}{1.776894in}}%
\pgfpathlineto{\pgfqpoint{4.381955in}{1.776894in}}%
\pgfpathlineto{\pgfqpoint{4.423000in}{1.776894in}}%
\pgfpathlineto{\pgfqpoint{4.464045in}{1.776894in}}%
\pgfpathlineto{\pgfqpoint{4.505091in}{1.776894in}}%
\pgfpathlineto{\pgfqpoint{4.546136in}{1.776894in}}%
\pgfpathlineto{\pgfqpoint{4.587181in}{1.776894in}}%
\pgfpathlineto{\pgfqpoint{4.628227in}{1.776894in}}%
\pgfpathlineto{\pgfqpoint{4.669272in}{1.776894in}}%
\pgfpathlineto{\pgfqpoint{4.669272in}{1.764968in}}%
\pgfpathlineto{\pgfqpoint{4.669272in}{1.744340in}}%
\pgfpathlineto{\pgfqpoint{4.669272in}{1.723713in}}%
\pgfpathlineto{\pgfqpoint{4.669272in}{1.703085in}}%
\pgfpathlineto{\pgfqpoint{4.669272in}{1.682457in}}%
\pgfpathlineto{\pgfqpoint{4.669272in}{1.675593in}}%
\pgfpathlineto{\pgfqpoint{4.628227in}{1.675593in}}%
\pgfpathlineto{\pgfqpoint{4.587181in}{1.675593in}}%
\pgfpathlineto{\pgfqpoint{4.546136in}{1.675593in}}%
\pgfpathlineto{\pgfqpoint{4.505091in}{1.675593in}}%
\pgfpathlineto{\pgfqpoint{4.464045in}{1.675593in}}%
\pgfpathlineto{\pgfqpoint{4.423000in}{1.675593in}}%
\pgfpathlineto{\pgfqpoint{4.381955in}{1.675593in}}%
\pgfpathlineto{\pgfqpoint{4.340909in}{1.675593in}}%
\pgfpathlineto{\pgfqpoint{4.299864in}{1.675593in}}%
\pgfpathlineto{\pgfqpoint{4.258819in}{1.675593in}}%
\pgfpathlineto{\pgfqpoint{4.217773in}{1.675593in}}%
\pgfpathlineto{\pgfqpoint{4.176728in}{1.675593in}}%
\pgfpathlineto{\pgfqpoint{4.135683in}{1.675593in}}%
\pgfpathlineto{\pgfqpoint{4.094637in}{1.675593in}}%
\pgfpathlineto{\pgfqpoint{4.053592in}{1.675593in}}%
\pgfpathlineto{\pgfqpoint{4.012547in}{1.675593in}}%
\pgfpathlineto{\pgfqpoint{3.971501in}{1.675593in}}%
\pgfpathlineto{\pgfqpoint{3.930456in}{1.675593in}}%
\pgfpathlineto{\pgfqpoint{3.889411in}{1.675593in}}%
\pgfpathlineto{\pgfqpoint{3.848365in}{1.675593in}}%
\pgfpathlineto{\pgfqpoint{3.807320in}{1.675593in}}%
\pgfpathlineto{\pgfqpoint{3.766275in}{1.675593in}}%
\pgfpathlineto{\pgfqpoint{3.725229in}{1.675593in}}%
\pgfpathlineto{\pgfqpoint{3.684184in}{1.675593in}}%
\pgfpathlineto{\pgfqpoint{3.682564in}{1.682457in}}%
\pgfpathlineto{\pgfqpoint{3.677825in}{1.703085in}}%
\pgfpathlineto{\pgfqpoint{3.673352in}{1.723713in}}%
\pgfpathlineto{\pgfqpoint{3.669108in}{1.744340in}}%
\pgfpathlineto{\pgfqpoint{3.665065in}{1.764968in}}%
\pgfpathlineto{\pgfqpoint{3.661196in}{1.785596in}}%
\pgfpathlineto{\pgfqpoint{3.657481in}{1.806224in}}%
\pgfpathlineto{\pgfqpoint{3.653903in}{1.826852in}}%
\pgfpathlineto{\pgfqpoint{3.650445in}{1.847480in}}%
\pgfpathlineto{\pgfqpoint{3.647095in}{1.868107in}}%
\pgfpathlineto{\pgfqpoint{3.643842in}{1.888735in}}%
\pgfpathlineto{\pgfqpoint{3.643139in}{1.893201in}}%
\pgfpathlineto{\pgfqpoint{3.602093in}{1.893201in}}%
\pgfpathlineto{\pgfqpoint{3.561048in}{1.893201in}}%
\pgfpathlineto{\pgfqpoint{3.520003in}{1.893201in}}%
\pgfpathlineto{\pgfqpoint{3.478957in}{1.893201in}}%
\pgfpathlineto{\pgfqpoint{3.437912in}{1.893201in}}%
\pgfpathlineto{\pgfqpoint{3.396867in}{1.893201in}}%
\pgfpathlineto{\pgfqpoint{3.355821in}{1.893201in}}%
\pgfpathlineto{\pgfqpoint{3.314776in}{1.893201in}}%
\pgfpathlineto{\pgfqpoint{3.273731in}{1.893201in}}%
\pgfpathlineto{\pgfqpoint{3.232685in}{1.893201in}}%
\pgfpathlineto{\pgfqpoint{3.191640in}{1.893201in}}%
\pgfpathlineto{\pgfqpoint{3.150595in}{1.893201in}}%
\pgfpathlineto{\pgfqpoint{3.109549in}{1.893201in}}%
\pgfpathlineto{\pgfqpoint{3.068504in}{1.893201in}}%
\pgfpathlineto{\pgfqpoint{3.027459in}{1.893201in}}%
\pgfpathlineto{\pgfqpoint{2.986413in}{1.893201in}}%
\pgfpathlineto{\pgfqpoint{2.945368in}{1.893201in}}%
\pgfpathlineto{\pgfqpoint{2.904323in}{1.893201in}}%
\pgfpathlineto{\pgfqpoint{2.863277in}{1.893201in}}%
\pgfpathlineto{\pgfqpoint{2.822232in}{1.893201in}}%
\pgfpathlineto{\pgfqpoint{2.781187in}{1.893201in}}%
\pgfpathlineto{\pgfqpoint{2.740141in}{1.893201in}}%
\pgfpathlineto{\pgfqpoint{2.699096in}{1.893201in}}%
\pgfpathlineto{\pgfqpoint{2.658051in}{1.893201in}}%
\pgfpathlineto{\pgfqpoint{2.617005in}{1.893201in}}%
\pgfpathlineto{\pgfqpoint{2.575960in}{1.893201in}}%
\pgfpathlineto{\pgfqpoint{2.534915in}{1.893201in}}%
\pgfpathlineto{\pgfqpoint{2.528792in}{1.909363in}}%
\pgfpathlineto{\pgfqpoint{2.522591in}{1.929991in}}%
\pgfpathlineto{\pgfqpoint{2.517628in}{1.950619in}}%
\pgfpathlineto{\pgfqpoint{2.513506in}{1.971247in}}%
\pgfpathlineto{\pgfqpoint{2.509982in}{1.991874in}}%
\pgfpathlineto{\pgfqpoint{2.506897in}{2.012502in}}%
\pgfpathlineto{\pgfqpoint{2.504146in}{2.033130in}}%
\pgfpathlineto{\pgfqpoint{2.501653in}{2.053758in}}%
\pgfpathlineto{\pgfqpoint{2.499364in}{2.074386in}}%
\pgfpathlineto{\pgfqpoint{2.497239in}{2.095013in}}%
\pgfpathlineto{\pgfqpoint{2.495249in}{2.115641in}}%
\pgfpathlineto{\pgfqpoint{2.493869in}{2.130540in}}%
\pgfpathlineto{\pgfqpoint{2.487625in}{2.136269in}}%
\pgfpathlineto{\pgfqpoint{2.463557in}{2.156897in}}%
\pgfpathlineto{\pgfqpoint{2.452824in}{2.165521in}}%
\pgfpathlineto{\pgfqpoint{2.444846in}{2.177525in}}%
\pgfpathlineto{\pgfqpoint{2.430491in}{2.198153in}}%
\pgfpathlineto{\pgfqpoint{2.415542in}{2.218780in}}%
\pgfpathlineto{\pgfqpoint{2.411779in}{2.223754in}}%
\pgfpathlineto{\pgfqpoint{2.402631in}{2.239408in}}%
\pgfpathlineto{\pgfqpoint{2.390193in}{2.260036in}}%
\pgfpathlineto{\pgfqpoint{2.377410in}{2.280664in}}%
\pgfpathlineto{\pgfqpoint{2.370733in}{2.291114in}}%
\pgfpathlineto{\pgfqpoint{2.364648in}{2.301292in}}%
\pgfpathlineto{\pgfqpoint{2.351993in}{2.321920in}}%
\pgfpathlineto{\pgfqpoint{2.339102in}{2.342547in}}%
\pgfpathlineto{\pgfqpoint{2.329688in}{2.357300in}}%
\pgfpathlineto{\pgfqpoint{2.325642in}{2.363175in}}%
\pgfpathlineto{\pgfqpoint{2.311116in}{2.383803in}}%
\pgfpathlineto{\pgfqpoint{2.296422in}{2.404431in}}%
\pgfpathlineto{\pgfqpoint{2.288643in}{2.415163in}}%
\pgfpathclose%
\pgfusepath{stroke,fill}%
\end{pgfscope}%
\begin{pgfscope}%
\pgfpathrectangle{\pgfqpoint{0.605784in}{0.382904in}}{\pgfqpoint{4.063488in}{2.042155in}}%
\pgfusepath{clip}%
\pgfsetbuttcap%
\pgfsetroundjoin%
\definecolor{currentfill}{rgb}{0.195860,0.395433,0.555276}%
\pgfsetfillcolor{currentfill}%
\pgfsetlinewidth{1.003750pt}%
\definecolor{currentstroke}{rgb}{0.195860,0.395433,0.555276}%
\pgfsetstrokecolor{currentstroke}%
\pgfsetdash{}{0pt}%
\pgfpathmoveto{\pgfqpoint{0.641478in}{0.568554in}}%
\pgfpathlineto{\pgfqpoint{0.605784in}{0.583374in}}%
\pgfpathlineto{\pgfqpoint{0.605784in}{0.589182in}}%
\pgfpathlineto{\pgfqpoint{0.605784in}{0.609810in}}%
\pgfpathlineto{\pgfqpoint{0.605784in}{0.630438in}}%
\pgfpathlineto{\pgfqpoint{0.605784in}{0.651066in}}%
\pgfpathlineto{\pgfqpoint{0.605784in}{0.666235in}}%
\pgfpathlineto{\pgfqpoint{0.643201in}{0.651066in}}%
\pgfpathlineto{\pgfqpoint{0.646829in}{0.649596in}}%
\pgfpathlineto{\pgfqpoint{0.687875in}{0.642148in}}%
\pgfpathlineto{\pgfqpoint{0.728920in}{0.643688in}}%
\pgfpathlineto{\pgfqpoint{0.763234in}{0.651066in}}%
\pgfpathlineto{\pgfqpoint{0.769965in}{0.652572in}}%
\pgfpathlineto{\pgfqpoint{0.811011in}{0.666637in}}%
\pgfpathlineto{\pgfqpoint{0.823007in}{0.671694in}}%
\pgfpathlineto{\pgfqpoint{0.852056in}{0.684014in}}%
\pgfpathlineto{\pgfqpoint{0.868684in}{0.692321in}}%
\pgfpathlineto{\pgfqpoint{0.893101in}{0.704386in}}%
\pgfpathlineto{\pgfqpoint{0.907284in}{0.712949in}}%
\pgfpathlineto{\pgfqpoint{0.934147in}{0.728758in}}%
\pgfpathlineto{\pgfqpoint{0.940620in}{0.733577in}}%
\pgfpathlineto{\pgfqpoint{0.969050in}{0.754205in}}%
\pgfpathlineto{\pgfqpoint{0.975192in}{0.758548in}}%
\pgfpathlineto{\pgfqpoint{0.993472in}{0.774833in}}%
\pgfpathlineto{\pgfqpoint{1.016237in}{0.794270in}}%
\pgfpathlineto{\pgfqpoint{1.017406in}{0.795460in}}%
\pgfpathlineto{\pgfqpoint{1.037803in}{0.816088in}}%
\pgfpathlineto{\pgfqpoint{1.057283in}{0.834984in}}%
\pgfpathlineto{\pgfqpoint{1.058905in}{0.836716in}}%
\pgfpathlineto{\pgfqpoint{1.078311in}{0.857344in}}%
\pgfpathlineto{\pgfqpoint{1.098328in}{0.877839in}}%
\pgfpathlineto{\pgfqpoint{1.098458in}{0.877972in}}%
\pgfpathlineto{\pgfqpoint{1.118802in}{0.898600in}}%
\pgfpathlineto{\pgfqpoint{1.139373in}{0.918797in}}%
\pgfpathlineto{\pgfqpoint{1.139881in}{0.919227in}}%
\pgfpathlineto{\pgfqpoint{1.164192in}{0.939855in}}%
\pgfpathlineto{\pgfqpoint{1.180419in}{0.953349in}}%
\pgfpathlineto{\pgfqpoint{1.192483in}{0.960483in}}%
\pgfpathlineto{\pgfqpoint{1.221464in}{0.977516in}}%
\pgfpathlineto{\pgfqpoint{1.233869in}{0.981111in}}%
\pgfpathlineto{\pgfqpoint{1.262509in}{0.989493in}}%
\pgfpathlineto{\pgfqpoint{1.303555in}{0.989739in}}%
\pgfpathlineto{\pgfqpoint{1.344600in}{0.981175in}}%
\pgfpathlineto{\pgfqpoint{1.344807in}{0.981111in}}%
\pgfpathlineto{\pgfqpoint{1.385645in}{0.968621in}}%
\pgfpathlineto{\pgfqpoint{1.413976in}{0.960483in}}%
\pgfpathlineto{\pgfqpoint{1.426691in}{0.956845in}}%
\pgfpathlineto{\pgfqpoint{1.467736in}{0.950359in}}%
\pgfpathlineto{\pgfqpoint{1.508781in}{0.951763in}}%
\pgfpathlineto{\pgfqpoint{1.545781in}{0.960483in}}%
\pgfpathlineto{\pgfqpoint{1.549827in}{0.961498in}}%
\pgfpathlineto{\pgfqpoint{1.590872in}{0.977843in}}%
\pgfpathlineto{\pgfqpoint{1.598105in}{0.981111in}}%
\pgfpathlineto{\pgfqpoint{1.631917in}{0.996995in}}%
\pgfpathlineto{\pgfqpoint{1.642894in}{1.001739in}}%
\pgfpathlineto{\pgfqpoint{1.672963in}{1.015013in}}%
\pgfpathlineto{\pgfqpoint{1.693562in}{1.022367in}}%
\pgfpathlineto{\pgfqpoint{1.714008in}{1.029677in}}%
\pgfpathlineto{\pgfqpoint{1.755053in}{1.041687in}}%
\pgfpathlineto{\pgfqpoint{1.759228in}{1.042994in}}%
\pgfpathlineto{\pgfqpoint{1.796099in}{1.054194in}}%
\pgfpathlineto{\pgfqpoint{1.819930in}{1.063622in}}%
\pgfpathlineto{\pgfqpoint{1.837144in}{1.070163in}}%
\pgfpathlineto{\pgfqpoint{1.863859in}{1.084250in}}%
\pgfpathlineto{\pgfqpoint{1.878189in}{1.091479in}}%
\pgfpathlineto{\pgfqpoint{1.898618in}{1.104878in}}%
\pgfpathlineto{\pgfqpoint{1.919235in}{1.117833in}}%
\pgfpathlineto{\pgfqpoint{1.929760in}{1.125506in}}%
\pgfpathlineto{\pgfqpoint{1.959066in}{1.146134in}}%
\pgfpathlineto{\pgfqpoint{1.960280in}{1.146971in}}%
\pgfpathlineto{\pgfqpoint{1.988296in}{1.166761in}}%
\pgfpathlineto{\pgfqpoint{2.001325in}{1.175707in}}%
\pgfpathlineto{\pgfqpoint{2.020695in}{1.187389in}}%
\pgfpathlineto{\pgfqpoint{2.042371in}{1.200255in}}%
\pgfpathlineto{\pgfqpoint{2.060348in}{1.208017in}}%
\pgfpathlineto{\pgfqpoint{2.083416in}{1.217959in}}%
\pgfpathlineto{\pgfqpoint{2.124461in}{1.227705in}}%
\pgfpathlineto{\pgfqpoint{2.137653in}{1.228645in}}%
\pgfpathlineto{\pgfqpoint{2.165507in}{1.230685in}}%
\pgfpathlineto{\pgfqpoint{2.206552in}{1.229703in}}%
\pgfpathlineto{\pgfqpoint{2.247597in}{1.228845in}}%
\pgfpathlineto{\pgfqpoint{2.288643in}{1.232375in}}%
\pgfpathlineto{\pgfqpoint{2.329688in}{1.243505in}}%
\pgfpathlineto{\pgfqpoint{2.341992in}{1.249273in}}%
\pgfpathlineto{\pgfqpoint{2.370733in}{1.263674in}}%
\pgfpathlineto{\pgfqpoint{2.380358in}{1.269900in}}%
\pgfpathlineto{\pgfqpoint{2.410327in}{1.290528in}}%
\pgfpathlineto{\pgfqpoint{2.411779in}{1.291588in}}%
\pgfpathlineto{\pgfqpoint{2.437727in}{1.311156in}}%
\pgfpathlineto{\pgfqpoint{2.452824in}{1.323185in}}%
\pgfpathlineto{\pgfqpoint{2.465628in}{1.331784in}}%
\pgfpathlineto{\pgfqpoint{2.493869in}{1.351500in}}%
\pgfpathlineto{\pgfqpoint{2.495934in}{1.331784in}}%
\pgfpathlineto{\pgfqpoint{2.497998in}{1.311156in}}%
\pgfpathlineto{\pgfqpoint{2.499974in}{1.290528in}}%
\pgfpathlineto{\pgfqpoint{2.501873in}{1.269900in}}%
\pgfpathlineto{\pgfqpoint{2.503704in}{1.249273in}}%
\pgfpathlineto{\pgfqpoint{2.505475in}{1.228645in}}%
\pgfpathlineto{\pgfqpoint{2.507192in}{1.208017in}}%
\pgfpathlineto{\pgfqpoint{2.508860in}{1.187389in}}%
\pgfpathlineto{\pgfqpoint{2.510485in}{1.166761in}}%
\pgfpathlineto{\pgfqpoint{2.512071in}{1.146134in}}%
\pgfpathlineto{\pgfqpoint{2.513622in}{1.125506in}}%
\pgfpathlineto{\pgfqpoint{2.515140in}{1.104878in}}%
\pgfpathlineto{\pgfqpoint{2.516630in}{1.084250in}}%
\pgfpathlineto{\pgfqpoint{2.518092in}{1.063622in}}%
\pgfpathlineto{\pgfqpoint{2.519530in}{1.042994in}}%
\pgfpathlineto{\pgfqpoint{2.520946in}{1.022367in}}%
\pgfpathlineto{\pgfqpoint{2.522340in}{1.001739in}}%
\pgfpathlineto{\pgfqpoint{2.523716in}{0.981111in}}%
\pgfpathlineto{\pgfqpoint{2.525074in}{0.960483in}}%
\pgfpathlineto{\pgfqpoint{2.526416in}{0.939855in}}%
\pgfpathlineto{\pgfqpoint{2.527743in}{0.919227in}}%
\pgfpathlineto{\pgfqpoint{2.529055in}{0.898600in}}%
\pgfpathlineto{\pgfqpoint{2.530355in}{0.877972in}}%
\pgfpathlineto{\pgfqpoint{2.531642in}{0.857344in}}%
\pgfpathlineto{\pgfqpoint{2.532917in}{0.836716in}}%
\pgfpathlineto{\pgfqpoint{2.534183in}{0.816088in}}%
\pgfpathlineto{\pgfqpoint{2.534915in}{0.804162in}}%
\pgfpathlineto{\pgfqpoint{2.575960in}{0.804162in}}%
\pgfpathlineto{\pgfqpoint{2.617005in}{0.804162in}}%
\pgfpathlineto{\pgfqpoint{2.658051in}{0.804162in}}%
\pgfpathlineto{\pgfqpoint{2.699096in}{0.804162in}}%
\pgfpathlineto{\pgfqpoint{2.740141in}{0.804162in}}%
\pgfpathlineto{\pgfqpoint{2.781187in}{0.804162in}}%
\pgfpathlineto{\pgfqpoint{2.822232in}{0.804162in}}%
\pgfpathlineto{\pgfqpoint{2.863277in}{0.804162in}}%
\pgfpathlineto{\pgfqpoint{2.904323in}{0.804162in}}%
\pgfpathlineto{\pgfqpoint{2.945368in}{0.804162in}}%
\pgfpathlineto{\pgfqpoint{2.986413in}{0.804162in}}%
\pgfpathlineto{\pgfqpoint{3.027459in}{0.804162in}}%
\pgfpathlineto{\pgfqpoint{3.068504in}{0.804162in}}%
\pgfpathlineto{\pgfqpoint{3.109549in}{0.804162in}}%
\pgfpathlineto{\pgfqpoint{3.150595in}{0.804162in}}%
\pgfpathlineto{\pgfqpoint{3.191640in}{0.804162in}}%
\pgfpathlineto{\pgfqpoint{3.232685in}{0.804162in}}%
\pgfpathlineto{\pgfqpoint{3.273731in}{0.804162in}}%
\pgfpathlineto{\pgfqpoint{3.314776in}{0.804162in}}%
\pgfpathlineto{\pgfqpoint{3.355821in}{0.804162in}}%
\pgfpathlineto{\pgfqpoint{3.396867in}{0.804162in}}%
\pgfpathlineto{\pgfqpoint{3.437912in}{0.804162in}}%
\pgfpathlineto{\pgfqpoint{3.478957in}{0.804162in}}%
\pgfpathlineto{\pgfqpoint{3.520003in}{0.804162in}}%
\pgfpathlineto{\pgfqpoint{3.561048in}{0.804162in}}%
\pgfpathlineto{\pgfqpoint{3.602093in}{0.804162in}}%
\pgfpathlineto{\pgfqpoint{3.643139in}{0.804162in}}%
\pgfpathlineto{\pgfqpoint{3.645107in}{0.795460in}}%
\pgfpathlineto{\pgfqpoint{3.649662in}{0.774833in}}%
\pgfpathlineto{\pgfqpoint{3.654017in}{0.754205in}}%
\pgfpathlineto{\pgfqpoint{3.658193in}{0.733577in}}%
\pgfpathlineto{\pgfqpoint{3.662210in}{0.712949in}}%
\pgfpathlineto{\pgfqpoint{3.666085in}{0.692321in}}%
\pgfpathlineto{\pgfqpoint{3.669830in}{0.671694in}}%
\pgfpathlineto{\pgfqpoint{3.673460in}{0.651066in}}%
\pgfpathlineto{\pgfqpoint{3.676984in}{0.630438in}}%
\pgfpathlineto{\pgfqpoint{3.680412in}{0.609810in}}%
\pgfpathlineto{\pgfqpoint{3.683754in}{0.589182in}}%
\pgfpathlineto{\pgfqpoint{3.684184in}{0.586530in}}%
\pgfpathlineto{\pgfqpoint{3.725229in}{0.586530in}}%
\pgfpathlineto{\pgfqpoint{3.766275in}{0.586530in}}%
\pgfpathlineto{\pgfqpoint{3.807320in}{0.586530in}}%
\pgfpathlineto{\pgfqpoint{3.848365in}{0.586530in}}%
\pgfpathlineto{\pgfqpoint{3.889411in}{0.586530in}}%
\pgfpathlineto{\pgfqpoint{3.930456in}{0.586530in}}%
\pgfpathlineto{\pgfqpoint{3.971501in}{0.586530in}}%
\pgfpathlineto{\pgfqpoint{4.012547in}{0.586530in}}%
\pgfpathlineto{\pgfqpoint{4.053592in}{0.586530in}}%
\pgfpathlineto{\pgfqpoint{4.094637in}{0.586530in}}%
\pgfpathlineto{\pgfqpoint{4.135683in}{0.586530in}}%
\pgfpathlineto{\pgfqpoint{4.176728in}{0.586530in}}%
\pgfpathlineto{\pgfqpoint{4.217773in}{0.586530in}}%
\pgfpathlineto{\pgfqpoint{4.258819in}{0.586530in}}%
\pgfpathlineto{\pgfqpoint{4.299864in}{0.586530in}}%
\pgfpathlineto{\pgfqpoint{4.340909in}{0.586530in}}%
\pgfpathlineto{\pgfqpoint{4.381955in}{0.586530in}}%
\pgfpathlineto{\pgfqpoint{4.423000in}{0.586530in}}%
\pgfpathlineto{\pgfqpoint{4.464045in}{0.586530in}}%
\pgfpathlineto{\pgfqpoint{4.505091in}{0.586530in}}%
\pgfpathlineto{\pgfqpoint{4.546136in}{0.586530in}}%
\pgfpathlineto{\pgfqpoint{4.587181in}{0.586530in}}%
\pgfpathlineto{\pgfqpoint{4.628227in}{0.586530in}}%
\pgfpathlineto{\pgfqpoint{4.669272in}{0.586530in}}%
\pgfpathlineto{\pgfqpoint{4.669272in}{0.568554in}}%
\pgfpathlineto{\pgfqpoint{4.669272in}{0.547927in}}%
\pgfpathlineto{\pgfqpoint{4.669272in}{0.527299in}}%
\pgfpathlineto{\pgfqpoint{4.669272in}{0.506671in}}%
\pgfpathlineto{\pgfqpoint{4.669272in}{0.500154in}}%
\pgfpathlineto{\pgfqpoint{4.628227in}{0.500154in}}%
\pgfpathlineto{\pgfqpoint{4.587181in}{0.500154in}}%
\pgfpathlineto{\pgfqpoint{4.546136in}{0.500154in}}%
\pgfpathlineto{\pgfqpoint{4.505091in}{0.500154in}}%
\pgfpathlineto{\pgfqpoint{4.464045in}{0.500154in}}%
\pgfpathlineto{\pgfqpoint{4.423000in}{0.500154in}}%
\pgfpathlineto{\pgfqpoint{4.381955in}{0.500154in}}%
\pgfpathlineto{\pgfqpoint{4.340909in}{0.500154in}}%
\pgfpathlineto{\pgfqpoint{4.299864in}{0.500154in}}%
\pgfpathlineto{\pgfqpoint{4.258819in}{0.500154in}}%
\pgfpathlineto{\pgfqpoint{4.217773in}{0.500154in}}%
\pgfpathlineto{\pgfqpoint{4.176728in}{0.500154in}}%
\pgfpathlineto{\pgfqpoint{4.135683in}{0.500154in}}%
\pgfpathlineto{\pgfqpoint{4.094637in}{0.500154in}}%
\pgfpathlineto{\pgfqpoint{4.053592in}{0.500154in}}%
\pgfpathlineto{\pgfqpoint{4.012547in}{0.500154in}}%
\pgfpathlineto{\pgfqpoint{3.971501in}{0.500154in}}%
\pgfpathlineto{\pgfqpoint{3.930456in}{0.500154in}}%
\pgfpathlineto{\pgfqpoint{3.889411in}{0.500154in}}%
\pgfpathlineto{\pgfqpoint{3.848365in}{0.500154in}}%
\pgfpathlineto{\pgfqpoint{3.807320in}{0.500154in}}%
\pgfpathlineto{\pgfqpoint{3.766275in}{0.500154in}}%
\pgfpathlineto{\pgfqpoint{3.725229in}{0.500154in}}%
\pgfpathlineto{\pgfqpoint{3.684184in}{0.500154in}}%
\pgfpathlineto{\pgfqpoint{3.683109in}{0.506671in}}%
\pgfpathlineto{\pgfqpoint{3.679691in}{0.527299in}}%
\pgfpathlineto{\pgfqpoint{3.676191in}{0.547927in}}%
\pgfpathlineto{\pgfqpoint{3.672602in}{0.568554in}}%
\pgfpathlineto{\pgfqpoint{3.668916in}{0.589182in}}%
\pgfpathlineto{\pgfqpoint{3.665125in}{0.609810in}}%
\pgfpathlineto{\pgfqpoint{3.661219in}{0.630438in}}%
\pgfpathlineto{\pgfqpoint{3.657186in}{0.651066in}}%
\pgfpathlineto{\pgfqpoint{3.653015in}{0.671694in}}%
\pgfpathlineto{\pgfqpoint{3.648689in}{0.692321in}}%
\pgfpathlineto{\pgfqpoint{3.644194in}{0.712949in}}%
\pgfpathlineto{\pgfqpoint{3.643139in}{0.717728in}}%
\pgfpathlineto{\pgfqpoint{3.602093in}{0.717728in}}%
\pgfpathlineto{\pgfqpoint{3.561048in}{0.717728in}}%
\pgfpathlineto{\pgfqpoint{3.520003in}{0.717728in}}%
\pgfpathlineto{\pgfqpoint{3.478957in}{0.717728in}}%
\pgfpathlineto{\pgfqpoint{3.437912in}{0.717728in}}%
\pgfpathlineto{\pgfqpoint{3.396867in}{0.717728in}}%
\pgfpathlineto{\pgfqpoint{3.355821in}{0.717728in}}%
\pgfpathlineto{\pgfqpoint{3.314776in}{0.717728in}}%
\pgfpathlineto{\pgfqpoint{3.273731in}{0.717728in}}%
\pgfpathlineto{\pgfqpoint{3.232685in}{0.717728in}}%
\pgfpathlineto{\pgfqpoint{3.191640in}{0.717728in}}%
\pgfpathlineto{\pgfqpoint{3.150595in}{0.717728in}}%
\pgfpathlineto{\pgfqpoint{3.109549in}{0.717728in}}%
\pgfpathlineto{\pgfqpoint{3.068504in}{0.717728in}}%
\pgfpathlineto{\pgfqpoint{3.027459in}{0.717728in}}%
\pgfpathlineto{\pgfqpoint{2.986413in}{0.717728in}}%
\pgfpathlineto{\pgfqpoint{2.945368in}{0.717728in}}%
\pgfpathlineto{\pgfqpoint{2.904323in}{0.717728in}}%
\pgfpathlineto{\pgfqpoint{2.863277in}{0.717728in}}%
\pgfpathlineto{\pgfqpoint{2.822232in}{0.717728in}}%
\pgfpathlineto{\pgfqpoint{2.781187in}{0.717728in}}%
\pgfpathlineto{\pgfqpoint{2.740141in}{0.717728in}}%
\pgfpathlineto{\pgfqpoint{2.699096in}{0.717728in}}%
\pgfpathlineto{\pgfqpoint{2.658051in}{0.717728in}}%
\pgfpathlineto{\pgfqpoint{2.617005in}{0.717728in}}%
\pgfpathlineto{\pgfqpoint{2.575960in}{0.717728in}}%
\pgfpathlineto{\pgfqpoint{2.534915in}{0.717728in}}%
\pgfpathlineto{\pgfqpoint{2.533894in}{0.733577in}}%
\pgfpathlineto{\pgfqpoint{2.532560in}{0.754205in}}%
\pgfpathlineto{\pgfqpoint{2.531213in}{0.774833in}}%
\pgfpathlineto{\pgfqpoint{2.529852in}{0.795460in}}%
\pgfpathlineto{\pgfqpoint{2.528478in}{0.816088in}}%
\pgfpathlineto{\pgfqpoint{2.527088in}{0.836716in}}%
\pgfpathlineto{\pgfqpoint{2.525681in}{0.857344in}}%
\pgfpathlineto{\pgfqpoint{2.524258in}{0.877972in}}%
\pgfpathlineto{\pgfqpoint{2.522815in}{0.898600in}}%
\pgfpathlineto{\pgfqpoint{2.521353in}{0.919227in}}%
\pgfpathlineto{\pgfqpoint{2.519869in}{0.939855in}}%
\pgfpathlineto{\pgfqpoint{2.518362in}{0.960483in}}%
\pgfpathlineto{\pgfqpoint{2.516830in}{0.981111in}}%
\pgfpathlineto{\pgfqpoint{2.515271in}{1.001739in}}%
\pgfpathlineto{\pgfqpoint{2.513684in}{1.022367in}}%
\pgfpathlineto{\pgfqpoint{2.512064in}{1.042994in}}%
\pgfpathlineto{\pgfqpoint{2.510411in}{1.063622in}}%
\pgfpathlineto{\pgfqpoint{2.508720in}{1.084250in}}%
\pgfpathlineto{\pgfqpoint{2.506988in}{1.104878in}}%
\pgfpathlineto{\pgfqpoint{2.505212in}{1.125506in}}%
\pgfpathlineto{\pgfqpoint{2.503387in}{1.146134in}}%
\pgfpathlineto{\pgfqpoint{2.501508in}{1.166761in}}%
\pgfpathlineto{\pgfqpoint{2.499569in}{1.187389in}}%
\pgfpathlineto{\pgfqpoint{2.497565in}{1.208017in}}%
\pgfpathlineto{\pgfqpoint{2.495487in}{1.228645in}}%
\pgfpathlineto{\pgfqpoint{2.493869in}{1.244237in}}%
\pgfpathlineto{\pgfqpoint{2.467049in}{1.228645in}}%
\pgfpathlineto{\pgfqpoint{2.452824in}{1.220635in}}%
\pgfpathlineto{\pgfqpoint{2.432731in}{1.208017in}}%
\pgfpathlineto{\pgfqpoint{2.411779in}{1.195447in}}%
\pgfpathlineto{\pgfqpoint{2.397154in}{1.187389in}}%
\pgfpathlineto{\pgfqpoint{2.370733in}{1.173607in}}%
\pgfpathlineto{\pgfqpoint{2.352716in}{1.166761in}}%
\pgfpathlineto{\pgfqpoint{2.329688in}{1.158510in}}%
\pgfpathlineto{\pgfqpoint{2.288643in}{1.151105in}}%
\pgfpathlineto{\pgfqpoint{2.247597in}{1.150083in}}%
\pgfpathlineto{\pgfqpoint{2.206552in}{1.152382in}}%
\pgfpathlineto{\pgfqpoint{2.165507in}{1.154029in}}%
\pgfpathlineto{\pgfqpoint{2.124461in}{1.151205in}}%
\pgfpathlineto{\pgfqpoint{2.103739in}{1.146134in}}%
\pgfpathlineto{\pgfqpoint{2.083416in}{1.141219in}}%
\pgfpathlineto{\pgfqpoint{2.047856in}{1.125506in}}%
\pgfpathlineto{\pgfqpoint{2.042371in}{1.123078in}}%
\pgfpathlineto{\pgfqpoint{2.012637in}{1.104878in}}%
\pgfpathlineto{\pgfqpoint{2.001325in}{1.097854in}}%
\pgfpathlineto{\pgfqpoint{1.982338in}{1.084250in}}%
\pgfpathlineto{\pgfqpoint{1.960280in}{1.068042in}}%
\pgfpathlineto{\pgfqpoint{1.954273in}{1.063622in}}%
\pgfpathlineto{\pgfqpoint{1.926848in}{1.042994in}}%
\pgfpathlineto{\pgfqpoint{1.919235in}{1.037128in}}%
\pgfpathlineto{\pgfqpoint{1.897809in}{1.022367in}}%
\pgfpathlineto{\pgfqpoint{1.878189in}{1.008327in}}%
\pgfpathlineto{\pgfqpoint{1.866851in}{1.001739in}}%
\pgfpathlineto{\pgfqpoint{1.837144in}{0.983790in}}%
\pgfpathlineto{\pgfqpoint{1.831379in}{0.981111in}}%
\pgfpathlineto{\pgfqpoint{1.796099in}{0.964124in}}%
\pgfpathlineto{\pgfqpoint{1.786573in}{0.960483in}}%
\pgfpathlineto{\pgfqpoint{1.755053in}{0.948106in}}%
\pgfpathlineto{\pgfqpoint{1.731911in}{0.939855in}}%
\pgfpathlineto{\pgfqpoint{1.714008in}{0.933381in}}%
\pgfpathlineto{\pgfqpoint{1.676563in}{0.919227in}}%
\pgfpathlineto{\pgfqpoint{1.672963in}{0.917868in}}%
\pgfpathlineto{\pgfqpoint{1.631917in}{0.901221in}}%
\pgfpathlineto{\pgfqpoint{1.625449in}{0.898600in}}%
\pgfpathlineto{\pgfqpoint{1.590872in}{0.885055in}}%
\pgfpathlineto{\pgfqpoint{1.568559in}{0.877972in}}%
\pgfpathlineto{\pgfqpoint{1.549827in}{0.872291in}}%
\pgfpathlineto{\pgfqpoint{1.508781in}{0.865815in}}%
\pgfpathlineto{\pgfqpoint{1.467736in}{0.866971in}}%
\pgfpathlineto{\pgfqpoint{1.426691in}{0.875071in}}%
\pgfpathlineto{\pgfqpoint{1.417145in}{0.877972in}}%
\pgfpathlineto{\pgfqpoint{1.385645in}{0.887577in}}%
\pgfpathlineto{\pgfqpoint{1.349731in}{0.898600in}}%
\pgfpathlineto{\pgfqpoint{1.344600in}{0.900191in}}%
\pgfpathlineto{\pgfqpoint{1.303555in}{0.908372in}}%
\pgfpathlineto{\pgfqpoint{1.262509in}{0.907694in}}%
\pgfpathlineto{\pgfqpoint{1.232099in}{0.898600in}}%
\pgfpathlineto{\pgfqpoint{1.221464in}{0.895447in}}%
\pgfpathlineto{\pgfqpoint{1.191781in}{0.877972in}}%
\pgfpathlineto{\pgfqpoint{1.180419in}{0.871247in}}%
\pgfpathlineto{\pgfqpoint{1.163675in}{0.857344in}}%
\pgfpathlineto{\pgfqpoint{1.139373in}{0.836790in}}%
\pgfpathlineto{\pgfqpoint{1.139298in}{0.836716in}}%
\pgfpathlineto{\pgfqpoint{1.118337in}{0.816088in}}%
\pgfpathlineto{\pgfqpoint{1.098328in}{0.795826in}}%
\pgfpathlineto{\pgfqpoint{1.097974in}{0.795460in}}%
\pgfpathlineto{\pgfqpoint{1.078054in}{0.774833in}}%
\pgfpathlineto{\pgfqpoint{1.058775in}{0.754205in}}%
\pgfpathlineto{\pgfqpoint{1.057283in}{0.752602in}}%
\pgfpathlineto{\pgfqpoint{1.038120in}{0.733577in}}%
\pgfpathlineto{\pgfqpoint{1.018071in}{0.712949in}}%
\pgfpathlineto{\pgfqpoint{1.016237in}{0.711049in}}%
\pgfpathlineto{\pgfqpoint{0.995124in}{0.692321in}}%
\pgfpathlineto{\pgfqpoint{0.975192in}{0.673966in}}%
\pgfpathlineto{\pgfqpoint{0.972150in}{0.671694in}}%
\pgfpathlineto{\pgfqpoint{0.944973in}{0.651066in}}%
\pgfpathlineto{\pgfqpoint{0.934147in}{0.642622in}}%
\pgfpathlineto{\pgfqpoint{0.914586in}{0.630438in}}%
\pgfpathlineto{\pgfqpoint{0.893101in}{0.616752in}}%
\pgfpathlineto{\pgfqpoint{0.879607in}{0.609810in}}%
\pgfpathlineto{\pgfqpoint{0.852056in}{0.595499in}}%
\pgfpathlineto{\pgfqpoint{0.837121in}{0.589182in}}%
\pgfpathlineto{\pgfqpoint{0.811011in}{0.578197in}}%
\pgfpathlineto{\pgfqpoint{0.780958in}{0.568554in}}%
\pgfpathlineto{\pgfqpoint{0.769965in}{0.565099in}}%
\pgfpathlineto{\pgfqpoint{0.728920in}{0.557713in}}%
\pgfpathlineto{\pgfqpoint{0.687875in}{0.557731in}}%
\pgfpathlineto{\pgfqpoint{0.646829in}{0.566334in}}%
\pgfpathclose%
\pgfusepath{stroke,fill}%
\end{pgfscope}%
\begin{pgfscope}%
\pgfpathrectangle{\pgfqpoint{0.605784in}{0.382904in}}{\pgfqpoint{4.063488in}{2.042155in}}%
\pgfusepath{clip}%
\pgfsetbuttcap%
\pgfsetroundjoin%
\definecolor{currentfill}{rgb}{0.195860,0.395433,0.555276}%
\pgfsetfillcolor{currentfill}%
\pgfsetlinewidth{1.003750pt}%
\definecolor{currentstroke}{rgb}{0.195860,0.395433,0.555276}%
\pgfsetstrokecolor{currentstroke}%
\pgfsetdash{}{0pt}%
\pgfpathmoveto{\pgfqpoint{0.615291in}{1.909363in}}%
\pgfpathlineto{\pgfqpoint{0.605784in}{1.914822in}}%
\pgfpathlineto{\pgfqpoint{0.605784in}{1.929991in}}%
\pgfpathlineto{\pgfqpoint{0.605784in}{1.950619in}}%
\pgfpathlineto{\pgfqpoint{0.605784in}{1.971247in}}%
\pgfpathlineto{\pgfqpoint{0.605784in}{1.991874in}}%
\pgfpathlineto{\pgfqpoint{0.605784in}{1.997683in}}%
\pgfpathlineto{\pgfqpoint{0.616074in}{1.991874in}}%
\pgfpathlineto{\pgfqpoint{0.646829in}{1.974503in}}%
\pgfpathlineto{\pgfqpoint{0.652131in}{1.971247in}}%
\pgfpathlineto{\pgfqpoint{0.685495in}{1.950619in}}%
\pgfpathlineto{\pgfqpoint{0.687875in}{1.949135in}}%
\pgfpathlineto{\pgfqpoint{0.723245in}{1.929991in}}%
\pgfpathlineto{\pgfqpoint{0.728920in}{1.926876in}}%
\pgfpathlineto{\pgfqpoint{0.769965in}{1.913499in}}%
\pgfpathlineto{\pgfqpoint{0.811011in}{1.913848in}}%
\pgfpathlineto{\pgfqpoint{0.850462in}{1.929991in}}%
\pgfpathlineto{\pgfqpoint{0.852056in}{1.930640in}}%
\pgfpathlineto{\pgfqpoint{0.877155in}{1.950619in}}%
\pgfpathlineto{\pgfqpoint{0.893101in}{1.963434in}}%
\pgfpathlineto{\pgfqpoint{0.900138in}{1.971247in}}%
\pgfpathlineto{\pgfqpoint{0.918669in}{1.991874in}}%
\pgfpathlineto{\pgfqpoint{0.934147in}{2.009334in}}%
\pgfpathlineto{\pgfqpoint{0.936579in}{2.012502in}}%
\pgfpathlineto{\pgfqpoint{0.952456in}{2.033130in}}%
\pgfpathlineto{\pgfqpoint{0.968056in}{2.053758in}}%
\pgfpathlineto{\pgfqpoint{0.975192in}{2.063234in}}%
\pgfpathlineto{\pgfqpoint{0.983405in}{2.074386in}}%
\pgfpathlineto{\pgfqpoint{0.998466in}{2.095013in}}%
\pgfpathlineto{\pgfqpoint{1.013242in}{2.115641in}}%
\pgfpathlineto{\pgfqpoint{1.016237in}{2.119821in}}%
\pgfpathlineto{\pgfqpoint{1.028870in}{2.136269in}}%
\pgfpathlineto{\pgfqpoint{1.044439in}{2.156897in}}%
\pgfpathlineto{\pgfqpoint{1.057283in}{2.174218in}}%
\pgfpathlineto{\pgfqpoint{1.060112in}{2.177525in}}%
\pgfpathlineto{\pgfqpoint{1.077706in}{2.198153in}}%
\pgfpathlineto{\pgfqpoint{1.094853in}{2.218780in}}%
\pgfpathlineto{\pgfqpoint{1.098328in}{2.222975in}}%
\pgfpathlineto{\pgfqpoint{1.114928in}{2.239408in}}%
\pgfpathlineto{\pgfqpoint{1.135292in}{2.260036in}}%
\pgfpathlineto{\pgfqpoint{1.139373in}{2.264193in}}%
\pgfpathlineto{\pgfqpoint{1.160052in}{2.280664in}}%
\pgfpathlineto{\pgfqpoint{1.180419in}{2.297186in}}%
\pgfpathlineto{\pgfqpoint{1.187252in}{2.301292in}}%
\pgfpathlineto{\pgfqpoint{1.221404in}{2.321920in}}%
\pgfpathlineto{\pgfqpoint{1.221464in}{2.321956in}}%
\pgfpathlineto{\pgfqpoint{1.262509in}{2.338255in}}%
\pgfpathlineto{\pgfqpoint{1.286060in}{2.342547in}}%
\pgfpathlineto{\pgfqpoint{1.303555in}{2.345663in}}%
\pgfpathlineto{\pgfqpoint{1.344600in}{2.343178in}}%
\pgfpathlineto{\pgfqpoint{1.346569in}{2.342547in}}%
\pgfpathlineto{\pgfqpoint{1.385645in}{2.329902in}}%
\pgfpathlineto{\pgfqpoint{1.399461in}{2.321920in}}%
\pgfpathlineto{\pgfqpoint{1.426691in}{2.306132in}}%
\pgfpathlineto{\pgfqpoint{1.432807in}{2.301292in}}%
\pgfpathlineto{\pgfqpoint{1.458704in}{2.280664in}}%
\pgfpathlineto{\pgfqpoint{1.467736in}{2.273436in}}%
\pgfpathlineto{\pgfqpoint{1.482276in}{2.260036in}}%
\pgfpathlineto{\pgfqpoint{1.504560in}{2.239408in}}%
\pgfpathlineto{\pgfqpoint{1.508781in}{2.235455in}}%
\pgfpathlineto{\pgfqpoint{1.526936in}{2.218780in}}%
\pgfpathlineto{\pgfqpoint{1.549224in}{2.198153in}}%
\pgfpathlineto{\pgfqpoint{1.549827in}{2.197585in}}%
\pgfpathlineto{\pgfqpoint{1.576432in}{2.177525in}}%
\pgfpathlineto{\pgfqpoint{1.590872in}{2.166442in}}%
\pgfpathlineto{\pgfqpoint{1.613928in}{2.156897in}}%
\pgfpathlineto{\pgfqpoint{1.631917in}{2.149208in}}%
\pgfpathlineto{\pgfqpoint{1.672963in}{2.151404in}}%
\pgfpathlineto{\pgfqpoint{1.682310in}{2.156897in}}%
\pgfpathlineto{\pgfqpoint{1.714008in}{2.175499in}}%
\pgfpathlineto{\pgfqpoint{1.715875in}{2.177525in}}%
\pgfpathlineto{\pgfqpoint{1.735053in}{2.198153in}}%
\pgfpathlineto{\pgfqpoint{1.754024in}{2.218780in}}%
\pgfpathlineto{\pgfqpoint{1.755053in}{2.219889in}}%
\pgfpathlineto{\pgfqpoint{1.768566in}{2.239408in}}%
\pgfpathlineto{\pgfqpoint{1.782736in}{2.260036in}}%
\pgfpathlineto{\pgfqpoint{1.796099in}{2.279631in}}%
\pgfpathlineto{\pgfqpoint{1.796718in}{2.280664in}}%
\pgfpathlineto{\pgfqpoint{1.809217in}{2.301292in}}%
\pgfpathlineto{\pgfqpoint{1.821603in}{2.321920in}}%
\pgfpathlineto{\pgfqpoint{1.833882in}{2.342547in}}%
\pgfpathlineto{\pgfqpoint{1.837144in}{2.347996in}}%
\pgfpathlineto{\pgfqpoint{1.846088in}{2.363175in}}%
\pgfpathlineto{\pgfqpoint{1.858178in}{2.383803in}}%
\pgfpathlineto{\pgfqpoint{1.870144in}{2.404431in}}%
\pgfpathlineto{\pgfqpoint{1.878189in}{2.418350in}}%
\pgfpathlineto{\pgfqpoint{1.882354in}{2.425059in}}%
\pgfpathlineto{\pgfqpoint{1.919235in}{2.425059in}}%
\pgfpathlineto{\pgfqpoint{1.933387in}{2.425059in}}%
\pgfpathlineto{\pgfqpoint{1.919239in}{2.404431in}}%
\pgfpathlineto{\pgfqpoint{1.919235in}{2.404424in}}%
\pgfpathlineto{\pgfqpoint{1.907170in}{2.383803in}}%
\pgfpathlineto{\pgfqpoint{1.894957in}{2.363175in}}%
\pgfpathlineto{\pgfqpoint{1.882588in}{2.342547in}}%
\pgfpathlineto{\pgfqpoint{1.878189in}{2.335197in}}%
\pgfpathlineto{\pgfqpoint{1.870840in}{2.321920in}}%
\pgfpathlineto{\pgfqpoint{1.859402in}{2.301292in}}%
\pgfpathlineto{\pgfqpoint{1.847852in}{2.280664in}}%
\pgfpathlineto{\pgfqpoint{1.837144in}{2.261700in}}%
\pgfpathlineto{\pgfqpoint{1.836194in}{2.260036in}}%
\pgfpathlineto{\pgfqpoint{1.824558in}{2.239408in}}%
\pgfpathlineto{\pgfqpoint{1.812833in}{2.218780in}}%
\pgfpathlineto{\pgfqpoint{1.801015in}{2.198153in}}%
\pgfpathlineto{\pgfqpoint{1.796099in}{2.189532in}}%
\pgfpathlineto{\pgfqpoint{1.788330in}{2.177525in}}%
\pgfpathlineto{\pgfqpoint{1.775011in}{2.156897in}}%
\pgfpathlineto{\pgfqpoint{1.761595in}{2.136269in}}%
\pgfpathlineto{\pgfqpoint{1.755053in}{2.126170in}}%
\pgfpathlineto{\pgfqpoint{1.745856in}{2.115641in}}%
\pgfpathlineto{\pgfqpoint{1.727872in}{2.095013in}}%
\pgfpathlineto{\pgfqpoint{1.714008in}{2.079180in}}%
\pgfpathlineto{\pgfqpoint{1.706060in}{2.074386in}}%
\pgfpathlineto{\pgfqpoint{1.672963in}{2.054387in}}%
\pgfpathlineto{\pgfqpoint{1.642200in}{2.053758in}}%
\pgfpathlineto{\pgfqpoint{1.631917in}{2.053543in}}%
\pgfpathlineto{\pgfqpoint{1.631496in}{2.053758in}}%
\pgfpathlineto{\pgfqpoint{1.590872in}{2.073701in}}%
\pgfpathlineto{\pgfqpoint{1.590087in}{2.074386in}}%
\pgfpathlineto{\pgfqpoint{1.565764in}{2.095013in}}%
\pgfpathlineto{\pgfqpoint{1.549827in}{2.108254in}}%
\pgfpathlineto{\pgfqpoint{1.542607in}{2.115641in}}%
\pgfpathlineto{\pgfqpoint{1.522094in}{2.136269in}}%
\pgfpathlineto{\pgfqpoint{1.508781in}{2.149484in}}%
\pgfpathlineto{\pgfqpoint{1.501370in}{2.156897in}}%
\pgfpathlineto{\pgfqpoint{1.480499in}{2.177525in}}%
\pgfpathlineto{\pgfqpoint{1.467736in}{2.190046in}}%
\pgfpathlineto{\pgfqpoint{1.458093in}{2.198153in}}%
\pgfpathlineto{\pgfqpoint{1.433416in}{2.218780in}}%
\pgfpathlineto{\pgfqpoint{1.426691in}{2.224358in}}%
\pgfpathlineto{\pgfqpoint{1.401508in}{2.239408in}}%
\pgfpathlineto{\pgfqpoint{1.385645in}{2.248852in}}%
\pgfpathlineto{\pgfqpoint{1.351197in}{2.260036in}}%
\pgfpathlineto{\pgfqpoint{1.344600in}{2.262152in}}%
\pgfpathlineto{\pgfqpoint{1.303555in}{2.264286in}}%
\pgfpathlineto{\pgfqpoint{1.280986in}{2.260036in}}%
\pgfpathlineto{\pgfqpoint{1.262509in}{2.256465in}}%
\pgfpathlineto{\pgfqpoint{1.221464in}{2.239889in}}%
\pgfpathlineto{\pgfqpoint{1.220668in}{2.239408in}}%
\pgfpathlineto{\pgfqpoint{1.186551in}{2.218780in}}%
\pgfpathlineto{\pgfqpoint{1.180419in}{2.215093in}}%
\pgfpathlineto{\pgfqpoint{1.159504in}{2.198153in}}%
\pgfpathlineto{\pgfqpoint{1.139373in}{2.182181in}}%
\pgfpathlineto{\pgfqpoint{1.134811in}{2.177525in}}%
\pgfpathlineto{\pgfqpoint{1.114463in}{2.156897in}}%
\pgfpathlineto{\pgfqpoint{1.098328in}{2.140947in}}%
\pgfpathlineto{\pgfqpoint{1.094488in}{2.136269in}}%
\pgfpathlineto{\pgfqpoint{1.077484in}{2.115641in}}%
\pgfpathlineto{\pgfqpoint{1.059986in}{2.095013in}}%
\pgfpathlineto{\pgfqpoint{1.057283in}{2.091837in}}%
\pgfpathlineto{\pgfqpoint{1.044568in}{2.074386in}}%
\pgfpathlineto{\pgfqpoint{1.029245in}{2.053758in}}%
\pgfpathlineto{\pgfqpoint{1.016237in}{2.036594in}}%
\pgfpathlineto{\pgfqpoint{1.013813in}{2.033130in}}%
\pgfpathlineto{\pgfqpoint{0.999408in}{2.012502in}}%
\pgfpathlineto{\pgfqpoint{0.984707in}{1.991874in}}%
\pgfpathlineto{\pgfqpoint{0.975192in}{1.978675in}}%
\pgfpathlineto{\pgfqpoint{0.969757in}{1.971247in}}%
\pgfpathlineto{\pgfqpoint{0.954647in}{1.950619in}}%
\pgfpathlineto{\pgfqpoint{0.939258in}{1.929991in}}%
\pgfpathlineto{\pgfqpoint{0.934147in}{1.923148in}}%
\pgfpathlineto{\pgfqpoint{0.922298in}{1.909363in}}%
\pgfpathlineto{\pgfqpoint{0.904408in}{1.888735in}}%
\pgfpathlineto{\pgfqpoint{0.893101in}{1.875806in}}%
\pgfpathlineto{\pgfqpoint{0.883757in}{1.868107in}}%
\pgfpathlineto{\pgfqpoint{0.858631in}{1.847480in}}%
\pgfpathlineto{\pgfqpoint{0.852056in}{1.842084in}}%
\pgfpathlineto{\pgfqpoint{0.814499in}{1.826852in}}%
\pgfpathlineto{\pgfqpoint{0.811011in}{1.825428in}}%
\pgfpathlineto{\pgfqpoint{0.769965in}{1.826044in}}%
\pgfpathlineto{\pgfqpoint{0.767792in}{1.826852in}}%
\pgfpathlineto{\pgfqpoint{0.728920in}{1.840867in}}%
\pgfpathlineto{\pgfqpoint{0.717662in}{1.847480in}}%
\pgfpathlineto{\pgfqpoint{0.687875in}{1.864701in}}%
\pgfpathlineto{\pgfqpoint{0.682649in}{1.868107in}}%
\pgfpathlineto{\pgfqpoint{0.650737in}{1.888735in}}%
\pgfpathlineto{\pgfqpoint{0.646829in}{1.891238in}}%
\pgfpathclose%
\pgfusepath{stroke,fill}%
\end{pgfscope}%
\begin{pgfscope}%
\pgfpathrectangle{\pgfqpoint{0.605784in}{0.382904in}}{\pgfqpoint{4.063488in}{2.042155in}}%
\pgfusepath{clip}%
\pgfsetbuttcap%
\pgfsetroundjoin%
\definecolor{currentfill}{rgb}{0.195860,0.395433,0.555276}%
\pgfsetfillcolor{currentfill}%
\pgfsetlinewidth{1.003750pt}%
\definecolor{currentstroke}{rgb}{0.195860,0.395433,0.555276}%
\pgfsetstrokecolor{currentstroke}%
\pgfsetdash{}{0pt}%
\pgfpathmoveto{\pgfqpoint{2.366653in}{2.404431in}}%
\pgfpathlineto{\pgfqpoint{2.352125in}{2.425059in}}%
\pgfpathlineto{\pgfqpoint{2.370733in}{2.425059in}}%
\pgfpathlineto{\pgfqpoint{2.411779in}{2.425059in}}%
\pgfpathlineto{\pgfqpoint{2.425912in}{2.425059in}}%
\pgfpathlineto{\pgfqpoint{2.447388in}{2.404431in}}%
\pgfpathlineto{\pgfqpoint{2.452824in}{2.399066in}}%
\pgfpathlineto{\pgfqpoint{2.485146in}{2.383803in}}%
\pgfpathlineto{\pgfqpoint{2.493869in}{2.379517in}}%
\pgfpathlineto{\pgfqpoint{2.495430in}{2.363175in}}%
\pgfpathlineto{\pgfqpoint{2.497454in}{2.342547in}}%
\pgfpathlineto{\pgfqpoint{2.499554in}{2.321920in}}%
\pgfpathlineto{\pgfqpoint{2.501738in}{2.301292in}}%
\pgfpathlineto{\pgfqpoint{2.504018in}{2.280664in}}%
\pgfpathlineto{\pgfqpoint{2.506405in}{2.260036in}}%
\pgfpathlineto{\pgfqpoint{2.508914in}{2.239408in}}%
\pgfpathlineto{\pgfqpoint{2.511561in}{2.218780in}}%
\pgfpathlineto{\pgfqpoint{2.514368in}{2.198153in}}%
\pgfpathlineto{\pgfqpoint{2.517358in}{2.177525in}}%
\pgfpathlineto{\pgfqpoint{2.520563in}{2.156897in}}%
\pgfpathlineto{\pgfqpoint{2.524018in}{2.136269in}}%
\pgfpathlineto{\pgfqpoint{2.527771in}{2.115641in}}%
\pgfpathlineto{\pgfqpoint{2.531878in}{2.095013in}}%
\pgfpathlineto{\pgfqpoint{2.534915in}{2.080903in}}%
\pgfpathlineto{\pgfqpoint{2.575960in}{2.080903in}}%
\pgfpathlineto{\pgfqpoint{2.617005in}{2.080903in}}%
\pgfpathlineto{\pgfqpoint{2.658051in}{2.080903in}}%
\pgfpathlineto{\pgfqpoint{2.699096in}{2.080903in}}%
\pgfpathlineto{\pgfqpoint{2.740141in}{2.080903in}}%
\pgfpathlineto{\pgfqpoint{2.781187in}{2.080903in}}%
\pgfpathlineto{\pgfqpoint{2.822232in}{2.080903in}}%
\pgfpathlineto{\pgfqpoint{2.863277in}{2.080903in}}%
\pgfpathlineto{\pgfqpoint{2.904323in}{2.080903in}}%
\pgfpathlineto{\pgfqpoint{2.945368in}{2.080903in}}%
\pgfpathlineto{\pgfqpoint{2.986413in}{2.080903in}}%
\pgfpathlineto{\pgfqpoint{3.027459in}{2.080903in}}%
\pgfpathlineto{\pgfqpoint{3.068504in}{2.080903in}}%
\pgfpathlineto{\pgfqpoint{3.109549in}{2.080903in}}%
\pgfpathlineto{\pgfqpoint{3.150595in}{2.080903in}}%
\pgfpathlineto{\pgfqpoint{3.191640in}{2.080903in}}%
\pgfpathlineto{\pgfqpoint{3.232685in}{2.080903in}}%
\pgfpathlineto{\pgfqpoint{3.273731in}{2.080903in}}%
\pgfpathlineto{\pgfqpoint{3.314776in}{2.080903in}}%
\pgfpathlineto{\pgfqpoint{3.355821in}{2.080903in}}%
\pgfpathlineto{\pgfqpoint{3.396867in}{2.080903in}}%
\pgfpathlineto{\pgfqpoint{3.437912in}{2.080903in}}%
\pgfpathlineto{\pgfqpoint{3.478957in}{2.080903in}}%
\pgfpathlineto{\pgfqpoint{3.520003in}{2.080903in}}%
\pgfpathlineto{\pgfqpoint{3.561048in}{2.080903in}}%
\pgfpathlineto{\pgfqpoint{3.602093in}{2.080903in}}%
\pgfpathlineto{\pgfqpoint{3.643139in}{2.080903in}}%
\pgfpathlineto{\pgfqpoint{3.644214in}{2.074386in}}%
\pgfpathlineto{\pgfqpoint{3.647632in}{2.053758in}}%
\pgfpathlineto{\pgfqpoint{3.651132in}{2.033130in}}%
\pgfpathlineto{\pgfqpoint{3.654721in}{2.012502in}}%
\pgfpathlineto{\pgfqpoint{3.658406in}{1.991874in}}%
\pgfpathlineto{\pgfqpoint{3.662197in}{1.971247in}}%
\pgfpathlineto{\pgfqpoint{3.666103in}{1.950619in}}%
\pgfpathlineto{\pgfqpoint{3.670136in}{1.929991in}}%
\pgfpathlineto{\pgfqpoint{3.674308in}{1.909363in}}%
\pgfpathlineto{\pgfqpoint{3.678633in}{1.888735in}}%
\pgfpathlineto{\pgfqpoint{3.683129in}{1.868107in}}%
\pgfpathlineto{\pgfqpoint{3.684184in}{1.863329in}}%
\pgfpathlineto{\pgfqpoint{3.725229in}{1.863329in}}%
\pgfpathlineto{\pgfqpoint{3.766275in}{1.863329in}}%
\pgfpathlineto{\pgfqpoint{3.807320in}{1.863329in}}%
\pgfpathlineto{\pgfqpoint{3.848365in}{1.863329in}}%
\pgfpathlineto{\pgfqpoint{3.889411in}{1.863329in}}%
\pgfpathlineto{\pgfqpoint{3.930456in}{1.863329in}}%
\pgfpathlineto{\pgfqpoint{3.971501in}{1.863329in}}%
\pgfpathlineto{\pgfqpoint{4.012547in}{1.863329in}}%
\pgfpathlineto{\pgfqpoint{4.053592in}{1.863329in}}%
\pgfpathlineto{\pgfqpoint{4.094637in}{1.863329in}}%
\pgfpathlineto{\pgfqpoint{4.135683in}{1.863329in}}%
\pgfpathlineto{\pgfqpoint{4.176728in}{1.863329in}}%
\pgfpathlineto{\pgfqpoint{4.217773in}{1.863329in}}%
\pgfpathlineto{\pgfqpoint{4.258819in}{1.863329in}}%
\pgfpathlineto{\pgfqpoint{4.299864in}{1.863329in}}%
\pgfpathlineto{\pgfqpoint{4.340909in}{1.863329in}}%
\pgfpathlineto{\pgfqpoint{4.381955in}{1.863329in}}%
\pgfpathlineto{\pgfqpoint{4.423000in}{1.863329in}}%
\pgfpathlineto{\pgfqpoint{4.464045in}{1.863329in}}%
\pgfpathlineto{\pgfqpoint{4.505091in}{1.863329in}}%
\pgfpathlineto{\pgfqpoint{4.546136in}{1.863329in}}%
\pgfpathlineto{\pgfqpoint{4.587181in}{1.863329in}}%
\pgfpathlineto{\pgfqpoint{4.628227in}{1.863329in}}%
\pgfpathlineto{\pgfqpoint{4.669272in}{1.863329in}}%
\pgfpathlineto{\pgfqpoint{4.669272in}{1.847480in}}%
\pgfpathlineto{\pgfqpoint{4.669272in}{1.826852in}}%
\pgfpathlineto{\pgfqpoint{4.669272in}{1.806224in}}%
\pgfpathlineto{\pgfqpoint{4.669272in}{1.785596in}}%
\pgfpathlineto{\pgfqpoint{4.669272in}{1.776894in}}%
\pgfpathlineto{\pgfqpoint{4.628227in}{1.776894in}}%
\pgfpathlineto{\pgfqpoint{4.587181in}{1.776894in}}%
\pgfpathlineto{\pgfqpoint{4.546136in}{1.776894in}}%
\pgfpathlineto{\pgfqpoint{4.505091in}{1.776894in}}%
\pgfpathlineto{\pgfqpoint{4.464045in}{1.776894in}}%
\pgfpathlineto{\pgfqpoint{4.423000in}{1.776894in}}%
\pgfpathlineto{\pgfqpoint{4.381955in}{1.776894in}}%
\pgfpathlineto{\pgfqpoint{4.340909in}{1.776894in}}%
\pgfpathlineto{\pgfqpoint{4.299864in}{1.776894in}}%
\pgfpathlineto{\pgfqpoint{4.258819in}{1.776894in}}%
\pgfpathlineto{\pgfqpoint{4.217773in}{1.776894in}}%
\pgfpathlineto{\pgfqpoint{4.176728in}{1.776894in}}%
\pgfpathlineto{\pgfqpoint{4.135683in}{1.776894in}}%
\pgfpathlineto{\pgfqpoint{4.094637in}{1.776894in}}%
\pgfpathlineto{\pgfqpoint{4.053592in}{1.776894in}}%
\pgfpathlineto{\pgfqpoint{4.012547in}{1.776894in}}%
\pgfpathlineto{\pgfqpoint{3.971501in}{1.776894in}}%
\pgfpathlineto{\pgfqpoint{3.930456in}{1.776894in}}%
\pgfpathlineto{\pgfqpoint{3.889411in}{1.776894in}}%
\pgfpathlineto{\pgfqpoint{3.848365in}{1.776894in}}%
\pgfpathlineto{\pgfqpoint{3.807320in}{1.776894in}}%
\pgfpathlineto{\pgfqpoint{3.766275in}{1.776894in}}%
\pgfpathlineto{\pgfqpoint{3.725229in}{1.776894in}}%
\pgfpathlineto{\pgfqpoint{3.684184in}{1.776894in}}%
\pgfpathlineto{\pgfqpoint{3.682216in}{1.785596in}}%
\pgfpathlineto{\pgfqpoint{3.677660in}{1.806224in}}%
\pgfpathlineto{\pgfqpoint{3.673305in}{1.826852in}}%
\pgfpathlineto{\pgfqpoint{3.669129in}{1.847480in}}%
\pgfpathlineto{\pgfqpoint{3.665112in}{1.868107in}}%
\pgfpathlineto{\pgfqpoint{3.661238in}{1.888735in}}%
\pgfpathlineto{\pgfqpoint{3.657492in}{1.909363in}}%
\pgfpathlineto{\pgfqpoint{3.653863in}{1.929991in}}%
\pgfpathlineto{\pgfqpoint{3.650339in}{1.950619in}}%
\pgfpathlineto{\pgfqpoint{3.646910in}{1.971247in}}%
\pgfpathlineto{\pgfqpoint{3.643569in}{1.991874in}}%
\pgfpathlineto{\pgfqpoint{3.643139in}{1.994527in}}%
\pgfpathlineto{\pgfqpoint{3.602093in}{1.994527in}}%
\pgfpathlineto{\pgfqpoint{3.561048in}{1.994527in}}%
\pgfpathlineto{\pgfqpoint{3.520003in}{1.994527in}}%
\pgfpathlineto{\pgfqpoint{3.478957in}{1.994527in}}%
\pgfpathlineto{\pgfqpoint{3.437912in}{1.994527in}}%
\pgfpathlineto{\pgfqpoint{3.396867in}{1.994527in}}%
\pgfpathlineto{\pgfqpoint{3.355821in}{1.994527in}}%
\pgfpathlineto{\pgfqpoint{3.314776in}{1.994527in}}%
\pgfpathlineto{\pgfqpoint{3.273731in}{1.994527in}}%
\pgfpathlineto{\pgfqpoint{3.232685in}{1.994527in}}%
\pgfpathlineto{\pgfqpoint{3.191640in}{1.994527in}}%
\pgfpathlineto{\pgfqpoint{3.150595in}{1.994527in}}%
\pgfpathlineto{\pgfqpoint{3.109549in}{1.994527in}}%
\pgfpathlineto{\pgfqpoint{3.068504in}{1.994527in}}%
\pgfpathlineto{\pgfqpoint{3.027459in}{1.994527in}}%
\pgfpathlineto{\pgfqpoint{2.986413in}{1.994527in}}%
\pgfpathlineto{\pgfqpoint{2.945368in}{1.994527in}}%
\pgfpathlineto{\pgfqpoint{2.904323in}{1.994527in}}%
\pgfpathlineto{\pgfqpoint{2.863277in}{1.994527in}}%
\pgfpathlineto{\pgfqpoint{2.822232in}{1.994527in}}%
\pgfpathlineto{\pgfqpoint{2.781187in}{1.994527in}}%
\pgfpathlineto{\pgfqpoint{2.740141in}{1.994527in}}%
\pgfpathlineto{\pgfqpoint{2.699096in}{1.994527in}}%
\pgfpathlineto{\pgfqpoint{2.658051in}{1.994527in}}%
\pgfpathlineto{\pgfqpoint{2.617005in}{1.994527in}}%
\pgfpathlineto{\pgfqpoint{2.575960in}{1.994527in}}%
\pgfpathlineto{\pgfqpoint{2.534915in}{1.994527in}}%
\pgfpathlineto{\pgfqpoint{2.530314in}{2.012502in}}%
\pgfpathlineto{\pgfqpoint{2.525669in}{2.033130in}}%
\pgfpathlineto{\pgfqpoint{2.521565in}{2.053758in}}%
\pgfpathlineto{\pgfqpoint{2.517889in}{2.074386in}}%
\pgfpathlineto{\pgfqpoint{2.514559in}{2.095013in}}%
\pgfpathlineto{\pgfqpoint{2.511510in}{2.115641in}}%
\pgfpathlineto{\pgfqpoint{2.508694in}{2.136269in}}%
\pgfpathlineto{\pgfqpoint{2.506073in}{2.156897in}}%
\pgfpathlineto{\pgfqpoint{2.503617in}{2.177525in}}%
\pgfpathlineto{\pgfqpoint{2.501301in}{2.198153in}}%
\pgfpathlineto{\pgfqpoint{2.499106in}{2.218780in}}%
\pgfpathlineto{\pgfqpoint{2.497016in}{2.239408in}}%
\pgfpathlineto{\pgfqpoint{2.495016in}{2.260036in}}%
\pgfpathlineto{\pgfqpoint{2.493869in}{2.272154in}}%
\pgfpathlineto{\pgfqpoint{2.479986in}{2.280664in}}%
\pgfpathlineto{\pgfqpoint{2.452824in}{2.296528in}}%
\pgfpathlineto{\pgfqpoint{2.448723in}{2.301292in}}%
\pgfpathlineto{\pgfqpoint{2.430347in}{2.321920in}}%
\pgfpathlineto{\pgfqpoint{2.411779in}{2.342185in}}%
\pgfpathlineto{\pgfqpoint{2.411526in}{2.342547in}}%
\pgfpathlineto{\pgfqpoint{2.396754in}{2.363175in}}%
\pgfpathlineto{\pgfqpoint{2.381697in}{2.383803in}}%
\pgfpathlineto{\pgfqpoint{2.370733in}{2.398514in}}%
\pgfpathclose%
\pgfusepath{stroke,fill}%
\end{pgfscope}%
\begin{pgfscope}%
\pgfpathrectangle{\pgfqpoint{0.605784in}{0.382904in}}{\pgfqpoint{4.063488in}{2.042155in}}%
\pgfusepath{clip}%
\pgfsetbuttcap%
\pgfsetroundjoin%
\definecolor{currentfill}{rgb}{0.166617,0.463708,0.558119}%
\pgfsetfillcolor{currentfill}%
\pgfsetlinewidth{1.003750pt}%
\definecolor{currentstroke}{rgb}{0.166617,0.463708,0.558119}%
\pgfsetstrokecolor{currentstroke}%
\pgfsetdash{}{0pt}%
\pgfpathmoveto{\pgfqpoint{0.611848in}{0.506671in}}%
\pgfpathlineto{\pgfqpoint{0.605784in}{0.509225in}}%
\pgfpathlineto{\pgfqpoint{0.605784in}{0.527299in}}%
\pgfpathlineto{\pgfqpoint{0.605784in}{0.547927in}}%
\pgfpathlineto{\pgfqpoint{0.605784in}{0.568554in}}%
\pgfpathlineto{\pgfqpoint{0.605784in}{0.583374in}}%
\pgfpathlineto{\pgfqpoint{0.641478in}{0.568554in}}%
\pgfpathlineto{\pgfqpoint{0.646829in}{0.566334in}}%
\pgfpathlineto{\pgfqpoint{0.687875in}{0.557731in}}%
\pgfpathlineto{\pgfqpoint{0.728920in}{0.557713in}}%
\pgfpathlineto{\pgfqpoint{0.769965in}{0.565099in}}%
\pgfpathlineto{\pgfqpoint{0.780958in}{0.568554in}}%
\pgfpathlineto{\pgfqpoint{0.811011in}{0.578197in}}%
\pgfpathlineto{\pgfqpoint{0.837121in}{0.589182in}}%
\pgfpathlineto{\pgfqpoint{0.852056in}{0.595499in}}%
\pgfpathlineto{\pgfqpoint{0.879607in}{0.609810in}}%
\pgfpathlineto{\pgfqpoint{0.893101in}{0.616752in}}%
\pgfpathlineto{\pgfqpoint{0.914586in}{0.630438in}}%
\pgfpathlineto{\pgfqpoint{0.934147in}{0.642622in}}%
\pgfpathlineto{\pgfqpoint{0.944973in}{0.651066in}}%
\pgfpathlineto{\pgfqpoint{0.972150in}{0.671694in}}%
\pgfpathlineto{\pgfqpoint{0.975192in}{0.673966in}}%
\pgfpathlineto{\pgfqpoint{0.995124in}{0.692321in}}%
\pgfpathlineto{\pgfqpoint{1.016237in}{0.711049in}}%
\pgfpathlineto{\pgfqpoint{1.018071in}{0.712949in}}%
\pgfpathlineto{\pgfqpoint{1.038120in}{0.733577in}}%
\pgfpathlineto{\pgfqpoint{1.057283in}{0.752602in}}%
\pgfpathlineto{\pgfqpoint{1.058775in}{0.754205in}}%
\pgfpathlineto{\pgfqpoint{1.078054in}{0.774833in}}%
\pgfpathlineto{\pgfqpoint{1.097974in}{0.795460in}}%
\pgfpathlineto{\pgfqpoint{1.098328in}{0.795826in}}%
\pgfpathlineto{\pgfqpoint{1.118337in}{0.816088in}}%
\pgfpathlineto{\pgfqpoint{1.139298in}{0.836716in}}%
\pgfpathlineto{\pgfqpoint{1.139373in}{0.836790in}}%
\pgfpathlineto{\pgfqpoint{1.163675in}{0.857344in}}%
\pgfpathlineto{\pgfqpoint{1.180419in}{0.871247in}}%
\pgfpathlineto{\pgfqpoint{1.191781in}{0.877972in}}%
\pgfpathlineto{\pgfqpoint{1.221464in}{0.895447in}}%
\pgfpathlineto{\pgfqpoint{1.232099in}{0.898600in}}%
\pgfpathlineto{\pgfqpoint{1.262509in}{0.907694in}}%
\pgfpathlineto{\pgfqpoint{1.303555in}{0.908372in}}%
\pgfpathlineto{\pgfqpoint{1.344600in}{0.900191in}}%
\pgfpathlineto{\pgfqpoint{1.349731in}{0.898600in}}%
\pgfpathlineto{\pgfqpoint{1.385645in}{0.887577in}}%
\pgfpathlineto{\pgfqpoint{1.417145in}{0.877972in}}%
\pgfpathlineto{\pgfqpoint{1.426691in}{0.875071in}}%
\pgfpathlineto{\pgfqpoint{1.467736in}{0.866971in}}%
\pgfpathlineto{\pgfqpoint{1.508781in}{0.865815in}}%
\pgfpathlineto{\pgfqpoint{1.549827in}{0.872291in}}%
\pgfpathlineto{\pgfqpoint{1.568559in}{0.877972in}}%
\pgfpathlineto{\pgfqpoint{1.590872in}{0.885055in}}%
\pgfpathlineto{\pgfqpoint{1.625449in}{0.898600in}}%
\pgfpathlineto{\pgfqpoint{1.631917in}{0.901221in}}%
\pgfpathlineto{\pgfqpoint{1.672963in}{0.917868in}}%
\pgfpathlineto{\pgfqpoint{1.676563in}{0.919227in}}%
\pgfpathlineto{\pgfqpoint{1.714008in}{0.933381in}}%
\pgfpathlineto{\pgfqpoint{1.731911in}{0.939855in}}%
\pgfpathlineto{\pgfqpoint{1.755053in}{0.948106in}}%
\pgfpathlineto{\pgfqpoint{1.786573in}{0.960483in}}%
\pgfpathlineto{\pgfqpoint{1.796099in}{0.964124in}}%
\pgfpathlineto{\pgfqpoint{1.831379in}{0.981111in}}%
\pgfpathlineto{\pgfqpoint{1.837144in}{0.983790in}}%
\pgfpathlineto{\pgfqpoint{1.866851in}{1.001739in}}%
\pgfpathlineto{\pgfqpoint{1.878189in}{1.008327in}}%
\pgfpathlineto{\pgfqpoint{1.897809in}{1.022367in}}%
\pgfpathlineto{\pgfqpoint{1.919235in}{1.037128in}}%
\pgfpathlineto{\pgfqpoint{1.926848in}{1.042994in}}%
\pgfpathlineto{\pgfqpoint{1.954273in}{1.063622in}}%
\pgfpathlineto{\pgfqpoint{1.960280in}{1.068042in}}%
\pgfpathlineto{\pgfqpoint{1.982338in}{1.084250in}}%
\pgfpathlineto{\pgfqpoint{2.001325in}{1.097854in}}%
\pgfpathlineto{\pgfqpoint{2.012637in}{1.104878in}}%
\pgfpathlineto{\pgfqpoint{2.042371in}{1.123078in}}%
\pgfpathlineto{\pgfqpoint{2.047856in}{1.125506in}}%
\pgfpathlineto{\pgfqpoint{2.083416in}{1.141219in}}%
\pgfpathlineto{\pgfqpoint{2.103739in}{1.146134in}}%
\pgfpathlineto{\pgfqpoint{2.124461in}{1.151205in}}%
\pgfpathlineto{\pgfqpoint{2.165507in}{1.154029in}}%
\pgfpathlineto{\pgfqpoint{2.206552in}{1.152382in}}%
\pgfpathlineto{\pgfqpoint{2.247597in}{1.150083in}}%
\pgfpathlineto{\pgfqpoint{2.288643in}{1.151105in}}%
\pgfpathlineto{\pgfqpoint{2.329688in}{1.158510in}}%
\pgfpathlineto{\pgfqpoint{2.352716in}{1.166761in}}%
\pgfpathlineto{\pgfqpoint{2.370733in}{1.173607in}}%
\pgfpathlineto{\pgfqpoint{2.397154in}{1.187389in}}%
\pgfpathlineto{\pgfqpoint{2.411779in}{1.195447in}}%
\pgfpathlineto{\pgfqpoint{2.432731in}{1.208017in}}%
\pgfpathlineto{\pgfqpoint{2.452824in}{1.220635in}}%
\pgfpathlineto{\pgfqpoint{2.467049in}{1.228645in}}%
\pgfpathlineto{\pgfqpoint{2.493869in}{1.244237in}}%
\pgfpathlineto{\pgfqpoint{2.495487in}{1.228645in}}%
\pgfpathlineto{\pgfqpoint{2.497565in}{1.208017in}}%
\pgfpathlineto{\pgfqpoint{2.499569in}{1.187389in}}%
\pgfpathlineto{\pgfqpoint{2.501508in}{1.166761in}}%
\pgfpathlineto{\pgfqpoint{2.503387in}{1.146134in}}%
\pgfpathlineto{\pgfqpoint{2.505212in}{1.125506in}}%
\pgfpathlineto{\pgfqpoint{2.506988in}{1.104878in}}%
\pgfpathlineto{\pgfqpoint{2.508720in}{1.084250in}}%
\pgfpathlineto{\pgfqpoint{2.510411in}{1.063622in}}%
\pgfpathlineto{\pgfqpoint{2.512064in}{1.042994in}}%
\pgfpathlineto{\pgfqpoint{2.513684in}{1.022367in}}%
\pgfpathlineto{\pgfqpoint{2.515271in}{1.001739in}}%
\pgfpathlineto{\pgfqpoint{2.516830in}{0.981111in}}%
\pgfpathlineto{\pgfqpoint{2.518362in}{0.960483in}}%
\pgfpathlineto{\pgfqpoint{2.519869in}{0.939855in}}%
\pgfpathlineto{\pgfqpoint{2.521353in}{0.919227in}}%
\pgfpathlineto{\pgfqpoint{2.522815in}{0.898600in}}%
\pgfpathlineto{\pgfqpoint{2.524258in}{0.877972in}}%
\pgfpathlineto{\pgfqpoint{2.525681in}{0.857344in}}%
\pgfpathlineto{\pgfqpoint{2.527088in}{0.836716in}}%
\pgfpathlineto{\pgfqpoint{2.528478in}{0.816088in}}%
\pgfpathlineto{\pgfqpoint{2.529852in}{0.795460in}}%
\pgfpathlineto{\pgfqpoint{2.531213in}{0.774833in}}%
\pgfpathlineto{\pgfqpoint{2.532560in}{0.754205in}}%
\pgfpathlineto{\pgfqpoint{2.533894in}{0.733577in}}%
\pgfpathlineto{\pgfqpoint{2.534915in}{0.717728in}}%
\pgfpathlineto{\pgfqpoint{2.575960in}{0.717728in}}%
\pgfpathlineto{\pgfqpoint{2.617005in}{0.717728in}}%
\pgfpathlineto{\pgfqpoint{2.658051in}{0.717728in}}%
\pgfpathlineto{\pgfqpoint{2.699096in}{0.717728in}}%
\pgfpathlineto{\pgfqpoint{2.740141in}{0.717728in}}%
\pgfpathlineto{\pgfqpoint{2.781187in}{0.717728in}}%
\pgfpathlineto{\pgfqpoint{2.822232in}{0.717728in}}%
\pgfpathlineto{\pgfqpoint{2.863277in}{0.717728in}}%
\pgfpathlineto{\pgfqpoint{2.904323in}{0.717728in}}%
\pgfpathlineto{\pgfqpoint{2.945368in}{0.717728in}}%
\pgfpathlineto{\pgfqpoint{2.986413in}{0.717728in}}%
\pgfpathlineto{\pgfqpoint{3.027459in}{0.717728in}}%
\pgfpathlineto{\pgfqpoint{3.068504in}{0.717728in}}%
\pgfpathlineto{\pgfqpoint{3.109549in}{0.717728in}}%
\pgfpathlineto{\pgfqpoint{3.150595in}{0.717728in}}%
\pgfpathlineto{\pgfqpoint{3.191640in}{0.717728in}}%
\pgfpathlineto{\pgfqpoint{3.232685in}{0.717728in}}%
\pgfpathlineto{\pgfqpoint{3.273731in}{0.717728in}}%
\pgfpathlineto{\pgfqpoint{3.314776in}{0.717728in}}%
\pgfpathlineto{\pgfqpoint{3.355821in}{0.717728in}}%
\pgfpathlineto{\pgfqpoint{3.396867in}{0.717728in}}%
\pgfpathlineto{\pgfqpoint{3.437912in}{0.717728in}}%
\pgfpathlineto{\pgfqpoint{3.478957in}{0.717728in}}%
\pgfpathlineto{\pgfqpoint{3.520003in}{0.717728in}}%
\pgfpathlineto{\pgfqpoint{3.561048in}{0.717728in}}%
\pgfpathlineto{\pgfqpoint{3.602093in}{0.717728in}}%
\pgfpathlineto{\pgfqpoint{3.643139in}{0.717728in}}%
\pgfpathlineto{\pgfqpoint{3.644194in}{0.712949in}}%
\pgfpathlineto{\pgfqpoint{3.648689in}{0.692321in}}%
\pgfpathlineto{\pgfqpoint{3.653015in}{0.671694in}}%
\pgfpathlineto{\pgfqpoint{3.657186in}{0.651066in}}%
\pgfpathlineto{\pgfqpoint{3.661219in}{0.630438in}}%
\pgfpathlineto{\pgfqpoint{3.665125in}{0.609810in}}%
\pgfpathlineto{\pgfqpoint{3.668916in}{0.589182in}}%
\pgfpathlineto{\pgfqpoint{3.672602in}{0.568554in}}%
\pgfpathlineto{\pgfqpoint{3.676191in}{0.547927in}}%
\pgfpathlineto{\pgfqpoint{3.679691in}{0.527299in}}%
\pgfpathlineto{\pgfqpoint{3.683109in}{0.506671in}}%
\pgfpathlineto{\pgfqpoint{3.684184in}{0.500154in}}%
\pgfpathlineto{\pgfqpoint{3.725229in}{0.500154in}}%
\pgfpathlineto{\pgfqpoint{3.766275in}{0.500154in}}%
\pgfpathlineto{\pgfqpoint{3.807320in}{0.500154in}}%
\pgfpathlineto{\pgfqpoint{3.848365in}{0.500154in}}%
\pgfpathlineto{\pgfqpoint{3.889411in}{0.500154in}}%
\pgfpathlineto{\pgfqpoint{3.930456in}{0.500154in}}%
\pgfpathlineto{\pgfqpoint{3.971501in}{0.500154in}}%
\pgfpathlineto{\pgfqpoint{4.012547in}{0.500154in}}%
\pgfpathlineto{\pgfqpoint{4.053592in}{0.500154in}}%
\pgfpathlineto{\pgfqpoint{4.094637in}{0.500154in}}%
\pgfpathlineto{\pgfqpoint{4.135683in}{0.500154in}}%
\pgfpathlineto{\pgfqpoint{4.176728in}{0.500154in}}%
\pgfpathlineto{\pgfqpoint{4.217773in}{0.500154in}}%
\pgfpathlineto{\pgfqpoint{4.258819in}{0.500154in}}%
\pgfpathlineto{\pgfqpoint{4.299864in}{0.500154in}}%
\pgfpathlineto{\pgfqpoint{4.340909in}{0.500154in}}%
\pgfpathlineto{\pgfqpoint{4.381955in}{0.500154in}}%
\pgfpathlineto{\pgfqpoint{4.423000in}{0.500154in}}%
\pgfpathlineto{\pgfqpoint{4.464045in}{0.500154in}}%
\pgfpathlineto{\pgfqpoint{4.505091in}{0.500154in}}%
\pgfpathlineto{\pgfqpoint{4.546136in}{0.500154in}}%
\pgfpathlineto{\pgfqpoint{4.587181in}{0.500154in}}%
\pgfpathlineto{\pgfqpoint{4.628227in}{0.500154in}}%
\pgfpathlineto{\pgfqpoint{4.669272in}{0.500154in}}%
\pgfpathlineto{\pgfqpoint{4.669272in}{0.486043in}}%
\pgfpathlineto{\pgfqpoint{4.669272in}{0.465415in}}%
\pgfpathlineto{\pgfqpoint{4.669272in}{0.444787in}}%
\pgfpathlineto{\pgfqpoint{4.669272in}{0.424160in}}%
\pgfpathlineto{\pgfqpoint{4.669272in}{0.423474in}}%
\pgfpathlineto{\pgfqpoint{4.628227in}{0.423474in}}%
\pgfpathlineto{\pgfqpoint{4.587181in}{0.423474in}}%
\pgfpathlineto{\pgfqpoint{4.546136in}{0.423474in}}%
\pgfpathlineto{\pgfqpoint{4.505091in}{0.423474in}}%
\pgfpathlineto{\pgfqpoint{4.464045in}{0.423474in}}%
\pgfpathlineto{\pgfqpoint{4.423000in}{0.423474in}}%
\pgfpathlineto{\pgfqpoint{4.381955in}{0.423474in}}%
\pgfpathlineto{\pgfqpoint{4.340909in}{0.423474in}}%
\pgfpathlineto{\pgfqpoint{4.299864in}{0.423474in}}%
\pgfpathlineto{\pgfqpoint{4.258819in}{0.423474in}}%
\pgfpathlineto{\pgfqpoint{4.217773in}{0.423474in}}%
\pgfpathlineto{\pgfqpoint{4.176728in}{0.423474in}}%
\pgfpathlineto{\pgfqpoint{4.135683in}{0.423474in}}%
\pgfpathlineto{\pgfqpoint{4.094637in}{0.423474in}}%
\pgfpathlineto{\pgfqpoint{4.053592in}{0.423474in}}%
\pgfpathlineto{\pgfqpoint{4.012547in}{0.423474in}}%
\pgfpathlineto{\pgfqpoint{3.971501in}{0.423474in}}%
\pgfpathlineto{\pgfqpoint{3.930456in}{0.423474in}}%
\pgfpathlineto{\pgfqpoint{3.889411in}{0.423474in}}%
\pgfpathlineto{\pgfqpoint{3.848365in}{0.423474in}}%
\pgfpathlineto{\pgfqpoint{3.807320in}{0.423474in}}%
\pgfpathlineto{\pgfqpoint{3.766275in}{0.423474in}}%
\pgfpathlineto{\pgfqpoint{3.725229in}{0.423474in}}%
\pgfpathlineto{\pgfqpoint{3.684184in}{0.423474in}}%
\pgfpathlineto{\pgfqpoint{3.684069in}{0.424160in}}%
\pgfpathlineto{\pgfqpoint{3.680628in}{0.444787in}}%
\pgfpathlineto{\pgfqpoint{3.677113in}{0.465415in}}%
\pgfpathlineto{\pgfqpoint{3.673516in}{0.486043in}}%
\pgfpathlineto{\pgfqpoint{3.669833in}{0.506671in}}%
\pgfpathlineto{\pgfqpoint{3.666056in}{0.527299in}}%
\pgfpathlineto{\pgfqpoint{3.662178in}{0.547927in}}%
\pgfpathlineto{\pgfqpoint{3.658188in}{0.568554in}}%
\pgfpathlineto{\pgfqpoint{3.654079in}{0.589182in}}%
\pgfpathlineto{\pgfqpoint{3.649838in}{0.609810in}}%
\pgfpathlineto{\pgfqpoint{3.645454in}{0.630438in}}%
\pgfpathlineto{\pgfqpoint{3.643139in}{0.641121in}}%
\pgfpathlineto{\pgfqpoint{3.602093in}{0.641121in}}%
\pgfpathlineto{\pgfqpoint{3.561048in}{0.641121in}}%
\pgfpathlineto{\pgfqpoint{3.520003in}{0.641121in}}%
\pgfpathlineto{\pgfqpoint{3.478957in}{0.641121in}}%
\pgfpathlineto{\pgfqpoint{3.437912in}{0.641121in}}%
\pgfpathlineto{\pgfqpoint{3.396867in}{0.641121in}}%
\pgfpathlineto{\pgfqpoint{3.355821in}{0.641121in}}%
\pgfpathlineto{\pgfqpoint{3.314776in}{0.641121in}}%
\pgfpathlineto{\pgfqpoint{3.273731in}{0.641121in}}%
\pgfpathlineto{\pgfqpoint{3.232685in}{0.641121in}}%
\pgfpathlineto{\pgfqpoint{3.191640in}{0.641121in}}%
\pgfpathlineto{\pgfqpoint{3.150595in}{0.641121in}}%
\pgfpathlineto{\pgfqpoint{3.109549in}{0.641121in}}%
\pgfpathlineto{\pgfqpoint{3.068504in}{0.641121in}}%
\pgfpathlineto{\pgfqpoint{3.027459in}{0.641121in}}%
\pgfpathlineto{\pgfqpoint{2.986413in}{0.641121in}}%
\pgfpathlineto{\pgfqpoint{2.945368in}{0.641121in}}%
\pgfpathlineto{\pgfqpoint{2.904323in}{0.641121in}}%
\pgfpathlineto{\pgfqpoint{2.863277in}{0.641121in}}%
\pgfpathlineto{\pgfqpoint{2.822232in}{0.641121in}}%
\pgfpathlineto{\pgfqpoint{2.781187in}{0.641121in}}%
\pgfpathlineto{\pgfqpoint{2.740141in}{0.641121in}}%
\pgfpathlineto{\pgfqpoint{2.699096in}{0.641121in}}%
\pgfpathlineto{\pgfqpoint{2.658051in}{0.641121in}}%
\pgfpathlineto{\pgfqpoint{2.617005in}{0.641121in}}%
\pgfpathlineto{\pgfqpoint{2.575960in}{0.641121in}}%
\pgfpathlineto{\pgfqpoint{2.534915in}{0.641121in}}%
\pgfpathlineto{\pgfqpoint{2.534249in}{0.651066in}}%
\pgfpathlineto{\pgfqpoint{2.532867in}{0.671694in}}%
\pgfpathlineto{\pgfqpoint{2.531473in}{0.692321in}}%
\pgfpathlineto{\pgfqpoint{2.530064in}{0.712949in}}%
\pgfpathlineto{\pgfqpoint{2.528640in}{0.733577in}}%
\pgfpathlineto{\pgfqpoint{2.527200in}{0.754205in}}%
\pgfpathlineto{\pgfqpoint{2.525742in}{0.774833in}}%
\pgfpathlineto{\pgfqpoint{2.524267in}{0.795460in}}%
\pgfpathlineto{\pgfqpoint{2.522773in}{0.816088in}}%
\pgfpathlineto{\pgfqpoint{2.521258in}{0.836716in}}%
\pgfpathlineto{\pgfqpoint{2.519721in}{0.857344in}}%
\pgfpathlineto{\pgfqpoint{2.518160in}{0.877972in}}%
\pgfpathlineto{\pgfqpoint{2.516575in}{0.898600in}}%
\pgfpathlineto{\pgfqpoint{2.514963in}{0.919227in}}%
\pgfpathlineto{\pgfqpoint{2.513322in}{0.939855in}}%
\pgfpathlineto{\pgfqpoint{2.511650in}{0.960483in}}%
\pgfpathlineto{\pgfqpoint{2.509944in}{0.981111in}}%
\pgfpathlineto{\pgfqpoint{2.508202in}{1.001739in}}%
\pgfpathlineto{\pgfqpoint{2.506422in}{1.022367in}}%
\pgfpathlineto{\pgfqpoint{2.504599in}{1.042994in}}%
\pgfpathlineto{\pgfqpoint{2.502730in}{1.063622in}}%
\pgfpathlineto{\pgfqpoint{2.500810in}{1.084250in}}%
\pgfpathlineto{\pgfqpoint{2.498836in}{1.104878in}}%
\pgfpathlineto{\pgfqpoint{2.496803in}{1.125506in}}%
\pgfpathlineto{\pgfqpoint{2.494703in}{1.146134in}}%
\pgfpathlineto{\pgfqpoint{2.493869in}{1.154214in}}%
\pgfpathlineto{\pgfqpoint{2.478161in}{1.146134in}}%
\pgfpathlineto{\pgfqpoint{2.452824in}{1.133460in}}%
\pgfpathlineto{\pgfqpoint{2.437820in}{1.125506in}}%
\pgfpathlineto{\pgfqpoint{2.411779in}{1.112245in}}%
\pgfpathlineto{\pgfqpoint{2.395241in}{1.104878in}}%
\pgfpathlineto{\pgfqpoint{2.370733in}{1.094477in}}%
\pgfpathlineto{\pgfqpoint{2.334972in}{1.084250in}}%
\pgfpathlineto{\pgfqpoint{2.329688in}{1.082816in}}%
\pgfpathlineto{\pgfqpoint{2.288643in}{1.078126in}}%
\pgfpathlineto{\pgfqpoint{2.247597in}{1.078910in}}%
\pgfpathlineto{\pgfqpoint{2.206552in}{1.082231in}}%
\pgfpathlineto{\pgfqpoint{2.167413in}{1.084250in}}%
\pgfpathlineto{\pgfqpoint{2.165507in}{1.084350in}}%
\pgfpathlineto{\pgfqpoint{2.164003in}{1.084250in}}%
\pgfpathlineto{\pgfqpoint{2.124461in}{1.081683in}}%
\pgfpathlineto{\pgfqpoint{2.083416in}{1.071602in}}%
\pgfpathlineto{\pgfqpoint{2.065630in}{1.063622in}}%
\pgfpathlineto{\pgfqpoint{2.042371in}{1.053172in}}%
\pgfpathlineto{\pgfqpoint{2.026049in}{1.042994in}}%
\pgfpathlineto{\pgfqpoint{2.001325in}{1.027373in}}%
\pgfpathlineto{\pgfqpoint{1.994513in}{1.022367in}}%
\pgfpathlineto{\pgfqpoint{1.966873in}{1.001739in}}%
\pgfpathlineto{\pgfqpoint{1.960280in}{0.996744in}}%
\pgfpathlineto{\pgfqpoint{1.940034in}{0.981111in}}%
\pgfpathlineto{\pgfqpoint{1.919235in}{0.964555in}}%
\pgfpathlineto{\pgfqpoint{1.913627in}{0.960483in}}%
\pgfpathlineto{\pgfqpoint{1.885872in}{0.939855in}}%
\pgfpathlineto{\pgfqpoint{1.878189in}{0.933996in}}%
\pgfpathlineto{\pgfqpoint{1.855170in}{0.919227in}}%
\pgfpathlineto{\pgfqpoint{1.837144in}{0.907253in}}%
\pgfpathlineto{\pgfqpoint{1.820849in}{0.898600in}}%
\pgfpathlineto{\pgfqpoint{1.796099in}{0.885036in}}%
\pgfpathlineto{\pgfqpoint{1.780157in}{0.877972in}}%
\pgfpathlineto{\pgfqpoint{1.755053in}{0.866580in}}%
\pgfpathlineto{\pgfqpoint{1.731869in}{0.857344in}}%
\pgfpathlineto{\pgfqpoint{1.714008in}{0.850139in}}%
\pgfpathlineto{\pgfqpoint{1.679596in}{0.836716in}}%
\pgfpathlineto{\pgfqpoint{1.672963in}{0.834132in}}%
\pgfpathlineto{\pgfqpoint{1.631917in}{0.818321in}}%
\pgfpathlineto{\pgfqpoint{1.625644in}{0.816088in}}%
\pgfpathlineto{\pgfqpoint{1.590872in}{0.804078in}}%
\pgfpathlineto{\pgfqpoint{1.557142in}{0.795460in}}%
\pgfpathlineto{\pgfqpoint{1.549827in}{0.793666in}}%
\pgfpathlineto{\pgfqpoint{1.508781in}{0.789501in}}%
\pgfpathlineto{\pgfqpoint{1.467736in}{0.792439in}}%
\pgfpathlineto{\pgfqpoint{1.454428in}{0.795460in}}%
\pgfpathlineto{\pgfqpoint{1.426691in}{0.801759in}}%
\pgfpathlineto{\pgfqpoint{1.385645in}{0.814710in}}%
\pgfpathlineto{\pgfqpoint{1.381167in}{0.816088in}}%
\pgfpathlineto{\pgfqpoint{1.344600in}{0.827451in}}%
\pgfpathlineto{\pgfqpoint{1.303555in}{0.835272in}}%
\pgfpathlineto{\pgfqpoint{1.262509in}{0.834294in}}%
\pgfpathlineto{\pgfqpoint{1.221464in}{0.821916in}}%
\pgfpathlineto{\pgfqpoint{1.211575in}{0.816088in}}%
\pgfpathlineto{\pgfqpoint{1.180419in}{0.797636in}}%
\pgfpathlineto{\pgfqpoint{1.177796in}{0.795460in}}%
\pgfpathlineto{\pgfqpoint{1.152991in}{0.774833in}}%
\pgfpathlineto{\pgfqpoint{1.139373in}{0.763351in}}%
\pgfpathlineto{\pgfqpoint{1.130091in}{0.754205in}}%
\pgfpathlineto{\pgfqpoint{1.109407in}{0.733577in}}%
\pgfpathlineto{\pgfqpoint{1.098328in}{0.722371in}}%
\pgfpathlineto{\pgfqpoint{1.089283in}{0.712949in}}%
\pgfpathlineto{\pgfqpoint{1.069768in}{0.692321in}}%
\pgfpathlineto{\pgfqpoint{1.057283in}{0.678866in}}%
\pgfpathlineto{\pgfqpoint{1.050158in}{0.671694in}}%
\pgfpathlineto{\pgfqpoint{1.029965in}{0.651066in}}%
\pgfpathlineto{\pgfqpoint{1.016237in}{0.636698in}}%
\pgfpathlineto{\pgfqpoint{1.009337in}{0.630438in}}%
\pgfpathlineto{\pgfqpoint{0.986976in}{0.609810in}}%
\pgfpathlineto{\pgfqpoint{0.975192in}{0.598680in}}%
\pgfpathlineto{\pgfqpoint{0.962991in}{0.589182in}}%
\pgfpathlineto{\pgfqpoint{0.937145in}{0.568554in}}%
\pgfpathlineto{\pgfqpoint{0.934147in}{0.566132in}}%
\pgfpathlineto{\pgfqpoint{0.906130in}{0.547927in}}%
\pgfpathlineto{\pgfqpoint{0.893101in}{0.539288in}}%
\pgfpathlineto{\pgfqpoint{0.870492in}{0.527299in}}%
\pgfpathlineto{\pgfqpoint{0.852056in}{0.517438in}}%
\pgfpathlineto{\pgfqpoint{0.826548in}{0.506671in}}%
\pgfpathlineto{\pgfqpoint{0.811011in}{0.500144in}}%
\pgfpathlineto{\pgfqpoint{0.769965in}{0.487700in}}%
\pgfpathlineto{\pgfqpoint{0.759476in}{0.486043in}}%
\pgfpathlineto{\pgfqpoint{0.728920in}{0.481364in}}%
\pgfpathlineto{\pgfqpoint{0.687875in}{0.482474in}}%
\pgfpathlineto{\pgfqpoint{0.672367in}{0.486043in}}%
\pgfpathlineto{\pgfqpoint{0.646829in}{0.491949in}}%
\pgfpathclose%
\pgfusepath{stroke,fill}%
\end{pgfscope}%
\begin{pgfscope}%
\pgfpathrectangle{\pgfqpoint{0.605784in}{0.382904in}}{\pgfqpoint{4.063488in}{2.042155in}}%
\pgfusepath{clip}%
\pgfsetbuttcap%
\pgfsetroundjoin%
\definecolor{currentfill}{rgb}{0.166617,0.463708,0.558119}%
\pgfsetfillcolor{currentfill}%
\pgfsetlinewidth{1.003750pt}%
\definecolor{currentstroke}{rgb}{0.166617,0.463708,0.558119}%
\pgfsetstrokecolor{currentstroke}%
\pgfsetdash{}{0pt}%
\pgfpathmoveto{\pgfqpoint{0.616074in}{1.991874in}}%
\pgfpathlineto{\pgfqpoint{0.605784in}{1.997683in}}%
\pgfpathlineto{\pgfqpoint{0.605784in}{2.012502in}}%
\pgfpathlineto{\pgfqpoint{0.605784in}{2.033130in}}%
\pgfpathlineto{\pgfqpoint{0.605784in}{2.053758in}}%
\pgfpathlineto{\pgfqpoint{0.605784in}{2.071832in}}%
\pgfpathlineto{\pgfqpoint{0.638148in}{2.053758in}}%
\pgfpathlineto{\pgfqpoint{0.646829in}{2.048907in}}%
\pgfpathlineto{\pgfqpoint{0.673199in}{2.033130in}}%
\pgfpathlineto{\pgfqpoint{0.687875in}{2.024309in}}%
\pgfpathlineto{\pgfqpoint{0.710922in}{2.012502in}}%
\pgfpathlineto{\pgfqpoint{0.728920in}{2.003167in}}%
\pgfpathlineto{\pgfqpoint{0.766815in}{1.991874in}}%
\pgfpathlineto{\pgfqpoint{0.769965in}{1.990913in}}%
\pgfpathlineto{\pgfqpoint{0.808451in}{1.991874in}}%
\pgfpathlineto{\pgfqpoint{0.811011in}{1.991937in}}%
\pgfpathlineto{\pgfqpoint{0.852056in}{2.008671in}}%
\pgfpathlineto{\pgfqpoint{0.856939in}{2.012502in}}%
\pgfpathlineto{\pgfqpoint{0.883234in}{2.033130in}}%
\pgfpathlineto{\pgfqpoint{0.893101in}{2.040903in}}%
\pgfpathlineto{\pgfqpoint{0.904955in}{2.053758in}}%
\pgfpathlineto{\pgfqpoint{0.923821in}{2.074386in}}%
\pgfpathlineto{\pgfqpoint{0.934147in}{2.085748in}}%
\pgfpathlineto{\pgfqpoint{0.941421in}{2.095013in}}%
\pgfpathlineto{\pgfqpoint{0.957537in}{2.115641in}}%
\pgfpathlineto{\pgfqpoint{0.973390in}{2.136269in}}%
\pgfpathlineto{\pgfqpoint{0.975192in}{2.138607in}}%
\pgfpathlineto{\pgfqpoint{0.988894in}{2.156897in}}%
\pgfpathlineto{\pgfqpoint{1.004101in}{2.177525in}}%
\pgfpathlineto{\pgfqpoint{1.016237in}{2.194211in}}%
\pgfpathlineto{\pgfqpoint{1.019289in}{2.198153in}}%
\pgfpathlineto{\pgfqpoint{1.035232in}{2.218780in}}%
\pgfpathlineto{\pgfqpoint{1.050845in}{2.239408in}}%
\pgfpathlineto{\pgfqpoint{1.057283in}{2.247968in}}%
\pgfpathlineto{\pgfqpoint{1.067653in}{2.260036in}}%
\pgfpathlineto{\pgfqpoint{1.085141in}{2.280664in}}%
\pgfpathlineto{\pgfqpoint{1.098328in}{2.296492in}}%
\pgfpathlineto{\pgfqpoint{1.103172in}{2.301292in}}%
\pgfpathlineto{\pgfqpoint{1.123860in}{2.321920in}}%
\pgfpathlineto{\pgfqpoint{1.139373in}{2.337697in}}%
\pgfpathlineto{\pgfqpoint{1.145448in}{2.342547in}}%
\pgfpathlineto{\pgfqpoint{1.171103in}{2.363175in}}%
\pgfpathlineto{\pgfqpoint{1.180419in}{2.370742in}}%
\pgfpathlineto{\pgfqpoint{1.202177in}{2.383803in}}%
\pgfpathlineto{\pgfqpoint{1.221464in}{2.395437in}}%
\pgfpathlineto{\pgfqpoint{1.244225in}{2.404431in}}%
\pgfpathlineto{\pgfqpoint{1.262509in}{2.411597in}}%
\pgfpathlineto{\pgfqpoint{1.303555in}{2.418687in}}%
\pgfpathlineto{\pgfqpoint{1.344600in}{2.415906in}}%
\pgfpathlineto{\pgfqpoint{1.380480in}{2.404431in}}%
\pgfpathlineto{\pgfqpoint{1.385645in}{2.402763in}}%
\pgfpathlineto{\pgfqpoint{1.419080in}{2.383803in}}%
\pgfpathlineto{\pgfqpoint{1.426691in}{2.379474in}}%
\pgfpathlineto{\pgfqpoint{1.447910in}{2.363175in}}%
\pgfpathlineto{\pgfqpoint{1.467736in}{2.347943in}}%
\pgfpathlineto{\pgfqpoint{1.473904in}{2.342547in}}%
\pgfpathlineto{\pgfqpoint{1.497288in}{2.321920in}}%
\pgfpathlineto{\pgfqpoint{1.508781in}{2.311728in}}%
\pgfpathlineto{\pgfqpoint{1.520912in}{2.301292in}}%
\pgfpathlineto{\pgfqpoint{1.544670in}{2.280664in}}%
\pgfpathlineto{\pgfqpoint{1.549827in}{2.276128in}}%
\pgfpathlineto{\pgfqpoint{1.573000in}{2.260036in}}%
\pgfpathlineto{\pgfqpoint{1.590872in}{2.247428in}}%
\pgfpathlineto{\pgfqpoint{1.612650in}{2.239408in}}%
\pgfpathlineto{\pgfqpoint{1.631917in}{2.232110in}}%
\pgfpathlineto{\pgfqpoint{1.672963in}{2.235170in}}%
\pgfpathlineto{\pgfqpoint{1.680332in}{2.239408in}}%
\pgfpathlineto{\pgfqpoint{1.714008in}{2.258749in}}%
\pgfpathlineto{\pgfqpoint{1.715243in}{2.260036in}}%
\pgfpathlineto{\pgfqpoint{1.735195in}{2.280664in}}%
\pgfpathlineto{\pgfqpoint{1.754926in}{2.301292in}}%
\pgfpathlineto{\pgfqpoint{1.755053in}{2.301423in}}%
\pgfpathlineto{\pgfqpoint{1.769847in}{2.321920in}}%
\pgfpathlineto{\pgfqpoint{1.784600in}{2.342547in}}%
\pgfpathlineto{\pgfqpoint{1.796099in}{2.358704in}}%
\pgfpathlineto{\pgfqpoint{1.798893in}{2.363175in}}%
\pgfpathlineto{\pgfqpoint{1.811852in}{2.383803in}}%
\pgfpathlineto{\pgfqpoint{1.824688in}{2.404431in}}%
\pgfpathlineto{\pgfqpoint{1.837144in}{2.424622in}}%
\pgfpathlineto{\pgfqpoint{1.837408in}{2.425059in}}%
\pgfpathlineto{\pgfqpoint{1.878189in}{2.425059in}}%
\pgfpathlineto{\pgfqpoint{1.882354in}{2.425059in}}%
\pgfpathlineto{\pgfqpoint{1.878189in}{2.418350in}}%
\pgfpathlineto{\pgfqpoint{1.870144in}{2.404431in}}%
\pgfpathlineto{\pgfqpoint{1.858178in}{2.383803in}}%
\pgfpathlineto{\pgfqpoint{1.846088in}{2.363175in}}%
\pgfpathlineto{\pgfqpoint{1.837144in}{2.347996in}}%
\pgfpathlineto{\pgfqpoint{1.833882in}{2.342547in}}%
\pgfpathlineto{\pgfqpoint{1.821603in}{2.321920in}}%
\pgfpathlineto{\pgfqpoint{1.809217in}{2.301292in}}%
\pgfpathlineto{\pgfqpoint{1.796718in}{2.280664in}}%
\pgfpathlineto{\pgfqpoint{1.796099in}{2.279631in}}%
\pgfpathlineto{\pgfqpoint{1.782736in}{2.260036in}}%
\pgfpathlineto{\pgfqpoint{1.768566in}{2.239408in}}%
\pgfpathlineto{\pgfqpoint{1.755053in}{2.219889in}}%
\pgfpathlineto{\pgfqpoint{1.754024in}{2.218780in}}%
\pgfpathlineto{\pgfqpoint{1.735053in}{2.198153in}}%
\pgfpathlineto{\pgfqpoint{1.715875in}{2.177525in}}%
\pgfpathlineto{\pgfqpoint{1.714008in}{2.175499in}}%
\pgfpathlineto{\pgfqpoint{1.682310in}{2.156897in}}%
\pgfpathlineto{\pgfqpoint{1.672963in}{2.151404in}}%
\pgfpathlineto{\pgfqpoint{1.631917in}{2.149208in}}%
\pgfpathlineto{\pgfqpoint{1.613928in}{2.156897in}}%
\pgfpathlineto{\pgfqpoint{1.590872in}{2.166442in}}%
\pgfpathlineto{\pgfqpoint{1.576432in}{2.177525in}}%
\pgfpathlineto{\pgfqpoint{1.549827in}{2.197585in}}%
\pgfpathlineto{\pgfqpoint{1.549224in}{2.198153in}}%
\pgfpathlineto{\pgfqpoint{1.526936in}{2.218780in}}%
\pgfpathlineto{\pgfqpoint{1.508781in}{2.235455in}}%
\pgfpathlineto{\pgfqpoint{1.504560in}{2.239408in}}%
\pgfpathlineto{\pgfqpoint{1.482276in}{2.260036in}}%
\pgfpathlineto{\pgfqpoint{1.467736in}{2.273436in}}%
\pgfpathlineto{\pgfqpoint{1.458704in}{2.280664in}}%
\pgfpathlineto{\pgfqpoint{1.432807in}{2.301292in}}%
\pgfpathlineto{\pgfqpoint{1.426691in}{2.306132in}}%
\pgfpathlineto{\pgfqpoint{1.399461in}{2.321920in}}%
\pgfpathlineto{\pgfqpoint{1.385645in}{2.329902in}}%
\pgfpathlineto{\pgfqpoint{1.346569in}{2.342547in}}%
\pgfpathlineto{\pgfqpoint{1.344600in}{2.343178in}}%
\pgfpathlineto{\pgfqpoint{1.303555in}{2.345663in}}%
\pgfpathlineto{\pgfqpoint{1.286060in}{2.342547in}}%
\pgfpathlineto{\pgfqpoint{1.262509in}{2.338255in}}%
\pgfpathlineto{\pgfqpoint{1.221464in}{2.321956in}}%
\pgfpathlineto{\pgfqpoint{1.221404in}{2.321920in}}%
\pgfpathlineto{\pgfqpoint{1.187252in}{2.301292in}}%
\pgfpathlineto{\pgfqpoint{1.180419in}{2.297186in}}%
\pgfpathlineto{\pgfqpoint{1.160052in}{2.280664in}}%
\pgfpathlineto{\pgfqpoint{1.139373in}{2.264193in}}%
\pgfpathlineto{\pgfqpoint{1.135292in}{2.260036in}}%
\pgfpathlineto{\pgfqpoint{1.114928in}{2.239408in}}%
\pgfpathlineto{\pgfqpoint{1.098328in}{2.222975in}}%
\pgfpathlineto{\pgfqpoint{1.094853in}{2.218780in}}%
\pgfpathlineto{\pgfqpoint{1.077706in}{2.198153in}}%
\pgfpathlineto{\pgfqpoint{1.060112in}{2.177525in}}%
\pgfpathlineto{\pgfqpoint{1.057283in}{2.174218in}}%
\pgfpathlineto{\pgfqpoint{1.044439in}{2.156897in}}%
\pgfpathlineto{\pgfqpoint{1.028870in}{2.136269in}}%
\pgfpathlineto{\pgfqpoint{1.016237in}{2.119821in}}%
\pgfpathlineto{\pgfqpoint{1.013242in}{2.115641in}}%
\pgfpathlineto{\pgfqpoint{0.998466in}{2.095013in}}%
\pgfpathlineto{\pgfqpoint{0.983405in}{2.074386in}}%
\pgfpathlineto{\pgfqpoint{0.975192in}{2.063234in}}%
\pgfpathlineto{\pgfqpoint{0.968056in}{2.053758in}}%
\pgfpathlineto{\pgfqpoint{0.952456in}{2.033130in}}%
\pgfpathlineto{\pgfqpoint{0.936579in}{2.012502in}}%
\pgfpathlineto{\pgfqpoint{0.934147in}{2.009334in}}%
\pgfpathlineto{\pgfqpoint{0.918669in}{1.991874in}}%
\pgfpathlineto{\pgfqpoint{0.900138in}{1.971247in}}%
\pgfpathlineto{\pgfqpoint{0.893101in}{1.963434in}}%
\pgfpathlineto{\pgfqpoint{0.877155in}{1.950619in}}%
\pgfpathlineto{\pgfqpoint{0.852056in}{1.930640in}}%
\pgfpathlineto{\pgfqpoint{0.850462in}{1.929991in}}%
\pgfpathlineto{\pgfqpoint{0.811011in}{1.913848in}}%
\pgfpathlineto{\pgfqpoint{0.769965in}{1.913499in}}%
\pgfpathlineto{\pgfqpoint{0.728920in}{1.926876in}}%
\pgfpathlineto{\pgfqpoint{0.723245in}{1.929991in}}%
\pgfpathlineto{\pgfqpoint{0.687875in}{1.949135in}}%
\pgfpathlineto{\pgfqpoint{0.685495in}{1.950619in}}%
\pgfpathlineto{\pgfqpoint{0.652131in}{1.971247in}}%
\pgfpathlineto{\pgfqpoint{0.646829in}{1.974503in}}%
\pgfpathclose%
\pgfusepath{stroke,fill}%
\end{pgfscope}%
\begin{pgfscope}%
\pgfpathrectangle{\pgfqpoint{0.605784in}{0.382904in}}{\pgfqpoint{4.063488in}{2.042155in}}%
\pgfusepath{clip}%
\pgfsetbuttcap%
\pgfsetroundjoin%
\definecolor{currentfill}{rgb}{0.166617,0.463708,0.558119}%
\pgfsetfillcolor{currentfill}%
\pgfsetlinewidth{1.003750pt}%
\definecolor{currentstroke}{rgb}{0.166617,0.463708,0.558119}%
\pgfsetstrokecolor{currentstroke}%
\pgfsetdash{}{0pt}%
\pgfpathmoveto{\pgfqpoint{2.447388in}{2.404431in}}%
\pgfpathlineto{\pgfqpoint{2.425912in}{2.425059in}}%
\pgfpathlineto{\pgfqpoint{2.452824in}{2.425059in}}%
\pgfpathlineto{\pgfqpoint{2.493869in}{2.425059in}}%
\pgfpathlineto{\pgfqpoint{2.498223in}{2.425059in}}%
\pgfpathlineto{\pgfqpoint{2.500342in}{2.404431in}}%
\pgfpathlineto{\pgfqpoint{2.502535in}{2.383803in}}%
\pgfpathlineto{\pgfqpoint{2.504811in}{2.363175in}}%
\pgfpathlineto{\pgfqpoint{2.507178in}{2.342547in}}%
\pgfpathlineto{\pgfqpoint{2.509647in}{2.321920in}}%
\pgfpathlineto{\pgfqpoint{2.512229in}{2.301292in}}%
\pgfpathlineto{\pgfqpoint{2.514939in}{2.280664in}}%
\pgfpathlineto{\pgfqpoint{2.517794in}{2.260036in}}%
\pgfpathlineto{\pgfqpoint{2.520812in}{2.239408in}}%
\pgfpathlineto{\pgfqpoint{2.524016in}{2.218780in}}%
\pgfpathlineto{\pgfqpoint{2.527434in}{2.198153in}}%
\pgfpathlineto{\pgfqpoint{2.531100in}{2.177525in}}%
\pgfpathlineto{\pgfqpoint{2.534915in}{2.157582in}}%
\pgfpathlineto{\pgfqpoint{2.575960in}{2.157582in}}%
\pgfpathlineto{\pgfqpoint{2.617005in}{2.157582in}}%
\pgfpathlineto{\pgfqpoint{2.658051in}{2.157582in}}%
\pgfpathlineto{\pgfqpoint{2.699096in}{2.157582in}}%
\pgfpathlineto{\pgfqpoint{2.740141in}{2.157582in}}%
\pgfpathlineto{\pgfqpoint{2.781187in}{2.157582in}}%
\pgfpathlineto{\pgfqpoint{2.822232in}{2.157582in}}%
\pgfpathlineto{\pgfqpoint{2.863277in}{2.157582in}}%
\pgfpathlineto{\pgfqpoint{2.904323in}{2.157582in}}%
\pgfpathlineto{\pgfqpoint{2.945368in}{2.157582in}}%
\pgfpathlineto{\pgfqpoint{2.986413in}{2.157582in}}%
\pgfpathlineto{\pgfqpoint{3.027459in}{2.157582in}}%
\pgfpathlineto{\pgfqpoint{3.068504in}{2.157582in}}%
\pgfpathlineto{\pgfqpoint{3.109549in}{2.157582in}}%
\pgfpathlineto{\pgfqpoint{3.150595in}{2.157582in}}%
\pgfpathlineto{\pgfqpoint{3.191640in}{2.157582in}}%
\pgfpathlineto{\pgfqpoint{3.232685in}{2.157582in}}%
\pgfpathlineto{\pgfqpoint{3.273731in}{2.157582in}}%
\pgfpathlineto{\pgfqpoint{3.314776in}{2.157582in}}%
\pgfpathlineto{\pgfqpoint{3.355821in}{2.157582in}}%
\pgfpathlineto{\pgfqpoint{3.396867in}{2.157582in}}%
\pgfpathlineto{\pgfqpoint{3.437912in}{2.157582in}}%
\pgfpathlineto{\pgfqpoint{3.478957in}{2.157582in}}%
\pgfpathlineto{\pgfqpoint{3.520003in}{2.157582in}}%
\pgfpathlineto{\pgfqpoint{3.561048in}{2.157582in}}%
\pgfpathlineto{\pgfqpoint{3.602093in}{2.157582in}}%
\pgfpathlineto{\pgfqpoint{3.643139in}{2.157582in}}%
\pgfpathlineto{\pgfqpoint{3.643253in}{2.156897in}}%
\pgfpathlineto{\pgfqpoint{3.646694in}{2.136269in}}%
\pgfpathlineto{\pgfqpoint{3.650210in}{2.115641in}}%
\pgfpathlineto{\pgfqpoint{3.653806in}{2.095013in}}%
\pgfpathlineto{\pgfqpoint{3.657489in}{2.074386in}}%
\pgfpathlineto{\pgfqpoint{3.661266in}{2.053758in}}%
\pgfpathlineto{\pgfqpoint{3.665145in}{2.033130in}}%
\pgfpathlineto{\pgfqpoint{3.669134in}{2.012502in}}%
\pgfpathlineto{\pgfqpoint{3.673243in}{1.991874in}}%
\pgfpathlineto{\pgfqpoint{3.677484in}{1.971247in}}%
\pgfpathlineto{\pgfqpoint{3.681868in}{1.950619in}}%
\pgfpathlineto{\pgfqpoint{3.684184in}{1.939935in}}%
\pgfpathlineto{\pgfqpoint{3.725229in}{1.939935in}}%
\pgfpathlineto{\pgfqpoint{3.766275in}{1.939935in}}%
\pgfpathlineto{\pgfqpoint{3.807320in}{1.939935in}}%
\pgfpathlineto{\pgfqpoint{3.848365in}{1.939935in}}%
\pgfpathlineto{\pgfqpoint{3.889411in}{1.939935in}}%
\pgfpathlineto{\pgfqpoint{3.930456in}{1.939935in}}%
\pgfpathlineto{\pgfqpoint{3.971501in}{1.939935in}}%
\pgfpathlineto{\pgfqpoint{4.012547in}{1.939935in}}%
\pgfpathlineto{\pgfqpoint{4.053592in}{1.939935in}}%
\pgfpathlineto{\pgfqpoint{4.094637in}{1.939935in}}%
\pgfpathlineto{\pgfqpoint{4.135683in}{1.939935in}}%
\pgfpathlineto{\pgfqpoint{4.176728in}{1.939935in}}%
\pgfpathlineto{\pgfqpoint{4.217773in}{1.939935in}}%
\pgfpathlineto{\pgfqpoint{4.258819in}{1.939935in}}%
\pgfpathlineto{\pgfqpoint{4.299864in}{1.939935in}}%
\pgfpathlineto{\pgfqpoint{4.340909in}{1.939935in}}%
\pgfpathlineto{\pgfqpoint{4.381955in}{1.939935in}}%
\pgfpathlineto{\pgfqpoint{4.423000in}{1.939935in}}%
\pgfpathlineto{\pgfqpoint{4.464045in}{1.939935in}}%
\pgfpathlineto{\pgfqpoint{4.505091in}{1.939935in}}%
\pgfpathlineto{\pgfqpoint{4.546136in}{1.939935in}}%
\pgfpathlineto{\pgfqpoint{4.587181in}{1.939935in}}%
\pgfpathlineto{\pgfqpoint{4.628227in}{1.939935in}}%
\pgfpathlineto{\pgfqpoint{4.669272in}{1.939935in}}%
\pgfpathlineto{\pgfqpoint{4.669272in}{1.929991in}}%
\pgfpathlineto{\pgfqpoint{4.669272in}{1.909363in}}%
\pgfpathlineto{\pgfqpoint{4.669272in}{1.888735in}}%
\pgfpathlineto{\pgfqpoint{4.669272in}{1.868107in}}%
\pgfpathlineto{\pgfqpoint{4.669272in}{1.863329in}}%
\pgfpathlineto{\pgfqpoint{4.628227in}{1.863329in}}%
\pgfpathlineto{\pgfqpoint{4.587181in}{1.863329in}}%
\pgfpathlineto{\pgfqpoint{4.546136in}{1.863329in}}%
\pgfpathlineto{\pgfqpoint{4.505091in}{1.863329in}}%
\pgfpathlineto{\pgfqpoint{4.464045in}{1.863329in}}%
\pgfpathlineto{\pgfqpoint{4.423000in}{1.863329in}}%
\pgfpathlineto{\pgfqpoint{4.381955in}{1.863329in}}%
\pgfpathlineto{\pgfqpoint{4.340909in}{1.863329in}}%
\pgfpathlineto{\pgfqpoint{4.299864in}{1.863329in}}%
\pgfpathlineto{\pgfqpoint{4.258819in}{1.863329in}}%
\pgfpathlineto{\pgfqpoint{4.217773in}{1.863329in}}%
\pgfpathlineto{\pgfqpoint{4.176728in}{1.863329in}}%
\pgfpathlineto{\pgfqpoint{4.135683in}{1.863329in}}%
\pgfpathlineto{\pgfqpoint{4.094637in}{1.863329in}}%
\pgfpathlineto{\pgfqpoint{4.053592in}{1.863329in}}%
\pgfpathlineto{\pgfqpoint{4.012547in}{1.863329in}}%
\pgfpathlineto{\pgfqpoint{3.971501in}{1.863329in}}%
\pgfpathlineto{\pgfqpoint{3.930456in}{1.863329in}}%
\pgfpathlineto{\pgfqpoint{3.889411in}{1.863329in}}%
\pgfpathlineto{\pgfqpoint{3.848365in}{1.863329in}}%
\pgfpathlineto{\pgfqpoint{3.807320in}{1.863329in}}%
\pgfpathlineto{\pgfqpoint{3.766275in}{1.863329in}}%
\pgfpathlineto{\pgfqpoint{3.725229in}{1.863329in}}%
\pgfpathlineto{\pgfqpoint{3.684184in}{1.863329in}}%
\pgfpathlineto{\pgfqpoint{3.683129in}{1.868107in}}%
\pgfpathlineto{\pgfqpoint{3.678633in}{1.888735in}}%
\pgfpathlineto{\pgfqpoint{3.674308in}{1.909363in}}%
\pgfpathlineto{\pgfqpoint{3.670136in}{1.929991in}}%
\pgfpathlineto{\pgfqpoint{3.666103in}{1.950619in}}%
\pgfpathlineto{\pgfqpoint{3.662197in}{1.971247in}}%
\pgfpathlineto{\pgfqpoint{3.658406in}{1.991874in}}%
\pgfpathlineto{\pgfqpoint{3.654721in}{2.012502in}}%
\pgfpathlineto{\pgfqpoint{3.651132in}{2.033130in}}%
\pgfpathlineto{\pgfqpoint{3.647632in}{2.053758in}}%
\pgfpathlineto{\pgfqpoint{3.644214in}{2.074386in}}%
\pgfpathlineto{\pgfqpoint{3.643139in}{2.080903in}}%
\pgfpathlineto{\pgfqpoint{3.602093in}{2.080903in}}%
\pgfpathlineto{\pgfqpoint{3.561048in}{2.080903in}}%
\pgfpathlineto{\pgfqpoint{3.520003in}{2.080903in}}%
\pgfpathlineto{\pgfqpoint{3.478957in}{2.080903in}}%
\pgfpathlineto{\pgfqpoint{3.437912in}{2.080903in}}%
\pgfpathlineto{\pgfqpoint{3.396867in}{2.080903in}}%
\pgfpathlineto{\pgfqpoint{3.355821in}{2.080903in}}%
\pgfpathlineto{\pgfqpoint{3.314776in}{2.080903in}}%
\pgfpathlineto{\pgfqpoint{3.273731in}{2.080903in}}%
\pgfpathlineto{\pgfqpoint{3.232685in}{2.080903in}}%
\pgfpathlineto{\pgfqpoint{3.191640in}{2.080903in}}%
\pgfpathlineto{\pgfqpoint{3.150595in}{2.080903in}}%
\pgfpathlineto{\pgfqpoint{3.109549in}{2.080903in}}%
\pgfpathlineto{\pgfqpoint{3.068504in}{2.080903in}}%
\pgfpathlineto{\pgfqpoint{3.027459in}{2.080903in}}%
\pgfpathlineto{\pgfqpoint{2.986413in}{2.080903in}}%
\pgfpathlineto{\pgfqpoint{2.945368in}{2.080903in}}%
\pgfpathlineto{\pgfqpoint{2.904323in}{2.080903in}}%
\pgfpathlineto{\pgfqpoint{2.863277in}{2.080903in}}%
\pgfpathlineto{\pgfqpoint{2.822232in}{2.080903in}}%
\pgfpathlineto{\pgfqpoint{2.781187in}{2.080903in}}%
\pgfpathlineto{\pgfqpoint{2.740141in}{2.080903in}}%
\pgfpathlineto{\pgfqpoint{2.699096in}{2.080903in}}%
\pgfpathlineto{\pgfqpoint{2.658051in}{2.080903in}}%
\pgfpathlineto{\pgfqpoint{2.617005in}{2.080903in}}%
\pgfpathlineto{\pgfqpoint{2.575960in}{2.080903in}}%
\pgfpathlineto{\pgfqpoint{2.534915in}{2.080903in}}%
\pgfpathlineto{\pgfqpoint{2.531878in}{2.095013in}}%
\pgfpathlineto{\pgfqpoint{2.527771in}{2.115641in}}%
\pgfpathlineto{\pgfqpoint{2.524018in}{2.136269in}}%
\pgfpathlineto{\pgfqpoint{2.520563in}{2.156897in}}%
\pgfpathlineto{\pgfqpoint{2.517358in}{2.177525in}}%
\pgfpathlineto{\pgfqpoint{2.514368in}{2.198153in}}%
\pgfpathlineto{\pgfqpoint{2.511561in}{2.218780in}}%
\pgfpathlineto{\pgfqpoint{2.508914in}{2.239408in}}%
\pgfpathlineto{\pgfqpoint{2.506405in}{2.260036in}}%
\pgfpathlineto{\pgfqpoint{2.504018in}{2.280664in}}%
\pgfpathlineto{\pgfqpoint{2.501738in}{2.301292in}}%
\pgfpathlineto{\pgfqpoint{2.499554in}{2.321920in}}%
\pgfpathlineto{\pgfqpoint{2.497454in}{2.342547in}}%
\pgfpathlineto{\pgfqpoint{2.495430in}{2.363175in}}%
\pgfpathlineto{\pgfqpoint{2.493869in}{2.379517in}}%
\pgfpathlineto{\pgfqpoint{2.485146in}{2.383803in}}%
\pgfpathlineto{\pgfqpoint{2.452824in}{2.399066in}}%
\pgfpathclose%
\pgfusepath{stroke,fill}%
\end{pgfscope}%
\begin{pgfscope}%
\pgfpathrectangle{\pgfqpoint{0.605784in}{0.382904in}}{\pgfqpoint{4.063488in}{2.042155in}}%
\pgfusepath{clip}%
\pgfsetbuttcap%
\pgfsetroundjoin%
\definecolor{currentfill}{rgb}{0.140536,0.530132,0.555659}%
\pgfsetfillcolor{currentfill}%
\pgfsetlinewidth{1.003750pt}%
\definecolor{currentstroke}{rgb}{0.140536,0.530132,0.555659}%
\pgfsetstrokecolor{currentstroke}%
\pgfsetdash{}{0pt}%
\pgfpathmoveto{\pgfqpoint{0.646453in}{0.424160in}}%
\pgfpathlineto{\pgfqpoint{0.605784in}{0.441557in}}%
\pgfpathlineto{\pgfqpoint{0.605784in}{0.444787in}}%
\pgfpathlineto{\pgfqpoint{0.605784in}{0.465415in}}%
\pgfpathlineto{\pgfqpoint{0.605784in}{0.486043in}}%
\pgfpathlineto{\pgfqpoint{0.605784in}{0.506671in}}%
\pgfpathlineto{\pgfqpoint{0.605784in}{0.509225in}}%
\pgfpathlineto{\pgfqpoint{0.611848in}{0.506671in}}%
\pgfpathlineto{\pgfqpoint{0.646829in}{0.491949in}}%
\pgfpathlineto{\pgfqpoint{0.672367in}{0.486043in}}%
\pgfpathlineto{\pgfqpoint{0.687875in}{0.482474in}}%
\pgfpathlineto{\pgfqpoint{0.728920in}{0.481364in}}%
\pgfpathlineto{\pgfqpoint{0.759476in}{0.486043in}}%
\pgfpathlineto{\pgfqpoint{0.769965in}{0.487700in}}%
\pgfpathlineto{\pgfqpoint{0.811011in}{0.500144in}}%
\pgfpathlineto{\pgfqpoint{0.826548in}{0.506671in}}%
\pgfpathlineto{\pgfqpoint{0.852056in}{0.517438in}}%
\pgfpathlineto{\pgfqpoint{0.870492in}{0.527299in}}%
\pgfpathlineto{\pgfqpoint{0.893101in}{0.539288in}}%
\pgfpathlineto{\pgfqpoint{0.906130in}{0.547927in}}%
\pgfpathlineto{\pgfqpoint{0.934147in}{0.566132in}}%
\pgfpathlineto{\pgfqpoint{0.937145in}{0.568554in}}%
\pgfpathlineto{\pgfqpoint{0.962991in}{0.589182in}}%
\pgfpathlineto{\pgfqpoint{0.975192in}{0.598680in}}%
\pgfpathlineto{\pgfqpoint{0.986976in}{0.609810in}}%
\pgfpathlineto{\pgfqpoint{1.009337in}{0.630438in}}%
\pgfpathlineto{\pgfqpoint{1.016237in}{0.636698in}}%
\pgfpathlineto{\pgfqpoint{1.029965in}{0.651066in}}%
\pgfpathlineto{\pgfqpoint{1.050158in}{0.671694in}}%
\pgfpathlineto{\pgfqpoint{1.057283in}{0.678866in}}%
\pgfpathlineto{\pgfqpoint{1.069768in}{0.692321in}}%
\pgfpathlineto{\pgfqpoint{1.089283in}{0.712949in}}%
\pgfpathlineto{\pgfqpoint{1.098328in}{0.722371in}}%
\pgfpathlineto{\pgfqpoint{1.109407in}{0.733577in}}%
\pgfpathlineto{\pgfqpoint{1.130091in}{0.754205in}}%
\pgfpathlineto{\pgfqpoint{1.139373in}{0.763351in}}%
\pgfpathlineto{\pgfqpoint{1.152991in}{0.774833in}}%
\pgfpathlineto{\pgfqpoint{1.177796in}{0.795460in}}%
\pgfpathlineto{\pgfqpoint{1.180419in}{0.797636in}}%
\pgfpathlineto{\pgfqpoint{1.211575in}{0.816088in}}%
\pgfpathlineto{\pgfqpoint{1.221464in}{0.821916in}}%
\pgfpathlineto{\pgfqpoint{1.262509in}{0.834294in}}%
\pgfpathlineto{\pgfqpoint{1.303555in}{0.835272in}}%
\pgfpathlineto{\pgfqpoint{1.344600in}{0.827451in}}%
\pgfpathlineto{\pgfqpoint{1.381167in}{0.816088in}}%
\pgfpathlineto{\pgfqpoint{1.385645in}{0.814710in}}%
\pgfpathlineto{\pgfqpoint{1.426691in}{0.801759in}}%
\pgfpathlineto{\pgfqpoint{1.454428in}{0.795460in}}%
\pgfpathlineto{\pgfqpoint{1.467736in}{0.792439in}}%
\pgfpathlineto{\pgfqpoint{1.508781in}{0.789501in}}%
\pgfpathlineto{\pgfqpoint{1.549827in}{0.793666in}}%
\pgfpathlineto{\pgfqpoint{1.557142in}{0.795460in}}%
\pgfpathlineto{\pgfqpoint{1.590872in}{0.804078in}}%
\pgfpathlineto{\pgfqpoint{1.625644in}{0.816088in}}%
\pgfpathlineto{\pgfqpoint{1.631917in}{0.818321in}}%
\pgfpathlineto{\pgfqpoint{1.672963in}{0.834132in}}%
\pgfpathlineto{\pgfqpoint{1.679596in}{0.836716in}}%
\pgfpathlineto{\pgfqpoint{1.714008in}{0.850139in}}%
\pgfpathlineto{\pgfqpoint{1.731869in}{0.857344in}}%
\pgfpathlineto{\pgfqpoint{1.755053in}{0.866580in}}%
\pgfpathlineto{\pgfqpoint{1.780157in}{0.877972in}}%
\pgfpathlineto{\pgfqpoint{1.796099in}{0.885036in}}%
\pgfpathlineto{\pgfqpoint{1.820849in}{0.898600in}}%
\pgfpathlineto{\pgfqpoint{1.837144in}{0.907253in}}%
\pgfpathlineto{\pgfqpoint{1.855170in}{0.919227in}}%
\pgfpathlineto{\pgfqpoint{1.878189in}{0.933996in}}%
\pgfpathlineto{\pgfqpoint{1.885872in}{0.939855in}}%
\pgfpathlineto{\pgfqpoint{1.913627in}{0.960483in}}%
\pgfpathlineto{\pgfqpoint{1.919235in}{0.964555in}}%
\pgfpathlineto{\pgfqpoint{1.940034in}{0.981111in}}%
\pgfpathlineto{\pgfqpoint{1.960280in}{0.996744in}}%
\pgfpathlineto{\pgfqpoint{1.966873in}{1.001739in}}%
\pgfpathlineto{\pgfqpoint{1.994513in}{1.022367in}}%
\pgfpathlineto{\pgfqpoint{2.001325in}{1.027373in}}%
\pgfpathlineto{\pgfqpoint{2.026049in}{1.042994in}}%
\pgfpathlineto{\pgfqpoint{2.042371in}{1.053172in}}%
\pgfpathlineto{\pgfqpoint{2.065630in}{1.063622in}}%
\pgfpathlineto{\pgfqpoint{2.083416in}{1.071602in}}%
\pgfpathlineto{\pgfqpoint{2.124461in}{1.081683in}}%
\pgfpathlineto{\pgfqpoint{2.164003in}{1.084250in}}%
\pgfpathlineto{\pgfqpoint{2.165507in}{1.084350in}}%
\pgfpathlineto{\pgfqpoint{2.167413in}{1.084250in}}%
\pgfpathlineto{\pgfqpoint{2.206552in}{1.082231in}}%
\pgfpathlineto{\pgfqpoint{2.247597in}{1.078910in}}%
\pgfpathlineto{\pgfqpoint{2.288643in}{1.078126in}}%
\pgfpathlineto{\pgfqpoint{2.329688in}{1.082816in}}%
\pgfpathlineto{\pgfqpoint{2.334972in}{1.084250in}}%
\pgfpathlineto{\pgfqpoint{2.370733in}{1.094477in}}%
\pgfpathlineto{\pgfqpoint{2.395241in}{1.104878in}}%
\pgfpathlineto{\pgfqpoint{2.411779in}{1.112245in}}%
\pgfpathlineto{\pgfqpoint{2.437820in}{1.125506in}}%
\pgfpathlineto{\pgfqpoint{2.452824in}{1.133460in}}%
\pgfpathlineto{\pgfqpoint{2.478161in}{1.146134in}}%
\pgfpathlineto{\pgfqpoint{2.493869in}{1.154214in}}%
\pgfpathlineto{\pgfqpoint{2.494703in}{1.146134in}}%
\pgfpathlineto{\pgfqpoint{2.496803in}{1.125506in}}%
\pgfpathlineto{\pgfqpoint{2.498836in}{1.104878in}}%
\pgfpathlineto{\pgfqpoint{2.500810in}{1.084250in}}%
\pgfpathlineto{\pgfqpoint{2.502730in}{1.063622in}}%
\pgfpathlineto{\pgfqpoint{2.504599in}{1.042994in}}%
\pgfpathlineto{\pgfqpoint{2.506422in}{1.022367in}}%
\pgfpathlineto{\pgfqpoint{2.508202in}{1.001739in}}%
\pgfpathlineto{\pgfqpoint{2.509944in}{0.981111in}}%
\pgfpathlineto{\pgfqpoint{2.511650in}{0.960483in}}%
\pgfpathlineto{\pgfqpoint{2.513322in}{0.939855in}}%
\pgfpathlineto{\pgfqpoint{2.514963in}{0.919227in}}%
\pgfpathlineto{\pgfqpoint{2.516575in}{0.898600in}}%
\pgfpathlineto{\pgfqpoint{2.518160in}{0.877972in}}%
\pgfpathlineto{\pgfqpoint{2.519721in}{0.857344in}}%
\pgfpathlineto{\pgfqpoint{2.521258in}{0.836716in}}%
\pgfpathlineto{\pgfqpoint{2.522773in}{0.816088in}}%
\pgfpathlineto{\pgfqpoint{2.524267in}{0.795460in}}%
\pgfpathlineto{\pgfqpoint{2.525742in}{0.774833in}}%
\pgfpathlineto{\pgfqpoint{2.527200in}{0.754205in}}%
\pgfpathlineto{\pgfqpoint{2.528640in}{0.733577in}}%
\pgfpathlineto{\pgfqpoint{2.530064in}{0.712949in}}%
\pgfpathlineto{\pgfqpoint{2.531473in}{0.692321in}}%
\pgfpathlineto{\pgfqpoint{2.532867in}{0.671694in}}%
\pgfpathlineto{\pgfqpoint{2.534249in}{0.651066in}}%
\pgfpathlineto{\pgfqpoint{2.534915in}{0.641121in}}%
\pgfpathlineto{\pgfqpoint{2.575960in}{0.641121in}}%
\pgfpathlineto{\pgfqpoint{2.617005in}{0.641121in}}%
\pgfpathlineto{\pgfqpoint{2.658051in}{0.641121in}}%
\pgfpathlineto{\pgfqpoint{2.699096in}{0.641121in}}%
\pgfpathlineto{\pgfqpoint{2.740141in}{0.641121in}}%
\pgfpathlineto{\pgfqpoint{2.781187in}{0.641121in}}%
\pgfpathlineto{\pgfqpoint{2.822232in}{0.641121in}}%
\pgfpathlineto{\pgfqpoint{2.863277in}{0.641121in}}%
\pgfpathlineto{\pgfqpoint{2.904323in}{0.641121in}}%
\pgfpathlineto{\pgfqpoint{2.945368in}{0.641121in}}%
\pgfpathlineto{\pgfqpoint{2.986413in}{0.641121in}}%
\pgfpathlineto{\pgfqpoint{3.027459in}{0.641121in}}%
\pgfpathlineto{\pgfqpoint{3.068504in}{0.641121in}}%
\pgfpathlineto{\pgfqpoint{3.109549in}{0.641121in}}%
\pgfpathlineto{\pgfqpoint{3.150595in}{0.641121in}}%
\pgfpathlineto{\pgfqpoint{3.191640in}{0.641121in}}%
\pgfpathlineto{\pgfqpoint{3.232685in}{0.641121in}}%
\pgfpathlineto{\pgfqpoint{3.273731in}{0.641121in}}%
\pgfpathlineto{\pgfqpoint{3.314776in}{0.641121in}}%
\pgfpathlineto{\pgfqpoint{3.355821in}{0.641121in}}%
\pgfpathlineto{\pgfqpoint{3.396867in}{0.641121in}}%
\pgfpathlineto{\pgfqpoint{3.437912in}{0.641121in}}%
\pgfpathlineto{\pgfqpoint{3.478957in}{0.641121in}}%
\pgfpathlineto{\pgfqpoint{3.520003in}{0.641121in}}%
\pgfpathlineto{\pgfqpoint{3.561048in}{0.641121in}}%
\pgfpathlineto{\pgfqpoint{3.602093in}{0.641121in}}%
\pgfpathlineto{\pgfqpoint{3.643139in}{0.641121in}}%
\pgfpathlineto{\pgfqpoint{3.645454in}{0.630438in}}%
\pgfpathlineto{\pgfqpoint{3.649838in}{0.609810in}}%
\pgfpathlineto{\pgfqpoint{3.654079in}{0.589182in}}%
\pgfpathlineto{\pgfqpoint{3.658188in}{0.568554in}}%
\pgfpathlineto{\pgfqpoint{3.662178in}{0.547927in}}%
\pgfpathlineto{\pgfqpoint{3.666056in}{0.527299in}}%
\pgfpathlineto{\pgfqpoint{3.669833in}{0.506671in}}%
\pgfpathlineto{\pgfqpoint{3.673516in}{0.486043in}}%
\pgfpathlineto{\pgfqpoint{3.677113in}{0.465415in}}%
\pgfpathlineto{\pgfqpoint{3.680628in}{0.444787in}}%
\pgfpathlineto{\pgfqpoint{3.684069in}{0.424160in}}%
\pgfpathlineto{\pgfqpoint{3.684184in}{0.423474in}}%
\pgfpathlineto{\pgfqpoint{3.725229in}{0.423474in}}%
\pgfpathlineto{\pgfqpoint{3.766275in}{0.423474in}}%
\pgfpathlineto{\pgfqpoint{3.807320in}{0.423474in}}%
\pgfpathlineto{\pgfqpoint{3.848365in}{0.423474in}}%
\pgfpathlineto{\pgfqpoint{3.889411in}{0.423474in}}%
\pgfpathlineto{\pgfqpoint{3.930456in}{0.423474in}}%
\pgfpathlineto{\pgfqpoint{3.971501in}{0.423474in}}%
\pgfpathlineto{\pgfqpoint{4.012547in}{0.423474in}}%
\pgfpathlineto{\pgfqpoint{4.053592in}{0.423474in}}%
\pgfpathlineto{\pgfqpoint{4.094637in}{0.423474in}}%
\pgfpathlineto{\pgfqpoint{4.135683in}{0.423474in}}%
\pgfpathlineto{\pgfqpoint{4.176728in}{0.423474in}}%
\pgfpathlineto{\pgfqpoint{4.217773in}{0.423474in}}%
\pgfpathlineto{\pgfqpoint{4.258819in}{0.423474in}}%
\pgfpathlineto{\pgfqpoint{4.299864in}{0.423474in}}%
\pgfpathlineto{\pgfqpoint{4.340909in}{0.423474in}}%
\pgfpathlineto{\pgfqpoint{4.381955in}{0.423474in}}%
\pgfpathlineto{\pgfqpoint{4.423000in}{0.423474in}}%
\pgfpathlineto{\pgfqpoint{4.464045in}{0.423474in}}%
\pgfpathlineto{\pgfqpoint{4.505091in}{0.423474in}}%
\pgfpathlineto{\pgfqpoint{4.546136in}{0.423474in}}%
\pgfpathlineto{\pgfqpoint{4.587181in}{0.423474in}}%
\pgfpathlineto{\pgfqpoint{4.628227in}{0.423474in}}%
\pgfpathlineto{\pgfqpoint{4.669272in}{0.423474in}}%
\pgfpathlineto{\pgfqpoint{4.669272in}{0.403532in}}%
\pgfpathlineto{\pgfqpoint{4.669272in}{0.382904in}}%
\pgfpathlineto{\pgfqpoint{4.628227in}{0.382904in}}%
\pgfpathlineto{\pgfqpoint{4.587181in}{0.382904in}}%
\pgfpathlineto{\pgfqpoint{4.546136in}{0.382904in}}%
\pgfpathlineto{\pgfqpoint{4.505091in}{0.382904in}}%
\pgfpathlineto{\pgfqpoint{4.464045in}{0.382904in}}%
\pgfpathlineto{\pgfqpoint{4.423000in}{0.382904in}}%
\pgfpathlineto{\pgfqpoint{4.381955in}{0.382904in}}%
\pgfpathlineto{\pgfqpoint{4.340909in}{0.382904in}}%
\pgfpathlineto{\pgfqpoint{4.299864in}{0.382904in}}%
\pgfpathlineto{\pgfqpoint{4.258819in}{0.382904in}}%
\pgfpathlineto{\pgfqpoint{4.217773in}{0.382904in}}%
\pgfpathlineto{\pgfqpoint{4.176728in}{0.382904in}}%
\pgfpathlineto{\pgfqpoint{4.135683in}{0.382904in}}%
\pgfpathlineto{\pgfqpoint{4.094637in}{0.382904in}}%
\pgfpathlineto{\pgfqpoint{4.053592in}{0.382904in}}%
\pgfpathlineto{\pgfqpoint{4.012547in}{0.382904in}}%
\pgfpathlineto{\pgfqpoint{3.971501in}{0.382904in}}%
\pgfpathlineto{\pgfqpoint{3.930456in}{0.382904in}}%
\pgfpathlineto{\pgfqpoint{3.889411in}{0.382904in}}%
\pgfpathlineto{\pgfqpoint{3.848365in}{0.382904in}}%
\pgfpathlineto{\pgfqpoint{3.807320in}{0.382904in}}%
\pgfpathlineto{\pgfqpoint{3.766275in}{0.382904in}}%
\pgfpathlineto{\pgfqpoint{3.725229in}{0.382904in}}%
\pgfpathlineto{\pgfqpoint{3.684184in}{0.382904in}}%
\pgfpathlineto{\pgfqpoint{3.679282in}{0.382904in}}%
\pgfpathlineto{\pgfqpoint{3.675709in}{0.403532in}}%
\pgfpathlineto{\pgfqpoint{3.672058in}{0.424160in}}%
\pgfpathlineto{\pgfqpoint{3.668324in}{0.444787in}}%
\pgfpathlineto{\pgfqpoint{3.664501in}{0.465415in}}%
\pgfpathlineto{\pgfqpoint{3.660581in}{0.486043in}}%
\pgfpathlineto{\pgfqpoint{3.656558in}{0.506671in}}%
\pgfpathlineto{\pgfqpoint{3.652422in}{0.527299in}}%
\pgfpathlineto{\pgfqpoint{3.648165in}{0.547927in}}%
\pgfpathlineto{\pgfqpoint{3.643775in}{0.568554in}}%
\pgfpathlineto{\pgfqpoint{3.643139in}{0.571523in}}%
\pgfpathlineto{\pgfqpoint{3.602093in}{0.571523in}}%
\pgfpathlineto{\pgfqpoint{3.561048in}{0.571523in}}%
\pgfpathlineto{\pgfqpoint{3.520003in}{0.571523in}}%
\pgfpathlineto{\pgfqpoint{3.478957in}{0.571523in}}%
\pgfpathlineto{\pgfqpoint{3.437912in}{0.571523in}}%
\pgfpathlineto{\pgfqpoint{3.396867in}{0.571523in}}%
\pgfpathlineto{\pgfqpoint{3.355821in}{0.571523in}}%
\pgfpathlineto{\pgfqpoint{3.314776in}{0.571523in}}%
\pgfpathlineto{\pgfqpoint{3.273731in}{0.571523in}}%
\pgfpathlineto{\pgfqpoint{3.232685in}{0.571523in}}%
\pgfpathlineto{\pgfqpoint{3.191640in}{0.571523in}}%
\pgfpathlineto{\pgfqpoint{3.150595in}{0.571523in}}%
\pgfpathlineto{\pgfqpoint{3.109549in}{0.571523in}}%
\pgfpathlineto{\pgfqpoint{3.068504in}{0.571523in}}%
\pgfpathlineto{\pgfqpoint{3.027459in}{0.571523in}}%
\pgfpathlineto{\pgfqpoint{2.986413in}{0.571523in}}%
\pgfpathlineto{\pgfqpoint{2.945368in}{0.571523in}}%
\pgfpathlineto{\pgfqpoint{2.904323in}{0.571523in}}%
\pgfpathlineto{\pgfqpoint{2.863277in}{0.571523in}}%
\pgfpathlineto{\pgfqpoint{2.822232in}{0.571523in}}%
\pgfpathlineto{\pgfqpoint{2.781187in}{0.571523in}}%
\pgfpathlineto{\pgfqpoint{2.740141in}{0.571523in}}%
\pgfpathlineto{\pgfqpoint{2.699096in}{0.571523in}}%
\pgfpathlineto{\pgfqpoint{2.658051in}{0.571523in}}%
\pgfpathlineto{\pgfqpoint{2.617005in}{0.571523in}}%
\pgfpathlineto{\pgfqpoint{2.575960in}{0.571523in}}%
\pgfpathlineto{\pgfqpoint{2.534915in}{0.571523in}}%
\pgfpathlineto{\pgfqpoint{2.533702in}{0.589182in}}%
\pgfpathlineto{\pgfqpoint{2.532276in}{0.609810in}}%
\pgfpathlineto{\pgfqpoint{2.530835in}{0.630438in}}%
\pgfpathlineto{\pgfqpoint{2.529379in}{0.651066in}}%
\pgfpathlineto{\pgfqpoint{2.527907in}{0.671694in}}%
\pgfpathlineto{\pgfqpoint{2.526418in}{0.692321in}}%
\pgfpathlineto{\pgfqpoint{2.524911in}{0.712949in}}%
\pgfpathlineto{\pgfqpoint{2.523385in}{0.733577in}}%
\pgfpathlineto{\pgfqpoint{2.521839in}{0.754205in}}%
\pgfpathlineto{\pgfqpoint{2.520272in}{0.774833in}}%
\pgfpathlineto{\pgfqpoint{2.518682in}{0.795460in}}%
\pgfpathlineto{\pgfqpoint{2.517068in}{0.816088in}}%
\pgfpathlineto{\pgfqpoint{2.515428in}{0.836716in}}%
\pgfpathlineto{\pgfqpoint{2.513760in}{0.857344in}}%
\pgfpathlineto{\pgfqpoint{2.512063in}{0.877972in}}%
\pgfpathlineto{\pgfqpoint{2.510335in}{0.898600in}}%
\pgfpathlineto{\pgfqpoint{2.508573in}{0.919227in}}%
\pgfpathlineto{\pgfqpoint{2.506775in}{0.939855in}}%
\pgfpathlineto{\pgfqpoint{2.504937in}{0.960483in}}%
\pgfpathlineto{\pgfqpoint{2.503058in}{0.981111in}}%
\pgfpathlineto{\pgfqpoint{2.501133in}{1.001739in}}%
\pgfpathlineto{\pgfqpoint{2.499160in}{1.022367in}}%
\pgfpathlineto{\pgfqpoint{2.497133in}{1.042994in}}%
\pgfpathlineto{\pgfqpoint{2.495048in}{1.063622in}}%
\pgfpathlineto{\pgfqpoint{2.493869in}{1.075096in}}%
\pgfpathlineto{\pgfqpoint{2.469056in}{1.063622in}}%
\pgfpathlineto{\pgfqpoint{2.452824in}{1.056301in}}%
\pgfpathlineto{\pgfqpoint{2.423488in}{1.042994in}}%
\pgfpathlineto{\pgfqpoint{2.411779in}{1.037870in}}%
\pgfpathlineto{\pgfqpoint{2.370733in}{1.022993in}}%
\pgfpathlineto{\pgfqpoint{2.367989in}{1.022367in}}%
\pgfpathlineto{\pgfqpoint{2.329688in}{1.014037in}}%
\pgfpathlineto{\pgfqpoint{2.288643in}{1.011327in}}%
\pgfpathlineto{\pgfqpoint{2.247597in}{1.013505in}}%
\pgfpathlineto{\pgfqpoint{2.206552in}{1.017665in}}%
\pgfpathlineto{\pgfqpoint{2.165507in}{1.020177in}}%
\pgfpathlineto{\pgfqpoint{2.124461in}{1.017594in}}%
\pgfpathlineto{\pgfqpoint{2.083416in}{1.007383in}}%
\pgfpathlineto{\pgfqpoint{2.070995in}{1.001739in}}%
\pgfpathlineto{\pgfqpoint{2.042371in}{0.988713in}}%
\pgfpathlineto{\pgfqpoint{2.030369in}{0.981111in}}%
\pgfpathlineto{\pgfqpoint{2.001325in}{0.962487in}}%
\pgfpathlineto{\pgfqpoint{1.998657in}{0.960483in}}%
\pgfpathlineto{\pgfqpoint{1.971433in}{0.939855in}}%
\pgfpathlineto{\pgfqpoint{1.960280in}{0.931249in}}%
\pgfpathlineto{\pgfqpoint{1.945159in}{0.919227in}}%
\pgfpathlineto{\pgfqpoint{1.919911in}{0.898600in}}%
\pgfpathlineto{\pgfqpoint{1.919235in}{0.898042in}}%
\pgfpathlineto{\pgfqpoint{1.893083in}{0.877972in}}%
\pgfpathlineto{\pgfqpoint{1.878189in}{0.866175in}}%
\pgfpathlineto{\pgfqpoint{1.865186in}{0.857344in}}%
\pgfpathlineto{\pgfqpoint{1.837144in}{0.837675in}}%
\pgfpathlineto{\pgfqpoint{1.835468in}{0.836716in}}%
\pgfpathlineto{\pgfqpoint{1.800242in}{0.816088in}}%
\pgfpathlineto{\pgfqpoint{1.796099in}{0.813597in}}%
\pgfpathlineto{\pgfqpoint{1.759209in}{0.795460in}}%
\pgfpathlineto{\pgfqpoint{1.755053in}{0.793373in}}%
\pgfpathlineto{\pgfqpoint{1.714008in}{0.775690in}}%
\pgfpathlineto{\pgfqpoint{1.711850in}{0.774833in}}%
\pgfpathlineto{\pgfqpoint{1.672963in}{0.759396in}}%
\pgfpathlineto{\pgfqpoint{1.659059in}{0.754205in}}%
\pgfpathlineto{\pgfqpoint{1.631917in}{0.744216in}}%
\pgfpathlineto{\pgfqpoint{1.598350in}{0.733577in}}%
\pgfpathlineto{\pgfqpoint{1.590872in}{0.731269in}}%
\pgfpathlineto{\pgfqpoint{1.549827in}{0.722692in}}%
\pgfpathlineto{\pgfqpoint{1.508781in}{0.720187in}}%
\pgfpathlineto{\pgfqpoint{1.467736in}{0.724512in}}%
\pgfpathlineto{\pgfqpoint{1.431002in}{0.733577in}}%
\pgfpathlineto{\pgfqpoint{1.426691in}{0.734641in}}%
\pgfpathlineto{\pgfqpoint{1.385645in}{0.748072in}}%
\pgfpathlineto{\pgfqpoint{1.365753in}{0.754205in}}%
\pgfpathlineto{\pgfqpoint{1.344600in}{0.760785in}}%
\pgfpathlineto{\pgfqpoint{1.303555in}{0.768456in}}%
\pgfpathlineto{\pgfqpoint{1.262509in}{0.767234in}}%
\pgfpathlineto{\pgfqpoint{1.221464in}{0.754629in}}%
\pgfpathlineto{\pgfqpoint{1.220745in}{0.754205in}}%
\pgfpathlineto{\pgfqpoint{1.185799in}{0.733577in}}%
\pgfpathlineto{\pgfqpoint{1.180419in}{0.730389in}}%
\pgfpathlineto{\pgfqpoint{1.159376in}{0.712949in}}%
\pgfpathlineto{\pgfqpoint{1.139373in}{0.696117in}}%
\pgfpathlineto{\pgfqpoint{1.135526in}{0.692321in}}%
\pgfpathlineto{\pgfqpoint{1.114717in}{0.671694in}}%
\pgfpathlineto{\pgfqpoint{1.098328in}{0.655129in}}%
\pgfpathlineto{\pgfqpoint{1.094448in}{0.651066in}}%
\pgfpathlineto{\pgfqpoint{1.074887in}{0.630438in}}%
\pgfpathlineto{\pgfqpoint{1.057283in}{0.611412in}}%
\pgfpathlineto{\pgfqpoint{1.055710in}{0.609810in}}%
\pgfpathlineto{\pgfqpoint{1.035574in}{0.589182in}}%
\pgfpathlineto{\pgfqpoint{1.016237in}{0.568769in}}%
\pgfpathlineto{\pgfqpoint{1.016005in}{0.568554in}}%
\pgfpathlineto{\pgfqpoint{0.993826in}{0.547927in}}%
\pgfpathlineto{\pgfqpoint{0.975192in}{0.530055in}}%
\pgfpathlineto{\pgfqpoint{0.971738in}{0.527299in}}%
\pgfpathlineto{\pgfqpoint{0.946181in}{0.506671in}}%
\pgfpathlineto{\pgfqpoint{0.934147in}{0.496741in}}%
\pgfpathlineto{\pgfqpoint{0.918096in}{0.486043in}}%
\pgfpathlineto{\pgfqpoint{0.893101in}{0.469071in}}%
\pgfpathlineto{\pgfqpoint{0.886332in}{0.465415in}}%
\pgfpathlineto{\pgfqpoint{0.852056in}{0.446756in}}%
\pgfpathlineto{\pgfqpoint{0.847386in}{0.444787in}}%
\pgfpathlineto{\pgfqpoint{0.811011in}{0.429521in}}%
\pgfpathlineto{\pgfqpoint{0.792702in}{0.424160in}}%
\pgfpathlineto{\pgfqpoint{0.769965in}{0.417613in}}%
\pgfpathlineto{\pgfqpoint{0.728920in}{0.412044in}}%
\pgfpathlineto{\pgfqpoint{0.687875in}{0.413990in}}%
\pgfpathlineto{\pgfqpoint{0.646829in}{0.423999in}}%
\pgfpathclose%
\pgfusepath{stroke,fill}%
\end{pgfscope}%
\begin{pgfscope}%
\pgfpathrectangle{\pgfqpoint{0.605784in}{0.382904in}}{\pgfqpoint{4.063488in}{2.042155in}}%
\pgfusepath{clip}%
\pgfsetbuttcap%
\pgfsetroundjoin%
\definecolor{currentfill}{rgb}{0.140536,0.530132,0.555659}%
\pgfsetfillcolor{currentfill}%
\pgfsetlinewidth{1.003750pt}%
\definecolor{currentstroke}{rgb}{0.140536,0.530132,0.555659}%
\pgfsetstrokecolor{currentstroke}%
\pgfsetdash{}{0pt}%
\pgfpathmoveto{\pgfqpoint{0.638148in}{2.053758in}}%
\pgfpathlineto{\pgfqpoint{0.605784in}{2.071832in}}%
\pgfpathlineto{\pgfqpoint{0.605784in}{2.074386in}}%
\pgfpathlineto{\pgfqpoint{0.605784in}{2.095013in}}%
\pgfpathlineto{\pgfqpoint{0.605784in}{2.115641in}}%
\pgfpathlineto{\pgfqpoint{0.605784in}{2.136269in}}%
\pgfpathlineto{\pgfqpoint{0.605784in}{2.139500in}}%
\pgfpathlineto{\pgfqpoint{0.611639in}{2.136269in}}%
\pgfpathlineto{\pgfqpoint{0.646829in}{2.116838in}}%
\pgfpathlineto{\pgfqpoint{0.648890in}{2.115641in}}%
\pgfpathlineto{\pgfqpoint{0.684208in}{2.095013in}}%
\pgfpathlineto{\pgfqpoint{0.687875in}{2.092860in}}%
\pgfpathlineto{\pgfqpoint{0.725247in}{2.074386in}}%
\pgfpathlineto{\pgfqpoint{0.728920in}{2.072549in}}%
\pgfpathlineto{\pgfqpoint{0.769965in}{2.060989in}}%
\pgfpathlineto{\pgfqpoint{0.811011in}{2.062486in}}%
\pgfpathlineto{\pgfqpoint{0.840021in}{2.074386in}}%
\pgfpathlineto{\pgfqpoint{0.852056in}{2.079300in}}%
\pgfpathlineto{\pgfqpoint{0.872411in}{2.095013in}}%
\pgfpathlineto{\pgfqpoint{0.893101in}{2.111111in}}%
\pgfpathlineto{\pgfqpoint{0.897341in}{2.115641in}}%
\pgfpathlineto{\pgfqpoint{0.916649in}{2.136269in}}%
\pgfpathlineto{\pgfqpoint{0.934147in}{2.155225in}}%
\pgfpathlineto{\pgfqpoint{0.935478in}{2.156897in}}%
\pgfpathlineto{\pgfqpoint{0.951943in}{2.177525in}}%
\pgfpathlineto{\pgfqpoint{0.968148in}{2.198153in}}%
\pgfpathlineto{\pgfqpoint{0.975192in}{2.207158in}}%
\pgfpathlineto{\pgfqpoint{0.983993in}{2.218780in}}%
\pgfpathlineto{\pgfqpoint{0.999482in}{2.239408in}}%
\pgfpathlineto{\pgfqpoint{1.014708in}{2.260036in}}%
\pgfpathlineto{\pgfqpoint{1.016237in}{2.262105in}}%
\pgfpathlineto{\pgfqpoint{1.030743in}{2.280664in}}%
\pgfpathlineto{\pgfqpoint{1.046585in}{2.301292in}}%
\pgfpathlineto{\pgfqpoint{1.057283in}{2.315391in}}%
\pgfpathlineto{\pgfqpoint{1.062908in}{2.321920in}}%
\pgfpathlineto{\pgfqpoint{1.080568in}{2.342547in}}%
\pgfpathlineto{\pgfqpoint{1.097849in}{2.363175in}}%
\pgfpathlineto{\pgfqpoint{1.098328in}{2.363747in}}%
\pgfpathlineto{\pgfqpoint{1.118550in}{2.383803in}}%
\pgfpathlineto{\pgfqpoint{1.138867in}{2.404431in}}%
\pgfpathlineto{\pgfqpoint{1.139373in}{2.404945in}}%
\pgfpathlineto{\pgfqpoint{1.164497in}{2.425059in}}%
\pgfpathlineto{\pgfqpoint{1.180419in}{2.425059in}}%
\pgfpathlineto{\pgfqpoint{1.221464in}{2.425059in}}%
\pgfpathlineto{\pgfqpoint{1.262509in}{2.425059in}}%
\pgfpathlineto{\pgfqpoint{1.303555in}{2.425059in}}%
\pgfpathlineto{\pgfqpoint{1.344600in}{2.425059in}}%
\pgfpathlineto{\pgfqpoint{1.385645in}{2.425059in}}%
\pgfpathlineto{\pgfqpoint{1.426691in}{2.425059in}}%
\pgfpathlineto{\pgfqpoint{1.455483in}{2.425059in}}%
\pgfpathlineto{\pgfqpoint{1.467736in}{2.415888in}}%
\pgfpathlineto{\pgfqpoint{1.481285in}{2.404431in}}%
\pgfpathlineto{\pgfqpoint{1.505605in}{2.383803in}}%
\pgfpathlineto{\pgfqpoint{1.508781in}{2.381085in}}%
\pgfpathlineto{\pgfqpoint{1.530504in}{2.363175in}}%
\pgfpathlineto{\pgfqpoint{1.549827in}{2.347149in}}%
\pgfpathlineto{\pgfqpoint{1.556940in}{2.342547in}}%
\pgfpathlineto{\pgfqpoint{1.588358in}{2.321920in}}%
\pgfpathlineto{\pgfqpoint{1.590872in}{2.320242in}}%
\pgfpathlineto{\pgfqpoint{1.631917in}{2.306273in}}%
\pgfpathlineto{\pgfqpoint{1.672963in}{2.309907in}}%
\pgfpathlineto{\pgfqpoint{1.694219in}{2.321920in}}%
\pgfpathlineto{\pgfqpoint{1.714008in}{2.333090in}}%
\pgfpathlineto{\pgfqpoint{1.723393in}{2.342547in}}%
\pgfpathlineto{\pgfqpoint{1.743841in}{2.363175in}}%
\pgfpathlineto{\pgfqpoint{1.755053in}{2.374517in}}%
\pgfpathlineto{\pgfqpoint{1.761936in}{2.383803in}}%
\pgfpathlineto{\pgfqpoint{1.777228in}{2.404431in}}%
\pgfpathlineto{\pgfqpoint{1.792377in}{2.425059in}}%
\pgfpathlineto{\pgfqpoint{1.796099in}{2.425059in}}%
\pgfpathlineto{\pgfqpoint{1.837144in}{2.425059in}}%
\pgfpathlineto{\pgfqpoint{1.837408in}{2.425059in}}%
\pgfpathlineto{\pgfqpoint{1.837144in}{2.424622in}}%
\pgfpathlineto{\pgfqpoint{1.824688in}{2.404431in}}%
\pgfpathlineto{\pgfqpoint{1.811852in}{2.383803in}}%
\pgfpathlineto{\pgfqpoint{1.798893in}{2.363175in}}%
\pgfpathlineto{\pgfqpoint{1.796099in}{2.358704in}}%
\pgfpathlineto{\pgfqpoint{1.784600in}{2.342547in}}%
\pgfpathlineto{\pgfqpoint{1.769847in}{2.321920in}}%
\pgfpathlineto{\pgfqpoint{1.755053in}{2.301423in}}%
\pgfpathlineto{\pgfqpoint{1.754926in}{2.301292in}}%
\pgfpathlineto{\pgfqpoint{1.735195in}{2.280664in}}%
\pgfpathlineto{\pgfqpoint{1.715243in}{2.260036in}}%
\pgfpathlineto{\pgfqpoint{1.714008in}{2.258749in}}%
\pgfpathlineto{\pgfqpoint{1.680332in}{2.239408in}}%
\pgfpathlineto{\pgfqpoint{1.672963in}{2.235170in}}%
\pgfpathlineto{\pgfqpoint{1.631917in}{2.232110in}}%
\pgfpathlineto{\pgfqpoint{1.612650in}{2.239408in}}%
\pgfpathlineto{\pgfqpoint{1.590872in}{2.247428in}}%
\pgfpathlineto{\pgfqpoint{1.573000in}{2.260036in}}%
\pgfpathlineto{\pgfqpoint{1.549827in}{2.276128in}}%
\pgfpathlineto{\pgfqpoint{1.544670in}{2.280664in}}%
\pgfpathlineto{\pgfqpoint{1.520912in}{2.301292in}}%
\pgfpathlineto{\pgfqpoint{1.508781in}{2.311728in}}%
\pgfpathlineto{\pgfqpoint{1.497288in}{2.321920in}}%
\pgfpathlineto{\pgfqpoint{1.473904in}{2.342547in}}%
\pgfpathlineto{\pgfqpoint{1.467736in}{2.347943in}}%
\pgfpathlineto{\pgfqpoint{1.447910in}{2.363175in}}%
\pgfpathlineto{\pgfqpoint{1.426691in}{2.379474in}}%
\pgfpathlineto{\pgfqpoint{1.419080in}{2.383803in}}%
\pgfpathlineto{\pgfqpoint{1.385645in}{2.402763in}}%
\pgfpathlineto{\pgfqpoint{1.380480in}{2.404431in}}%
\pgfpathlineto{\pgfqpoint{1.344600in}{2.415906in}}%
\pgfpathlineto{\pgfqpoint{1.303555in}{2.418687in}}%
\pgfpathlineto{\pgfqpoint{1.262509in}{2.411597in}}%
\pgfpathlineto{\pgfqpoint{1.244225in}{2.404431in}}%
\pgfpathlineto{\pgfqpoint{1.221464in}{2.395437in}}%
\pgfpathlineto{\pgfqpoint{1.202177in}{2.383803in}}%
\pgfpathlineto{\pgfqpoint{1.180419in}{2.370742in}}%
\pgfpathlineto{\pgfqpoint{1.171103in}{2.363175in}}%
\pgfpathlineto{\pgfqpoint{1.145448in}{2.342547in}}%
\pgfpathlineto{\pgfqpoint{1.139373in}{2.337697in}}%
\pgfpathlineto{\pgfqpoint{1.123860in}{2.321920in}}%
\pgfpathlineto{\pgfqpoint{1.103172in}{2.301292in}}%
\pgfpathlineto{\pgfqpoint{1.098328in}{2.296492in}}%
\pgfpathlineto{\pgfqpoint{1.085141in}{2.280664in}}%
\pgfpathlineto{\pgfqpoint{1.067653in}{2.260036in}}%
\pgfpathlineto{\pgfqpoint{1.057283in}{2.247968in}}%
\pgfpathlineto{\pgfqpoint{1.050845in}{2.239408in}}%
\pgfpathlineto{\pgfqpoint{1.035232in}{2.218780in}}%
\pgfpathlineto{\pgfqpoint{1.019289in}{2.198153in}}%
\pgfpathlineto{\pgfqpoint{1.016237in}{2.194211in}}%
\pgfpathlineto{\pgfqpoint{1.004101in}{2.177525in}}%
\pgfpathlineto{\pgfqpoint{0.988894in}{2.156897in}}%
\pgfpathlineto{\pgfqpoint{0.975192in}{2.138607in}}%
\pgfpathlineto{\pgfqpoint{0.973390in}{2.136269in}}%
\pgfpathlineto{\pgfqpoint{0.957537in}{2.115641in}}%
\pgfpathlineto{\pgfqpoint{0.941421in}{2.095013in}}%
\pgfpathlineto{\pgfqpoint{0.934147in}{2.085748in}}%
\pgfpathlineto{\pgfqpoint{0.923821in}{2.074386in}}%
\pgfpathlineto{\pgfqpoint{0.904955in}{2.053758in}}%
\pgfpathlineto{\pgfqpoint{0.893101in}{2.040903in}}%
\pgfpathlineto{\pgfqpoint{0.883234in}{2.033130in}}%
\pgfpathlineto{\pgfqpoint{0.856939in}{2.012502in}}%
\pgfpathlineto{\pgfqpoint{0.852056in}{2.008671in}}%
\pgfpathlineto{\pgfqpoint{0.811011in}{1.991937in}}%
\pgfpathlineto{\pgfqpoint{0.808451in}{1.991874in}}%
\pgfpathlineto{\pgfqpoint{0.769965in}{1.990913in}}%
\pgfpathlineto{\pgfqpoint{0.766815in}{1.991874in}}%
\pgfpathlineto{\pgfqpoint{0.728920in}{2.003167in}}%
\pgfpathlineto{\pgfqpoint{0.710922in}{2.012502in}}%
\pgfpathlineto{\pgfqpoint{0.687875in}{2.024309in}}%
\pgfpathlineto{\pgfqpoint{0.673199in}{2.033130in}}%
\pgfpathlineto{\pgfqpoint{0.646829in}{2.048907in}}%
\pgfpathclose%
\pgfusepath{stroke,fill}%
\end{pgfscope}%
\begin{pgfscope}%
\pgfpathrectangle{\pgfqpoint{0.605784in}{0.382904in}}{\pgfqpoint{4.063488in}{2.042155in}}%
\pgfusepath{clip}%
\pgfsetbuttcap%
\pgfsetroundjoin%
\definecolor{currentfill}{rgb}{0.140536,0.530132,0.555659}%
\pgfsetfillcolor{currentfill}%
\pgfsetlinewidth{1.003750pt}%
\definecolor{currentstroke}{rgb}{0.140536,0.530132,0.555659}%
\pgfsetstrokecolor{currentstroke}%
\pgfsetdash{}{0pt}%
\pgfpathmoveto{\pgfqpoint{2.531100in}{2.177525in}}%
\pgfpathlineto{\pgfqpoint{2.527434in}{2.198153in}}%
\pgfpathlineto{\pgfqpoint{2.524016in}{2.218780in}}%
\pgfpathlineto{\pgfqpoint{2.520812in}{2.239408in}}%
\pgfpathlineto{\pgfqpoint{2.517794in}{2.260036in}}%
\pgfpathlineto{\pgfqpoint{2.514939in}{2.280664in}}%
\pgfpathlineto{\pgfqpoint{2.512229in}{2.301292in}}%
\pgfpathlineto{\pgfqpoint{2.509647in}{2.321920in}}%
\pgfpathlineto{\pgfqpoint{2.507178in}{2.342547in}}%
\pgfpathlineto{\pgfqpoint{2.504811in}{2.363175in}}%
\pgfpathlineto{\pgfqpoint{2.502535in}{2.383803in}}%
\pgfpathlineto{\pgfqpoint{2.500342in}{2.404431in}}%
\pgfpathlineto{\pgfqpoint{2.498223in}{2.425059in}}%
\pgfpathlineto{\pgfqpoint{2.506707in}{2.425059in}}%
\pgfpathlineto{\pgfqpoint{2.509105in}{2.404431in}}%
\pgfpathlineto{\pgfqpoint{2.511597in}{2.383803in}}%
\pgfpathlineto{\pgfqpoint{2.514192in}{2.363175in}}%
\pgfpathlineto{\pgfqpoint{2.516902in}{2.342547in}}%
\pgfpathlineto{\pgfqpoint{2.519739in}{2.321920in}}%
\pgfpathlineto{\pgfqpoint{2.522720in}{2.301292in}}%
\pgfpathlineto{\pgfqpoint{2.525861in}{2.280664in}}%
\pgfpathlineto{\pgfqpoint{2.529183in}{2.260036in}}%
\pgfpathlineto{\pgfqpoint{2.532710in}{2.239408in}}%
\pgfpathlineto{\pgfqpoint{2.534915in}{2.227089in}}%
\pgfpathlineto{\pgfqpoint{2.575960in}{2.227089in}}%
\pgfpathlineto{\pgfqpoint{2.617005in}{2.227089in}}%
\pgfpathlineto{\pgfqpoint{2.658051in}{2.227089in}}%
\pgfpathlineto{\pgfqpoint{2.699096in}{2.227089in}}%
\pgfpathlineto{\pgfqpoint{2.740141in}{2.227089in}}%
\pgfpathlineto{\pgfqpoint{2.781187in}{2.227089in}}%
\pgfpathlineto{\pgfqpoint{2.822232in}{2.227089in}}%
\pgfpathlineto{\pgfqpoint{2.863277in}{2.227089in}}%
\pgfpathlineto{\pgfqpoint{2.904323in}{2.227089in}}%
\pgfpathlineto{\pgfqpoint{2.945368in}{2.227089in}}%
\pgfpathlineto{\pgfqpoint{2.986413in}{2.227089in}}%
\pgfpathlineto{\pgfqpoint{3.027459in}{2.227089in}}%
\pgfpathlineto{\pgfqpoint{3.068504in}{2.227089in}}%
\pgfpathlineto{\pgfqpoint{3.109549in}{2.227089in}}%
\pgfpathlineto{\pgfqpoint{3.150595in}{2.227089in}}%
\pgfpathlineto{\pgfqpoint{3.191640in}{2.227089in}}%
\pgfpathlineto{\pgfqpoint{3.232685in}{2.227089in}}%
\pgfpathlineto{\pgfqpoint{3.273731in}{2.227089in}}%
\pgfpathlineto{\pgfqpoint{3.314776in}{2.227089in}}%
\pgfpathlineto{\pgfqpoint{3.355821in}{2.227089in}}%
\pgfpathlineto{\pgfqpoint{3.396867in}{2.227089in}}%
\pgfpathlineto{\pgfqpoint{3.437912in}{2.227089in}}%
\pgfpathlineto{\pgfqpoint{3.478957in}{2.227089in}}%
\pgfpathlineto{\pgfqpoint{3.520003in}{2.227089in}}%
\pgfpathlineto{\pgfqpoint{3.561048in}{2.227089in}}%
\pgfpathlineto{\pgfqpoint{3.602093in}{2.227089in}}%
\pgfpathlineto{\pgfqpoint{3.643139in}{2.227089in}}%
\pgfpathlineto{\pgfqpoint{3.644540in}{2.218780in}}%
\pgfpathlineto{\pgfqpoint{3.648040in}{2.198153in}}%
\pgfpathlineto{\pgfqpoint{3.651614in}{2.177525in}}%
\pgfpathlineto{\pgfqpoint{3.655264in}{2.156897in}}%
\pgfpathlineto{\pgfqpoint{3.658998in}{2.136269in}}%
\pgfpathlineto{\pgfqpoint{3.662822in}{2.115641in}}%
\pgfpathlineto{\pgfqpoint{3.666741in}{2.095013in}}%
\pgfpathlineto{\pgfqpoint{3.670765in}{2.074386in}}%
\pgfpathlineto{\pgfqpoint{3.674900in}{2.053758in}}%
\pgfpathlineto{\pgfqpoint{3.679158in}{2.033130in}}%
\pgfpathlineto{\pgfqpoint{3.683547in}{2.012502in}}%
\pgfpathlineto{\pgfqpoint{3.684184in}{2.009533in}}%
\pgfpathlineto{\pgfqpoint{3.725229in}{2.009533in}}%
\pgfpathlineto{\pgfqpoint{3.766275in}{2.009533in}}%
\pgfpathlineto{\pgfqpoint{3.807320in}{2.009533in}}%
\pgfpathlineto{\pgfqpoint{3.848365in}{2.009533in}}%
\pgfpathlineto{\pgfqpoint{3.889411in}{2.009533in}}%
\pgfpathlineto{\pgfqpoint{3.930456in}{2.009533in}}%
\pgfpathlineto{\pgfqpoint{3.971501in}{2.009533in}}%
\pgfpathlineto{\pgfqpoint{4.012547in}{2.009533in}}%
\pgfpathlineto{\pgfqpoint{4.053592in}{2.009533in}}%
\pgfpathlineto{\pgfqpoint{4.094637in}{2.009533in}}%
\pgfpathlineto{\pgfqpoint{4.135683in}{2.009533in}}%
\pgfpathlineto{\pgfqpoint{4.176728in}{2.009533in}}%
\pgfpathlineto{\pgfqpoint{4.217773in}{2.009533in}}%
\pgfpathlineto{\pgfqpoint{4.258819in}{2.009533in}}%
\pgfpathlineto{\pgfqpoint{4.299864in}{2.009533in}}%
\pgfpathlineto{\pgfqpoint{4.340909in}{2.009533in}}%
\pgfpathlineto{\pgfqpoint{4.381955in}{2.009533in}}%
\pgfpathlineto{\pgfqpoint{4.423000in}{2.009533in}}%
\pgfpathlineto{\pgfqpoint{4.464045in}{2.009533in}}%
\pgfpathlineto{\pgfqpoint{4.505091in}{2.009533in}}%
\pgfpathlineto{\pgfqpoint{4.546136in}{2.009533in}}%
\pgfpathlineto{\pgfqpoint{4.587181in}{2.009533in}}%
\pgfpathlineto{\pgfqpoint{4.628227in}{2.009533in}}%
\pgfpathlineto{\pgfqpoint{4.669272in}{2.009533in}}%
\pgfpathlineto{\pgfqpoint{4.669272in}{1.991874in}}%
\pgfpathlineto{\pgfqpoint{4.669272in}{1.971247in}}%
\pgfpathlineto{\pgfqpoint{4.669272in}{1.950619in}}%
\pgfpathlineto{\pgfqpoint{4.669272in}{1.939935in}}%
\pgfpathlineto{\pgfqpoint{4.628227in}{1.939935in}}%
\pgfpathlineto{\pgfqpoint{4.587181in}{1.939935in}}%
\pgfpathlineto{\pgfqpoint{4.546136in}{1.939935in}}%
\pgfpathlineto{\pgfqpoint{4.505091in}{1.939935in}}%
\pgfpathlineto{\pgfqpoint{4.464045in}{1.939935in}}%
\pgfpathlineto{\pgfqpoint{4.423000in}{1.939935in}}%
\pgfpathlineto{\pgfqpoint{4.381955in}{1.939935in}}%
\pgfpathlineto{\pgfqpoint{4.340909in}{1.939935in}}%
\pgfpathlineto{\pgfqpoint{4.299864in}{1.939935in}}%
\pgfpathlineto{\pgfqpoint{4.258819in}{1.939935in}}%
\pgfpathlineto{\pgfqpoint{4.217773in}{1.939935in}}%
\pgfpathlineto{\pgfqpoint{4.176728in}{1.939935in}}%
\pgfpathlineto{\pgfqpoint{4.135683in}{1.939935in}}%
\pgfpathlineto{\pgfqpoint{4.094637in}{1.939935in}}%
\pgfpathlineto{\pgfqpoint{4.053592in}{1.939935in}}%
\pgfpathlineto{\pgfqpoint{4.012547in}{1.939935in}}%
\pgfpathlineto{\pgfqpoint{3.971501in}{1.939935in}}%
\pgfpathlineto{\pgfqpoint{3.930456in}{1.939935in}}%
\pgfpathlineto{\pgfqpoint{3.889411in}{1.939935in}}%
\pgfpathlineto{\pgfqpoint{3.848365in}{1.939935in}}%
\pgfpathlineto{\pgfqpoint{3.807320in}{1.939935in}}%
\pgfpathlineto{\pgfqpoint{3.766275in}{1.939935in}}%
\pgfpathlineto{\pgfqpoint{3.725229in}{1.939935in}}%
\pgfpathlineto{\pgfqpoint{3.684184in}{1.939935in}}%
\pgfpathlineto{\pgfqpoint{3.681868in}{1.950619in}}%
\pgfpathlineto{\pgfqpoint{3.677484in}{1.971247in}}%
\pgfpathlineto{\pgfqpoint{3.673243in}{1.991874in}}%
\pgfpathlineto{\pgfqpoint{3.669134in}{2.012502in}}%
\pgfpathlineto{\pgfqpoint{3.665145in}{2.033130in}}%
\pgfpathlineto{\pgfqpoint{3.661266in}{2.053758in}}%
\pgfpathlineto{\pgfqpoint{3.657489in}{2.074386in}}%
\pgfpathlineto{\pgfqpoint{3.653806in}{2.095013in}}%
\pgfpathlineto{\pgfqpoint{3.650210in}{2.115641in}}%
\pgfpathlineto{\pgfqpoint{3.646694in}{2.136269in}}%
\pgfpathlineto{\pgfqpoint{3.643253in}{2.156897in}}%
\pgfpathlineto{\pgfqpoint{3.643139in}{2.157582in}}%
\pgfpathlineto{\pgfqpoint{3.602093in}{2.157582in}}%
\pgfpathlineto{\pgfqpoint{3.561048in}{2.157582in}}%
\pgfpathlineto{\pgfqpoint{3.520003in}{2.157582in}}%
\pgfpathlineto{\pgfqpoint{3.478957in}{2.157582in}}%
\pgfpathlineto{\pgfqpoint{3.437912in}{2.157582in}}%
\pgfpathlineto{\pgfqpoint{3.396867in}{2.157582in}}%
\pgfpathlineto{\pgfqpoint{3.355821in}{2.157582in}}%
\pgfpathlineto{\pgfqpoint{3.314776in}{2.157582in}}%
\pgfpathlineto{\pgfqpoint{3.273731in}{2.157582in}}%
\pgfpathlineto{\pgfqpoint{3.232685in}{2.157582in}}%
\pgfpathlineto{\pgfqpoint{3.191640in}{2.157582in}}%
\pgfpathlineto{\pgfqpoint{3.150595in}{2.157582in}}%
\pgfpathlineto{\pgfqpoint{3.109549in}{2.157582in}}%
\pgfpathlineto{\pgfqpoint{3.068504in}{2.157582in}}%
\pgfpathlineto{\pgfqpoint{3.027459in}{2.157582in}}%
\pgfpathlineto{\pgfqpoint{2.986413in}{2.157582in}}%
\pgfpathlineto{\pgfqpoint{2.945368in}{2.157582in}}%
\pgfpathlineto{\pgfqpoint{2.904323in}{2.157582in}}%
\pgfpathlineto{\pgfqpoint{2.863277in}{2.157582in}}%
\pgfpathlineto{\pgfqpoint{2.822232in}{2.157582in}}%
\pgfpathlineto{\pgfqpoint{2.781187in}{2.157582in}}%
\pgfpathlineto{\pgfqpoint{2.740141in}{2.157582in}}%
\pgfpathlineto{\pgfqpoint{2.699096in}{2.157582in}}%
\pgfpathlineto{\pgfqpoint{2.658051in}{2.157582in}}%
\pgfpathlineto{\pgfqpoint{2.617005in}{2.157582in}}%
\pgfpathlineto{\pgfqpoint{2.575960in}{2.157582in}}%
\pgfpathlineto{\pgfqpoint{2.534915in}{2.157582in}}%
\pgfpathclose%
\pgfusepath{stroke,fill}%
\end{pgfscope}%
\begin{pgfscope}%
\pgfpathrectangle{\pgfqpoint{0.605784in}{0.382904in}}{\pgfqpoint{4.063488in}{2.042155in}}%
\pgfusepath{clip}%
\pgfsetbuttcap%
\pgfsetroundjoin%
\definecolor{currentfill}{rgb}{0.120092,0.600104,0.542530}%
\pgfsetfillcolor{currentfill}%
\pgfsetlinewidth{1.003750pt}%
\definecolor{currentstroke}{rgb}{0.120092,0.600104,0.542530}%
\pgfsetstrokecolor{currentstroke}%
\pgfsetdash{}{0pt}%
\pgfpathmoveto{\pgfqpoint{0.605784in}{0.403532in}}%
\pgfpathlineto{\pgfqpoint{0.605784in}{0.424160in}}%
\pgfpathlineto{\pgfqpoint{0.605784in}{0.441557in}}%
\pgfpathlineto{\pgfqpoint{0.646453in}{0.424160in}}%
\pgfpathlineto{\pgfqpoint{0.646829in}{0.423999in}}%
\pgfpathlineto{\pgfqpoint{0.687875in}{0.413990in}}%
\pgfpathlineto{\pgfqpoint{0.728920in}{0.412044in}}%
\pgfpathlineto{\pgfqpoint{0.769965in}{0.417613in}}%
\pgfpathlineto{\pgfqpoint{0.792702in}{0.424160in}}%
\pgfpathlineto{\pgfqpoint{0.811011in}{0.429521in}}%
\pgfpathlineto{\pgfqpoint{0.847386in}{0.444787in}}%
\pgfpathlineto{\pgfqpoint{0.852056in}{0.446756in}}%
\pgfpathlineto{\pgfqpoint{0.886332in}{0.465415in}}%
\pgfpathlineto{\pgfqpoint{0.893101in}{0.469071in}}%
\pgfpathlineto{\pgfqpoint{0.918096in}{0.486043in}}%
\pgfpathlineto{\pgfqpoint{0.934147in}{0.496741in}}%
\pgfpathlineto{\pgfqpoint{0.946181in}{0.506671in}}%
\pgfpathlineto{\pgfqpoint{0.971738in}{0.527299in}}%
\pgfpathlineto{\pgfqpoint{0.975192in}{0.530055in}}%
\pgfpathlineto{\pgfqpoint{0.993826in}{0.547927in}}%
\pgfpathlineto{\pgfqpoint{1.016005in}{0.568554in}}%
\pgfpathlineto{\pgfqpoint{1.016237in}{0.568769in}}%
\pgfpathlineto{\pgfqpoint{1.035574in}{0.589182in}}%
\pgfpathlineto{\pgfqpoint{1.055710in}{0.609810in}}%
\pgfpathlineto{\pgfqpoint{1.057283in}{0.611412in}}%
\pgfpathlineto{\pgfqpoint{1.074887in}{0.630438in}}%
\pgfpathlineto{\pgfqpoint{1.094448in}{0.651066in}}%
\pgfpathlineto{\pgfqpoint{1.098328in}{0.655129in}}%
\pgfpathlineto{\pgfqpoint{1.114717in}{0.671694in}}%
\pgfpathlineto{\pgfqpoint{1.135526in}{0.692321in}}%
\pgfpathlineto{\pgfqpoint{1.139373in}{0.696117in}}%
\pgfpathlineto{\pgfqpoint{1.159376in}{0.712949in}}%
\pgfpathlineto{\pgfqpoint{1.180419in}{0.730389in}}%
\pgfpathlineto{\pgfqpoint{1.185799in}{0.733577in}}%
\pgfpathlineto{\pgfqpoint{1.220745in}{0.754205in}}%
\pgfpathlineto{\pgfqpoint{1.221464in}{0.754629in}}%
\pgfpathlineto{\pgfqpoint{1.262509in}{0.767234in}}%
\pgfpathlineto{\pgfqpoint{1.303555in}{0.768456in}}%
\pgfpathlineto{\pgfqpoint{1.344600in}{0.760785in}}%
\pgfpathlineto{\pgfqpoint{1.365753in}{0.754205in}}%
\pgfpathlineto{\pgfqpoint{1.385645in}{0.748072in}}%
\pgfpathlineto{\pgfqpoint{1.426691in}{0.734641in}}%
\pgfpathlineto{\pgfqpoint{1.431002in}{0.733577in}}%
\pgfpathlineto{\pgfqpoint{1.467736in}{0.724512in}}%
\pgfpathlineto{\pgfqpoint{1.508781in}{0.720187in}}%
\pgfpathlineto{\pgfqpoint{1.549827in}{0.722692in}}%
\pgfpathlineto{\pgfqpoint{1.590872in}{0.731269in}}%
\pgfpathlineto{\pgfqpoint{1.598350in}{0.733577in}}%
\pgfpathlineto{\pgfqpoint{1.631917in}{0.744216in}}%
\pgfpathlineto{\pgfqpoint{1.659059in}{0.754205in}}%
\pgfpathlineto{\pgfqpoint{1.672963in}{0.759396in}}%
\pgfpathlineto{\pgfqpoint{1.711850in}{0.774833in}}%
\pgfpathlineto{\pgfqpoint{1.714008in}{0.775690in}}%
\pgfpathlineto{\pgfqpoint{1.755053in}{0.793373in}}%
\pgfpathlineto{\pgfqpoint{1.759209in}{0.795460in}}%
\pgfpathlineto{\pgfqpoint{1.796099in}{0.813597in}}%
\pgfpathlineto{\pgfqpoint{1.800242in}{0.816088in}}%
\pgfpathlineto{\pgfqpoint{1.835468in}{0.836716in}}%
\pgfpathlineto{\pgfqpoint{1.837144in}{0.837675in}}%
\pgfpathlineto{\pgfqpoint{1.865186in}{0.857344in}}%
\pgfpathlineto{\pgfqpoint{1.878189in}{0.866175in}}%
\pgfpathlineto{\pgfqpoint{1.893083in}{0.877972in}}%
\pgfpathlineto{\pgfqpoint{1.919235in}{0.898042in}}%
\pgfpathlineto{\pgfqpoint{1.919911in}{0.898600in}}%
\pgfpathlineto{\pgfqpoint{1.945159in}{0.919227in}}%
\pgfpathlineto{\pgfqpoint{1.960280in}{0.931249in}}%
\pgfpathlineto{\pgfqpoint{1.971433in}{0.939855in}}%
\pgfpathlineto{\pgfqpoint{1.998657in}{0.960483in}}%
\pgfpathlineto{\pgfqpoint{2.001325in}{0.962487in}}%
\pgfpathlineto{\pgfqpoint{2.030369in}{0.981111in}}%
\pgfpathlineto{\pgfqpoint{2.042371in}{0.988713in}}%
\pgfpathlineto{\pgfqpoint{2.070995in}{1.001739in}}%
\pgfpathlineto{\pgfqpoint{2.083416in}{1.007383in}}%
\pgfpathlineto{\pgfqpoint{2.124461in}{1.017594in}}%
\pgfpathlineto{\pgfqpoint{2.165507in}{1.020177in}}%
\pgfpathlineto{\pgfqpoint{2.206552in}{1.017665in}}%
\pgfpathlineto{\pgfqpoint{2.247597in}{1.013505in}}%
\pgfpathlineto{\pgfqpoint{2.288643in}{1.011327in}}%
\pgfpathlineto{\pgfqpoint{2.329688in}{1.014037in}}%
\pgfpathlineto{\pgfqpoint{2.367989in}{1.022367in}}%
\pgfpathlineto{\pgfqpoint{2.370733in}{1.022993in}}%
\pgfpathlineto{\pgfqpoint{2.411779in}{1.037870in}}%
\pgfpathlineto{\pgfqpoint{2.423488in}{1.042994in}}%
\pgfpathlineto{\pgfqpoint{2.452824in}{1.056301in}}%
\pgfpathlineto{\pgfqpoint{2.469056in}{1.063622in}}%
\pgfpathlineto{\pgfqpoint{2.493869in}{1.075096in}}%
\pgfpathlineto{\pgfqpoint{2.495048in}{1.063622in}}%
\pgfpathlineto{\pgfqpoint{2.497133in}{1.042994in}}%
\pgfpathlineto{\pgfqpoint{2.499160in}{1.022367in}}%
\pgfpathlineto{\pgfqpoint{2.501133in}{1.001739in}}%
\pgfpathlineto{\pgfqpoint{2.503058in}{0.981111in}}%
\pgfpathlineto{\pgfqpoint{2.504937in}{0.960483in}}%
\pgfpathlineto{\pgfqpoint{2.506775in}{0.939855in}}%
\pgfpathlineto{\pgfqpoint{2.508573in}{0.919227in}}%
\pgfpathlineto{\pgfqpoint{2.510335in}{0.898600in}}%
\pgfpathlineto{\pgfqpoint{2.512063in}{0.877972in}}%
\pgfpathlineto{\pgfqpoint{2.513760in}{0.857344in}}%
\pgfpathlineto{\pgfqpoint{2.515428in}{0.836716in}}%
\pgfpathlineto{\pgfqpoint{2.517068in}{0.816088in}}%
\pgfpathlineto{\pgfqpoint{2.518682in}{0.795460in}}%
\pgfpathlineto{\pgfqpoint{2.520272in}{0.774833in}}%
\pgfpathlineto{\pgfqpoint{2.521839in}{0.754205in}}%
\pgfpathlineto{\pgfqpoint{2.523385in}{0.733577in}}%
\pgfpathlineto{\pgfqpoint{2.524911in}{0.712949in}}%
\pgfpathlineto{\pgfqpoint{2.526418in}{0.692321in}}%
\pgfpathlineto{\pgfqpoint{2.527907in}{0.671694in}}%
\pgfpathlineto{\pgfqpoint{2.529379in}{0.651066in}}%
\pgfpathlineto{\pgfqpoint{2.530835in}{0.630438in}}%
\pgfpathlineto{\pgfqpoint{2.532276in}{0.609810in}}%
\pgfpathlineto{\pgfqpoint{2.533702in}{0.589182in}}%
\pgfpathlineto{\pgfqpoint{2.534915in}{0.571523in}}%
\pgfpathlineto{\pgfqpoint{2.575960in}{0.571523in}}%
\pgfpathlineto{\pgfqpoint{2.617005in}{0.571523in}}%
\pgfpathlineto{\pgfqpoint{2.658051in}{0.571523in}}%
\pgfpathlineto{\pgfqpoint{2.699096in}{0.571523in}}%
\pgfpathlineto{\pgfqpoint{2.740141in}{0.571523in}}%
\pgfpathlineto{\pgfqpoint{2.781187in}{0.571523in}}%
\pgfpathlineto{\pgfqpoint{2.822232in}{0.571523in}}%
\pgfpathlineto{\pgfqpoint{2.863277in}{0.571523in}}%
\pgfpathlineto{\pgfqpoint{2.904323in}{0.571523in}}%
\pgfpathlineto{\pgfqpoint{2.945368in}{0.571523in}}%
\pgfpathlineto{\pgfqpoint{2.986413in}{0.571523in}}%
\pgfpathlineto{\pgfqpoint{3.027459in}{0.571523in}}%
\pgfpathlineto{\pgfqpoint{3.068504in}{0.571523in}}%
\pgfpathlineto{\pgfqpoint{3.109549in}{0.571523in}}%
\pgfpathlineto{\pgfqpoint{3.150595in}{0.571523in}}%
\pgfpathlineto{\pgfqpoint{3.191640in}{0.571523in}}%
\pgfpathlineto{\pgfqpoint{3.232685in}{0.571523in}}%
\pgfpathlineto{\pgfqpoint{3.273731in}{0.571523in}}%
\pgfpathlineto{\pgfqpoint{3.314776in}{0.571523in}}%
\pgfpathlineto{\pgfqpoint{3.355821in}{0.571523in}}%
\pgfpathlineto{\pgfqpoint{3.396867in}{0.571523in}}%
\pgfpathlineto{\pgfqpoint{3.437912in}{0.571523in}}%
\pgfpathlineto{\pgfqpoint{3.478957in}{0.571523in}}%
\pgfpathlineto{\pgfqpoint{3.520003in}{0.571523in}}%
\pgfpathlineto{\pgfqpoint{3.561048in}{0.571523in}}%
\pgfpathlineto{\pgfqpoint{3.602093in}{0.571523in}}%
\pgfpathlineto{\pgfqpoint{3.643139in}{0.571523in}}%
\pgfpathlineto{\pgfqpoint{3.643775in}{0.568554in}}%
\pgfpathlineto{\pgfqpoint{3.648165in}{0.547927in}}%
\pgfpathlineto{\pgfqpoint{3.652422in}{0.527299in}}%
\pgfpathlineto{\pgfqpoint{3.656558in}{0.506671in}}%
\pgfpathlineto{\pgfqpoint{3.660581in}{0.486043in}}%
\pgfpathlineto{\pgfqpoint{3.664501in}{0.465415in}}%
\pgfpathlineto{\pgfqpoint{3.668324in}{0.444787in}}%
\pgfpathlineto{\pgfqpoint{3.672058in}{0.424160in}}%
\pgfpathlineto{\pgfqpoint{3.675709in}{0.403532in}}%
\pgfpathlineto{\pgfqpoint{3.679282in}{0.382904in}}%
\pgfpathlineto{\pgfqpoint{3.667817in}{0.382904in}}%
\pgfpathlineto{\pgfqpoint{3.663977in}{0.403532in}}%
\pgfpathlineto{\pgfqpoint{3.660047in}{0.424160in}}%
\pgfpathlineto{\pgfqpoint{3.656020in}{0.444787in}}%
\pgfpathlineto{\pgfqpoint{3.651889in}{0.465415in}}%
\pgfpathlineto{\pgfqpoint{3.647646in}{0.486043in}}%
\pgfpathlineto{\pgfqpoint{3.643282in}{0.506671in}}%
\pgfpathlineto{\pgfqpoint{3.643139in}{0.507348in}}%
\pgfpathlineto{\pgfqpoint{3.602093in}{0.507348in}}%
\pgfpathlineto{\pgfqpoint{3.561048in}{0.507348in}}%
\pgfpathlineto{\pgfqpoint{3.520003in}{0.507348in}}%
\pgfpathlineto{\pgfqpoint{3.478957in}{0.507348in}}%
\pgfpathlineto{\pgfqpoint{3.437912in}{0.507348in}}%
\pgfpathlineto{\pgfqpoint{3.396867in}{0.507348in}}%
\pgfpathlineto{\pgfqpoint{3.355821in}{0.507348in}}%
\pgfpathlineto{\pgfqpoint{3.314776in}{0.507348in}}%
\pgfpathlineto{\pgfqpoint{3.273731in}{0.507348in}}%
\pgfpathlineto{\pgfqpoint{3.232685in}{0.507348in}}%
\pgfpathlineto{\pgfqpoint{3.191640in}{0.507348in}}%
\pgfpathlineto{\pgfqpoint{3.150595in}{0.507348in}}%
\pgfpathlineto{\pgfqpoint{3.109549in}{0.507348in}}%
\pgfpathlineto{\pgfqpoint{3.068504in}{0.507348in}}%
\pgfpathlineto{\pgfqpoint{3.027459in}{0.507348in}}%
\pgfpathlineto{\pgfqpoint{2.986413in}{0.507348in}}%
\pgfpathlineto{\pgfqpoint{2.945368in}{0.507348in}}%
\pgfpathlineto{\pgfqpoint{2.904323in}{0.507348in}}%
\pgfpathlineto{\pgfqpoint{2.863277in}{0.507348in}}%
\pgfpathlineto{\pgfqpoint{2.822232in}{0.507348in}}%
\pgfpathlineto{\pgfqpoint{2.781187in}{0.507348in}}%
\pgfpathlineto{\pgfqpoint{2.740141in}{0.507348in}}%
\pgfpathlineto{\pgfqpoint{2.699096in}{0.507348in}}%
\pgfpathlineto{\pgfqpoint{2.658051in}{0.507348in}}%
\pgfpathlineto{\pgfqpoint{2.617005in}{0.507348in}}%
\pgfpathlineto{\pgfqpoint{2.575960in}{0.507348in}}%
\pgfpathlineto{\pgfqpoint{2.534915in}{0.507348in}}%
\pgfpathlineto{\pgfqpoint{2.533514in}{0.527299in}}%
\pgfpathlineto{\pgfqpoint{2.532053in}{0.547927in}}%
\pgfpathlineto{\pgfqpoint{2.530578in}{0.568554in}}%
\pgfpathlineto{\pgfqpoint{2.529086in}{0.589182in}}%
\pgfpathlineto{\pgfqpoint{2.527578in}{0.609810in}}%
\pgfpathlineto{\pgfqpoint{2.526053in}{0.630438in}}%
\pgfpathlineto{\pgfqpoint{2.524509in}{0.651066in}}%
\pgfpathlineto{\pgfqpoint{2.522947in}{0.671694in}}%
\pgfpathlineto{\pgfqpoint{2.521364in}{0.692321in}}%
\pgfpathlineto{\pgfqpoint{2.519759in}{0.712949in}}%
\pgfpathlineto{\pgfqpoint{2.518131in}{0.733577in}}%
\pgfpathlineto{\pgfqpoint{2.516479in}{0.754205in}}%
\pgfpathlineto{\pgfqpoint{2.514802in}{0.774833in}}%
\pgfpathlineto{\pgfqpoint{2.513097in}{0.795460in}}%
\pgfpathlineto{\pgfqpoint{2.511363in}{0.816088in}}%
\pgfpathlineto{\pgfqpoint{2.509598in}{0.836716in}}%
\pgfpathlineto{\pgfqpoint{2.507800in}{0.857344in}}%
\pgfpathlineto{\pgfqpoint{2.505966in}{0.877972in}}%
\pgfpathlineto{\pgfqpoint{2.504095in}{0.898600in}}%
\pgfpathlineto{\pgfqpoint{2.502183in}{0.919227in}}%
\pgfpathlineto{\pgfqpoint{2.500227in}{0.939855in}}%
\pgfpathlineto{\pgfqpoint{2.498225in}{0.960483in}}%
\pgfpathlineto{\pgfqpoint{2.496172in}{0.981111in}}%
\pgfpathlineto{\pgfqpoint{2.494064in}{1.001739in}}%
\pgfpathlineto{\pgfqpoint{2.493869in}{1.003642in}}%
\pgfpathlineto{\pgfqpoint{2.489447in}{1.001739in}}%
\pgfpathlineto{\pgfqpoint{2.452824in}{0.986332in}}%
\pgfpathlineto{\pgfqpoint{2.439975in}{0.981111in}}%
\pgfpathlineto{\pgfqpoint{2.411779in}{0.970028in}}%
\pgfpathlineto{\pgfqpoint{2.381024in}{0.960483in}}%
\pgfpathlineto{\pgfqpoint{2.370733in}{0.957416in}}%
\pgfpathlineto{\pgfqpoint{2.329688in}{0.950463in}}%
\pgfpathlineto{\pgfqpoint{2.288643in}{0.949351in}}%
\pgfpathlineto{\pgfqpoint{2.247597in}{0.952641in}}%
\pgfpathlineto{\pgfqpoint{2.206552in}{0.957461in}}%
\pgfpathlineto{\pgfqpoint{2.165507in}{0.960284in}}%
\pgfpathlineto{\pgfqpoint{2.124461in}{0.957778in}}%
\pgfpathlineto{\pgfqpoint{2.083416in}{0.947507in}}%
\pgfpathlineto{\pgfqpoint{2.066762in}{0.939855in}}%
\pgfpathlineto{\pgfqpoint{2.042371in}{0.928633in}}%
\pgfpathlineto{\pgfqpoint{2.027720in}{0.919227in}}%
\pgfpathlineto{\pgfqpoint{2.001325in}{0.902086in}}%
\pgfpathlineto{\pgfqpoint{1.996764in}{0.898600in}}%
\pgfpathlineto{\pgfqpoint{1.970051in}{0.877972in}}%
\pgfpathlineto{\pgfqpoint{1.960280in}{0.870311in}}%
\pgfpathlineto{\pgfqpoint{1.944366in}{0.857344in}}%
\pgfpathlineto{\pgfqpoint{1.919695in}{0.836716in}}%
\pgfpathlineto{\pgfqpoint{1.919235in}{0.836329in}}%
\pgfpathlineto{\pgfqpoint{1.893732in}{0.816088in}}%
\pgfpathlineto{\pgfqpoint{1.878189in}{0.803384in}}%
\pgfpathlineto{\pgfqpoint{1.867055in}{0.795460in}}%
\pgfpathlineto{\pgfqpoint{1.838862in}{0.774833in}}%
\pgfpathlineto{\pgfqpoint{1.837144in}{0.773556in}}%
\pgfpathlineto{\pgfqpoint{1.805826in}{0.754205in}}%
\pgfpathlineto{\pgfqpoint{1.796099in}{0.748034in}}%
\pgfpathlineto{\pgfqpoint{1.768408in}{0.733577in}}%
\pgfpathlineto{\pgfqpoint{1.755053in}{0.726465in}}%
\pgfpathlineto{\pgfqpoint{1.725171in}{0.712949in}}%
\pgfpathlineto{\pgfqpoint{1.714008in}{0.707847in}}%
\pgfpathlineto{\pgfqpoint{1.675724in}{0.692321in}}%
\pgfpathlineto{\pgfqpoint{1.672963in}{0.691202in}}%
\pgfpathlineto{\pgfqpoint{1.631917in}{0.676501in}}%
\pgfpathlineto{\pgfqpoint{1.615495in}{0.671694in}}%
\pgfpathlineto{\pgfqpoint{1.590872in}{0.664661in}}%
\pgfpathlineto{\pgfqpoint{1.549827in}{0.657389in}}%
\pgfpathlineto{\pgfqpoint{1.508781in}{0.656212in}}%
\pgfpathlineto{\pgfqpoint{1.467736in}{0.661622in}}%
\pgfpathlineto{\pgfqpoint{1.429560in}{0.671694in}}%
\pgfpathlineto{\pgfqpoint{1.426691in}{0.672451in}}%
\pgfpathlineto{\pgfqpoint{1.385645in}{0.686201in}}%
\pgfpathlineto{\pgfqpoint{1.365825in}{0.692321in}}%
\pgfpathlineto{\pgfqpoint{1.344600in}{0.698930in}}%
\pgfpathlineto{\pgfqpoint{1.303555in}{0.706443in}}%
\pgfpathlineto{\pgfqpoint{1.262509in}{0.705032in}}%
\pgfpathlineto{\pgfqpoint{1.221509in}{0.692321in}}%
\pgfpathlineto{\pgfqpoint{1.221464in}{0.692307in}}%
\pgfpathlineto{\pgfqpoint{1.186549in}{0.671694in}}%
\pgfpathlineto{\pgfqpoint{1.180419in}{0.668059in}}%
\pgfpathlineto{\pgfqpoint{1.159901in}{0.651066in}}%
\pgfpathlineto{\pgfqpoint{1.139373in}{0.633820in}}%
\pgfpathlineto{\pgfqpoint{1.135948in}{0.630438in}}%
\pgfpathlineto{\pgfqpoint{1.115143in}{0.609810in}}%
\pgfpathlineto{\pgfqpoint{1.098328in}{0.592826in}}%
\pgfpathlineto{\pgfqpoint{1.094866in}{0.589182in}}%
\pgfpathlineto{\pgfqpoint{1.075376in}{0.568554in}}%
\pgfpathlineto{\pgfqpoint{1.057283in}{0.548949in}}%
\pgfpathlineto{\pgfqpoint{1.056289in}{0.547927in}}%
\pgfpathlineto{\pgfqpoint{1.036326in}{0.527299in}}%
\pgfpathlineto{\pgfqpoint{1.016912in}{0.506671in}}%
\pgfpathlineto{\pgfqpoint{1.016237in}{0.505952in}}%
\pgfpathlineto{\pgfqpoint{0.995147in}{0.486043in}}%
\pgfpathlineto{\pgfqpoint{0.975192in}{0.466653in}}%
\pgfpathlineto{\pgfqpoint{0.973673in}{0.465415in}}%
\pgfpathlineto{\pgfqpoint{0.948551in}{0.444787in}}%
\pgfpathlineto{\pgfqpoint{0.934147in}{0.432688in}}%
\pgfpathlineto{\pgfqpoint{0.921622in}{0.424160in}}%
\pgfpathlineto{\pgfqpoint{0.893101in}{0.404401in}}%
\pgfpathlineto{\pgfqpoint{0.891516in}{0.403532in}}%
\pgfpathlineto{\pgfqpoint{0.854125in}{0.382904in}}%
\pgfpathlineto{\pgfqpoint{0.852056in}{0.382904in}}%
\pgfpathlineto{\pgfqpoint{0.811011in}{0.382904in}}%
\pgfpathlineto{\pgfqpoint{0.769965in}{0.382904in}}%
\pgfpathlineto{\pgfqpoint{0.728920in}{0.382904in}}%
\pgfpathlineto{\pgfqpoint{0.687875in}{0.382904in}}%
\pgfpathlineto{\pgfqpoint{0.646829in}{0.382904in}}%
\pgfpathlineto{\pgfqpoint{0.605784in}{0.382904in}}%
\pgfpathclose%
\pgfusepath{stroke,fill}%
\end{pgfscope}%
\begin{pgfscope}%
\pgfpathrectangle{\pgfqpoint{0.605784in}{0.382904in}}{\pgfqpoint{4.063488in}{2.042155in}}%
\pgfusepath{clip}%
\pgfsetbuttcap%
\pgfsetroundjoin%
\definecolor{currentfill}{rgb}{0.120092,0.600104,0.542530}%
\pgfsetfillcolor{currentfill}%
\pgfsetlinewidth{1.003750pt}%
\definecolor{currentstroke}{rgb}{0.120092,0.600104,0.542530}%
\pgfsetstrokecolor{currentstroke}%
\pgfsetdash{}{0pt}%
\pgfpathmoveto{\pgfqpoint{0.611639in}{2.136269in}}%
\pgfpathlineto{\pgfqpoint{0.605784in}{2.139500in}}%
\pgfpathlineto{\pgfqpoint{0.605784in}{2.156897in}}%
\pgfpathlineto{\pgfqpoint{0.605784in}{2.177525in}}%
\pgfpathlineto{\pgfqpoint{0.605784in}{2.198153in}}%
\pgfpathlineto{\pgfqpoint{0.605784in}{2.202159in}}%
\pgfpathlineto{\pgfqpoint{0.613099in}{2.198153in}}%
\pgfpathlineto{\pgfqpoint{0.646829in}{2.179670in}}%
\pgfpathlineto{\pgfqpoint{0.650597in}{2.177525in}}%
\pgfpathlineto{\pgfqpoint{0.686663in}{2.156897in}}%
\pgfpathlineto{\pgfqpoint{0.687875in}{2.156200in}}%
\pgfpathlineto{\pgfqpoint{0.728920in}{2.136541in}}%
\pgfpathlineto{\pgfqpoint{0.729959in}{2.136269in}}%
\pgfpathlineto{\pgfqpoint{0.769965in}{2.125580in}}%
\pgfpathlineto{\pgfqpoint{0.811011in}{2.127464in}}%
\pgfpathlineto{\pgfqpoint{0.832460in}{2.136269in}}%
\pgfpathlineto{\pgfqpoint{0.852056in}{2.144281in}}%
\pgfpathlineto{\pgfqpoint{0.868568in}{2.156897in}}%
\pgfpathlineto{\pgfqpoint{0.893101in}{2.175779in}}%
\pgfpathlineto{\pgfqpoint{0.894757in}{2.177525in}}%
\pgfpathlineto{\pgfqpoint{0.914361in}{2.198153in}}%
\pgfpathlineto{\pgfqpoint{0.933672in}{2.218780in}}%
\pgfpathlineto{\pgfqpoint{0.934147in}{2.219285in}}%
\pgfpathlineto{\pgfqpoint{0.950424in}{2.239408in}}%
\pgfpathlineto{\pgfqpoint{0.966863in}{2.260036in}}%
\pgfpathlineto{\pgfqpoint{0.975192in}{2.270547in}}%
\pgfpathlineto{\pgfqpoint{0.982925in}{2.280664in}}%
\pgfpathlineto{\pgfqpoint{0.998587in}{2.301292in}}%
\pgfpathlineto{\pgfqpoint{1.013994in}{2.321920in}}%
\pgfpathlineto{\pgfqpoint{1.016237in}{2.324921in}}%
\pgfpathlineto{\pgfqpoint{1.030095in}{2.342547in}}%
\pgfpathlineto{\pgfqpoint{1.046060in}{2.363175in}}%
\pgfpathlineto{\pgfqpoint{1.057283in}{2.377853in}}%
\pgfpathlineto{\pgfqpoint{1.062422in}{2.383803in}}%
\pgfpathlineto{\pgfqpoint{1.080140in}{2.404431in}}%
\pgfpathlineto{\pgfqpoint{1.097501in}{2.425059in}}%
\pgfpathlineto{\pgfqpoint{1.098328in}{2.425059in}}%
\pgfpathlineto{\pgfqpoint{1.139373in}{2.425059in}}%
\pgfpathlineto{\pgfqpoint{1.164497in}{2.425059in}}%
\pgfpathlineto{\pgfqpoint{1.139373in}{2.404945in}}%
\pgfpathlineto{\pgfqpoint{1.138867in}{2.404431in}}%
\pgfpathlineto{\pgfqpoint{1.118550in}{2.383803in}}%
\pgfpathlineto{\pgfqpoint{1.098328in}{2.363747in}}%
\pgfpathlineto{\pgfqpoint{1.097849in}{2.363175in}}%
\pgfpathlineto{\pgfqpoint{1.080568in}{2.342547in}}%
\pgfpathlineto{\pgfqpoint{1.062908in}{2.321920in}}%
\pgfpathlineto{\pgfqpoint{1.057283in}{2.315391in}}%
\pgfpathlineto{\pgfqpoint{1.046585in}{2.301292in}}%
\pgfpathlineto{\pgfqpoint{1.030743in}{2.280664in}}%
\pgfpathlineto{\pgfqpoint{1.016237in}{2.262105in}}%
\pgfpathlineto{\pgfqpoint{1.014708in}{2.260036in}}%
\pgfpathlineto{\pgfqpoint{0.999482in}{2.239408in}}%
\pgfpathlineto{\pgfqpoint{0.983993in}{2.218780in}}%
\pgfpathlineto{\pgfqpoint{0.975192in}{2.207158in}}%
\pgfpathlineto{\pgfqpoint{0.968148in}{2.198153in}}%
\pgfpathlineto{\pgfqpoint{0.951943in}{2.177525in}}%
\pgfpathlineto{\pgfqpoint{0.935478in}{2.156897in}}%
\pgfpathlineto{\pgfqpoint{0.934147in}{2.155225in}}%
\pgfpathlineto{\pgfqpoint{0.916649in}{2.136269in}}%
\pgfpathlineto{\pgfqpoint{0.897341in}{2.115641in}}%
\pgfpathlineto{\pgfqpoint{0.893101in}{2.111111in}}%
\pgfpathlineto{\pgfqpoint{0.872411in}{2.095013in}}%
\pgfpathlineto{\pgfqpoint{0.852056in}{2.079300in}}%
\pgfpathlineto{\pgfqpoint{0.840021in}{2.074386in}}%
\pgfpathlineto{\pgfqpoint{0.811011in}{2.062486in}}%
\pgfpathlineto{\pgfqpoint{0.769965in}{2.060989in}}%
\pgfpathlineto{\pgfqpoint{0.728920in}{2.072549in}}%
\pgfpathlineto{\pgfqpoint{0.725247in}{2.074386in}}%
\pgfpathlineto{\pgfqpoint{0.687875in}{2.092860in}}%
\pgfpathlineto{\pgfqpoint{0.684208in}{2.095013in}}%
\pgfpathlineto{\pgfqpoint{0.648890in}{2.115641in}}%
\pgfpathlineto{\pgfqpoint{0.646829in}{2.116838in}}%
\pgfpathclose%
\pgfusepath{stroke,fill}%
\end{pgfscope}%
\begin{pgfscope}%
\pgfpathrectangle{\pgfqpoint{0.605784in}{0.382904in}}{\pgfqpoint{4.063488in}{2.042155in}}%
\pgfusepath{clip}%
\pgfsetbuttcap%
\pgfsetroundjoin%
\definecolor{currentfill}{rgb}{0.120092,0.600104,0.542530}%
\pgfsetfillcolor{currentfill}%
\pgfsetlinewidth{1.003750pt}%
\definecolor{currentstroke}{rgb}{0.120092,0.600104,0.542530}%
\pgfsetstrokecolor{currentstroke}%
\pgfsetdash{}{0pt}%
\pgfpathmoveto{\pgfqpoint{1.455483in}{2.425059in}}%
\pgfpathlineto{\pgfqpoint{1.467736in}{2.425059in}}%
\pgfpathlineto{\pgfqpoint{1.508781in}{2.425059in}}%
\pgfpathlineto{\pgfqpoint{1.533995in}{2.425059in}}%
\pgfpathlineto{\pgfqpoint{1.549827in}{2.412425in}}%
\pgfpathlineto{\pgfqpoint{1.562781in}{2.404431in}}%
\pgfpathlineto{\pgfqpoint{1.590872in}{2.386868in}}%
\pgfpathlineto{\pgfqpoint{1.600786in}{2.383803in}}%
\pgfpathlineto{\pgfqpoint{1.631917in}{2.373951in}}%
\pgfpathlineto{\pgfqpoint{1.672963in}{2.378078in}}%
\pgfpathlineto{\pgfqpoint{1.683205in}{2.383803in}}%
\pgfpathlineto{\pgfqpoint{1.714008in}{2.401002in}}%
\pgfpathlineto{\pgfqpoint{1.717484in}{2.404431in}}%
\pgfpathlineto{\pgfqpoint{1.738470in}{2.425059in}}%
\pgfpathlineto{\pgfqpoint{1.755053in}{2.425059in}}%
\pgfpathlineto{\pgfqpoint{1.792377in}{2.425059in}}%
\pgfpathlineto{\pgfqpoint{1.777228in}{2.404431in}}%
\pgfpathlineto{\pgfqpoint{1.761936in}{2.383803in}}%
\pgfpathlineto{\pgfqpoint{1.755053in}{2.374517in}}%
\pgfpathlineto{\pgfqpoint{1.743841in}{2.363175in}}%
\pgfpathlineto{\pgfqpoint{1.723393in}{2.342547in}}%
\pgfpathlineto{\pgfqpoint{1.714008in}{2.333090in}}%
\pgfpathlineto{\pgfqpoint{1.694219in}{2.321920in}}%
\pgfpathlineto{\pgfqpoint{1.672963in}{2.309907in}}%
\pgfpathlineto{\pgfqpoint{1.631917in}{2.306273in}}%
\pgfpathlineto{\pgfqpoint{1.590872in}{2.320242in}}%
\pgfpathlineto{\pgfqpoint{1.588358in}{2.321920in}}%
\pgfpathlineto{\pgfqpoint{1.556940in}{2.342547in}}%
\pgfpathlineto{\pgfqpoint{1.549827in}{2.347149in}}%
\pgfpathlineto{\pgfqpoint{1.530504in}{2.363175in}}%
\pgfpathlineto{\pgfqpoint{1.508781in}{2.381085in}}%
\pgfpathlineto{\pgfqpoint{1.505605in}{2.383803in}}%
\pgfpathlineto{\pgfqpoint{1.481285in}{2.404431in}}%
\pgfpathlineto{\pgfqpoint{1.467736in}{2.415888in}}%
\pgfpathclose%
\pgfusepath{stroke,fill}%
\end{pgfscope}%
\begin{pgfscope}%
\pgfpathrectangle{\pgfqpoint{0.605784in}{0.382904in}}{\pgfqpoint{4.063488in}{2.042155in}}%
\pgfusepath{clip}%
\pgfsetbuttcap%
\pgfsetroundjoin%
\definecolor{currentfill}{rgb}{0.120092,0.600104,0.542530}%
\pgfsetfillcolor{currentfill}%
\pgfsetlinewidth{1.003750pt}%
\definecolor{currentstroke}{rgb}{0.120092,0.600104,0.542530}%
\pgfsetstrokecolor{currentstroke}%
\pgfsetdash{}{0pt}%
\pgfpathmoveto{\pgfqpoint{2.532710in}{2.239408in}}%
\pgfpathlineto{\pgfqpoint{2.529183in}{2.260036in}}%
\pgfpathlineto{\pgfqpoint{2.525861in}{2.280664in}}%
\pgfpathlineto{\pgfqpoint{2.522720in}{2.301292in}}%
\pgfpathlineto{\pgfqpoint{2.519739in}{2.321920in}}%
\pgfpathlineto{\pgfqpoint{2.516902in}{2.342547in}}%
\pgfpathlineto{\pgfqpoint{2.514192in}{2.363175in}}%
\pgfpathlineto{\pgfqpoint{2.511597in}{2.383803in}}%
\pgfpathlineto{\pgfqpoint{2.509105in}{2.404431in}}%
\pgfpathlineto{\pgfqpoint{2.506707in}{2.425059in}}%
\pgfpathlineto{\pgfqpoint{2.515190in}{2.425059in}}%
\pgfpathlineto{\pgfqpoint{2.517868in}{2.404431in}}%
\pgfpathlineto{\pgfqpoint{2.520658in}{2.383803in}}%
\pgfpathlineto{\pgfqpoint{2.523573in}{2.363175in}}%
\pgfpathlineto{\pgfqpoint{2.526626in}{2.342547in}}%
\pgfpathlineto{\pgfqpoint{2.529832in}{2.321920in}}%
\pgfpathlineto{\pgfqpoint{2.533210in}{2.301292in}}%
\pgfpathlineto{\pgfqpoint{2.534915in}{2.291242in}}%
\pgfpathlineto{\pgfqpoint{2.575960in}{2.291242in}}%
\pgfpathlineto{\pgfqpoint{2.617005in}{2.291242in}}%
\pgfpathlineto{\pgfqpoint{2.658051in}{2.291242in}}%
\pgfpathlineto{\pgfqpoint{2.699096in}{2.291242in}}%
\pgfpathlineto{\pgfqpoint{2.740141in}{2.291242in}}%
\pgfpathlineto{\pgfqpoint{2.781187in}{2.291242in}}%
\pgfpathlineto{\pgfqpoint{2.822232in}{2.291242in}}%
\pgfpathlineto{\pgfqpoint{2.863277in}{2.291242in}}%
\pgfpathlineto{\pgfqpoint{2.904323in}{2.291242in}}%
\pgfpathlineto{\pgfqpoint{2.945368in}{2.291242in}}%
\pgfpathlineto{\pgfqpoint{2.986413in}{2.291242in}}%
\pgfpathlineto{\pgfqpoint{3.027459in}{2.291242in}}%
\pgfpathlineto{\pgfqpoint{3.068504in}{2.291242in}}%
\pgfpathlineto{\pgfqpoint{3.109549in}{2.291242in}}%
\pgfpathlineto{\pgfqpoint{3.150595in}{2.291242in}}%
\pgfpathlineto{\pgfqpoint{3.191640in}{2.291242in}}%
\pgfpathlineto{\pgfqpoint{3.232685in}{2.291242in}}%
\pgfpathlineto{\pgfqpoint{3.273731in}{2.291242in}}%
\pgfpathlineto{\pgfqpoint{3.314776in}{2.291242in}}%
\pgfpathlineto{\pgfqpoint{3.355821in}{2.291242in}}%
\pgfpathlineto{\pgfqpoint{3.396867in}{2.291242in}}%
\pgfpathlineto{\pgfqpoint{3.437912in}{2.291242in}}%
\pgfpathlineto{\pgfqpoint{3.478957in}{2.291242in}}%
\pgfpathlineto{\pgfqpoint{3.520003in}{2.291242in}}%
\pgfpathlineto{\pgfqpoint{3.561048in}{2.291242in}}%
\pgfpathlineto{\pgfqpoint{3.602093in}{2.291242in}}%
\pgfpathlineto{\pgfqpoint{3.643139in}{2.291242in}}%
\pgfpathlineto{\pgfqpoint{3.644936in}{2.280664in}}%
\pgfpathlineto{\pgfqpoint{3.648470in}{2.260036in}}%
\pgfpathlineto{\pgfqpoint{3.652073in}{2.239408in}}%
\pgfpathlineto{\pgfqpoint{3.655750in}{2.218780in}}%
\pgfpathlineto{\pgfqpoint{3.659506in}{2.198153in}}%
\pgfpathlineto{\pgfqpoint{3.663345in}{2.177525in}}%
\pgfpathlineto{\pgfqpoint{3.667276in}{2.156897in}}%
\pgfpathlineto{\pgfqpoint{3.671302in}{2.136269in}}%
\pgfpathlineto{\pgfqpoint{3.675433in}{2.115641in}}%
\pgfpathlineto{\pgfqpoint{3.679676in}{2.095013in}}%
\pgfpathlineto{\pgfqpoint{3.684040in}{2.074386in}}%
\pgfpathlineto{\pgfqpoint{3.684184in}{2.073708in}}%
\pgfpathlineto{\pgfqpoint{3.725229in}{2.073708in}}%
\pgfpathlineto{\pgfqpoint{3.766275in}{2.073708in}}%
\pgfpathlineto{\pgfqpoint{3.807320in}{2.073708in}}%
\pgfpathlineto{\pgfqpoint{3.848365in}{2.073708in}}%
\pgfpathlineto{\pgfqpoint{3.889411in}{2.073708in}}%
\pgfpathlineto{\pgfqpoint{3.930456in}{2.073708in}}%
\pgfpathlineto{\pgfqpoint{3.971501in}{2.073708in}}%
\pgfpathlineto{\pgfqpoint{4.012547in}{2.073708in}}%
\pgfpathlineto{\pgfqpoint{4.053592in}{2.073708in}}%
\pgfpathlineto{\pgfqpoint{4.094637in}{2.073708in}}%
\pgfpathlineto{\pgfqpoint{4.135683in}{2.073708in}}%
\pgfpathlineto{\pgfqpoint{4.176728in}{2.073708in}}%
\pgfpathlineto{\pgfqpoint{4.217773in}{2.073708in}}%
\pgfpathlineto{\pgfqpoint{4.258819in}{2.073708in}}%
\pgfpathlineto{\pgfqpoint{4.299864in}{2.073708in}}%
\pgfpathlineto{\pgfqpoint{4.340909in}{2.073708in}}%
\pgfpathlineto{\pgfqpoint{4.381955in}{2.073708in}}%
\pgfpathlineto{\pgfqpoint{4.423000in}{2.073708in}}%
\pgfpathlineto{\pgfqpoint{4.464045in}{2.073708in}}%
\pgfpathlineto{\pgfqpoint{4.505091in}{2.073708in}}%
\pgfpathlineto{\pgfqpoint{4.546136in}{2.073708in}}%
\pgfpathlineto{\pgfqpoint{4.587181in}{2.073708in}}%
\pgfpathlineto{\pgfqpoint{4.628227in}{2.073708in}}%
\pgfpathlineto{\pgfqpoint{4.669272in}{2.073708in}}%
\pgfpathlineto{\pgfqpoint{4.669272in}{2.053758in}}%
\pgfpathlineto{\pgfqpoint{4.669272in}{2.033130in}}%
\pgfpathlineto{\pgfqpoint{4.669272in}{2.012502in}}%
\pgfpathlineto{\pgfqpoint{4.669272in}{2.009533in}}%
\pgfpathlineto{\pgfqpoint{4.628227in}{2.009533in}}%
\pgfpathlineto{\pgfqpoint{4.587181in}{2.009533in}}%
\pgfpathlineto{\pgfqpoint{4.546136in}{2.009533in}}%
\pgfpathlineto{\pgfqpoint{4.505091in}{2.009533in}}%
\pgfpathlineto{\pgfqpoint{4.464045in}{2.009533in}}%
\pgfpathlineto{\pgfqpoint{4.423000in}{2.009533in}}%
\pgfpathlineto{\pgfqpoint{4.381955in}{2.009533in}}%
\pgfpathlineto{\pgfqpoint{4.340909in}{2.009533in}}%
\pgfpathlineto{\pgfqpoint{4.299864in}{2.009533in}}%
\pgfpathlineto{\pgfqpoint{4.258819in}{2.009533in}}%
\pgfpathlineto{\pgfqpoint{4.217773in}{2.009533in}}%
\pgfpathlineto{\pgfqpoint{4.176728in}{2.009533in}}%
\pgfpathlineto{\pgfqpoint{4.135683in}{2.009533in}}%
\pgfpathlineto{\pgfqpoint{4.094637in}{2.009533in}}%
\pgfpathlineto{\pgfqpoint{4.053592in}{2.009533in}}%
\pgfpathlineto{\pgfqpoint{4.012547in}{2.009533in}}%
\pgfpathlineto{\pgfqpoint{3.971501in}{2.009533in}}%
\pgfpathlineto{\pgfqpoint{3.930456in}{2.009533in}}%
\pgfpathlineto{\pgfqpoint{3.889411in}{2.009533in}}%
\pgfpathlineto{\pgfqpoint{3.848365in}{2.009533in}}%
\pgfpathlineto{\pgfqpoint{3.807320in}{2.009533in}}%
\pgfpathlineto{\pgfqpoint{3.766275in}{2.009533in}}%
\pgfpathlineto{\pgfqpoint{3.725229in}{2.009533in}}%
\pgfpathlineto{\pgfqpoint{3.684184in}{2.009533in}}%
\pgfpathlineto{\pgfqpoint{3.683547in}{2.012502in}}%
\pgfpathlineto{\pgfqpoint{3.679158in}{2.033130in}}%
\pgfpathlineto{\pgfqpoint{3.674900in}{2.053758in}}%
\pgfpathlineto{\pgfqpoint{3.670765in}{2.074386in}}%
\pgfpathlineto{\pgfqpoint{3.666741in}{2.095013in}}%
\pgfpathlineto{\pgfqpoint{3.662822in}{2.115641in}}%
\pgfpathlineto{\pgfqpoint{3.658998in}{2.136269in}}%
\pgfpathlineto{\pgfqpoint{3.655264in}{2.156897in}}%
\pgfpathlineto{\pgfqpoint{3.651614in}{2.177525in}}%
\pgfpathlineto{\pgfqpoint{3.648040in}{2.198153in}}%
\pgfpathlineto{\pgfqpoint{3.644540in}{2.218780in}}%
\pgfpathlineto{\pgfqpoint{3.643139in}{2.227089in}}%
\pgfpathlineto{\pgfqpoint{3.602093in}{2.227089in}}%
\pgfpathlineto{\pgfqpoint{3.561048in}{2.227089in}}%
\pgfpathlineto{\pgfqpoint{3.520003in}{2.227089in}}%
\pgfpathlineto{\pgfqpoint{3.478957in}{2.227089in}}%
\pgfpathlineto{\pgfqpoint{3.437912in}{2.227089in}}%
\pgfpathlineto{\pgfqpoint{3.396867in}{2.227089in}}%
\pgfpathlineto{\pgfqpoint{3.355821in}{2.227089in}}%
\pgfpathlineto{\pgfqpoint{3.314776in}{2.227089in}}%
\pgfpathlineto{\pgfqpoint{3.273731in}{2.227089in}}%
\pgfpathlineto{\pgfqpoint{3.232685in}{2.227089in}}%
\pgfpathlineto{\pgfqpoint{3.191640in}{2.227089in}}%
\pgfpathlineto{\pgfqpoint{3.150595in}{2.227089in}}%
\pgfpathlineto{\pgfqpoint{3.109549in}{2.227089in}}%
\pgfpathlineto{\pgfqpoint{3.068504in}{2.227089in}}%
\pgfpathlineto{\pgfqpoint{3.027459in}{2.227089in}}%
\pgfpathlineto{\pgfqpoint{2.986413in}{2.227089in}}%
\pgfpathlineto{\pgfqpoint{2.945368in}{2.227089in}}%
\pgfpathlineto{\pgfqpoint{2.904323in}{2.227089in}}%
\pgfpathlineto{\pgfqpoint{2.863277in}{2.227089in}}%
\pgfpathlineto{\pgfqpoint{2.822232in}{2.227089in}}%
\pgfpathlineto{\pgfqpoint{2.781187in}{2.227089in}}%
\pgfpathlineto{\pgfqpoint{2.740141in}{2.227089in}}%
\pgfpathlineto{\pgfqpoint{2.699096in}{2.227089in}}%
\pgfpathlineto{\pgfqpoint{2.658051in}{2.227089in}}%
\pgfpathlineto{\pgfqpoint{2.617005in}{2.227089in}}%
\pgfpathlineto{\pgfqpoint{2.575960in}{2.227089in}}%
\pgfpathlineto{\pgfqpoint{2.534915in}{2.227089in}}%
\pgfpathclose%
\pgfusepath{stroke,fill}%
\end{pgfscope}%
\begin{pgfscope}%
\pgfpathrectangle{\pgfqpoint{0.605784in}{0.382904in}}{\pgfqpoint{4.063488in}{2.042155in}}%
\pgfusepath{clip}%
\pgfsetbuttcap%
\pgfsetroundjoin%
\definecolor{currentfill}{rgb}{0.140210,0.665859,0.513427}%
\pgfsetfillcolor{currentfill}%
\pgfsetlinewidth{1.003750pt}%
\definecolor{currentstroke}{rgb}{0.140210,0.665859,0.513427}%
\pgfsetstrokecolor{currentstroke}%
\pgfsetdash{}{0pt}%
\pgfpathmoveto{\pgfqpoint{0.613099in}{2.198153in}}%
\pgfpathlineto{\pgfqpoint{0.605784in}{2.202159in}}%
\pgfpathlineto{\pgfqpoint{0.605784in}{2.218780in}}%
\pgfpathlineto{\pgfqpoint{0.605784in}{2.239408in}}%
\pgfpathlineto{\pgfqpoint{0.605784in}{2.260036in}}%
\pgfpathlineto{\pgfqpoint{0.605784in}{2.260812in}}%
\pgfpathlineto{\pgfqpoint{0.607209in}{2.260036in}}%
\pgfpathlineto{\pgfqpoint{0.645075in}{2.239408in}}%
\pgfpathlineto{\pgfqpoint{0.646829in}{2.238451in}}%
\pgfpathlineto{\pgfqpoint{0.681786in}{2.218780in}}%
\pgfpathlineto{\pgfqpoint{0.687875in}{2.215341in}}%
\pgfpathlineto{\pgfqpoint{0.724807in}{2.198153in}}%
\pgfpathlineto{\pgfqpoint{0.728920in}{2.196219in}}%
\pgfpathlineto{\pgfqpoint{0.769965in}{2.185780in}}%
\pgfpathlineto{\pgfqpoint{0.811011in}{2.187969in}}%
\pgfpathlineto{\pgfqpoint{0.835801in}{2.198153in}}%
\pgfpathlineto{\pgfqpoint{0.852056in}{2.204805in}}%
\pgfpathlineto{\pgfqpoint{0.870513in}{2.218780in}}%
\pgfpathlineto{\pgfqpoint{0.893101in}{2.236001in}}%
\pgfpathlineto{\pgfqpoint{0.896369in}{2.239408in}}%
\pgfpathlineto{\pgfqpoint{0.916161in}{2.260036in}}%
\pgfpathlineto{\pgfqpoint{0.934147in}{2.279027in}}%
\pgfpathlineto{\pgfqpoint{0.935480in}{2.280664in}}%
\pgfpathlineto{\pgfqpoint{0.952314in}{2.301292in}}%
\pgfpathlineto{\pgfqpoint{0.968901in}{2.321920in}}%
\pgfpathlineto{\pgfqpoint{0.975192in}{2.329769in}}%
\pgfpathlineto{\pgfqpoint{0.985041in}{2.342547in}}%
\pgfpathlineto{\pgfqpoint{1.000797in}{2.363175in}}%
\pgfpathlineto{\pgfqpoint{1.016237in}{2.383704in}}%
\pgfpathlineto{\pgfqpoint{1.016315in}{2.383803in}}%
\pgfpathlineto{\pgfqpoint{1.032615in}{2.404431in}}%
\pgfpathlineto{\pgfqpoint{1.048628in}{2.425059in}}%
\pgfpathlineto{\pgfqpoint{1.057283in}{2.425059in}}%
\pgfpathlineto{\pgfqpoint{1.097501in}{2.425059in}}%
\pgfpathlineto{\pgfqpoint{1.080140in}{2.404431in}}%
\pgfpathlineto{\pgfqpoint{1.062422in}{2.383803in}}%
\pgfpathlineto{\pgfqpoint{1.057283in}{2.377853in}}%
\pgfpathlineto{\pgfqpoint{1.046060in}{2.363175in}}%
\pgfpathlineto{\pgfqpoint{1.030095in}{2.342547in}}%
\pgfpathlineto{\pgfqpoint{1.016237in}{2.324921in}}%
\pgfpathlineto{\pgfqpoint{1.013994in}{2.321920in}}%
\pgfpathlineto{\pgfqpoint{0.998587in}{2.301292in}}%
\pgfpathlineto{\pgfqpoint{0.982925in}{2.280664in}}%
\pgfpathlineto{\pgfqpoint{0.975192in}{2.270547in}}%
\pgfpathlineto{\pgfqpoint{0.966863in}{2.260036in}}%
\pgfpathlineto{\pgfqpoint{0.950424in}{2.239408in}}%
\pgfpathlineto{\pgfqpoint{0.934147in}{2.219285in}}%
\pgfpathlineto{\pgfqpoint{0.933672in}{2.218780in}}%
\pgfpathlineto{\pgfqpoint{0.914361in}{2.198153in}}%
\pgfpathlineto{\pgfqpoint{0.894757in}{2.177525in}}%
\pgfpathlineto{\pgfqpoint{0.893101in}{2.175779in}}%
\pgfpathlineto{\pgfqpoint{0.868568in}{2.156897in}}%
\pgfpathlineto{\pgfqpoint{0.852056in}{2.144281in}}%
\pgfpathlineto{\pgfqpoint{0.832460in}{2.136269in}}%
\pgfpathlineto{\pgfqpoint{0.811011in}{2.127464in}}%
\pgfpathlineto{\pgfqpoint{0.769965in}{2.125580in}}%
\pgfpathlineto{\pgfqpoint{0.729959in}{2.136269in}}%
\pgfpathlineto{\pgfqpoint{0.728920in}{2.136541in}}%
\pgfpathlineto{\pgfqpoint{0.687875in}{2.156200in}}%
\pgfpathlineto{\pgfqpoint{0.686663in}{2.156897in}}%
\pgfpathlineto{\pgfqpoint{0.650597in}{2.177525in}}%
\pgfpathlineto{\pgfqpoint{0.646829in}{2.179670in}}%
\pgfpathclose%
\pgfusepath{stroke,fill}%
\end{pgfscope}%
\begin{pgfscope}%
\pgfpathrectangle{\pgfqpoint{0.605784in}{0.382904in}}{\pgfqpoint{4.063488in}{2.042155in}}%
\pgfusepath{clip}%
\pgfsetbuttcap%
\pgfsetroundjoin%
\definecolor{currentfill}{rgb}{0.140210,0.665859,0.513427}%
\pgfsetfillcolor{currentfill}%
\pgfsetlinewidth{1.003750pt}%
\definecolor{currentstroke}{rgb}{0.140210,0.665859,0.513427}%
\pgfsetstrokecolor{currentstroke}%
\pgfsetdash{}{0pt}%
\pgfpathmoveto{\pgfqpoint{0.891516in}{0.403532in}}%
\pgfpathlineto{\pgfqpoint{0.893101in}{0.404401in}}%
\pgfpathlineto{\pgfqpoint{0.921622in}{0.424160in}}%
\pgfpathlineto{\pgfqpoint{0.934147in}{0.432688in}}%
\pgfpathlineto{\pgfqpoint{0.948551in}{0.444787in}}%
\pgfpathlineto{\pgfqpoint{0.973673in}{0.465415in}}%
\pgfpathlineto{\pgfqpoint{0.975192in}{0.466653in}}%
\pgfpathlineto{\pgfqpoint{0.995147in}{0.486043in}}%
\pgfpathlineto{\pgfqpoint{1.016237in}{0.505952in}}%
\pgfpathlineto{\pgfqpoint{1.016912in}{0.506671in}}%
\pgfpathlineto{\pgfqpoint{1.036326in}{0.527299in}}%
\pgfpathlineto{\pgfqpoint{1.056289in}{0.547927in}}%
\pgfpathlineto{\pgfqpoint{1.057283in}{0.548949in}}%
\pgfpathlineto{\pgfqpoint{1.075376in}{0.568554in}}%
\pgfpathlineto{\pgfqpoint{1.094866in}{0.589182in}}%
\pgfpathlineto{\pgfqpoint{1.098328in}{0.592826in}}%
\pgfpathlineto{\pgfqpoint{1.115143in}{0.609810in}}%
\pgfpathlineto{\pgfqpoint{1.135948in}{0.630438in}}%
\pgfpathlineto{\pgfqpoint{1.139373in}{0.633820in}}%
\pgfpathlineto{\pgfqpoint{1.159901in}{0.651066in}}%
\pgfpathlineto{\pgfqpoint{1.180419in}{0.668059in}}%
\pgfpathlineto{\pgfqpoint{1.186549in}{0.671694in}}%
\pgfpathlineto{\pgfqpoint{1.221464in}{0.692307in}}%
\pgfpathlineto{\pgfqpoint{1.221509in}{0.692321in}}%
\pgfpathlineto{\pgfqpoint{1.262509in}{0.705032in}}%
\pgfpathlineto{\pgfqpoint{1.303555in}{0.706443in}}%
\pgfpathlineto{\pgfqpoint{1.344600in}{0.698930in}}%
\pgfpathlineto{\pgfqpoint{1.365825in}{0.692321in}}%
\pgfpathlineto{\pgfqpoint{1.385645in}{0.686201in}}%
\pgfpathlineto{\pgfqpoint{1.426691in}{0.672451in}}%
\pgfpathlineto{\pgfqpoint{1.429560in}{0.671694in}}%
\pgfpathlineto{\pgfqpoint{1.467736in}{0.661622in}}%
\pgfpathlineto{\pgfqpoint{1.508781in}{0.656212in}}%
\pgfpathlineto{\pgfqpoint{1.549827in}{0.657389in}}%
\pgfpathlineto{\pgfqpoint{1.590872in}{0.664661in}}%
\pgfpathlineto{\pgfqpoint{1.615495in}{0.671694in}}%
\pgfpathlineto{\pgfqpoint{1.631917in}{0.676501in}}%
\pgfpathlineto{\pgfqpoint{1.672963in}{0.691202in}}%
\pgfpathlineto{\pgfqpoint{1.675724in}{0.692321in}}%
\pgfpathlineto{\pgfqpoint{1.714008in}{0.707847in}}%
\pgfpathlineto{\pgfqpoint{1.725171in}{0.712949in}}%
\pgfpathlineto{\pgfqpoint{1.755053in}{0.726465in}}%
\pgfpathlineto{\pgfqpoint{1.768408in}{0.733577in}}%
\pgfpathlineto{\pgfqpoint{1.796099in}{0.748034in}}%
\pgfpathlineto{\pgfqpoint{1.805826in}{0.754205in}}%
\pgfpathlineto{\pgfqpoint{1.837144in}{0.773556in}}%
\pgfpathlineto{\pgfqpoint{1.838862in}{0.774833in}}%
\pgfpathlineto{\pgfqpoint{1.867055in}{0.795460in}}%
\pgfpathlineto{\pgfqpoint{1.878189in}{0.803384in}}%
\pgfpathlineto{\pgfqpoint{1.893732in}{0.816088in}}%
\pgfpathlineto{\pgfqpoint{1.919235in}{0.836329in}}%
\pgfpathlineto{\pgfqpoint{1.919695in}{0.836716in}}%
\pgfpathlineto{\pgfqpoint{1.944366in}{0.857344in}}%
\pgfpathlineto{\pgfqpoint{1.960280in}{0.870311in}}%
\pgfpathlineto{\pgfqpoint{1.970051in}{0.877972in}}%
\pgfpathlineto{\pgfqpoint{1.996764in}{0.898600in}}%
\pgfpathlineto{\pgfqpoint{2.001325in}{0.902086in}}%
\pgfpathlineto{\pgfqpoint{2.027720in}{0.919227in}}%
\pgfpathlineto{\pgfqpoint{2.042371in}{0.928633in}}%
\pgfpathlineto{\pgfqpoint{2.066762in}{0.939855in}}%
\pgfpathlineto{\pgfqpoint{2.083416in}{0.947507in}}%
\pgfpathlineto{\pgfqpoint{2.124461in}{0.957778in}}%
\pgfpathlineto{\pgfqpoint{2.165507in}{0.960284in}}%
\pgfpathlineto{\pgfqpoint{2.206552in}{0.957461in}}%
\pgfpathlineto{\pgfqpoint{2.247597in}{0.952641in}}%
\pgfpathlineto{\pgfqpoint{2.288643in}{0.949351in}}%
\pgfpathlineto{\pgfqpoint{2.329688in}{0.950463in}}%
\pgfpathlineto{\pgfqpoint{2.370733in}{0.957416in}}%
\pgfpathlineto{\pgfqpoint{2.381024in}{0.960483in}}%
\pgfpathlineto{\pgfqpoint{2.411779in}{0.970028in}}%
\pgfpathlineto{\pgfqpoint{2.439975in}{0.981111in}}%
\pgfpathlineto{\pgfqpoint{2.452824in}{0.986332in}}%
\pgfpathlineto{\pgfqpoint{2.489447in}{1.001739in}}%
\pgfpathlineto{\pgfqpoint{2.493869in}{1.003642in}}%
\pgfpathlineto{\pgfqpoint{2.494064in}{1.001739in}}%
\pgfpathlineto{\pgfqpoint{2.496172in}{0.981111in}}%
\pgfpathlineto{\pgfqpoint{2.498225in}{0.960483in}}%
\pgfpathlineto{\pgfqpoint{2.500227in}{0.939855in}}%
\pgfpathlineto{\pgfqpoint{2.502183in}{0.919227in}}%
\pgfpathlineto{\pgfqpoint{2.504095in}{0.898600in}}%
\pgfpathlineto{\pgfqpoint{2.505966in}{0.877972in}}%
\pgfpathlineto{\pgfqpoint{2.507800in}{0.857344in}}%
\pgfpathlineto{\pgfqpoint{2.509598in}{0.836716in}}%
\pgfpathlineto{\pgfqpoint{2.511363in}{0.816088in}}%
\pgfpathlineto{\pgfqpoint{2.513097in}{0.795460in}}%
\pgfpathlineto{\pgfqpoint{2.514802in}{0.774833in}}%
\pgfpathlineto{\pgfqpoint{2.516479in}{0.754205in}}%
\pgfpathlineto{\pgfqpoint{2.518131in}{0.733577in}}%
\pgfpathlineto{\pgfqpoint{2.519759in}{0.712949in}}%
\pgfpathlineto{\pgfqpoint{2.521364in}{0.692321in}}%
\pgfpathlineto{\pgfqpoint{2.522947in}{0.671694in}}%
\pgfpathlineto{\pgfqpoint{2.524509in}{0.651066in}}%
\pgfpathlineto{\pgfqpoint{2.526053in}{0.630438in}}%
\pgfpathlineto{\pgfqpoint{2.527578in}{0.609810in}}%
\pgfpathlineto{\pgfqpoint{2.529086in}{0.589182in}}%
\pgfpathlineto{\pgfqpoint{2.530578in}{0.568554in}}%
\pgfpathlineto{\pgfqpoint{2.532053in}{0.547927in}}%
\pgfpathlineto{\pgfqpoint{2.533514in}{0.527299in}}%
\pgfpathlineto{\pgfqpoint{2.534915in}{0.507348in}}%
\pgfpathlineto{\pgfqpoint{2.575960in}{0.507348in}}%
\pgfpathlineto{\pgfqpoint{2.617005in}{0.507348in}}%
\pgfpathlineto{\pgfqpoint{2.658051in}{0.507348in}}%
\pgfpathlineto{\pgfqpoint{2.699096in}{0.507348in}}%
\pgfpathlineto{\pgfqpoint{2.740141in}{0.507348in}}%
\pgfpathlineto{\pgfqpoint{2.781187in}{0.507348in}}%
\pgfpathlineto{\pgfqpoint{2.822232in}{0.507348in}}%
\pgfpathlineto{\pgfqpoint{2.863277in}{0.507348in}}%
\pgfpathlineto{\pgfqpoint{2.904323in}{0.507348in}}%
\pgfpathlineto{\pgfqpoint{2.945368in}{0.507348in}}%
\pgfpathlineto{\pgfqpoint{2.986413in}{0.507348in}}%
\pgfpathlineto{\pgfqpoint{3.027459in}{0.507348in}}%
\pgfpathlineto{\pgfqpoint{3.068504in}{0.507348in}}%
\pgfpathlineto{\pgfqpoint{3.109549in}{0.507348in}}%
\pgfpathlineto{\pgfqpoint{3.150595in}{0.507348in}}%
\pgfpathlineto{\pgfqpoint{3.191640in}{0.507348in}}%
\pgfpathlineto{\pgfqpoint{3.232685in}{0.507348in}}%
\pgfpathlineto{\pgfqpoint{3.273731in}{0.507348in}}%
\pgfpathlineto{\pgfqpoint{3.314776in}{0.507348in}}%
\pgfpathlineto{\pgfqpoint{3.355821in}{0.507348in}}%
\pgfpathlineto{\pgfqpoint{3.396867in}{0.507348in}}%
\pgfpathlineto{\pgfqpoint{3.437912in}{0.507348in}}%
\pgfpathlineto{\pgfqpoint{3.478957in}{0.507348in}}%
\pgfpathlineto{\pgfqpoint{3.520003in}{0.507348in}}%
\pgfpathlineto{\pgfqpoint{3.561048in}{0.507348in}}%
\pgfpathlineto{\pgfqpoint{3.602093in}{0.507348in}}%
\pgfpathlineto{\pgfqpoint{3.643139in}{0.507348in}}%
\pgfpathlineto{\pgfqpoint{3.643282in}{0.506671in}}%
\pgfpathlineto{\pgfqpoint{3.647646in}{0.486043in}}%
\pgfpathlineto{\pgfqpoint{3.651889in}{0.465415in}}%
\pgfpathlineto{\pgfqpoint{3.656020in}{0.444787in}}%
\pgfpathlineto{\pgfqpoint{3.660047in}{0.424160in}}%
\pgfpathlineto{\pgfqpoint{3.663977in}{0.403532in}}%
\pgfpathlineto{\pgfqpoint{3.667817in}{0.382904in}}%
\pgfpathlineto{\pgfqpoint{3.656352in}{0.382904in}}%
\pgfpathlineto{\pgfqpoint{3.652245in}{0.403532in}}%
\pgfpathlineto{\pgfqpoint{3.648036in}{0.424160in}}%
\pgfpathlineto{\pgfqpoint{3.643716in}{0.444787in}}%
\pgfpathlineto{\pgfqpoint{3.643139in}{0.447529in}}%
\pgfpathlineto{\pgfqpoint{3.602093in}{0.447529in}}%
\pgfpathlineto{\pgfqpoint{3.561048in}{0.447529in}}%
\pgfpathlineto{\pgfqpoint{3.520003in}{0.447529in}}%
\pgfpathlineto{\pgfqpoint{3.478957in}{0.447529in}}%
\pgfpathlineto{\pgfqpoint{3.437912in}{0.447529in}}%
\pgfpathlineto{\pgfqpoint{3.396867in}{0.447529in}}%
\pgfpathlineto{\pgfqpoint{3.355821in}{0.447529in}}%
\pgfpathlineto{\pgfqpoint{3.314776in}{0.447529in}}%
\pgfpathlineto{\pgfqpoint{3.273731in}{0.447529in}}%
\pgfpathlineto{\pgfqpoint{3.232685in}{0.447529in}}%
\pgfpathlineto{\pgfqpoint{3.191640in}{0.447529in}}%
\pgfpathlineto{\pgfqpoint{3.150595in}{0.447529in}}%
\pgfpathlineto{\pgfqpoint{3.109549in}{0.447529in}}%
\pgfpathlineto{\pgfqpoint{3.068504in}{0.447529in}}%
\pgfpathlineto{\pgfqpoint{3.027459in}{0.447529in}}%
\pgfpathlineto{\pgfqpoint{2.986413in}{0.447529in}}%
\pgfpathlineto{\pgfqpoint{2.945368in}{0.447529in}}%
\pgfpathlineto{\pgfqpoint{2.904323in}{0.447529in}}%
\pgfpathlineto{\pgfqpoint{2.863277in}{0.447529in}}%
\pgfpathlineto{\pgfqpoint{2.822232in}{0.447529in}}%
\pgfpathlineto{\pgfqpoint{2.781187in}{0.447529in}}%
\pgfpathlineto{\pgfqpoint{2.740141in}{0.447529in}}%
\pgfpathlineto{\pgfqpoint{2.699096in}{0.447529in}}%
\pgfpathlineto{\pgfqpoint{2.658051in}{0.447529in}}%
\pgfpathlineto{\pgfqpoint{2.617005in}{0.447529in}}%
\pgfpathlineto{\pgfqpoint{2.575960in}{0.447529in}}%
\pgfpathlineto{\pgfqpoint{2.534915in}{0.447529in}}%
\pgfpathlineto{\pgfqpoint{2.533635in}{0.465415in}}%
\pgfpathlineto{\pgfqpoint{2.532147in}{0.486043in}}%
\pgfpathlineto{\pgfqpoint{2.530645in}{0.506671in}}%
\pgfpathlineto{\pgfqpoint{2.529127in}{0.527299in}}%
\pgfpathlineto{\pgfqpoint{2.527592in}{0.547927in}}%
\pgfpathlineto{\pgfqpoint{2.526040in}{0.568554in}}%
\pgfpathlineto{\pgfqpoint{2.524470in}{0.589182in}}%
\pgfpathlineto{\pgfqpoint{2.522881in}{0.609810in}}%
\pgfpathlineto{\pgfqpoint{2.521271in}{0.630438in}}%
\pgfpathlineto{\pgfqpoint{2.519640in}{0.651066in}}%
\pgfpathlineto{\pgfqpoint{2.517986in}{0.671694in}}%
\pgfpathlineto{\pgfqpoint{2.516309in}{0.692321in}}%
\pgfpathlineto{\pgfqpoint{2.514606in}{0.712949in}}%
\pgfpathlineto{\pgfqpoint{2.512877in}{0.733577in}}%
\pgfpathlineto{\pgfqpoint{2.511119in}{0.754205in}}%
\pgfpathlineto{\pgfqpoint{2.509332in}{0.774833in}}%
\pgfpathlineto{\pgfqpoint{2.507512in}{0.795460in}}%
\pgfpathlineto{\pgfqpoint{2.505658in}{0.816088in}}%
\pgfpathlineto{\pgfqpoint{2.503768in}{0.836716in}}%
\pgfpathlineto{\pgfqpoint{2.501839in}{0.857344in}}%
\pgfpathlineto{\pgfqpoint{2.499869in}{0.877972in}}%
\pgfpathlineto{\pgfqpoint{2.497855in}{0.898600in}}%
\pgfpathlineto{\pgfqpoint{2.495793in}{0.919227in}}%
\pgfpathlineto{\pgfqpoint{2.493869in}{0.938050in}}%
\pgfpathlineto{\pgfqpoint{2.452824in}{0.921850in}}%
\pgfpathlineto{\pgfqpoint{2.445654in}{0.919227in}}%
\pgfpathlineto{\pgfqpoint{2.411779in}{0.907216in}}%
\pgfpathlineto{\pgfqpoint{2.379338in}{0.898600in}}%
\pgfpathlineto{\pgfqpoint{2.370733in}{0.896399in}}%
\pgfpathlineto{\pgfqpoint{2.329688in}{0.891089in}}%
\pgfpathlineto{\pgfqpoint{2.288643in}{0.891286in}}%
\pgfpathlineto{\pgfqpoint{2.247597in}{0.895477in}}%
\pgfpathlineto{\pgfqpoint{2.223766in}{0.898600in}}%
\pgfpathlineto{\pgfqpoint{2.206552in}{0.900869in}}%
\pgfpathlineto{\pgfqpoint{2.165507in}{0.904002in}}%
\pgfpathlineto{\pgfqpoint{2.124461in}{0.901519in}}%
\pgfpathlineto{\pgfqpoint{2.112951in}{0.898600in}}%
\pgfpathlineto{\pgfqpoint{2.083416in}{0.891176in}}%
\pgfpathlineto{\pgfqpoint{2.054954in}{0.877972in}}%
\pgfpathlineto{\pgfqpoint{2.042371in}{0.872127in}}%
\pgfpathlineto{\pgfqpoint{2.019611in}{0.857344in}}%
\pgfpathlineto{\pgfqpoint{2.001325in}{0.845338in}}%
\pgfpathlineto{\pgfqpoint{1.990222in}{0.836716in}}%
\pgfpathlineto{\pgfqpoint{1.964059in}{0.816088in}}%
\pgfpathlineto{\pgfqpoint{1.960280in}{0.813084in}}%
\pgfpathlineto{\pgfqpoint{1.939103in}{0.795460in}}%
\pgfpathlineto{\pgfqpoint{1.919235in}{0.778513in}}%
\pgfpathlineto{\pgfqpoint{1.914687in}{0.774833in}}%
\pgfpathlineto{\pgfqpoint{1.889561in}{0.754205in}}%
\pgfpathlineto{\pgfqpoint{1.878189in}{0.744659in}}%
\pgfpathlineto{\pgfqpoint{1.863213in}{0.733577in}}%
\pgfpathlineto{\pgfqpoint{1.837144in}{0.713737in}}%
\pgfpathlineto{\pgfqpoint{1.835909in}{0.712949in}}%
\pgfpathlineto{\pgfqpoint{1.804077in}{0.692321in}}%
\pgfpathlineto{\pgfqpoint{1.796099in}{0.687030in}}%
\pgfpathlineto{\pgfqpoint{1.768138in}{0.671694in}}%
\pgfpathlineto{\pgfqpoint{1.755053in}{0.664382in}}%
\pgfpathlineto{\pgfqpoint{1.726736in}{0.651066in}}%
\pgfpathlineto{\pgfqpoint{1.714008in}{0.645022in}}%
\pgfpathlineto{\pgfqpoint{1.678496in}{0.630438in}}%
\pgfpathlineto{\pgfqpoint{1.672963in}{0.628167in}}%
\pgfpathlineto{\pgfqpoint{1.631917in}{0.613815in}}%
\pgfpathlineto{\pgfqpoint{1.617186in}{0.609810in}}%
\pgfpathlineto{\pgfqpoint{1.590872in}{0.602819in}}%
\pgfpathlineto{\pgfqpoint{1.549827in}{0.596634in}}%
\pgfpathlineto{\pgfqpoint{1.508781in}{0.596537in}}%
\pgfpathlineto{\pgfqpoint{1.467736in}{0.602805in}}%
\pgfpathlineto{\pgfqpoint{1.442602in}{0.609810in}}%
\pgfpathlineto{\pgfqpoint{1.426691in}{0.614246in}}%
\pgfpathlineto{\pgfqpoint{1.385645in}{0.628203in}}%
\pgfpathlineto{\pgfqpoint{1.378416in}{0.630438in}}%
\pgfpathlineto{\pgfqpoint{1.344600in}{0.640975in}}%
\pgfpathlineto{\pgfqpoint{1.303555in}{0.648329in}}%
\pgfpathlineto{\pgfqpoint{1.262509in}{0.646768in}}%
\pgfpathlineto{\pgfqpoint{1.221464in}{0.633998in}}%
\pgfpathlineto{\pgfqpoint{1.215437in}{0.630438in}}%
\pgfpathlineto{\pgfqpoint{1.180648in}{0.609810in}}%
\pgfpathlineto{\pgfqpoint{1.180419in}{0.609674in}}%
\pgfpathlineto{\pgfqpoint{1.155663in}{0.589182in}}%
\pgfpathlineto{\pgfqpoint{1.139373in}{0.575517in}}%
\pgfpathlineto{\pgfqpoint{1.132329in}{0.568554in}}%
\pgfpathlineto{\pgfqpoint{1.111613in}{0.547927in}}%
\pgfpathlineto{\pgfqpoint{1.098328in}{0.534516in}}%
\pgfpathlineto{\pgfqpoint{1.091499in}{0.527299in}}%
\pgfpathlineto{\pgfqpoint{1.072155in}{0.506671in}}%
\pgfpathlineto{\pgfqpoint{1.057283in}{0.490521in}}%
\pgfpathlineto{\pgfqpoint{1.052968in}{0.486043in}}%
\pgfpathlineto{\pgfqpoint{1.033246in}{0.465415in}}%
\pgfpathlineto{\pgfqpoint{1.016237in}{0.447212in}}%
\pgfpathlineto{\pgfqpoint{1.013692in}{0.444787in}}%
\pgfpathlineto{\pgfqpoint{0.992187in}{0.424160in}}%
\pgfpathlineto{\pgfqpoint{0.975192in}{0.407459in}}%
\pgfpathlineto{\pgfqpoint{0.970456in}{0.403532in}}%
\pgfpathlineto{\pgfqpoint{0.945837in}{0.382904in}}%
\pgfpathlineto{\pgfqpoint{0.934147in}{0.382904in}}%
\pgfpathlineto{\pgfqpoint{0.893101in}{0.382904in}}%
\pgfpathlineto{\pgfqpoint{0.854125in}{0.382904in}}%
\pgfpathclose%
\pgfusepath{stroke,fill}%
\end{pgfscope}%
\begin{pgfscope}%
\pgfpathrectangle{\pgfqpoint{0.605784in}{0.382904in}}{\pgfqpoint{4.063488in}{2.042155in}}%
\pgfusepath{clip}%
\pgfsetbuttcap%
\pgfsetroundjoin%
\definecolor{currentfill}{rgb}{0.140210,0.665859,0.513427}%
\pgfsetfillcolor{currentfill}%
\pgfsetlinewidth{1.003750pt}%
\definecolor{currentstroke}{rgb}{0.140210,0.665859,0.513427}%
\pgfsetstrokecolor{currentstroke}%
\pgfsetdash{}{0pt}%
\pgfpathmoveto{\pgfqpoint{1.533995in}{2.425059in}}%
\pgfpathlineto{\pgfqpoint{1.549827in}{2.425059in}}%
\pgfpathlineto{\pgfqpoint{1.590872in}{2.425059in}}%
\pgfpathlineto{\pgfqpoint{1.631917in}{2.425059in}}%
\pgfpathlineto{\pgfqpoint{1.672963in}{2.425059in}}%
\pgfpathlineto{\pgfqpoint{1.714008in}{2.425059in}}%
\pgfpathlineto{\pgfqpoint{1.738470in}{2.425059in}}%
\pgfpathlineto{\pgfqpoint{1.717484in}{2.404431in}}%
\pgfpathlineto{\pgfqpoint{1.714008in}{2.401002in}}%
\pgfpathlineto{\pgfqpoint{1.683205in}{2.383803in}}%
\pgfpathlineto{\pgfqpoint{1.672963in}{2.378078in}}%
\pgfpathlineto{\pgfqpoint{1.631917in}{2.373951in}}%
\pgfpathlineto{\pgfqpoint{1.600786in}{2.383803in}}%
\pgfpathlineto{\pgfqpoint{1.590872in}{2.386868in}}%
\pgfpathlineto{\pgfqpoint{1.562781in}{2.404431in}}%
\pgfpathlineto{\pgfqpoint{1.549827in}{2.412425in}}%
\pgfpathclose%
\pgfusepath{stroke,fill}%
\end{pgfscope}%
\begin{pgfscope}%
\pgfpathrectangle{\pgfqpoint{0.605784in}{0.382904in}}{\pgfqpoint{4.063488in}{2.042155in}}%
\pgfusepath{clip}%
\pgfsetbuttcap%
\pgfsetroundjoin%
\definecolor{currentfill}{rgb}{0.140210,0.665859,0.513427}%
\pgfsetfillcolor{currentfill}%
\pgfsetlinewidth{1.003750pt}%
\definecolor{currentstroke}{rgb}{0.140210,0.665859,0.513427}%
\pgfsetstrokecolor{currentstroke}%
\pgfsetdash{}{0pt}%
\pgfpathmoveto{\pgfqpoint{2.533210in}{2.301292in}}%
\pgfpathlineto{\pgfqpoint{2.529832in}{2.321920in}}%
\pgfpathlineto{\pgfqpoint{2.526626in}{2.342547in}}%
\pgfpathlineto{\pgfqpoint{2.523573in}{2.363175in}}%
\pgfpathlineto{\pgfqpoint{2.520658in}{2.383803in}}%
\pgfpathlineto{\pgfqpoint{2.517868in}{2.404431in}}%
\pgfpathlineto{\pgfqpoint{2.515190in}{2.425059in}}%
\pgfpathlineto{\pgfqpoint{2.523674in}{2.425059in}}%
\pgfpathlineto{\pgfqpoint{2.526631in}{2.404431in}}%
\pgfpathlineto{\pgfqpoint{2.529720in}{2.383803in}}%
\pgfpathlineto{\pgfqpoint{2.532954in}{2.363175in}}%
\pgfpathlineto{\pgfqpoint{2.534915in}{2.351085in}}%
\pgfpathlineto{\pgfqpoint{2.575960in}{2.351085in}}%
\pgfpathlineto{\pgfqpoint{2.617005in}{2.351085in}}%
\pgfpathlineto{\pgfqpoint{2.658051in}{2.351085in}}%
\pgfpathlineto{\pgfqpoint{2.699096in}{2.351085in}}%
\pgfpathlineto{\pgfqpoint{2.740141in}{2.351085in}}%
\pgfpathlineto{\pgfqpoint{2.781187in}{2.351085in}}%
\pgfpathlineto{\pgfqpoint{2.822232in}{2.351085in}}%
\pgfpathlineto{\pgfqpoint{2.863277in}{2.351085in}}%
\pgfpathlineto{\pgfqpoint{2.904323in}{2.351085in}}%
\pgfpathlineto{\pgfqpoint{2.945368in}{2.351085in}}%
\pgfpathlineto{\pgfqpoint{2.986413in}{2.351085in}}%
\pgfpathlineto{\pgfqpoint{3.027459in}{2.351085in}}%
\pgfpathlineto{\pgfqpoint{3.068504in}{2.351085in}}%
\pgfpathlineto{\pgfqpoint{3.109549in}{2.351085in}}%
\pgfpathlineto{\pgfqpoint{3.150595in}{2.351085in}}%
\pgfpathlineto{\pgfqpoint{3.191640in}{2.351085in}}%
\pgfpathlineto{\pgfqpoint{3.232685in}{2.351085in}}%
\pgfpathlineto{\pgfqpoint{3.273731in}{2.351085in}}%
\pgfpathlineto{\pgfqpoint{3.314776in}{2.351085in}}%
\pgfpathlineto{\pgfqpoint{3.355821in}{2.351085in}}%
\pgfpathlineto{\pgfqpoint{3.396867in}{2.351085in}}%
\pgfpathlineto{\pgfqpoint{3.437912in}{2.351085in}}%
\pgfpathlineto{\pgfqpoint{3.478957in}{2.351085in}}%
\pgfpathlineto{\pgfqpoint{3.520003in}{2.351085in}}%
\pgfpathlineto{\pgfqpoint{3.561048in}{2.351085in}}%
\pgfpathlineto{\pgfqpoint{3.602093in}{2.351085in}}%
\pgfpathlineto{\pgfqpoint{3.643139in}{2.351085in}}%
\pgfpathlineto{\pgfqpoint{3.644598in}{2.342547in}}%
\pgfpathlineto{\pgfqpoint{3.648147in}{2.321920in}}%
\pgfpathlineto{\pgfqpoint{3.651762in}{2.301292in}}%
\pgfpathlineto{\pgfqpoint{3.655445in}{2.280664in}}%
\pgfpathlineto{\pgfqpoint{3.659203in}{2.260036in}}%
\pgfpathlineto{\pgfqpoint{3.663040in}{2.239408in}}%
\pgfpathlineto{\pgfqpoint{3.666960in}{2.218780in}}%
\pgfpathlineto{\pgfqpoint{3.670971in}{2.198153in}}%
\pgfpathlineto{\pgfqpoint{3.675077in}{2.177525in}}%
\pgfpathlineto{\pgfqpoint{3.679287in}{2.156897in}}%
\pgfpathlineto{\pgfqpoint{3.683607in}{2.136269in}}%
\pgfpathlineto{\pgfqpoint{3.684184in}{2.133528in}}%
\pgfpathlineto{\pgfqpoint{3.725229in}{2.133528in}}%
\pgfpathlineto{\pgfqpoint{3.766275in}{2.133528in}}%
\pgfpathlineto{\pgfqpoint{3.807320in}{2.133528in}}%
\pgfpathlineto{\pgfqpoint{3.848365in}{2.133528in}}%
\pgfpathlineto{\pgfqpoint{3.889411in}{2.133528in}}%
\pgfpathlineto{\pgfqpoint{3.930456in}{2.133528in}}%
\pgfpathlineto{\pgfqpoint{3.971501in}{2.133528in}}%
\pgfpathlineto{\pgfqpoint{4.012547in}{2.133528in}}%
\pgfpathlineto{\pgfqpoint{4.053592in}{2.133528in}}%
\pgfpathlineto{\pgfqpoint{4.094637in}{2.133528in}}%
\pgfpathlineto{\pgfqpoint{4.135683in}{2.133528in}}%
\pgfpathlineto{\pgfqpoint{4.176728in}{2.133528in}}%
\pgfpathlineto{\pgfqpoint{4.217773in}{2.133528in}}%
\pgfpathlineto{\pgfqpoint{4.258819in}{2.133528in}}%
\pgfpathlineto{\pgfqpoint{4.299864in}{2.133528in}}%
\pgfpathlineto{\pgfqpoint{4.340909in}{2.133528in}}%
\pgfpathlineto{\pgfqpoint{4.381955in}{2.133528in}}%
\pgfpathlineto{\pgfqpoint{4.423000in}{2.133528in}}%
\pgfpathlineto{\pgfqpoint{4.464045in}{2.133528in}}%
\pgfpathlineto{\pgfqpoint{4.505091in}{2.133528in}}%
\pgfpathlineto{\pgfqpoint{4.546136in}{2.133528in}}%
\pgfpathlineto{\pgfqpoint{4.587181in}{2.133528in}}%
\pgfpathlineto{\pgfqpoint{4.628227in}{2.133528in}}%
\pgfpathlineto{\pgfqpoint{4.669272in}{2.133528in}}%
\pgfpathlineto{\pgfqpoint{4.669272in}{2.115641in}}%
\pgfpathlineto{\pgfqpoint{4.669272in}{2.095013in}}%
\pgfpathlineto{\pgfqpoint{4.669272in}{2.074386in}}%
\pgfpathlineto{\pgfqpoint{4.669272in}{2.073708in}}%
\pgfpathlineto{\pgfqpoint{4.628227in}{2.073708in}}%
\pgfpathlineto{\pgfqpoint{4.587181in}{2.073708in}}%
\pgfpathlineto{\pgfqpoint{4.546136in}{2.073708in}}%
\pgfpathlineto{\pgfqpoint{4.505091in}{2.073708in}}%
\pgfpathlineto{\pgfqpoint{4.464045in}{2.073708in}}%
\pgfpathlineto{\pgfqpoint{4.423000in}{2.073708in}}%
\pgfpathlineto{\pgfqpoint{4.381955in}{2.073708in}}%
\pgfpathlineto{\pgfqpoint{4.340909in}{2.073708in}}%
\pgfpathlineto{\pgfqpoint{4.299864in}{2.073708in}}%
\pgfpathlineto{\pgfqpoint{4.258819in}{2.073708in}}%
\pgfpathlineto{\pgfqpoint{4.217773in}{2.073708in}}%
\pgfpathlineto{\pgfqpoint{4.176728in}{2.073708in}}%
\pgfpathlineto{\pgfqpoint{4.135683in}{2.073708in}}%
\pgfpathlineto{\pgfqpoint{4.094637in}{2.073708in}}%
\pgfpathlineto{\pgfqpoint{4.053592in}{2.073708in}}%
\pgfpathlineto{\pgfqpoint{4.012547in}{2.073708in}}%
\pgfpathlineto{\pgfqpoint{3.971501in}{2.073708in}}%
\pgfpathlineto{\pgfqpoint{3.930456in}{2.073708in}}%
\pgfpathlineto{\pgfqpoint{3.889411in}{2.073708in}}%
\pgfpathlineto{\pgfqpoint{3.848365in}{2.073708in}}%
\pgfpathlineto{\pgfqpoint{3.807320in}{2.073708in}}%
\pgfpathlineto{\pgfqpoint{3.766275in}{2.073708in}}%
\pgfpathlineto{\pgfqpoint{3.725229in}{2.073708in}}%
\pgfpathlineto{\pgfqpoint{3.684184in}{2.073708in}}%
\pgfpathlineto{\pgfqpoint{3.684040in}{2.074386in}}%
\pgfpathlineto{\pgfqpoint{3.679676in}{2.095013in}}%
\pgfpathlineto{\pgfqpoint{3.675433in}{2.115641in}}%
\pgfpathlineto{\pgfqpoint{3.671302in}{2.136269in}}%
\pgfpathlineto{\pgfqpoint{3.667276in}{2.156897in}}%
\pgfpathlineto{\pgfqpoint{3.663345in}{2.177525in}}%
\pgfpathlineto{\pgfqpoint{3.659506in}{2.198153in}}%
\pgfpathlineto{\pgfqpoint{3.655750in}{2.218780in}}%
\pgfpathlineto{\pgfqpoint{3.652073in}{2.239408in}}%
\pgfpathlineto{\pgfqpoint{3.648470in}{2.260036in}}%
\pgfpathlineto{\pgfqpoint{3.644936in}{2.280664in}}%
\pgfpathlineto{\pgfqpoint{3.643139in}{2.291242in}}%
\pgfpathlineto{\pgfqpoint{3.602093in}{2.291242in}}%
\pgfpathlineto{\pgfqpoint{3.561048in}{2.291242in}}%
\pgfpathlineto{\pgfqpoint{3.520003in}{2.291242in}}%
\pgfpathlineto{\pgfqpoint{3.478957in}{2.291242in}}%
\pgfpathlineto{\pgfqpoint{3.437912in}{2.291242in}}%
\pgfpathlineto{\pgfqpoint{3.396867in}{2.291242in}}%
\pgfpathlineto{\pgfqpoint{3.355821in}{2.291242in}}%
\pgfpathlineto{\pgfqpoint{3.314776in}{2.291242in}}%
\pgfpathlineto{\pgfqpoint{3.273731in}{2.291242in}}%
\pgfpathlineto{\pgfqpoint{3.232685in}{2.291242in}}%
\pgfpathlineto{\pgfqpoint{3.191640in}{2.291242in}}%
\pgfpathlineto{\pgfqpoint{3.150595in}{2.291242in}}%
\pgfpathlineto{\pgfqpoint{3.109549in}{2.291242in}}%
\pgfpathlineto{\pgfqpoint{3.068504in}{2.291242in}}%
\pgfpathlineto{\pgfqpoint{3.027459in}{2.291242in}}%
\pgfpathlineto{\pgfqpoint{2.986413in}{2.291242in}}%
\pgfpathlineto{\pgfqpoint{2.945368in}{2.291242in}}%
\pgfpathlineto{\pgfqpoint{2.904323in}{2.291242in}}%
\pgfpathlineto{\pgfqpoint{2.863277in}{2.291242in}}%
\pgfpathlineto{\pgfqpoint{2.822232in}{2.291242in}}%
\pgfpathlineto{\pgfqpoint{2.781187in}{2.291242in}}%
\pgfpathlineto{\pgfqpoint{2.740141in}{2.291242in}}%
\pgfpathlineto{\pgfqpoint{2.699096in}{2.291242in}}%
\pgfpathlineto{\pgfqpoint{2.658051in}{2.291242in}}%
\pgfpathlineto{\pgfqpoint{2.617005in}{2.291242in}}%
\pgfpathlineto{\pgfqpoint{2.575960in}{2.291242in}}%
\pgfpathlineto{\pgfqpoint{2.534915in}{2.291242in}}%
\pgfpathclose%
\pgfusepath{stroke,fill}%
\end{pgfscope}%
\begin{pgfscope}%
\pgfpathrectangle{\pgfqpoint{0.605784in}{0.382904in}}{\pgfqpoint{4.063488in}{2.042155in}}%
\pgfusepath{clip}%
\pgfsetbuttcap%
\pgfsetroundjoin%
\definecolor{currentfill}{rgb}{0.226397,0.728888,0.462789}%
\pgfsetfillcolor{currentfill}%
\pgfsetlinewidth{1.003750pt}%
\definecolor{currentstroke}{rgb}{0.226397,0.728888,0.462789}%
\pgfsetstrokecolor{currentstroke}%
\pgfsetdash{}{0pt}%
\pgfpathmoveto{\pgfqpoint{0.645075in}{2.239408in}}%
\pgfpathlineto{\pgfqpoint{0.607209in}{2.260036in}}%
\pgfpathlineto{\pgfqpoint{0.605784in}{2.260812in}}%
\pgfpathlineto{\pgfqpoint{0.605784in}{2.280664in}}%
\pgfpathlineto{\pgfqpoint{0.605784in}{2.301292in}}%
\pgfpathlineto{\pgfqpoint{0.605784in}{2.316056in}}%
\pgfpathlineto{\pgfqpoint{0.633033in}{2.301292in}}%
\pgfpathlineto{\pgfqpoint{0.646829in}{2.293813in}}%
\pgfpathlineto{\pgfqpoint{0.670575in}{2.280664in}}%
\pgfpathlineto{\pgfqpoint{0.687875in}{2.271051in}}%
\pgfpathlineto{\pgfqpoint{0.712158in}{2.260036in}}%
\pgfpathlineto{\pgfqpoint{0.728920in}{2.252361in}}%
\pgfpathlineto{\pgfqpoint{0.769965in}{2.242396in}}%
\pgfpathlineto{\pgfqpoint{0.811011in}{2.244828in}}%
\pgfpathlineto{\pgfqpoint{0.848007in}{2.260036in}}%
\pgfpathlineto{\pgfqpoint{0.852056in}{2.261694in}}%
\pgfpathlineto{\pgfqpoint{0.877309in}{2.280664in}}%
\pgfpathlineto{\pgfqpoint{0.893101in}{2.292602in}}%
\pgfpathlineto{\pgfqpoint{0.901519in}{2.301292in}}%
\pgfpathlineto{\pgfqpoint{0.921419in}{2.321920in}}%
\pgfpathlineto{\pgfqpoint{0.934147in}{2.335216in}}%
\pgfpathlineto{\pgfqpoint{0.940175in}{2.342547in}}%
\pgfpathlineto{\pgfqpoint{0.957084in}{2.363175in}}%
\pgfpathlineto{\pgfqpoint{0.973756in}{2.383803in}}%
\pgfpathlineto{\pgfqpoint{0.975192in}{2.385577in}}%
\pgfpathlineto{\pgfqpoint{0.989830in}{2.404431in}}%
\pgfpathlineto{\pgfqpoint{1.005625in}{2.425059in}}%
\pgfpathlineto{\pgfqpoint{1.016237in}{2.425059in}}%
\pgfpathlineto{\pgfqpoint{1.048628in}{2.425059in}}%
\pgfpathlineto{\pgfqpoint{1.032615in}{2.404431in}}%
\pgfpathlineto{\pgfqpoint{1.016315in}{2.383803in}}%
\pgfpathlineto{\pgfqpoint{1.016237in}{2.383704in}}%
\pgfpathlineto{\pgfqpoint{1.000797in}{2.363175in}}%
\pgfpathlineto{\pgfqpoint{0.985041in}{2.342547in}}%
\pgfpathlineto{\pgfqpoint{0.975192in}{2.329769in}}%
\pgfpathlineto{\pgfqpoint{0.968901in}{2.321920in}}%
\pgfpathlineto{\pgfqpoint{0.952314in}{2.301292in}}%
\pgfpathlineto{\pgfqpoint{0.935480in}{2.280664in}}%
\pgfpathlineto{\pgfqpoint{0.934147in}{2.279027in}}%
\pgfpathlineto{\pgfqpoint{0.916161in}{2.260036in}}%
\pgfpathlineto{\pgfqpoint{0.896369in}{2.239408in}}%
\pgfpathlineto{\pgfqpoint{0.893101in}{2.236001in}}%
\pgfpathlineto{\pgfqpoint{0.870513in}{2.218780in}}%
\pgfpathlineto{\pgfqpoint{0.852056in}{2.204805in}}%
\pgfpathlineto{\pgfqpoint{0.835801in}{2.198153in}}%
\pgfpathlineto{\pgfqpoint{0.811011in}{2.187969in}}%
\pgfpathlineto{\pgfqpoint{0.769965in}{2.185780in}}%
\pgfpathlineto{\pgfqpoint{0.728920in}{2.196219in}}%
\pgfpathlineto{\pgfqpoint{0.724807in}{2.198153in}}%
\pgfpathlineto{\pgfqpoint{0.687875in}{2.215341in}}%
\pgfpathlineto{\pgfqpoint{0.681786in}{2.218780in}}%
\pgfpathlineto{\pgfqpoint{0.646829in}{2.238451in}}%
\pgfpathclose%
\pgfusepath{stroke,fill}%
\end{pgfscope}%
\begin{pgfscope}%
\pgfpathrectangle{\pgfqpoint{0.605784in}{0.382904in}}{\pgfqpoint{4.063488in}{2.042155in}}%
\pgfusepath{clip}%
\pgfsetbuttcap%
\pgfsetroundjoin%
\definecolor{currentfill}{rgb}{0.226397,0.728888,0.462789}%
\pgfsetfillcolor{currentfill}%
\pgfsetlinewidth{1.003750pt}%
\definecolor{currentstroke}{rgb}{0.226397,0.728888,0.462789}%
\pgfsetstrokecolor{currentstroke}%
\pgfsetdash{}{0pt}%
\pgfpathmoveto{\pgfqpoint{0.970456in}{0.403532in}}%
\pgfpathlineto{\pgfqpoint{0.975192in}{0.407459in}}%
\pgfpathlineto{\pgfqpoint{0.992187in}{0.424160in}}%
\pgfpathlineto{\pgfqpoint{1.013692in}{0.444787in}}%
\pgfpathlineto{\pgfqpoint{1.016237in}{0.447212in}}%
\pgfpathlineto{\pgfqpoint{1.033246in}{0.465415in}}%
\pgfpathlineto{\pgfqpoint{1.052968in}{0.486043in}}%
\pgfpathlineto{\pgfqpoint{1.057283in}{0.490521in}}%
\pgfpathlineto{\pgfqpoint{1.072155in}{0.506671in}}%
\pgfpathlineto{\pgfqpoint{1.091499in}{0.527299in}}%
\pgfpathlineto{\pgfqpoint{1.098328in}{0.534516in}}%
\pgfpathlineto{\pgfqpoint{1.111613in}{0.547927in}}%
\pgfpathlineto{\pgfqpoint{1.132329in}{0.568554in}}%
\pgfpathlineto{\pgfqpoint{1.139373in}{0.575517in}}%
\pgfpathlineto{\pgfqpoint{1.155663in}{0.589182in}}%
\pgfpathlineto{\pgfqpoint{1.180419in}{0.609674in}}%
\pgfpathlineto{\pgfqpoint{1.180648in}{0.609810in}}%
\pgfpathlineto{\pgfqpoint{1.215437in}{0.630438in}}%
\pgfpathlineto{\pgfqpoint{1.221464in}{0.633998in}}%
\pgfpathlineto{\pgfqpoint{1.262509in}{0.646768in}}%
\pgfpathlineto{\pgfqpoint{1.303555in}{0.648329in}}%
\pgfpathlineto{\pgfqpoint{1.344600in}{0.640975in}}%
\pgfpathlineto{\pgfqpoint{1.378416in}{0.630438in}}%
\pgfpathlineto{\pgfqpoint{1.385645in}{0.628203in}}%
\pgfpathlineto{\pgfqpoint{1.426691in}{0.614246in}}%
\pgfpathlineto{\pgfqpoint{1.442602in}{0.609810in}}%
\pgfpathlineto{\pgfqpoint{1.467736in}{0.602805in}}%
\pgfpathlineto{\pgfqpoint{1.508781in}{0.596537in}}%
\pgfpathlineto{\pgfqpoint{1.549827in}{0.596634in}}%
\pgfpathlineto{\pgfqpoint{1.590872in}{0.602819in}}%
\pgfpathlineto{\pgfqpoint{1.617186in}{0.609810in}}%
\pgfpathlineto{\pgfqpoint{1.631917in}{0.613815in}}%
\pgfpathlineto{\pgfqpoint{1.672963in}{0.628167in}}%
\pgfpathlineto{\pgfqpoint{1.678496in}{0.630438in}}%
\pgfpathlineto{\pgfqpoint{1.714008in}{0.645022in}}%
\pgfpathlineto{\pgfqpoint{1.726736in}{0.651066in}}%
\pgfpathlineto{\pgfqpoint{1.755053in}{0.664382in}}%
\pgfpathlineto{\pgfqpoint{1.768138in}{0.671694in}}%
\pgfpathlineto{\pgfqpoint{1.796099in}{0.687030in}}%
\pgfpathlineto{\pgfqpoint{1.804077in}{0.692321in}}%
\pgfpathlineto{\pgfqpoint{1.835909in}{0.712949in}}%
\pgfpathlineto{\pgfqpoint{1.837144in}{0.713737in}}%
\pgfpathlineto{\pgfqpoint{1.863213in}{0.733577in}}%
\pgfpathlineto{\pgfqpoint{1.878189in}{0.744659in}}%
\pgfpathlineto{\pgfqpoint{1.889561in}{0.754205in}}%
\pgfpathlineto{\pgfqpoint{1.914687in}{0.774833in}}%
\pgfpathlineto{\pgfqpoint{1.919235in}{0.778513in}}%
\pgfpathlineto{\pgfqpoint{1.939103in}{0.795460in}}%
\pgfpathlineto{\pgfqpoint{1.960280in}{0.813084in}}%
\pgfpathlineto{\pgfqpoint{1.964059in}{0.816088in}}%
\pgfpathlineto{\pgfqpoint{1.990222in}{0.836716in}}%
\pgfpathlineto{\pgfqpoint{2.001325in}{0.845338in}}%
\pgfpathlineto{\pgfqpoint{2.019611in}{0.857344in}}%
\pgfpathlineto{\pgfqpoint{2.042371in}{0.872127in}}%
\pgfpathlineto{\pgfqpoint{2.054954in}{0.877972in}}%
\pgfpathlineto{\pgfqpoint{2.083416in}{0.891176in}}%
\pgfpathlineto{\pgfqpoint{2.112951in}{0.898600in}}%
\pgfpathlineto{\pgfqpoint{2.124461in}{0.901519in}}%
\pgfpathlineto{\pgfqpoint{2.165507in}{0.904002in}}%
\pgfpathlineto{\pgfqpoint{2.206552in}{0.900869in}}%
\pgfpathlineto{\pgfqpoint{2.223766in}{0.898600in}}%
\pgfpathlineto{\pgfqpoint{2.247597in}{0.895477in}}%
\pgfpathlineto{\pgfqpoint{2.288643in}{0.891286in}}%
\pgfpathlineto{\pgfqpoint{2.329688in}{0.891089in}}%
\pgfpathlineto{\pgfqpoint{2.370733in}{0.896399in}}%
\pgfpathlineto{\pgfqpoint{2.379338in}{0.898600in}}%
\pgfpathlineto{\pgfqpoint{2.411779in}{0.907216in}}%
\pgfpathlineto{\pgfqpoint{2.445654in}{0.919227in}}%
\pgfpathlineto{\pgfqpoint{2.452824in}{0.921850in}}%
\pgfpathlineto{\pgfqpoint{2.493869in}{0.938050in}}%
\pgfpathlineto{\pgfqpoint{2.495793in}{0.919227in}}%
\pgfpathlineto{\pgfqpoint{2.497855in}{0.898600in}}%
\pgfpathlineto{\pgfqpoint{2.499869in}{0.877972in}}%
\pgfpathlineto{\pgfqpoint{2.501839in}{0.857344in}}%
\pgfpathlineto{\pgfqpoint{2.503768in}{0.836716in}}%
\pgfpathlineto{\pgfqpoint{2.505658in}{0.816088in}}%
\pgfpathlineto{\pgfqpoint{2.507512in}{0.795460in}}%
\pgfpathlineto{\pgfqpoint{2.509332in}{0.774833in}}%
\pgfpathlineto{\pgfqpoint{2.511119in}{0.754205in}}%
\pgfpathlineto{\pgfqpoint{2.512877in}{0.733577in}}%
\pgfpathlineto{\pgfqpoint{2.514606in}{0.712949in}}%
\pgfpathlineto{\pgfqpoint{2.516309in}{0.692321in}}%
\pgfpathlineto{\pgfqpoint{2.517986in}{0.671694in}}%
\pgfpathlineto{\pgfqpoint{2.519640in}{0.651066in}}%
\pgfpathlineto{\pgfqpoint{2.521271in}{0.630438in}}%
\pgfpathlineto{\pgfqpoint{2.522881in}{0.609810in}}%
\pgfpathlineto{\pgfqpoint{2.524470in}{0.589182in}}%
\pgfpathlineto{\pgfqpoint{2.526040in}{0.568554in}}%
\pgfpathlineto{\pgfqpoint{2.527592in}{0.547927in}}%
\pgfpathlineto{\pgfqpoint{2.529127in}{0.527299in}}%
\pgfpathlineto{\pgfqpoint{2.530645in}{0.506671in}}%
\pgfpathlineto{\pgfqpoint{2.532147in}{0.486043in}}%
\pgfpathlineto{\pgfqpoint{2.533635in}{0.465415in}}%
\pgfpathlineto{\pgfqpoint{2.534915in}{0.447529in}}%
\pgfpathlineto{\pgfqpoint{2.575960in}{0.447529in}}%
\pgfpathlineto{\pgfqpoint{2.617005in}{0.447529in}}%
\pgfpathlineto{\pgfqpoint{2.658051in}{0.447529in}}%
\pgfpathlineto{\pgfqpoint{2.699096in}{0.447529in}}%
\pgfpathlineto{\pgfqpoint{2.740141in}{0.447529in}}%
\pgfpathlineto{\pgfqpoint{2.781187in}{0.447529in}}%
\pgfpathlineto{\pgfqpoint{2.822232in}{0.447529in}}%
\pgfpathlineto{\pgfqpoint{2.863277in}{0.447529in}}%
\pgfpathlineto{\pgfqpoint{2.904323in}{0.447529in}}%
\pgfpathlineto{\pgfqpoint{2.945368in}{0.447529in}}%
\pgfpathlineto{\pgfqpoint{2.986413in}{0.447529in}}%
\pgfpathlineto{\pgfqpoint{3.027459in}{0.447529in}}%
\pgfpathlineto{\pgfqpoint{3.068504in}{0.447529in}}%
\pgfpathlineto{\pgfqpoint{3.109549in}{0.447529in}}%
\pgfpathlineto{\pgfqpoint{3.150595in}{0.447529in}}%
\pgfpathlineto{\pgfqpoint{3.191640in}{0.447529in}}%
\pgfpathlineto{\pgfqpoint{3.232685in}{0.447529in}}%
\pgfpathlineto{\pgfqpoint{3.273731in}{0.447529in}}%
\pgfpathlineto{\pgfqpoint{3.314776in}{0.447529in}}%
\pgfpathlineto{\pgfqpoint{3.355821in}{0.447529in}}%
\pgfpathlineto{\pgfqpoint{3.396867in}{0.447529in}}%
\pgfpathlineto{\pgfqpoint{3.437912in}{0.447529in}}%
\pgfpathlineto{\pgfqpoint{3.478957in}{0.447529in}}%
\pgfpathlineto{\pgfqpoint{3.520003in}{0.447529in}}%
\pgfpathlineto{\pgfqpoint{3.561048in}{0.447529in}}%
\pgfpathlineto{\pgfqpoint{3.602093in}{0.447529in}}%
\pgfpathlineto{\pgfqpoint{3.643139in}{0.447529in}}%
\pgfpathlineto{\pgfqpoint{3.643716in}{0.444787in}}%
\pgfpathlineto{\pgfqpoint{3.648036in}{0.424160in}}%
\pgfpathlineto{\pgfqpoint{3.652245in}{0.403532in}}%
\pgfpathlineto{\pgfqpoint{3.656352in}{0.382904in}}%
\pgfpathlineto{\pgfqpoint{3.644886in}{0.382904in}}%
\pgfpathlineto{\pgfqpoint{3.643139in}{0.391262in}}%
\pgfpathlineto{\pgfqpoint{3.602093in}{0.391262in}}%
\pgfpathlineto{\pgfqpoint{3.561048in}{0.391262in}}%
\pgfpathlineto{\pgfqpoint{3.520003in}{0.391262in}}%
\pgfpathlineto{\pgfqpoint{3.478957in}{0.391262in}}%
\pgfpathlineto{\pgfqpoint{3.437912in}{0.391262in}}%
\pgfpathlineto{\pgfqpoint{3.396867in}{0.391262in}}%
\pgfpathlineto{\pgfqpoint{3.355821in}{0.391262in}}%
\pgfpathlineto{\pgfqpoint{3.314776in}{0.391262in}}%
\pgfpathlineto{\pgfqpoint{3.273731in}{0.391262in}}%
\pgfpathlineto{\pgfqpoint{3.232685in}{0.391262in}}%
\pgfpathlineto{\pgfqpoint{3.191640in}{0.391262in}}%
\pgfpathlineto{\pgfqpoint{3.150595in}{0.391262in}}%
\pgfpathlineto{\pgfqpoint{3.109549in}{0.391262in}}%
\pgfpathlineto{\pgfqpoint{3.068504in}{0.391262in}}%
\pgfpathlineto{\pgfqpoint{3.027459in}{0.391262in}}%
\pgfpathlineto{\pgfqpoint{2.986413in}{0.391262in}}%
\pgfpathlineto{\pgfqpoint{2.945368in}{0.391262in}}%
\pgfpathlineto{\pgfqpoint{2.904323in}{0.391262in}}%
\pgfpathlineto{\pgfqpoint{2.863277in}{0.391262in}}%
\pgfpathlineto{\pgfqpoint{2.822232in}{0.391262in}}%
\pgfpathlineto{\pgfqpoint{2.781187in}{0.391262in}}%
\pgfpathlineto{\pgfqpoint{2.740141in}{0.391262in}}%
\pgfpathlineto{\pgfqpoint{2.699096in}{0.391262in}}%
\pgfpathlineto{\pgfqpoint{2.658051in}{0.391262in}}%
\pgfpathlineto{\pgfqpoint{2.617005in}{0.391262in}}%
\pgfpathlineto{\pgfqpoint{2.575960in}{0.391262in}}%
\pgfpathlineto{\pgfqpoint{2.534915in}{0.391262in}}%
\pgfpathlineto{\pgfqpoint{2.534021in}{0.403532in}}%
\pgfpathlineto{\pgfqpoint{2.532514in}{0.424160in}}%
\pgfpathlineto{\pgfqpoint{2.530991in}{0.444787in}}%
\pgfpathlineto{\pgfqpoint{2.529454in}{0.465415in}}%
\pgfpathlineto{\pgfqpoint{2.527899in}{0.486043in}}%
\pgfpathlineto{\pgfqpoint{2.526328in}{0.506671in}}%
\pgfpathlineto{\pgfqpoint{2.524739in}{0.527299in}}%
\pgfpathlineto{\pgfqpoint{2.523131in}{0.547927in}}%
\pgfpathlineto{\pgfqpoint{2.521503in}{0.568554in}}%
\pgfpathlineto{\pgfqpoint{2.519854in}{0.589182in}}%
\pgfpathlineto{\pgfqpoint{2.518183in}{0.609810in}}%
\pgfpathlineto{\pgfqpoint{2.516489in}{0.630438in}}%
\pgfpathlineto{\pgfqpoint{2.514770in}{0.651066in}}%
\pgfpathlineto{\pgfqpoint{2.513026in}{0.671694in}}%
\pgfpathlineto{\pgfqpoint{2.511254in}{0.692321in}}%
\pgfpathlineto{\pgfqpoint{2.509454in}{0.712949in}}%
\pgfpathlineto{\pgfqpoint{2.507623in}{0.733577in}}%
\pgfpathlineto{\pgfqpoint{2.505759in}{0.754205in}}%
\pgfpathlineto{\pgfqpoint{2.503861in}{0.774833in}}%
\pgfpathlineto{\pgfqpoint{2.501927in}{0.795460in}}%
\pgfpathlineto{\pgfqpoint{2.499953in}{0.816088in}}%
\pgfpathlineto{\pgfqpoint{2.497938in}{0.836716in}}%
\pgfpathlineto{\pgfqpoint{2.495879in}{0.857344in}}%
\pgfpathlineto{\pgfqpoint{2.493869in}{0.877041in}}%
\pgfpathlineto{\pgfqpoint{2.452824in}{0.861774in}}%
\pgfpathlineto{\pgfqpoint{2.439418in}{0.857344in}}%
\pgfpathlineto{\pgfqpoint{2.411779in}{0.848471in}}%
\pgfpathlineto{\pgfqpoint{2.370733in}{0.839136in}}%
\pgfpathlineto{\pgfqpoint{2.346184in}{0.836716in}}%
\pgfpathlineto{\pgfqpoint{2.329688in}{0.835153in}}%
\pgfpathlineto{\pgfqpoint{2.288643in}{0.836437in}}%
\pgfpathlineto{\pgfqpoint{2.286353in}{0.836716in}}%
\pgfpathlineto{\pgfqpoint{2.247597in}{0.841457in}}%
\pgfpathlineto{\pgfqpoint{2.206552in}{0.847340in}}%
\pgfpathlineto{\pgfqpoint{2.165507in}{0.850670in}}%
\pgfpathlineto{\pgfqpoint{2.124461in}{0.848260in}}%
\pgfpathlineto{\pgfqpoint{2.083416in}{0.837796in}}%
\pgfpathlineto{\pgfqpoint{2.081102in}{0.836716in}}%
\pgfpathlineto{\pgfqpoint{2.042371in}{0.818624in}}%
\pgfpathlineto{\pgfqpoint{2.038493in}{0.816088in}}%
\pgfpathlineto{\pgfqpoint{2.007146in}{0.795460in}}%
\pgfpathlineto{\pgfqpoint{2.001325in}{0.791602in}}%
\pgfpathlineto{\pgfqpoint{1.980021in}{0.774833in}}%
\pgfpathlineto{\pgfqpoint{1.960280in}{0.759014in}}%
\pgfpathlineto{\pgfqpoint{1.954573in}{0.754205in}}%
\pgfpathlineto{\pgfqpoint{1.930378in}{0.733577in}}%
\pgfpathlineto{\pgfqpoint{1.919235in}{0.723910in}}%
\pgfpathlineto{\pgfqpoint{1.906035in}{0.712949in}}%
\pgfpathlineto{\pgfqpoint{1.881744in}{0.692321in}}%
\pgfpathlineto{\pgfqpoint{1.878189in}{0.689268in}}%
\pgfpathlineto{\pgfqpoint{1.855214in}{0.671694in}}%
\pgfpathlineto{\pgfqpoint{1.837144in}{0.657502in}}%
\pgfpathlineto{\pgfqpoint{1.827459in}{0.651066in}}%
\pgfpathlineto{\pgfqpoint{1.797113in}{0.630438in}}%
\pgfpathlineto{\pgfqpoint{1.796099in}{0.629740in}}%
\pgfpathlineto{\pgfqpoint{1.761235in}{0.609810in}}%
\pgfpathlineto{\pgfqpoint{1.755053in}{0.606214in}}%
\pgfpathlineto{\pgfqpoint{1.720004in}{0.589182in}}%
\pgfpathlineto{\pgfqpoint{1.714008in}{0.586242in}}%
\pgfpathlineto{\pgfqpoint{1.672963in}{0.569214in}}%
\pgfpathlineto{\pgfqpoint{1.671052in}{0.568554in}}%
\pgfpathlineto{\pgfqpoint{1.631917in}{0.555192in}}%
\pgfpathlineto{\pgfqpoint{1.603277in}{0.547927in}}%
\pgfpathlineto{\pgfqpoint{1.590872in}{0.544847in}}%
\pgfpathlineto{\pgfqpoint{1.549827in}{0.539578in}}%
\pgfpathlineto{\pgfqpoint{1.508781in}{0.540372in}}%
\pgfpathlineto{\pgfqpoint{1.467736in}{0.547329in}}%
\pgfpathlineto{\pgfqpoint{1.465688in}{0.547927in}}%
\pgfpathlineto{\pgfqpoint{1.426691in}{0.559309in}}%
\pgfpathlineto{\pgfqpoint{1.399906in}{0.568554in}}%
\pgfpathlineto{\pgfqpoint{1.385645in}{0.573489in}}%
\pgfpathlineto{\pgfqpoint{1.344600in}{0.586226in}}%
\pgfpathlineto{\pgfqpoint{1.327795in}{0.589182in}}%
\pgfpathlineto{\pgfqpoint{1.303555in}{0.593510in}}%
\pgfpathlineto{\pgfqpoint{1.262509in}{0.591795in}}%
\pgfpathlineto{\pgfqpoint{1.254184in}{0.589182in}}%
\pgfpathlineto{\pgfqpoint{1.221464in}{0.578977in}}%
\pgfpathlineto{\pgfqpoint{1.203828in}{0.568554in}}%
\pgfpathlineto{\pgfqpoint{1.180419in}{0.554668in}}%
\pgfpathlineto{\pgfqpoint{1.172272in}{0.547927in}}%
\pgfpathlineto{\pgfqpoint{1.147499in}{0.527299in}}%
\pgfpathlineto{\pgfqpoint{1.139373in}{0.520491in}}%
\pgfpathlineto{\pgfqpoint{1.125402in}{0.506671in}}%
\pgfpathlineto{\pgfqpoint{1.104830in}{0.486043in}}%
\pgfpathlineto{\pgfqpoint{1.098328in}{0.479483in}}%
\pgfpathlineto{\pgfqpoint{1.085065in}{0.465415in}}%
\pgfpathlineto{\pgfqpoint{1.065912in}{0.444787in}}%
\pgfpathlineto{\pgfqpoint{1.057283in}{0.435399in}}%
\pgfpathlineto{\pgfqpoint{1.046535in}{0.424160in}}%
\pgfpathlineto{\pgfqpoint{1.027090in}{0.403532in}}%
\pgfpathlineto{\pgfqpoint{1.016237in}{0.391852in}}%
\pgfpathlineto{\pgfqpoint{1.006954in}{0.382904in}}%
\pgfpathlineto{\pgfqpoint{0.975192in}{0.382904in}}%
\pgfpathlineto{\pgfqpoint{0.945837in}{0.382904in}}%
\pgfpathclose%
\pgfusepath{stroke,fill}%
\end{pgfscope}%
\begin{pgfscope}%
\pgfpathrectangle{\pgfqpoint{0.605784in}{0.382904in}}{\pgfqpoint{4.063488in}{2.042155in}}%
\pgfusepath{clip}%
\pgfsetbuttcap%
\pgfsetroundjoin%
\definecolor{currentfill}{rgb}{0.226397,0.728888,0.462789}%
\pgfsetfillcolor{currentfill}%
\pgfsetlinewidth{1.003750pt}%
\definecolor{currentstroke}{rgb}{0.226397,0.728888,0.462789}%
\pgfsetstrokecolor{currentstroke}%
\pgfsetdash{}{0pt}%
\pgfpathmoveto{\pgfqpoint{2.532954in}{2.363175in}}%
\pgfpathlineto{\pgfqpoint{2.529720in}{2.383803in}}%
\pgfpathlineto{\pgfqpoint{2.526631in}{2.404431in}}%
\pgfpathlineto{\pgfqpoint{2.523674in}{2.425059in}}%
\pgfpathlineto{\pgfqpoint{2.532158in}{2.425059in}}%
\pgfpathlineto{\pgfqpoint{2.534915in}{2.407405in}}%
\pgfpathlineto{\pgfqpoint{2.575960in}{2.407405in}}%
\pgfpathlineto{\pgfqpoint{2.617005in}{2.407405in}}%
\pgfpathlineto{\pgfqpoint{2.658051in}{2.407405in}}%
\pgfpathlineto{\pgfqpoint{2.699096in}{2.407405in}}%
\pgfpathlineto{\pgfqpoint{2.740141in}{2.407405in}}%
\pgfpathlineto{\pgfqpoint{2.781187in}{2.407405in}}%
\pgfpathlineto{\pgfqpoint{2.822232in}{2.407405in}}%
\pgfpathlineto{\pgfqpoint{2.863277in}{2.407405in}}%
\pgfpathlineto{\pgfqpoint{2.904323in}{2.407405in}}%
\pgfpathlineto{\pgfqpoint{2.945368in}{2.407405in}}%
\pgfpathlineto{\pgfqpoint{2.986413in}{2.407405in}}%
\pgfpathlineto{\pgfqpoint{3.027459in}{2.407405in}}%
\pgfpathlineto{\pgfqpoint{3.068504in}{2.407405in}}%
\pgfpathlineto{\pgfqpoint{3.109549in}{2.407405in}}%
\pgfpathlineto{\pgfqpoint{3.150595in}{2.407405in}}%
\pgfpathlineto{\pgfqpoint{3.191640in}{2.407405in}}%
\pgfpathlineto{\pgfqpoint{3.232685in}{2.407405in}}%
\pgfpathlineto{\pgfqpoint{3.273731in}{2.407405in}}%
\pgfpathlineto{\pgfqpoint{3.314776in}{2.407405in}}%
\pgfpathlineto{\pgfqpoint{3.355821in}{2.407405in}}%
\pgfpathlineto{\pgfqpoint{3.396867in}{2.407405in}}%
\pgfpathlineto{\pgfqpoint{3.437912in}{2.407405in}}%
\pgfpathlineto{\pgfqpoint{3.478957in}{2.407405in}}%
\pgfpathlineto{\pgfqpoint{3.520003in}{2.407405in}}%
\pgfpathlineto{\pgfqpoint{3.561048in}{2.407405in}}%
\pgfpathlineto{\pgfqpoint{3.602093in}{2.407405in}}%
\pgfpathlineto{\pgfqpoint{3.643139in}{2.407405in}}%
\pgfpathlineto{\pgfqpoint{3.643650in}{2.404431in}}%
\pgfpathlineto{\pgfqpoint{3.647200in}{2.383803in}}%
\pgfpathlineto{\pgfqpoint{3.650812in}{2.363175in}}%
\pgfpathlineto{\pgfqpoint{3.654490in}{2.342547in}}%
\pgfpathlineto{\pgfqpoint{3.658237in}{2.321920in}}%
\pgfpathlineto{\pgfqpoint{3.662057in}{2.301292in}}%
\pgfpathlineto{\pgfqpoint{3.665955in}{2.280664in}}%
\pgfpathlineto{\pgfqpoint{3.669937in}{2.260036in}}%
\pgfpathlineto{\pgfqpoint{3.674007in}{2.239408in}}%
\pgfpathlineto{\pgfqpoint{3.678171in}{2.218780in}}%
\pgfpathlineto{\pgfqpoint{3.682436in}{2.198153in}}%
\pgfpathlineto{\pgfqpoint{3.684184in}{2.189794in}}%
\pgfpathlineto{\pgfqpoint{3.725229in}{2.189794in}}%
\pgfpathlineto{\pgfqpoint{3.766275in}{2.189794in}}%
\pgfpathlineto{\pgfqpoint{3.807320in}{2.189794in}}%
\pgfpathlineto{\pgfqpoint{3.848365in}{2.189794in}}%
\pgfpathlineto{\pgfqpoint{3.889411in}{2.189794in}}%
\pgfpathlineto{\pgfqpoint{3.930456in}{2.189794in}}%
\pgfpathlineto{\pgfqpoint{3.971501in}{2.189794in}}%
\pgfpathlineto{\pgfqpoint{4.012547in}{2.189794in}}%
\pgfpathlineto{\pgfqpoint{4.053592in}{2.189794in}}%
\pgfpathlineto{\pgfqpoint{4.094637in}{2.189794in}}%
\pgfpathlineto{\pgfqpoint{4.135683in}{2.189794in}}%
\pgfpathlineto{\pgfqpoint{4.176728in}{2.189794in}}%
\pgfpathlineto{\pgfqpoint{4.217773in}{2.189794in}}%
\pgfpathlineto{\pgfqpoint{4.258819in}{2.189794in}}%
\pgfpathlineto{\pgfqpoint{4.299864in}{2.189794in}}%
\pgfpathlineto{\pgfqpoint{4.340909in}{2.189794in}}%
\pgfpathlineto{\pgfqpoint{4.381955in}{2.189794in}}%
\pgfpathlineto{\pgfqpoint{4.423000in}{2.189794in}}%
\pgfpathlineto{\pgfqpoint{4.464045in}{2.189794in}}%
\pgfpathlineto{\pgfqpoint{4.505091in}{2.189794in}}%
\pgfpathlineto{\pgfqpoint{4.546136in}{2.189794in}}%
\pgfpathlineto{\pgfqpoint{4.587181in}{2.189794in}}%
\pgfpathlineto{\pgfqpoint{4.628227in}{2.189794in}}%
\pgfpathlineto{\pgfqpoint{4.669272in}{2.189794in}}%
\pgfpathlineto{\pgfqpoint{4.669272in}{2.177525in}}%
\pgfpathlineto{\pgfqpoint{4.669272in}{2.156897in}}%
\pgfpathlineto{\pgfqpoint{4.669272in}{2.136269in}}%
\pgfpathlineto{\pgfqpoint{4.669272in}{2.133528in}}%
\pgfpathlineto{\pgfqpoint{4.628227in}{2.133528in}}%
\pgfpathlineto{\pgfqpoint{4.587181in}{2.133528in}}%
\pgfpathlineto{\pgfqpoint{4.546136in}{2.133528in}}%
\pgfpathlineto{\pgfqpoint{4.505091in}{2.133528in}}%
\pgfpathlineto{\pgfqpoint{4.464045in}{2.133528in}}%
\pgfpathlineto{\pgfqpoint{4.423000in}{2.133528in}}%
\pgfpathlineto{\pgfqpoint{4.381955in}{2.133528in}}%
\pgfpathlineto{\pgfqpoint{4.340909in}{2.133528in}}%
\pgfpathlineto{\pgfqpoint{4.299864in}{2.133528in}}%
\pgfpathlineto{\pgfqpoint{4.258819in}{2.133528in}}%
\pgfpathlineto{\pgfqpoint{4.217773in}{2.133528in}}%
\pgfpathlineto{\pgfqpoint{4.176728in}{2.133528in}}%
\pgfpathlineto{\pgfqpoint{4.135683in}{2.133528in}}%
\pgfpathlineto{\pgfqpoint{4.094637in}{2.133528in}}%
\pgfpathlineto{\pgfqpoint{4.053592in}{2.133528in}}%
\pgfpathlineto{\pgfqpoint{4.012547in}{2.133528in}}%
\pgfpathlineto{\pgfqpoint{3.971501in}{2.133528in}}%
\pgfpathlineto{\pgfqpoint{3.930456in}{2.133528in}}%
\pgfpathlineto{\pgfqpoint{3.889411in}{2.133528in}}%
\pgfpathlineto{\pgfqpoint{3.848365in}{2.133528in}}%
\pgfpathlineto{\pgfqpoint{3.807320in}{2.133528in}}%
\pgfpathlineto{\pgfqpoint{3.766275in}{2.133528in}}%
\pgfpathlineto{\pgfqpoint{3.725229in}{2.133528in}}%
\pgfpathlineto{\pgfqpoint{3.684184in}{2.133528in}}%
\pgfpathlineto{\pgfqpoint{3.683607in}{2.136269in}}%
\pgfpathlineto{\pgfqpoint{3.679287in}{2.156897in}}%
\pgfpathlineto{\pgfqpoint{3.675077in}{2.177525in}}%
\pgfpathlineto{\pgfqpoint{3.670971in}{2.198153in}}%
\pgfpathlineto{\pgfqpoint{3.666960in}{2.218780in}}%
\pgfpathlineto{\pgfqpoint{3.663040in}{2.239408in}}%
\pgfpathlineto{\pgfqpoint{3.659203in}{2.260036in}}%
\pgfpathlineto{\pgfqpoint{3.655445in}{2.280664in}}%
\pgfpathlineto{\pgfqpoint{3.651762in}{2.301292in}}%
\pgfpathlineto{\pgfqpoint{3.648147in}{2.321920in}}%
\pgfpathlineto{\pgfqpoint{3.644598in}{2.342547in}}%
\pgfpathlineto{\pgfqpoint{3.643139in}{2.351085in}}%
\pgfpathlineto{\pgfqpoint{3.602093in}{2.351085in}}%
\pgfpathlineto{\pgfqpoint{3.561048in}{2.351085in}}%
\pgfpathlineto{\pgfqpoint{3.520003in}{2.351085in}}%
\pgfpathlineto{\pgfqpoint{3.478957in}{2.351085in}}%
\pgfpathlineto{\pgfqpoint{3.437912in}{2.351085in}}%
\pgfpathlineto{\pgfqpoint{3.396867in}{2.351085in}}%
\pgfpathlineto{\pgfqpoint{3.355821in}{2.351085in}}%
\pgfpathlineto{\pgfqpoint{3.314776in}{2.351085in}}%
\pgfpathlineto{\pgfqpoint{3.273731in}{2.351085in}}%
\pgfpathlineto{\pgfqpoint{3.232685in}{2.351085in}}%
\pgfpathlineto{\pgfqpoint{3.191640in}{2.351085in}}%
\pgfpathlineto{\pgfqpoint{3.150595in}{2.351085in}}%
\pgfpathlineto{\pgfqpoint{3.109549in}{2.351085in}}%
\pgfpathlineto{\pgfqpoint{3.068504in}{2.351085in}}%
\pgfpathlineto{\pgfqpoint{3.027459in}{2.351085in}}%
\pgfpathlineto{\pgfqpoint{2.986413in}{2.351085in}}%
\pgfpathlineto{\pgfqpoint{2.945368in}{2.351085in}}%
\pgfpathlineto{\pgfqpoint{2.904323in}{2.351085in}}%
\pgfpathlineto{\pgfqpoint{2.863277in}{2.351085in}}%
\pgfpathlineto{\pgfqpoint{2.822232in}{2.351085in}}%
\pgfpathlineto{\pgfqpoint{2.781187in}{2.351085in}}%
\pgfpathlineto{\pgfqpoint{2.740141in}{2.351085in}}%
\pgfpathlineto{\pgfqpoint{2.699096in}{2.351085in}}%
\pgfpathlineto{\pgfqpoint{2.658051in}{2.351085in}}%
\pgfpathlineto{\pgfqpoint{2.617005in}{2.351085in}}%
\pgfpathlineto{\pgfqpoint{2.575960in}{2.351085in}}%
\pgfpathlineto{\pgfqpoint{2.534915in}{2.351085in}}%
\pgfpathclose%
\pgfusepath{stroke,fill}%
\end{pgfscope}%
\begin{pgfscope}%
\pgfpathrectangle{\pgfqpoint{0.605784in}{0.382904in}}{\pgfqpoint{4.063488in}{2.042155in}}%
\pgfusepath{clip}%
\pgfsetbuttcap%
\pgfsetroundjoin%
\definecolor{currentfill}{rgb}{0.369214,0.788888,0.382914}%
\pgfsetfillcolor{currentfill}%
\pgfsetlinewidth{1.003750pt}%
\definecolor{currentstroke}{rgb}{0.369214,0.788888,0.382914}%
\pgfsetstrokecolor{currentstroke}%
\pgfsetdash{}{0pt}%
\pgfpathmoveto{\pgfqpoint{0.633033in}{2.301292in}}%
\pgfpathlineto{\pgfqpoint{0.605784in}{2.316056in}}%
\pgfpathlineto{\pgfqpoint{0.605784in}{2.321920in}}%
\pgfpathlineto{\pgfqpoint{0.605784in}{2.342547in}}%
\pgfpathlineto{\pgfqpoint{0.605784in}{2.363175in}}%
\pgfpathlineto{\pgfqpoint{0.605784in}{2.368508in}}%
\pgfpathlineto{\pgfqpoint{0.615680in}{2.363175in}}%
\pgfpathlineto{\pgfqpoint{0.646829in}{2.346383in}}%
\pgfpathlineto{\pgfqpoint{0.653856in}{2.342547in}}%
\pgfpathlineto{\pgfqpoint{0.687875in}{2.323919in}}%
\pgfpathlineto{\pgfqpoint{0.692386in}{2.321920in}}%
\pgfpathlineto{\pgfqpoint{0.728920in}{2.305587in}}%
\pgfpathlineto{\pgfqpoint{0.747375in}{2.301292in}}%
\pgfpathlineto{\pgfqpoint{0.769965in}{2.295946in}}%
\pgfpathlineto{\pgfqpoint{0.811011in}{2.298626in}}%
\pgfpathlineto{\pgfqpoint{0.817493in}{2.301292in}}%
\pgfpathlineto{\pgfqpoint{0.852056in}{2.315456in}}%
\pgfpathlineto{\pgfqpoint{0.860703in}{2.321920in}}%
\pgfpathlineto{\pgfqpoint{0.888213in}{2.342547in}}%
\pgfpathlineto{\pgfqpoint{0.893101in}{2.346215in}}%
\pgfpathlineto{\pgfqpoint{0.909678in}{2.363175in}}%
\pgfpathlineto{\pgfqpoint{0.929627in}{2.383803in}}%
\pgfpathlineto{\pgfqpoint{0.934147in}{2.388480in}}%
\pgfpathlineto{\pgfqpoint{0.947374in}{2.404431in}}%
\pgfpathlineto{\pgfqpoint{0.964309in}{2.425059in}}%
\pgfpathlineto{\pgfqpoint{0.975192in}{2.425059in}}%
\pgfpathlineto{\pgfqpoint{1.005625in}{2.425059in}}%
\pgfpathlineto{\pgfqpoint{0.989830in}{2.404431in}}%
\pgfpathlineto{\pgfqpoint{0.975192in}{2.385577in}}%
\pgfpathlineto{\pgfqpoint{0.973756in}{2.383803in}}%
\pgfpathlineto{\pgfqpoint{0.957084in}{2.363175in}}%
\pgfpathlineto{\pgfqpoint{0.940175in}{2.342547in}}%
\pgfpathlineto{\pgfqpoint{0.934147in}{2.335216in}}%
\pgfpathlineto{\pgfqpoint{0.921419in}{2.321920in}}%
\pgfpathlineto{\pgfqpoint{0.901519in}{2.301292in}}%
\pgfpathlineto{\pgfqpoint{0.893101in}{2.292602in}}%
\pgfpathlineto{\pgfqpoint{0.877309in}{2.280664in}}%
\pgfpathlineto{\pgfqpoint{0.852056in}{2.261694in}}%
\pgfpathlineto{\pgfqpoint{0.848007in}{2.260036in}}%
\pgfpathlineto{\pgfqpoint{0.811011in}{2.244828in}}%
\pgfpathlineto{\pgfqpoint{0.769965in}{2.242396in}}%
\pgfpathlineto{\pgfqpoint{0.728920in}{2.252361in}}%
\pgfpathlineto{\pgfqpoint{0.712158in}{2.260036in}}%
\pgfpathlineto{\pgfqpoint{0.687875in}{2.271051in}}%
\pgfpathlineto{\pgfqpoint{0.670575in}{2.280664in}}%
\pgfpathlineto{\pgfqpoint{0.646829in}{2.293813in}}%
\pgfpathclose%
\pgfusepath{stroke,fill}%
\end{pgfscope}%
\begin{pgfscope}%
\pgfpathrectangle{\pgfqpoint{0.605784in}{0.382904in}}{\pgfqpoint{4.063488in}{2.042155in}}%
\pgfusepath{clip}%
\pgfsetbuttcap%
\pgfsetroundjoin%
\definecolor{currentfill}{rgb}{0.369214,0.788888,0.382914}%
\pgfsetfillcolor{currentfill}%
\pgfsetlinewidth{1.003750pt}%
\definecolor{currentstroke}{rgb}{0.369214,0.788888,0.382914}%
\pgfsetstrokecolor{currentstroke}%
\pgfsetdash{}{0pt}%
\pgfpathmoveto{\pgfqpoint{1.016237in}{0.391852in}}%
\pgfpathlineto{\pgfqpoint{1.027090in}{0.403532in}}%
\pgfpathlineto{\pgfqpoint{1.046535in}{0.424160in}}%
\pgfpathlineto{\pgfqpoint{1.057283in}{0.435399in}}%
\pgfpathlineto{\pgfqpoint{1.065912in}{0.444787in}}%
\pgfpathlineto{\pgfqpoint{1.085065in}{0.465415in}}%
\pgfpathlineto{\pgfqpoint{1.098328in}{0.479483in}}%
\pgfpathlineto{\pgfqpoint{1.104830in}{0.486043in}}%
\pgfpathlineto{\pgfqpoint{1.125402in}{0.506671in}}%
\pgfpathlineto{\pgfqpoint{1.139373in}{0.520491in}}%
\pgfpathlineto{\pgfqpoint{1.147499in}{0.527299in}}%
\pgfpathlineto{\pgfqpoint{1.172272in}{0.547927in}}%
\pgfpathlineto{\pgfqpoint{1.180419in}{0.554668in}}%
\pgfpathlineto{\pgfqpoint{1.203828in}{0.568554in}}%
\pgfpathlineto{\pgfqpoint{1.221464in}{0.578977in}}%
\pgfpathlineto{\pgfqpoint{1.254184in}{0.589182in}}%
\pgfpathlineto{\pgfqpoint{1.262509in}{0.591795in}}%
\pgfpathlineto{\pgfqpoint{1.303555in}{0.593510in}}%
\pgfpathlineto{\pgfqpoint{1.327795in}{0.589182in}}%
\pgfpathlineto{\pgfqpoint{1.344600in}{0.586226in}}%
\pgfpathlineto{\pgfqpoint{1.385645in}{0.573489in}}%
\pgfpathlineto{\pgfqpoint{1.399906in}{0.568554in}}%
\pgfpathlineto{\pgfqpoint{1.426691in}{0.559309in}}%
\pgfpathlineto{\pgfqpoint{1.465688in}{0.547927in}}%
\pgfpathlineto{\pgfqpoint{1.467736in}{0.547329in}}%
\pgfpathlineto{\pgfqpoint{1.508781in}{0.540372in}}%
\pgfpathlineto{\pgfqpoint{1.549827in}{0.539578in}}%
\pgfpathlineto{\pgfqpoint{1.590872in}{0.544847in}}%
\pgfpathlineto{\pgfqpoint{1.603277in}{0.547927in}}%
\pgfpathlineto{\pgfqpoint{1.631917in}{0.555192in}}%
\pgfpathlineto{\pgfqpoint{1.671052in}{0.568554in}}%
\pgfpathlineto{\pgfqpoint{1.672963in}{0.569214in}}%
\pgfpathlineto{\pgfqpoint{1.714008in}{0.586242in}}%
\pgfpathlineto{\pgfqpoint{1.720004in}{0.589182in}}%
\pgfpathlineto{\pgfqpoint{1.755053in}{0.606214in}}%
\pgfpathlineto{\pgfqpoint{1.761235in}{0.609810in}}%
\pgfpathlineto{\pgfqpoint{1.796099in}{0.629740in}}%
\pgfpathlineto{\pgfqpoint{1.797113in}{0.630438in}}%
\pgfpathlineto{\pgfqpoint{1.827459in}{0.651066in}}%
\pgfpathlineto{\pgfqpoint{1.837144in}{0.657502in}}%
\pgfpathlineto{\pgfqpoint{1.855214in}{0.671694in}}%
\pgfpathlineto{\pgfqpoint{1.878189in}{0.689268in}}%
\pgfpathlineto{\pgfqpoint{1.881744in}{0.692321in}}%
\pgfpathlineto{\pgfqpoint{1.906035in}{0.712949in}}%
\pgfpathlineto{\pgfqpoint{1.919235in}{0.723910in}}%
\pgfpathlineto{\pgfqpoint{1.930378in}{0.733577in}}%
\pgfpathlineto{\pgfqpoint{1.954573in}{0.754205in}}%
\pgfpathlineto{\pgfqpoint{1.960280in}{0.759014in}}%
\pgfpathlineto{\pgfqpoint{1.980021in}{0.774833in}}%
\pgfpathlineto{\pgfqpoint{2.001325in}{0.791602in}}%
\pgfpathlineto{\pgfqpoint{2.007146in}{0.795460in}}%
\pgfpathlineto{\pgfqpoint{2.038493in}{0.816088in}}%
\pgfpathlineto{\pgfqpoint{2.042371in}{0.818624in}}%
\pgfpathlineto{\pgfqpoint{2.081102in}{0.836716in}}%
\pgfpathlineto{\pgfqpoint{2.083416in}{0.837796in}}%
\pgfpathlineto{\pgfqpoint{2.124461in}{0.848260in}}%
\pgfpathlineto{\pgfqpoint{2.165507in}{0.850670in}}%
\pgfpathlineto{\pgfqpoint{2.206552in}{0.847340in}}%
\pgfpathlineto{\pgfqpoint{2.247597in}{0.841457in}}%
\pgfpathlineto{\pgfqpoint{2.286353in}{0.836716in}}%
\pgfpathlineto{\pgfqpoint{2.288643in}{0.836437in}}%
\pgfpathlineto{\pgfqpoint{2.329688in}{0.835153in}}%
\pgfpathlineto{\pgfqpoint{2.346184in}{0.836716in}}%
\pgfpathlineto{\pgfqpoint{2.370733in}{0.839136in}}%
\pgfpathlineto{\pgfqpoint{2.411779in}{0.848471in}}%
\pgfpathlineto{\pgfqpoint{2.439418in}{0.857344in}}%
\pgfpathlineto{\pgfqpoint{2.452824in}{0.861774in}}%
\pgfpathlineto{\pgfqpoint{2.493869in}{0.877041in}}%
\pgfpathlineto{\pgfqpoint{2.495879in}{0.857344in}}%
\pgfpathlineto{\pgfqpoint{2.497938in}{0.836716in}}%
\pgfpathlineto{\pgfqpoint{2.499953in}{0.816088in}}%
\pgfpathlineto{\pgfqpoint{2.501927in}{0.795460in}}%
\pgfpathlineto{\pgfqpoint{2.503861in}{0.774833in}}%
\pgfpathlineto{\pgfqpoint{2.505759in}{0.754205in}}%
\pgfpathlineto{\pgfqpoint{2.507623in}{0.733577in}}%
\pgfpathlineto{\pgfqpoint{2.509454in}{0.712949in}}%
\pgfpathlineto{\pgfqpoint{2.511254in}{0.692321in}}%
\pgfpathlineto{\pgfqpoint{2.513026in}{0.671694in}}%
\pgfpathlineto{\pgfqpoint{2.514770in}{0.651066in}}%
\pgfpathlineto{\pgfqpoint{2.516489in}{0.630438in}}%
\pgfpathlineto{\pgfqpoint{2.518183in}{0.609810in}}%
\pgfpathlineto{\pgfqpoint{2.519854in}{0.589182in}}%
\pgfpathlineto{\pgfqpoint{2.521503in}{0.568554in}}%
\pgfpathlineto{\pgfqpoint{2.523131in}{0.547927in}}%
\pgfpathlineto{\pgfqpoint{2.524739in}{0.527299in}}%
\pgfpathlineto{\pgfqpoint{2.526328in}{0.506671in}}%
\pgfpathlineto{\pgfqpoint{2.527899in}{0.486043in}}%
\pgfpathlineto{\pgfqpoint{2.529454in}{0.465415in}}%
\pgfpathlineto{\pgfqpoint{2.530991in}{0.444787in}}%
\pgfpathlineto{\pgfqpoint{2.532514in}{0.424160in}}%
\pgfpathlineto{\pgfqpoint{2.534021in}{0.403532in}}%
\pgfpathlineto{\pgfqpoint{2.534915in}{0.391262in}}%
\pgfpathlineto{\pgfqpoint{2.575960in}{0.391262in}}%
\pgfpathlineto{\pgfqpoint{2.617005in}{0.391262in}}%
\pgfpathlineto{\pgfqpoint{2.658051in}{0.391262in}}%
\pgfpathlineto{\pgfqpoint{2.699096in}{0.391262in}}%
\pgfpathlineto{\pgfqpoint{2.740141in}{0.391262in}}%
\pgfpathlineto{\pgfqpoint{2.781187in}{0.391262in}}%
\pgfpathlineto{\pgfqpoint{2.822232in}{0.391262in}}%
\pgfpathlineto{\pgfqpoint{2.863277in}{0.391262in}}%
\pgfpathlineto{\pgfqpoint{2.904323in}{0.391262in}}%
\pgfpathlineto{\pgfqpoint{2.945368in}{0.391262in}}%
\pgfpathlineto{\pgfqpoint{2.986413in}{0.391262in}}%
\pgfpathlineto{\pgfqpoint{3.027459in}{0.391262in}}%
\pgfpathlineto{\pgfqpoint{3.068504in}{0.391262in}}%
\pgfpathlineto{\pgfqpoint{3.109549in}{0.391262in}}%
\pgfpathlineto{\pgfqpoint{3.150595in}{0.391262in}}%
\pgfpathlineto{\pgfqpoint{3.191640in}{0.391262in}}%
\pgfpathlineto{\pgfqpoint{3.232685in}{0.391262in}}%
\pgfpathlineto{\pgfqpoint{3.273731in}{0.391262in}}%
\pgfpathlineto{\pgfqpoint{3.314776in}{0.391262in}}%
\pgfpathlineto{\pgfqpoint{3.355821in}{0.391262in}}%
\pgfpathlineto{\pgfqpoint{3.396867in}{0.391262in}}%
\pgfpathlineto{\pgfqpoint{3.437912in}{0.391262in}}%
\pgfpathlineto{\pgfqpoint{3.478957in}{0.391262in}}%
\pgfpathlineto{\pgfqpoint{3.520003in}{0.391262in}}%
\pgfpathlineto{\pgfqpoint{3.561048in}{0.391262in}}%
\pgfpathlineto{\pgfqpoint{3.602093in}{0.391262in}}%
\pgfpathlineto{\pgfqpoint{3.643139in}{0.391262in}}%
\pgfpathlineto{\pgfqpoint{3.644886in}{0.382904in}}%
\pgfpathlineto{\pgfqpoint{3.643139in}{0.382904in}}%
\pgfpathlineto{\pgfqpoint{3.602093in}{0.382904in}}%
\pgfpathlineto{\pgfqpoint{3.561048in}{0.382904in}}%
\pgfpathlineto{\pgfqpoint{3.520003in}{0.382904in}}%
\pgfpathlineto{\pgfqpoint{3.478957in}{0.382904in}}%
\pgfpathlineto{\pgfqpoint{3.437912in}{0.382904in}}%
\pgfpathlineto{\pgfqpoint{3.396867in}{0.382904in}}%
\pgfpathlineto{\pgfqpoint{3.355821in}{0.382904in}}%
\pgfpathlineto{\pgfqpoint{3.314776in}{0.382904in}}%
\pgfpathlineto{\pgfqpoint{3.273731in}{0.382904in}}%
\pgfpathlineto{\pgfqpoint{3.232685in}{0.382904in}}%
\pgfpathlineto{\pgfqpoint{3.191640in}{0.382904in}}%
\pgfpathlineto{\pgfqpoint{3.150595in}{0.382904in}}%
\pgfpathlineto{\pgfqpoint{3.109549in}{0.382904in}}%
\pgfpathlineto{\pgfqpoint{3.068504in}{0.382904in}}%
\pgfpathlineto{\pgfqpoint{3.027459in}{0.382904in}}%
\pgfpathlineto{\pgfqpoint{2.986413in}{0.382904in}}%
\pgfpathlineto{\pgfqpoint{2.945368in}{0.382904in}}%
\pgfpathlineto{\pgfqpoint{2.904323in}{0.382904in}}%
\pgfpathlineto{\pgfqpoint{2.863277in}{0.382904in}}%
\pgfpathlineto{\pgfqpoint{2.822232in}{0.382904in}}%
\pgfpathlineto{\pgfqpoint{2.781187in}{0.382904in}}%
\pgfpathlineto{\pgfqpoint{2.740141in}{0.382904in}}%
\pgfpathlineto{\pgfqpoint{2.699096in}{0.382904in}}%
\pgfpathlineto{\pgfqpoint{2.658051in}{0.382904in}}%
\pgfpathlineto{\pgfqpoint{2.617005in}{0.382904in}}%
\pgfpathlineto{\pgfqpoint{2.575960in}{0.382904in}}%
\pgfpathlineto{\pgfqpoint{2.534915in}{0.382904in}}%
\pgfpathlineto{\pgfqpoint{2.531581in}{0.382904in}}%
\pgfpathlineto{\pgfqpoint{2.530028in}{0.403532in}}%
\pgfpathlineto{\pgfqpoint{2.528460in}{0.424160in}}%
\pgfpathlineto{\pgfqpoint{2.526875in}{0.444787in}}%
\pgfpathlineto{\pgfqpoint{2.525273in}{0.465415in}}%
\pgfpathlineto{\pgfqpoint{2.523652in}{0.486043in}}%
\pgfpathlineto{\pgfqpoint{2.522012in}{0.506671in}}%
\pgfpathlineto{\pgfqpoint{2.520351in}{0.527299in}}%
\pgfpathlineto{\pgfqpoint{2.518669in}{0.547927in}}%
\pgfpathlineto{\pgfqpoint{2.516965in}{0.568554in}}%
\pgfpathlineto{\pgfqpoint{2.515238in}{0.589182in}}%
\pgfpathlineto{\pgfqpoint{2.513485in}{0.609810in}}%
\pgfpathlineto{\pgfqpoint{2.511707in}{0.630438in}}%
\pgfpathlineto{\pgfqpoint{2.509901in}{0.651066in}}%
\pgfpathlineto{\pgfqpoint{2.508066in}{0.671694in}}%
\pgfpathlineto{\pgfqpoint{2.506200in}{0.692321in}}%
\pgfpathlineto{\pgfqpoint{2.504301in}{0.712949in}}%
\pgfpathlineto{\pgfqpoint{2.502368in}{0.733577in}}%
\pgfpathlineto{\pgfqpoint{2.500399in}{0.754205in}}%
\pgfpathlineto{\pgfqpoint{2.498391in}{0.774833in}}%
\pgfpathlineto{\pgfqpoint{2.496342in}{0.795460in}}%
\pgfpathlineto{\pgfqpoint{2.494248in}{0.816088in}}%
\pgfpathlineto{\pgfqpoint{2.493869in}{0.819808in}}%
\pgfpathlineto{\pgfqpoint{2.483500in}{0.816088in}}%
\pgfpathlineto{\pgfqpoint{2.452824in}{0.805289in}}%
\pgfpathlineto{\pgfqpoint{2.420008in}{0.795460in}}%
\pgfpathlineto{\pgfqpoint{2.411779in}{0.793062in}}%
\pgfpathlineto{\pgfqpoint{2.370733in}{0.785032in}}%
\pgfpathlineto{\pgfqpoint{2.329688in}{0.782207in}}%
\pgfpathlineto{\pgfqpoint{2.288643in}{0.784442in}}%
\pgfpathlineto{\pgfqpoint{2.247597in}{0.790082in}}%
\pgfpathlineto{\pgfqpoint{2.212207in}{0.795460in}}%
\pgfpathlineto{\pgfqpoint{2.206552in}{0.796325in}}%
\pgfpathlineto{\pgfqpoint{2.165507in}{0.799879in}}%
\pgfpathlineto{\pgfqpoint{2.124461in}{0.797493in}}%
\pgfpathlineto{\pgfqpoint{2.116528in}{0.795460in}}%
\pgfpathlineto{\pgfqpoint{2.083416in}{0.787047in}}%
\pgfpathlineto{\pgfqpoint{2.057438in}{0.774833in}}%
\pgfpathlineto{\pgfqpoint{2.042371in}{0.767740in}}%
\pgfpathlineto{\pgfqpoint{2.021870in}{0.754205in}}%
\pgfpathlineto{\pgfqpoint{2.001325in}{0.740508in}}%
\pgfpathlineto{\pgfqpoint{1.992591in}{0.733577in}}%
\pgfpathlineto{\pgfqpoint{1.966890in}{0.712949in}}%
\pgfpathlineto{\pgfqpoint{1.960280in}{0.707594in}}%
\pgfpathlineto{\pgfqpoint{1.942455in}{0.692321in}}%
\pgfpathlineto{\pgfqpoint{1.919235in}{0.671976in}}%
\pgfpathlineto{\pgfqpoint{1.918900in}{0.671694in}}%
\pgfpathlineto{\pgfqpoint{1.894616in}{0.651066in}}%
\pgfpathlineto{\pgfqpoint{1.878189in}{0.636765in}}%
\pgfpathlineto{\pgfqpoint{1.870077in}{0.630438in}}%
\pgfpathlineto{\pgfqpoint{1.844093in}{0.609810in}}%
\pgfpathlineto{\pgfqpoint{1.837144in}{0.604203in}}%
\pgfpathlineto{\pgfqpoint{1.815322in}{0.589182in}}%
\pgfpathlineto{\pgfqpoint{1.796099in}{0.575653in}}%
\pgfpathlineto{\pgfqpoint{1.783975in}{0.568554in}}%
\pgfpathlineto{\pgfqpoint{1.755053in}{0.551334in}}%
\pgfpathlineto{\pgfqpoint{1.748175in}{0.547927in}}%
\pgfpathlineto{\pgfqpoint{1.714008in}{0.530853in}}%
\pgfpathlineto{\pgfqpoint{1.705501in}{0.527299in}}%
\pgfpathlineto{\pgfqpoint{1.672963in}{0.513714in}}%
\pgfpathlineto{\pgfqpoint{1.652150in}{0.506671in}}%
\pgfpathlineto{\pgfqpoint{1.631917in}{0.499899in}}%
\pgfpathlineto{\pgfqpoint{1.590872in}{0.490142in}}%
\pgfpathlineto{\pgfqpoint{1.554155in}{0.486043in}}%
\pgfpathlineto{\pgfqpoint{1.549827in}{0.485574in}}%
\pgfpathlineto{\pgfqpoint{1.537419in}{0.486043in}}%
\pgfpathlineto{\pgfqpoint{1.508781in}{0.487132in}}%
\pgfpathlineto{\pgfqpoint{1.467736in}{0.494793in}}%
\pgfpathlineto{\pgfqpoint{1.428103in}{0.506671in}}%
\pgfpathlineto{\pgfqpoint{1.426691in}{0.507094in}}%
\pgfpathlineto{\pgfqpoint{1.385645in}{0.521512in}}%
\pgfpathlineto{\pgfqpoint{1.366963in}{0.527299in}}%
\pgfpathlineto{\pgfqpoint{1.344600in}{0.534275in}}%
\pgfpathlineto{\pgfqpoint{1.303555in}{0.541454in}}%
\pgfpathlineto{\pgfqpoint{1.262509in}{0.539644in}}%
\pgfpathlineto{\pgfqpoint{1.223414in}{0.527299in}}%
\pgfpathlineto{\pgfqpoint{1.221464in}{0.526687in}}%
\pgfpathlineto{\pgfqpoint{1.187614in}{0.506671in}}%
\pgfpathlineto{\pgfqpoint{1.180419in}{0.502401in}}%
\pgfpathlineto{\pgfqpoint{1.160641in}{0.486043in}}%
\pgfpathlineto{\pgfqpoint{1.139373in}{0.468239in}}%
\pgfpathlineto{\pgfqpoint{1.136520in}{0.465415in}}%
\pgfpathlineto{\pgfqpoint{1.115735in}{0.444787in}}%
\pgfpathlineto{\pgfqpoint{1.098328in}{0.427229in}}%
\pgfpathlineto{\pgfqpoint{1.095440in}{0.424160in}}%
\pgfpathlineto{\pgfqpoint{1.076111in}{0.403532in}}%
\pgfpathlineto{\pgfqpoint{1.057283in}{0.383023in}}%
\pgfpathlineto{\pgfqpoint{1.057169in}{0.382904in}}%
\pgfpathlineto{\pgfqpoint{1.016237in}{0.382904in}}%
\pgfpathlineto{\pgfqpoint{1.006954in}{0.382904in}}%
\pgfpathclose%
\pgfusepath{stroke,fill}%
\end{pgfscope}%
\begin{pgfscope}%
\pgfpathrectangle{\pgfqpoint{0.605784in}{0.382904in}}{\pgfqpoint{4.063488in}{2.042155in}}%
\pgfusepath{clip}%
\pgfsetbuttcap%
\pgfsetroundjoin%
\definecolor{currentfill}{rgb}{0.369214,0.788888,0.382914}%
\pgfsetfillcolor{currentfill}%
\pgfsetlinewidth{1.003750pt}%
\definecolor{currentstroke}{rgb}{0.369214,0.788888,0.382914}%
\pgfsetstrokecolor{currentstroke}%
\pgfsetdash{}{0pt}%
\pgfpathmoveto{\pgfqpoint{2.532158in}{2.425059in}}%
\pgfpathlineto{\pgfqpoint{2.534915in}{2.425059in}}%
\pgfpathlineto{\pgfqpoint{2.575960in}{2.425059in}}%
\pgfpathlineto{\pgfqpoint{2.617005in}{2.425059in}}%
\pgfpathlineto{\pgfqpoint{2.658051in}{2.425059in}}%
\pgfpathlineto{\pgfqpoint{2.699096in}{2.425059in}}%
\pgfpathlineto{\pgfqpoint{2.740141in}{2.425059in}}%
\pgfpathlineto{\pgfqpoint{2.781187in}{2.425059in}}%
\pgfpathlineto{\pgfqpoint{2.822232in}{2.425059in}}%
\pgfpathlineto{\pgfqpoint{2.863277in}{2.425059in}}%
\pgfpathlineto{\pgfqpoint{2.904323in}{2.425059in}}%
\pgfpathlineto{\pgfqpoint{2.945368in}{2.425059in}}%
\pgfpathlineto{\pgfqpoint{2.986413in}{2.425059in}}%
\pgfpathlineto{\pgfqpoint{3.027459in}{2.425059in}}%
\pgfpathlineto{\pgfqpoint{3.068504in}{2.425059in}}%
\pgfpathlineto{\pgfqpoint{3.109549in}{2.425059in}}%
\pgfpathlineto{\pgfqpoint{3.150595in}{2.425059in}}%
\pgfpathlineto{\pgfqpoint{3.191640in}{2.425059in}}%
\pgfpathlineto{\pgfqpoint{3.232685in}{2.425059in}}%
\pgfpathlineto{\pgfqpoint{3.273731in}{2.425059in}}%
\pgfpathlineto{\pgfqpoint{3.314776in}{2.425059in}}%
\pgfpathlineto{\pgfqpoint{3.355821in}{2.425059in}}%
\pgfpathlineto{\pgfqpoint{3.396867in}{2.425059in}}%
\pgfpathlineto{\pgfqpoint{3.437912in}{2.425059in}}%
\pgfpathlineto{\pgfqpoint{3.478957in}{2.425059in}}%
\pgfpathlineto{\pgfqpoint{3.520003in}{2.425059in}}%
\pgfpathlineto{\pgfqpoint{3.561048in}{2.425059in}}%
\pgfpathlineto{\pgfqpoint{3.602093in}{2.425059in}}%
\pgfpathlineto{\pgfqpoint{3.643139in}{2.425059in}}%
\pgfpathlineto{\pgfqpoint{3.649330in}{2.425059in}}%
\pgfpathlineto{\pgfqpoint{3.652992in}{2.404431in}}%
\pgfpathlineto{\pgfqpoint{3.656719in}{2.383803in}}%
\pgfpathlineto{\pgfqpoint{3.660514in}{2.363175in}}%
\pgfpathlineto{\pgfqpoint{3.664381in}{2.342547in}}%
\pgfpathlineto{\pgfqpoint{3.668326in}{2.321920in}}%
\pgfpathlineto{\pgfqpoint{3.672352in}{2.301292in}}%
\pgfpathlineto{\pgfqpoint{3.676465in}{2.280664in}}%
\pgfpathlineto{\pgfqpoint{3.680670in}{2.260036in}}%
\pgfpathlineto{\pgfqpoint{3.684184in}{2.243126in}}%
\pgfpathlineto{\pgfqpoint{3.725229in}{2.243126in}}%
\pgfpathlineto{\pgfqpoint{3.766275in}{2.243126in}}%
\pgfpathlineto{\pgfqpoint{3.807320in}{2.243126in}}%
\pgfpathlineto{\pgfqpoint{3.848365in}{2.243126in}}%
\pgfpathlineto{\pgfqpoint{3.889411in}{2.243126in}}%
\pgfpathlineto{\pgfqpoint{3.930456in}{2.243126in}}%
\pgfpathlineto{\pgfqpoint{3.971501in}{2.243126in}}%
\pgfpathlineto{\pgfqpoint{4.012547in}{2.243126in}}%
\pgfpathlineto{\pgfqpoint{4.053592in}{2.243126in}}%
\pgfpathlineto{\pgfqpoint{4.094637in}{2.243126in}}%
\pgfpathlineto{\pgfqpoint{4.135683in}{2.243126in}}%
\pgfpathlineto{\pgfqpoint{4.176728in}{2.243126in}}%
\pgfpathlineto{\pgfqpoint{4.217773in}{2.243126in}}%
\pgfpathlineto{\pgfqpoint{4.258819in}{2.243126in}}%
\pgfpathlineto{\pgfqpoint{4.299864in}{2.243126in}}%
\pgfpathlineto{\pgfqpoint{4.340909in}{2.243126in}}%
\pgfpathlineto{\pgfqpoint{4.381955in}{2.243126in}}%
\pgfpathlineto{\pgfqpoint{4.423000in}{2.243126in}}%
\pgfpathlineto{\pgfqpoint{4.464045in}{2.243126in}}%
\pgfpathlineto{\pgfqpoint{4.505091in}{2.243126in}}%
\pgfpathlineto{\pgfqpoint{4.546136in}{2.243126in}}%
\pgfpathlineto{\pgfqpoint{4.587181in}{2.243126in}}%
\pgfpathlineto{\pgfqpoint{4.628227in}{2.243126in}}%
\pgfpathlineto{\pgfqpoint{4.669272in}{2.243126in}}%
\pgfpathlineto{\pgfqpoint{4.669272in}{2.239408in}}%
\pgfpathlineto{\pgfqpoint{4.669272in}{2.218780in}}%
\pgfpathlineto{\pgfqpoint{4.669272in}{2.198153in}}%
\pgfpathlineto{\pgfqpoint{4.669272in}{2.189794in}}%
\pgfpathlineto{\pgfqpoint{4.628227in}{2.189794in}}%
\pgfpathlineto{\pgfqpoint{4.587181in}{2.189794in}}%
\pgfpathlineto{\pgfqpoint{4.546136in}{2.189794in}}%
\pgfpathlineto{\pgfqpoint{4.505091in}{2.189794in}}%
\pgfpathlineto{\pgfqpoint{4.464045in}{2.189794in}}%
\pgfpathlineto{\pgfqpoint{4.423000in}{2.189794in}}%
\pgfpathlineto{\pgfqpoint{4.381955in}{2.189794in}}%
\pgfpathlineto{\pgfqpoint{4.340909in}{2.189794in}}%
\pgfpathlineto{\pgfqpoint{4.299864in}{2.189794in}}%
\pgfpathlineto{\pgfqpoint{4.258819in}{2.189794in}}%
\pgfpathlineto{\pgfqpoint{4.217773in}{2.189794in}}%
\pgfpathlineto{\pgfqpoint{4.176728in}{2.189794in}}%
\pgfpathlineto{\pgfqpoint{4.135683in}{2.189794in}}%
\pgfpathlineto{\pgfqpoint{4.094637in}{2.189794in}}%
\pgfpathlineto{\pgfqpoint{4.053592in}{2.189794in}}%
\pgfpathlineto{\pgfqpoint{4.012547in}{2.189794in}}%
\pgfpathlineto{\pgfqpoint{3.971501in}{2.189794in}}%
\pgfpathlineto{\pgfqpoint{3.930456in}{2.189794in}}%
\pgfpathlineto{\pgfqpoint{3.889411in}{2.189794in}}%
\pgfpathlineto{\pgfqpoint{3.848365in}{2.189794in}}%
\pgfpathlineto{\pgfqpoint{3.807320in}{2.189794in}}%
\pgfpathlineto{\pgfqpoint{3.766275in}{2.189794in}}%
\pgfpathlineto{\pgfqpoint{3.725229in}{2.189794in}}%
\pgfpathlineto{\pgfqpoint{3.684184in}{2.189794in}}%
\pgfpathlineto{\pgfqpoint{3.682436in}{2.198153in}}%
\pgfpathlineto{\pgfqpoint{3.678171in}{2.218780in}}%
\pgfpathlineto{\pgfqpoint{3.674007in}{2.239408in}}%
\pgfpathlineto{\pgfqpoint{3.669937in}{2.260036in}}%
\pgfpathlineto{\pgfqpoint{3.665955in}{2.280664in}}%
\pgfpathlineto{\pgfqpoint{3.662057in}{2.301292in}}%
\pgfpathlineto{\pgfqpoint{3.658237in}{2.321920in}}%
\pgfpathlineto{\pgfqpoint{3.654490in}{2.342547in}}%
\pgfpathlineto{\pgfqpoint{3.650812in}{2.363175in}}%
\pgfpathlineto{\pgfqpoint{3.647200in}{2.383803in}}%
\pgfpathlineto{\pgfqpoint{3.643650in}{2.404431in}}%
\pgfpathlineto{\pgfqpoint{3.643139in}{2.407405in}}%
\pgfpathlineto{\pgfqpoint{3.602093in}{2.407405in}}%
\pgfpathlineto{\pgfqpoint{3.561048in}{2.407405in}}%
\pgfpathlineto{\pgfqpoint{3.520003in}{2.407405in}}%
\pgfpathlineto{\pgfqpoint{3.478957in}{2.407405in}}%
\pgfpathlineto{\pgfqpoint{3.437912in}{2.407405in}}%
\pgfpathlineto{\pgfqpoint{3.396867in}{2.407405in}}%
\pgfpathlineto{\pgfqpoint{3.355821in}{2.407405in}}%
\pgfpathlineto{\pgfqpoint{3.314776in}{2.407405in}}%
\pgfpathlineto{\pgfqpoint{3.273731in}{2.407405in}}%
\pgfpathlineto{\pgfqpoint{3.232685in}{2.407405in}}%
\pgfpathlineto{\pgfqpoint{3.191640in}{2.407405in}}%
\pgfpathlineto{\pgfqpoint{3.150595in}{2.407405in}}%
\pgfpathlineto{\pgfqpoint{3.109549in}{2.407405in}}%
\pgfpathlineto{\pgfqpoint{3.068504in}{2.407405in}}%
\pgfpathlineto{\pgfqpoint{3.027459in}{2.407405in}}%
\pgfpathlineto{\pgfqpoint{2.986413in}{2.407405in}}%
\pgfpathlineto{\pgfqpoint{2.945368in}{2.407405in}}%
\pgfpathlineto{\pgfqpoint{2.904323in}{2.407405in}}%
\pgfpathlineto{\pgfqpoint{2.863277in}{2.407405in}}%
\pgfpathlineto{\pgfqpoint{2.822232in}{2.407405in}}%
\pgfpathlineto{\pgfqpoint{2.781187in}{2.407405in}}%
\pgfpathlineto{\pgfqpoint{2.740141in}{2.407405in}}%
\pgfpathlineto{\pgfqpoint{2.699096in}{2.407405in}}%
\pgfpathlineto{\pgfqpoint{2.658051in}{2.407405in}}%
\pgfpathlineto{\pgfqpoint{2.617005in}{2.407405in}}%
\pgfpathlineto{\pgfqpoint{2.575960in}{2.407405in}}%
\pgfpathlineto{\pgfqpoint{2.534915in}{2.407405in}}%
\pgfpathclose%
\pgfusepath{stroke,fill}%
\end{pgfscope}%
\begin{pgfscope}%
\pgfpathrectangle{\pgfqpoint{0.605784in}{0.382904in}}{\pgfqpoint{4.063488in}{2.042155in}}%
\pgfusepath{clip}%
\pgfsetbuttcap%
\pgfsetroundjoin%
\definecolor{currentfill}{rgb}{0.535621,0.835785,0.281908}%
\pgfsetfillcolor{currentfill}%
\pgfsetlinewidth{1.003750pt}%
\definecolor{currentstroke}{rgb}{0.535621,0.835785,0.281908}%
\pgfsetstrokecolor{currentstroke}%
\pgfsetdash{}{0pt}%
\pgfpathmoveto{\pgfqpoint{0.615680in}{2.363175in}}%
\pgfpathlineto{\pgfqpoint{0.605784in}{2.368508in}}%
\pgfpathlineto{\pgfqpoint{0.605784in}{2.383803in}}%
\pgfpathlineto{\pgfqpoint{0.605784in}{2.404431in}}%
\pgfpathlineto{\pgfqpoint{0.605784in}{2.418516in}}%
\pgfpathlineto{\pgfqpoint{0.632012in}{2.404431in}}%
\pgfpathlineto{\pgfqpoint{0.646829in}{2.396470in}}%
\pgfpathlineto{\pgfqpoint{0.670244in}{2.383803in}}%
\pgfpathlineto{\pgfqpoint{0.687875in}{2.374235in}}%
\pgfpathlineto{\pgfqpoint{0.713194in}{2.363175in}}%
\pgfpathlineto{\pgfqpoint{0.728920in}{2.356247in}}%
\pgfpathlineto{\pgfqpoint{0.769965in}{2.346928in}}%
\pgfpathlineto{\pgfqpoint{0.811011in}{2.349767in}}%
\pgfpathlineto{\pgfqpoint{0.843601in}{2.363175in}}%
\pgfpathlineto{\pgfqpoint{0.852056in}{2.366643in}}%
\pgfpathlineto{\pgfqpoint{0.875164in}{2.383803in}}%
\pgfpathlineto{\pgfqpoint{0.893101in}{2.397201in}}%
\pgfpathlineto{\pgfqpoint{0.900207in}{2.404431in}}%
\pgfpathlineto{\pgfqpoint{0.920420in}{2.425059in}}%
\pgfpathlineto{\pgfqpoint{0.934147in}{2.425059in}}%
\pgfpathlineto{\pgfqpoint{0.964309in}{2.425059in}}%
\pgfpathlineto{\pgfqpoint{0.947374in}{2.404431in}}%
\pgfpathlineto{\pgfqpoint{0.934147in}{2.388480in}}%
\pgfpathlineto{\pgfqpoint{0.929627in}{2.383803in}}%
\pgfpathlineto{\pgfqpoint{0.909678in}{2.363175in}}%
\pgfpathlineto{\pgfqpoint{0.893101in}{2.346215in}}%
\pgfpathlineto{\pgfqpoint{0.888213in}{2.342547in}}%
\pgfpathlineto{\pgfqpoint{0.860703in}{2.321920in}}%
\pgfpathlineto{\pgfqpoint{0.852056in}{2.315456in}}%
\pgfpathlineto{\pgfqpoint{0.817493in}{2.301292in}}%
\pgfpathlineto{\pgfqpoint{0.811011in}{2.298626in}}%
\pgfpathlineto{\pgfqpoint{0.769965in}{2.295946in}}%
\pgfpathlineto{\pgfqpoint{0.747375in}{2.301292in}}%
\pgfpathlineto{\pgfqpoint{0.728920in}{2.305587in}}%
\pgfpathlineto{\pgfqpoint{0.692386in}{2.321920in}}%
\pgfpathlineto{\pgfqpoint{0.687875in}{2.323919in}}%
\pgfpathlineto{\pgfqpoint{0.653856in}{2.342547in}}%
\pgfpathlineto{\pgfqpoint{0.646829in}{2.346383in}}%
\pgfpathclose%
\pgfusepath{stroke,fill}%
\end{pgfscope}%
\begin{pgfscope}%
\pgfpathrectangle{\pgfqpoint{0.605784in}{0.382904in}}{\pgfqpoint{4.063488in}{2.042155in}}%
\pgfusepath{clip}%
\pgfsetbuttcap%
\pgfsetroundjoin%
\definecolor{currentfill}{rgb}{0.535621,0.835785,0.281908}%
\pgfsetfillcolor{currentfill}%
\pgfsetlinewidth{1.003750pt}%
\definecolor{currentstroke}{rgb}{0.535621,0.835785,0.281908}%
\pgfsetstrokecolor{currentstroke}%
\pgfsetdash{}{0pt}%
\pgfpathmoveto{\pgfqpoint{1.057283in}{0.383023in}}%
\pgfpathlineto{\pgfqpoint{1.076111in}{0.403532in}}%
\pgfpathlineto{\pgfqpoint{1.095440in}{0.424160in}}%
\pgfpathlineto{\pgfqpoint{1.098328in}{0.427229in}}%
\pgfpathlineto{\pgfqpoint{1.115735in}{0.444787in}}%
\pgfpathlineto{\pgfqpoint{1.136520in}{0.465415in}}%
\pgfpathlineto{\pgfqpoint{1.139373in}{0.468239in}}%
\pgfpathlineto{\pgfqpoint{1.160641in}{0.486043in}}%
\pgfpathlineto{\pgfqpoint{1.180419in}{0.502401in}}%
\pgfpathlineto{\pgfqpoint{1.187614in}{0.506671in}}%
\pgfpathlineto{\pgfqpoint{1.221464in}{0.526687in}}%
\pgfpathlineto{\pgfqpoint{1.223414in}{0.527299in}}%
\pgfpathlineto{\pgfqpoint{1.262509in}{0.539644in}}%
\pgfpathlineto{\pgfqpoint{1.303555in}{0.541454in}}%
\pgfpathlineto{\pgfqpoint{1.344600in}{0.534275in}}%
\pgfpathlineto{\pgfqpoint{1.366963in}{0.527299in}}%
\pgfpathlineto{\pgfqpoint{1.385645in}{0.521512in}}%
\pgfpathlineto{\pgfqpoint{1.426691in}{0.507094in}}%
\pgfpathlineto{\pgfqpoint{1.428103in}{0.506671in}}%
\pgfpathlineto{\pgfqpoint{1.467736in}{0.494793in}}%
\pgfpathlineto{\pgfqpoint{1.508781in}{0.487132in}}%
\pgfpathlineto{\pgfqpoint{1.537419in}{0.486043in}}%
\pgfpathlineto{\pgfqpoint{1.549827in}{0.485574in}}%
\pgfpathlineto{\pgfqpoint{1.554155in}{0.486043in}}%
\pgfpathlineto{\pgfqpoint{1.590872in}{0.490142in}}%
\pgfpathlineto{\pgfqpoint{1.631917in}{0.499899in}}%
\pgfpathlineto{\pgfqpoint{1.652150in}{0.506671in}}%
\pgfpathlineto{\pgfqpoint{1.672963in}{0.513714in}}%
\pgfpathlineto{\pgfqpoint{1.705501in}{0.527299in}}%
\pgfpathlineto{\pgfqpoint{1.714008in}{0.530853in}}%
\pgfpathlineto{\pgfqpoint{1.748175in}{0.547927in}}%
\pgfpathlineto{\pgfqpoint{1.755053in}{0.551334in}}%
\pgfpathlineto{\pgfqpoint{1.783975in}{0.568554in}}%
\pgfpathlineto{\pgfqpoint{1.796099in}{0.575653in}}%
\pgfpathlineto{\pgfqpoint{1.815322in}{0.589182in}}%
\pgfpathlineto{\pgfqpoint{1.837144in}{0.604203in}}%
\pgfpathlineto{\pgfqpoint{1.844093in}{0.609810in}}%
\pgfpathlineto{\pgfqpoint{1.870077in}{0.630438in}}%
\pgfpathlineto{\pgfqpoint{1.878189in}{0.636765in}}%
\pgfpathlineto{\pgfqpoint{1.894616in}{0.651066in}}%
\pgfpathlineto{\pgfqpoint{1.918900in}{0.671694in}}%
\pgfpathlineto{\pgfqpoint{1.919235in}{0.671976in}}%
\pgfpathlineto{\pgfqpoint{1.942455in}{0.692321in}}%
\pgfpathlineto{\pgfqpoint{1.960280in}{0.707594in}}%
\pgfpathlineto{\pgfqpoint{1.966890in}{0.712949in}}%
\pgfpathlineto{\pgfqpoint{1.992591in}{0.733577in}}%
\pgfpathlineto{\pgfqpoint{2.001325in}{0.740508in}}%
\pgfpathlineto{\pgfqpoint{2.021870in}{0.754205in}}%
\pgfpathlineto{\pgfqpoint{2.042371in}{0.767740in}}%
\pgfpathlineto{\pgfqpoint{2.057438in}{0.774833in}}%
\pgfpathlineto{\pgfqpoint{2.083416in}{0.787047in}}%
\pgfpathlineto{\pgfqpoint{2.116528in}{0.795460in}}%
\pgfpathlineto{\pgfqpoint{2.124461in}{0.797493in}}%
\pgfpathlineto{\pgfqpoint{2.165507in}{0.799879in}}%
\pgfpathlineto{\pgfqpoint{2.206552in}{0.796325in}}%
\pgfpathlineto{\pgfqpoint{2.212207in}{0.795460in}}%
\pgfpathlineto{\pgfqpoint{2.247597in}{0.790082in}}%
\pgfpathlineto{\pgfqpoint{2.288643in}{0.784442in}}%
\pgfpathlineto{\pgfqpoint{2.329688in}{0.782207in}}%
\pgfpathlineto{\pgfqpoint{2.370733in}{0.785032in}}%
\pgfpathlineto{\pgfqpoint{2.411779in}{0.793062in}}%
\pgfpathlineto{\pgfqpoint{2.420008in}{0.795460in}}%
\pgfpathlineto{\pgfqpoint{2.452824in}{0.805289in}}%
\pgfpathlineto{\pgfqpoint{2.483500in}{0.816088in}}%
\pgfpathlineto{\pgfqpoint{2.493869in}{0.819808in}}%
\pgfpathlineto{\pgfqpoint{2.494248in}{0.816088in}}%
\pgfpathlineto{\pgfqpoint{2.496342in}{0.795460in}}%
\pgfpathlineto{\pgfqpoint{2.498391in}{0.774833in}}%
\pgfpathlineto{\pgfqpoint{2.500399in}{0.754205in}}%
\pgfpathlineto{\pgfqpoint{2.502368in}{0.733577in}}%
\pgfpathlineto{\pgfqpoint{2.504301in}{0.712949in}}%
\pgfpathlineto{\pgfqpoint{2.506200in}{0.692321in}}%
\pgfpathlineto{\pgfqpoint{2.508066in}{0.671694in}}%
\pgfpathlineto{\pgfqpoint{2.509901in}{0.651066in}}%
\pgfpathlineto{\pgfqpoint{2.511707in}{0.630438in}}%
\pgfpathlineto{\pgfqpoint{2.513485in}{0.609810in}}%
\pgfpathlineto{\pgfqpoint{2.515238in}{0.589182in}}%
\pgfpathlineto{\pgfqpoint{2.516965in}{0.568554in}}%
\pgfpathlineto{\pgfqpoint{2.518669in}{0.547927in}}%
\pgfpathlineto{\pgfqpoint{2.520351in}{0.527299in}}%
\pgfpathlineto{\pgfqpoint{2.522012in}{0.506671in}}%
\pgfpathlineto{\pgfqpoint{2.523652in}{0.486043in}}%
\pgfpathlineto{\pgfqpoint{2.525273in}{0.465415in}}%
\pgfpathlineto{\pgfqpoint{2.526875in}{0.444787in}}%
\pgfpathlineto{\pgfqpoint{2.528460in}{0.424160in}}%
\pgfpathlineto{\pgfqpoint{2.530028in}{0.403532in}}%
\pgfpathlineto{\pgfqpoint{2.531581in}{0.382904in}}%
\pgfpathlineto{\pgfqpoint{2.527647in}{0.382904in}}%
\pgfpathlineto{\pgfqpoint{2.526036in}{0.403532in}}%
\pgfpathlineto{\pgfqpoint{2.524407in}{0.424160in}}%
\pgfpathlineto{\pgfqpoint{2.522759in}{0.444787in}}%
\pgfpathlineto{\pgfqpoint{2.521092in}{0.465415in}}%
\pgfpathlineto{\pgfqpoint{2.519404in}{0.486043in}}%
\pgfpathlineto{\pgfqpoint{2.517695in}{0.506671in}}%
\pgfpathlineto{\pgfqpoint{2.515963in}{0.527299in}}%
\pgfpathlineto{\pgfqpoint{2.514208in}{0.547927in}}%
\pgfpathlineto{\pgfqpoint{2.512428in}{0.568554in}}%
\pgfpathlineto{\pgfqpoint{2.510622in}{0.589182in}}%
\pgfpathlineto{\pgfqpoint{2.508788in}{0.609810in}}%
\pgfpathlineto{\pgfqpoint{2.506925in}{0.630438in}}%
\pgfpathlineto{\pgfqpoint{2.505031in}{0.651066in}}%
\pgfpathlineto{\pgfqpoint{2.503105in}{0.671694in}}%
\pgfpathlineto{\pgfqpoint{2.501145in}{0.692321in}}%
\pgfpathlineto{\pgfqpoint{2.499149in}{0.712949in}}%
\pgfpathlineto{\pgfqpoint{2.497114in}{0.733577in}}%
\pgfpathlineto{\pgfqpoint{2.495039in}{0.754205in}}%
\pgfpathlineto{\pgfqpoint{2.493869in}{0.765699in}}%
\pgfpathlineto{\pgfqpoint{2.460104in}{0.754205in}}%
\pgfpathlineto{\pgfqpoint{2.452824in}{0.751770in}}%
\pgfpathlineto{\pgfqpoint{2.411779in}{0.740553in}}%
\pgfpathlineto{\pgfqpoint{2.371105in}{0.733577in}}%
\pgfpathlineto{\pgfqpoint{2.370733in}{0.733515in}}%
\pgfpathlineto{\pgfqpoint{2.329688in}{0.731707in}}%
\pgfpathlineto{\pgfqpoint{2.304413in}{0.733577in}}%
\pgfpathlineto{\pgfqpoint{2.288643in}{0.734744in}}%
\pgfpathlineto{\pgfqpoint{2.247597in}{0.741007in}}%
\pgfpathlineto{\pgfqpoint{2.206552in}{0.747629in}}%
\pgfpathlineto{\pgfqpoint{2.165507in}{0.751310in}}%
\pgfpathlineto{\pgfqpoint{2.124461in}{0.748991in}}%
\pgfpathlineto{\pgfqpoint{2.083416in}{0.738470in}}%
\pgfpathlineto{\pgfqpoint{2.073057in}{0.733577in}}%
\pgfpathlineto{\pgfqpoint{2.042371in}{0.719066in}}%
\pgfpathlineto{\pgfqpoint{2.033159in}{0.712949in}}%
\pgfpathlineto{\pgfqpoint{2.002360in}{0.692321in}}%
\pgfpathlineto{\pgfqpoint{2.001325in}{0.691626in}}%
\pgfpathlineto{\pgfqpoint{1.976483in}{0.671694in}}%
\pgfpathlineto{\pgfqpoint{1.960280in}{0.658480in}}%
\pgfpathlineto{\pgfqpoint{1.951712in}{0.651066in}}%
\pgfpathlineto{\pgfqpoint{1.928180in}{0.630438in}}%
\pgfpathlineto{\pgfqpoint{1.919235in}{0.622493in}}%
\pgfpathlineto{\pgfqpoint{1.904495in}{0.609810in}}%
\pgfpathlineto{\pgfqpoint{1.881024in}{0.589182in}}%
\pgfpathlineto{\pgfqpoint{1.878189in}{0.586669in}}%
\pgfpathlineto{\pgfqpoint{1.855559in}{0.568554in}}%
\pgfpathlineto{\pgfqpoint{1.837144in}{0.553458in}}%
\pgfpathlineto{\pgfqpoint{1.829272in}{0.547927in}}%
\pgfpathlineto{\pgfqpoint{1.800414in}{0.527299in}}%
\pgfpathlineto{\pgfqpoint{1.796099in}{0.524173in}}%
\pgfpathlineto{\pgfqpoint{1.767145in}{0.506671in}}%
\pgfpathlineto{\pgfqpoint{1.755053in}{0.499246in}}%
\pgfpathlineto{\pgfqpoint{1.729070in}{0.486043in}}%
\pgfpathlineto{\pgfqpoint{1.714008in}{0.478326in}}%
\pgfpathlineto{\pgfqpoint{1.683377in}{0.465415in}}%
\pgfpathlineto{\pgfqpoint{1.672963in}{0.461029in}}%
\pgfpathlineto{\pgfqpoint{1.631917in}{0.447418in}}%
\pgfpathlineto{\pgfqpoint{1.620380in}{0.444787in}}%
\pgfpathlineto{\pgfqpoint{1.590872in}{0.438190in}}%
\pgfpathlineto{\pgfqpoint{1.549827in}{0.434302in}}%
\pgfpathlineto{\pgfqpoint{1.508781in}{0.436504in}}%
\pgfpathlineto{\pgfqpoint{1.467736in}{0.444620in}}%
\pgfpathlineto{\pgfqpoint{1.467198in}{0.444787in}}%
\pgfpathlineto{\pgfqpoint{1.426691in}{0.457366in}}%
\pgfpathlineto{\pgfqpoint{1.403949in}{0.465415in}}%
\pgfpathlineto{\pgfqpoint{1.385645in}{0.471908in}}%
\pgfpathlineto{\pgfqpoint{1.344600in}{0.484646in}}%
\pgfpathlineto{\pgfqpoint{1.336465in}{0.486043in}}%
\pgfpathlineto{\pgfqpoint{1.303555in}{0.491772in}}%
\pgfpathlineto{\pgfqpoint{1.262509in}{0.489849in}}%
\pgfpathlineto{\pgfqpoint{1.250500in}{0.486043in}}%
\pgfpathlineto{\pgfqpoint{1.221464in}{0.476892in}}%
\pgfpathlineto{\pgfqpoint{1.202061in}{0.465415in}}%
\pgfpathlineto{\pgfqpoint{1.180419in}{0.452571in}}%
\pgfpathlineto{\pgfqpoint{1.171006in}{0.444787in}}%
\pgfpathlineto{\pgfqpoint{1.146224in}{0.424160in}}%
\pgfpathlineto{\pgfqpoint{1.139373in}{0.418431in}}%
\pgfpathlineto{\pgfqpoint{1.124327in}{0.403532in}}%
\pgfpathlineto{\pgfqpoint{1.103772in}{0.382904in}}%
\pgfpathlineto{\pgfqpoint{1.098328in}{0.382904in}}%
\pgfpathlineto{\pgfqpoint{1.057283in}{0.382904in}}%
\pgfpathlineto{\pgfqpoint{1.057169in}{0.382904in}}%
\pgfpathclose%
\pgfusepath{stroke,fill}%
\end{pgfscope}%
\begin{pgfscope}%
\pgfpathrectangle{\pgfqpoint{0.605784in}{0.382904in}}{\pgfqpoint{4.063488in}{2.042155in}}%
\pgfusepath{clip}%
\pgfsetbuttcap%
\pgfsetroundjoin%
\definecolor{currentfill}{rgb}{0.535621,0.835785,0.281908}%
\pgfsetfillcolor{currentfill}%
\pgfsetlinewidth{1.003750pt}%
\definecolor{currentstroke}{rgb}{0.535621,0.835785,0.281908}%
\pgfsetstrokecolor{currentstroke}%
\pgfsetdash{}{0pt}%
\pgfpathmoveto{\pgfqpoint{3.680670in}{2.260036in}}%
\pgfpathlineto{\pgfqpoint{3.676465in}{2.280664in}}%
\pgfpathlineto{\pgfqpoint{3.672352in}{2.301292in}}%
\pgfpathlineto{\pgfqpoint{3.668326in}{2.321920in}}%
\pgfpathlineto{\pgfqpoint{3.664381in}{2.342547in}}%
\pgfpathlineto{\pgfqpoint{3.660514in}{2.363175in}}%
\pgfpathlineto{\pgfqpoint{3.656719in}{2.383803in}}%
\pgfpathlineto{\pgfqpoint{3.652992in}{2.404431in}}%
\pgfpathlineto{\pgfqpoint{3.649330in}{2.425059in}}%
\pgfpathlineto{\pgfqpoint{3.658502in}{2.425059in}}%
\pgfpathlineto{\pgfqpoint{3.662334in}{2.404431in}}%
\pgfpathlineto{\pgfqpoint{3.666237in}{2.383803in}}%
\pgfpathlineto{\pgfqpoint{3.670215in}{2.363175in}}%
\pgfpathlineto{\pgfqpoint{3.674273in}{2.342547in}}%
\pgfpathlineto{\pgfqpoint{3.678415in}{2.321920in}}%
\pgfpathlineto{\pgfqpoint{3.682647in}{2.301292in}}%
\pgfpathlineto{\pgfqpoint{3.684184in}{2.293870in}}%
\pgfpathlineto{\pgfqpoint{3.725229in}{2.293870in}}%
\pgfpathlineto{\pgfqpoint{3.766275in}{2.293870in}}%
\pgfpathlineto{\pgfqpoint{3.807320in}{2.293870in}}%
\pgfpathlineto{\pgfqpoint{3.848365in}{2.293870in}}%
\pgfpathlineto{\pgfqpoint{3.889411in}{2.293870in}}%
\pgfpathlineto{\pgfqpoint{3.930456in}{2.293870in}}%
\pgfpathlineto{\pgfqpoint{3.971501in}{2.293870in}}%
\pgfpathlineto{\pgfqpoint{4.012547in}{2.293870in}}%
\pgfpathlineto{\pgfqpoint{4.053592in}{2.293870in}}%
\pgfpathlineto{\pgfqpoint{4.094637in}{2.293870in}}%
\pgfpathlineto{\pgfqpoint{4.135683in}{2.293870in}}%
\pgfpathlineto{\pgfqpoint{4.176728in}{2.293870in}}%
\pgfpathlineto{\pgfqpoint{4.217773in}{2.293870in}}%
\pgfpathlineto{\pgfqpoint{4.258819in}{2.293870in}}%
\pgfpathlineto{\pgfqpoint{4.299864in}{2.293870in}}%
\pgfpathlineto{\pgfqpoint{4.340909in}{2.293870in}}%
\pgfpathlineto{\pgfqpoint{4.381955in}{2.293870in}}%
\pgfpathlineto{\pgfqpoint{4.423000in}{2.293870in}}%
\pgfpathlineto{\pgfqpoint{4.464045in}{2.293870in}}%
\pgfpathlineto{\pgfqpoint{4.505091in}{2.293870in}}%
\pgfpathlineto{\pgfqpoint{4.546136in}{2.293870in}}%
\pgfpathlineto{\pgfqpoint{4.587181in}{2.293870in}}%
\pgfpathlineto{\pgfqpoint{4.628227in}{2.293870in}}%
\pgfpathlineto{\pgfqpoint{4.669272in}{2.293870in}}%
\pgfpathlineto{\pgfqpoint{4.669272in}{2.280664in}}%
\pgfpathlineto{\pgfqpoint{4.669272in}{2.260036in}}%
\pgfpathlineto{\pgfqpoint{4.669272in}{2.243126in}}%
\pgfpathlineto{\pgfqpoint{4.628227in}{2.243126in}}%
\pgfpathlineto{\pgfqpoint{4.587181in}{2.243126in}}%
\pgfpathlineto{\pgfqpoint{4.546136in}{2.243126in}}%
\pgfpathlineto{\pgfqpoint{4.505091in}{2.243126in}}%
\pgfpathlineto{\pgfqpoint{4.464045in}{2.243126in}}%
\pgfpathlineto{\pgfqpoint{4.423000in}{2.243126in}}%
\pgfpathlineto{\pgfqpoint{4.381955in}{2.243126in}}%
\pgfpathlineto{\pgfqpoint{4.340909in}{2.243126in}}%
\pgfpathlineto{\pgfqpoint{4.299864in}{2.243126in}}%
\pgfpathlineto{\pgfqpoint{4.258819in}{2.243126in}}%
\pgfpathlineto{\pgfqpoint{4.217773in}{2.243126in}}%
\pgfpathlineto{\pgfqpoint{4.176728in}{2.243126in}}%
\pgfpathlineto{\pgfqpoint{4.135683in}{2.243126in}}%
\pgfpathlineto{\pgfqpoint{4.094637in}{2.243126in}}%
\pgfpathlineto{\pgfqpoint{4.053592in}{2.243126in}}%
\pgfpathlineto{\pgfqpoint{4.012547in}{2.243126in}}%
\pgfpathlineto{\pgfqpoint{3.971501in}{2.243126in}}%
\pgfpathlineto{\pgfqpoint{3.930456in}{2.243126in}}%
\pgfpathlineto{\pgfqpoint{3.889411in}{2.243126in}}%
\pgfpathlineto{\pgfqpoint{3.848365in}{2.243126in}}%
\pgfpathlineto{\pgfqpoint{3.807320in}{2.243126in}}%
\pgfpathlineto{\pgfqpoint{3.766275in}{2.243126in}}%
\pgfpathlineto{\pgfqpoint{3.725229in}{2.243126in}}%
\pgfpathlineto{\pgfqpoint{3.684184in}{2.243126in}}%
\pgfpathclose%
\pgfusepath{stroke,fill}%
\end{pgfscope}%
\begin{pgfscope}%
\pgfpathrectangle{\pgfqpoint{0.605784in}{0.382904in}}{\pgfqpoint{4.063488in}{2.042155in}}%
\pgfusepath{clip}%
\pgfsetbuttcap%
\pgfsetroundjoin%
\definecolor{currentfill}{rgb}{0.720391,0.870350,0.162603}%
\pgfsetfillcolor{currentfill}%
\pgfsetlinewidth{1.003750pt}%
\definecolor{currentstroke}{rgb}{0.720391,0.870350,0.162603}%
\pgfsetstrokecolor{currentstroke}%
\pgfsetdash{}{0pt}%
\pgfpathmoveto{\pgfqpoint{0.632012in}{2.404431in}}%
\pgfpathlineto{\pgfqpoint{0.605784in}{2.418516in}}%
\pgfpathlineto{\pgfqpoint{0.605784in}{2.425059in}}%
\pgfpathlineto{\pgfqpoint{0.646829in}{2.425059in}}%
\pgfpathlineto{\pgfqpoint{0.683011in}{2.425059in}}%
\pgfpathlineto{\pgfqpoint{0.687875in}{2.422441in}}%
\pgfpathlineto{\pgfqpoint{0.728920in}{2.404751in}}%
\pgfpathlineto{\pgfqpoint{0.730386in}{2.404431in}}%
\pgfpathlineto{\pgfqpoint{0.769965in}{2.395646in}}%
\pgfpathlineto{\pgfqpoint{0.811011in}{2.398667in}}%
\pgfpathlineto{\pgfqpoint{0.825015in}{2.404431in}}%
\pgfpathlineto{\pgfqpoint{0.852056in}{2.415524in}}%
\pgfpathlineto{\pgfqpoint{0.864946in}{2.425059in}}%
\pgfpathlineto{\pgfqpoint{0.893101in}{2.425059in}}%
\pgfpathlineto{\pgfqpoint{0.920420in}{2.425059in}}%
\pgfpathlineto{\pgfqpoint{0.900207in}{2.404431in}}%
\pgfpathlineto{\pgfqpoint{0.893101in}{2.397201in}}%
\pgfpathlineto{\pgfqpoint{0.875164in}{2.383803in}}%
\pgfpathlineto{\pgfqpoint{0.852056in}{2.366643in}}%
\pgfpathlineto{\pgfqpoint{0.843601in}{2.363175in}}%
\pgfpathlineto{\pgfqpoint{0.811011in}{2.349767in}}%
\pgfpathlineto{\pgfqpoint{0.769965in}{2.346928in}}%
\pgfpathlineto{\pgfqpoint{0.728920in}{2.356247in}}%
\pgfpathlineto{\pgfqpoint{0.713194in}{2.363175in}}%
\pgfpathlineto{\pgfqpoint{0.687875in}{2.374235in}}%
\pgfpathlineto{\pgfqpoint{0.670244in}{2.383803in}}%
\pgfpathlineto{\pgfqpoint{0.646829in}{2.396470in}}%
\pgfpathclose%
\pgfusepath{stroke,fill}%
\end{pgfscope}%
\begin{pgfscope}%
\pgfpathrectangle{\pgfqpoint{0.605784in}{0.382904in}}{\pgfqpoint{4.063488in}{2.042155in}}%
\pgfusepath{clip}%
\pgfsetbuttcap%
\pgfsetroundjoin%
\definecolor{currentfill}{rgb}{0.720391,0.870350,0.162603}%
\pgfsetfillcolor{currentfill}%
\pgfsetlinewidth{1.003750pt}%
\definecolor{currentstroke}{rgb}{0.720391,0.870350,0.162603}%
\pgfsetstrokecolor{currentstroke}%
\pgfsetdash{}{0pt}%
\pgfpathmoveto{\pgfqpoint{1.124327in}{0.403532in}}%
\pgfpathlineto{\pgfqpoint{1.139373in}{0.418431in}}%
\pgfpathlineto{\pgfqpoint{1.146224in}{0.424160in}}%
\pgfpathlineto{\pgfqpoint{1.171006in}{0.444787in}}%
\pgfpathlineto{\pgfqpoint{1.180419in}{0.452571in}}%
\pgfpathlineto{\pgfqpoint{1.202061in}{0.465415in}}%
\pgfpathlineto{\pgfqpoint{1.221464in}{0.476892in}}%
\pgfpathlineto{\pgfqpoint{1.250500in}{0.486043in}}%
\pgfpathlineto{\pgfqpoint{1.262509in}{0.489849in}}%
\pgfpathlineto{\pgfqpoint{1.303555in}{0.491772in}}%
\pgfpathlineto{\pgfqpoint{1.336465in}{0.486043in}}%
\pgfpathlineto{\pgfqpoint{1.344600in}{0.484646in}}%
\pgfpathlineto{\pgfqpoint{1.385645in}{0.471908in}}%
\pgfpathlineto{\pgfqpoint{1.403949in}{0.465415in}}%
\pgfpathlineto{\pgfqpoint{1.426691in}{0.457366in}}%
\pgfpathlineto{\pgfqpoint{1.467198in}{0.444787in}}%
\pgfpathlineto{\pgfqpoint{1.467736in}{0.444620in}}%
\pgfpathlineto{\pgfqpoint{1.508781in}{0.436504in}}%
\pgfpathlineto{\pgfqpoint{1.549827in}{0.434302in}}%
\pgfpathlineto{\pgfqpoint{1.590872in}{0.438190in}}%
\pgfpathlineto{\pgfqpoint{1.620380in}{0.444787in}}%
\pgfpathlineto{\pgfqpoint{1.631917in}{0.447418in}}%
\pgfpathlineto{\pgfqpoint{1.672963in}{0.461029in}}%
\pgfpathlineto{\pgfqpoint{1.683377in}{0.465415in}}%
\pgfpathlineto{\pgfqpoint{1.714008in}{0.478326in}}%
\pgfpathlineto{\pgfqpoint{1.729070in}{0.486043in}}%
\pgfpathlineto{\pgfqpoint{1.755053in}{0.499246in}}%
\pgfpathlineto{\pgfqpoint{1.767145in}{0.506671in}}%
\pgfpathlineto{\pgfqpoint{1.796099in}{0.524173in}}%
\pgfpathlineto{\pgfqpoint{1.800414in}{0.527299in}}%
\pgfpathlineto{\pgfqpoint{1.829272in}{0.547927in}}%
\pgfpathlineto{\pgfqpoint{1.837144in}{0.553458in}}%
\pgfpathlineto{\pgfqpoint{1.855559in}{0.568554in}}%
\pgfpathlineto{\pgfqpoint{1.878189in}{0.586669in}}%
\pgfpathlineto{\pgfqpoint{1.881024in}{0.589182in}}%
\pgfpathlineto{\pgfqpoint{1.904495in}{0.609810in}}%
\pgfpathlineto{\pgfqpoint{1.919235in}{0.622493in}}%
\pgfpathlineto{\pgfqpoint{1.928180in}{0.630438in}}%
\pgfpathlineto{\pgfqpoint{1.951712in}{0.651066in}}%
\pgfpathlineto{\pgfqpoint{1.960280in}{0.658480in}}%
\pgfpathlineto{\pgfqpoint{1.976483in}{0.671694in}}%
\pgfpathlineto{\pgfqpoint{2.001325in}{0.691626in}}%
\pgfpathlineto{\pgfqpoint{2.002360in}{0.692321in}}%
\pgfpathlineto{\pgfqpoint{2.033159in}{0.712949in}}%
\pgfpathlineto{\pgfqpoint{2.042371in}{0.719066in}}%
\pgfpathlineto{\pgfqpoint{2.073057in}{0.733577in}}%
\pgfpathlineto{\pgfqpoint{2.083416in}{0.738470in}}%
\pgfpathlineto{\pgfqpoint{2.124461in}{0.748991in}}%
\pgfpathlineto{\pgfqpoint{2.165507in}{0.751310in}}%
\pgfpathlineto{\pgfqpoint{2.206552in}{0.747629in}}%
\pgfpathlineto{\pgfqpoint{2.247597in}{0.741007in}}%
\pgfpathlineto{\pgfqpoint{2.288643in}{0.734744in}}%
\pgfpathlineto{\pgfqpoint{2.304413in}{0.733577in}}%
\pgfpathlineto{\pgfqpoint{2.329688in}{0.731707in}}%
\pgfpathlineto{\pgfqpoint{2.370733in}{0.733515in}}%
\pgfpathlineto{\pgfqpoint{2.371105in}{0.733577in}}%
\pgfpathlineto{\pgfqpoint{2.411779in}{0.740553in}}%
\pgfpathlineto{\pgfqpoint{2.452824in}{0.751770in}}%
\pgfpathlineto{\pgfqpoint{2.460104in}{0.754205in}}%
\pgfpathlineto{\pgfqpoint{2.493869in}{0.765699in}}%
\pgfpathlineto{\pgfqpoint{2.495039in}{0.754205in}}%
\pgfpathlineto{\pgfqpoint{2.497114in}{0.733577in}}%
\pgfpathlineto{\pgfqpoint{2.499149in}{0.712949in}}%
\pgfpathlineto{\pgfqpoint{2.501145in}{0.692321in}}%
\pgfpathlineto{\pgfqpoint{2.503105in}{0.671694in}}%
\pgfpathlineto{\pgfqpoint{2.505031in}{0.651066in}}%
\pgfpathlineto{\pgfqpoint{2.506925in}{0.630438in}}%
\pgfpathlineto{\pgfqpoint{2.508788in}{0.609810in}}%
\pgfpathlineto{\pgfqpoint{2.510622in}{0.589182in}}%
\pgfpathlineto{\pgfqpoint{2.512428in}{0.568554in}}%
\pgfpathlineto{\pgfqpoint{2.514208in}{0.547927in}}%
\pgfpathlineto{\pgfqpoint{2.515963in}{0.527299in}}%
\pgfpathlineto{\pgfqpoint{2.517695in}{0.506671in}}%
\pgfpathlineto{\pgfqpoint{2.519404in}{0.486043in}}%
\pgfpathlineto{\pgfqpoint{2.521092in}{0.465415in}}%
\pgfpathlineto{\pgfqpoint{2.522759in}{0.444787in}}%
\pgfpathlineto{\pgfqpoint{2.524407in}{0.424160in}}%
\pgfpathlineto{\pgfqpoint{2.526036in}{0.403532in}}%
\pgfpathlineto{\pgfqpoint{2.527647in}{0.382904in}}%
\pgfpathlineto{\pgfqpoint{2.523713in}{0.382904in}}%
\pgfpathlineto{\pgfqpoint{2.522043in}{0.403532in}}%
\pgfpathlineto{\pgfqpoint{2.520353in}{0.424160in}}%
\pgfpathlineto{\pgfqpoint{2.518643in}{0.444787in}}%
\pgfpathlineto{\pgfqpoint{2.516911in}{0.465415in}}%
\pgfpathlineto{\pgfqpoint{2.515157in}{0.486043in}}%
\pgfpathlineto{\pgfqpoint{2.513378in}{0.506671in}}%
\pgfpathlineto{\pgfqpoint{2.511576in}{0.527299in}}%
\pgfpathlineto{\pgfqpoint{2.509747in}{0.547927in}}%
\pgfpathlineto{\pgfqpoint{2.507890in}{0.568554in}}%
\pgfpathlineto{\pgfqpoint{2.506005in}{0.589182in}}%
\pgfpathlineto{\pgfqpoint{2.504090in}{0.609810in}}%
\pgfpathlineto{\pgfqpoint{2.502143in}{0.630438in}}%
\pgfpathlineto{\pgfqpoint{2.500162in}{0.651066in}}%
\pgfpathlineto{\pgfqpoint{2.498145in}{0.671694in}}%
\pgfpathlineto{\pgfqpoint{2.496090in}{0.692321in}}%
\pgfpathlineto{\pgfqpoint{2.493996in}{0.712949in}}%
\pgfpathlineto{\pgfqpoint{2.493869in}{0.714199in}}%
\pgfpathlineto{\pgfqpoint{2.490074in}{0.712949in}}%
\pgfpathlineto{\pgfqpoint{2.452824in}{0.700885in}}%
\pgfpathlineto{\pgfqpoint{2.419324in}{0.692321in}}%
\pgfpathlineto{\pgfqpoint{2.411779in}{0.690440in}}%
\pgfpathlineto{\pgfqpoint{2.370733in}{0.684400in}}%
\pgfpathlineto{\pgfqpoint{2.329688in}{0.683447in}}%
\pgfpathlineto{\pgfqpoint{2.288643in}{0.687191in}}%
\pgfpathlineto{\pgfqpoint{2.257291in}{0.692321in}}%
\pgfpathlineto{\pgfqpoint{2.247597in}{0.693910in}}%
\pgfpathlineto{\pgfqpoint{2.206552in}{0.700876in}}%
\pgfpathlineto{\pgfqpoint{2.165507in}{0.704723in}}%
\pgfpathlineto{\pgfqpoint{2.124461in}{0.702428in}}%
\pgfpathlineto{\pgfqpoint{2.085332in}{0.692321in}}%
\pgfpathlineto{\pgfqpoint{2.083416in}{0.691830in}}%
\pgfpathlineto{\pgfqpoint{2.042371in}{0.672328in}}%
\pgfpathlineto{\pgfqpoint{2.041420in}{0.671694in}}%
\pgfpathlineto{\pgfqpoint{2.010610in}{0.651066in}}%
\pgfpathlineto{\pgfqpoint{2.001325in}{0.644798in}}%
\pgfpathlineto{\pgfqpoint{1.983547in}{0.630438in}}%
\pgfpathlineto{\pgfqpoint{1.960280in}{0.611349in}}%
\pgfpathlineto{\pgfqpoint{1.958518in}{0.609810in}}%
\pgfpathlineto{\pgfqpoint{1.935042in}{0.589182in}}%
\pgfpathlineto{\pgfqpoint{1.919235in}{0.575029in}}%
\pgfpathlineto{\pgfqpoint{1.911801in}{0.568554in}}%
\pgfpathlineto{\pgfqpoint{1.888425in}{0.547927in}}%
\pgfpathlineto{\pgfqpoint{1.878189in}{0.538752in}}%
\pgfpathlineto{\pgfqpoint{1.864100in}{0.527299in}}%
\pgfpathlineto{\pgfqpoint{1.839250in}{0.506671in}}%
\pgfpathlineto{\pgfqpoint{1.837144in}{0.504907in}}%
\pgfpathlineto{\pgfqpoint{1.811027in}{0.486043in}}%
\pgfpathlineto{\pgfqpoint{1.796099in}{0.475041in}}%
\pgfpathlineto{\pgfqpoint{1.780471in}{0.465415in}}%
\pgfpathlineto{\pgfqpoint{1.755053in}{0.449520in}}%
\pgfpathlineto{\pgfqpoint{1.745880in}{0.444787in}}%
\pgfpathlineto{\pgfqpoint{1.714008in}{0.428215in}}%
\pgfpathlineto{\pgfqpoint{1.704438in}{0.424160in}}%
\pgfpathlineto{\pgfqpoint{1.672963in}{0.410834in}}%
\pgfpathlineto{\pgfqpoint{1.650727in}{0.403532in}}%
\pgfpathlineto{\pgfqpoint{1.631917in}{0.397416in}}%
\pgfpathlineto{\pgfqpoint{1.590872in}{0.388594in}}%
\pgfpathlineto{\pgfqpoint{1.549827in}{0.385245in}}%
\pgfpathlineto{\pgfqpoint{1.508781in}{0.388035in}}%
\pgfpathlineto{\pgfqpoint{1.467736in}{0.396673in}}%
\pgfpathlineto{\pgfqpoint{1.446100in}{0.403532in}}%
\pgfpathlineto{\pgfqpoint{1.426691in}{0.409686in}}%
\pgfpathlineto{\pgfqpoint{1.386149in}{0.424160in}}%
\pgfpathlineto{\pgfqpoint{1.385645in}{0.424340in}}%
\pgfpathlineto{\pgfqpoint{1.344600in}{0.437154in}}%
\pgfpathlineto{\pgfqpoint{1.303555in}{0.444146in}}%
\pgfpathlineto{\pgfqpoint{1.262509in}{0.442162in}}%
\pgfpathlineto{\pgfqpoint{1.221464in}{0.429146in}}%
\pgfpathlineto{\pgfqpoint{1.213035in}{0.424160in}}%
\pgfpathlineto{\pgfqpoint{1.180419in}{0.404799in}}%
\pgfpathlineto{\pgfqpoint{1.178886in}{0.403532in}}%
\pgfpathlineto{\pgfqpoint{1.153956in}{0.382904in}}%
\pgfpathlineto{\pgfqpoint{1.139373in}{0.382904in}}%
\pgfpathlineto{\pgfqpoint{1.103772in}{0.382904in}}%
\pgfpathclose%
\pgfusepath{stroke,fill}%
\end{pgfscope}%
\begin{pgfscope}%
\pgfpathrectangle{\pgfqpoint{0.605784in}{0.382904in}}{\pgfqpoint{4.063488in}{2.042155in}}%
\pgfusepath{clip}%
\pgfsetbuttcap%
\pgfsetroundjoin%
\definecolor{currentfill}{rgb}{0.720391,0.870350,0.162603}%
\pgfsetfillcolor{currentfill}%
\pgfsetlinewidth{1.003750pt}%
\definecolor{currentstroke}{rgb}{0.720391,0.870350,0.162603}%
\pgfsetstrokecolor{currentstroke}%
\pgfsetdash{}{0pt}%
\pgfpathmoveto{\pgfqpoint{3.682647in}{2.301292in}}%
\pgfpathlineto{\pgfqpoint{3.678415in}{2.321920in}}%
\pgfpathlineto{\pgfqpoint{3.674273in}{2.342547in}}%
\pgfpathlineto{\pgfqpoint{3.670215in}{2.363175in}}%
\pgfpathlineto{\pgfqpoint{3.666237in}{2.383803in}}%
\pgfpathlineto{\pgfqpoint{3.662334in}{2.404431in}}%
\pgfpathlineto{\pgfqpoint{3.658502in}{2.425059in}}%
\pgfpathlineto{\pgfqpoint{3.667675in}{2.425059in}}%
\pgfpathlineto{\pgfqpoint{3.671676in}{2.404431in}}%
\pgfpathlineto{\pgfqpoint{3.675755in}{2.383803in}}%
\pgfpathlineto{\pgfqpoint{3.679916in}{2.363175in}}%
\pgfpathlineto{\pgfqpoint{3.684165in}{2.342547in}}%
\pgfpathlineto{\pgfqpoint{3.684184in}{2.342454in}}%
\pgfpathlineto{\pgfqpoint{3.725229in}{2.342454in}}%
\pgfpathlineto{\pgfqpoint{3.766275in}{2.342454in}}%
\pgfpathlineto{\pgfqpoint{3.807320in}{2.342454in}}%
\pgfpathlineto{\pgfqpoint{3.848365in}{2.342454in}}%
\pgfpathlineto{\pgfqpoint{3.889411in}{2.342454in}}%
\pgfpathlineto{\pgfqpoint{3.930456in}{2.342454in}}%
\pgfpathlineto{\pgfqpoint{3.971501in}{2.342454in}}%
\pgfpathlineto{\pgfqpoint{4.012547in}{2.342454in}}%
\pgfpathlineto{\pgfqpoint{4.053592in}{2.342454in}}%
\pgfpathlineto{\pgfqpoint{4.094637in}{2.342454in}}%
\pgfpathlineto{\pgfqpoint{4.135683in}{2.342454in}}%
\pgfpathlineto{\pgfqpoint{4.176728in}{2.342454in}}%
\pgfpathlineto{\pgfqpoint{4.217773in}{2.342454in}}%
\pgfpathlineto{\pgfqpoint{4.258819in}{2.342454in}}%
\pgfpathlineto{\pgfqpoint{4.299864in}{2.342454in}}%
\pgfpathlineto{\pgfqpoint{4.340909in}{2.342454in}}%
\pgfpathlineto{\pgfqpoint{4.381955in}{2.342454in}}%
\pgfpathlineto{\pgfqpoint{4.423000in}{2.342454in}}%
\pgfpathlineto{\pgfqpoint{4.464045in}{2.342454in}}%
\pgfpathlineto{\pgfqpoint{4.505091in}{2.342454in}}%
\pgfpathlineto{\pgfqpoint{4.546136in}{2.342454in}}%
\pgfpathlineto{\pgfqpoint{4.587181in}{2.342454in}}%
\pgfpathlineto{\pgfqpoint{4.628227in}{2.342454in}}%
\pgfpathlineto{\pgfqpoint{4.669272in}{2.342454in}}%
\pgfpathlineto{\pgfqpoint{4.669272in}{2.321920in}}%
\pgfpathlineto{\pgfqpoint{4.669272in}{2.301292in}}%
\pgfpathlineto{\pgfqpoint{4.669272in}{2.293870in}}%
\pgfpathlineto{\pgfqpoint{4.628227in}{2.293870in}}%
\pgfpathlineto{\pgfqpoint{4.587181in}{2.293870in}}%
\pgfpathlineto{\pgfqpoint{4.546136in}{2.293870in}}%
\pgfpathlineto{\pgfqpoint{4.505091in}{2.293870in}}%
\pgfpathlineto{\pgfqpoint{4.464045in}{2.293870in}}%
\pgfpathlineto{\pgfqpoint{4.423000in}{2.293870in}}%
\pgfpathlineto{\pgfqpoint{4.381955in}{2.293870in}}%
\pgfpathlineto{\pgfqpoint{4.340909in}{2.293870in}}%
\pgfpathlineto{\pgfqpoint{4.299864in}{2.293870in}}%
\pgfpathlineto{\pgfqpoint{4.258819in}{2.293870in}}%
\pgfpathlineto{\pgfqpoint{4.217773in}{2.293870in}}%
\pgfpathlineto{\pgfqpoint{4.176728in}{2.293870in}}%
\pgfpathlineto{\pgfqpoint{4.135683in}{2.293870in}}%
\pgfpathlineto{\pgfqpoint{4.094637in}{2.293870in}}%
\pgfpathlineto{\pgfqpoint{4.053592in}{2.293870in}}%
\pgfpathlineto{\pgfqpoint{4.012547in}{2.293870in}}%
\pgfpathlineto{\pgfqpoint{3.971501in}{2.293870in}}%
\pgfpathlineto{\pgfqpoint{3.930456in}{2.293870in}}%
\pgfpathlineto{\pgfqpoint{3.889411in}{2.293870in}}%
\pgfpathlineto{\pgfqpoint{3.848365in}{2.293870in}}%
\pgfpathlineto{\pgfqpoint{3.807320in}{2.293870in}}%
\pgfpathlineto{\pgfqpoint{3.766275in}{2.293870in}}%
\pgfpathlineto{\pgfqpoint{3.725229in}{2.293870in}}%
\pgfpathlineto{\pgfqpoint{3.684184in}{2.293870in}}%
\pgfpathclose%
\pgfusepath{stroke,fill}%
\end{pgfscope}%
\begin{pgfscope}%
\pgfpathrectangle{\pgfqpoint{0.605784in}{0.382904in}}{\pgfqpoint{4.063488in}{2.042155in}}%
\pgfusepath{clip}%
\pgfsetbuttcap%
\pgfsetroundjoin%
\definecolor{currentfill}{rgb}{0.906311,0.894855,0.098125}%
\pgfsetfillcolor{currentfill}%
\pgfsetlinewidth{1.003750pt}%
\definecolor{currentstroke}{rgb}{0.906311,0.894855,0.098125}%
\pgfsetstrokecolor{currentstroke}%
\pgfsetdash{}{0pt}%
\pgfpathmoveto{\pgfqpoint{0.683011in}{2.425059in}}%
\pgfpathlineto{\pgfqpoint{0.687875in}{2.425059in}}%
\pgfpathlineto{\pgfqpoint{0.728920in}{2.425059in}}%
\pgfpathlineto{\pgfqpoint{0.769965in}{2.425059in}}%
\pgfpathlineto{\pgfqpoint{0.811011in}{2.425059in}}%
\pgfpathlineto{\pgfqpoint{0.852056in}{2.425059in}}%
\pgfpathlineto{\pgfqpoint{0.864946in}{2.425059in}}%
\pgfpathlineto{\pgfqpoint{0.852056in}{2.415524in}}%
\pgfpathlineto{\pgfqpoint{0.825015in}{2.404431in}}%
\pgfpathlineto{\pgfqpoint{0.811011in}{2.398667in}}%
\pgfpathlineto{\pgfqpoint{0.769965in}{2.395646in}}%
\pgfpathlineto{\pgfqpoint{0.730386in}{2.404431in}}%
\pgfpathlineto{\pgfqpoint{0.728920in}{2.404751in}}%
\pgfpathlineto{\pgfqpoint{0.687875in}{2.422441in}}%
\pgfpathclose%
\pgfusepath{stroke,fill}%
\end{pgfscope}%
\begin{pgfscope}%
\pgfpathrectangle{\pgfqpoint{0.605784in}{0.382904in}}{\pgfqpoint{4.063488in}{2.042155in}}%
\pgfusepath{clip}%
\pgfsetbuttcap%
\pgfsetroundjoin%
\definecolor{currentfill}{rgb}{0.906311,0.894855,0.098125}%
\pgfsetfillcolor{currentfill}%
\pgfsetlinewidth{1.003750pt}%
\definecolor{currentstroke}{rgb}{0.906311,0.894855,0.098125}%
\pgfsetstrokecolor{currentstroke}%
\pgfsetdash{}{0pt}%
\pgfpathmoveto{\pgfqpoint{1.178886in}{0.403532in}}%
\pgfpathlineto{\pgfqpoint{1.180419in}{0.404799in}}%
\pgfpathlineto{\pgfqpoint{1.213035in}{0.424160in}}%
\pgfpathlineto{\pgfqpoint{1.221464in}{0.429146in}}%
\pgfpathlineto{\pgfqpoint{1.262509in}{0.442162in}}%
\pgfpathlineto{\pgfqpoint{1.303555in}{0.444146in}}%
\pgfpathlineto{\pgfqpoint{1.344600in}{0.437154in}}%
\pgfpathlineto{\pgfqpoint{1.385645in}{0.424340in}}%
\pgfpathlineto{\pgfqpoint{1.386149in}{0.424160in}}%
\pgfpathlineto{\pgfqpoint{1.426691in}{0.409686in}}%
\pgfpathlineto{\pgfqpoint{1.446100in}{0.403532in}}%
\pgfpathlineto{\pgfqpoint{1.467736in}{0.396673in}}%
\pgfpathlineto{\pgfqpoint{1.508781in}{0.388035in}}%
\pgfpathlineto{\pgfqpoint{1.549827in}{0.385245in}}%
\pgfpathlineto{\pgfqpoint{1.590872in}{0.388594in}}%
\pgfpathlineto{\pgfqpoint{1.631917in}{0.397416in}}%
\pgfpathlineto{\pgfqpoint{1.650727in}{0.403532in}}%
\pgfpathlineto{\pgfqpoint{1.672963in}{0.410834in}}%
\pgfpathlineto{\pgfqpoint{1.704438in}{0.424160in}}%
\pgfpathlineto{\pgfqpoint{1.714008in}{0.428215in}}%
\pgfpathlineto{\pgfqpoint{1.745880in}{0.444787in}}%
\pgfpathlineto{\pgfqpoint{1.755053in}{0.449520in}}%
\pgfpathlineto{\pgfqpoint{1.780471in}{0.465415in}}%
\pgfpathlineto{\pgfqpoint{1.796099in}{0.475041in}}%
\pgfpathlineto{\pgfqpoint{1.811027in}{0.486043in}}%
\pgfpathlineto{\pgfqpoint{1.837144in}{0.504907in}}%
\pgfpathlineto{\pgfqpoint{1.839250in}{0.506671in}}%
\pgfpathlineto{\pgfqpoint{1.864100in}{0.527299in}}%
\pgfpathlineto{\pgfqpoint{1.878189in}{0.538752in}}%
\pgfpathlineto{\pgfqpoint{1.888425in}{0.547927in}}%
\pgfpathlineto{\pgfqpoint{1.911801in}{0.568554in}}%
\pgfpathlineto{\pgfqpoint{1.919235in}{0.575029in}}%
\pgfpathlineto{\pgfqpoint{1.935042in}{0.589182in}}%
\pgfpathlineto{\pgfqpoint{1.958518in}{0.609810in}}%
\pgfpathlineto{\pgfqpoint{1.960280in}{0.611349in}}%
\pgfpathlineto{\pgfqpoint{1.983547in}{0.630438in}}%
\pgfpathlineto{\pgfqpoint{2.001325in}{0.644798in}}%
\pgfpathlineto{\pgfqpoint{2.010610in}{0.651066in}}%
\pgfpathlineto{\pgfqpoint{2.041420in}{0.671694in}}%
\pgfpathlineto{\pgfqpoint{2.042371in}{0.672328in}}%
\pgfpathlineto{\pgfqpoint{2.083416in}{0.691830in}}%
\pgfpathlineto{\pgfqpoint{2.085332in}{0.692321in}}%
\pgfpathlineto{\pgfqpoint{2.124461in}{0.702428in}}%
\pgfpathlineto{\pgfqpoint{2.165507in}{0.704723in}}%
\pgfpathlineto{\pgfqpoint{2.206552in}{0.700876in}}%
\pgfpathlineto{\pgfqpoint{2.247597in}{0.693910in}}%
\pgfpathlineto{\pgfqpoint{2.257291in}{0.692321in}}%
\pgfpathlineto{\pgfqpoint{2.288643in}{0.687191in}}%
\pgfpathlineto{\pgfqpoint{2.329688in}{0.683447in}}%
\pgfpathlineto{\pgfqpoint{2.370733in}{0.684400in}}%
\pgfpathlineto{\pgfqpoint{2.411779in}{0.690440in}}%
\pgfpathlineto{\pgfqpoint{2.419324in}{0.692321in}}%
\pgfpathlineto{\pgfqpoint{2.452824in}{0.700885in}}%
\pgfpathlineto{\pgfqpoint{2.490074in}{0.712949in}}%
\pgfpathlineto{\pgfqpoint{2.493869in}{0.714199in}}%
\pgfpathlineto{\pgfqpoint{2.493996in}{0.712949in}}%
\pgfpathlineto{\pgfqpoint{2.496090in}{0.692321in}}%
\pgfpathlineto{\pgfqpoint{2.498145in}{0.671694in}}%
\pgfpathlineto{\pgfqpoint{2.500162in}{0.651066in}}%
\pgfpathlineto{\pgfqpoint{2.502143in}{0.630438in}}%
\pgfpathlineto{\pgfqpoint{2.504090in}{0.609810in}}%
\pgfpathlineto{\pgfqpoint{2.506005in}{0.589182in}}%
\pgfpathlineto{\pgfqpoint{2.507890in}{0.568554in}}%
\pgfpathlineto{\pgfqpoint{2.509747in}{0.547927in}}%
\pgfpathlineto{\pgfqpoint{2.511576in}{0.527299in}}%
\pgfpathlineto{\pgfqpoint{2.513378in}{0.506671in}}%
\pgfpathlineto{\pgfqpoint{2.515157in}{0.486043in}}%
\pgfpathlineto{\pgfqpoint{2.516911in}{0.465415in}}%
\pgfpathlineto{\pgfqpoint{2.518643in}{0.444787in}}%
\pgfpathlineto{\pgfqpoint{2.520353in}{0.424160in}}%
\pgfpathlineto{\pgfqpoint{2.522043in}{0.403532in}}%
\pgfpathlineto{\pgfqpoint{2.523713in}{0.382904in}}%
\pgfpathlineto{\pgfqpoint{2.519780in}{0.382904in}}%
\pgfpathlineto{\pgfqpoint{2.518050in}{0.403532in}}%
\pgfpathlineto{\pgfqpoint{2.516300in}{0.424160in}}%
\pgfpathlineto{\pgfqpoint{2.514527in}{0.444787in}}%
\pgfpathlineto{\pgfqpoint{2.512730in}{0.465415in}}%
\pgfpathlineto{\pgfqpoint{2.510909in}{0.486043in}}%
\pgfpathlineto{\pgfqpoint{2.509062in}{0.506671in}}%
\pgfpathlineto{\pgfqpoint{2.507188in}{0.527299in}}%
\pgfpathlineto{\pgfqpoint{2.505285in}{0.547927in}}%
\pgfpathlineto{\pgfqpoint{2.503353in}{0.568554in}}%
\pgfpathlineto{\pgfqpoint{2.501389in}{0.589182in}}%
\pgfpathlineto{\pgfqpoint{2.499392in}{0.609810in}}%
\pgfpathlineto{\pgfqpoint{2.497361in}{0.630438in}}%
\pgfpathlineto{\pgfqpoint{2.495292in}{0.651066in}}%
\pgfpathlineto{\pgfqpoint{2.493869in}{0.665074in}}%
\pgfpathlineto{\pgfqpoint{2.452824in}{0.652174in}}%
\pgfpathlineto{\pgfqpoint{2.448212in}{0.651066in}}%
\pgfpathlineto{\pgfqpoint{2.411779in}{0.642517in}}%
\pgfpathlineto{\pgfqpoint{2.370733in}{0.637259in}}%
\pgfpathlineto{\pgfqpoint{2.329688in}{0.637073in}}%
\pgfpathlineto{\pgfqpoint{2.288643in}{0.641465in}}%
\pgfpathlineto{\pgfqpoint{2.247597in}{0.648632in}}%
\pgfpathlineto{\pgfqpoint{2.233727in}{0.651066in}}%
\pgfpathlineto{\pgfqpoint{2.206552in}{0.655858in}}%
\pgfpathlineto{\pgfqpoint{2.165507in}{0.659849in}}%
\pgfpathlineto{\pgfqpoint{2.124461in}{0.657579in}}%
\pgfpathlineto{\pgfqpoint{2.099318in}{0.651066in}}%
\pgfpathlineto{\pgfqpoint{2.083416in}{0.646976in}}%
\pgfpathlineto{\pgfqpoint{2.048763in}{0.630438in}}%
\pgfpathlineto{\pgfqpoint{2.042371in}{0.627384in}}%
\pgfpathlineto{\pgfqpoint{2.016228in}{0.609810in}}%
\pgfpathlineto{\pgfqpoint{2.001325in}{0.599705in}}%
\pgfpathlineto{\pgfqpoint{1.988378in}{0.589182in}}%
\pgfpathlineto{\pgfqpoint{1.963318in}{0.568554in}}%
\pgfpathlineto{\pgfqpoint{1.960280in}{0.566041in}}%
\pgfpathlineto{\pgfqpoint{1.939799in}{0.547927in}}%
\pgfpathlineto{\pgfqpoint{1.919235in}{0.529375in}}%
\pgfpathlineto{\pgfqpoint{1.916877in}{0.527299in}}%
\pgfpathlineto{\pgfqpoint{1.893636in}{0.506671in}}%
\pgfpathlineto{\pgfqpoint{1.878189in}{0.492686in}}%
\pgfpathlineto{\pgfqpoint{1.870133in}{0.486043in}}%
\pgfpathlineto{\pgfqpoint{1.845468in}{0.465415in}}%
\pgfpathlineto{\pgfqpoint{1.837144in}{0.458353in}}%
\pgfpathlineto{\pgfqpoint{1.818668in}{0.444787in}}%
\pgfpathlineto{\pgfqpoint{1.796099in}{0.427890in}}%
\pgfpathlineto{\pgfqpoint{1.790146in}{0.424160in}}%
\pgfpathlineto{\pgfqpoint{1.757639in}{0.403532in}}%
\pgfpathlineto{\pgfqpoint{1.755053in}{0.401875in}}%
\pgfpathlineto{\pgfqpoint{1.719020in}{0.382904in}}%
\pgfpathlineto{\pgfqpoint{1.714008in}{0.382904in}}%
\pgfpathlineto{\pgfqpoint{1.672963in}{0.382904in}}%
\pgfpathlineto{\pgfqpoint{1.631917in}{0.382904in}}%
\pgfpathlineto{\pgfqpoint{1.590872in}{0.382904in}}%
\pgfpathlineto{\pgfqpoint{1.549827in}{0.382904in}}%
\pgfpathlineto{\pgfqpoint{1.508781in}{0.382904in}}%
\pgfpathlineto{\pgfqpoint{1.467736in}{0.382904in}}%
\pgfpathlineto{\pgfqpoint{1.426691in}{0.382904in}}%
\pgfpathlineto{\pgfqpoint{1.385645in}{0.382904in}}%
\pgfpathlineto{\pgfqpoint{1.372013in}{0.382904in}}%
\pgfpathlineto{\pgfqpoint{1.344600in}{0.391466in}}%
\pgfpathlineto{\pgfqpoint{1.303555in}{0.398421in}}%
\pgfpathlineto{\pgfqpoint{1.262509in}{0.396354in}}%
\pgfpathlineto{\pgfqpoint{1.221464in}{0.383242in}}%
\pgfpathlineto{\pgfqpoint{1.220893in}{0.382904in}}%
\pgfpathlineto{\pgfqpoint{1.180419in}{0.382904in}}%
\pgfpathlineto{\pgfqpoint{1.153956in}{0.382904in}}%
\pgfpathclose%
\pgfusepath{stroke,fill}%
\end{pgfscope}%
\begin{pgfscope}%
\pgfpathrectangle{\pgfqpoint{0.605784in}{0.382904in}}{\pgfqpoint{4.063488in}{2.042155in}}%
\pgfusepath{clip}%
\pgfsetbuttcap%
\pgfsetroundjoin%
\definecolor{currentfill}{rgb}{0.906311,0.894855,0.098125}%
\pgfsetfillcolor{currentfill}%
\pgfsetlinewidth{1.003750pt}%
\definecolor{currentstroke}{rgb}{0.906311,0.894855,0.098125}%
\pgfsetstrokecolor{currentstroke}%
\pgfsetdash{}{0pt}%
\pgfpathmoveto{\pgfqpoint{3.684165in}{2.342547in}}%
\pgfpathlineto{\pgfqpoint{3.679916in}{2.363175in}}%
\pgfpathlineto{\pgfqpoint{3.675755in}{2.383803in}}%
\pgfpathlineto{\pgfqpoint{3.671676in}{2.404431in}}%
\pgfpathlineto{\pgfqpoint{3.667675in}{2.425059in}}%
\pgfpathlineto{\pgfqpoint{3.676847in}{2.425059in}}%
\pgfpathlineto{\pgfqpoint{3.681018in}{2.404431in}}%
\pgfpathlineto{\pgfqpoint{3.684184in}{2.389012in}}%
\pgfpathlineto{\pgfqpoint{3.725229in}{2.389012in}}%
\pgfpathlineto{\pgfqpoint{3.766275in}{2.389012in}}%
\pgfpathlineto{\pgfqpoint{3.807320in}{2.389012in}}%
\pgfpathlineto{\pgfqpoint{3.848365in}{2.389012in}}%
\pgfpathlineto{\pgfqpoint{3.889411in}{2.389012in}}%
\pgfpathlineto{\pgfqpoint{3.930456in}{2.389012in}}%
\pgfpathlineto{\pgfqpoint{3.971501in}{2.389012in}}%
\pgfpathlineto{\pgfqpoint{4.012547in}{2.389012in}}%
\pgfpathlineto{\pgfqpoint{4.053592in}{2.389012in}}%
\pgfpathlineto{\pgfqpoint{4.094637in}{2.389012in}}%
\pgfpathlineto{\pgfqpoint{4.135683in}{2.389012in}}%
\pgfpathlineto{\pgfqpoint{4.176728in}{2.389012in}}%
\pgfpathlineto{\pgfqpoint{4.217773in}{2.389012in}}%
\pgfpathlineto{\pgfqpoint{4.258819in}{2.389012in}}%
\pgfpathlineto{\pgfqpoint{4.299864in}{2.389012in}}%
\pgfpathlineto{\pgfqpoint{4.340909in}{2.389012in}}%
\pgfpathlineto{\pgfqpoint{4.381955in}{2.389012in}}%
\pgfpathlineto{\pgfqpoint{4.423000in}{2.389012in}}%
\pgfpathlineto{\pgfqpoint{4.464045in}{2.389012in}}%
\pgfpathlineto{\pgfqpoint{4.505091in}{2.389012in}}%
\pgfpathlineto{\pgfqpoint{4.546136in}{2.389012in}}%
\pgfpathlineto{\pgfqpoint{4.587181in}{2.389012in}}%
\pgfpathlineto{\pgfqpoint{4.628227in}{2.389012in}}%
\pgfpathlineto{\pgfqpoint{4.669272in}{2.389012in}}%
\pgfpathlineto{\pgfqpoint{4.669272in}{2.383803in}}%
\pgfpathlineto{\pgfqpoint{4.669272in}{2.363175in}}%
\pgfpathlineto{\pgfqpoint{4.669272in}{2.342547in}}%
\pgfpathlineto{\pgfqpoint{4.669272in}{2.342454in}}%
\pgfpathlineto{\pgfqpoint{4.628227in}{2.342454in}}%
\pgfpathlineto{\pgfqpoint{4.587181in}{2.342454in}}%
\pgfpathlineto{\pgfqpoint{4.546136in}{2.342454in}}%
\pgfpathlineto{\pgfqpoint{4.505091in}{2.342454in}}%
\pgfpathlineto{\pgfqpoint{4.464045in}{2.342454in}}%
\pgfpathlineto{\pgfqpoint{4.423000in}{2.342454in}}%
\pgfpathlineto{\pgfqpoint{4.381955in}{2.342454in}}%
\pgfpathlineto{\pgfqpoint{4.340909in}{2.342454in}}%
\pgfpathlineto{\pgfqpoint{4.299864in}{2.342454in}}%
\pgfpathlineto{\pgfqpoint{4.258819in}{2.342454in}}%
\pgfpathlineto{\pgfqpoint{4.217773in}{2.342454in}}%
\pgfpathlineto{\pgfqpoint{4.176728in}{2.342454in}}%
\pgfpathlineto{\pgfqpoint{4.135683in}{2.342454in}}%
\pgfpathlineto{\pgfqpoint{4.094637in}{2.342454in}}%
\pgfpathlineto{\pgfqpoint{4.053592in}{2.342454in}}%
\pgfpathlineto{\pgfqpoint{4.012547in}{2.342454in}}%
\pgfpathlineto{\pgfqpoint{3.971501in}{2.342454in}}%
\pgfpathlineto{\pgfqpoint{3.930456in}{2.342454in}}%
\pgfpathlineto{\pgfqpoint{3.889411in}{2.342454in}}%
\pgfpathlineto{\pgfqpoint{3.848365in}{2.342454in}}%
\pgfpathlineto{\pgfqpoint{3.807320in}{2.342454in}}%
\pgfpathlineto{\pgfqpoint{3.766275in}{2.342454in}}%
\pgfpathlineto{\pgfqpoint{3.725229in}{2.342454in}}%
\pgfpathlineto{\pgfqpoint{3.684184in}{2.342454in}}%
\pgfpathclose%
\pgfusepath{stroke,fill}%
\end{pgfscope}%
\begin{pgfscope}%
\pgfpathrectangle{\pgfqpoint{0.605784in}{0.382904in}}{\pgfqpoint{4.063488in}{2.042155in}}%
\pgfusepath{clip}%
\pgfsetbuttcap%
\pgfsetroundjoin%
\definecolor{currentfill}{rgb}{0.993248,0.906157,0.143936}%
\pgfsetfillcolor{currentfill}%
\pgfsetlinewidth{1.003750pt}%
\definecolor{currentstroke}{rgb}{0.993248,0.906157,0.143936}%
\pgfsetstrokecolor{currentstroke}%
\pgfsetdash{}{0pt}%
\pgfpathmoveto{\pgfqpoint{1.221464in}{0.383242in}}%
\pgfpathlineto{\pgfqpoint{1.262509in}{0.396354in}}%
\pgfpathlineto{\pgfqpoint{1.303555in}{0.398421in}}%
\pgfpathlineto{\pgfqpoint{1.344600in}{0.391466in}}%
\pgfpathlineto{\pgfqpoint{1.372013in}{0.382904in}}%
\pgfpathlineto{\pgfqpoint{1.344600in}{0.382904in}}%
\pgfpathlineto{\pgfqpoint{1.303555in}{0.382904in}}%
\pgfpathlineto{\pgfqpoint{1.262509in}{0.382904in}}%
\pgfpathlineto{\pgfqpoint{1.221464in}{0.382904in}}%
\pgfpathlineto{\pgfqpoint{1.220893in}{0.382904in}}%
\pgfpathclose%
\pgfusepath{stroke,fill}%
\end{pgfscope}%
\begin{pgfscope}%
\pgfpathrectangle{\pgfqpoint{0.605784in}{0.382904in}}{\pgfqpoint{4.063488in}{2.042155in}}%
\pgfusepath{clip}%
\pgfsetbuttcap%
\pgfsetroundjoin%
\definecolor{currentfill}{rgb}{0.993248,0.906157,0.143936}%
\pgfsetfillcolor{currentfill}%
\pgfsetlinewidth{1.003750pt}%
\definecolor{currentstroke}{rgb}{0.993248,0.906157,0.143936}%
\pgfsetstrokecolor{currentstroke}%
\pgfsetdash{}{0pt}%
\pgfpathmoveto{\pgfqpoint{1.755053in}{0.401875in}}%
\pgfpathlineto{\pgfqpoint{1.757639in}{0.403532in}}%
\pgfpathlineto{\pgfqpoint{1.790146in}{0.424160in}}%
\pgfpathlineto{\pgfqpoint{1.796099in}{0.427890in}}%
\pgfpathlineto{\pgfqpoint{1.818668in}{0.444787in}}%
\pgfpathlineto{\pgfqpoint{1.837144in}{0.458353in}}%
\pgfpathlineto{\pgfqpoint{1.845468in}{0.465415in}}%
\pgfpathlineto{\pgfqpoint{1.870133in}{0.486043in}}%
\pgfpathlineto{\pgfqpoint{1.878189in}{0.492686in}}%
\pgfpathlineto{\pgfqpoint{1.893636in}{0.506671in}}%
\pgfpathlineto{\pgfqpoint{1.916877in}{0.527299in}}%
\pgfpathlineto{\pgfqpoint{1.919235in}{0.529375in}}%
\pgfpathlineto{\pgfqpoint{1.939799in}{0.547927in}}%
\pgfpathlineto{\pgfqpoint{1.960280in}{0.566041in}}%
\pgfpathlineto{\pgfqpoint{1.963318in}{0.568554in}}%
\pgfpathlineto{\pgfqpoint{1.988378in}{0.589182in}}%
\pgfpathlineto{\pgfqpoint{2.001325in}{0.599705in}}%
\pgfpathlineto{\pgfqpoint{2.016228in}{0.609810in}}%
\pgfpathlineto{\pgfqpoint{2.042371in}{0.627384in}}%
\pgfpathlineto{\pgfqpoint{2.048763in}{0.630438in}}%
\pgfpathlineto{\pgfqpoint{2.083416in}{0.646976in}}%
\pgfpathlineto{\pgfqpoint{2.099318in}{0.651066in}}%
\pgfpathlineto{\pgfqpoint{2.124461in}{0.657579in}}%
\pgfpathlineto{\pgfqpoint{2.165507in}{0.659849in}}%
\pgfpathlineto{\pgfqpoint{2.206552in}{0.655858in}}%
\pgfpathlineto{\pgfqpoint{2.233727in}{0.651066in}}%
\pgfpathlineto{\pgfqpoint{2.247597in}{0.648632in}}%
\pgfpathlineto{\pgfqpoint{2.288643in}{0.641465in}}%
\pgfpathlineto{\pgfqpoint{2.329688in}{0.637073in}}%
\pgfpathlineto{\pgfqpoint{2.370733in}{0.637259in}}%
\pgfpathlineto{\pgfqpoint{2.411779in}{0.642517in}}%
\pgfpathlineto{\pgfqpoint{2.448212in}{0.651066in}}%
\pgfpathlineto{\pgfqpoint{2.452824in}{0.652174in}}%
\pgfpathlineto{\pgfqpoint{2.493869in}{0.665074in}}%
\pgfpathlineto{\pgfqpoint{2.495292in}{0.651066in}}%
\pgfpathlineto{\pgfqpoint{2.497361in}{0.630438in}}%
\pgfpathlineto{\pgfqpoint{2.499392in}{0.609810in}}%
\pgfpathlineto{\pgfqpoint{2.501389in}{0.589182in}}%
\pgfpathlineto{\pgfqpoint{2.503353in}{0.568554in}}%
\pgfpathlineto{\pgfqpoint{2.505285in}{0.547927in}}%
\pgfpathlineto{\pgfqpoint{2.507188in}{0.527299in}}%
\pgfpathlineto{\pgfqpoint{2.509062in}{0.506671in}}%
\pgfpathlineto{\pgfqpoint{2.510909in}{0.486043in}}%
\pgfpathlineto{\pgfqpoint{2.512730in}{0.465415in}}%
\pgfpathlineto{\pgfqpoint{2.514527in}{0.444787in}}%
\pgfpathlineto{\pgfqpoint{2.516300in}{0.424160in}}%
\pgfpathlineto{\pgfqpoint{2.518050in}{0.403532in}}%
\pgfpathlineto{\pgfqpoint{2.519780in}{0.382904in}}%
\pgfpathlineto{\pgfqpoint{2.493869in}{0.382904in}}%
\pgfpathlineto{\pgfqpoint{2.452824in}{0.382904in}}%
\pgfpathlineto{\pgfqpoint{2.411779in}{0.382904in}}%
\pgfpathlineto{\pgfqpoint{2.370733in}{0.382904in}}%
\pgfpathlineto{\pgfqpoint{2.329688in}{0.382904in}}%
\pgfpathlineto{\pgfqpoint{2.288643in}{0.382904in}}%
\pgfpathlineto{\pgfqpoint{2.247597in}{0.382904in}}%
\pgfpathlineto{\pgfqpoint{2.206552in}{0.382904in}}%
\pgfpathlineto{\pgfqpoint{2.165507in}{0.382904in}}%
\pgfpathlineto{\pgfqpoint{2.124461in}{0.382904in}}%
\pgfpathlineto{\pgfqpoint{2.083416in}{0.382904in}}%
\pgfpathlineto{\pgfqpoint{2.042371in}{0.382904in}}%
\pgfpathlineto{\pgfqpoint{2.001325in}{0.382904in}}%
\pgfpathlineto{\pgfqpoint{1.960280in}{0.382904in}}%
\pgfpathlineto{\pgfqpoint{1.919235in}{0.382904in}}%
\pgfpathlineto{\pgfqpoint{1.878189in}{0.382904in}}%
\pgfpathlineto{\pgfqpoint{1.837144in}{0.382904in}}%
\pgfpathlineto{\pgfqpoint{1.796099in}{0.382904in}}%
\pgfpathlineto{\pgfqpoint{1.755053in}{0.382904in}}%
\pgfpathlineto{\pgfqpoint{1.719020in}{0.382904in}}%
\pgfpathclose%
\pgfusepath{stroke,fill}%
\end{pgfscope}%
\begin{pgfscope}%
\pgfpathrectangle{\pgfqpoint{0.605784in}{0.382904in}}{\pgfqpoint{4.063488in}{2.042155in}}%
\pgfusepath{clip}%
\pgfsetbuttcap%
\pgfsetroundjoin%
\definecolor{currentfill}{rgb}{0.993248,0.906157,0.143936}%
\pgfsetfillcolor{currentfill}%
\pgfsetlinewidth{1.003750pt}%
\definecolor{currentstroke}{rgb}{0.993248,0.906157,0.143936}%
\pgfsetstrokecolor{currentstroke}%
\pgfsetdash{}{0pt}%
\pgfpathmoveto{\pgfqpoint{3.681018in}{2.404431in}}%
\pgfpathlineto{\pgfqpoint{3.676847in}{2.425059in}}%
\pgfpathlineto{\pgfqpoint{3.684184in}{2.425059in}}%
\pgfpathlineto{\pgfqpoint{3.725229in}{2.425059in}}%
\pgfpathlineto{\pgfqpoint{3.766275in}{2.425059in}}%
\pgfpathlineto{\pgfqpoint{3.807320in}{2.425059in}}%
\pgfpathlineto{\pgfqpoint{3.848365in}{2.425059in}}%
\pgfpathlineto{\pgfqpoint{3.889411in}{2.425059in}}%
\pgfpathlineto{\pgfqpoint{3.930456in}{2.425059in}}%
\pgfpathlineto{\pgfqpoint{3.971501in}{2.425059in}}%
\pgfpathlineto{\pgfqpoint{4.012547in}{2.425059in}}%
\pgfpathlineto{\pgfqpoint{4.053592in}{2.425059in}}%
\pgfpathlineto{\pgfqpoint{4.094637in}{2.425059in}}%
\pgfpathlineto{\pgfqpoint{4.135683in}{2.425059in}}%
\pgfpathlineto{\pgfqpoint{4.176728in}{2.425059in}}%
\pgfpathlineto{\pgfqpoint{4.217773in}{2.425059in}}%
\pgfpathlineto{\pgfqpoint{4.258819in}{2.425059in}}%
\pgfpathlineto{\pgfqpoint{4.299864in}{2.425059in}}%
\pgfpathlineto{\pgfqpoint{4.340909in}{2.425059in}}%
\pgfpathlineto{\pgfqpoint{4.381955in}{2.425059in}}%
\pgfpathlineto{\pgfqpoint{4.423000in}{2.425059in}}%
\pgfpathlineto{\pgfqpoint{4.464045in}{2.425059in}}%
\pgfpathlineto{\pgfqpoint{4.505091in}{2.425059in}}%
\pgfpathlineto{\pgfqpoint{4.546136in}{2.425059in}}%
\pgfpathlineto{\pgfqpoint{4.587181in}{2.425059in}}%
\pgfpathlineto{\pgfqpoint{4.628227in}{2.425059in}}%
\pgfpathlineto{\pgfqpoint{4.669272in}{2.425059in}}%
\pgfpathlineto{\pgfqpoint{4.669272in}{2.404431in}}%
\pgfpathlineto{\pgfqpoint{4.669272in}{2.389012in}}%
\pgfpathlineto{\pgfqpoint{4.628227in}{2.389012in}}%
\pgfpathlineto{\pgfqpoint{4.587181in}{2.389012in}}%
\pgfpathlineto{\pgfqpoint{4.546136in}{2.389012in}}%
\pgfpathlineto{\pgfqpoint{4.505091in}{2.389012in}}%
\pgfpathlineto{\pgfqpoint{4.464045in}{2.389012in}}%
\pgfpathlineto{\pgfqpoint{4.423000in}{2.389012in}}%
\pgfpathlineto{\pgfqpoint{4.381955in}{2.389012in}}%
\pgfpathlineto{\pgfqpoint{4.340909in}{2.389012in}}%
\pgfpathlineto{\pgfqpoint{4.299864in}{2.389012in}}%
\pgfpathlineto{\pgfqpoint{4.258819in}{2.389012in}}%
\pgfpathlineto{\pgfqpoint{4.217773in}{2.389012in}}%
\pgfpathlineto{\pgfqpoint{4.176728in}{2.389012in}}%
\pgfpathlineto{\pgfqpoint{4.135683in}{2.389012in}}%
\pgfpathlineto{\pgfqpoint{4.094637in}{2.389012in}}%
\pgfpathlineto{\pgfqpoint{4.053592in}{2.389012in}}%
\pgfpathlineto{\pgfqpoint{4.012547in}{2.389012in}}%
\pgfpathlineto{\pgfqpoint{3.971501in}{2.389012in}}%
\pgfpathlineto{\pgfqpoint{3.930456in}{2.389012in}}%
\pgfpathlineto{\pgfqpoint{3.889411in}{2.389012in}}%
\pgfpathlineto{\pgfqpoint{3.848365in}{2.389012in}}%
\pgfpathlineto{\pgfqpoint{3.807320in}{2.389012in}}%
\pgfpathlineto{\pgfqpoint{3.766275in}{2.389012in}}%
\pgfpathlineto{\pgfqpoint{3.725229in}{2.389012in}}%
\pgfpathlineto{\pgfqpoint{3.684184in}{2.389012in}}%
\pgfpathclose%
\pgfusepath{stroke,fill}%
\end{pgfscope}%
\begin{pgfscope}%
\pgfsetbuttcap%
\pgfsetroundjoin%
\definecolor{currentfill}{rgb}{0.000000,0.000000,0.000000}%
\pgfsetfillcolor{currentfill}%
\pgfsetlinewidth{0.803000pt}%
\definecolor{currentstroke}{rgb}{0.000000,0.000000,0.000000}%
\pgfsetstrokecolor{currentstroke}%
\pgfsetdash{}{0pt}%
\pgfsys@defobject{currentmarker}{\pgfqpoint{0.000000in}{-0.048611in}}{\pgfqpoint{0.000000in}{0.000000in}}{%
\pgfpathmoveto{\pgfqpoint{0.000000in}{0.000000in}}%
\pgfpathlineto{\pgfqpoint{0.000000in}{-0.048611in}}%
\pgfusepath{stroke,fill}%
}%
\begin{pgfscope}%
\pgfsys@transformshift{0.605784in}{0.382904in}%
\pgfsys@useobject{currentmarker}{}%
\end{pgfscope}%
\end{pgfscope}%
\begin{pgfscope}%
\definecolor{textcolor}{rgb}{0.000000,0.000000,0.000000}%
\pgfsetstrokecolor{textcolor}%
\pgfsetfillcolor{textcolor}%
\pgftext[x=0.605784in,y=0.285682in,,top]{\color{textcolor}\rmfamily\fontsize{10.000000}{12.000000}\selectfont \(\displaystyle {0}\)}%
\end{pgfscope}%
\begin{pgfscope}%
\pgfsetbuttcap%
\pgfsetroundjoin%
\definecolor{currentfill}{rgb}{0.000000,0.000000,0.000000}%
\pgfsetfillcolor{currentfill}%
\pgfsetlinewidth{0.803000pt}%
\definecolor{currentstroke}{rgb}{0.000000,0.000000,0.000000}%
\pgfsetstrokecolor{currentstroke}%
\pgfsetdash{}{0pt}%
\pgfsys@defobject{currentmarker}{\pgfqpoint{0.000000in}{-0.048611in}}{\pgfqpoint{0.000000in}{0.000000in}}{%
\pgfpathmoveto{\pgfqpoint{0.000000in}{0.000000in}}%
\pgfpathlineto{\pgfqpoint{0.000000in}{-0.048611in}}%
\pgfusepath{stroke,fill}%
}%
\begin{pgfscope}%
\pgfsys@transformshift{1.283032in}{0.382904in}%
\pgfsys@useobject{currentmarker}{}%
\end{pgfscope}%
\end{pgfscope}%
\begin{pgfscope}%
\definecolor{textcolor}{rgb}{0.000000,0.000000,0.000000}%
\pgfsetstrokecolor{textcolor}%
\pgfsetfillcolor{textcolor}%
\pgftext[x=1.283032in,y=0.285682in,,top]{\color{textcolor}\rmfamily\fontsize{10.000000}{12.000000}\selectfont \(\displaystyle {50}\)}%
\end{pgfscope}%
\begin{pgfscope}%
\pgfsetbuttcap%
\pgfsetroundjoin%
\definecolor{currentfill}{rgb}{0.000000,0.000000,0.000000}%
\pgfsetfillcolor{currentfill}%
\pgfsetlinewidth{0.803000pt}%
\definecolor{currentstroke}{rgb}{0.000000,0.000000,0.000000}%
\pgfsetstrokecolor{currentstroke}%
\pgfsetdash{}{0pt}%
\pgfsys@defobject{currentmarker}{\pgfqpoint{0.000000in}{-0.048611in}}{\pgfqpoint{0.000000in}{0.000000in}}{%
\pgfpathmoveto{\pgfqpoint{0.000000in}{0.000000in}}%
\pgfpathlineto{\pgfqpoint{0.000000in}{-0.048611in}}%
\pgfusepath{stroke,fill}%
}%
\begin{pgfscope}%
\pgfsys@transformshift{1.960280in}{0.382904in}%
\pgfsys@useobject{currentmarker}{}%
\end{pgfscope}%
\end{pgfscope}%
\begin{pgfscope}%
\definecolor{textcolor}{rgb}{0.000000,0.000000,0.000000}%
\pgfsetstrokecolor{textcolor}%
\pgfsetfillcolor{textcolor}%
\pgftext[x=1.960280in,y=0.285682in,,top]{\color{textcolor}\rmfamily\fontsize{10.000000}{12.000000}\selectfont \(\displaystyle {100}\)}%
\end{pgfscope}%
\begin{pgfscope}%
\pgfsetbuttcap%
\pgfsetroundjoin%
\definecolor{currentfill}{rgb}{0.000000,0.000000,0.000000}%
\pgfsetfillcolor{currentfill}%
\pgfsetlinewidth{0.803000pt}%
\definecolor{currentstroke}{rgb}{0.000000,0.000000,0.000000}%
\pgfsetstrokecolor{currentstroke}%
\pgfsetdash{}{0pt}%
\pgfsys@defobject{currentmarker}{\pgfqpoint{0.000000in}{-0.048611in}}{\pgfqpoint{0.000000in}{0.000000in}}{%
\pgfpathmoveto{\pgfqpoint{0.000000in}{0.000000in}}%
\pgfpathlineto{\pgfqpoint{0.000000in}{-0.048611in}}%
\pgfusepath{stroke,fill}%
}%
\begin{pgfscope}%
\pgfsys@transformshift{2.637528in}{0.382904in}%
\pgfsys@useobject{currentmarker}{}%
\end{pgfscope}%
\end{pgfscope}%
\begin{pgfscope}%
\definecolor{textcolor}{rgb}{0.000000,0.000000,0.000000}%
\pgfsetstrokecolor{textcolor}%
\pgfsetfillcolor{textcolor}%
\pgftext[x=2.637528in,y=0.285682in,,top]{\color{textcolor}\rmfamily\fontsize{10.000000}{12.000000}\selectfont \(\displaystyle {150}\)}%
\end{pgfscope}%
\begin{pgfscope}%
\pgfsetbuttcap%
\pgfsetroundjoin%
\definecolor{currentfill}{rgb}{0.000000,0.000000,0.000000}%
\pgfsetfillcolor{currentfill}%
\pgfsetlinewidth{0.803000pt}%
\definecolor{currentstroke}{rgb}{0.000000,0.000000,0.000000}%
\pgfsetstrokecolor{currentstroke}%
\pgfsetdash{}{0pt}%
\pgfsys@defobject{currentmarker}{\pgfqpoint{0.000000in}{-0.048611in}}{\pgfqpoint{0.000000in}{0.000000in}}{%
\pgfpathmoveto{\pgfqpoint{0.000000in}{0.000000in}}%
\pgfpathlineto{\pgfqpoint{0.000000in}{-0.048611in}}%
\pgfusepath{stroke,fill}%
}%
\begin{pgfscope}%
\pgfsys@transformshift{3.314776in}{0.382904in}%
\pgfsys@useobject{currentmarker}{}%
\end{pgfscope}%
\end{pgfscope}%
\begin{pgfscope}%
\definecolor{textcolor}{rgb}{0.000000,0.000000,0.000000}%
\pgfsetstrokecolor{textcolor}%
\pgfsetfillcolor{textcolor}%
\pgftext[x=3.314776in,y=0.285682in,,top]{\color{textcolor}\rmfamily\fontsize{10.000000}{12.000000}\selectfont \(\displaystyle {200}\)}%
\end{pgfscope}%
\begin{pgfscope}%
\pgfsetbuttcap%
\pgfsetroundjoin%
\definecolor{currentfill}{rgb}{0.000000,0.000000,0.000000}%
\pgfsetfillcolor{currentfill}%
\pgfsetlinewidth{0.803000pt}%
\definecolor{currentstroke}{rgb}{0.000000,0.000000,0.000000}%
\pgfsetstrokecolor{currentstroke}%
\pgfsetdash{}{0pt}%
\pgfsys@defobject{currentmarker}{\pgfqpoint{0.000000in}{-0.048611in}}{\pgfqpoint{0.000000in}{0.000000in}}{%
\pgfpathmoveto{\pgfqpoint{0.000000in}{0.000000in}}%
\pgfpathlineto{\pgfqpoint{0.000000in}{-0.048611in}}%
\pgfusepath{stroke,fill}%
}%
\begin{pgfscope}%
\pgfsys@transformshift{3.992024in}{0.382904in}%
\pgfsys@useobject{currentmarker}{}%
\end{pgfscope}%
\end{pgfscope}%
\begin{pgfscope}%
\definecolor{textcolor}{rgb}{0.000000,0.000000,0.000000}%
\pgfsetstrokecolor{textcolor}%
\pgfsetfillcolor{textcolor}%
\pgftext[x=3.992024in,y=0.285682in,,top]{\color{textcolor}\rmfamily\fontsize{10.000000}{12.000000}\selectfont \(\displaystyle {250}\)}%
\end{pgfscope}%
\begin{pgfscope}%
\pgfsetbuttcap%
\pgfsetroundjoin%
\definecolor{currentfill}{rgb}{0.000000,0.000000,0.000000}%
\pgfsetfillcolor{currentfill}%
\pgfsetlinewidth{0.803000pt}%
\definecolor{currentstroke}{rgb}{0.000000,0.000000,0.000000}%
\pgfsetstrokecolor{currentstroke}%
\pgfsetdash{}{0pt}%
\pgfsys@defobject{currentmarker}{\pgfqpoint{0.000000in}{-0.048611in}}{\pgfqpoint{0.000000in}{0.000000in}}{%
\pgfpathmoveto{\pgfqpoint{0.000000in}{0.000000in}}%
\pgfpathlineto{\pgfqpoint{0.000000in}{-0.048611in}}%
\pgfusepath{stroke,fill}%
}%
\begin{pgfscope}%
\pgfsys@transformshift{4.669272in}{0.382904in}%
\pgfsys@useobject{currentmarker}{}%
\end{pgfscope}%
\end{pgfscope}%
\begin{pgfscope}%
\definecolor{textcolor}{rgb}{0.000000,0.000000,0.000000}%
\pgfsetstrokecolor{textcolor}%
\pgfsetfillcolor{textcolor}%
\pgftext[x=4.669272in,y=0.285682in,,top]{\color{textcolor}\rmfamily\fontsize{10.000000}{12.000000}\selectfont \(\displaystyle {300}\)}%
\end{pgfscope}%
\begin{pgfscope}%
\definecolor{textcolor}{rgb}{0.000000,0.000000,0.000000}%
\pgfsetstrokecolor{textcolor}%
\pgfsetfillcolor{textcolor}%
\pgftext[x=2.637528in,y=0.106793in,,top]{\color{textcolor}\rmfamily\fontsize{10.000000}{12.000000}\selectfont \(\displaystyle t\)}%
\end{pgfscope}%
\begin{pgfscope}%
\pgfsetbuttcap%
\pgfsetroundjoin%
\definecolor{currentfill}{rgb}{0.000000,0.000000,0.000000}%
\pgfsetfillcolor{currentfill}%
\pgfsetlinewidth{0.803000pt}%
\definecolor{currentstroke}{rgb}{0.000000,0.000000,0.000000}%
\pgfsetstrokecolor{currentstroke}%
\pgfsetdash{}{0pt}%
\pgfsys@defobject{currentmarker}{\pgfqpoint{-0.048611in}{0.000000in}}{\pgfqpoint{-0.000000in}{0.000000in}}{%
\pgfpathmoveto{\pgfqpoint{-0.000000in}{0.000000in}}%
\pgfpathlineto{\pgfqpoint{-0.048611in}{0.000000in}}%
\pgfusepath{stroke,fill}%
}%
\begin{pgfscope}%
\pgfsys@transformshift{0.605784in}{0.723263in}%
\pgfsys@useobject{currentmarker}{}%
\end{pgfscope}%
\end{pgfscope}%
\begin{pgfscope}%
\definecolor{textcolor}{rgb}{0.000000,0.000000,0.000000}%
\pgfsetstrokecolor{textcolor}%
\pgfsetfillcolor{textcolor}%
\pgftext[x=0.331092in, y=0.675069in, left, base]{\color{textcolor}\rmfamily\fontsize{10.000000}{12.000000}\selectfont \(\displaystyle {\ensuremath{-}5}\)}%
\end{pgfscope}%
\begin{pgfscope}%
\pgfsetbuttcap%
\pgfsetroundjoin%
\definecolor{currentfill}{rgb}{0.000000,0.000000,0.000000}%
\pgfsetfillcolor{currentfill}%
\pgfsetlinewidth{0.803000pt}%
\definecolor{currentstroke}{rgb}{0.000000,0.000000,0.000000}%
\pgfsetstrokecolor{currentstroke}%
\pgfsetdash{}{0pt}%
\pgfsys@defobject{currentmarker}{\pgfqpoint{-0.048611in}{0.000000in}}{\pgfqpoint{-0.000000in}{0.000000in}}{%
\pgfpathmoveto{\pgfqpoint{-0.000000in}{0.000000in}}%
\pgfpathlineto{\pgfqpoint{-0.048611in}{0.000000in}}%
\pgfusepath{stroke,fill}%
}%
\begin{pgfscope}%
\pgfsys@transformshift{0.605784in}{1.290528in}%
\pgfsys@useobject{currentmarker}{}%
\end{pgfscope}%
\end{pgfscope}%
\begin{pgfscope}%
\definecolor{textcolor}{rgb}{0.000000,0.000000,0.000000}%
\pgfsetstrokecolor{textcolor}%
\pgfsetfillcolor{textcolor}%
\pgftext[x=0.439117in, y=1.242334in, left, base]{\color{textcolor}\rmfamily\fontsize{10.000000}{12.000000}\selectfont \(\displaystyle {0}\)}%
\end{pgfscope}%
\begin{pgfscope}%
\pgfsetbuttcap%
\pgfsetroundjoin%
\definecolor{currentfill}{rgb}{0.000000,0.000000,0.000000}%
\pgfsetfillcolor{currentfill}%
\pgfsetlinewidth{0.803000pt}%
\definecolor{currentstroke}{rgb}{0.000000,0.000000,0.000000}%
\pgfsetstrokecolor{currentstroke}%
\pgfsetdash{}{0pt}%
\pgfsys@defobject{currentmarker}{\pgfqpoint{-0.048611in}{0.000000in}}{\pgfqpoint{-0.000000in}{0.000000in}}{%
\pgfpathmoveto{\pgfqpoint{-0.000000in}{0.000000in}}%
\pgfpathlineto{\pgfqpoint{-0.048611in}{0.000000in}}%
\pgfusepath{stroke,fill}%
}%
\begin{pgfscope}%
\pgfsys@transformshift{0.605784in}{1.857793in}%
\pgfsys@useobject{currentmarker}{}%
\end{pgfscope}%
\end{pgfscope}%
\begin{pgfscope}%
\definecolor{textcolor}{rgb}{0.000000,0.000000,0.000000}%
\pgfsetstrokecolor{textcolor}%
\pgfsetfillcolor{textcolor}%
\pgftext[x=0.439117in, y=1.809599in, left, base]{\color{textcolor}\rmfamily\fontsize{10.000000}{12.000000}\selectfont \(\displaystyle {5}\)}%
\end{pgfscope}%
\begin{pgfscope}%
\pgfsetbuttcap%
\pgfsetroundjoin%
\definecolor{currentfill}{rgb}{0.000000,0.000000,0.000000}%
\pgfsetfillcolor{currentfill}%
\pgfsetlinewidth{0.803000pt}%
\definecolor{currentstroke}{rgb}{0.000000,0.000000,0.000000}%
\pgfsetstrokecolor{currentstroke}%
\pgfsetdash{}{0pt}%
\pgfsys@defobject{currentmarker}{\pgfqpoint{-0.048611in}{0.000000in}}{\pgfqpoint{-0.000000in}{0.000000in}}{%
\pgfpathmoveto{\pgfqpoint{-0.000000in}{0.000000in}}%
\pgfpathlineto{\pgfqpoint{-0.048611in}{0.000000in}}%
\pgfusepath{stroke,fill}%
}%
\begin{pgfscope}%
\pgfsys@transformshift{0.605784in}{2.425059in}%
\pgfsys@useobject{currentmarker}{}%
\end{pgfscope}%
\end{pgfscope}%
\begin{pgfscope}%
\definecolor{textcolor}{rgb}{0.000000,0.000000,0.000000}%
\pgfsetstrokecolor{textcolor}%
\pgfsetfillcolor{textcolor}%
\pgftext[x=0.369672in, y=2.376864in, left, base]{\color{textcolor}\rmfamily\fontsize{10.000000}{12.000000}\selectfont \(\displaystyle {10}\)}%
\end{pgfscope}%
\begin{pgfscope}%
\definecolor{textcolor}{rgb}{0.000000,0.000000,0.000000}%
\pgfsetstrokecolor{textcolor}%
\pgfsetfillcolor{textcolor}%
\pgftext[x=0.275536in,y=1.403981in,,bottom,rotate=90.000000]{\color{textcolor}\rmfamily\fontsize{10.000000}{12.000000}\selectfont \(\displaystyle x\)}%
\end{pgfscope}%
\begin{pgfscope}%
\pgfpathrectangle{\pgfqpoint{0.605784in}{0.382904in}}{\pgfqpoint{4.063488in}{2.042155in}}%
\pgfusepath{clip}%
\pgfsetrectcap%
\pgfsetroundjoin%
\pgfsetlinewidth{1.505625pt}%
\definecolor{currentstroke}{rgb}{0.000000,0.000000,0.000000}%
\pgfsetstrokecolor{currentstroke}%
\pgfsetdash{}{0pt}%
\pgfpathmoveto{\pgfqpoint{0.605784in}{0.723263in}}%
\pgfpathlineto{\pgfqpoint{0.646829in}{0.723263in}}%
\pgfpathlineto{\pgfqpoint{0.687875in}{0.723263in}}%
\pgfpathlineto{\pgfqpoint{0.728920in}{0.723263in}}%
\pgfpathlineto{\pgfqpoint{0.769965in}{0.723263in}}%
\pgfpathlineto{\pgfqpoint{0.811011in}{0.723263in}}%
\pgfpathlineto{\pgfqpoint{0.852056in}{0.723263in}}%
\pgfpathlineto{\pgfqpoint{0.893101in}{0.723263in}}%
\pgfpathlineto{\pgfqpoint{0.934147in}{0.723263in}}%
\pgfpathlineto{\pgfqpoint{0.975192in}{0.723263in}}%
\pgfpathlineto{\pgfqpoint{1.016237in}{0.723263in}}%
\pgfpathlineto{\pgfqpoint{1.057283in}{0.723263in}}%
\pgfpathlineto{\pgfqpoint{1.098328in}{0.723263in}}%
\pgfpathlineto{\pgfqpoint{1.139373in}{0.723263in}}%
\pgfpathlineto{\pgfqpoint{1.180419in}{0.723263in}}%
\pgfpathlineto{\pgfqpoint{1.221464in}{0.723263in}}%
\pgfpathlineto{\pgfqpoint{1.262509in}{0.723263in}}%
\pgfpathlineto{\pgfqpoint{1.303555in}{0.723263in}}%
\pgfpathlineto{\pgfqpoint{1.344600in}{0.723263in}}%
\pgfpathlineto{\pgfqpoint{1.385645in}{0.723263in}}%
\pgfpathlineto{\pgfqpoint{1.426691in}{0.723263in}}%
\pgfpathlineto{\pgfqpoint{1.467736in}{0.723263in}}%
\pgfpathlineto{\pgfqpoint{1.508781in}{0.723263in}}%
\pgfpathlineto{\pgfqpoint{1.549827in}{0.723263in}}%
\pgfpathlineto{\pgfqpoint{1.590872in}{0.723263in}}%
\pgfpathlineto{\pgfqpoint{1.631917in}{0.723263in}}%
\pgfpathlineto{\pgfqpoint{1.672963in}{0.723263in}}%
\pgfpathlineto{\pgfqpoint{1.714008in}{0.723263in}}%
\pgfpathlineto{\pgfqpoint{1.755053in}{0.723263in}}%
\pgfpathlineto{\pgfqpoint{1.796099in}{0.723263in}}%
\pgfpathlineto{\pgfqpoint{1.837144in}{0.723263in}}%
\pgfpathlineto{\pgfqpoint{1.878189in}{0.723263in}}%
\pgfpathlineto{\pgfqpoint{1.919235in}{0.723263in}}%
\pgfpathlineto{\pgfqpoint{1.960280in}{0.723263in}}%
\pgfpathlineto{\pgfqpoint{2.001325in}{0.723263in}}%
\pgfpathlineto{\pgfqpoint{2.042371in}{0.723263in}}%
\pgfpathlineto{\pgfqpoint{2.083416in}{0.723263in}}%
\pgfpathlineto{\pgfqpoint{2.124461in}{0.723263in}}%
\pgfpathlineto{\pgfqpoint{2.165507in}{0.723263in}}%
\pgfpathlineto{\pgfqpoint{2.206552in}{0.723263in}}%
\pgfpathlineto{\pgfqpoint{2.247597in}{0.723263in}}%
\pgfpathlineto{\pgfqpoint{2.288643in}{0.723263in}}%
\pgfpathlineto{\pgfqpoint{2.329688in}{0.723263in}}%
\pgfpathlineto{\pgfqpoint{2.370733in}{0.723263in}}%
\pgfpathlineto{\pgfqpoint{2.411779in}{0.723263in}}%
\pgfpathlineto{\pgfqpoint{2.452824in}{0.723263in}}%
\pgfpathlineto{\pgfqpoint{2.493869in}{0.723263in}}%
\pgfpathlineto{\pgfqpoint{2.534915in}{0.723263in}}%
\pgfpathlineto{\pgfqpoint{2.575960in}{0.723263in}}%
\pgfpathlineto{\pgfqpoint{2.617005in}{0.723263in}}%
\pgfpathlineto{\pgfqpoint{2.658051in}{0.723263in}}%
\pgfpathlineto{\pgfqpoint{2.699096in}{0.723263in}}%
\pgfpathlineto{\pgfqpoint{2.740141in}{0.723263in}}%
\pgfpathlineto{\pgfqpoint{2.781187in}{0.723263in}}%
\pgfpathlineto{\pgfqpoint{2.822232in}{0.723263in}}%
\pgfpathlineto{\pgfqpoint{2.863277in}{0.723263in}}%
\pgfpathlineto{\pgfqpoint{2.904323in}{0.723263in}}%
\pgfpathlineto{\pgfqpoint{2.945368in}{0.723263in}}%
\pgfpathlineto{\pgfqpoint{2.986413in}{0.723263in}}%
\pgfpathlineto{\pgfqpoint{3.027459in}{0.723263in}}%
\pgfpathlineto{\pgfqpoint{3.068504in}{0.723263in}}%
\pgfpathlineto{\pgfqpoint{3.109549in}{0.723263in}}%
\pgfpathlineto{\pgfqpoint{3.150595in}{0.723263in}}%
\pgfpathlineto{\pgfqpoint{3.191640in}{0.723263in}}%
\pgfpathlineto{\pgfqpoint{3.232685in}{0.723263in}}%
\pgfpathlineto{\pgfqpoint{3.273731in}{0.723263in}}%
\pgfpathlineto{\pgfqpoint{3.314776in}{0.723263in}}%
\pgfpathlineto{\pgfqpoint{3.355821in}{0.723263in}}%
\pgfpathlineto{\pgfqpoint{3.396867in}{0.723263in}}%
\pgfpathlineto{\pgfqpoint{3.437912in}{0.723263in}}%
\pgfpathlineto{\pgfqpoint{3.478957in}{0.723263in}}%
\pgfpathlineto{\pgfqpoint{3.520003in}{0.723263in}}%
\pgfpathlineto{\pgfqpoint{3.561048in}{0.723263in}}%
\pgfpathlineto{\pgfqpoint{3.602093in}{0.723263in}}%
\pgfpathlineto{\pgfqpoint{3.643139in}{0.723263in}}%
\pgfpathlineto{\pgfqpoint{3.684184in}{0.723263in}}%
\pgfpathlineto{\pgfqpoint{3.725229in}{0.723263in}}%
\pgfpathlineto{\pgfqpoint{3.766275in}{0.723263in}}%
\pgfpathlineto{\pgfqpoint{3.807320in}{0.723263in}}%
\pgfpathlineto{\pgfqpoint{3.848365in}{0.723263in}}%
\pgfpathlineto{\pgfqpoint{3.889411in}{0.723263in}}%
\pgfpathlineto{\pgfqpoint{3.930456in}{0.723263in}}%
\pgfpathlineto{\pgfqpoint{3.971501in}{0.723263in}}%
\pgfpathlineto{\pgfqpoint{4.012547in}{0.723263in}}%
\pgfpathlineto{\pgfqpoint{4.053592in}{0.723263in}}%
\pgfpathlineto{\pgfqpoint{4.094637in}{0.723263in}}%
\pgfpathlineto{\pgfqpoint{4.135683in}{0.723263in}}%
\pgfpathlineto{\pgfqpoint{4.176728in}{0.723263in}}%
\pgfpathlineto{\pgfqpoint{4.217773in}{0.723263in}}%
\pgfpathlineto{\pgfqpoint{4.258819in}{0.723263in}}%
\pgfpathlineto{\pgfqpoint{4.299864in}{0.723263in}}%
\pgfpathlineto{\pgfqpoint{4.340909in}{0.723263in}}%
\pgfpathlineto{\pgfqpoint{4.381955in}{0.723263in}}%
\pgfpathlineto{\pgfqpoint{4.423000in}{0.723263in}}%
\pgfpathlineto{\pgfqpoint{4.464045in}{0.723263in}}%
\pgfpathlineto{\pgfqpoint{4.505091in}{0.723263in}}%
\pgfpathlineto{\pgfqpoint{4.546136in}{0.723263in}}%
\pgfpathlineto{\pgfqpoint{4.587181in}{0.723263in}}%
\pgfpathlineto{\pgfqpoint{4.628227in}{0.723263in}}%
\pgfpathlineto{\pgfqpoint{4.669272in}{0.723263in}}%
\pgfusepath{stroke}%
\end{pgfscope}%
\begin{pgfscope}%
\pgfpathrectangle{\pgfqpoint{0.605784in}{0.382904in}}{\pgfqpoint{4.063488in}{2.042155in}}%
\pgfusepath{clip}%
\pgfsetrectcap%
\pgfsetroundjoin%
\pgfsetlinewidth{1.505625pt}%
\definecolor{currentstroke}{rgb}{0.000000,0.000000,0.000000}%
\pgfsetstrokecolor{currentstroke}%
\pgfsetdash{}{0pt}%
\pgfpathmoveto{\pgfqpoint{0.605784in}{2.311606in}}%
\pgfpathlineto{\pgfqpoint{0.646829in}{2.311606in}}%
\pgfpathlineto{\pgfqpoint{0.687875in}{2.311606in}}%
\pgfpathlineto{\pgfqpoint{0.728920in}{2.311606in}}%
\pgfpathlineto{\pgfqpoint{0.769965in}{2.311606in}}%
\pgfpathlineto{\pgfqpoint{0.811011in}{2.311606in}}%
\pgfpathlineto{\pgfqpoint{0.852056in}{2.311606in}}%
\pgfpathlineto{\pgfqpoint{0.893101in}{2.311606in}}%
\pgfpathlineto{\pgfqpoint{0.934147in}{2.311606in}}%
\pgfpathlineto{\pgfqpoint{0.975192in}{2.311606in}}%
\pgfpathlineto{\pgfqpoint{1.016237in}{2.311606in}}%
\pgfpathlineto{\pgfqpoint{1.057283in}{2.311606in}}%
\pgfpathlineto{\pgfqpoint{1.098328in}{2.311606in}}%
\pgfpathlineto{\pgfqpoint{1.139373in}{2.311606in}}%
\pgfpathlineto{\pgfqpoint{1.180419in}{2.311606in}}%
\pgfpathlineto{\pgfqpoint{1.221464in}{2.311606in}}%
\pgfpathlineto{\pgfqpoint{1.262509in}{2.311606in}}%
\pgfpathlineto{\pgfqpoint{1.303555in}{2.311606in}}%
\pgfpathlineto{\pgfqpoint{1.344600in}{2.311606in}}%
\pgfpathlineto{\pgfqpoint{1.385645in}{2.311606in}}%
\pgfpathlineto{\pgfqpoint{1.426691in}{2.311606in}}%
\pgfpathlineto{\pgfqpoint{1.467736in}{2.311606in}}%
\pgfpathlineto{\pgfqpoint{1.508781in}{2.311606in}}%
\pgfpathlineto{\pgfqpoint{1.549827in}{2.311606in}}%
\pgfpathlineto{\pgfqpoint{1.590872in}{2.311606in}}%
\pgfpathlineto{\pgfqpoint{1.631917in}{2.311606in}}%
\pgfpathlineto{\pgfqpoint{1.672963in}{2.311606in}}%
\pgfpathlineto{\pgfqpoint{1.714008in}{2.311606in}}%
\pgfpathlineto{\pgfqpoint{1.755053in}{2.311606in}}%
\pgfpathlineto{\pgfqpoint{1.796099in}{2.311606in}}%
\pgfpathlineto{\pgfqpoint{1.837144in}{2.311606in}}%
\pgfpathlineto{\pgfqpoint{1.878189in}{2.311606in}}%
\pgfpathlineto{\pgfqpoint{1.919235in}{2.311606in}}%
\pgfpathlineto{\pgfqpoint{1.960280in}{2.311606in}}%
\pgfpathlineto{\pgfqpoint{2.001325in}{2.311606in}}%
\pgfpathlineto{\pgfqpoint{2.042371in}{2.311606in}}%
\pgfpathlineto{\pgfqpoint{2.083416in}{2.311606in}}%
\pgfpathlineto{\pgfqpoint{2.124461in}{2.311606in}}%
\pgfpathlineto{\pgfqpoint{2.165507in}{2.311606in}}%
\pgfpathlineto{\pgfqpoint{2.206552in}{2.311606in}}%
\pgfpathlineto{\pgfqpoint{2.247597in}{2.311606in}}%
\pgfpathlineto{\pgfqpoint{2.288643in}{2.311606in}}%
\pgfpathlineto{\pgfqpoint{2.329688in}{2.311606in}}%
\pgfpathlineto{\pgfqpoint{2.370733in}{2.311606in}}%
\pgfpathlineto{\pgfqpoint{2.411779in}{2.311606in}}%
\pgfpathlineto{\pgfqpoint{2.452824in}{2.311606in}}%
\pgfpathlineto{\pgfqpoint{2.493869in}{2.311606in}}%
\pgfpathlineto{\pgfqpoint{2.534915in}{2.311606in}}%
\pgfpathlineto{\pgfqpoint{2.575960in}{2.311606in}}%
\pgfpathlineto{\pgfqpoint{2.617005in}{2.311606in}}%
\pgfpathlineto{\pgfqpoint{2.658051in}{2.311606in}}%
\pgfpathlineto{\pgfqpoint{2.699096in}{2.311606in}}%
\pgfpathlineto{\pgfqpoint{2.740141in}{2.311606in}}%
\pgfpathlineto{\pgfqpoint{2.781187in}{2.311606in}}%
\pgfpathlineto{\pgfqpoint{2.822232in}{2.311606in}}%
\pgfpathlineto{\pgfqpoint{2.863277in}{2.311606in}}%
\pgfpathlineto{\pgfqpoint{2.904323in}{2.311606in}}%
\pgfpathlineto{\pgfqpoint{2.945368in}{2.311606in}}%
\pgfpathlineto{\pgfqpoint{2.986413in}{2.311606in}}%
\pgfpathlineto{\pgfqpoint{3.027459in}{2.311606in}}%
\pgfpathlineto{\pgfqpoint{3.068504in}{2.311606in}}%
\pgfpathlineto{\pgfqpoint{3.109549in}{2.311606in}}%
\pgfpathlineto{\pgfqpoint{3.150595in}{2.311606in}}%
\pgfpathlineto{\pgfqpoint{3.191640in}{2.311606in}}%
\pgfpathlineto{\pgfqpoint{3.232685in}{2.311606in}}%
\pgfpathlineto{\pgfqpoint{3.273731in}{2.311606in}}%
\pgfpathlineto{\pgfqpoint{3.314776in}{2.311606in}}%
\pgfpathlineto{\pgfqpoint{3.355821in}{2.311606in}}%
\pgfpathlineto{\pgfqpoint{3.396867in}{2.311606in}}%
\pgfpathlineto{\pgfqpoint{3.437912in}{2.311606in}}%
\pgfpathlineto{\pgfqpoint{3.478957in}{2.311606in}}%
\pgfpathlineto{\pgfqpoint{3.520003in}{2.311606in}}%
\pgfpathlineto{\pgfqpoint{3.561048in}{2.311606in}}%
\pgfpathlineto{\pgfqpoint{3.602093in}{2.311606in}}%
\pgfpathlineto{\pgfqpoint{3.643139in}{2.311606in}}%
\pgfpathlineto{\pgfqpoint{3.684184in}{2.311606in}}%
\pgfpathlineto{\pgfqpoint{3.725229in}{2.311606in}}%
\pgfpathlineto{\pgfqpoint{3.766275in}{2.311606in}}%
\pgfpathlineto{\pgfqpoint{3.807320in}{2.311606in}}%
\pgfpathlineto{\pgfqpoint{3.848365in}{2.311606in}}%
\pgfpathlineto{\pgfqpoint{3.889411in}{2.311606in}}%
\pgfpathlineto{\pgfqpoint{3.930456in}{2.311606in}}%
\pgfpathlineto{\pgfqpoint{3.971501in}{2.311606in}}%
\pgfpathlineto{\pgfqpoint{4.012547in}{2.311606in}}%
\pgfpathlineto{\pgfqpoint{4.053592in}{2.311606in}}%
\pgfpathlineto{\pgfqpoint{4.094637in}{2.311606in}}%
\pgfpathlineto{\pgfqpoint{4.135683in}{2.311606in}}%
\pgfpathlineto{\pgfqpoint{4.176728in}{2.311606in}}%
\pgfpathlineto{\pgfqpoint{4.217773in}{2.311606in}}%
\pgfpathlineto{\pgfqpoint{4.258819in}{2.311606in}}%
\pgfpathlineto{\pgfqpoint{4.299864in}{2.311606in}}%
\pgfpathlineto{\pgfqpoint{4.340909in}{2.311606in}}%
\pgfpathlineto{\pgfqpoint{4.381955in}{2.311606in}}%
\pgfpathlineto{\pgfqpoint{4.423000in}{2.311606in}}%
\pgfpathlineto{\pgfqpoint{4.464045in}{2.311606in}}%
\pgfpathlineto{\pgfqpoint{4.505091in}{2.311606in}}%
\pgfpathlineto{\pgfqpoint{4.546136in}{2.311606in}}%
\pgfpathlineto{\pgfqpoint{4.587181in}{2.311606in}}%
\pgfpathlineto{\pgfqpoint{4.628227in}{2.311606in}}%
\pgfpathlineto{\pgfqpoint{4.669272in}{2.311606in}}%
\pgfusepath{stroke}%
\end{pgfscope}%
\begin{pgfscope}%
\pgfpathrectangle{\pgfqpoint{0.605784in}{0.382904in}}{\pgfqpoint{4.063488in}{2.042155in}}%
\pgfusepath{clip}%
\pgfsetrectcap%
\pgfsetroundjoin%
\pgfsetlinewidth{1.003750pt}%
\definecolor{currentstroke}{rgb}{1.000000,1.000000,1.000000}%
\pgfsetstrokecolor{currentstroke}%
\pgfsetdash{}{0pt}%
\pgfpathmoveto{\pgfqpoint{0.605784in}{1.290528in}}%
\pgfpathlineto{\pgfqpoint{0.646829in}{1.270424in}}%
\pgfpathlineto{\pgfqpoint{0.687875in}{1.253405in}}%
\pgfpathlineto{\pgfqpoint{0.728920in}{1.242276in}}%
\pgfpathlineto{\pgfqpoint{0.769965in}{1.239303in}}%
\pgfpathlineto{\pgfqpoint{0.811011in}{1.246010in}}%
\pgfpathlineto{\pgfqpoint{0.852056in}{1.263041in}}%
\pgfpathlineto{\pgfqpoint{0.893101in}{1.290095in}}%
\pgfpathlineto{\pgfqpoint{0.934147in}{1.325960in}}%
\pgfpathlineto{\pgfqpoint{0.975192in}{1.368623in}}%
\pgfpathlineto{\pgfqpoint{1.016237in}{1.415447in}}%
\pgfpathlineto{\pgfqpoint{1.057283in}{1.463420in}}%
\pgfpathlineto{\pgfqpoint{1.098328in}{1.509422in}}%
\pgfpathlineto{\pgfqpoint{1.139373in}{1.550515in}}%
\pgfpathlineto{\pgfqpoint{1.180419in}{1.584208in}}%
\pgfpathlineto{\pgfqpoint{1.221464in}{1.608682in}}%
\pgfpathlineto{\pgfqpoint{1.262509in}{1.622962in}}%
\pgfpathlineto{\pgfqpoint{1.303555in}{1.626999in}}%
\pgfpathlineto{\pgfqpoint{1.344600in}{1.621679in}}%
\pgfpathlineto{\pgfqpoint{1.385645in}{1.608738in}}%
\pgfpathlineto{\pgfqpoint{1.426691in}{1.590610in}}%
\pgfpathlineto{\pgfqpoint{1.467736in}{1.570200in}}%
\pgfpathlineto{\pgfqpoint{1.508781in}{1.550621in}}%
\pgfpathlineto{\pgfqpoint{1.549827in}{1.534910in}}%
\pgfpathlineto{\pgfqpoint{1.590872in}{1.525752in}}%
\pgfpathlineto{\pgfqpoint{1.631917in}{1.525235in}}%
\pgfpathlineto{\pgfqpoint{1.672963in}{1.534659in}}%
\pgfpathlineto{\pgfqpoint{1.714008in}{1.554418in}}%
\pgfpathlineto{\pgfqpoint{1.755053in}{1.583966in}}%
\pgfpathlineto{\pgfqpoint{1.796099in}{1.621862in}}%
\pgfpathlineto{\pgfqpoint{1.837144in}{1.665906in}}%
\pgfpathlineto{\pgfqpoint{1.878189in}{1.713338in}}%
\pgfpathlineto{\pgfqpoint{1.919235in}{1.761090in}}%
\pgfpathlineto{\pgfqpoint{1.960280in}{1.806061in}}%
\pgfpathlineto{\pgfqpoint{2.001325in}{1.845409in}}%
\pgfpathlineto{\pgfqpoint{2.042371in}{1.876799in}}%
\pgfpathlineto{\pgfqpoint{2.083416in}{1.898625in}}%
\pgfpathlineto{\pgfqpoint{2.124461in}{1.910151in}}%
\pgfpathlineto{\pgfqpoint{2.165507in}{1.911580in}}%
\pgfpathlineto{\pgfqpoint{2.206552in}{1.904035in}}%
\pgfpathlineto{\pgfqpoint{2.247597in}{1.889457in}}%
\pgfpathlineto{\pgfqpoint{2.288643in}{1.870426in}}%
\pgfpathlineto{\pgfqpoint{2.329688in}{1.849931in}}%
\pgfpathlineto{\pgfqpoint{2.370733in}{1.831093in}}%
\pgfpathlineto{\pgfqpoint{2.411779in}{1.816880in}}%
\pgfpathlineto{\pgfqpoint{2.452824in}{1.809842in}}%
\pgfpathlineto{\pgfqpoint{2.493869in}{1.811873in}}%
\pgfpathlineto{\pgfqpoint{2.534915in}{1.399321in}}%
\pgfpathlineto{\pgfqpoint{2.575960in}{1.399321in}}%
\pgfpathlineto{\pgfqpoint{2.617005in}{1.399321in}}%
\pgfpathlineto{\pgfqpoint{2.658051in}{1.399321in}}%
\pgfpathlineto{\pgfqpoint{2.699096in}{1.399321in}}%
\pgfpathlineto{\pgfqpoint{2.740141in}{1.399321in}}%
\pgfpathlineto{\pgfqpoint{2.781187in}{1.399321in}}%
\pgfpathlineto{\pgfqpoint{2.822232in}{1.399321in}}%
\pgfpathlineto{\pgfqpoint{2.863277in}{1.399321in}}%
\pgfpathlineto{\pgfqpoint{2.904323in}{1.399321in}}%
\pgfpathlineto{\pgfqpoint{2.945368in}{1.399321in}}%
\pgfpathlineto{\pgfqpoint{2.986413in}{1.399321in}}%
\pgfpathlineto{\pgfqpoint{3.027459in}{1.399321in}}%
\pgfpathlineto{\pgfqpoint{3.068504in}{1.399321in}}%
\pgfpathlineto{\pgfqpoint{3.109549in}{1.399321in}}%
\pgfpathlineto{\pgfqpoint{3.150595in}{1.399321in}}%
\pgfpathlineto{\pgfqpoint{3.191640in}{1.399321in}}%
\pgfpathlineto{\pgfqpoint{3.232685in}{1.399321in}}%
\pgfpathlineto{\pgfqpoint{3.273731in}{1.399321in}}%
\pgfpathlineto{\pgfqpoint{3.314776in}{1.399321in}}%
\pgfpathlineto{\pgfqpoint{3.355821in}{1.399321in}}%
\pgfpathlineto{\pgfqpoint{3.396867in}{1.399321in}}%
\pgfpathlineto{\pgfqpoint{3.437912in}{1.399321in}}%
\pgfpathlineto{\pgfqpoint{3.478957in}{1.399321in}}%
\pgfpathlineto{\pgfqpoint{3.520003in}{1.399321in}}%
\pgfpathlineto{\pgfqpoint{3.561048in}{1.399321in}}%
\pgfpathlineto{\pgfqpoint{3.602093in}{1.399321in}}%
\pgfpathlineto{\pgfqpoint{3.643139in}{1.399321in}}%
\pgfpathlineto{\pgfqpoint{3.684184in}{1.181735in}}%
\pgfpathlineto{\pgfqpoint{3.725229in}{1.181735in}}%
\pgfpathlineto{\pgfqpoint{3.766275in}{1.181735in}}%
\pgfpathlineto{\pgfqpoint{3.807320in}{1.181735in}}%
\pgfpathlineto{\pgfqpoint{3.848365in}{1.181735in}}%
\pgfpathlineto{\pgfqpoint{3.889411in}{1.181735in}}%
\pgfpathlineto{\pgfqpoint{3.930456in}{1.181735in}}%
\pgfpathlineto{\pgfqpoint{3.971501in}{1.181735in}}%
\pgfpathlineto{\pgfqpoint{4.012547in}{1.181735in}}%
\pgfpathlineto{\pgfqpoint{4.053592in}{1.181735in}}%
\pgfpathlineto{\pgfqpoint{4.094637in}{1.181735in}}%
\pgfpathlineto{\pgfqpoint{4.135683in}{1.181735in}}%
\pgfpathlineto{\pgfqpoint{4.176728in}{1.181735in}}%
\pgfpathlineto{\pgfqpoint{4.217773in}{1.181735in}}%
\pgfpathlineto{\pgfqpoint{4.258819in}{1.181735in}}%
\pgfpathlineto{\pgfqpoint{4.299864in}{1.181735in}}%
\pgfpathlineto{\pgfqpoint{4.340909in}{1.181735in}}%
\pgfpathlineto{\pgfqpoint{4.381955in}{1.181735in}}%
\pgfpathlineto{\pgfqpoint{4.423000in}{1.181735in}}%
\pgfpathlineto{\pgfqpoint{4.464045in}{1.181735in}}%
\pgfpathlineto{\pgfqpoint{4.505091in}{1.181735in}}%
\pgfpathlineto{\pgfqpoint{4.546136in}{1.181735in}}%
\pgfpathlineto{\pgfqpoint{4.587181in}{1.181735in}}%
\pgfpathlineto{\pgfqpoint{4.628227in}{1.181735in}}%
\pgfpathlineto{\pgfqpoint{4.669272in}{1.181735in}}%
\pgfusepath{stroke}%
\end{pgfscope}%
\begin{pgfscope}%
\pgfsetrectcap%
\pgfsetmiterjoin%
\pgfsetlinewidth{0.803000pt}%
\definecolor{currentstroke}{rgb}{0.000000,0.000000,0.000000}%
\pgfsetstrokecolor{currentstroke}%
\pgfsetdash{}{0pt}%
\pgfpathmoveto{\pgfqpoint{0.605784in}{0.382904in}}%
\pgfpathlineto{\pgfqpoint{0.605784in}{2.425059in}}%
\pgfusepath{stroke}%
\end{pgfscope}%
\begin{pgfscope}%
\pgfsetrectcap%
\pgfsetmiterjoin%
\pgfsetlinewidth{0.803000pt}%
\definecolor{currentstroke}{rgb}{0.000000,0.000000,0.000000}%
\pgfsetstrokecolor{currentstroke}%
\pgfsetdash{}{0pt}%
\pgfpathmoveto{\pgfqpoint{4.669272in}{0.382904in}}%
\pgfpathlineto{\pgfqpoint{4.669272in}{2.425059in}}%
\pgfusepath{stroke}%
\end{pgfscope}%
\begin{pgfscope}%
\pgfsetrectcap%
\pgfsetmiterjoin%
\pgfsetlinewidth{0.803000pt}%
\definecolor{currentstroke}{rgb}{0.000000,0.000000,0.000000}%
\pgfsetstrokecolor{currentstroke}%
\pgfsetdash{}{0pt}%
\pgfpathmoveto{\pgfqpoint{0.605784in}{0.382904in}}%
\pgfpathlineto{\pgfqpoint{4.669272in}{0.382904in}}%
\pgfusepath{stroke}%
\end{pgfscope}%
\begin{pgfscope}%
\pgfsetrectcap%
\pgfsetmiterjoin%
\pgfsetlinewidth{0.803000pt}%
\definecolor{currentstroke}{rgb}{0.000000,0.000000,0.000000}%
\pgfsetstrokecolor{currentstroke}%
\pgfsetdash{}{0pt}%
\pgfpathmoveto{\pgfqpoint{0.605784in}{2.425059in}}%
\pgfpathlineto{\pgfqpoint{4.669272in}{2.425059in}}%
\pgfusepath{stroke}%
\end{pgfscope}%
\begin{pgfscope}%
\pgfsetbuttcap%
\pgfsetmiterjoin%
\definecolor{currentfill}{rgb}{0.556863,0.729412,0.898039}%
\pgfsetfillcolor{currentfill}%
\pgfsetlinewidth{0.752812pt}%
\definecolor{currentstroke}{rgb}{0.000000,0.000000,0.000000}%
\pgfsetstrokecolor{currentstroke}%
\pgfsetdash{}{0pt}%
\pgfpathmoveto{\pgfqpoint{3.246015in}{0.382904in}}%
\pgfpathlineto{\pgfqpoint{4.669272in}{0.382904in}}%
\pgfpathlineto{\pgfqpoint{4.669272in}{0.630932in}}%
\pgfpathlineto{\pgfqpoint{3.246015in}{0.630932in}}%
\pgfpathclose%
\pgfusepath{stroke,fill}%
\end{pgfscope}%
\begin{pgfscope}%
\pgfsetrectcap%
\pgfsetroundjoin%
\pgfsetlinewidth{1.505625pt}%
\definecolor{currentstroke}{rgb}{0.000000,0.000000,0.000000}%
\pgfsetstrokecolor{currentstroke}%
\pgfsetdash{}{0pt}%
\pgfpathmoveto{\pgfqpoint{3.301571in}{0.521406in}}%
\pgfpathlineto{\pgfqpoint{3.579349in}{0.521406in}}%
\pgfusepath{stroke}%
\end{pgfscope}%
\begin{pgfscope}%
\definecolor{textcolor}{rgb}{0.000000,0.000000,0.000000}%
\pgfsetstrokecolor{textcolor}%
\pgfsetfillcolor{textcolor}%
\pgftext[x=3.690460in,y=0.472795in,left,base]{\color{textcolor}\rmfamily\fontsize{10.000000}{12.000000}\selectfont \(\displaystyle \mathcal{X}\)}%
\end{pgfscope}%
\begin{pgfscope}%
\pgfsetrectcap%
\pgfsetroundjoin%
\pgfsetlinewidth{1.003750pt}%
\definecolor{currentstroke}{rgb}{1.000000,1.000000,1.000000}%
\pgfsetstrokecolor{currentstroke}%
\pgfsetdash{}{0pt}%
\pgfpathmoveto{\pgfqpoint{4.087644in}{0.521406in}}%
\pgfpathlineto{\pgfqpoint{4.365421in}{0.521406in}}%
\pgfusepath{stroke}%
\end{pgfscope}%
\begin{pgfscope}%
\definecolor{textcolor}{rgb}{0.000000,0.000000,0.000000}%
\pgfsetstrokecolor{textcolor}%
\pgfsetfillcolor{textcolor}%
\pgftext[x=4.476533in,y=0.472795in,left,base]{\color{textcolor}\rmfamily\fontsize{10.000000}{12.000000}\selectfont \(\displaystyle \mathbf{x}_t^*\)}%
\end{pgfscope}%
\begin{pgfscope}%
\pgfsetbuttcap%
\pgfsetmiterjoin%
\definecolor{currentfill}{rgb}{1.000000,1.000000,1.000000}%
\pgfsetfillcolor{currentfill}%
\pgfsetlinewidth{0.000000pt}%
\definecolor{currentstroke}{rgb}{0.000000,0.000000,0.000000}%
\pgfsetstrokecolor{currentstroke}%
\pgfsetstrokeopacity{0.000000}%
\pgfsetdash{}{0pt}%
\pgfpathmoveto{\pgfqpoint{4.756284in}{0.382904in}}%
\pgfpathlineto{\pgfqpoint{4.849110in}{0.382904in}}%
\pgfpathlineto{\pgfqpoint{4.849110in}{2.425059in}}%
\pgfpathlineto{\pgfqpoint{4.756284in}{2.425059in}}%
\pgfpathclose%
\pgfusepath{fill}%
\end{pgfscope}%
\begin{pgfscope}%
\pgfpathrectangle{\pgfqpoint{4.756284in}{0.382904in}}{\pgfqpoint{0.092825in}{2.042155in}}%
\pgfusepath{clip}%
\pgfsetbuttcap%
\pgfsetmiterjoin%
\definecolor{currentfill}{rgb}{1.000000,1.000000,1.000000}%
\pgfsetfillcolor{currentfill}%
\pgfsetlinewidth{0.010037pt}%
\definecolor{currentstroke}{rgb}{1.000000,1.000000,1.000000}%
\pgfsetstrokecolor{currentstroke}%
\pgfsetdash{}{0pt}%
\pgfpathmoveto{\pgfqpoint{4.802697in}{0.382904in}}%
\pgfpathlineto{\pgfqpoint{4.756284in}{0.475729in}}%
\pgfpathlineto{\pgfqpoint{4.756284in}{2.332233in}}%
\pgfpathlineto{\pgfqpoint{4.802697in}{2.425059in}}%
\pgfpathlineto{\pgfqpoint{4.802697in}{2.425059in}}%
\pgfpathlineto{\pgfqpoint{4.849110in}{2.332233in}}%
\pgfpathlineto{\pgfqpoint{4.849110in}{0.475729in}}%
\pgfpathlineto{\pgfqpoint{4.802697in}{0.382904in}}%
\pgfpathclose%
\pgfusepath{stroke,fill}%
\end{pgfscope}%
\begin{pgfscope}%
\pgfpathrectangle{\pgfqpoint{4.756284in}{0.382904in}}{\pgfqpoint{0.092825in}{2.042155in}}%
\pgfusepath{clip}%
\pgfsetbuttcap%
\pgfsetroundjoin%
\definecolor{currentfill}{rgb}{0.267004,0.004874,0.329415}%
\pgfsetfillcolor{currentfill}%
\pgfsetlinewidth{0.000000pt}%
\definecolor{currentstroke}{rgb}{0.000000,0.000000,0.000000}%
\pgfsetstrokecolor{currentstroke}%
\pgfsetdash{}{0pt}%
\pgfpathmoveto{\pgfqpoint{4.802697in}{0.382904in}}%
\pgfpathlineto{\pgfqpoint{4.802697in}{0.382904in}}%
\pgfpathlineto{\pgfqpoint{4.849110in}{0.475729in}}%
\pgfpathlineto{\pgfqpoint{4.756284in}{0.475729in}}%
\pgfpathlineto{\pgfqpoint{4.802697in}{0.382904in}}%
\pgfusepath{fill}%
\end{pgfscope}%
\begin{pgfscope}%
\pgfpathrectangle{\pgfqpoint{4.756284in}{0.382904in}}{\pgfqpoint{0.092825in}{2.042155in}}%
\pgfusepath{clip}%
\pgfsetbuttcap%
\pgfsetroundjoin%
\definecolor{currentfill}{rgb}{0.277941,0.056324,0.381191}%
\pgfsetfillcolor{currentfill}%
\pgfsetlinewidth{0.000000pt}%
\definecolor{currentstroke}{rgb}{0.000000,0.000000,0.000000}%
\pgfsetstrokecolor{currentstroke}%
\pgfsetdash{}{0pt}%
\pgfpathmoveto{\pgfqpoint{4.756284in}{0.475729in}}%
\pgfpathlineto{\pgfqpoint{4.849110in}{0.475729in}}%
\pgfpathlineto{\pgfqpoint{4.849110in}{0.608337in}}%
\pgfpathlineto{\pgfqpoint{4.756284in}{0.608337in}}%
\pgfpathlineto{\pgfqpoint{4.756284in}{0.475729in}}%
\pgfusepath{fill}%
\end{pgfscope}%
\begin{pgfscope}%
\pgfpathrectangle{\pgfqpoint{4.756284in}{0.382904in}}{\pgfqpoint{0.092825in}{2.042155in}}%
\pgfusepath{clip}%
\pgfsetbuttcap%
\pgfsetroundjoin%
\definecolor{currentfill}{rgb}{0.281887,0.150881,0.465405}%
\pgfsetfillcolor{currentfill}%
\pgfsetlinewidth{0.000000pt}%
\definecolor{currentstroke}{rgb}{0.000000,0.000000,0.000000}%
\pgfsetstrokecolor{currentstroke}%
\pgfsetdash{}{0pt}%
\pgfpathmoveto{\pgfqpoint{4.756284in}{0.608337in}}%
\pgfpathlineto{\pgfqpoint{4.849110in}{0.608337in}}%
\pgfpathlineto{\pgfqpoint{4.849110in}{0.740944in}}%
\pgfpathlineto{\pgfqpoint{4.756284in}{0.740944in}}%
\pgfpathlineto{\pgfqpoint{4.756284in}{0.608337in}}%
\pgfusepath{fill}%
\end{pgfscope}%
\begin{pgfscope}%
\pgfpathrectangle{\pgfqpoint{4.756284in}{0.382904in}}{\pgfqpoint{0.092825in}{2.042155in}}%
\pgfusepath{clip}%
\pgfsetbuttcap%
\pgfsetroundjoin%
\definecolor{currentfill}{rgb}{0.263663,0.237631,0.518762}%
\pgfsetfillcolor{currentfill}%
\pgfsetlinewidth{0.000000pt}%
\definecolor{currentstroke}{rgb}{0.000000,0.000000,0.000000}%
\pgfsetstrokecolor{currentstroke}%
\pgfsetdash{}{0pt}%
\pgfpathmoveto{\pgfqpoint{4.756284in}{0.740944in}}%
\pgfpathlineto{\pgfqpoint{4.849110in}{0.740944in}}%
\pgfpathlineto{\pgfqpoint{4.849110in}{0.873552in}}%
\pgfpathlineto{\pgfqpoint{4.756284in}{0.873552in}}%
\pgfpathlineto{\pgfqpoint{4.756284in}{0.740944in}}%
\pgfusepath{fill}%
\end{pgfscope}%
\begin{pgfscope}%
\pgfpathrectangle{\pgfqpoint{4.756284in}{0.382904in}}{\pgfqpoint{0.092825in}{2.042155in}}%
\pgfusepath{clip}%
\pgfsetbuttcap%
\pgfsetroundjoin%
\definecolor{currentfill}{rgb}{0.229739,0.322361,0.545706}%
\pgfsetfillcolor{currentfill}%
\pgfsetlinewidth{0.000000pt}%
\definecolor{currentstroke}{rgb}{0.000000,0.000000,0.000000}%
\pgfsetstrokecolor{currentstroke}%
\pgfsetdash{}{0pt}%
\pgfpathmoveto{\pgfqpoint{4.756284in}{0.873552in}}%
\pgfpathlineto{\pgfqpoint{4.849110in}{0.873552in}}%
\pgfpathlineto{\pgfqpoint{4.849110in}{1.006159in}}%
\pgfpathlineto{\pgfqpoint{4.756284in}{1.006159in}}%
\pgfpathlineto{\pgfqpoint{4.756284in}{0.873552in}}%
\pgfusepath{fill}%
\end{pgfscope}%
\begin{pgfscope}%
\pgfpathrectangle{\pgfqpoint{4.756284in}{0.382904in}}{\pgfqpoint{0.092825in}{2.042155in}}%
\pgfusepath{clip}%
\pgfsetbuttcap%
\pgfsetroundjoin%
\definecolor{currentfill}{rgb}{0.195860,0.395433,0.555276}%
\pgfsetfillcolor{currentfill}%
\pgfsetlinewidth{0.000000pt}%
\definecolor{currentstroke}{rgb}{0.000000,0.000000,0.000000}%
\pgfsetstrokecolor{currentstroke}%
\pgfsetdash{}{0pt}%
\pgfpathmoveto{\pgfqpoint{4.756284in}{1.006159in}}%
\pgfpathlineto{\pgfqpoint{4.849110in}{1.006159in}}%
\pgfpathlineto{\pgfqpoint{4.849110in}{1.138766in}}%
\pgfpathlineto{\pgfqpoint{4.756284in}{1.138766in}}%
\pgfpathlineto{\pgfqpoint{4.756284in}{1.006159in}}%
\pgfusepath{fill}%
\end{pgfscope}%
\begin{pgfscope}%
\pgfpathrectangle{\pgfqpoint{4.756284in}{0.382904in}}{\pgfqpoint{0.092825in}{2.042155in}}%
\pgfusepath{clip}%
\pgfsetbuttcap%
\pgfsetroundjoin%
\definecolor{currentfill}{rgb}{0.166617,0.463708,0.558119}%
\pgfsetfillcolor{currentfill}%
\pgfsetlinewidth{0.000000pt}%
\definecolor{currentstroke}{rgb}{0.000000,0.000000,0.000000}%
\pgfsetstrokecolor{currentstroke}%
\pgfsetdash{}{0pt}%
\pgfpathmoveto{\pgfqpoint{4.756284in}{1.138766in}}%
\pgfpathlineto{\pgfqpoint{4.849110in}{1.138766in}}%
\pgfpathlineto{\pgfqpoint{4.849110in}{1.271374in}}%
\pgfpathlineto{\pgfqpoint{4.756284in}{1.271374in}}%
\pgfpathlineto{\pgfqpoint{4.756284in}{1.138766in}}%
\pgfusepath{fill}%
\end{pgfscope}%
\begin{pgfscope}%
\pgfpathrectangle{\pgfqpoint{4.756284in}{0.382904in}}{\pgfqpoint{0.092825in}{2.042155in}}%
\pgfusepath{clip}%
\pgfsetbuttcap%
\pgfsetroundjoin%
\definecolor{currentfill}{rgb}{0.140536,0.530132,0.555659}%
\pgfsetfillcolor{currentfill}%
\pgfsetlinewidth{0.000000pt}%
\definecolor{currentstroke}{rgb}{0.000000,0.000000,0.000000}%
\pgfsetstrokecolor{currentstroke}%
\pgfsetdash{}{0pt}%
\pgfpathmoveto{\pgfqpoint{4.756284in}{1.271374in}}%
\pgfpathlineto{\pgfqpoint{4.849110in}{1.271374in}}%
\pgfpathlineto{\pgfqpoint{4.849110in}{1.403981in}}%
\pgfpathlineto{\pgfqpoint{4.756284in}{1.403981in}}%
\pgfpathlineto{\pgfqpoint{4.756284in}{1.271374in}}%
\pgfusepath{fill}%
\end{pgfscope}%
\begin{pgfscope}%
\pgfpathrectangle{\pgfqpoint{4.756284in}{0.382904in}}{\pgfqpoint{0.092825in}{2.042155in}}%
\pgfusepath{clip}%
\pgfsetbuttcap%
\pgfsetroundjoin%
\definecolor{currentfill}{rgb}{0.120092,0.600104,0.542530}%
\pgfsetfillcolor{currentfill}%
\pgfsetlinewidth{0.000000pt}%
\definecolor{currentstroke}{rgb}{0.000000,0.000000,0.000000}%
\pgfsetstrokecolor{currentstroke}%
\pgfsetdash{}{0pt}%
\pgfpathmoveto{\pgfqpoint{4.756284in}{1.403981in}}%
\pgfpathlineto{\pgfqpoint{4.849110in}{1.403981in}}%
\pgfpathlineto{\pgfqpoint{4.849110in}{1.536589in}}%
\pgfpathlineto{\pgfqpoint{4.756284in}{1.536589in}}%
\pgfpathlineto{\pgfqpoint{4.756284in}{1.403981in}}%
\pgfusepath{fill}%
\end{pgfscope}%
\begin{pgfscope}%
\pgfpathrectangle{\pgfqpoint{4.756284in}{0.382904in}}{\pgfqpoint{0.092825in}{2.042155in}}%
\pgfusepath{clip}%
\pgfsetbuttcap%
\pgfsetroundjoin%
\definecolor{currentfill}{rgb}{0.140210,0.665859,0.513427}%
\pgfsetfillcolor{currentfill}%
\pgfsetlinewidth{0.000000pt}%
\definecolor{currentstroke}{rgb}{0.000000,0.000000,0.000000}%
\pgfsetstrokecolor{currentstroke}%
\pgfsetdash{}{0pt}%
\pgfpathmoveto{\pgfqpoint{4.756284in}{1.536589in}}%
\pgfpathlineto{\pgfqpoint{4.849110in}{1.536589in}}%
\pgfpathlineto{\pgfqpoint{4.849110in}{1.669196in}}%
\pgfpathlineto{\pgfqpoint{4.756284in}{1.669196in}}%
\pgfpathlineto{\pgfqpoint{4.756284in}{1.536589in}}%
\pgfusepath{fill}%
\end{pgfscope}%
\begin{pgfscope}%
\pgfpathrectangle{\pgfqpoint{4.756284in}{0.382904in}}{\pgfqpoint{0.092825in}{2.042155in}}%
\pgfusepath{clip}%
\pgfsetbuttcap%
\pgfsetroundjoin%
\definecolor{currentfill}{rgb}{0.226397,0.728888,0.462789}%
\pgfsetfillcolor{currentfill}%
\pgfsetlinewidth{0.000000pt}%
\definecolor{currentstroke}{rgb}{0.000000,0.000000,0.000000}%
\pgfsetstrokecolor{currentstroke}%
\pgfsetdash{}{0pt}%
\pgfpathmoveto{\pgfqpoint{4.756284in}{1.669196in}}%
\pgfpathlineto{\pgfqpoint{4.849110in}{1.669196in}}%
\pgfpathlineto{\pgfqpoint{4.849110in}{1.801804in}}%
\pgfpathlineto{\pgfqpoint{4.756284in}{1.801804in}}%
\pgfpathlineto{\pgfqpoint{4.756284in}{1.669196in}}%
\pgfusepath{fill}%
\end{pgfscope}%
\begin{pgfscope}%
\pgfpathrectangle{\pgfqpoint{4.756284in}{0.382904in}}{\pgfqpoint{0.092825in}{2.042155in}}%
\pgfusepath{clip}%
\pgfsetbuttcap%
\pgfsetroundjoin%
\definecolor{currentfill}{rgb}{0.369214,0.788888,0.382914}%
\pgfsetfillcolor{currentfill}%
\pgfsetlinewidth{0.000000pt}%
\definecolor{currentstroke}{rgb}{0.000000,0.000000,0.000000}%
\pgfsetstrokecolor{currentstroke}%
\pgfsetdash{}{0pt}%
\pgfpathmoveto{\pgfqpoint{4.756284in}{1.801804in}}%
\pgfpathlineto{\pgfqpoint{4.849110in}{1.801804in}}%
\pgfpathlineto{\pgfqpoint{4.849110in}{1.934411in}}%
\pgfpathlineto{\pgfqpoint{4.756284in}{1.934411in}}%
\pgfpathlineto{\pgfqpoint{4.756284in}{1.801804in}}%
\pgfusepath{fill}%
\end{pgfscope}%
\begin{pgfscope}%
\pgfpathrectangle{\pgfqpoint{4.756284in}{0.382904in}}{\pgfqpoint{0.092825in}{2.042155in}}%
\pgfusepath{clip}%
\pgfsetbuttcap%
\pgfsetroundjoin%
\definecolor{currentfill}{rgb}{0.535621,0.835785,0.281908}%
\pgfsetfillcolor{currentfill}%
\pgfsetlinewidth{0.000000pt}%
\definecolor{currentstroke}{rgb}{0.000000,0.000000,0.000000}%
\pgfsetstrokecolor{currentstroke}%
\pgfsetdash{}{0pt}%
\pgfpathmoveto{\pgfqpoint{4.756284in}{1.934411in}}%
\pgfpathlineto{\pgfqpoint{4.849110in}{1.934411in}}%
\pgfpathlineto{\pgfqpoint{4.849110in}{2.067019in}}%
\pgfpathlineto{\pgfqpoint{4.756284in}{2.067019in}}%
\pgfpathlineto{\pgfqpoint{4.756284in}{1.934411in}}%
\pgfusepath{fill}%
\end{pgfscope}%
\begin{pgfscope}%
\pgfpathrectangle{\pgfqpoint{4.756284in}{0.382904in}}{\pgfqpoint{0.092825in}{2.042155in}}%
\pgfusepath{clip}%
\pgfsetbuttcap%
\pgfsetroundjoin%
\definecolor{currentfill}{rgb}{0.720391,0.870350,0.162603}%
\pgfsetfillcolor{currentfill}%
\pgfsetlinewidth{0.000000pt}%
\definecolor{currentstroke}{rgb}{0.000000,0.000000,0.000000}%
\pgfsetstrokecolor{currentstroke}%
\pgfsetdash{}{0pt}%
\pgfpathmoveto{\pgfqpoint{4.756284in}{2.067019in}}%
\pgfpathlineto{\pgfqpoint{4.849110in}{2.067019in}}%
\pgfpathlineto{\pgfqpoint{4.849110in}{2.199626in}}%
\pgfpathlineto{\pgfqpoint{4.756284in}{2.199626in}}%
\pgfpathlineto{\pgfqpoint{4.756284in}{2.067019in}}%
\pgfusepath{fill}%
\end{pgfscope}%
\begin{pgfscope}%
\pgfpathrectangle{\pgfqpoint{4.756284in}{0.382904in}}{\pgfqpoint{0.092825in}{2.042155in}}%
\pgfusepath{clip}%
\pgfsetbuttcap%
\pgfsetroundjoin%
\definecolor{currentfill}{rgb}{0.906311,0.894855,0.098125}%
\pgfsetfillcolor{currentfill}%
\pgfsetlinewidth{0.000000pt}%
\definecolor{currentstroke}{rgb}{0.000000,0.000000,0.000000}%
\pgfsetstrokecolor{currentstroke}%
\pgfsetdash{}{0pt}%
\pgfpathmoveto{\pgfqpoint{4.756284in}{2.199626in}}%
\pgfpathlineto{\pgfqpoint{4.849110in}{2.199626in}}%
\pgfpathlineto{\pgfqpoint{4.849110in}{2.332233in}}%
\pgfpathlineto{\pgfqpoint{4.756284in}{2.332233in}}%
\pgfpathlineto{\pgfqpoint{4.756284in}{2.199626in}}%
\pgfusepath{fill}%
\end{pgfscope}%
\begin{pgfscope}%
\pgfpathrectangle{\pgfqpoint{4.756284in}{0.382904in}}{\pgfqpoint{0.092825in}{2.042155in}}%
\pgfusepath{clip}%
\pgfsetbuttcap%
\pgfsetroundjoin%
\definecolor{currentfill}{rgb}{0.993248,0.906157,0.143936}%
\pgfsetfillcolor{currentfill}%
\pgfsetlinewidth{0.000000pt}%
\definecolor{currentstroke}{rgb}{0.000000,0.000000,0.000000}%
\pgfsetstrokecolor{currentstroke}%
\pgfsetdash{}{0pt}%
\pgfpathmoveto{\pgfqpoint{4.756284in}{2.332233in}}%
\pgfpathlineto{\pgfqpoint{4.849110in}{2.332233in}}%
\pgfpathlineto{\pgfqpoint{4.802697in}{2.425059in}}%
\pgfpathlineto{\pgfqpoint{4.802697in}{2.425059in}}%
\pgfpathlineto{\pgfqpoint{4.756284in}{2.332233in}}%
\pgfusepath{fill}%
\end{pgfscope}%
\begin{pgfscope}%
\pgfsetbuttcap%
\pgfsetroundjoin%
\definecolor{currentfill}{rgb}{0.000000,0.000000,0.000000}%
\pgfsetfillcolor{currentfill}%
\pgfsetlinewidth{0.803000pt}%
\definecolor{currentstroke}{rgb}{0.000000,0.000000,0.000000}%
\pgfsetstrokecolor{currentstroke}%
\pgfsetdash{}{0pt}%
\pgfsys@defobject{currentmarker}{\pgfqpoint{0.000000in}{0.000000in}}{\pgfqpoint{0.048611in}{0.000000in}}{%
\pgfpathmoveto{\pgfqpoint{0.000000in}{0.000000in}}%
\pgfpathlineto{\pgfqpoint{0.048611in}{0.000000in}}%
\pgfusepath{stroke,fill}%
}%
\begin{pgfscope}%
\pgfsys@transformshift{4.849110in}{0.475729in}%
\pgfsys@useobject{currentmarker}{}%
\end{pgfscope}%
\end{pgfscope}%
\begin{pgfscope}%
\definecolor{textcolor}{rgb}{0.000000,0.000000,0.000000}%
\pgfsetstrokecolor{textcolor}%
\pgfsetfillcolor{textcolor}%
\pgftext[x=4.946332in, y=0.427535in, left, base]{\color{textcolor}\rmfamily\fontsize{10.000000}{12.000000}\selectfont \(\displaystyle {3}\)}%
\end{pgfscope}%
\begin{pgfscope}%
\pgfsetbuttcap%
\pgfsetroundjoin%
\definecolor{currentfill}{rgb}{0.000000,0.000000,0.000000}%
\pgfsetfillcolor{currentfill}%
\pgfsetlinewidth{0.803000pt}%
\definecolor{currentstroke}{rgb}{0.000000,0.000000,0.000000}%
\pgfsetstrokecolor{currentstroke}%
\pgfsetdash{}{0pt}%
\pgfsys@defobject{currentmarker}{\pgfqpoint{0.000000in}{0.000000in}}{\pgfqpoint{0.048611in}{0.000000in}}{%
\pgfpathmoveto{\pgfqpoint{0.000000in}{0.000000in}}%
\pgfpathlineto{\pgfqpoint{0.048611in}{0.000000in}}%
\pgfusepath{stroke,fill}%
}%
\begin{pgfscope}%
\pgfsys@transformshift{4.849110in}{0.847030in}%
\pgfsys@useobject{currentmarker}{}%
\end{pgfscope}%
\end{pgfscope}%
\begin{pgfscope}%
\definecolor{textcolor}{rgb}{0.000000,0.000000,0.000000}%
\pgfsetstrokecolor{textcolor}%
\pgfsetfillcolor{textcolor}%
\pgftext[x=4.946332in, y=0.798836in, left, base]{\color{textcolor}\rmfamily\fontsize{10.000000}{12.000000}\selectfont \(\displaystyle {9}\)}%
\end{pgfscope}%
\begin{pgfscope}%
\pgfsetbuttcap%
\pgfsetroundjoin%
\definecolor{currentfill}{rgb}{0.000000,0.000000,0.000000}%
\pgfsetfillcolor{currentfill}%
\pgfsetlinewidth{0.803000pt}%
\definecolor{currentstroke}{rgb}{0.000000,0.000000,0.000000}%
\pgfsetstrokecolor{currentstroke}%
\pgfsetdash{}{0pt}%
\pgfsys@defobject{currentmarker}{\pgfqpoint{0.000000in}{0.000000in}}{\pgfqpoint{0.048611in}{0.000000in}}{%
\pgfpathmoveto{\pgfqpoint{0.000000in}{0.000000in}}%
\pgfpathlineto{\pgfqpoint{0.048611in}{0.000000in}}%
\pgfusepath{stroke,fill}%
}%
\begin{pgfscope}%
\pgfsys@transformshift{4.849110in}{1.218331in}%
\pgfsys@useobject{currentmarker}{}%
\end{pgfscope}%
\end{pgfscope}%
\begin{pgfscope}%
\definecolor{textcolor}{rgb}{0.000000,0.000000,0.000000}%
\pgfsetstrokecolor{textcolor}%
\pgfsetfillcolor{textcolor}%
\pgftext[x=4.946332in, y=1.170136in, left, base]{\color{textcolor}\rmfamily\fontsize{10.000000}{12.000000}\selectfont \(\displaystyle {15}\)}%
\end{pgfscope}%
\begin{pgfscope}%
\pgfsetbuttcap%
\pgfsetroundjoin%
\definecolor{currentfill}{rgb}{0.000000,0.000000,0.000000}%
\pgfsetfillcolor{currentfill}%
\pgfsetlinewidth{0.803000pt}%
\definecolor{currentstroke}{rgb}{0.000000,0.000000,0.000000}%
\pgfsetstrokecolor{currentstroke}%
\pgfsetdash{}{0pt}%
\pgfsys@defobject{currentmarker}{\pgfqpoint{0.000000in}{0.000000in}}{\pgfqpoint{0.048611in}{0.000000in}}{%
\pgfpathmoveto{\pgfqpoint{0.000000in}{0.000000in}}%
\pgfpathlineto{\pgfqpoint{0.048611in}{0.000000in}}%
\pgfusepath{stroke,fill}%
}%
\begin{pgfscope}%
\pgfsys@transformshift{4.849110in}{1.589632in}%
\pgfsys@useobject{currentmarker}{}%
\end{pgfscope}%
\end{pgfscope}%
\begin{pgfscope}%
\definecolor{textcolor}{rgb}{0.000000,0.000000,0.000000}%
\pgfsetstrokecolor{textcolor}%
\pgfsetfillcolor{textcolor}%
\pgftext[x=4.946332in, y=1.541437in, left, base]{\color{textcolor}\rmfamily\fontsize{10.000000}{12.000000}\selectfont \(\displaystyle {21}\)}%
\end{pgfscope}%
\begin{pgfscope}%
\pgfsetbuttcap%
\pgfsetroundjoin%
\definecolor{currentfill}{rgb}{0.000000,0.000000,0.000000}%
\pgfsetfillcolor{currentfill}%
\pgfsetlinewidth{0.803000pt}%
\definecolor{currentstroke}{rgb}{0.000000,0.000000,0.000000}%
\pgfsetstrokecolor{currentstroke}%
\pgfsetdash{}{0pt}%
\pgfsys@defobject{currentmarker}{\pgfqpoint{0.000000in}{0.000000in}}{\pgfqpoint{0.048611in}{0.000000in}}{%
\pgfpathmoveto{\pgfqpoint{0.000000in}{0.000000in}}%
\pgfpathlineto{\pgfqpoint{0.048611in}{0.000000in}}%
\pgfusepath{stroke,fill}%
}%
\begin{pgfscope}%
\pgfsys@transformshift{4.849110in}{1.960933in}%
\pgfsys@useobject{currentmarker}{}%
\end{pgfscope}%
\end{pgfscope}%
\begin{pgfscope}%
\definecolor{textcolor}{rgb}{0.000000,0.000000,0.000000}%
\pgfsetstrokecolor{textcolor}%
\pgfsetfillcolor{textcolor}%
\pgftext[x=4.946332in, y=1.912738in, left, base]{\color{textcolor}\rmfamily\fontsize{10.000000}{12.000000}\selectfont \(\displaystyle {27}\)}%
\end{pgfscope}%
\begin{pgfscope}%
\pgfsetbuttcap%
\pgfsetroundjoin%
\definecolor{currentfill}{rgb}{0.000000,0.000000,0.000000}%
\pgfsetfillcolor{currentfill}%
\pgfsetlinewidth{0.803000pt}%
\definecolor{currentstroke}{rgb}{0.000000,0.000000,0.000000}%
\pgfsetstrokecolor{currentstroke}%
\pgfsetdash{}{0pt}%
\pgfsys@defobject{currentmarker}{\pgfqpoint{0.000000in}{0.000000in}}{\pgfqpoint{0.048611in}{0.000000in}}{%
\pgfpathmoveto{\pgfqpoint{0.000000in}{0.000000in}}%
\pgfpathlineto{\pgfqpoint{0.048611in}{0.000000in}}%
\pgfusepath{stroke,fill}%
}%
\begin{pgfscope}%
\pgfsys@transformshift{4.849110in}{2.332233in}%
\pgfsys@useobject{currentmarker}{}%
\end{pgfscope}%
\end{pgfscope}%
\begin{pgfscope}%
\definecolor{textcolor}{rgb}{0.000000,0.000000,0.000000}%
\pgfsetstrokecolor{textcolor}%
\pgfsetfillcolor{textcolor}%
\pgftext[x=4.946332in, y=2.284039in, left, base]{\color{textcolor}\rmfamily\fontsize{10.000000}{12.000000}\selectfont \(\displaystyle {33}\)}%
\end{pgfscope}%
\begin{pgfscope}%
\definecolor{textcolor}{rgb}{0.000000,0.000000,0.000000}%
\pgfsetstrokecolor{textcolor}%
\pgfsetfillcolor{textcolor}%
\pgftext[x=5.140777in,y=1.403981in,,top,rotate=90.000000]{\color{textcolor}\rmfamily\fontsize{10.000000}{12.000000}\selectfont \(\displaystyle f_t(x)\)}%
\end{pgfscope}%
\begin{pgfscope}%
\pgfsetrectcap%
\pgfsetmiterjoin%
\pgfsetlinewidth{0.803000pt}%
\definecolor{currentstroke}{rgb}{0.000000,0.000000,0.000000}%
\pgfsetstrokecolor{currentstroke}%
\pgfsetdash{}{0pt}%
\pgfpathmoveto{\pgfqpoint{4.802697in}{0.382904in}}%
\pgfpathlineto{\pgfqpoint{4.756284in}{0.475729in}}%
\pgfpathlineto{\pgfqpoint{4.756284in}{2.332233in}}%
\pgfpathlineto{\pgfqpoint{4.802697in}{2.425059in}}%
\pgfpathlineto{\pgfqpoint{4.802697in}{2.425059in}}%
\pgfpathlineto{\pgfqpoint{4.849110in}{2.332233in}}%
\pgfpathlineto{\pgfqpoint{4.849110in}{0.475729in}}%
\pgfpathlineto{\pgfqpoint{4.802697in}{0.382904in}}%
\pgfpathclose%
\pgfusepath{stroke}%
\end{pgfscope}%
\end{pgfpicture}%
\makeatother%
\endgroup%

    \caption[Objective function of the one-dimensional moving parabola.]{Objective function in \eqref{eq:1d_parabola} of the one-dimensional moving parabola.}
    \label{fig:Parabola1D}
\end{figure}

For the variations without data selection strategy, a sensitivity analysis regarding the forgetting factor was performed and the results are shown in Figure~\ref{fig:Parabola1D_forgetting_factors}.

The forgetting factor for \gls{b2p} forgetting is only defined as $\epsilon \in (0, 1)$, whereas the \gls{ui} forgetting factor is defined as $\hat{\sigma}_w^2 \in (0, \infty)$. Therefore, the significantly higher regret for \gls{b2p} forgetting at high forgetting factors compared to \gls{ui} forgetting is expected. Figure~\ref{fig:Parabola1D_forgetting_factors} shows that for \gls{b2p} forgetting, constraining the \gls{gp} posterior by using the proposed method \gls{ctvbo} reduces the regret for forgetting factors up to $\epsilon = 0.0215$. However, for higher forgetting factors, constraining the posterior increases the regret. This is a result of \gls{b2p} forgetting. As the constraints limit the exploration of the acquisition function, queries around the predicted optimum are chosen. However, since the optimum is never chosen directly due to the \gls{lcb} acquisition function, the expectation of the optimal function value $\EX[f_t(\mathbf{\hat{x}}_t^*)]$ propagates towards the prior mean. $\mu_0 > \EX[f_t(\mathbf{\hat{x}}_t^*)]$ suggests that the resulting posterior becomes more level over time. Therefore, the sampling radius around the predicted optimum of the acquisition function is increased due to the constant $\beta_{t+1}$, resulting in increased regret compared to the unconstrained case as in \Cref{sec:out_of_model}.
The exception for $\epsilon = 0.3$ seems to be an outlier, where the objective function is constructed in such a way that the regret with \gls{b2p} forgetting and \gls{ctvbo} is significantly reduced for this specific forgetting factor.
\begin{figure}[h!]
    \centering
    %% Creator: Matplotlib, PGF backend
%%
%% To include the figure in your LaTeX document, write
%%   \input{<filename>.pgf}
%%
%% Make sure the required packages are loaded in your preamble
%%   \usepackage{pgf}
%%
%% Figures using additional raster images can only be included by \input if
%% they are in the same directory as the main LaTeX file. For loading figures
%% from other directories you can use the `import` package
%%   \usepackage{import}
%%
%% and then include the figures with
%%   \import{<path to file>}{<filename>.pgf}
%%
%% Matplotlib used the following preamble
%%   \usepackage{fontspec}
%%
\begingroup%
\makeatletter%
\begin{pgfpicture}%
\pgfpathrectangle{\pgfpointorigin}{\pgfqpoint{5.507126in}{5.105387in}}%
\pgfusepath{use as bounding box, clip}%
\begin{pgfscope}%
\pgfsetbuttcap%
\pgfsetmiterjoin%
\definecolor{currentfill}{rgb}{1.000000,1.000000,1.000000}%
\pgfsetfillcolor{currentfill}%
\pgfsetlinewidth{0.000000pt}%
\definecolor{currentstroke}{rgb}{1.000000,1.000000,1.000000}%
\pgfsetstrokecolor{currentstroke}%
\pgfsetdash{}{0pt}%
\pgfpathmoveto{\pgfqpoint{0.000000in}{0.000000in}}%
\pgfpathlineto{\pgfqpoint{5.507126in}{0.000000in}}%
\pgfpathlineto{\pgfqpoint{5.507126in}{5.105387in}}%
\pgfpathlineto{\pgfqpoint{0.000000in}{5.105387in}}%
\pgfpathclose%
\pgfusepath{fill}%
\end{pgfscope}%
\begin{pgfscope}%
\pgfsetbuttcap%
\pgfsetmiterjoin%
\definecolor{currentfill}{rgb}{1.000000,1.000000,1.000000}%
\pgfsetfillcolor{currentfill}%
\pgfsetlinewidth{0.000000pt}%
\definecolor{currentstroke}{rgb}{0.000000,0.000000,0.000000}%
\pgfsetstrokecolor{currentstroke}%
\pgfsetstrokeopacity{0.000000}%
\pgfsetdash{}{0pt}%
\pgfpathmoveto{\pgfqpoint{0.550713in}{3.620038in}}%
\pgfpathlineto{\pgfqpoint{5.341912in}{3.620038in}}%
\pgfpathlineto{\pgfqpoint{5.341912in}{4.952225in}}%
\pgfpathlineto{\pgfqpoint{0.550713in}{4.952225in}}%
\pgfpathclose%
\pgfusepath{fill}%
\end{pgfscope}%
\begin{pgfscope}%
\pgfpathrectangle{\pgfqpoint{0.550713in}{3.620038in}}{\pgfqpoint{4.791200in}{1.332187in}}%
\pgfusepath{clip}%
\pgfsetbuttcap%
\pgfsetroundjoin%
\definecolor{currentfill}{rgb}{0.934548,0.870850,0.887213}%
\pgfsetfillcolor{currentfill}%
\pgfsetlinewidth{0.752812pt}%
\definecolor{currentstroke}{rgb}{0.152941,0.152941,0.152941}%
\pgfsetstrokecolor{currentstroke}%
\pgfsetdash{}{0pt}%
\pgfpathmoveto{\pgfqpoint{0.630885in}{3.774083in}}%
\pgfpathlineto{\pgfqpoint{0.662188in}{3.774083in}}%
\pgfpathlineto{\pgfqpoint{0.662188in}{4.053215in}}%
\pgfpathlineto{\pgfqpoint{0.630885in}{4.053215in}}%
\pgfpathclose%
\pgfusepath{stroke,fill}%
\end{pgfscope}%
\begin{pgfscope}%
\pgfpathrectangle{\pgfqpoint{0.550713in}{3.620038in}}{\pgfqpoint{4.791200in}{1.332187in}}%
\pgfusepath{clip}%
\pgfsetbuttcap%
\pgfsetroundjoin%
\definecolor{currentfill}{rgb}{0.748399,0.503538,0.566438}%
\pgfsetfillcolor{currentfill}%
\pgfsetlinewidth{0.752812pt}%
\definecolor{currentstroke}{rgb}{0.152941,0.152941,0.152941}%
\pgfsetstrokecolor{currentstroke}%
\pgfsetdash{}{0pt}%
\pgfpathmoveto{\pgfqpoint{0.615234in}{3.789364in}}%
\pgfpathlineto{\pgfqpoint{0.677839in}{3.789364in}}%
\pgfpathlineto{\pgfqpoint{0.677839in}{4.039764in}}%
\pgfpathlineto{\pgfqpoint{0.615234in}{4.039764in}}%
\pgfpathclose%
\pgfusepath{stroke,fill}%
\end{pgfscope}%
\begin{pgfscope}%
\pgfpathrectangle{\pgfqpoint{0.550713in}{3.620038in}}{\pgfqpoint{4.791200in}{1.332187in}}%
\pgfusepath{clip}%
\pgfsetbuttcap%
\pgfsetroundjoin%
\definecolor{currentfill}{rgb}{0.560784,0.133333,0.243137}%
\pgfsetfillcolor{currentfill}%
\pgfsetlinewidth{0.752812pt}%
\definecolor{currentstroke}{rgb}{0.152941,0.152941,0.152941}%
\pgfsetstrokecolor{currentstroke}%
\pgfsetdash{}{0pt}%
\pgfpathmoveto{\pgfqpoint{0.583932in}{3.819927in}}%
\pgfpathlineto{\pgfqpoint{0.709142in}{3.819927in}}%
\pgfpathlineto{\pgfqpoint{0.709142in}{4.012862in}}%
\pgfpathlineto{\pgfqpoint{0.583932in}{4.012862in}}%
\pgfpathclose%
\pgfusepath{stroke,fill}%
\end{pgfscope}%
\begin{pgfscope}%
\pgfpathrectangle{\pgfqpoint{0.550713in}{3.620038in}}{\pgfqpoint{4.791200in}{1.332187in}}%
\pgfusepath{clip}%
\pgfsetbuttcap%
\pgfsetroundjoin%
\definecolor{currentfill}{rgb}{0.902991,0.874940,0.898900}%
\pgfsetfillcolor{currentfill}%
\pgfsetlinewidth{0.752812pt}%
\definecolor{currentstroke}{rgb}{0.152941,0.152941,0.152941}%
\pgfsetstrokecolor{currentstroke}%
\pgfsetdash{}{0pt}%
\pgfpathmoveto{\pgfqpoint{0.758651in}{3.688621in}}%
\pgfpathlineto{\pgfqpoint{0.789953in}{3.688621in}}%
\pgfpathlineto{\pgfqpoint{0.789953in}{3.751901in}}%
\pgfpathlineto{\pgfqpoint{0.758651in}{3.751901in}}%
\pgfpathclose%
\pgfusepath{stroke,fill}%
\end{pgfscope}%
\begin{pgfscope}%
\pgfpathrectangle{\pgfqpoint{0.550713in}{3.620038in}}{\pgfqpoint{4.791200in}{1.332187in}}%
\pgfusepath{clip}%
\pgfsetbuttcap%
\pgfsetroundjoin%
\definecolor{currentfill}{rgb}{0.627092,0.519263,0.611367}%
\pgfsetfillcolor{currentfill}%
\pgfsetlinewidth{0.752812pt}%
\definecolor{currentstroke}{rgb}{0.152941,0.152941,0.152941}%
\pgfsetstrokecolor{currentstroke}%
\pgfsetdash{}{0pt}%
\pgfpathmoveto{\pgfqpoint{0.742999in}{3.689849in}}%
\pgfpathlineto{\pgfqpoint{0.805604in}{3.689849in}}%
\pgfpathlineto{\pgfqpoint{0.805604in}{3.746612in}}%
\pgfpathlineto{\pgfqpoint{0.742999in}{3.746612in}}%
\pgfpathclose%
\pgfusepath{stroke,fill}%
\end{pgfscope}%
\begin{pgfscope}%
\pgfpathrectangle{\pgfqpoint{0.550713in}{3.620038in}}{\pgfqpoint{4.791200in}{1.332187in}}%
\pgfusepath{clip}%
\pgfsetbuttcap%
\pgfsetroundjoin%
\definecolor{currentfill}{rgb}{0.349020,0.160784,0.321569}%
\pgfsetfillcolor{currentfill}%
\pgfsetlinewidth{0.752812pt}%
\definecolor{currentstroke}{rgb}{0.152941,0.152941,0.152941}%
\pgfsetstrokecolor{currentstroke}%
\pgfsetdash{}{0pt}%
\pgfpathmoveto{\pgfqpoint{0.711697in}{3.692304in}}%
\pgfpathlineto{\pgfqpoint{0.836907in}{3.692304in}}%
\pgfpathlineto{\pgfqpoint{0.836907in}{3.736035in}}%
\pgfpathlineto{\pgfqpoint{0.711697in}{3.736035in}}%
\pgfpathclose%
\pgfusepath{stroke,fill}%
\end{pgfscope}%
\begin{pgfscope}%
\pgfpathrectangle{\pgfqpoint{0.550713in}{3.620038in}}{\pgfqpoint{4.791200in}{1.332187in}}%
\pgfusepath{clip}%
\pgfsetbuttcap%
\pgfsetroundjoin%
\definecolor{currentfill}{rgb}{0.934548,0.870850,0.887213}%
\pgfsetfillcolor{currentfill}%
\pgfsetlinewidth{0.752812pt}%
\definecolor{currentstroke}{rgb}{0.152941,0.152941,0.152941}%
\pgfsetstrokecolor{currentstroke}%
\pgfsetdash{}{0pt}%
\pgfpathmoveto{\pgfqpoint{0.950299in}{3.879174in}}%
\pgfpathlineto{\pgfqpoint{0.981601in}{3.879174in}}%
\pgfpathlineto{\pgfqpoint{0.981601in}{4.014495in}}%
\pgfpathlineto{\pgfqpoint{0.950299in}{4.014495in}}%
\pgfpathclose%
\pgfusepath{stroke,fill}%
\end{pgfscope}%
\begin{pgfscope}%
\pgfpathrectangle{\pgfqpoint{0.550713in}{3.620038in}}{\pgfqpoint{4.791200in}{1.332187in}}%
\pgfusepath{clip}%
\pgfsetbuttcap%
\pgfsetroundjoin%
\definecolor{currentfill}{rgb}{0.748399,0.503538,0.566438}%
\pgfsetfillcolor{currentfill}%
\pgfsetlinewidth{0.752812pt}%
\definecolor{currentstroke}{rgb}{0.152941,0.152941,0.152941}%
\pgfsetstrokecolor{currentstroke}%
\pgfsetdash{}{0pt}%
\pgfpathmoveto{\pgfqpoint{0.934647in}{3.887126in}}%
\pgfpathlineto{\pgfqpoint{0.997252in}{3.887126in}}%
\pgfpathlineto{\pgfqpoint{0.997252in}{3.999616in}}%
\pgfpathlineto{\pgfqpoint{0.934647in}{3.999616in}}%
\pgfpathclose%
\pgfusepath{stroke,fill}%
\end{pgfscope}%
\begin{pgfscope}%
\pgfpathrectangle{\pgfqpoint{0.550713in}{3.620038in}}{\pgfqpoint{4.791200in}{1.332187in}}%
\pgfusepath{clip}%
\pgfsetbuttcap%
\pgfsetroundjoin%
\definecolor{currentfill}{rgb}{0.560784,0.133333,0.243137}%
\pgfsetfillcolor{currentfill}%
\pgfsetlinewidth{0.752812pt}%
\definecolor{currentstroke}{rgb}{0.152941,0.152941,0.152941}%
\pgfsetstrokecolor{currentstroke}%
\pgfsetdash{}{0pt}%
\pgfpathmoveto{\pgfqpoint{0.903345in}{3.903031in}}%
\pgfpathlineto{\pgfqpoint{1.028555in}{3.903031in}}%
\pgfpathlineto{\pgfqpoint{1.028555in}{3.969858in}}%
\pgfpathlineto{\pgfqpoint{0.903345in}{3.969858in}}%
\pgfpathclose%
\pgfusepath{stroke,fill}%
\end{pgfscope}%
\begin{pgfscope}%
\pgfpathrectangle{\pgfqpoint{0.550713in}{3.620038in}}{\pgfqpoint{4.791200in}{1.332187in}}%
\pgfusepath{clip}%
\pgfsetbuttcap%
\pgfsetroundjoin%
\definecolor{currentfill}{rgb}{0.902991,0.874940,0.898900}%
\pgfsetfillcolor{currentfill}%
\pgfsetlinewidth{0.752812pt}%
\definecolor{currentstroke}{rgb}{0.152941,0.152941,0.152941}%
\pgfsetstrokecolor{currentstroke}%
\pgfsetdash{}{0pt}%
\pgfpathmoveto{\pgfqpoint{1.078064in}{3.696734in}}%
\pgfpathlineto{\pgfqpoint{1.109366in}{3.696734in}}%
\pgfpathlineto{\pgfqpoint{1.109366in}{3.745651in}}%
\pgfpathlineto{\pgfqpoint{1.078064in}{3.745651in}}%
\pgfpathclose%
\pgfusepath{stroke,fill}%
\end{pgfscope}%
\begin{pgfscope}%
\pgfpathrectangle{\pgfqpoint{0.550713in}{3.620038in}}{\pgfqpoint{4.791200in}{1.332187in}}%
\pgfusepath{clip}%
\pgfsetbuttcap%
\pgfsetroundjoin%
\definecolor{currentfill}{rgb}{0.627092,0.519263,0.611367}%
\pgfsetfillcolor{currentfill}%
\pgfsetlinewidth{0.752812pt}%
\definecolor{currentstroke}{rgb}{0.152941,0.152941,0.152941}%
\pgfsetstrokecolor{currentstroke}%
\pgfsetdash{}{0pt}%
\pgfpathmoveto{\pgfqpoint{1.062413in}{3.697371in}}%
\pgfpathlineto{\pgfqpoint{1.125018in}{3.697371in}}%
\pgfpathlineto{\pgfqpoint{1.125018in}{3.743202in}}%
\pgfpathlineto{\pgfqpoint{1.062413in}{3.743202in}}%
\pgfpathclose%
\pgfusepath{stroke,fill}%
\end{pgfscope}%
\begin{pgfscope}%
\pgfpathrectangle{\pgfqpoint{0.550713in}{3.620038in}}{\pgfqpoint{4.791200in}{1.332187in}}%
\pgfusepath{clip}%
\pgfsetbuttcap%
\pgfsetroundjoin%
\definecolor{currentfill}{rgb}{0.349020,0.160784,0.321569}%
\pgfsetfillcolor{currentfill}%
\pgfsetlinewidth{0.752812pt}%
\definecolor{currentstroke}{rgb}{0.152941,0.152941,0.152941}%
\pgfsetstrokecolor{currentstroke}%
\pgfsetdash{}{0pt}%
\pgfpathmoveto{\pgfqpoint{1.031110in}{3.698645in}}%
\pgfpathlineto{\pgfqpoint{1.156320in}{3.698645in}}%
\pgfpathlineto{\pgfqpoint{1.156320in}{3.738302in}}%
\pgfpathlineto{\pgfqpoint{1.031110in}{3.738302in}}%
\pgfpathclose%
\pgfusepath{stroke,fill}%
\end{pgfscope}%
\begin{pgfscope}%
\pgfpathrectangle{\pgfqpoint{0.550713in}{3.620038in}}{\pgfqpoint{4.791200in}{1.332187in}}%
\pgfusepath{clip}%
\pgfsetbuttcap%
\pgfsetroundjoin%
\definecolor{currentfill}{rgb}{0.934548,0.870850,0.887213}%
\pgfsetfillcolor{currentfill}%
\pgfsetlinewidth{0.752812pt}%
\definecolor{currentstroke}{rgb}{0.152941,0.152941,0.152941}%
\pgfsetstrokecolor{currentstroke}%
\pgfsetdash{}{0pt}%
\pgfpathmoveto{\pgfqpoint{1.269712in}{3.726746in}}%
\pgfpathlineto{\pgfqpoint{1.301014in}{3.726746in}}%
\pgfpathlineto{\pgfqpoint{1.301014in}{3.942291in}}%
\pgfpathlineto{\pgfqpoint{1.269712in}{3.942291in}}%
\pgfpathclose%
\pgfusepath{stroke,fill}%
\end{pgfscope}%
\begin{pgfscope}%
\pgfpathrectangle{\pgfqpoint{0.550713in}{3.620038in}}{\pgfqpoint{4.791200in}{1.332187in}}%
\pgfusepath{clip}%
\pgfsetbuttcap%
\pgfsetroundjoin%
\definecolor{currentfill}{rgb}{0.748399,0.503538,0.566438}%
\pgfsetfillcolor{currentfill}%
\pgfsetlinewidth{0.752812pt}%
\definecolor{currentstroke}{rgb}{0.152941,0.152941,0.152941}%
\pgfsetstrokecolor{currentstroke}%
\pgfsetdash{}{0pt}%
\pgfpathmoveto{\pgfqpoint{1.254061in}{3.726932in}}%
\pgfpathlineto{\pgfqpoint{1.316666in}{3.726932in}}%
\pgfpathlineto{\pgfqpoint{1.316666in}{3.893843in}}%
\pgfpathlineto{\pgfqpoint{1.254061in}{3.893843in}}%
\pgfpathclose%
\pgfusepath{stroke,fill}%
\end{pgfscope}%
\begin{pgfscope}%
\pgfpathrectangle{\pgfqpoint{0.550713in}{3.620038in}}{\pgfqpoint{4.791200in}{1.332187in}}%
\pgfusepath{clip}%
\pgfsetbuttcap%
\pgfsetroundjoin%
\definecolor{currentfill}{rgb}{0.560784,0.133333,0.243137}%
\pgfsetfillcolor{currentfill}%
\pgfsetlinewidth{0.752812pt}%
\definecolor{currentstroke}{rgb}{0.152941,0.152941,0.152941}%
\pgfsetstrokecolor{currentstroke}%
\pgfsetdash{}{0pt}%
\pgfpathmoveto{\pgfqpoint{1.222758in}{3.727303in}}%
\pgfpathlineto{\pgfqpoint{1.347968in}{3.727303in}}%
\pgfpathlineto{\pgfqpoint{1.347968in}{3.796947in}}%
\pgfpathlineto{\pgfqpoint{1.222758in}{3.796947in}}%
\pgfpathclose%
\pgfusepath{stroke,fill}%
\end{pgfscope}%
\begin{pgfscope}%
\pgfpathrectangle{\pgfqpoint{0.550713in}{3.620038in}}{\pgfqpoint{4.791200in}{1.332187in}}%
\pgfusepath{clip}%
\pgfsetbuttcap%
\pgfsetroundjoin%
\definecolor{currentfill}{rgb}{0.902991,0.874940,0.898900}%
\pgfsetfillcolor{currentfill}%
\pgfsetlinewidth{0.752812pt}%
\definecolor{currentstroke}{rgb}{0.152941,0.152941,0.152941}%
\pgfsetstrokecolor{currentstroke}%
\pgfsetdash{}{0pt}%
\pgfpathmoveto{\pgfqpoint{1.397477in}{3.697606in}}%
\pgfpathlineto{\pgfqpoint{1.428780in}{3.697606in}}%
\pgfpathlineto{\pgfqpoint{1.428780in}{3.720940in}}%
\pgfpathlineto{\pgfqpoint{1.397477in}{3.720940in}}%
\pgfpathclose%
\pgfusepath{stroke,fill}%
\end{pgfscope}%
\begin{pgfscope}%
\pgfpathrectangle{\pgfqpoint{0.550713in}{3.620038in}}{\pgfqpoint{4.791200in}{1.332187in}}%
\pgfusepath{clip}%
\pgfsetbuttcap%
\pgfsetroundjoin%
\definecolor{currentfill}{rgb}{0.627092,0.519263,0.611367}%
\pgfsetfillcolor{currentfill}%
\pgfsetlinewidth{0.752812pt}%
\definecolor{currentstroke}{rgb}{0.152941,0.152941,0.152941}%
\pgfsetstrokecolor{currentstroke}%
\pgfsetdash{}{0pt}%
\pgfpathmoveto{\pgfqpoint{1.381826in}{3.701455in}}%
\pgfpathlineto{\pgfqpoint{1.444431in}{3.701455in}}%
\pgfpathlineto{\pgfqpoint{1.444431in}{3.718473in}}%
\pgfpathlineto{\pgfqpoint{1.381826in}{3.718473in}}%
\pgfpathclose%
\pgfusepath{stroke,fill}%
\end{pgfscope}%
\begin{pgfscope}%
\pgfpathrectangle{\pgfqpoint{0.550713in}{3.620038in}}{\pgfqpoint{4.791200in}{1.332187in}}%
\pgfusepath{clip}%
\pgfsetbuttcap%
\pgfsetroundjoin%
\definecolor{currentfill}{rgb}{0.349020,0.160784,0.321569}%
\pgfsetfillcolor{currentfill}%
\pgfsetlinewidth{0.752812pt}%
\definecolor{currentstroke}{rgb}{0.152941,0.152941,0.152941}%
\pgfsetstrokecolor{currentstroke}%
\pgfsetdash{}{0pt}%
\pgfpathmoveto{\pgfqpoint{1.350524in}{3.709154in}}%
\pgfpathlineto{\pgfqpoint{1.475734in}{3.709154in}}%
\pgfpathlineto{\pgfqpoint{1.475734in}{3.713539in}}%
\pgfpathlineto{\pgfqpoint{1.350524in}{3.713539in}}%
\pgfpathclose%
\pgfusepath{stroke,fill}%
\end{pgfscope}%
\begin{pgfscope}%
\pgfpathrectangle{\pgfqpoint{0.550713in}{3.620038in}}{\pgfqpoint{4.791200in}{1.332187in}}%
\pgfusepath{clip}%
\pgfsetbuttcap%
\pgfsetroundjoin%
\definecolor{currentfill}{rgb}{0.934548,0.870850,0.887213}%
\pgfsetfillcolor{currentfill}%
\pgfsetlinewidth{0.752812pt}%
\definecolor{currentstroke}{rgb}{0.152941,0.152941,0.152941}%
\pgfsetstrokecolor{currentstroke}%
\pgfsetdash{}{0pt}%
\pgfpathmoveto{\pgfqpoint{1.589125in}{3.737527in}}%
\pgfpathlineto{\pgfqpoint{1.620428in}{3.737527in}}%
\pgfpathlineto{\pgfqpoint{1.620428in}{3.909638in}}%
\pgfpathlineto{\pgfqpoint{1.589125in}{3.909638in}}%
\pgfpathclose%
\pgfusepath{stroke,fill}%
\end{pgfscope}%
\begin{pgfscope}%
\pgfpathrectangle{\pgfqpoint{0.550713in}{3.620038in}}{\pgfqpoint{4.791200in}{1.332187in}}%
\pgfusepath{clip}%
\pgfsetbuttcap%
\pgfsetroundjoin%
\definecolor{currentfill}{rgb}{0.748399,0.503538,0.566438}%
\pgfsetfillcolor{currentfill}%
\pgfsetlinewidth{0.752812pt}%
\definecolor{currentstroke}{rgb}{0.152941,0.152941,0.152941}%
\pgfsetstrokecolor{currentstroke}%
\pgfsetdash{}{0pt}%
\pgfpathmoveto{\pgfqpoint{1.573474in}{3.746197in}}%
\pgfpathlineto{\pgfqpoint{1.636079in}{3.746197in}}%
\pgfpathlineto{\pgfqpoint{1.636079in}{3.903535in}}%
\pgfpathlineto{\pgfqpoint{1.573474in}{3.903535in}}%
\pgfpathclose%
\pgfusepath{stroke,fill}%
\end{pgfscope}%
\begin{pgfscope}%
\pgfpathrectangle{\pgfqpoint{0.550713in}{3.620038in}}{\pgfqpoint{4.791200in}{1.332187in}}%
\pgfusepath{clip}%
\pgfsetbuttcap%
\pgfsetroundjoin%
\definecolor{currentfill}{rgb}{0.560784,0.133333,0.243137}%
\pgfsetfillcolor{currentfill}%
\pgfsetlinewidth{0.752812pt}%
\definecolor{currentstroke}{rgb}{0.152941,0.152941,0.152941}%
\pgfsetstrokecolor{currentstroke}%
\pgfsetdash{}{0pt}%
\pgfpathmoveto{\pgfqpoint{1.542172in}{3.763538in}}%
\pgfpathlineto{\pgfqpoint{1.667382in}{3.763538in}}%
\pgfpathlineto{\pgfqpoint{1.667382in}{3.891327in}}%
\pgfpathlineto{\pgfqpoint{1.542172in}{3.891327in}}%
\pgfpathclose%
\pgfusepath{stroke,fill}%
\end{pgfscope}%
\begin{pgfscope}%
\pgfpathrectangle{\pgfqpoint{0.550713in}{3.620038in}}{\pgfqpoint{4.791200in}{1.332187in}}%
\pgfusepath{clip}%
\pgfsetbuttcap%
\pgfsetroundjoin%
\definecolor{currentfill}{rgb}{0.902991,0.874940,0.898900}%
\pgfsetfillcolor{currentfill}%
\pgfsetlinewidth{0.752812pt}%
\definecolor{currentstroke}{rgb}{0.152941,0.152941,0.152941}%
\pgfsetstrokecolor{currentstroke}%
\pgfsetdash{}{0pt}%
\pgfpathmoveto{\pgfqpoint{1.716891in}{3.690569in}}%
\pgfpathlineto{\pgfqpoint{1.748193in}{3.690569in}}%
\pgfpathlineto{\pgfqpoint{1.748193in}{3.729628in}}%
\pgfpathlineto{\pgfqpoint{1.716891in}{3.729628in}}%
\pgfpathclose%
\pgfusepath{stroke,fill}%
\end{pgfscope}%
\begin{pgfscope}%
\pgfpathrectangle{\pgfqpoint{0.550713in}{3.620038in}}{\pgfqpoint{4.791200in}{1.332187in}}%
\pgfusepath{clip}%
\pgfsetbuttcap%
\pgfsetroundjoin%
\definecolor{currentfill}{rgb}{0.627092,0.519263,0.611367}%
\pgfsetfillcolor{currentfill}%
\pgfsetlinewidth{0.752812pt}%
\definecolor{currentstroke}{rgb}{0.152941,0.152941,0.152941}%
\pgfsetstrokecolor{currentstroke}%
\pgfsetdash{}{0pt}%
\pgfpathmoveto{\pgfqpoint{1.701239in}{3.693005in}}%
\pgfpathlineto{\pgfqpoint{1.763844in}{3.693005in}}%
\pgfpathlineto{\pgfqpoint{1.763844in}{3.726293in}}%
\pgfpathlineto{\pgfqpoint{1.701239in}{3.726293in}}%
\pgfpathclose%
\pgfusepath{stroke,fill}%
\end{pgfscope}%
\begin{pgfscope}%
\pgfpathrectangle{\pgfqpoint{0.550713in}{3.620038in}}{\pgfqpoint{4.791200in}{1.332187in}}%
\pgfusepath{clip}%
\pgfsetbuttcap%
\pgfsetroundjoin%
\definecolor{currentfill}{rgb}{0.349020,0.160784,0.321569}%
\pgfsetfillcolor{currentfill}%
\pgfsetlinewidth{0.752812pt}%
\definecolor{currentstroke}{rgb}{0.152941,0.152941,0.152941}%
\pgfsetstrokecolor{currentstroke}%
\pgfsetdash{}{0pt}%
\pgfpathmoveto{\pgfqpoint{1.669937in}{3.697877in}}%
\pgfpathlineto{\pgfqpoint{1.795147in}{3.697877in}}%
\pgfpathlineto{\pgfqpoint{1.795147in}{3.719624in}}%
\pgfpathlineto{\pgfqpoint{1.669937in}{3.719624in}}%
\pgfpathclose%
\pgfusepath{stroke,fill}%
\end{pgfscope}%
\begin{pgfscope}%
\pgfpathrectangle{\pgfqpoint{0.550713in}{3.620038in}}{\pgfqpoint{4.791200in}{1.332187in}}%
\pgfusepath{clip}%
\pgfsetbuttcap%
\pgfsetroundjoin%
\definecolor{currentfill}{rgb}{0.934548,0.870850,0.887213}%
\pgfsetfillcolor{currentfill}%
\pgfsetlinewidth{0.752812pt}%
\definecolor{currentstroke}{rgb}{0.152941,0.152941,0.152941}%
\pgfsetstrokecolor{currentstroke}%
\pgfsetdash{}{0pt}%
\pgfpathmoveto{\pgfqpoint{1.908539in}{3.733892in}}%
\pgfpathlineto{\pgfqpoint{1.939841in}{3.733892in}}%
\pgfpathlineto{\pgfqpoint{1.939841in}{3.891857in}}%
\pgfpathlineto{\pgfqpoint{1.908539in}{3.891857in}}%
\pgfpathclose%
\pgfusepath{stroke,fill}%
\end{pgfscope}%
\begin{pgfscope}%
\pgfpathrectangle{\pgfqpoint{0.550713in}{3.620038in}}{\pgfqpoint{4.791200in}{1.332187in}}%
\pgfusepath{clip}%
\pgfsetbuttcap%
\pgfsetroundjoin%
\definecolor{currentfill}{rgb}{0.748399,0.503538,0.566438}%
\pgfsetfillcolor{currentfill}%
\pgfsetlinewidth{0.752812pt}%
\definecolor{currentstroke}{rgb}{0.152941,0.152941,0.152941}%
\pgfsetstrokecolor{currentstroke}%
\pgfsetdash{}{0pt}%
\pgfpathmoveto{\pgfqpoint{1.892887in}{3.740511in}}%
\pgfpathlineto{\pgfqpoint{1.955492in}{3.740511in}}%
\pgfpathlineto{\pgfqpoint{1.955492in}{3.865962in}}%
\pgfpathlineto{\pgfqpoint{1.892887in}{3.865962in}}%
\pgfpathclose%
\pgfusepath{stroke,fill}%
\end{pgfscope}%
\begin{pgfscope}%
\pgfpathrectangle{\pgfqpoint{0.550713in}{3.620038in}}{\pgfqpoint{4.791200in}{1.332187in}}%
\pgfusepath{clip}%
\pgfsetbuttcap%
\pgfsetroundjoin%
\definecolor{currentfill}{rgb}{0.560784,0.133333,0.243137}%
\pgfsetfillcolor{currentfill}%
\pgfsetlinewidth{0.752812pt}%
\definecolor{currentstroke}{rgb}{0.152941,0.152941,0.152941}%
\pgfsetstrokecolor{currentstroke}%
\pgfsetdash{}{0pt}%
\pgfpathmoveto{\pgfqpoint{1.861585in}{3.753748in}}%
\pgfpathlineto{\pgfqpoint{1.986795in}{3.753748in}}%
\pgfpathlineto{\pgfqpoint{1.986795in}{3.814172in}}%
\pgfpathlineto{\pgfqpoint{1.861585in}{3.814172in}}%
\pgfpathclose%
\pgfusepath{stroke,fill}%
\end{pgfscope}%
\begin{pgfscope}%
\pgfpathrectangle{\pgfqpoint{0.550713in}{3.620038in}}{\pgfqpoint{4.791200in}{1.332187in}}%
\pgfusepath{clip}%
\pgfsetbuttcap%
\pgfsetroundjoin%
\definecolor{currentfill}{rgb}{0.902991,0.874940,0.898900}%
\pgfsetfillcolor{currentfill}%
\pgfsetlinewidth{0.752812pt}%
\definecolor{currentstroke}{rgb}{0.152941,0.152941,0.152941}%
\pgfsetstrokecolor{currentstroke}%
\pgfsetdash{}{0pt}%
\pgfpathmoveto{\pgfqpoint{2.036304in}{3.698289in}}%
\pgfpathlineto{\pgfqpoint{2.067606in}{3.698289in}}%
\pgfpathlineto{\pgfqpoint{2.067606in}{3.752949in}}%
\pgfpathlineto{\pgfqpoint{2.036304in}{3.752949in}}%
\pgfpathclose%
\pgfusepath{stroke,fill}%
\end{pgfscope}%
\begin{pgfscope}%
\pgfpathrectangle{\pgfqpoint{0.550713in}{3.620038in}}{\pgfqpoint{4.791200in}{1.332187in}}%
\pgfusepath{clip}%
\pgfsetbuttcap%
\pgfsetroundjoin%
\definecolor{currentfill}{rgb}{0.627092,0.519263,0.611367}%
\pgfsetfillcolor{currentfill}%
\pgfsetlinewidth{0.752812pt}%
\definecolor{currentstroke}{rgb}{0.152941,0.152941,0.152941}%
\pgfsetstrokecolor{currentstroke}%
\pgfsetdash{}{0pt}%
\pgfpathmoveto{\pgfqpoint{2.020653in}{3.701993in}}%
\pgfpathlineto{\pgfqpoint{2.083258in}{3.701993in}}%
\pgfpathlineto{\pgfqpoint{2.083258in}{3.745730in}}%
\pgfpathlineto{\pgfqpoint{2.020653in}{3.745730in}}%
\pgfpathclose%
\pgfusepath{stroke,fill}%
\end{pgfscope}%
\begin{pgfscope}%
\pgfpathrectangle{\pgfqpoint{0.550713in}{3.620038in}}{\pgfqpoint{4.791200in}{1.332187in}}%
\pgfusepath{clip}%
\pgfsetbuttcap%
\pgfsetroundjoin%
\definecolor{currentfill}{rgb}{0.349020,0.160784,0.321569}%
\pgfsetfillcolor{currentfill}%
\pgfsetlinewidth{0.752812pt}%
\definecolor{currentstroke}{rgb}{0.152941,0.152941,0.152941}%
\pgfsetstrokecolor{currentstroke}%
\pgfsetdash{}{0pt}%
\pgfpathmoveto{\pgfqpoint{1.989350in}{3.709401in}}%
\pgfpathlineto{\pgfqpoint{2.114560in}{3.709401in}}%
\pgfpathlineto{\pgfqpoint{2.114560in}{3.731291in}}%
\pgfpathlineto{\pgfqpoint{1.989350in}{3.731291in}}%
\pgfpathclose%
\pgfusepath{stroke,fill}%
\end{pgfscope}%
\begin{pgfscope}%
\pgfpathrectangle{\pgfqpoint{0.550713in}{3.620038in}}{\pgfqpoint{4.791200in}{1.332187in}}%
\pgfusepath{clip}%
\pgfsetbuttcap%
\pgfsetroundjoin%
\definecolor{currentfill}{rgb}{0.934548,0.870850,0.887213}%
\pgfsetfillcolor{currentfill}%
\pgfsetlinewidth{0.752812pt}%
\definecolor{currentstroke}{rgb}{0.152941,0.152941,0.152941}%
\pgfsetstrokecolor{currentstroke}%
\pgfsetdash{}{0pt}%
\pgfpathmoveto{\pgfqpoint{2.227952in}{3.760850in}}%
\pgfpathlineto{\pgfqpoint{2.259254in}{3.760850in}}%
\pgfpathlineto{\pgfqpoint{2.259254in}{3.865892in}}%
\pgfpathlineto{\pgfqpoint{2.227952in}{3.865892in}}%
\pgfpathclose%
\pgfusepath{stroke,fill}%
\end{pgfscope}%
\begin{pgfscope}%
\pgfpathrectangle{\pgfqpoint{0.550713in}{3.620038in}}{\pgfqpoint{4.791200in}{1.332187in}}%
\pgfusepath{clip}%
\pgfsetbuttcap%
\pgfsetroundjoin%
\definecolor{currentfill}{rgb}{0.748399,0.503538,0.566438}%
\pgfsetfillcolor{currentfill}%
\pgfsetlinewidth{0.752812pt}%
\definecolor{currentstroke}{rgb}{0.152941,0.152941,0.152941}%
\pgfsetstrokecolor{currentstroke}%
\pgfsetdash{}{0pt}%
\pgfpathmoveto{\pgfqpoint{2.212301in}{3.762918in}}%
\pgfpathlineto{\pgfqpoint{2.274906in}{3.762918in}}%
\pgfpathlineto{\pgfqpoint{2.274906in}{3.849636in}}%
\pgfpathlineto{\pgfqpoint{2.212301in}{3.849636in}}%
\pgfpathclose%
\pgfusepath{stroke,fill}%
\end{pgfscope}%
\begin{pgfscope}%
\pgfpathrectangle{\pgfqpoint{0.550713in}{3.620038in}}{\pgfqpoint{4.791200in}{1.332187in}}%
\pgfusepath{clip}%
\pgfsetbuttcap%
\pgfsetroundjoin%
\definecolor{currentfill}{rgb}{0.560784,0.133333,0.243137}%
\pgfsetfillcolor{currentfill}%
\pgfsetlinewidth{0.752812pt}%
\definecolor{currentstroke}{rgb}{0.152941,0.152941,0.152941}%
\pgfsetstrokecolor{currentstroke}%
\pgfsetdash{}{0pt}%
\pgfpathmoveto{\pgfqpoint{2.180998in}{3.767054in}}%
\pgfpathlineto{\pgfqpoint{2.306208in}{3.767054in}}%
\pgfpathlineto{\pgfqpoint{2.306208in}{3.817125in}}%
\pgfpathlineto{\pgfqpoint{2.180998in}{3.817125in}}%
\pgfpathclose%
\pgfusepath{stroke,fill}%
\end{pgfscope}%
\begin{pgfscope}%
\pgfpathrectangle{\pgfqpoint{0.550713in}{3.620038in}}{\pgfqpoint{4.791200in}{1.332187in}}%
\pgfusepath{clip}%
\pgfsetbuttcap%
\pgfsetroundjoin%
\definecolor{currentfill}{rgb}{0.902991,0.874940,0.898900}%
\pgfsetfillcolor{currentfill}%
\pgfsetlinewidth{0.752812pt}%
\definecolor{currentstroke}{rgb}{0.152941,0.152941,0.152941}%
\pgfsetstrokecolor{currentstroke}%
\pgfsetdash{}{0pt}%
\pgfpathmoveto{\pgfqpoint{2.355717in}{3.734802in}}%
\pgfpathlineto{\pgfqpoint{2.387020in}{3.734802in}}%
\pgfpathlineto{\pgfqpoint{2.387020in}{3.765414in}}%
\pgfpathlineto{\pgfqpoint{2.355717in}{3.765414in}}%
\pgfpathclose%
\pgfusepath{stroke,fill}%
\end{pgfscope}%
\begin{pgfscope}%
\pgfpathrectangle{\pgfqpoint{0.550713in}{3.620038in}}{\pgfqpoint{4.791200in}{1.332187in}}%
\pgfusepath{clip}%
\pgfsetbuttcap%
\pgfsetroundjoin%
\definecolor{currentfill}{rgb}{0.627092,0.519263,0.611367}%
\pgfsetfillcolor{currentfill}%
\pgfsetlinewidth{0.752812pt}%
\definecolor{currentstroke}{rgb}{0.152941,0.152941,0.152941}%
\pgfsetstrokecolor{currentstroke}%
\pgfsetdash{}{0pt}%
\pgfpathmoveto{\pgfqpoint{2.340066in}{3.738576in}}%
\pgfpathlineto{\pgfqpoint{2.402671in}{3.738576in}}%
\pgfpathlineto{\pgfqpoint{2.402671in}{3.765299in}}%
\pgfpathlineto{\pgfqpoint{2.340066in}{3.765299in}}%
\pgfpathclose%
\pgfusepath{stroke,fill}%
\end{pgfscope}%
\begin{pgfscope}%
\pgfpathrectangle{\pgfqpoint{0.550713in}{3.620038in}}{\pgfqpoint{4.791200in}{1.332187in}}%
\pgfusepath{clip}%
\pgfsetbuttcap%
\pgfsetroundjoin%
\definecolor{currentfill}{rgb}{0.349020,0.160784,0.321569}%
\pgfsetfillcolor{currentfill}%
\pgfsetlinewidth{0.752812pt}%
\definecolor{currentstroke}{rgb}{0.152941,0.152941,0.152941}%
\pgfsetstrokecolor{currentstroke}%
\pgfsetdash{}{0pt}%
\pgfpathmoveto{\pgfqpoint{2.308763in}{3.746124in}}%
\pgfpathlineto{\pgfqpoint{2.433973in}{3.746124in}}%
\pgfpathlineto{\pgfqpoint{2.433973in}{3.765068in}}%
\pgfpathlineto{\pgfqpoint{2.308763in}{3.765068in}}%
\pgfpathclose%
\pgfusepath{stroke,fill}%
\end{pgfscope}%
\begin{pgfscope}%
\pgfpathrectangle{\pgfqpoint{0.550713in}{3.620038in}}{\pgfqpoint{4.791200in}{1.332187in}}%
\pgfusepath{clip}%
\pgfsetbuttcap%
\pgfsetroundjoin%
\definecolor{currentfill}{rgb}{0.934548,0.870850,0.887213}%
\pgfsetfillcolor{currentfill}%
\pgfsetlinewidth{0.752812pt}%
\definecolor{currentstroke}{rgb}{0.152941,0.152941,0.152941}%
\pgfsetstrokecolor{currentstroke}%
\pgfsetdash{}{0pt}%
\pgfpathmoveto{\pgfqpoint{2.547365in}{3.753645in}}%
\pgfpathlineto{\pgfqpoint{2.578668in}{3.753645in}}%
\pgfpathlineto{\pgfqpoint{2.578668in}{3.849077in}}%
\pgfpathlineto{\pgfqpoint{2.547365in}{3.849077in}}%
\pgfpathclose%
\pgfusepath{stroke,fill}%
\end{pgfscope}%
\begin{pgfscope}%
\pgfpathrectangle{\pgfqpoint{0.550713in}{3.620038in}}{\pgfqpoint{4.791200in}{1.332187in}}%
\pgfusepath{clip}%
\pgfsetbuttcap%
\pgfsetroundjoin%
\definecolor{currentfill}{rgb}{0.748399,0.503538,0.566438}%
\pgfsetfillcolor{currentfill}%
\pgfsetlinewidth{0.752812pt}%
\definecolor{currentstroke}{rgb}{0.152941,0.152941,0.152941}%
\pgfsetstrokecolor{currentstroke}%
\pgfsetdash{}{0pt}%
\pgfpathmoveto{\pgfqpoint{2.531714in}{3.760379in}}%
\pgfpathlineto{\pgfqpoint{2.594319in}{3.760379in}}%
\pgfpathlineto{\pgfqpoint{2.594319in}{3.838410in}}%
\pgfpathlineto{\pgfqpoint{2.531714in}{3.838410in}}%
\pgfpathclose%
\pgfusepath{stroke,fill}%
\end{pgfscope}%
\begin{pgfscope}%
\pgfpathrectangle{\pgfqpoint{0.550713in}{3.620038in}}{\pgfqpoint{4.791200in}{1.332187in}}%
\pgfusepath{clip}%
\pgfsetbuttcap%
\pgfsetroundjoin%
\definecolor{currentfill}{rgb}{0.560784,0.133333,0.243137}%
\pgfsetfillcolor{currentfill}%
\pgfsetlinewidth{0.752812pt}%
\definecolor{currentstroke}{rgb}{0.152941,0.152941,0.152941}%
\pgfsetstrokecolor{currentstroke}%
\pgfsetdash{}{0pt}%
\pgfpathmoveto{\pgfqpoint{2.500411in}{3.773849in}}%
\pgfpathlineto{\pgfqpoint{2.625621in}{3.773849in}}%
\pgfpathlineto{\pgfqpoint{2.625621in}{3.817076in}}%
\pgfpathlineto{\pgfqpoint{2.500411in}{3.817076in}}%
\pgfpathclose%
\pgfusepath{stroke,fill}%
\end{pgfscope}%
\begin{pgfscope}%
\pgfpathrectangle{\pgfqpoint{0.550713in}{3.620038in}}{\pgfqpoint{4.791200in}{1.332187in}}%
\pgfusepath{clip}%
\pgfsetbuttcap%
\pgfsetroundjoin%
\definecolor{currentfill}{rgb}{0.902991,0.874940,0.898900}%
\pgfsetfillcolor{currentfill}%
\pgfsetlinewidth{0.752812pt}%
\definecolor{currentstroke}{rgb}{0.152941,0.152941,0.152941}%
\pgfsetstrokecolor{currentstroke}%
\pgfsetdash{}{0pt}%
\pgfpathmoveto{\pgfqpoint{2.675131in}{3.743982in}}%
\pgfpathlineto{\pgfqpoint{2.706433in}{3.743982in}}%
\pgfpathlineto{\pgfqpoint{2.706433in}{3.812532in}}%
\pgfpathlineto{\pgfqpoint{2.675131in}{3.812532in}}%
\pgfpathclose%
\pgfusepath{stroke,fill}%
\end{pgfscope}%
\begin{pgfscope}%
\pgfpathrectangle{\pgfqpoint{0.550713in}{3.620038in}}{\pgfqpoint{4.791200in}{1.332187in}}%
\pgfusepath{clip}%
\pgfsetbuttcap%
\pgfsetroundjoin%
\definecolor{currentfill}{rgb}{0.627092,0.519263,0.611367}%
\pgfsetfillcolor{currentfill}%
\pgfsetlinewidth{0.752812pt}%
\definecolor{currentstroke}{rgb}{0.152941,0.152941,0.152941}%
\pgfsetstrokecolor{currentstroke}%
\pgfsetdash{}{0pt}%
\pgfpathmoveto{\pgfqpoint{2.659479in}{3.751122in}}%
\pgfpathlineto{\pgfqpoint{2.722084in}{3.751122in}}%
\pgfpathlineto{\pgfqpoint{2.722084in}{3.809646in}}%
\pgfpathlineto{\pgfqpoint{2.659479in}{3.809646in}}%
\pgfpathclose%
\pgfusepath{stroke,fill}%
\end{pgfscope}%
\begin{pgfscope}%
\pgfpathrectangle{\pgfqpoint{0.550713in}{3.620038in}}{\pgfqpoint{4.791200in}{1.332187in}}%
\pgfusepath{clip}%
\pgfsetbuttcap%
\pgfsetroundjoin%
\definecolor{currentfill}{rgb}{0.349020,0.160784,0.321569}%
\pgfsetfillcolor{currentfill}%
\pgfsetlinewidth{0.752812pt}%
\definecolor{currentstroke}{rgb}{0.152941,0.152941,0.152941}%
\pgfsetstrokecolor{currentstroke}%
\pgfsetdash{}{0pt}%
\pgfpathmoveto{\pgfqpoint{2.628177in}{3.765402in}}%
\pgfpathlineto{\pgfqpoint{2.753387in}{3.765402in}}%
\pgfpathlineto{\pgfqpoint{2.753387in}{3.803874in}}%
\pgfpathlineto{\pgfqpoint{2.628177in}{3.803874in}}%
\pgfpathclose%
\pgfusepath{stroke,fill}%
\end{pgfscope}%
\begin{pgfscope}%
\pgfpathrectangle{\pgfqpoint{0.550713in}{3.620038in}}{\pgfqpoint{4.791200in}{1.332187in}}%
\pgfusepath{clip}%
\pgfsetbuttcap%
\pgfsetroundjoin%
\definecolor{currentfill}{rgb}{0.934548,0.870850,0.887213}%
\pgfsetfillcolor{currentfill}%
\pgfsetlinewidth{0.752812pt}%
\definecolor{currentstroke}{rgb}{0.152941,0.152941,0.152941}%
\pgfsetstrokecolor{currentstroke}%
\pgfsetdash{}{0pt}%
\pgfpathmoveto{\pgfqpoint{2.866779in}{3.765875in}}%
\pgfpathlineto{\pgfqpoint{2.898081in}{3.765875in}}%
\pgfpathlineto{\pgfqpoint{2.898081in}{3.800379in}}%
\pgfpathlineto{\pgfqpoint{2.866779in}{3.800379in}}%
\pgfpathclose%
\pgfusepath{stroke,fill}%
\end{pgfscope}%
\begin{pgfscope}%
\pgfpathrectangle{\pgfqpoint{0.550713in}{3.620038in}}{\pgfqpoint{4.791200in}{1.332187in}}%
\pgfusepath{clip}%
\pgfsetbuttcap%
\pgfsetroundjoin%
\definecolor{currentfill}{rgb}{0.748399,0.503538,0.566438}%
\pgfsetfillcolor{currentfill}%
\pgfsetlinewidth{0.752812pt}%
\definecolor{currentstroke}{rgb}{0.152941,0.152941,0.152941}%
\pgfsetstrokecolor{currentstroke}%
\pgfsetdash{}{0pt}%
\pgfpathmoveto{\pgfqpoint{2.851127in}{3.769632in}}%
\pgfpathlineto{\pgfqpoint{2.913732in}{3.769632in}}%
\pgfpathlineto{\pgfqpoint{2.913732in}{3.799598in}}%
\pgfpathlineto{\pgfqpoint{2.851127in}{3.799598in}}%
\pgfpathclose%
\pgfusepath{stroke,fill}%
\end{pgfscope}%
\begin{pgfscope}%
\pgfpathrectangle{\pgfqpoint{0.550713in}{3.620038in}}{\pgfqpoint{4.791200in}{1.332187in}}%
\pgfusepath{clip}%
\pgfsetbuttcap%
\pgfsetroundjoin%
\definecolor{currentfill}{rgb}{0.560784,0.133333,0.243137}%
\pgfsetfillcolor{currentfill}%
\pgfsetlinewidth{0.752812pt}%
\definecolor{currentstroke}{rgb}{0.152941,0.152941,0.152941}%
\pgfsetstrokecolor{currentstroke}%
\pgfsetdash{}{0pt}%
\pgfpathmoveto{\pgfqpoint{2.819825in}{3.777146in}}%
\pgfpathlineto{\pgfqpoint{2.945035in}{3.777146in}}%
\pgfpathlineto{\pgfqpoint{2.945035in}{3.798036in}}%
\pgfpathlineto{\pgfqpoint{2.819825in}{3.798036in}}%
\pgfpathclose%
\pgfusepath{stroke,fill}%
\end{pgfscope}%
\begin{pgfscope}%
\pgfpathrectangle{\pgfqpoint{0.550713in}{3.620038in}}{\pgfqpoint{4.791200in}{1.332187in}}%
\pgfusepath{clip}%
\pgfsetbuttcap%
\pgfsetroundjoin%
\definecolor{currentfill}{rgb}{0.902991,0.874940,0.898900}%
\pgfsetfillcolor{currentfill}%
\pgfsetlinewidth{0.752812pt}%
\definecolor{currentstroke}{rgb}{0.152941,0.152941,0.152941}%
\pgfsetstrokecolor{currentstroke}%
\pgfsetdash{}{0pt}%
\pgfpathmoveto{\pgfqpoint{2.994544in}{3.735886in}}%
\pgfpathlineto{\pgfqpoint{3.025846in}{3.735886in}}%
\pgfpathlineto{\pgfqpoint{3.025846in}{3.862389in}}%
\pgfpathlineto{\pgfqpoint{2.994544in}{3.862389in}}%
\pgfpathclose%
\pgfusepath{stroke,fill}%
\end{pgfscope}%
\begin{pgfscope}%
\pgfpathrectangle{\pgfqpoint{0.550713in}{3.620038in}}{\pgfqpoint{4.791200in}{1.332187in}}%
\pgfusepath{clip}%
\pgfsetbuttcap%
\pgfsetroundjoin%
\definecolor{currentfill}{rgb}{0.627092,0.519263,0.611367}%
\pgfsetfillcolor{currentfill}%
\pgfsetlinewidth{0.752812pt}%
\definecolor{currentstroke}{rgb}{0.152941,0.152941,0.152941}%
\pgfsetstrokecolor{currentstroke}%
\pgfsetdash{}{0pt}%
\pgfpathmoveto{\pgfqpoint{2.978893in}{3.765233in}}%
\pgfpathlineto{\pgfqpoint{3.041498in}{3.765233in}}%
\pgfpathlineto{\pgfqpoint{3.041498in}{3.859147in}}%
\pgfpathlineto{\pgfqpoint{2.978893in}{3.859147in}}%
\pgfpathclose%
\pgfusepath{stroke,fill}%
\end{pgfscope}%
\begin{pgfscope}%
\pgfpathrectangle{\pgfqpoint{0.550713in}{3.620038in}}{\pgfqpoint{4.791200in}{1.332187in}}%
\pgfusepath{clip}%
\pgfsetbuttcap%
\pgfsetroundjoin%
\definecolor{currentfill}{rgb}{0.349020,0.160784,0.321569}%
\pgfsetfillcolor{currentfill}%
\pgfsetlinewidth{0.752812pt}%
\definecolor{currentstroke}{rgb}{0.152941,0.152941,0.152941}%
\pgfsetstrokecolor{currentstroke}%
\pgfsetdash{}{0pt}%
\pgfpathmoveto{\pgfqpoint{2.947590in}{3.823927in}}%
\pgfpathlineto{\pgfqpoint{3.072800in}{3.823927in}}%
\pgfpathlineto{\pgfqpoint{3.072800in}{3.852663in}}%
\pgfpathlineto{\pgfqpoint{2.947590in}{3.852663in}}%
\pgfpathclose%
\pgfusepath{stroke,fill}%
\end{pgfscope}%
\begin{pgfscope}%
\pgfpathrectangle{\pgfqpoint{0.550713in}{3.620038in}}{\pgfqpoint{4.791200in}{1.332187in}}%
\pgfusepath{clip}%
\pgfsetbuttcap%
\pgfsetroundjoin%
\definecolor{currentfill}{rgb}{0.934548,0.870850,0.887213}%
\pgfsetfillcolor{currentfill}%
\pgfsetlinewidth{0.752812pt}%
\definecolor{currentstroke}{rgb}{0.152941,0.152941,0.152941}%
\pgfsetstrokecolor{currentstroke}%
\pgfsetdash{}{0pt}%
\pgfpathmoveto{\pgfqpoint{3.186192in}{3.772627in}}%
\pgfpathlineto{\pgfqpoint{3.217494in}{3.772627in}}%
\pgfpathlineto{\pgfqpoint{3.217494in}{3.873416in}}%
\pgfpathlineto{\pgfqpoint{3.186192in}{3.873416in}}%
\pgfpathclose%
\pgfusepath{stroke,fill}%
\end{pgfscope}%
\begin{pgfscope}%
\pgfpathrectangle{\pgfqpoint{0.550713in}{3.620038in}}{\pgfqpoint{4.791200in}{1.332187in}}%
\pgfusepath{clip}%
\pgfsetbuttcap%
\pgfsetroundjoin%
\definecolor{currentfill}{rgb}{0.748399,0.503538,0.566438}%
\pgfsetfillcolor{currentfill}%
\pgfsetlinewidth{0.752812pt}%
\definecolor{currentstroke}{rgb}{0.152941,0.152941,0.152941}%
\pgfsetstrokecolor{currentstroke}%
\pgfsetdash{}{0pt}%
\pgfpathmoveto{\pgfqpoint{3.170541in}{3.781810in}}%
\pgfpathlineto{\pgfqpoint{3.233146in}{3.781810in}}%
\pgfpathlineto{\pgfqpoint{3.233146in}{3.856178in}}%
\pgfpathlineto{\pgfqpoint{3.170541in}{3.856178in}}%
\pgfpathclose%
\pgfusepath{stroke,fill}%
\end{pgfscope}%
\begin{pgfscope}%
\pgfpathrectangle{\pgfqpoint{0.550713in}{3.620038in}}{\pgfqpoint{4.791200in}{1.332187in}}%
\pgfusepath{clip}%
\pgfsetbuttcap%
\pgfsetroundjoin%
\definecolor{currentfill}{rgb}{0.560784,0.133333,0.243137}%
\pgfsetfillcolor{currentfill}%
\pgfsetlinewidth{0.752812pt}%
\definecolor{currentstroke}{rgb}{0.152941,0.152941,0.152941}%
\pgfsetstrokecolor{currentstroke}%
\pgfsetdash{}{0pt}%
\pgfpathmoveto{\pgfqpoint{3.139238in}{3.800176in}}%
\pgfpathlineto{\pgfqpoint{3.264448in}{3.800176in}}%
\pgfpathlineto{\pgfqpoint{3.264448in}{3.821703in}}%
\pgfpathlineto{\pgfqpoint{3.139238in}{3.821703in}}%
\pgfpathclose%
\pgfusepath{stroke,fill}%
\end{pgfscope}%
\begin{pgfscope}%
\pgfpathrectangle{\pgfqpoint{0.550713in}{3.620038in}}{\pgfqpoint{4.791200in}{1.332187in}}%
\pgfusepath{clip}%
\pgfsetbuttcap%
\pgfsetroundjoin%
\definecolor{currentfill}{rgb}{0.902991,0.874940,0.898900}%
\pgfsetfillcolor{currentfill}%
\pgfsetlinewidth{0.752812pt}%
\definecolor{currentstroke}{rgb}{0.152941,0.152941,0.152941}%
\pgfsetstrokecolor{currentstroke}%
\pgfsetdash{}{0pt}%
\pgfpathmoveto{\pgfqpoint{3.313957in}{3.759716in}}%
\pgfpathlineto{\pgfqpoint{3.345260in}{3.759716in}}%
\pgfpathlineto{\pgfqpoint{3.345260in}{3.908570in}}%
\pgfpathlineto{\pgfqpoint{3.313957in}{3.908570in}}%
\pgfpathclose%
\pgfusepath{stroke,fill}%
\end{pgfscope}%
\begin{pgfscope}%
\pgfpathrectangle{\pgfqpoint{0.550713in}{3.620038in}}{\pgfqpoint{4.791200in}{1.332187in}}%
\pgfusepath{clip}%
\pgfsetbuttcap%
\pgfsetroundjoin%
\definecolor{currentfill}{rgb}{0.627092,0.519263,0.611367}%
\pgfsetfillcolor{currentfill}%
\pgfsetlinewidth{0.752812pt}%
\definecolor{currentstroke}{rgb}{0.152941,0.152941,0.152941}%
\pgfsetstrokecolor{currentstroke}%
\pgfsetdash{}{0pt}%
\pgfpathmoveto{\pgfqpoint{3.298306in}{3.798452in}}%
\pgfpathlineto{\pgfqpoint{3.360911in}{3.798452in}}%
\pgfpathlineto{\pgfqpoint{3.360911in}{3.908372in}}%
\pgfpathlineto{\pgfqpoint{3.298306in}{3.908372in}}%
\pgfpathclose%
\pgfusepath{stroke,fill}%
\end{pgfscope}%
\begin{pgfscope}%
\pgfpathrectangle{\pgfqpoint{0.550713in}{3.620038in}}{\pgfqpoint{4.791200in}{1.332187in}}%
\pgfusepath{clip}%
\pgfsetbuttcap%
\pgfsetroundjoin%
\definecolor{currentfill}{rgb}{0.349020,0.160784,0.321569}%
\pgfsetfillcolor{currentfill}%
\pgfsetlinewidth{0.752812pt}%
\definecolor{currentstroke}{rgb}{0.152941,0.152941,0.152941}%
\pgfsetstrokecolor{currentstroke}%
\pgfsetdash{}{0pt}%
\pgfpathmoveto{\pgfqpoint{3.267003in}{3.875923in}}%
\pgfpathlineto{\pgfqpoint{3.392213in}{3.875923in}}%
\pgfpathlineto{\pgfqpoint{3.392213in}{3.907975in}}%
\pgfpathlineto{\pgfqpoint{3.267003in}{3.907975in}}%
\pgfpathclose%
\pgfusepath{stroke,fill}%
\end{pgfscope}%
\begin{pgfscope}%
\pgfpathrectangle{\pgfqpoint{0.550713in}{3.620038in}}{\pgfqpoint{4.791200in}{1.332187in}}%
\pgfusepath{clip}%
\pgfsetbuttcap%
\pgfsetroundjoin%
\definecolor{currentfill}{rgb}{0.934548,0.870850,0.887213}%
\pgfsetfillcolor{currentfill}%
\pgfsetlinewidth{0.752812pt}%
\definecolor{currentstroke}{rgb}{0.152941,0.152941,0.152941}%
\pgfsetstrokecolor{currentstroke}%
\pgfsetdash{}{0pt}%
\pgfpathmoveto{\pgfqpoint{3.505605in}{3.812146in}}%
\pgfpathlineto{\pgfqpoint{3.536908in}{3.812146in}}%
\pgfpathlineto{\pgfqpoint{3.536908in}{3.918720in}}%
\pgfpathlineto{\pgfqpoint{3.505605in}{3.918720in}}%
\pgfpathclose%
\pgfusepath{stroke,fill}%
\end{pgfscope}%
\begin{pgfscope}%
\pgfpathrectangle{\pgfqpoint{0.550713in}{3.620038in}}{\pgfqpoint{4.791200in}{1.332187in}}%
\pgfusepath{clip}%
\pgfsetbuttcap%
\pgfsetroundjoin%
\definecolor{currentfill}{rgb}{0.748399,0.503538,0.566438}%
\pgfsetfillcolor{currentfill}%
\pgfsetlinewidth{0.752812pt}%
\definecolor{currentstroke}{rgb}{0.152941,0.152941,0.152941}%
\pgfsetstrokecolor{currentstroke}%
\pgfsetdash{}{0pt}%
\pgfpathmoveto{\pgfqpoint{3.489954in}{3.813258in}}%
\pgfpathlineto{\pgfqpoint{3.552559in}{3.813258in}}%
\pgfpathlineto{\pgfqpoint{3.552559in}{3.916509in}}%
\pgfpathlineto{\pgfqpoint{3.489954in}{3.916509in}}%
\pgfpathclose%
\pgfusepath{stroke,fill}%
\end{pgfscope}%
\begin{pgfscope}%
\pgfpathrectangle{\pgfqpoint{0.550713in}{3.620038in}}{\pgfqpoint{4.791200in}{1.332187in}}%
\pgfusepath{clip}%
\pgfsetbuttcap%
\pgfsetroundjoin%
\definecolor{currentfill}{rgb}{0.560784,0.133333,0.243137}%
\pgfsetfillcolor{currentfill}%
\pgfsetlinewidth{0.752812pt}%
\definecolor{currentstroke}{rgb}{0.152941,0.152941,0.152941}%
\pgfsetstrokecolor{currentstroke}%
\pgfsetdash{}{0pt}%
\pgfpathmoveto{\pgfqpoint{3.458651in}{3.815482in}}%
\pgfpathlineto{\pgfqpoint{3.583861in}{3.815482in}}%
\pgfpathlineto{\pgfqpoint{3.583861in}{3.912087in}}%
\pgfpathlineto{\pgfqpoint{3.458651in}{3.912087in}}%
\pgfpathclose%
\pgfusepath{stroke,fill}%
\end{pgfscope}%
\begin{pgfscope}%
\pgfpathrectangle{\pgfqpoint{0.550713in}{3.620038in}}{\pgfqpoint{4.791200in}{1.332187in}}%
\pgfusepath{clip}%
\pgfsetbuttcap%
\pgfsetroundjoin%
\definecolor{currentfill}{rgb}{0.902991,0.874940,0.898900}%
\pgfsetfillcolor{currentfill}%
\pgfsetlinewidth{0.752812pt}%
\definecolor{currentstroke}{rgb}{0.152941,0.152941,0.152941}%
\pgfsetstrokecolor{currentstroke}%
\pgfsetdash{}{0pt}%
\pgfpathmoveto{\pgfqpoint{3.633370in}{3.824995in}}%
\pgfpathlineto{\pgfqpoint{3.664673in}{3.824995in}}%
\pgfpathlineto{\pgfqpoint{3.664673in}{3.965899in}}%
\pgfpathlineto{\pgfqpoint{3.633370in}{3.965899in}}%
\pgfpathclose%
\pgfusepath{stroke,fill}%
\end{pgfscope}%
\begin{pgfscope}%
\pgfpathrectangle{\pgfqpoint{0.550713in}{3.620038in}}{\pgfqpoint{4.791200in}{1.332187in}}%
\pgfusepath{clip}%
\pgfsetbuttcap%
\pgfsetroundjoin%
\definecolor{currentfill}{rgb}{0.627092,0.519263,0.611367}%
\pgfsetfillcolor{currentfill}%
\pgfsetlinewidth{0.752812pt}%
\definecolor{currentstroke}{rgb}{0.152941,0.152941,0.152941}%
\pgfsetstrokecolor{currentstroke}%
\pgfsetdash{}{0pt}%
\pgfpathmoveto{\pgfqpoint{3.617719in}{3.856278in}}%
\pgfpathlineto{\pgfqpoint{3.680324in}{3.856278in}}%
\pgfpathlineto{\pgfqpoint{3.680324in}{3.962639in}}%
\pgfpathlineto{\pgfqpoint{3.617719in}{3.962639in}}%
\pgfpathclose%
\pgfusepath{stroke,fill}%
\end{pgfscope}%
\begin{pgfscope}%
\pgfpathrectangle{\pgfqpoint{0.550713in}{3.620038in}}{\pgfqpoint{4.791200in}{1.332187in}}%
\pgfusepath{clip}%
\pgfsetbuttcap%
\pgfsetroundjoin%
\definecolor{currentfill}{rgb}{0.349020,0.160784,0.321569}%
\pgfsetfillcolor{currentfill}%
\pgfsetlinewidth{0.752812pt}%
\definecolor{currentstroke}{rgb}{0.152941,0.152941,0.152941}%
\pgfsetstrokecolor{currentstroke}%
\pgfsetdash{}{0pt}%
\pgfpathmoveto{\pgfqpoint{3.586417in}{3.918845in}}%
\pgfpathlineto{\pgfqpoint{3.711627in}{3.918845in}}%
\pgfpathlineto{\pgfqpoint{3.711627in}{3.956120in}}%
\pgfpathlineto{\pgfqpoint{3.586417in}{3.956120in}}%
\pgfpathclose%
\pgfusepath{stroke,fill}%
\end{pgfscope}%
\begin{pgfscope}%
\pgfpathrectangle{\pgfqpoint{0.550713in}{3.620038in}}{\pgfqpoint{4.791200in}{1.332187in}}%
\pgfusepath{clip}%
\pgfsetbuttcap%
\pgfsetroundjoin%
\definecolor{currentfill}{rgb}{0.934548,0.870850,0.887213}%
\pgfsetfillcolor{currentfill}%
\pgfsetlinewidth{0.752812pt}%
\definecolor{currentstroke}{rgb}{0.152941,0.152941,0.152941}%
\pgfsetstrokecolor{currentstroke}%
\pgfsetdash{}{0pt}%
\pgfpathmoveto{\pgfqpoint{3.825018in}{3.842365in}}%
\pgfpathlineto{\pgfqpoint{3.856321in}{3.842365in}}%
\pgfpathlineto{\pgfqpoint{3.856321in}{3.925596in}}%
\pgfpathlineto{\pgfqpoint{3.825018in}{3.925596in}}%
\pgfpathclose%
\pgfusepath{stroke,fill}%
\end{pgfscope}%
\begin{pgfscope}%
\pgfpathrectangle{\pgfqpoint{0.550713in}{3.620038in}}{\pgfqpoint{4.791200in}{1.332187in}}%
\pgfusepath{clip}%
\pgfsetbuttcap%
\pgfsetroundjoin%
\definecolor{currentfill}{rgb}{0.748399,0.503538,0.566438}%
\pgfsetfillcolor{currentfill}%
\pgfsetlinewidth{0.752812pt}%
\definecolor{currentstroke}{rgb}{0.152941,0.152941,0.152941}%
\pgfsetstrokecolor{currentstroke}%
\pgfsetdash{}{0pt}%
\pgfpathmoveto{\pgfqpoint{3.809367in}{3.847520in}}%
\pgfpathlineto{\pgfqpoint{3.871972in}{3.847520in}}%
\pgfpathlineto{\pgfqpoint{3.871972in}{3.912164in}}%
\pgfpathlineto{\pgfqpoint{3.809367in}{3.912164in}}%
\pgfpathclose%
\pgfusepath{stroke,fill}%
\end{pgfscope}%
\begin{pgfscope}%
\pgfpathrectangle{\pgfqpoint{0.550713in}{3.620038in}}{\pgfqpoint{4.791200in}{1.332187in}}%
\pgfusepath{clip}%
\pgfsetbuttcap%
\pgfsetroundjoin%
\definecolor{currentfill}{rgb}{0.560784,0.133333,0.243137}%
\pgfsetfillcolor{currentfill}%
\pgfsetlinewidth{0.752812pt}%
\definecolor{currentstroke}{rgb}{0.152941,0.152941,0.152941}%
\pgfsetstrokecolor{currentstroke}%
\pgfsetdash{}{0pt}%
\pgfpathmoveto{\pgfqpoint{3.778065in}{3.857830in}}%
\pgfpathlineto{\pgfqpoint{3.903275in}{3.857830in}}%
\pgfpathlineto{\pgfqpoint{3.903275in}{3.885299in}}%
\pgfpathlineto{\pgfqpoint{3.778065in}{3.885299in}}%
\pgfpathclose%
\pgfusepath{stroke,fill}%
\end{pgfscope}%
\begin{pgfscope}%
\pgfpathrectangle{\pgfqpoint{0.550713in}{3.620038in}}{\pgfqpoint{4.791200in}{1.332187in}}%
\pgfusepath{clip}%
\pgfsetbuttcap%
\pgfsetroundjoin%
\definecolor{currentfill}{rgb}{0.902991,0.874940,0.898900}%
\pgfsetfillcolor{currentfill}%
\pgfsetlinewidth{0.752812pt}%
\definecolor{currentstroke}{rgb}{0.152941,0.152941,0.152941}%
\pgfsetstrokecolor{currentstroke}%
\pgfsetdash{}{0pt}%
\pgfpathmoveto{\pgfqpoint{3.952784in}{3.822752in}}%
\pgfpathlineto{\pgfqpoint{3.984086in}{3.822752in}}%
\pgfpathlineto{\pgfqpoint{3.984086in}{4.005226in}}%
\pgfpathlineto{\pgfqpoint{3.952784in}{4.005226in}}%
\pgfpathclose%
\pgfusepath{stroke,fill}%
\end{pgfscope}%
\begin{pgfscope}%
\pgfpathrectangle{\pgfqpoint{0.550713in}{3.620038in}}{\pgfqpoint{4.791200in}{1.332187in}}%
\pgfusepath{clip}%
\pgfsetbuttcap%
\pgfsetroundjoin%
\definecolor{currentfill}{rgb}{0.627092,0.519263,0.611367}%
\pgfsetfillcolor{currentfill}%
\pgfsetlinewidth{0.752812pt}%
\definecolor{currentstroke}{rgb}{0.152941,0.152941,0.152941}%
\pgfsetstrokecolor{currentstroke}%
\pgfsetdash{}{0pt}%
\pgfpathmoveto{\pgfqpoint{3.937133in}{3.865614in}}%
\pgfpathlineto{\pgfqpoint{3.999738in}{3.865614in}}%
\pgfpathlineto{\pgfqpoint{3.999738in}{3.999368in}}%
\pgfpathlineto{\pgfqpoint{3.937133in}{3.999368in}}%
\pgfpathclose%
\pgfusepath{stroke,fill}%
\end{pgfscope}%
\begin{pgfscope}%
\pgfpathrectangle{\pgfqpoint{0.550713in}{3.620038in}}{\pgfqpoint{4.791200in}{1.332187in}}%
\pgfusepath{clip}%
\pgfsetbuttcap%
\pgfsetroundjoin%
\definecolor{currentfill}{rgb}{0.349020,0.160784,0.321569}%
\pgfsetfillcolor{currentfill}%
\pgfsetlinewidth{0.752812pt}%
\definecolor{currentstroke}{rgb}{0.152941,0.152941,0.152941}%
\pgfsetstrokecolor{currentstroke}%
\pgfsetdash{}{0pt}%
\pgfpathmoveto{\pgfqpoint{3.905830in}{3.951339in}}%
\pgfpathlineto{\pgfqpoint{4.031040in}{3.951339in}}%
\pgfpathlineto{\pgfqpoint{4.031040in}{3.987653in}}%
\pgfpathlineto{\pgfqpoint{3.905830in}{3.987653in}}%
\pgfpathclose%
\pgfusepath{stroke,fill}%
\end{pgfscope}%
\begin{pgfscope}%
\pgfpathrectangle{\pgfqpoint{0.550713in}{3.620038in}}{\pgfqpoint{4.791200in}{1.332187in}}%
\pgfusepath{clip}%
\pgfsetbuttcap%
\pgfsetroundjoin%
\definecolor{currentfill}{rgb}{0.934548,0.870850,0.887213}%
\pgfsetfillcolor{currentfill}%
\pgfsetlinewidth{0.752812pt}%
\definecolor{currentstroke}{rgb}{0.152941,0.152941,0.152941}%
\pgfsetstrokecolor{currentstroke}%
\pgfsetdash{}{0pt}%
\pgfpathmoveto{\pgfqpoint{4.144432in}{3.848412in}}%
\pgfpathlineto{\pgfqpoint{4.175734in}{3.848412in}}%
\pgfpathlineto{\pgfqpoint{4.175734in}{3.959026in}}%
\pgfpathlineto{\pgfqpoint{4.144432in}{3.959026in}}%
\pgfpathclose%
\pgfusepath{stroke,fill}%
\end{pgfscope}%
\begin{pgfscope}%
\pgfpathrectangle{\pgfqpoint{0.550713in}{3.620038in}}{\pgfqpoint{4.791200in}{1.332187in}}%
\pgfusepath{clip}%
\pgfsetbuttcap%
\pgfsetroundjoin%
\definecolor{currentfill}{rgb}{0.748399,0.503538,0.566438}%
\pgfsetfillcolor{currentfill}%
\pgfsetlinewidth{0.752812pt}%
\definecolor{currentstroke}{rgb}{0.152941,0.152941,0.152941}%
\pgfsetstrokecolor{currentstroke}%
\pgfsetdash{}{0pt}%
\pgfpathmoveto{\pgfqpoint{4.128781in}{3.867388in}}%
\pgfpathlineto{\pgfqpoint{4.191386in}{3.867388in}}%
\pgfpathlineto{\pgfqpoint{4.191386in}{3.949723in}}%
\pgfpathlineto{\pgfqpoint{4.128781in}{3.949723in}}%
\pgfpathclose%
\pgfusepath{stroke,fill}%
\end{pgfscope}%
\begin{pgfscope}%
\pgfpathrectangle{\pgfqpoint{0.550713in}{3.620038in}}{\pgfqpoint{4.791200in}{1.332187in}}%
\pgfusepath{clip}%
\pgfsetbuttcap%
\pgfsetroundjoin%
\definecolor{currentfill}{rgb}{0.560784,0.133333,0.243137}%
\pgfsetfillcolor{currentfill}%
\pgfsetlinewidth{0.752812pt}%
\definecolor{currentstroke}{rgb}{0.152941,0.152941,0.152941}%
\pgfsetstrokecolor{currentstroke}%
\pgfsetdash{}{0pt}%
\pgfpathmoveto{\pgfqpoint{4.097478in}{3.905340in}}%
\pgfpathlineto{\pgfqpoint{4.222688in}{3.905340in}}%
\pgfpathlineto{\pgfqpoint{4.222688in}{3.931116in}}%
\pgfpathlineto{\pgfqpoint{4.097478in}{3.931116in}}%
\pgfpathclose%
\pgfusepath{stroke,fill}%
\end{pgfscope}%
\begin{pgfscope}%
\pgfpathrectangle{\pgfqpoint{0.550713in}{3.620038in}}{\pgfqpoint{4.791200in}{1.332187in}}%
\pgfusepath{clip}%
\pgfsetbuttcap%
\pgfsetroundjoin%
\definecolor{currentfill}{rgb}{0.902991,0.874940,0.898900}%
\pgfsetfillcolor{currentfill}%
\pgfsetlinewidth{0.752812pt}%
\definecolor{currentstroke}{rgb}{0.152941,0.152941,0.152941}%
\pgfsetstrokecolor{currentstroke}%
\pgfsetdash{}{0pt}%
\pgfpathmoveto{\pgfqpoint{4.272197in}{3.881854in}}%
\pgfpathlineto{\pgfqpoint{4.303500in}{3.881854in}}%
\pgfpathlineto{\pgfqpoint{4.303500in}{4.058817in}}%
\pgfpathlineto{\pgfqpoint{4.272197in}{4.058817in}}%
\pgfpathclose%
\pgfusepath{stroke,fill}%
\end{pgfscope}%
\begin{pgfscope}%
\pgfpathrectangle{\pgfqpoint{0.550713in}{3.620038in}}{\pgfqpoint{4.791200in}{1.332187in}}%
\pgfusepath{clip}%
\pgfsetbuttcap%
\pgfsetroundjoin%
\definecolor{currentfill}{rgb}{0.627092,0.519263,0.611367}%
\pgfsetfillcolor{currentfill}%
\pgfsetlinewidth{0.752812pt}%
\definecolor{currentstroke}{rgb}{0.152941,0.152941,0.152941}%
\pgfsetstrokecolor{currentstroke}%
\pgfsetdash{}{0pt}%
\pgfpathmoveto{\pgfqpoint{4.256546in}{3.936355in}}%
\pgfpathlineto{\pgfqpoint{4.319151in}{3.936355in}}%
\pgfpathlineto{\pgfqpoint{4.319151in}{4.058460in}}%
\pgfpathlineto{\pgfqpoint{4.256546in}{4.058460in}}%
\pgfpathclose%
\pgfusepath{stroke,fill}%
\end{pgfscope}%
\begin{pgfscope}%
\pgfpathrectangle{\pgfqpoint{0.550713in}{3.620038in}}{\pgfqpoint{4.791200in}{1.332187in}}%
\pgfusepath{clip}%
\pgfsetbuttcap%
\pgfsetroundjoin%
\definecolor{currentfill}{rgb}{0.349020,0.160784,0.321569}%
\pgfsetfillcolor{currentfill}%
\pgfsetlinewidth{0.752812pt}%
\definecolor{currentstroke}{rgb}{0.152941,0.152941,0.152941}%
\pgfsetstrokecolor{currentstroke}%
\pgfsetdash{}{0pt}%
\pgfpathmoveto{\pgfqpoint{4.225243in}{4.045355in}}%
\pgfpathlineto{\pgfqpoint{4.350453in}{4.045355in}}%
\pgfpathlineto{\pgfqpoint{4.350453in}{4.057748in}}%
\pgfpathlineto{\pgfqpoint{4.225243in}{4.057748in}}%
\pgfpathclose%
\pgfusepath{stroke,fill}%
\end{pgfscope}%
\begin{pgfscope}%
\pgfpathrectangle{\pgfqpoint{0.550713in}{3.620038in}}{\pgfqpoint{4.791200in}{1.332187in}}%
\pgfusepath{clip}%
\pgfsetbuttcap%
\pgfsetroundjoin%
\definecolor{currentfill}{rgb}{0.934548,0.870850,0.887213}%
\pgfsetfillcolor{currentfill}%
\pgfsetlinewidth{0.752812pt}%
\definecolor{currentstroke}{rgb}{0.152941,0.152941,0.152941}%
\pgfsetstrokecolor{currentstroke}%
\pgfsetdash{}{0pt}%
\pgfpathmoveto{\pgfqpoint{4.463845in}{3.910920in}}%
\pgfpathlineto{\pgfqpoint{4.495148in}{3.910920in}}%
\pgfpathlineto{\pgfqpoint{4.495148in}{4.050728in}}%
\pgfpathlineto{\pgfqpoint{4.463845in}{4.050728in}}%
\pgfpathclose%
\pgfusepath{stroke,fill}%
\end{pgfscope}%
\begin{pgfscope}%
\pgfpathrectangle{\pgfqpoint{0.550713in}{3.620038in}}{\pgfqpoint{4.791200in}{1.332187in}}%
\pgfusepath{clip}%
\pgfsetbuttcap%
\pgfsetroundjoin%
\definecolor{currentfill}{rgb}{0.748399,0.503538,0.566438}%
\pgfsetfillcolor{currentfill}%
\pgfsetlinewidth{0.752812pt}%
\definecolor{currentstroke}{rgb}{0.152941,0.152941,0.152941}%
\pgfsetstrokecolor{currentstroke}%
\pgfsetdash{}{0pt}%
\pgfpathmoveto{\pgfqpoint{4.448194in}{3.911291in}}%
\pgfpathlineto{\pgfqpoint{4.510799in}{3.911291in}}%
\pgfpathlineto{\pgfqpoint{4.510799in}{4.045046in}}%
\pgfpathlineto{\pgfqpoint{4.448194in}{4.045046in}}%
\pgfpathclose%
\pgfusepath{stroke,fill}%
\end{pgfscope}%
\begin{pgfscope}%
\pgfpathrectangle{\pgfqpoint{0.550713in}{3.620038in}}{\pgfqpoint{4.791200in}{1.332187in}}%
\pgfusepath{clip}%
\pgfsetbuttcap%
\pgfsetroundjoin%
\definecolor{currentfill}{rgb}{0.560784,0.133333,0.243137}%
\pgfsetfillcolor{currentfill}%
\pgfsetlinewidth{0.752812pt}%
\definecolor{currentstroke}{rgb}{0.152941,0.152941,0.152941}%
\pgfsetstrokecolor{currentstroke}%
\pgfsetdash{}{0pt}%
\pgfpathmoveto{\pgfqpoint{4.416891in}{3.912034in}}%
\pgfpathlineto{\pgfqpoint{4.542101in}{3.912034in}}%
\pgfpathlineto{\pgfqpoint{4.542101in}{4.033683in}}%
\pgfpathlineto{\pgfqpoint{4.416891in}{4.033683in}}%
\pgfpathclose%
\pgfusepath{stroke,fill}%
\end{pgfscope}%
\begin{pgfscope}%
\pgfpathrectangle{\pgfqpoint{0.550713in}{3.620038in}}{\pgfqpoint{4.791200in}{1.332187in}}%
\pgfusepath{clip}%
\pgfsetbuttcap%
\pgfsetroundjoin%
\definecolor{currentfill}{rgb}{0.902991,0.874940,0.898900}%
\pgfsetfillcolor{currentfill}%
\pgfsetlinewidth{0.752812pt}%
\definecolor{currentstroke}{rgb}{0.152941,0.152941,0.152941}%
\pgfsetstrokecolor{currentstroke}%
\pgfsetdash{}{0pt}%
\pgfpathmoveto{\pgfqpoint{4.591610in}{3.858584in}}%
\pgfpathlineto{\pgfqpoint{4.622913in}{3.858584in}}%
\pgfpathlineto{\pgfqpoint{4.622913in}{4.086197in}}%
\pgfpathlineto{\pgfqpoint{4.591610in}{4.086197in}}%
\pgfpathclose%
\pgfusepath{stroke,fill}%
\end{pgfscope}%
\begin{pgfscope}%
\pgfpathrectangle{\pgfqpoint{0.550713in}{3.620038in}}{\pgfqpoint{4.791200in}{1.332187in}}%
\pgfusepath{clip}%
\pgfsetbuttcap%
\pgfsetroundjoin%
\definecolor{currentfill}{rgb}{0.627092,0.519263,0.611367}%
\pgfsetfillcolor{currentfill}%
\pgfsetlinewidth{0.752812pt}%
\definecolor{currentstroke}{rgb}{0.152941,0.152941,0.152941}%
\pgfsetstrokecolor{currentstroke}%
\pgfsetdash{}{0pt}%
\pgfpathmoveto{\pgfqpoint{4.575959in}{3.917455in}}%
\pgfpathlineto{\pgfqpoint{4.638564in}{3.917455in}}%
\pgfpathlineto{\pgfqpoint{4.638564in}{4.084732in}}%
\pgfpathlineto{\pgfqpoint{4.575959in}{4.084732in}}%
\pgfpathclose%
\pgfusepath{stroke,fill}%
\end{pgfscope}%
\begin{pgfscope}%
\pgfpathrectangle{\pgfqpoint{0.550713in}{3.620038in}}{\pgfqpoint{4.791200in}{1.332187in}}%
\pgfusepath{clip}%
\pgfsetbuttcap%
\pgfsetroundjoin%
\definecolor{currentfill}{rgb}{0.349020,0.160784,0.321569}%
\pgfsetfillcolor{currentfill}%
\pgfsetlinewidth{0.752812pt}%
\definecolor{currentstroke}{rgb}{0.152941,0.152941,0.152941}%
\pgfsetstrokecolor{currentstroke}%
\pgfsetdash{}{0pt}%
\pgfpathmoveto{\pgfqpoint{4.544657in}{4.035196in}}%
\pgfpathlineto{\pgfqpoint{4.669867in}{4.035196in}}%
\pgfpathlineto{\pgfqpoint{4.669867in}{4.081801in}}%
\pgfpathlineto{\pgfqpoint{4.544657in}{4.081801in}}%
\pgfpathclose%
\pgfusepath{stroke,fill}%
\end{pgfscope}%
\begin{pgfscope}%
\pgfpathrectangle{\pgfqpoint{0.550713in}{3.620038in}}{\pgfqpoint{4.791200in}{1.332187in}}%
\pgfusepath{clip}%
\pgfsetbuttcap%
\pgfsetroundjoin%
\definecolor{currentfill}{rgb}{0.934548,0.870850,0.887213}%
\pgfsetfillcolor{currentfill}%
\pgfsetlinewidth{0.752812pt}%
\definecolor{currentstroke}{rgb}{0.152941,0.152941,0.152941}%
\pgfsetstrokecolor{currentstroke}%
\pgfsetdash{}{0pt}%
\pgfpathmoveto{\pgfqpoint{4.783258in}{4.005266in}}%
\pgfpathlineto{\pgfqpoint{4.814561in}{4.005266in}}%
\pgfpathlineto{\pgfqpoint{4.814561in}{4.511142in}}%
\pgfpathlineto{\pgfqpoint{4.783258in}{4.511142in}}%
\pgfpathclose%
\pgfusepath{stroke,fill}%
\end{pgfscope}%
\begin{pgfscope}%
\pgfpathrectangle{\pgfqpoint{0.550713in}{3.620038in}}{\pgfqpoint{4.791200in}{1.332187in}}%
\pgfusepath{clip}%
\pgfsetbuttcap%
\pgfsetroundjoin%
\definecolor{currentfill}{rgb}{0.748399,0.503538,0.566438}%
\pgfsetfillcolor{currentfill}%
\pgfsetlinewidth{0.752812pt}%
\definecolor{currentstroke}{rgb}{0.152941,0.152941,0.152941}%
\pgfsetstrokecolor{currentstroke}%
\pgfsetdash{}{0pt}%
\pgfpathmoveto{\pgfqpoint{4.767607in}{4.060176in}}%
\pgfpathlineto{\pgfqpoint{4.830212in}{4.060176in}}%
\pgfpathlineto{\pgfqpoint{4.830212in}{4.472947in}}%
\pgfpathlineto{\pgfqpoint{4.767607in}{4.472947in}}%
\pgfpathclose%
\pgfusepath{stroke,fill}%
\end{pgfscope}%
\begin{pgfscope}%
\pgfpathrectangle{\pgfqpoint{0.550713in}{3.620038in}}{\pgfqpoint{4.791200in}{1.332187in}}%
\pgfusepath{clip}%
\pgfsetbuttcap%
\pgfsetroundjoin%
\definecolor{currentfill}{rgb}{0.560784,0.133333,0.243137}%
\pgfsetfillcolor{currentfill}%
\pgfsetlinewidth{0.752812pt}%
\definecolor{currentstroke}{rgb}{0.152941,0.152941,0.152941}%
\pgfsetstrokecolor{currentstroke}%
\pgfsetdash{}{0pt}%
\pgfpathmoveto{\pgfqpoint{4.736305in}{4.169997in}}%
\pgfpathlineto{\pgfqpoint{4.861515in}{4.169997in}}%
\pgfpathlineto{\pgfqpoint{4.861515in}{4.396557in}}%
\pgfpathlineto{\pgfqpoint{4.736305in}{4.396557in}}%
\pgfpathclose%
\pgfusepath{stroke,fill}%
\end{pgfscope}%
\begin{pgfscope}%
\pgfpathrectangle{\pgfqpoint{0.550713in}{3.620038in}}{\pgfqpoint{4.791200in}{1.332187in}}%
\pgfusepath{clip}%
\pgfsetbuttcap%
\pgfsetroundjoin%
\definecolor{currentfill}{rgb}{0.902991,0.874940,0.898900}%
\pgfsetfillcolor{currentfill}%
\pgfsetlinewidth{0.752812pt}%
\definecolor{currentstroke}{rgb}{0.152941,0.152941,0.152941}%
\pgfsetstrokecolor{currentstroke}%
\pgfsetdash{}{0pt}%
\pgfpathmoveto{\pgfqpoint{4.911024in}{3.766201in}}%
\pgfpathlineto{\pgfqpoint{4.942326in}{3.766201in}}%
\pgfpathlineto{\pgfqpoint{4.942326in}{3.860806in}}%
\pgfpathlineto{\pgfqpoint{4.911024in}{3.860806in}}%
\pgfpathclose%
\pgfusepath{stroke,fill}%
\end{pgfscope}%
\begin{pgfscope}%
\pgfpathrectangle{\pgfqpoint{0.550713in}{3.620038in}}{\pgfqpoint{4.791200in}{1.332187in}}%
\pgfusepath{clip}%
\pgfsetbuttcap%
\pgfsetroundjoin%
\definecolor{currentfill}{rgb}{0.627092,0.519263,0.611367}%
\pgfsetfillcolor{currentfill}%
\pgfsetlinewidth{0.752812pt}%
\definecolor{currentstroke}{rgb}{0.152941,0.152941,0.152941}%
\pgfsetstrokecolor{currentstroke}%
\pgfsetdash{}{0pt}%
\pgfpathmoveto{\pgfqpoint{4.895372in}{3.776273in}}%
\pgfpathlineto{\pgfqpoint{4.957977in}{3.776273in}}%
\pgfpathlineto{\pgfqpoint{4.957977in}{3.858917in}}%
\pgfpathlineto{\pgfqpoint{4.895372in}{3.858917in}}%
\pgfpathclose%
\pgfusepath{stroke,fill}%
\end{pgfscope}%
\begin{pgfscope}%
\pgfpathrectangle{\pgfqpoint{0.550713in}{3.620038in}}{\pgfqpoint{4.791200in}{1.332187in}}%
\pgfusepath{clip}%
\pgfsetbuttcap%
\pgfsetroundjoin%
\definecolor{currentfill}{rgb}{0.349020,0.160784,0.321569}%
\pgfsetfillcolor{currentfill}%
\pgfsetlinewidth{0.752812pt}%
\definecolor{currentstroke}{rgb}{0.152941,0.152941,0.152941}%
\pgfsetstrokecolor{currentstroke}%
\pgfsetdash{}{0pt}%
\pgfpathmoveto{\pgfqpoint{4.864070in}{3.796416in}}%
\pgfpathlineto{\pgfqpoint{4.989280in}{3.796416in}}%
\pgfpathlineto{\pgfqpoint{4.989280in}{3.855138in}}%
\pgfpathlineto{\pgfqpoint{4.864070in}{3.855138in}}%
\pgfpathclose%
\pgfusepath{stroke,fill}%
\end{pgfscope}%
\begin{pgfscope}%
\pgfpathrectangle{\pgfqpoint{0.550713in}{3.620038in}}{\pgfqpoint{4.791200in}{1.332187in}}%
\pgfusepath{clip}%
\pgfsetbuttcap%
\pgfsetroundjoin%
\definecolor{currentfill}{rgb}{0.934548,0.870850,0.887213}%
\pgfsetfillcolor{currentfill}%
\pgfsetlinewidth{0.752812pt}%
\definecolor{currentstroke}{rgb}{0.152941,0.152941,0.152941}%
\pgfsetstrokecolor{currentstroke}%
\pgfsetdash{}{0pt}%
\pgfpathmoveto{\pgfqpoint{5.102672in}{4.024948in}}%
\pgfpathlineto{\pgfqpoint{5.133974in}{4.024948in}}%
\pgfpathlineto{\pgfqpoint{5.133974in}{4.744682in}}%
\pgfpathlineto{\pgfqpoint{5.102672in}{4.744682in}}%
\pgfpathclose%
\pgfusepath{stroke,fill}%
\end{pgfscope}%
\begin{pgfscope}%
\pgfpathrectangle{\pgfqpoint{0.550713in}{3.620038in}}{\pgfqpoint{4.791200in}{1.332187in}}%
\pgfusepath{clip}%
\pgfsetbuttcap%
\pgfsetroundjoin%
\definecolor{currentfill}{rgb}{0.748399,0.503538,0.566438}%
\pgfsetfillcolor{currentfill}%
\pgfsetlinewidth{0.752812pt}%
\definecolor{currentstroke}{rgb}{0.152941,0.152941,0.152941}%
\pgfsetstrokecolor{currentstroke}%
\pgfsetdash{}{0pt}%
\pgfpathmoveto{\pgfqpoint{5.087020in}{4.071309in}}%
\pgfpathlineto{\pgfqpoint{5.149625in}{4.071309in}}%
\pgfpathlineto{\pgfqpoint{5.149625in}{4.736209in}}%
\pgfpathlineto{\pgfqpoint{5.087020in}{4.736209in}}%
\pgfpathclose%
\pgfusepath{stroke,fill}%
\end{pgfscope}%
\begin{pgfscope}%
\pgfpathrectangle{\pgfqpoint{0.550713in}{3.620038in}}{\pgfqpoint{4.791200in}{1.332187in}}%
\pgfusepath{clip}%
\pgfsetbuttcap%
\pgfsetroundjoin%
\definecolor{currentfill}{rgb}{0.560784,0.133333,0.243137}%
\pgfsetfillcolor{currentfill}%
\pgfsetlinewidth{0.752812pt}%
\definecolor{currentstroke}{rgb}{0.152941,0.152941,0.152941}%
\pgfsetstrokecolor{currentstroke}%
\pgfsetdash{}{0pt}%
\pgfpathmoveto{\pgfqpoint{5.055718in}{4.164029in}}%
\pgfpathlineto{\pgfqpoint{5.180928in}{4.164029in}}%
\pgfpathlineto{\pgfqpoint{5.180928in}{4.719263in}}%
\pgfpathlineto{\pgfqpoint{5.055718in}{4.719263in}}%
\pgfpathclose%
\pgfusepath{stroke,fill}%
\end{pgfscope}%
\begin{pgfscope}%
\pgfpathrectangle{\pgfqpoint{0.550713in}{3.620038in}}{\pgfqpoint{4.791200in}{1.332187in}}%
\pgfusepath{clip}%
\pgfsetbuttcap%
\pgfsetroundjoin%
\definecolor{currentfill}{rgb}{0.902991,0.874940,0.898900}%
\pgfsetfillcolor{currentfill}%
\pgfsetlinewidth{0.752812pt}%
\definecolor{currentstroke}{rgb}{0.152941,0.152941,0.152941}%
\pgfsetstrokecolor{currentstroke}%
\pgfsetdash{}{0pt}%
\pgfpathmoveto{\pgfqpoint{5.230437in}{4.356336in}}%
\pgfpathlineto{\pgfqpoint{5.261740in}{4.356336in}}%
\pgfpathlineto{\pgfqpoint{5.261740in}{5.439292in}}%
\pgfpathlineto{\pgfqpoint{5.230437in}{5.439292in}}%
\pgfpathclose%
\pgfusepath{stroke,fill}%
\end{pgfscope}%
\begin{pgfscope}%
\pgfpathrectangle{\pgfqpoint{0.550713in}{3.620038in}}{\pgfqpoint{4.791200in}{1.332187in}}%
\pgfusepath{clip}%
\pgfsetbuttcap%
\pgfsetroundjoin%
\definecolor{currentfill}{rgb}{0.627092,0.519263,0.611367}%
\pgfsetfillcolor{currentfill}%
\pgfsetlinewidth{0.752812pt}%
\definecolor{currentstroke}{rgb}{0.152941,0.152941,0.152941}%
\pgfsetstrokecolor{currentstroke}%
\pgfsetdash{}{0pt}%
\pgfpathmoveto{\pgfqpoint{5.214786in}{4.420797in}}%
\pgfpathlineto{\pgfqpoint{5.277391in}{4.420797in}}%
\pgfpathlineto{\pgfqpoint{5.277391in}{5.330290in}}%
\pgfpathlineto{\pgfqpoint{5.214786in}{5.330290in}}%
\pgfpathclose%
\pgfusepath{stroke,fill}%
\end{pgfscope}%
\begin{pgfscope}%
\pgfpathrectangle{\pgfqpoint{0.550713in}{3.620038in}}{\pgfqpoint{4.791200in}{1.332187in}}%
\pgfusepath{clip}%
\pgfsetbuttcap%
\pgfsetroundjoin%
\definecolor{currentfill}{rgb}{0.349020,0.160784,0.321569}%
\pgfsetfillcolor{currentfill}%
\pgfsetlinewidth{0.752812pt}%
\definecolor{currentstroke}{rgb}{0.152941,0.152941,0.152941}%
\pgfsetstrokecolor{currentstroke}%
\pgfsetdash{}{0pt}%
\pgfpathmoveto{\pgfqpoint{5.183483in}{4.549718in}}%
\pgfpathlineto{\pgfqpoint{5.308693in}{4.549718in}}%
\pgfpathlineto{\pgfqpoint{5.308693in}{5.112285in}}%
\pgfpathlineto{\pgfqpoint{5.183483in}{5.112285in}}%
\pgfpathclose%
\pgfusepath{stroke,fill}%
\end{pgfscope}%
\begin{pgfscope}%
\pgfpathrectangle{\pgfqpoint{0.550713in}{3.620038in}}{\pgfqpoint{4.791200in}{1.332187in}}%
\pgfusepath{clip}%
\pgfsetbuttcap%
\pgfsetmiterjoin%
\definecolor{currentfill}{rgb}{0.560294,0.133824,0.242647}%
\pgfsetfillcolor{currentfill}%
\pgfsetlinewidth{0.376406pt}%
\definecolor{currentstroke}{rgb}{0.152941,0.152941,0.152941}%
\pgfsetstrokecolor{currentstroke}%
\pgfsetdash{}{0pt}%
\pgfpathmoveto{\pgfqpoint{0.710419in}{3.620038in}}%
\pgfpathlineto{\pgfqpoint{0.710419in}{3.620038in}}%
\pgfpathlineto{\pgfqpoint{0.710419in}{3.620038in}}%
\pgfpathlineto{\pgfqpoint{0.710419in}{3.620038in}}%
\pgfpathclose%
\pgfusepath{stroke,fill}%
\end{pgfscope}%
\begin{pgfscope}%
\pgfpathrectangle{\pgfqpoint{0.550713in}{3.620038in}}{\pgfqpoint{4.791200in}{1.332187in}}%
\pgfusepath{clip}%
\pgfsetbuttcap%
\pgfsetmiterjoin%
\definecolor{currentfill}{rgb}{0.349020,0.160784,0.322549}%
\pgfsetfillcolor{currentfill}%
\pgfsetlinewidth{0.376406pt}%
\definecolor{currentstroke}{rgb}{0.152941,0.152941,0.152941}%
\pgfsetstrokecolor{currentstroke}%
\pgfsetdash{}{0pt}%
\pgfpathmoveto{\pgfqpoint{0.710419in}{3.620038in}}%
\pgfpathlineto{\pgfqpoint{0.710419in}{3.620038in}}%
\pgfpathlineto{\pgfqpoint{0.710419in}{3.620038in}}%
\pgfpathlineto{\pgfqpoint{0.710419in}{3.620038in}}%
\pgfpathclose%
\pgfusepath{stroke,fill}%
\end{pgfscope}%
\begin{pgfscope}%
\pgfsetbuttcap%
\pgfsetroundjoin%
\definecolor{currentfill}{rgb}{0.000000,0.000000,0.000000}%
\pgfsetfillcolor{currentfill}%
\pgfsetlinewidth{0.803000pt}%
\definecolor{currentstroke}{rgb}{0.000000,0.000000,0.000000}%
\pgfsetstrokecolor{currentstroke}%
\pgfsetdash{}{0pt}%
\pgfsys@defobject{currentmarker}{\pgfqpoint{0.000000in}{-0.048611in}}{\pgfqpoint{0.000000in}{0.000000in}}{%
\pgfpathmoveto{\pgfqpoint{0.000000in}{0.000000in}}%
\pgfpathlineto{\pgfqpoint{0.000000in}{-0.048611in}}%
\pgfusepath{stroke,fill}%
}%
\begin{pgfscope}%
\pgfsys@transformshift{0.710419in}{3.620038in}%
\pgfsys@useobject{currentmarker}{}%
\end{pgfscope}%
\end{pgfscope}%
\begin{pgfscope}%
\pgfsetbuttcap%
\pgfsetroundjoin%
\definecolor{currentfill}{rgb}{0.000000,0.000000,0.000000}%
\pgfsetfillcolor{currentfill}%
\pgfsetlinewidth{0.803000pt}%
\definecolor{currentstroke}{rgb}{0.000000,0.000000,0.000000}%
\pgfsetstrokecolor{currentstroke}%
\pgfsetdash{}{0pt}%
\pgfsys@defobject{currentmarker}{\pgfqpoint{0.000000in}{-0.048611in}}{\pgfqpoint{0.000000in}{0.000000in}}{%
\pgfpathmoveto{\pgfqpoint{0.000000in}{0.000000in}}%
\pgfpathlineto{\pgfqpoint{0.000000in}{-0.048611in}}%
\pgfusepath{stroke,fill}%
}%
\begin{pgfscope}%
\pgfsys@transformshift{1.029833in}{3.620038in}%
\pgfsys@useobject{currentmarker}{}%
\end{pgfscope}%
\end{pgfscope}%
\begin{pgfscope}%
\pgfsetbuttcap%
\pgfsetroundjoin%
\definecolor{currentfill}{rgb}{0.000000,0.000000,0.000000}%
\pgfsetfillcolor{currentfill}%
\pgfsetlinewidth{0.803000pt}%
\definecolor{currentstroke}{rgb}{0.000000,0.000000,0.000000}%
\pgfsetstrokecolor{currentstroke}%
\pgfsetdash{}{0pt}%
\pgfsys@defobject{currentmarker}{\pgfqpoint{0.000000in}{-0.048611in}}{\pgfqpoint{0.000000in}{0.000000in}}{%
\pgfpathmoveto{\pgfqpoint{0.000000in}{0.000000in}}%
\pgfpathlineto{\pgfqpoint{0.000000in}{-0.048611in}}%
\pgfusepath{stroke,fill}%
}%
\begin{pgfscope}%
\pgfsys@transformshift{1.349246in}{3.620038in}%
\pgfsys@useobject{currentmarker}{}%
\end{pgfscope}%
\end{pgfscope}%
\begin{pgfscope}%
\pgfsetbuttcap%
\pgfsetroundjoin%
\definecolor{currentfill}{rgb}{0.000000,0.000000,0.000000}%
\pgfsetfillcolor{currentfill}%
\pgfsetlinewidth{0.803000pt}%
\definecolor{currentstroke}{rgb}{0.000000,0.000000,0.000000}%
\pgfsetstrokecolor{currentstroke}%
\pgfsetdash{}{0pt}%
\pgfsys@defobject{currentmarker}{\pgfqpoint{0.000000in}{-0.048611in}}{\pgfqpoint{0.000000in}{0.000000in}}{%
\pgfpathmoveto{\pgfqpoint{0.000000in}{0.000000in}}%
\pgfpathlineto{\pgfqpoint{0.000000in}{-0.048611in}}%
\pgfusepath{stroke,fill}%
}%
\begin{pgfscope}%
\pgfsys@transformshift{1.668659in}{3.620038in}%
\pgfsys@useobject{currentmarker}{}%
\end{pgfscope}%
\end{pgfscope}%
\begin{pgfscope}%
\pgfsetbuttcap%
\pgfsetroundjoin%
\definecolor{currentfill}{rgb}{0.000000,0.000000,0.000000}%
\pgfsetfillcolor{currentfill}%
\pgfsetlinewidth{0.803000pt}%
\definecolor{currentstroke}{rgb}{0.000000,0.000000,0.000000}%
\pgfsetstrokecolor{currentstroke}%
\pgfsetdash{}{0pt}%
\pgfsys@defobject{currentmarker}{\pgfqpoint{0.000000in}{-0.048611in}}{\pgfqpoint{0.000000in}{0.000000in}}{%
\pgfpathmoveto{\pgfqpoint{0.000000in}{0.000000in}}%
\pgfpathlineto{\pgfqpoint{0.000000in}{-0.048611in}}%
\pgfusepath{stroke,fill}%
}%
\begin{pgfscope}%
\pgfsys@transformshift{1.988073in}{3.620038in}%
\pgfsys@useobject{currentmarker}{}%
\end{pgfscope}%
\end{pgfscope}%
\begin{pgfscope}%
\pgfsetbuttcap%
\pgfsetroundjoin%
\definecolor{currentfill}{rgb}{0.000000,0.000000,0.000000}%
\pgfsetfillcolor{currentfill}%
\pgfsetlinewidth{0.803000pt}%
\definecolor{currentstroke}{rgb}{0.000000,0.000000,0.000000}%
\pgfsetstrokecolor{currentstroke}%
\pgfsetdash{}{0pt}%
\pgfsys@defobject{currentmarker}{\pgfqpoint{0.000000in}{-0.048611in}}{\pgfqpoint{0.000000in}{0.000000in}}{%
\pgfpathmoveto{\pgfqpoint{0.000000in}{0.000000in}}%
\pgfpathlineto{\pgfqpoint{0.000000in}{-0.048611in}}%
\pgfusepath{stroke,fill}%
}%
\begin{pgfscope}%
\pgfsys@transformshift{2.307486in}{3.620038in}%
\pgfsys@useobject{currentmarker}{}%
\end{pgfscope}%
\end{pgfscope}%
\begin{pgfscope}%
\pgfsetbuttcap%
\pgfsetroundjoin%
\definecolor{currentfill}{rgb}{0.000000,0.000000,0.000000}%
\pgfsetfillcolor{currentfill}%
\pgfsetlinewidth{0.803000pt}%
\definecolor{currentstroke}{rgb}{0.000000,0.000000,0.000000}%
\pgfsetstrokecolor{currentstroke}%
\pgfsetdash{}{0pt}%
\pgfsys@defobject{currentmarker}{\pgfqpoint{0.000000in}{-0.048611in}}{\pgfqpoint{0.000000in}{0.000000in}}{%
\pgfpathmoveto{\pgfqpoint{0.000000in}{0.000000in}}%
\pgfpathlineto{\pgfqpoint{0.000000in}{-0.048611in}}%
\pgfusepath{stroke,fill}%
}%
\begin{pgfscope}%
\pgfsys@transformshift{2.626899in}{3.620038in}%
\pgfsys@useobject{currentmarker}{}%
\end{pgfscope}%
\end{pgfscope}%
\begin{pgfscope}%
\pgfsetbuttcap%
\pgfsetroundjoin%
\definecolor{currentfill}{rgb}{0.000000,0.000000,0.000000}%
\pgfsetfillcolor{currentfill}%
\pgfsetlinewidth{0.803000pt}%
\definecolor{currentstroke}{rgb}{0.000000,0.000000,0.000000}%
\pgfsetstrokecolor{currentstroke}%
\pgfsetdash{}{0pt}%
\pgfsys@defobject{currentmarker}{\pgfqpoint{0.000000in}{-0.048611in}}{\pgfqpoint{0.000000in}{0.000000in}}{%
\pgfpathmoveto{\pgfqpoint{0.000000in}{0.000000in}}%
\pgfpathlineto{\pgfqpoint{0.000000in}{-0.048611in}}%
\pgfusepath{stroke,fill}%
}%
\begin{pgfscope}%
\pgfsys@transformshift{2.946312in}{3.620038in}%
\pgfsys@useobject{currentmarker}{}%
\end{pgfscope}%
\end{pgfscope}%
\begin{pgfscope}%
\pgfsetbuttcap%
\pgfsetroundjoin%
\definecolor{currentfill}{rgb}{0.000000,0.000000,0.000000}%
\pgfsetfillcolor{currentfill}%
\pgfsetlinewidth{0.803000pt}%
\definecolor{currentstroke}{rgb}{0.000000,0.000000,0.000000}%
\pgfsetstrokecolor{currentstroke}%
\pgfsetdash{}{0pt}%
\pgfsys@defobject{currentmarker}{\pgfqpoint{0.000000in}{-0.048611in}}{\pgfqpoint{0.000000in}{0.000000in}}{%
\pgfpathmoveto{\pgfqpoint{0.000000in}{0.000000in}}%
\pgfpathlineto{\pgfqpoint{0.000000in}{-0.048611in}}%
\pgfusepath{stroke,fill}%
}%
\begin{pgfscope}%
\pgfsys@transformshift{3.265726in}{3.620038in}%
\pgfsys@useobject{currentmarker}{}%
\end{pgfscope}%
\end{pgfscope}%
\begin{pgfscope}%
\pgfsetbuttcap%
\pgfsetroundjoin%
\definecolor{currentfill}{rgb}{0.000000,0.000000,0.000000}%
\pgfsetfillcolor{currentfill}%
\pgfsetlinewidth{0.803000pt}%
\definecolor{currentstroke}{rgb}{0.000000,0.000000,0.000000}%
\pgfsetstrokecolor{currentstroke}%
\pgfsetdash{}{0pt}%
\pgfsys@defobject{currentmarker}{\pgfqpoint{0.000000in}{-0.048611in}}{\pgfqpoint{0.000000in}{0.000000in}}{%
\pgfpathmoveto{\pgfqpoint{0.000000in}{0.000000in}}%
\pgfpathlineto{\pgfqpoint{0.000000in}{-0.048611in}}%
\pgfusepath{stroke,fill}%
}%
\begin{pgfscope}%
\pgfsys@transformshift{3.585139in}{3.620038in}%
\pgfsys@useobject{currentmarker}{}%
\end{pgfscope}%
\end{pgfscope}%
\begin{pgfscope}%
\pgfsetbuttcap%
\pgfsetroundjoin%
\definecolor{currentfill}{rgb}{0.000000,0.000000,0.000000}%
\pgfsetfillcolor{currentfill}%
\pgfsetlinewidth{0.803000pt}%
\definecolor{currentstroke}{rgb}{0.000000,0.000000,0.000000}%
\pgfsetstrokecolor{currentstroke}%
\pgfsetdash{}{0pt}%
\pgfsys@defobject{currentmarker}{\pgfqpoint{0.000000in}{-0.048611in}}{\pgfqpoint{0.000000in}{0.000000in}}{%
\pgfpathmoveto{\pgfqpoint{0.000000in}{0.000000in}}%
\pgfpathlineto{\pgfqpoint{0.000000in}{-0.048611in}}%
\pgfusepath{stroke,fill}%
}%
\begin{pgfscope}%
\pgfsys@transformshift{3.904552in}{3.620038in}%
\pgfsys@useobject{currentmarker}{}%
\end{pgfscope}%
\end{pgfscope}%
\begin{pgfscope}%
\pgfsetbuttcap%
\pgfsetroundjoin%
\definecolor{currentfill}{rgb}{0.000000,0.000000,0.000000}%
\pgfsetfillcolor{currentfill}%
\pgfsetlinewidth{0.803000pt}%
\definecolor{currentstroke}{rgb}{0.000000,0.000000,0.000000}%
\pgfsetstrokecolor{currentstroke}%
\pgfsetdash{}{0pt}%
\pgfsys@defobject{currentmarker}{\pgfqpoint{0.000000in}{-0.048611in}}{\pgfqpoint{0.000000in}{0.000000in}}{%
\pgfpathmoveto{\pgfqpoint{0.000000in}{0.000000in}}%
\pgfpathlineto{\pgfqpoint{0.000000in}{-0.048611in}}%
\pgfusepath{stroke,fill}%
}%
\begin{pgfscope}%
\pgfsys@transformshift{4.223966in}{3.620038in}%
\pgfsys@useobject{currentmarker}{}%
\end{pgfscope}%
\end{pgfscope}%
\begin{pgfscope}%
\pgfsetbuttcap%
\pgfsetroundjoin%
\definecolor{currentfill}{rgb}{0.000000,0.000000,0.000000}%
\pgfsetfillcolor{currentfill}%
\pgfsetlinewidth{0.803000pt}%
\definecolor{currentstroke}{rgb}{0.000000,0.000000,0.000000}%
\pgfsetstrokecolor{currentstroke}%
\pgfsetdash{}{0pt}%
\pgfsys@defobject{currentmarker}{\pgfqpoint{0.000000in}{-0.048611in}}{\pgfqpoint{0.000000in}{0.000000in}}{%
\pgfpathmoveto{\pgfqpoint{0.000000in}{0.000000in}}%
\pgfpathlineto{\pgfqpoint{0.000000in}{-0.048611in}}%
\pgfusepath{stroke,fill}%
}%
\begin{pgfscope}%
\pgfsys@transformshift{4.543379in}{3.620038in}%
\pgfsys@useobject{currentmarker}{}%
\end{pgfscope}%
\end{pgfscope}%
\begin{pgfscope}%
\pgfsetbuttcap%
\pgfsetroundjoin%
\definecolor{currentfill}{rgb}{0.000000,0.000000,0.000000}%
\pgfsetfillcolor{currentfill}%
\pgfsetlinewidth{0.803000pt}%
\definecolor{currentstroke}{rgb}{0.000000,0.000000,0.000000}%
\pgfsetstrokecolor{currentstroke}%
\pgfsetdash{}{0pt}%
\pgfsys@defobject{currentmarker}{\pgfqpoint{0.000000in}{-0.048611in}}{\pgfqpoint{0.000000in}{0.000000in}}{%
\pgfpathmoveto{\pgfqpoint{0.000000in}{0.000000in}}%
\pgfpathlineto{\pgfqpoint{0.000000in}{-0.048611in}}%
\pgfusepath{stroke,fill}%
}%
\begin{pgfscope}%
\pgfsys@transformshift{4.862792in}{3.620038in}%
\pgfsys@useobject{currentmarker}{}%
\end{pgfscope}%
\end{pgfscope}%
\begin{pgfscope}%
\pgfsetbuttcap%
\pgfsetroundjoin%
\definecolor{currentfill}{rgb}{0.000000,0.000000,0.000000}%
\pgfsetfillcolor{currentfill}%
\pgfsetlinewidth{0.803000pt}%
\definecolor{currentstroke}{rgb}{0.000000,0.000000,0.000000}%
\pgfsetstrokecolor{currentstroke}%
\pgfsetdash{}{0pt}%
\pgfsys@defobject{currentmarker}{\pgfqpoint{0.000000in}{-0.048611in}}{\pgfqpoint{0.000000in}{0.000000in}}{%
\pgfpathmoveto{\pgfqpoint{0.000000in}{0.000000in}}%
\pgfpathlineto{\pgfqpoint{0.000000in}{-0.048611in}}%
\pgfusepath{stroke,fill}%
}%
\begin{pgfscope}%
\pgfsys@transformshift{5.182206in}{3.620038in}%
\pgfsys@useobject{currentmarker}{}%
\end{pgfscope}%
\end{pgfscope}%
\begin{pgfscope}%
\pgfsetbuttcap%
\pgfsetroundjoin%
\definecolor{currentfill}{rgb}{0.000000,0.000000,0.000000}%
\pgfsetfillcolor{currentfill}%
\pgfsetlinewidth{0.803000pt}%
\definecolor{currentstroke}{rgb}{0.000000,0.000000,0.000000}%
\pgfsetstrokecolor{currentstroke}%
\pgfsetdash{}{0pt}%
\pgfsys@defobject{currentmarker}{\pgfqpoint{-0.048611in}{0.000000in}}{\pgfqpoint{-0.000000in}{0.000000in}}{%
\pgfpathmoveto{\pgfqpoint{-0.000000in}{0.000000in}}%
\pgfpathlineto{\pgfqpoint{-0.048611in}{0.000000in}}%
\pgfusepath{stroke,fill}%
}%
\begin{pgfscope}%
\pgfsys@transformshift{0.550713in}{3.620038in}%
\pgfsys@useobject{currentmarker}{}%
\end{pgfscope}%
\end{pgfscope}%
\begin{pgfscope}%
\definecolor{textcolor}{rgb}{0.000000,0.000000,0.000000}%
\pgfsetstrokecolor{textcolor}%
\pgfsetfillcolor{textcolor}%
\pgftext[x=0.384046in, y=3.571844in, left, base]{\color{textcolor}\rmfamily\fontsize{10.000000}{12.000000}\selectfont \(\displaystyle {0}\)}%
\end{pgfscope}%
\begin{pgfscope}%
\pgfsetbuttcap%
\pgfsetroundjoin%
\definecolor{currentfill}{rgb}{0.000000,0.000000,0.000000}%
\pgfsetfillcolor{currentfill}%
\pgfsetlinewidth{0.803000pt}%
\definecolor{currentstroke}{rgb}{0.000000,0.000000,0.000000}%
\pgfsetstrokecolor{currentstroke}%
\pgfsetdash{}{0pt}%
\pgfsys@defobject{currentmarker}{\pgfqpoint{-0.048611in}{0.000000in}}{\pgfqpoint{-0.000000in}{0.000000in}}{%
\pgfpathmoveto{\pgfqpoint{-0.000000in}{0.000000in}}%
\pgfpathlineto{\pgfqpoint{-0.048611in}{0.000000in}}%
\pgfusepath{stroke,fill}%
}%
\begin{pgfscope}%
\pgfsys@transformshift{0.550713in}{3.953085in}%
\pgfsys@useobject{currentmarker}{}%
\end{pgfscope}%
\end{pgfscope}%
\begin{pgfscope}%
\definecolor{textcolor}{rgb}{0.000000,0.000000,0.000000}%
\pgfsetstrokecolor{textcolor}%
\pgfsetfillcolor{textcolor}%
\pgftext[x=0.245156in, y=3.904890in, left, base]{\color{textcolor}\rmfamily\fontsize{10.000000}{12.000000}\selectfont \(\displaystyle {200}\)}%
\end{pgfscope}%
\begin{pgfscope}%
\pgfsetbuttcap%
\pgfsetroundjoin%
\definecolor{currentfill}{rgb}{0.000000,0.000000,0.000000}%
\pgfsetfillcolor{currentfill}%
\pgfsetlinewidth{0.803000pt}%
\definecolor{currentstroke}{rgb}{0.000000,0.000000,0.000000}%
\pgfsetstrokecolor{currentstroke}%
\pgfsetdash{}{0pt}%
\pgfsys@defobject{currentmarker}{\pgfqpoint{-0.048611in}{0.000000in}}{\pgfqpoint{-0.000000in}{0.000000in}}{%
\pgfpathmoveto{\pgfqpoint{-0.000000in}{0.000000in}}%
\pgfpathlineto{\pgfqpoint{-0.048611in}{0.000000in}}%
\pgfusepath{stroke,fill}%
}%
\begin{pgfscope}%
\pgfsys@transformshift{0.550713in}{4.286132in}%
\pgfsys@useobject{currentmarker}{}%
\end{pgfscope}%
\end{pgfscope}%
\begin{pgfscope}%
\definecolor{textcolor}{rgb}{0.000000,0.000000,0.000000}%
\pgfsetstrokecolor{textcolor}%
\pgfsetfillcolor{textcolor}%
\pgftext[x=0.245156in, y=4.237937in, left, base]{\color{textcolor}\rmfamily\fontsize{10.000000}{12.000000}\selectfont \(\displaystyle {400}\)}%
\end{pgfscope}%
\begin{pgfscope}%
\pgfsetbuttcap%
\pgfsetroundjoin%
\definecolor{currentfill}{rgb}{0.000000,0.000000,0.000000}%
\pgfsetfillcolor{currentfill}%
\pgfsetlinewidth{0.803000pt}%
\definecolor{currentstroke}{rgb}{0.000000,0.000000,0.000000}%
\pgfsetstrokecolor{currentstroke}%
\pgfsetdash{}{0pt}%
\pgfsys@defobject{currentmarker}{\pgfqpoint{-0.048611in}{0.000000in}}{\pgfqpoint{-0.000000in}{0.000000in}}{%
\pgfpathmoveto{\pgfqpoint{-0.000000in}{0.000000in}}%
\pgfpathlineto{\pgfqpoint{-0.048611in}{0.000000in}}%
\pgfusepath{stroke,fill}%
}%
\begin{pgfscope}%
\pgfsys@transformshift{0.550713in}{4.619178in}%
\pgfsys@useobject{currentmarker}{}%
\end{pgfscope}%
\end{pgfscope}%
\begin{pgfscope}%
\definecolor{textcolor}{rgb}{0.000000,0.000000,0.000000}%
\pgfsetstrokecolor{textcolor}%
\pgfsetfillcolor{textcolor}%
\pgftext[x=0.245156in, y=4.570984in, left, base]{\color{textcolor}\rmfamily\fontsize{10.000000}{12.000000}\selectfont \(\displaystyle {600}\)}%
\end{pgfscope}%
\begin{pgfscope}%
\pgfsetbuttcap%
\pgfsetroundjoin%
\definecolor{currentfill}{rgb}{0.000000,0.000000,0.000000}%
\pgfsetfillcolor{currentfill}%
\pgfsetlinewidth{0.803000pt}%
\definecolor{currentstroke}{rgb}{0.000000,0.000000,0.000000}%
\pgfsetstrokecolor{currentstroke}%
\pgfsetdash{}{0pt}%
\pgfsys@defobject{currentmarker}{\pgfqpoint{-0.048611in}{0.000000in}}{\pgfqpoint{-0.000000in}{0.000000in}}{%
\pgfpathmoveto{\pgfqpoint{-0.000000in}{0.000000in}}%
\pgfpathlineto{\pgfqpoint{-0.048611in}{0.000000in}}%
\pgfusepath{stroke,fill}%
}%
\begin{pgfscope}%
\pgfsys@transformshift{0.550713in}{4.952225in}%
\pgfsys@useobject{currentmarker}{}%
\end{pgfscope}%
\end{pgfscope}%
\begin{pgfscope}%
\definecolor{textcolor}{rgb}{0.000000,0.000000,0.000000}%
\pgfsetstrokecolor{textcolor}%
\pgfsetfillcolor{textcolor}%
\pgftext[x=0.245156in, y=4.904031in, left, base]{\color{textcolor}\rmfamily\fontsize{10.000000}{12.000000}\selectfont \(\displaystyle {800}\)}%
\end{pgfscope}%
\begin{pgfscope}%
\definecolor{textcolor}{rgb}{0.000000,0.000000,0.000000}%
\pgfsetstrokecolor{textcolor}%
\pgfsetfillcolor{textcolor}%
\pgftext[x=0.189601in,y=4.286132in,,bottom,rotate=90.000000]{\color{textcolor}\rmfamily\fontsize{10.000000}{12.000000}\selectfont \(\displaystyle R_T\)}%
\end{pgfscope}%
\begin{pgfscope}%
\pgfpathrectangle{\pgfqpoint{0.550713in}{3.620038in}}{\pgfqpoint{4.791200in}{1.332187in}}%
\pgfusepath{clip}%
\pgfsetbuttcap%
\pgfsetroundjoin%
\pgfsetlinewidth{0.501875pt}%
\definecolor{currentstroke}{rgb}{0.392157,0.396078,0.403922}%
\pgfsetstrokecolor{currentstroke}%
\pgfsetdash{}{0pt}%
\pgfpathmoveto{\pgfqpoint{0.870126in}{3.620038in}}%
\pgfpathlineto{\pgfqpoint{0.870126in}{4.952225in}}%
\pgfusepath{stroke}%
\end{pgfscope}%
\begin{pgfscope}%
\pgfpathrectangle{\pgfqpoint{0.550713in}{3.620038in}}{\pgfqpoint{4.791200in}{1.332187in}}%
\pgfusepath{clip}%
\pgfsetbuttcap%
\pgfsetroundjoin%
\pgfsetlinewidth{0.501875pt}%
\definecolor{currentstroke}{rgb}{0.392157,0.396078,0.403922}%
\pgfsetstrokecolor{currentstroke}%
\pgfsetdash{}{0pt}%
\pgfpathmoveto{\pgfqpoint{1.189539in}{3.620038in}}%
\pgfpathlineto{\pgfqpoint{1.189539in}{4.952225in}}%
\pgfusepath{stroke}%
\end{pgfscope}%
\begin{pgfscope}%
\pgfpathrectangle{\pgfqpoint{0.550713in}{3.620038in}}{\pgfqpoint{4.791200in}{1.332187in}}%
\pgfusepath{clip}%
\pgfsetbuttcap%
\pgfsetroundjoin%
\pgfsetlinewidth{0.501875pt}%
\definecolor{currentstroke}{rgb}{0.392157,0.396078,0.403922}%
\pgfsetstrokecolor{currentstroke}%
\pgfsetdash{}{0pt}%
\pgfpathmoveto{\pgfqpoint{1.508953in}{3.620038in}}%
\pgfpathlineto{\pgfqpoint{1.508953in}{4.952225in}}%
\pgfusepath{stroke}%
\end{pgfscope}%
\begin{pgfscope}%
\pgfpathrectangle{\pgfqpoint{0.550713in}{3.620038in}}{\pgfqpoint{4.791200in}{1.332187in}}%
\pgfusepath{clip}%
\pgfsetbuttcap%
\pgfsetroundjoin%
\pgfsetlinewidth{0.501875pt}%
\definecolor{currentstroke}{rgb}{0.392157,0.396078,0.403922}%
\pgfsetstrokecolor{currentstroke}%
\pgfsetdash{}{0pt}%
\pgfpathmoveto{\pgfqpoint{1.828366in}{3.620038in}}%
\pgfpathlineto{\pgfqpoint{1.828366in}{4.952225in}}%
\pgfusepath{stroke}%
\end{pgfscope}%
\begin{pgfscope}%
\pgfpathrectangle{\pgfqpoint{0.550713in}{3.620038in}}{\pgfqpoint{4.791200in}{1.332187in}}%
\pgfusepath{clip}%
\pgfsetbuttcap%
\pgfsetroundjoin%
\pgfsetlinewidth{0.501875pt}%
\definecolor{currentstroke}{rgb}{0.392157,0.396078,0.403922}%
\pgfsetstrokecolor{currentstroke}%
\pgfsetdash{}{0pt}%
\pgfpathmoveto{\pgfqpoint{2.147779in}{3.620038in}}%
\pgfpathlineto{\pgfqpoint{2.147779in}{4.952225in}}%
\pgfusepath{stroke}%
\end{pgfscope}%
\begin{pgfscope}%
\pgfpathrectangle{\pgfqpoint{0.550713in}{3.620038in}}{\pgfqpoint{4.791200in}{1.332187in}}%
\pgfusepath{clip}%
\pgfsetbuttcap%
\pgfsetroundjoin%
\pgfsetlinewidth{0.501875pt}%
\definecolor{currentstroke}{rgb}{0.392157,0.396078,0.403922}%
\pgfsetstrokecolor{currentstroke}%
\pgfsetdash{}{0pt}%
\pgfpathmoveto{\pgfqpoint{2.467192in}{3.620038in}}%
\pgfpathlineto{\pgfqpoint{2.467192in}{4.952225in}}%
\pgfusepath{stroke}%
\end{pgfscope}%
\begin{pgfscope}%
\pgfpathrectangle{\pgfqpoint{0.550713in}{3.620038in}}{\pgfqpoint{4.791200in}{1.332187in}}%
\pgfusepath{clip}%
\pgfsetbuttcap%
\pgfsetroundjoin%
\pgfsetlinewidth{0.501875pt}%
\definecolor{currentstroke}{rgb}{0.392157,0.396078,0.403922}%
\pgfsetstrokecolor{currentstroke}%
\pgfsetdash{}{0pt}%
\pgfpathmoveto{\pgfqpoint{2.786606in}{3.620038in}}%
\pgfpathlineto{\pgfqpoint{2.786606in}{4.952225in}}%
\pgfusepath{stroke}%
\end{pgfscope}%
\begin{pgfscope}%
\pgfpathrectangle{\pgfqpoint{0.550713in}{3.620038in}}{\pgfqpoint{4.791200in}{1.332187in}}%
\pgfusepath{clip}%
\pgfsetbuttcap%
\pgfsetroundjoin%
\pgfsetlinewidth{0.501875pt}%
\definecolor{currentstroke}{rgb}{0.392157,0.396078,0.403922}%
\pgfsetstrokecolor{currentstroke}%
\pgfsetdash{}{0pt}%
\pgfpathmoveto{\pgfqpoint{3.106019in}{3.620038in}}%
\pgfpathlineto{\pgfqpoint{3.106019in}{4.952225in}}%
\pgfusepath{stroke}%
\end{pgfscope}%
\begin{pgfscope}%
\pgfpathrectangle{\pgfqpoint{0.550713in}{3.620038in}}{\pgfqpoint{4.791200in}{1.332187in}}%
\pgfusepath{clip}%
\pgfsetbuttcap%
\pgfsetroundjoin%
\pgfsetlinewidth{0.501875pt}%
\definecolor{currentstroke}{rgb}{0.392157,0.396078,0.403922}%
\pgfsetstrokecolor{currentstroke}%
\pgfsetdash{}{0pt}%
\pgfpathmoveto{\pgfqpoint{3.425432in}{3.620038in}}%
\pgfpathlineto{\pgfqpoint{3.425432in}{4.952225in}}%
\pgfusepath{stroke}%
\end{pgfscope}%
\begin{pgfscope}%
\pgfpathrectangle{\pgfqpoint{0.550713in}{3.620038in}}{\pgfqpoint{4.791200in}{1.332187in}}%
\pgfusepath{clip}%
\pgfsetbuttcap%
\pgfsetroundjoin%
\pgfsetlinewidth{0.501875pt}%
\definecolor{currentstroke}{rgb}{0.392157,0.396078,0.403922}%
\pgfsetstrokecolor{currentstroke}%
\pgfsetdash{}{0pt}%
\pgfpathmoveto{\pgfqpoint{3.744846in}{3.620038in}}%
\pgfpathlineto{\pgfqpoint{3.744846in}{4.952225in}}%
\pgfusepath{stroke}%
\end{pgfscope}%
\begin{pgfscope}%
\pgfpathrectangle{\pgfqpoint{0.550713in}{3.620038in}}{\pgfqpoint{4.791200in}{1.332187in}}%
\pgfusepath{clip}%
\pgfsetbuttcap%
\pgfsetroundjoin%
\pgfsetlinewidth{0.501875pt}%
\definecolor{currentstroke}{rgb}{0.392157,0.396078,0.403922}%
\pgfsetstrokecolor{currentstroke}%
\pgfsetdash{}{0pt}%
\pgfpathmoveto{\pgfqpoint{4.064259in}{3.620038in}}%
\pgfpathlineto{\pgfqpoint{4.064259in}{4.952225in}}%
\pgfusepath{stroke}%
\end{pgfscope}%
\begin{pgfscope}%
\pgfpathrectangle{\pgfqpoint{0.550713in}{3.620038in}}{\pgfqpoint{4.791200in}{1.332187in}}%
\pgfusepath{clip}%
\pgfsetbuttcap%
\pgfsetroundjoin%
\pgfsetlinewidth{0.501875pt}%
\definecolor{currentstroke}{rgb}{0.392157,0.396078,0.403922}%
\pgfsetstrokecolor{currentstroke}%
\pgfsetdash{}{0pt}%
\pgfpathmoveto{\pgfqpoint{4.383672in}{3.620038in}}%
\pgfpathlineto{\pgfqpoint{4.383672in}{4.952225in}}%
\pgfusepath{stroke}%
\end{pgfscope}%
\begin{pgfscope}%
\pgfpathrectangle{\pgfqpoint{0.550713in}{3.620038in}}{\pgfqpoint{4.791200in}{1.332187in}}%
\pgfusepath{clip}%
\pgfsetbuttcap%
\pgfsetroundjoin%
\pgfsetlinewidth{0.501875pt}%
\definecolor{currentstroke}{rgb}{0.392157,0.396078,0.403922}%
\pgfsetstrokecolor{currentstroke}%
\pgfsetdash{}{0pt}%
\pgfpathmoveto{\pgfqpoint{4.703086in}{3.620038in}}%
\pgfpathlineto{\pgfqpoint{4.703086in}{4.952225in}}%
\pgfusepath{stroke}%
\end{pgfscope}%
\begin{pgfscope}%
\pgfpathrectangle{\pgfqpoint{0.550713in}{3.620038in}}{\pgfqpoint{4.791200in}{1.332187in}}%
\pgfusepath{clip}%
\pgfsetbuttcap%
\pgfsetroundjoin%
\pgfsetlinewidth{0.501875pt}%
\definecolor{currentstroke}{rgb}{0.392157,0.396078,0.403922}%
\pgfsetstrokecolor{currentstroke}%
\pgfsetdash{}{0pt}%
\pgfpathmoveto{\pgfqpoint{5.022499in}{3.620038in}}%
\pgfpathlineto{\pgfqpoint{5.022499in}{4.952225in}}%
\pgfusepath{stroke}%
\end{pgfscope}%
\begin{pgfscope}%
\pgfpathrectangle{\pgfqpoint{0.550713in}{3.620038in}}{\pgfqpoint{4.791200in}{1.332187in}}%
\pgfusepath{clip}%
\pgfsetbuttcap%
\pgfsetroundjoin%
\pgfsetlinewidth{0.752812pt}%
\definecolor{currentstroke}{rgb}{0.150000,0.150000,0.150000}%
\pgfsetstrokecolor{currentstroke}%
\pgfsetstrokeopacity{0.450000}%
\pgfsetdash{}{0pt}%
\pgfpathmoveto{\pgfqpoint{0.583932in}{3.891805in}}%
\pgfpathlineto{\pgfqpoint{0.709142in}{3.891805in}}%
\pgfusepath{stroke}%
\end{pgfscope}%
\begin{pgfscope}%
\pgfpathrectangle{\pgfqpoint{0.550713in}{3.620038in}}{\pgfqpoint{4.791200in}{1.332187in}}%
\pgfusepath{clip}%
\pgfsetbuttcap%
\pgfsetroundjoin%
\pgfsetlinewidth{0.752812pt}%
\definecolor{currentstroke}{rgb}{0.150000,0.150000,0.150000}%
\pgfsetstrokecolor{currentstroke}%
\pgfsetstrokeopacity{0.450000}%
\pgfsetdash{}{0pt}%
\pgfpathmoveto{\pgfqpoint{0.711697in}{3.693592in}}%
\pgfpathlineto{\pgfqpoint{0.836907in}{3.693592in}}%
\pgfusepath{stroke}%
\end{pgfscope}%
\begin{pgfscope}%
\pgfpathrectangle{\pgfqpoint{0.550713in}{3.620038in}}{\pgfqpoint{4.791200in}{1.332187in}}%
\pgfusepath{clip}%
\pgfsetbuttcap%
\pgfsetroundjoin%
\pgfsetlinewidth{0.752812pt}%
\definecolor{currentstroke}{rgb}{0.150000,0.150000,0.150000}%
\pgfsetstrokecolor{currentstroke}%
\pgfsetstrokeopacity{0.450000}%
\pgfsetdash{}{0pt}%
\pgfpathmoveto{\pgfqpoint{0.903345in}{3.922716in}}%
\pgfpathlineto{\pgfqpoint{1.028555in}{3.922716in}}%
\pgfusepath{stroke}%
\end{pgfscope}%
\begin{pgfscope}%
\pgfpathrectangle{\pgfqpoint{0.550713in}{3.620038in}}{\pgfqpoint{4.791200in}{1.332187in}}%
\pgfusepath{clip}%
\pgfsetbuttcap%
\pgfsetroundjoin%
\pgfsetlinewidth{0.752812pt}%
\definecolor{currentstroke}{rgb}{0.150000,0.150000,0.150000}%
\pgfsetstrokecolor{currentstroke}%
\pgfsetstrokeopacity{0.450000}%
\pgfsetdash{}{0pt}%
\pgfpathmoveto{\pgfqpoint{1.031110in}{3.707436in}}%
\pgfpathlineto{\pgfqpoint{1.156320in}{3.707436in}}%
\pgfusepath{stroke}%
\end{pgfscope}%
\begin{pgfscope}%
\pgfpathrectangle{\pgfqpoint{0.550713in}{3.620038in}}{\pgfqpoint{4.791200in}{1.332187in}}%
\pgfusepath{clip}%
\pgfsetbuttcap%
\pgfsetroundjoin%
\pgfsetlinewidth{0.752812pt}%
\definecolor{currentstroke}{rgb}{0.150000,0.150000,0.150000}%
\pgfsetstrokecolor{currentstroke}%
\pgfsetstrokeopacity{0.450000}%
\pgfsetdash{}{0pt}%
\pgfpathmoveto{\pgfqpoint{1.222758in}{3.741360in}}%
\pgfpathlineto{\pgfqpoint{1.347968in}{3.741360in}}%
\pgfusepath{stroke}%
\end{pgfscope}%
\begin{pgfscope}%
\pgfpathrectangle{\pgfqpoint{0.550713in}{3.620038in}}{\pgfqpoint{4.791200in}{1.332187in}}%
\pgfusepath{clip}%
\pgfsetbuttcap%
\pgfsetroundjoin%
\pgfsetlinewidth{0.752812pt}%
\definecolor{currentstroke}{rgb}{0.150000,0.150000,0.150000}%
\pgfsetstrokecolor{currentstroke}%
\pgfsetstrokeopacity{0.450000}%
\pgfsetdash{}{0pt}%
\pgfpathmoveto{\pgfqpoint{1.350524in}{3.709815in}}%
\pgfpathlineto{\pgfqpoint{1.475734in}{3.709815in}}%
\pgfusepath{stroke}%
\end{pgfscope}%
\begin{pgfscope}%
\pgfpathrectangle{\pgfqpoint{0.550713in}{3.620038in}}{\pgfqpoint{4.791200in}{1.332187in}}%
\pgfusepath{clip}%
\pgfsetbuttcap%
\pgfsetroundjoin%
\pgfsetlinewidth{0.752812pt}%
\definecolor{currentstroke}{rgb}{0.150000,0.150000,0.150000}%
\pgfsetstrokecolor{currentstroke}%
\pgfsetstrokeopacity{0.450000}%
\pgfsetdash{}{0pt}%
\pgfpathmoveto{\pgfqpoint{1.542172in}{3.806545in}}%
\pgfpathlineto{\pgfqpoint{1.667382in}{3.806545in}}%
\pgfusepath{stroke}%
\end{pgfscope}%
\begin{pgfscope}%
\pgfpathrectangle{\pgfqpoint{0.550713in}{3.620038in}}{\pgfqpoint{4.791200in}{1.332187in}}%
\pgfusepath{clip}%
\pgfsetbuttcap%
\pgfsetroundjoin%
\pgfsetlinewidth{0.752812pt}%
\definecolor{currentstroke}{rgb}{0.150000,0.150000,0.150000}%
\pgfsetstrokecolor{currentstroke}%
\pgfsetstrokeopacity{0.450000}%
\pgfsetdash{}{0pt}%
\pgfpathmoveto{\pgfqpoint{1.669937in}{3.714096in}}%
\pgfpathlineto{\pgfqpoint{1.795147in}{3.714096in}}%
\pgfusepath{stroke}%
\end{pgfscope}%
\begin{pgfscope}%
\pgfpathrectangle{\pgfqpoint{0.550713in}{3.620038in}}{\pgfqpoint{4.791200in}{1.332187in}}%
\pgfusepath{clip}%
\pgfsetbuttcap%
\pgfsetroundjoin%
\pgfsetlinewidth{0.752812pt}%
\definecolor{currentstroke}{rgb}{0.150000,0.150000,0.150000}%
\pgfsetstrokecolor{currentstroke}%
\pgfsetstrokeopacity{0.450000}%
\pgfsetdash{}{0pt}%
\pgfpathmoveto{\pgfqpoint{1.861585in}{3.803784in}}%
\pgfpathlineto{\pgfqpoint{1.986795in}{3.803784in}}%
\pgfusepath{stroke}%
\end{pgfscope}%
\begin{pgfscope}%
\pgfpathrectangle{\pgfqpoint{0.550713in}{3.620038in}}{\pgfqpoint{4.791200in}{1.332187in}}%
\pgfusepath{clip}%
\pgfsetbuttcap%
\pgfsetroundjoin%
\pgfsetlinewidth{0.752812pt}%
\definecolor{currentstroke}{rgb}{0.150000,0.150000,0.150000}%
\pgfsetstrokecolor{currentstroke}%
\pgfsetstrokeopacity{0.450000}%
\pgfsetdash{}{0pt}%
\pgfpathmoveto{\pgfqpoint{1.989350in}{3.710591in}}%
\pgfpathlineto{\pgfqpoint{2.114560in}{3.710591in}}%
\pgfusepath{stroke}%
\end{pgfscope}%
\begin{pgfscope}%
\pgfpathrectangle{\pgfqpoint{0.550713in}{3.620038in}}{\pgfqpoint{4.791200in}{1.332187in}}%
\pgfusepath{clip}%
\pgfsetbuttcap%
\pgfsetroundjoin%
\pgfsetlinewidth{0.752812pt}%
\definecolor{currentstroke}{rgb}{0.150000,0.150000,0.150000}%
\pgfsetstrokecolor{currentstroke}%
\pgfsetstrokeopacity{0.450000}%
\pgfsetdash{}{0pt}%
\pgfpathmoveto{\pgfqpoint{2.180998in}{3.798277in}}%
\pgfpathlineto{\pgfqpoint{2.306208in}{3.798277in}}%
\pgfusepath{stroke}%
\end{pgfscope}%
\begin{pgfscope}%
\pgfpathrectangle{\pgfqpoint{0.550713in}{3.620038in}}{\pgfqpoint{4.791200in}{1.332187in}}%
\pgfusepath{clip}%
\pgfsetbuttcap%
\pgfsetroundjoin%
\pgfsetlinewidth{0.752812pt}%
\definecolor{currentstroke}{rgb}{0.150000,0.150000,0.150000}%
\pgfsetstrokecolor{currentstroke}%
\pgfsetstrokeopacity{0.450000}%
\pgfsetdash{}{0pt}%
\pgfpathmoveto{\pgfqpoint{2.308763in}{3.759300in}}%
\pgfpathlineto{\pgfqpoint{2.433973in}{3.759300in}}%
\pgfusepath{stroke}%
\end{pgfscope}%
\begin{pgfscope}%
\pgfpathrectangle{\pgfqpoint{0.550713in}{3.620038in}}{\pgfqpoint{4.791200in}{1.332187in}}%
\pgfusepath{clip}%
\pgfsetbuttcap%
\pgfsetroundjoin%
\pgfsetlinewidth{0.752812pt}%
\definecolor{currentstroke}{rgb}{0.150000,0.150000,0.150000}%
\pgfsetstrokecolor{currentstroke}%
\pgfsetstrokeopacity{0.450000}%
\pgfsetdash{}{0pt}%
\pgfpathmoveto{\pgfqpoint{2.500411in}{3.781145in}}%
\pgfpathlineto{\pgfqpoint{2.625621in}{3.781145in}}%
\pgfusepath{stroke}%
\end{pgfscope}%
\begin{pgfscope}%
\pgfpathrectangle{\pgfqpoint{0.550713in}{3.620038in}}{\pgfqpoint{4.791200in}{1.332187in}}%
\pgfusepath{clip}%
\pgfsetbuttcap%
\pgfsetroundjoin%
\pgfsetlinewidth{0.752812pt}%
\definecolor{currentstroke}{rgb}{0.150000,0.150000,0.150000}%
\pgfsetstrokecolor{currentstroke}%
\pgfsetstrokeopacity{0.450000}%
\pgfsetdash{}{0pt}%
\pgfpathmoveto{\pgfqpoint{2.628177in}{3.783761in}}%
\pgfpathlineto{\pgfqpoint{2.753387in}{3.783761in}}%
\pgfusepath{stroke}%
\end{pgfscope}%
\begin{pgfscope}%
\pgfpathrectangle{\pgfqpoint{0.550713in}{3.620038in}}{\pgfqpoint{4.791200in}{1.332187in}}%
\pgfusepath{clip}%
\pgfsetbuttcap%
\pgfsetroundjoin%
\pgfsetlinewidth{0.752812pt}%
\definecolor{currentstroke}{rgb}{0.150000,0.150000,0.150000}%
\pgfsetstrokecolor{currentstroke}%
\pgfsetstrokeopacity{0.450000}%
\pgfsetdash{}{0pt}%
\pgfpathmoveto{\pgfqpoint{2.819825in}{3.790851in}}%
\pgfpathlineto{\pgfqpoint{2.945035in}{3.790851in}}%
\pgfusepath{stroke}%
\end{pgfscope}%
\begin{pgfscope}%
\pgfpathrectangle{\pgfqpoint{0.550713in}{3.620038in}}{\pgfqpoint{4.791200in}{1.332187in}}%
\pgfusepath{clip}%
\pgfsetbuttcap%
\pgfsetroundjoin%
\pgfsetlinewidth{0.752812pt}%
\definecolor{currentstroke}{rgb}{0.150000,0.150000,0.150000}%
\pgfsetstrokecolor{currentstroke}%
\pgfsetstrokeopacity{0.450000}%
\pgfsetdash{}{0pt}%
\pgfpathmoveto{\pgfqpoint{2.947590in}{3.851082in}}%
\pgfpathlineto{\pgfqpoint{3.072800in}{3.851082in}}%
\pgfusepath{stroke}%
\end{pgfscope}%
\begin{pgfscope}%
\pgfpathrectangle{\pgfqpoint{0.550713in}{3.620038in}}{\pgfqpoint{4.791200in}{1.332187in}}%
\pgfusepath{clip}%
\pgfsetbuttcap%
\pgfsetroundjoin%
\pgfsetlinewidth{0.752812pt}%
\definecolor{currentstroke}{rgb}{0.150000,0.150000,0.150000}%
\pgfsetstrokecolor{currentstroke}%
\pgfsetstrokeopacity{0.450000}%
\pgfsetdash{}{0pt}%
\pgfpathmoveto{\pgfqpoint{3.139238in}{3.811675in}}%
\pgfpathlineto{\pgfqpoint{3.264448in}{3.811675in}}%
\pgfusepath{stroke}%
\end{pgfscope}%
\begin{pgfscope}%
\pgfpathrectangle{\pgfqpoint{0.550713in}{3.620038in}}{\pgfqpoint{4.791200in}{1.332187in}}%
\pgfusepath{clip}%
\pgfsetbuttcap%
\pgfsetroundjoin%
\pgfsetlinewidth{0.752812pt}%
\definecolor{currentstroke}{rgb}{0.150000,0.150000,0.150000}%
\pgfsetstrokecolor{currentstroke}%
\pgfsetstrokeopacity{0.450000}%
\pgfsetdash{}{0pt}%
\pgfpathmoveto{\pgfqpoint{3.267003in}{3.882549in}}%
\pgfpathlineto{\pgfqpoint{3.392213in}{3.882549in}}%
\pgfusepath{stroke}%
\end{pgfscope}%
\begin{pgfscope}%
\pgfpathrectangle{\pgfqpoint{0.550713in}{3.620038in}}{\pgfqpoint{4.791200in}{1.332187in}}%
\pgfusepath{clip}%
\pgfsetbuttcap%
\pgfsetroundjoin%
\pgfsetlinewidth{0.752812pt}%
\definecolor{currentstroke}{rgb}{0.150000,0.150000,0.150000}%
\pgfsetstrokecolor{currentstroke}%
\pgfsetstrokeopacity{0.450000}%
\pgfsetdash{}{0pt}%
\pgfpathmoveto{\pgfqpoint{3.458651in}{3.819375in}}%
\pgfpathlineto{\pgfqpoint{3.583861in}{3.819375in}}%
\pgfusepath{stroke}%
\end{pgfscope}%
\begin{pgfscope}%
\pgfpathrectangle{\pgfqpoint{0.550713in}{3.620038in}}{\pgfqpoint{4.791200in}{1.332187in}}%
\pgfusepath{clip}%
\pgfsetbuttcap%
\pgfsetroundjoin%
\pgfsetlinewidth{0.752812pt}%
\definecolor{currentstroke}{rgb}{0.150000,0.150000,0.150000}%
\pgfsetstrokecolor{currentstroke}%
\pgfsetstrokeopacity{0.450000}%
\pgfsetdash{}{0pt}%
\pgfpathmoveto{\pgfqpoint{3.586417in}{3.946827in}}%
\pgfpathlineto{\pgfqpoint{3.711627in}{3.946827in}}%
\pgfusepath{stroke}%
\end{pgfscope}%
\begin{pgfscope}%
\pgfpathrectangle{\pgfqpoint{0.550713in}{3.620038in}}{\pgfqpoint{4.791200in}{1.332187in}}%
\pgfusepath{clip}%
\pgfsetbuttcap%
\pgfsetroundjoin%
\pgfsetlinewidth{0.752812pt}%
\definecolor{currentstroke}{rgb}{0.150000,0.150000,0.150000}%
\pgfsetstrokecolor{currentstroke}%
\pgfsetstrokeopacity{0.450000}%
\pgfsetdash{}{0pt}%
\pgfpathmoveto{\pgfqpoint{3.778065in}{3.883452in}}%
\pgfpathlineto{\pgfqpoint{3.903275in}{3.883452in}}%
\pgfusepath{stroke}%
\end{pgfscope}%
\begin{pgfscope}%
\pgfpathrectangle{\pgfqpoint{0.550713in}{3.620038in}}{\pgfqpoint{4.791200in}{1.332187in}}%
\pgfusepath{clip}%
\pgfsetbuttcap%
\pgfsetroundjoin%
\pgfsetlinewidth{0.752812pt}%
\definecolor{currentstroke}{rgb}{0.150000,0.150000,0.150000}%
\pgfsetstrokecolor{currentstroke}%
\pgfsetstrokeopacity{0.450000}%
\pgfsetdash{}{0pt}%
\pgfpathmoveto{\pgfqpoint{3.905830in}{3.972874in}}%
\pgfpathlineto{\pgfqpoint{4.031040in}{3.972874in}}%
\pgfusepath{stroke}%
\end{pgfscope}%
\begin{pgfscope}%
\pgfpathrectangle{\pgfqpoint{0.550713in}{3.620038in}}{\pgfqpoint{4.791200in}{1.332187in}}%
\pgfusepath{clip}%
\pgfsetbuttcap%
\pgfsetroundjoin%
\pgfsetlinewidth{0.752812pt}%
\definecolor{currentstroke}{rgb}{0.150000,0.150000,0.150000}%
\pgfsetstrokecolor{currentstroke}%
\pgfsetstrokeopacity{0.450000}%
\pgfsetdash{}{0pt}%
\pgfpathmoveto{\pgfqpoint{4.097478in}{3.911263in}}%
\pgfpathlineto{\pgfqpoint{4.222688in}{3.911263in}}%
\pgfusepath{stroke}%
\end{pgfscope}%
\begin{pgfscope}%
\pgfpathrectangle{\pgfqpoint{0.550713in}{3.620038in}}{\pgfqpoint{4.791200in}{1.332187in}}%
\pgfusepath{clip}%
\pgfsetbuttcap%
\pgfsetroundjoin%
\pgfsetlinewidth{0.752812pt}%
\definecolor{currentstroke}{rgb}{0.150000,0.150000,0.150000}%
\pgfsetstrokecolor{currentstroke}%
\pgfsetstrokeopacity{0.450000}%
\pgfsetdash{}{0pt}%
\pgfpathmoveto{\pgfqpoint{4.225243in}{4.054914in}}%
\pgfpathlineto{\pgfqpoint{4.350453in}{4.054914in}}%
\pgfusepath{stroke}%
\end{pgfscope}%
\begin{pgfscope}%
\pgfpathrectangle{\pgfqpoint{0.550713in}{3.620038in}}{\pgfqpoint{4.791200in}{1.332187in}}%
\pgfusepath{clip}%
\pgfsetbuttcap%
\pgfsetroundjoin%
\pgfsetlinewidth{0.752812pt}%
\definecolor{currentstroke}{rgb}{0.150000,0.150000,0.150000}%
\pgfsetstrokecolor{currentstroke}%
\pgfsetstrokeopacity{0.450000}%
\pgfsetdash{}{0pt}%
\pgfpathmoveto{\pgfqpoint{4.416891in}{3.960244in}}%
\pgfpathlineto{\pgfqpoint{4.542101in}{3.960244in}}%
\pgfusepath{stroke}%
\end{pgfscope}%
\begin{pgfscope}%
\pgfpathrectangle{\pgfqpoint{0.550713in}{3.620038in}}{\pgfqpoint{4.791200in}{1.332187in}}%
\pgfusepath{clip}%
\pgfsetbuttcap%
\pgfsetroundjoin%
\pgfsetlinewidth{0.752812pt}%
\definecolor{currentstroke}{rgb}{0.150000,0.150000,0.150000}%
\pgfsetstrokecolor{currentstroke}%
\pgfsetstrokeopacity{0.450000}%
\pgfsetdash{}{0pt}%
\pgfpathmoveto{\pgfqpoint{4.544657in}{4.072087in}}%
\pgfpathlineto{\pgfqpoint{4.669867in}{4.072087in}}%
\pgfusepath{stroke}%
\end{pgfscope}%
\begin{pgfscope}%
\pgfpathrectangle{\pgfqpoint{0.550713in}{3.620038in}}{\pgfqpoint{4.791200in}{1.332187in}}%
\pgfusepath{clip}%
\pgfsetbuttcap%
\pgfsetroundjoin%
\pgfsetlinewidth{0.752812pt}%
\definecolor{currentstroke}{rgb}{0.150000,0.150000,0.150000}%
\pgfsetstrokecolor{currentstroke}%
\pgfsetstrokeopacity{0.450000}%
\pgfsetdash{}{0pt}%
\pgfpathmoveto{\pgfqpoint{4.736305in}{4.259520in}}%
\pgfpathlineto{\pgfqpoint{4.861515in}{4.259520in}}%
\pgfusepath{stroke}%
\end{pgfscope}%
\begin{pgfscope}%
\pgfpathrectangle{\pgfqpoint{0.550713in}{3.620038in}}{\pgfqpoint{4.791200in}{1.332187in}}%
\pgfusepath{clip}%
\pgfsetbuttcap%
\pgfsetroundjoin%
\pgfsetlinewidth{0.752812pt}%
\definecolor{currentstroke}{rgb}{0.150000,0.150000,0.150000}%
\pgfsetstrokecolor{currentstroke}%
\pgfsetstrokeopacity{0.450000}%
\pgfsetdash{}{0pt}%
\pgfpathmoveto{\pgfqpoint{4.864070in}{3.829292in}}%
\pgfpathlineto{\pgfqpoint{4.989280in}{3.829292in}}%
\pgfusepath{stroke}%
\end{pgfscope}%
\begin{pgfscope}%
\pgfpathrectangle{\pgfqpoint{0.550713in}{3.620038in}}{\pgfqpoint{4.791200in}{1.332187in}}%
\pgfusepath{clip}%
\pgfsetbuttcap%
\pgfsetroundjoin%
\pgfsetlinewidth{0.752812pt}%
\definecolor{currentstroke}{rgb}{0.150000,0.150000,0.150000}%
\pgfsetstrokecolor{currentstroke}%
\pgfsetstrokeopacity{0.450000}%
\pgfsetdash{}{0pt}%
\pgfpathmoveto{\pgfqpoint{5.055718in}{4.519210in}}%
\pgfpathlineto{\pgfqpoint{5.180928in}{4.519210in}}%
\pgfusepath{stroke}%
\end{pgfscope}%
\begin{pgfscope}%
\pgfpathrectangle{\pgfqpoint{0.550713in}{3.620038in}}{\pgfqpoint{4.791200in}{1.332187in}}%
\pgfusepath{clip}%
\pgfsetbuttcap%
\pgfsetroundjoin%
\pgfsetlinewidth{0.752812pt}%
\definecolor{currentstroke}{rgb}{0.150000,0.150000,0.150000}%
\pgfsetstrokecolor{currentstroke}%
\pgfsetstrokeopacity{0.450000}%
\pgfsetdash{}{0pt}%
\pgfpathmoveto{\pgfqpoint{5.183483in}{4.678900in}}%
\pgfpathlineto{\pgfqpoint{5.308693in}{4.678900in}}%
\pgfusepath{stroke}%
\end{pgfscope}%
\begin{pgfscope}%
\pgfpathrectangle{\pgfqpoint{0.550713in}{3.620038in}}{\pgfqpoint{4.791200in}{1.332187in}}%
\pgfusepath{clip}%
\pgfsetbuttcap%
\pgfsetroundjoin%
\pgfsetlinewidth{0.853187pt}%
\definecolor{currentstroke}{rgb}{0.380392,0.129412,0.345098}%
\pgfsetstrokecolor{currentstroke}%
\pgfsetdash{{3.145000pt}{1.360000pt}}{0.000000pt}%
\pgfpathmoveto{\pgfqpoint{0.774302in}{3.713303in}}%
\pgfpathlineto{\pgfqpoint{1.093715in}{3.717716in}}%
\pgfpathlineto{\pgfqpoint{1.413129in}{3.709934in}}%
\pgfpathlineto{\pgfqpoint{1.732542in}{3.710539in}}%
\pgfpathlineto{\pgfqpoint{2.051955in}{3.721207in}}%
\pgfpathlineto{\pgfqpoint{2.371368in}{3.753410in}}%
\pgfpathlineto{\pgfqpoint{2.690782in}{3.781060in}}%
\pgfpathlineto{\pgfqpoint{3.010195in}{3.819969in}}%
\pgfpathlineto{\pgfqpoint{3.329608in}{3.859239in}}%
\pgfpathlineto{\pgfqpoint{3.649022in}{3.916932in}}%
\pgfpathlineto{\pgfqpoint{3.968435in}{3.940568in}}%
\pgfpathlineto{\pgfqpoint{4.287848in}{4.008909in}}%
\pgfpathlineto{\pgfqpoint{4.607262in}{4.015292in}}%
\pgfpathlineto{\pgfqpoint{4.926675in}{3.819934in}}%
\pgfpathlineto{\pgfqpoint{5.246088in}{4.836215in}}%
\pgfusepath{stroke}%
\end{pgfscope}%
\begin{pgfscope}%
\pgfpathrectangle{\pgfqpoint{0.550713in}{3.620038in}}{\pgfqpoint{4.791200in}{1.332187in}}%
\pgfusepath{clip}%
\pgfsetbuttcap%
\pgfsetroundjoin%
\definecolor{currentfill}{rgb}{0.380392,0.129412,0.345098}%
\pgfsetfillcolor{currentfill}%
\pgfsetlinewidth{0.752812pt}%
\definecolor{currentstroke}{rgb}{1.000000,1.000000,1.000000}%
\pgfsetstrokecolor{currentstroke}%
\pgfsetdash{}{0pt}%
\pgfsys@defobject{currentmarker}{\pgfqpoint{-0.027778in}{-0.027778in}}{\pgfqpoint{0.027778in}{0.027778in}}{%
\pgfpathmoveto{\pgfqpoint{0.000000in}{-0.027778in}}%
\pgfpathcurveto{\pgfqpoint{0.007367in}{-0.027778in}}{\pgfqpoint{0.014433in}{-0.024851in}}{\pgfqpoint{0.019642in}{-0.019642in}}%
\pgfpathcurveto{\pgfqpoint{0.024851in}{-0.014433in}}{\pgfqpoint{0.027778in}{-0.007367in}}{\pgfqpoint{0.027778in}{0.000000in}}%
\pgfpathcurveto{\pgfqpoint{0.027778in}{0.007367in}}{\pgfqpoint{0.024851in}{0.014433in}}{\pgfqpoint{0.019642in}{0.019642in}}%
\pgfpathcurveto{\pgfqpoint{0.014433in}{0.024851in}}{\pgfqpoint{0.007367in}{0.027778in}}{\pgfqpoint{0.000000in}{0.027778in}}%
\pgfpathcurveto{\pgfqpoint{-0.007367in}{0.027778in}}{\pgfqpoint{-0.014433in}{0.024851in}}{\pgfqpoint{-0.019642in}{0.019642in}}%
\pgfpathcurveto{\pgfqpoint{-0.024851in}{0.014433in}}{\pgfqpoint{-0.027778in}{0.007367in}}{\pgfqpoint{-0.027778in}{0.000000in}}%
\pgfpathcurveto{\pgfqpoint{-0.027778in}{-0.007367in}}{\pgfqpoint{-0.024851in}{-0.014433in}}{\pgfqpoint{-0.019642in}{-0.019642in}}%
\pgfpathcurveto{\pgfqpoint{-0.014433in}{-0.024851in}}{\pgfqpoint{-0.007367in}{-0.027778in}}{\pgfqpoint{0.000000in}{-0.027778in}}%
\pgfpathclose%
\pgfusepath{stroke,fill}%
}%
\begin{pgfscope}%
\pgfsys@transformshift{0.774302in}{3.713303in}%
\pgfsys@useobject{currentmarker}{}%
\end{pgfscope}%
\begin{pgfscope}%
\pgfsys@transformshift{1.093715in}{3.717716in}%
\pgfsys@useobject{currentmarker}{}%
\end{pgfscope}%
\begin{pgfscope}%
\pgfsys@transformshift{1.413129in}{3.709934in}%
\pgfsys@useobject{currentmarker}{}%
\end{pgfscope}%
\begin{pgfscope}%
\pgfsys@transformshift{1.732542in}{3.710539in}%
\pgfsys@useobject{currentmarker}{}%
\end{pgfscope}%
\begin{pgfscope}%
\pgfsys@transformshift{2.051955in}{3.721207in}%
\pgfsys@useobject{currentmarker}{}%
\end{pgfscope}%
\begin{pgfscope}%
\pgfsys@transformshift{2.371368in}{3.753410in}%
\pgfsys@useobject{currentmarker}{}%
\end{pgfscope}%
\begin{pgfscope}%
\pgfsys@transformshift{2.690782in}{3.781060in}%
\pgfsys@useobject{currentmarker}{}%
\end{pgfscope}%
\begin{pgfscope}%
\pgfsys@transformshift{3.010195in}{3.819969in}%
\pgfsys@useobject{currentmarker}{}%
\end{pgfscope}%
\begin{pgfscope}%
\pgfsys@transformshift{3.329608in}{3.859239in}%
\pgfsys@useobject{currentmarker}{}%
\end{pgfscope}%
\begin{pgfscope}%
\pgfsys@transformshift{3.649022in}{3.916932in}%
\pgfsys@useobject{currentmarker}{}%
\end{pgfscope}%
\begin{pgfscope}%
\pgfsys@transformshift{3.968435in}{3.940568in}%
\pgfsys@useobject{currentmarker}{}%
\end{pgfscope}%
\begin{pgfscope}%
\pgfsys@transformshift{4.287848in}{4.008909in}%
\pgfsys@useobject{currentmarker}{}%
\end{pgfscope}%
\begin{pgfscope}%
\pgfsys@transformshift{4.607262in}{4.015292in}%
\pgfsys@useobject{currentmarker}{}%
\end{pgfscope}%
\begin{pgfscope}%
\pgfsys@transformshift{4.926675in}{3.819934in}%
\pgfsys@useobject{currentmarker}{}%
\end{pgfscope}%
\begin{pgfscope}%
\pgfsys@transformshift{5.246088in}{4.836215in}%
\pgfsys@useobject{currentmarker}{}%
\end{pgfscope}%
\end{pgfscope}%
\begin{pgfscope}%
\pgfpathrectangle{\pgfqpoint{0.550713in}{3.620038in}}{\pgfqpoint{4.791200in}{1.332187in}}%
\pgfusepath{clip}%
\pgfsetbuttcap%
\pgfsetroundjoin%
\pgfsetlinewidth{0.853187pt}%
\definecolor{currentstroke}{rgb}{0.631373,0.062745,0.207843}%
\pgfsetstrokecolor{currentstroke}%
\pgfsetdash{{3.145000pt}{1.360000pt}}{0.000000pt}%
\pgfpathmoveto{\pgfqpoint{0.646537in}{3.910012in}}%
\pgfpathlineto{\pgfqpoint{0.965950in}{3.939240in}}%
\pgfpathlineto{\pgfqpoint{1.285363in}{3.796582in}}%
\pgfpathlineto{\pgfqpoint{1.604777in}{3.821202in}}%
\pgfpathlineto{\pgfqpoint{1.924190in}{3.803346in}}%
\pgfpathlineto{\pgfqpoint{2.243603in}{3.804677in}}%
\pgfpathlineto{\pgfqpoint{2.563016in}{3.795745in}}%
\pgfpathlineto{\pgfqpoint{2.882430in}{3.785862in}}%
\pgfpathlineto{\pgfqpoint{3.201843in}{3.817530in}}%
\pgfpathlineto{\pgfqpoint{3.521256in}{3.855782in}}%
\pgfpathlineto{\pgfqpoint{3.840670in}{3.880564in}}%
\pgfpathlineto{\pgfqpoint{4.160083in}{3.909097in}}%
\pgfpathlineto{\pgfqpoint{4.479496in}{3.974584in}}%
\pgfpathlineto{\pgfqpoint{4.798910in}{4.265153in}}%
\pgfpathlineto{\pgfqpoint{5.118323in}{4.426849in}}%
\pgfusepath{stroke}%
\end{pgfscope}%
\begin{pgfscope}%
\pgfpathrectangle{\pgfqpoint{0.550713in}{3.620038in}}{\pgfqpoint{4.791200in}{1.332187in}}%
\pgfusepath{clip}%
\pgfsetbuttcap%
\pgfsetroundjoin%
\definecolor{currentfill}{rgb}{0.631373,0.062745,0.207843}%
\pgfsetfillcolor{currentfill}%
\pgfsetlinewidth{0.752812pt}%
\definecolor{currentstroke}{rgb}{1.000000,1.000000,1.000000}%
\pgfsetstrokecolor{currentstroke}%
\pgfsetdash{}{0pt}%
\pgfsys@defobject{currentmarker}{\pgfqpoint{-0.027778in}{-0.027778in}}{\pgfqpoint{0.027778in}{0.027778in}}{%
\pgfpathmoveto{\pgfqpoint{0.000000in}{-0.027778in}}%
\pgfpathcurveto{\pgfqpoint{0.007367in}{-0.027778in}}{\pgfqpoint{0.014433in}{-0.024851in}}{\pgfqpoint{0.019642in}{-0.019642in}}%
\pgfpathcurveto{\pgfqpoint{0.024851in}{-0.014433in}}{\pgfqpoint{0.027778in}{-0.007367in}}{\pgfqpoint{0.027778in}{0.000000in}}%
\pgfpathcurveto{\pgfqpoint{0.027778in}{0.007367in}}{\pgfqpoint{0.024851in}{0.014433in}}{\pgfqpoint{0.019642in}{0.019642in}}%
\pgfpathcurveto{\pgfqpoint{0.014433in}{0.024851in}}{\pgfqpoint{0.007367in}{0.027778in}}{\pgfqpoint{0.000000in}{0.027778in}}%
\pgfpathcurveto{\pgfqpoint{-0.007367in}{0.027778in}}{\pgfqpoint{-0.014433in}{0.024851in}}{\pgfqpoint{-0.019642in}{0.019642in}}%
\pgfpathcurveto{\pgfqpoint{-0.024851in}{0.014433in}}{\pgfqpoint{-0.027778in}{0.007367in}}{\pgfqpoint{-0.027778in}{0.000000in}}%
\pgfpathcurveto{\pgfqpoint{-0.027778in}{-0.007367in}}{\pgfqpoint{-0.024851in}{-0.014433in}}{\pgfqpoint{-0.019642in}{-0.019642in}}%
\pgfpathcurveto{\pgfqpoint{-0.014433in}{-0.024851in}}{\pgfqpoint{-0.007367in}{-0.027778in}}{\pgfqpoint{0.000000in}{-0.027778in}}%
\pgfpathclose%
\pgfusepath{stroke,fill}%
}%
\begin{pgfscope}%
\pgfsys@transformshift{0.646537in}{3.910012in}%
\pgfsys@useobject{currentmarker}{}%
\end{pgfscope}%
\begin{pgfscope}%
\pgfsys@transformshift{0.965950in}{3.939240in}%
\pgfsys@useobject{currentmarker}{}%
\end{pgfscope}%
\begin{pgfscope}%
\pgfsys@transformshift{1.285363in}{3.796582in}%
\pgfsys@useobject{currentmarker}{}%
\end{pgfscope}%
\begin{pgfscope}%
\pgfsys@transformshift{1.604777in}{3.821202in}%
\pgfsys@useobject{currentmarker}{}%
\end{pgfscope}%
\begin{pgfscope}%
\pgfsys@transformshift{1.924190in}{3.803346in}%
\pgfsys@useobject{currentmarker}{}%
\end{pgfscope}%
\begin{pgfscope}%
\pgfsys@transformshift{2.243603in}{3.804677in}%
\pgfsys@useobject{currentmarker}{}%
\end{pgfscope}%
\begin{pgfscope}%
\pgfsys@transformshift{2.563016in}{3.795745in}%
\pgfsys@useobject{currentmarker}{}%
\end{pgfscope}%
\begin{pgfscope}%
\pgfsys@transformshift{2.882430in}{3.785862in}%
\pgfsys@useobject{currentmarker}{}%
\end{pgfscope}%
\begin{pgfscope}%
\pgfsys@transformshift{3.201843in}{3.817530in}%
\pgfsys@useobject{currentmarker}{}%
\end{pgfscope}%
\begin{pgfscope}%
\pgfsys@transformshift{3.521256in}{3.855782in}%
\pgfsys@useobject{currentmarker}{}%
\end{pgfscope}%
\begin{pgfscope}%
\pgfsys@transformshift{3.840670in}{3.880564in}%
\pgfsys@useobject{currentmarker}{}%
\end{pgfscope}%
\begin{pgfscope}%
\pgfsys@transformshift{4.160083in}{3.909097in}%
\pgfsys@useobject{currentmarker}{}%
\end{pgfscope}%
\begin{pgfscope}%
\pgfsys@transformshift{4.479496in}{3.974584in}%
\pgfsys@useobject{currentmarker}{}%
\end{pgfscope}%
\begin{pgfscope}%
\pgfsys@transformshift{4.798910in}{4.265153in}%
\pgfsys@useobject{currentmarker}{}%
\end{pgfscope}%
\begin{pgfscope}%
\pgfsys@transformshift{5.118323in}{4.426849in}%
\pgfsys@useobject{currentmarker}{}%
\end{pgfscope}%
\end{pgfscope}%
\begin{pgfscope}%
\pgfsetrectcap%
\pgfsetmiterjoin%
\pgfsetlinewidth{0.803000pt}%
\definecolor{currentstroke}{rgb}{0.000000,0.000000,0.000000}%
\pgfsetstrokecolor{currentstroke}%
\pgfsetdash{}{0pt}%
\pgfpathmoveto{\pgfqpoint{0.550713in}{3.620038in}}%
\pgfpathlineto{\pgfqpoint{0.550713in}{4.952225in}}%
\pgfusepath{stroke}%
\end{pgfscope}%
\begin{pgfscope}%
\pgfsetrectcap%
\pgfsetmiterjoin%
\pgfsetlinewidth{0.803000pt}%
\definecolor{currentstroke}{rgb}{0.000000,0.000000,0.000000}%
\pgfsetstrokecolor{currentstroke}%
\pgfsetdash{}{0pt}%
\pgfpathmoveto{\pgfqpoint{5.341912in}{3.620038in}}%
\pgfpathlineto{\pgfqpoint{5.341912in}{4.952225in}}%
\pgfusepath{stroke}%
\end{pgfscope}%
\begin{pgfscope}%
\pgfsetrectcap%
\pgfsetmiterjoin%
\pgfsetlinewidth{0.803000pt}%
\definecolor{currentstroke}{rgb}{0.000000,0.000000,0.000000}%
\pgfsetstrokecolor{currentstroke}%
\pgfsetdash{}{0pt}%
\pgfpathmoveto{\pgfqpoint{0.550713in}{3.620038in}}%
\pgfpathlineto{\pgfqpoint{5.341912in}{3.620038in}}%
\pgfusepath{stroke}%
\end{pgfscope}%
\begin{pgfscope}%
\pgfsetrectcap%
\pgfsetmiterjoin%
\pgfsetlinewidth{0.803000pt}%
\definecolor{currentstroke}{rgb}{0.000000,0.000000,0.000000}%
\pgfsetstrokecolor{currentstroke}%
\pgfsetdash{}{0pt}%
\pgfpathmoveto{\pgfqpoint{0.550713in}{4.952225in}}%
\pgfpathlineto{\pgfqpoint{5.341912in}{4.952225in}}%
\pgfusepath{stroke}%
\end{pgfscope}%
\begin{pgfscope}%
\pgfsetbuttcap%
\pgfsetmiterjoin%
\definecolor{currentfill}{rgb}{1.000000,1.000000,1.000000}%
\pgfsetfillcolor{currentfill}%
\pgfsetlinewidth{1.003750pt}%
\definecolor{currentstroke}{rgb}{1.000000,1.000000,1.000000}%
\pgfsetstrokecolor{currentstroke}%
\pgfsetdash{}{0pt}%
\pgfpathmoveto{\pgfqpoint{0.654864in}{4.666882in}}%
\pgfpathlineto{\pgfqpoint{1.665419in}{4.666882in}}%
\pgfpathlineto{\pgfqpoint{1.665419in}{4.904521in}}%
\pgfpathlineto{\pgfqpoint{0.654864in}{4.904521in}}%
\pgfpathclose%
\pgfusepath{stroke,fill}%
\end{pgfscope}%
\begin{pgfscope}%
\definecolor{textcolor}{rgb}{0.000000,0.000000,0.000000}%
\pgfsetstrokecolor{textcolor}%
\pgfsetfillcolor{textcolor}%
\pgftext[x=0.710419in,y=4.785702in,left,]{\color{textcolor}\rmfamily\fontsize{10.000000}{12.000000}\selectfont B2P forgetting}%
\end{pgfscope}%
\begin{pgfscope}%
\pgfsetbuttcap%
\pgfsetmiterjoin%
\definecolor{currentfill}{rgb}{1.000000,1.000000,1.000000}%
\pgfsetfillcolor{currentfill}%
\pgfsetlinewidth{1.003750pt}%
\definecolor{currentstroke}{rgb}{1.000000,1.000000,1.000000}%
\pgfsetstrokecolor{currentstroke}%
\pgfsetdash{}{0pt}%
\pgfpathmoveto{\pgfqpoint{2.551382in}{4.261114in}}%
\pgfpathlineto{\pgfqpoint{3.341243in}{4.261114in}}%
\pgfpathquadraticcurveto{\pgfqpoint{3.369021in}{4.261114in}}{\pgfqpoint{3.369021in}{4.288892in}}%
\pgfpathlineto{\pgfqpoint{3.369021in}{4.855003in}}%
\pgfpathquadraticcurveto{\pgfqpoint{3.369021in}{4.882781in}}{\pgfqpoint{3.341243in}{4.882781in}}%
\pgfpathlineto{\pgfqpoint{2.551382in}{4.882781in}}%
\pgfpathquadraticcurveto{\pgfqpoint{2.523604in}{4.882781in}}{\pgfqpoint{2.523604in}{4.855003in}}%
\pgfpathlineto{\pgfqpoint{2.523604in}{4.288892in}}%
\pgfpathquadraticcurveto{\pgfqpoint{2.523604in}{4.261114in}}{\pgfqpoint{2.551382in}{4.261114in}}%
\pgfpathclose%
\pgfusepath{stroke,fill}%
\end{pgfscope}%
\begin{pgfscope}%
\definecolor{textcolor}{rgb}{0.000000,0.000000,0.000000}%
\pgfsetstrokecolor{textcolor}%
\pgfsetfillcolor{textcolor}%
\pgftext[x=2.579160in,y=4.730836in,left,base]{\color{textcolor}\rmfamily\fontsize{10.000000}{12.000000}\selectfont Constrained}%
\end{pgfscope}%
\begin{pgfscope}%
\pgfsetbuttcap%
\pgfsetmiterjoin%
\definecolor{currentfill}{rgb}{0.560294,0.133824,0.242647}%
\pgfsetfillcolor{currentfill}%
\pgfsetlinewidth{0.376406pt}%
\definecolor{currentstroke}{rgb}{0.152941,0.152941,0.152941}%
\pgfsetstrokecolor{currentstroke}%
\pgfsetdash{}{0pt}%
\pgfpathmoveto{\pgfqpoint{2.600062in}{4.537225in}}%
\pgfpathlineto{\pgfqpoint{2.877840in}{4.537225in}}%
\pgfpathlineto{\pgfqpoint{2.877840in}{4.634447in}}%
\pgfpathlineto{\pgfqpoint{2.600062in}{4.634447in}}%
\pgfpathclose%
\pgfusepath{stroke,fill}%
\end{pgfscope}%
\begin{pgfscope}%
\definecolor{textcolor}{rgb}{0.000000,0.000000,0.000000}%
\pgfsetstrokecolor{textcolor}%
\pgfsetfillcolor{textcolor}%
\pgftext[x=2.988951in,y=4.537225in,left,base]{\color{textcolor}\rmfamily\fontsize{10.000000}{12.000000}\selectfont False}%
\end{pgfscope}%
\begin{pgfscope}%
\pgfsetbuttcap%
\pgfsetmiterjoin%
\definecolor{currentfill}{rgb}{0.349020,0.160784,0.322549}%
\pgfsetfillcolor{currentfill}%
\pgfsetlinewidth{0.376406pt}%
\definecolor{currentstroke}{rgb}{0.152941,0.152941,0.152941}%
\pgfsetstrokecolor{currentstroke}%
\pgfsetdash{}{0pt}%
\pgfpathmoveto{\pgfqpoint{2.600062in}{4.343614in}}%
\pgfpathlineto{\pgfqpoint{2.877840in}{4.343614in}}%
\pgfpathlineto{\pgfqpoint{2.877840in}{4.440836in}}%
\pgfpathlineto{\pgfqpoint{2.600062in}{4.440836in}}%
\pgfpathclose%
\pgfusepath{stroke,fill}%
\end{pgfscope}%
\begin{pgfscope}%
\definecolor{textcolor}{rgb}{0.000000,0.000000,0.000000}%
\pgfsetstrokecolor{textcolor}%
\pgfsetfillcolor{textcolor}%
\pgftext[x=2.988951in,y=4.343614in,left,base]{\color{textcolor}\rmfamily\fontsize{10.000000}{12.000000}\selectfont True}%
\end{pgfscope}%
\begin{pgfscope}%
\pgfsetbuttcap%
\pgfsetmiterjoin%
\definecolor{currentfill}{rgb}{1.000000,1.000000,1.000000}%
\pgfsetfillcolor{currentfill}%
\pgfsetlinewidth{0.000000pt}%
\definecolor{currentstroke}{rgb}{0.000000,0.000000,0.000000}%
\pgfsetstrokecolor{currentstroke}%
\pgfsetstrokeopacity{0.000000}%
\pgfsetdash{}{0pt}%
\pgfpathmoveto{\pgfqpoint{0.550713in}{2.154633in}}%
\pgfpathlineto{\pgfqpoint{5.341912in}{2.154633in}}%
\pgfpathlineto{\pgfqpoint{5.341912in}{3.486820in}}%
\pgfpathlineto{\pgfqpoint{0.550713in}{3.486820in}}%
\pgfpathclose%
\pgfusepath{fill}%
\end{pgfscope}%
\begin{pgfscope}%
\pgfpathrectangle{\pgfqpoint{0.550713in}{2.154633in}}{\pgfqpoint{4.791200in}{1.332187in}}%
\pgfusepath{clip}%
\pgfsetbuttcap%
\pgfsetroundjoin%
\definecolor{currentfill}{rgb}{0.862668,0.899485,0.932211}%
\pgfsetfillcolor{currentfill}%
\pgfsetlinewidth{0.752812pt}%
\definecolor{currentstroke}{rgb}{0.188235,0.188235,0.188235}%
\pgfsetstrokecolor{currentstroke}%
\pgfsetdash{}{0pt}%
\pgfpathmoveto{\pgfqpoint{0.630885in}{2.359506in}}%
\pgfpathlineto{\pgfqpoint{0.662188in}{2.359506in}}%
\pgfpathlineto{\pgfqpoint{0.662188in}{2.676663in}}%
\pgfpathlineto{\pgfqpoint{0.630885in}{2.676663in}}%
\pgfpathclose%
\pgfusepath{stroke,fill}%
\end{pgfscope}%
\begin{pgfscope}%
\pgfpathrectangle{\pgfqpoint{0.550713in}{2.154633in}}{\pgfqpoint{4.791200in}{1.332187in}}%
\pgfusepath{clip}%
\pgfsetbuttcap%
\pgfsetroundjoin%
\definecolor{currentfill}{rgb}{0.472088,0.613613,0.739413}%
\pgfsetfillcolor{currentfill}%
\pgfsetlinewidth{0.752812pt}%
\definecolor{currentstroke}{rgb}{0.188235,0.188235,0.188235}%
\pgfsetstrokecolor{currentstroke}%
\pgfsetdash{}{0pt}%
\pgfpathmoveto{\pgfqpoint{0.615234in}{2.404090in}}%
\pgfpathlineto{\pgfqpoint{0.677839in}{2.404090in}}%
\pgfpathlineto{\pgfqpoint{0.677839in}{2.642598in}}%
\pgfpathlineto{\pgfqpoint{0.615234in}{2.642598in}}%
\pgfpathclose%
\pgfusepath{stroke,fill}%
\end{pgfscope}%
\begin{pgfscope}%
\pgfpathrectangle{\pgfqpoint{0.550713in}{2.154633in}}{\pgfqpoint{4.791200in}{1.332187in}}%
\pgfusepath{clip}%
\pgfsetbuttcap%
\pgfsetroundjoin%
\definecolor{currentfill}{rgb}{0.078431,0.325490,0.545098}%
\pgfsetfillcolor{currentfill}%
\pgfsetlinewidth{0.752812pt}%
\definecolor{currentstroke}{rgb}{0.188235,0.188235,0.188235}%
\pgfsetstrokecolor{currentstroke}%
\pgfsetdash{}{0pt}%
\pgfpathmoveto{\pgfqpoint{0.583932in}{2.493259in}}%
\pgfpathlineto{\pgfqpoint{0.709142in}{2.493259in}}%
\pgfpathlineto{\pgfqpoint{0.709142in}{2.574469in}}%
\pgfpathlineto{\pgfqpoint{0.583932in}{2.574469in}}%
\pgfpathclose%
\pgfusepath{stroke,fill}%
\end{pgfscope}%
\begin{pgfscope}%
\pgfpathrectangle{\pgfqpoint{0.550713in}{2.154633in}}{\pgfqpoint{4.791200in}{1.332187in}}%
\pgfusepath{clip}%
\pgfsetbuttcap%
\pgfsetroundjoin%
\definecolor{currentfill}{rgb}{0.904744,0.941561,0.883122}%
\pgfsetfillcolor{currentfill}%
\pgfsetlinewidth{0.752812pt}%
\definecolor{currentstroke}{rgb}{0.188235,0.188235,0.188235}%
\pgfsetstrokecolor{currentstroke}%
\pgfsetdash{}{0pt}%
\pgfpathmoveto{\pgfqpoint{0.758651in}{2.252199in}}%
\pgfpathlineto{\pgfqpoint{0.789953in}{2.252199in}}%
\pgfpathlineto{\pgfqpoint{0.789953in}{2.337563in}}%
\pgfpathlineto{\pgfqpoint{0.758651in}{2.337563in}}%
\pgfpathclose%
\pgfusepath{stroke,fill}%
\end{pgfscope}%
\begin{pgfscope}%
\pgfpathrectangle{\pgfqpoint{0.550713in}{2.154633in}}{\pgfqpoint{4.791200in}{1.332187in}}%
\pgfusepath{clip}%
\pgfsetbuttcap%
\pgfsetroundjoin%
\definecolor{currentfill}{rgb}{0.633831,0.775356,0.550713}%
\pgfsetfillcolor{currentfill}%
\pgfsetlinewidth{0.752812pt}%
\definecolor{currentstroke}{rgb}{0.188235,0.188235,0.188235}%
\pgfsetstrokecolor{currentstroke}%
\pgfsetdash{}{0pt}%
\pgfpathmoveto{\pgfqpoint{0.742999in}{2.254919in}}%
\pgfpathlineto{\pgfqpoint{0.805604in}{2.254919in}}%
\pgfpathlineto{\pgfqpoint{0.805604in}{2.327833in}}%
\pgfpathlineto{\pgfqpoint{0.742999in}{2.327833in}}%
\pgfpathclose%
\pgfusepath{stroke,fill}%
\end{pgfscope}%
\begin{pgfscope}%
\pgfpathrectangle{\pgfqpoint{0.550713in}{2.154633in}}{\pgfqpoint{4.791200in}{1.332187in}}%
\pgfusepath{clip}%
\pgfsetbuttcap%
\pgfsetroundjoin%
\definecolor{currentfill}{rgb}{0.360784,0.607843,0.215686}%
\pgfsetfillcolor{currentfill}%
\pgfsetlinewidth{0.752812pt}%
\definecolor{currentstroke}{rgb}{0.188235,0.188235,0.188235}%
\pgfsetstrokecolor{currentstroke}%
\pgfsetdash{}{0pt}%
\pgfpathmoveto{\pgfqpoint{0.711697in}{2.260361in}}%
\pgfpathlineto{\pgfqpoint{0.836907in}{2.260361in}}%
\pgfpathlineto{\pgfqpoint{0.836907in}{2.308374in}}%
\pgfpathlineto{\pgfqpoint{0.711697in}{2.308374in}}%
\pgfpathclose%
\pgfusepath{stroke,fill}%
\end{pgfscope}%
\begin{pgfscope}%
\pgfpathrectangle{\pgfqpoint{0.550713in}{2.154633in}}{\pgfqpoint{4.791200in}{1.332187in}}%
\pgfusepath{clip}%
\pgfsetbuttcap%
\pgfsetroundjoin%
\definecolor{currentfill}{rgb}{0.862668,0.899485,0.932211}%
\pgfsetfillcolor{currentfill}%
\pgfsetlinewidth{0.752812pt}%
\definecolor{currentstroke}{rgb}{0.188235,0.188235,0.188235}%
\pgfsetstrokecolor{currentstroke}%
\pgfsetdash{}{0pt}%
\pgfpathmoveto{\pgfqpoint{0.950299in}{2.287567in}}%
\pgfpathlineto{\pgfqpoint{0.981601in}{2.287567in}}%
\pgfpathlineto{\pgfqpoint{0.981601in}{2.539041in}}%
\pgfpathlineto{\pgfqpoint{0.950299in}{2.539041in}}%
\pgfpathclose%
\pgfusepath{stroke,fill}%
\end{pgfscope}%
\begin{pgfscope}%
\pgfpathrectangle{\pgfqpoint{0.550713in}{2.154633in}}{\pgfqpoint{4.791200in}{1.332187in}}%
\pgfusepath{clip}%
\pgfsetbuttcap%
\pgfsetroundjoin%
\definecolor{currentfill}{rgb}{0.472088,0.613613,0.739413}%
\pgfsetfillcolor{currentfill}%
\pgfsetlinewidth{0.752812pt}%
\definecolor{currentstroke}{rgb}{0.188235,0.188235,0.188235}%
\pgfsetstrokecolor{currentstroke}%
\pgfsetdash{}{0pt}%
\pgfpathmoveto{\pgfqpoint{0.934647in}{2.295056in}}%
\pgfpathlineto{\pgfqpoint{0.997252in}{2.295056in}}%
\pgfpathlineto{\pgfqpoint{0.997252in}{2.517886in}}%
\pgfpathlineto{\pgfqpoint{0.934647in}{2.517886in}}%
\pgfpathclose%
\pgfusepath{stroke,fill}%
\end{pgfscope}%
\begin{pgfscope}%
\pgfpathrectangle{\pgfqpoint{0.550713in}{2.154633in}}{\pgfqpoint{4.791200in}{1.332187in}}%
\pgfusepath{clip}%
\pgfsetbuttcap%
\pgfsetroundjoin%
\definecolor{currentfill}{rgb}{0.078431,0.325490,0.545098}%
\pgfsetfillcolor{currentfill}%
\pgfsetlinewidth{0.752812pt}%
\definecolor{currentstroke}{rgb}{0.188235,0.188235,0.188235}%
\pgfsetstrokecolor{currentstroke}%
\pgfsetdash{}{0pt}%
\pgfpathmoveto{\pgfqpoint{0.903345in}{2.310035in}}%
\pgfpathlineto{\pgfqpoint{1.028555in}{2.310035in}}%
\pgfpathlineto{\pgfqpoint{1.028555in}{2.475576in}}%
\pgfpathlineto{\pgfqpoint{0.903345in}{2.475576in}}%
\pgfpathclose%
\pgfusepath{stroke,fill}%
\end{pgfscope}%
\begin{pgfscope}%
\pgfpathrectangle{\pgfqpoint{0.550713in}{2.154633in}}{\pgfqpoint{4.791200in}{1.332187in}}%
\pgfusepath{clip}%
\pgfsetbuttcap%
\pgfsetroundjoin%
\definecolor{currentfill}{rgb}{0.904744,0.941561,0.883122}%
\pgfsetfillcolor{currentfill}%
\pgfsetlinewidth{0.752812pt}%
\definecolor{currentstroke}{rgb}{0.188235,0.188235,0.188235}%
\pgfsetstrokecolor{currentstroke}%
\pgfsetdash{}{0pt}%
\pgfpathmoveto{\pgfqpoint{1.078064in}{2.226934in}}%
\pgfpathlineto{\pgfqpoint{1.109366in}{2.226934in}}%
\pgfpathlineto{\pgfqpoint{1.109366in}{2.265003in}}%
\pgfpathlineto{\pgfqpoint{1.078064in}{2.265003in}}%
\pgfpathclose%
\pgfusepath{stroke,fill}%
\end{pgfscope}%
\begin{pgfscope}%
\pgfpathrectangle{\pgfqpoint{0.550713in}{2.154633in}}{\pgfqpoint{4.791200in}{1.332187in}}%
\pgfusepath{clip}%
\pgfsetbuttcap%
\pgfsetroundjoin%
\definecolor{currentfill}{rgb}{0.633831,0.775356,0.550713}%
\pgfsetfillcolor{currentfill}%
\pgfsetlinewidth{0.752812pt}%
\definecolor{currentstroke}{rgb}{0.188235,0.188235,0.188235}%
\pgfsetstrokecolor{currentstroke}%
\pgfsetdash{}{0pt}%
\pgfpathmoveto{\pgfqpoint{1.062413in}{2.231371in}}%
\pgfpathlineto{\pgfqpoint{1.125018in}{2.231371in}}%
\pgfpathlineto{\pgfqpoint{1.125018in}{2.259797in}}%
\pgfpathlineto{\pgfqpoint{1.062413in}{2.259797in}}%
\pgfpathclose%
\pgfusepath{stroke,fill}%
\end{pgfscope}%
\begin{pgfscope}%
\pgfpathrectangle{\pgfqpoint{0.550713in}{2.154633in}}{\pgfqpoint{4.791200in}{1.332187in}}%
\pgfusepath{clip}%
\pgfsetbuttcap%
\pgfsetroundjoin%
\definecolor{currentfill}{rgb}{0.360784,0.607843,0.215686}%
\pgfsetfillcolor{currentfill}%
\pgfsetlinewidth{0.752812pt}%
\definecolor{currentstroke}{rgb}{0.188235,0.188235,0.188235}%
\pgfsetstrokecolor{currentstroke}%
\pgfsetdash{}{0pt}%
\pgfpathmoveto{\pgfqpoint{1.031110in}{2.240244in}}%
\pgfpathlineto{\pgfqpoint{1.156320in}{2.240244in}}%
\pgfpathlineto{\pgfqpoint{1.156320in}{2.249387in}}%
\pgfpathlineto{\pgfqpoint{1.031110in}{2.249387in}}%
\pgfpathclose%
\pgfusepath{stroke,fill}%
\end{pgfscope}%
\begin{pgfscope}%
\pgfpathrectangle{\pgfqpoint{0.550713in}{2.154633in}}{\pgfqpoint{4.791200in}{1.332187in}}%
\pgfusepath{clip}%
\pgfsetbuttcap%
\pgfsetroundjoin%
\definecolor{currentfill}{rgb}{0.862668,0.899485,0.932211}%
\pgfsetfillcolor{currentfill}%
\pgfsetlinewidth{0.752812pt}%
\definecolor{currentstroke}{rgb}{0.188235,0.188235,0.188235}%
\pgfsetstrokecolor{currentstroke}%
\pgfsetdash{}{0pt}%
\pgfpathmoveto{\pgfqpoint{1.269712in}{2.326149in}}%
\pgfpathlineto{\pgfqpoint{1.301014in}{2.326149in}}%
\pgfpathlineto{\pgfqpoint{1.301014in}{2.505432in}}%
\pgfpathlineto{\pgfqpoint{1.269712in}{2.505432in}}%
\pgfpathclose%
\pgfusepath{stroke,fill}%
\end{pgfscope}%
\begin{pgfscope}%
\pgfpathrectangle{\pgfqpoint{0.550713in}{2.154633in}}{\pgfqpoint{4.791200in}{1.332187in}}%
\pgfusepath{clip}%
\pgfsetbuttcap%
\pgfsetroundjoin%
\definecolor{currentfill}{rgb}{0.472088,0.613613,0.739413}%
\pgfsetfillcolor{currentfill}%
\pgfsetlinewidth{0.752812pt}%
\definecolor{currentstroke}{rgb}{0.188235,0.188235,0.188235}%
\pgfsetstrokecolor{currentstroke}%
\pgfsetdash{}{0pt}%
\pgfpathmoveto{\pgfqpoint{1.254061in}{2.327359in}}%
\pgfpathlineto{\pgfqpoint{1.316666in}{2.327359in}}%
\pgfpathlineto{\pgfqpoint{1.316666in}{2.504486in}}%
\pgfpathlineto{\pgfqpoint{1.254061in}{2.504486in}}%
\pgfpathclose%
\pgfusepath{stroke,fill}%
\end{pgfscope}%
\begin{pgfscope}%
\pgfpathrectangle{\pgfqpoint{0.550713in}{2.154633in}}{\pgfqpoint{4.791200in}{1.332187in}}%
\pgfusepath{clip}%
\pgfsetbuttcap%
\pgfsetroundjoin%
\definecolor{currentfill}{rgb}{0.078431,0.325490,0.545098}%
\pgfsetfillcolor{currentfill}%
\pgfsetlinewidth{0.752812pt}%
\definecolor{currentstroke}{rgb}{0.188235,0.188235,0.188235}%
\pgfsetstrokecolor{currentstroke}%
\pgfsetdash{}{0pt}%
\pgfpathmoveto{\pgfqpoint{1.222758in}{2.329779in}}%
\pgfpathlineto{\pgfqpoint{1.347968in}{2.329779in}}%
\pgfpathlineto{\pgfqpoint{1.347968in}{2.502596in}}%
\pgfpathlineto{\pgfqpoint{1.222758in}{2.502596in}}%
\pgfpathclose%
\pgfusepath{stroke,fill}%
\end{pgfscope}%
\begin{pgfscope}%
\pgfpathrectangle{\pgfqpoint{0.550713in}{2.154633in}}{\pgfqpoint{4.791200in}{1.332187in}}%
\pgfusepath{clip}%
\pgfsetbuttcap%
\pgfsetroundjoin%
\definecolor{currentfill}{rgb}{0.904744,0.941561,0.883122}%
\pgfsetfillcolor{currentfill}%
\pgfsetlinewidth{0.752812pt}%
\definecolor{currentstroke}{rgb}{0.188235,0.188235,0.188235}%
\pgfsetstrokecolor{currentstroke}%
\pgfsetdash{}{0pt}%
\pgfpathmoveto{\pgfqpoint{1.397477in}{2.225822in}}%
\pgfpathlineto{\pgfqpoint{1.428780in}{2.225822in}}%
\pgfpathlineto{\pgfqpoint{1.428780in}{2.247356in}}%
\pgfpathlineto{\pgfqpoint{1.397477in}{2.247356in}}%
\pgfpathclose%
\pgfusepath{stroke,fill}%
\end{pgfscope}%
\begin{pgfscope}%
\pgfpathrectangle{\pgfqpoint{0.550713in}{2.154633in}}{\pgfqpoint{4.791200in}{1.332187in}}%
\pgfusepath{clip}%
\pgfsetbuttcap%
\pgfsetroundjoin%
\definecolor{currentfill}{rgb}{0.633831,0.775356,0.550713}%
\pgfsetfillcolor{currentfill}%
\pgfsetlinewidth{0.752812pt}%
\definecolor{currentstroke}{rgb}{0.188235,0.188235,0.188235}%
\pgfsetstrokecolor{currentstroke}%
\pgfsetdash{}{0pt}%
\pgfpathmoveto{\pgfqpoint{1.381826in}{2.226446in}}%
\pgfpathlineto{\pgfqpoint{1.444431in}{2.226446in}}%
\pgfpathlineto{\pgfqpoint{1.444431in}{2.247110in}}%
\pgfpathlineto{\pgfqpoint{1.381826in}{2.247110in}}%
\pgfpathclose%
\pgfusepath{stroke,fill}%
\end{pgfscope}%
\begin{pgfscope}%
\pgfpathrectangle{\pgfqpoint{0.550713in}{2.154633in}}{\pgfqpoint{4.791200in}{1.332187in}}%
\pgfusepath{clip}%
\pgfsetbuttcap%
\pgfsetroundjoin%
\definecolor{currentfill}{rgb}{0.360784,0.607843,0.215686}%
\pgfsetfillcolor{currentfill}%
\pgfsetlinewidth{0.752812pt}%
\definecolor{currentstroke}{rgb}{0.188235,0.188235,0.188235}%
\pgfsetstrokecolor{currentstroke}%
\pgfsetdash{}{0pt}%
\pgfpathmoveto{\pgfqpoint{1.350524in}{2.227696in}}%
\pgfpathlineto{\pgfqpoint{1.475734in}{2.227696in}}%
\pgfpathlineto{\pgfqpoint{1.475734in}{2.246618in}}%
\pgfpathlineto{\pgfqpoint{1.350524in}{2.246618in}}%
\pgfpathclose%
\pgfusepath{stroke,fill}%
\end{pgfscope}%
\begin{pgfscope}%
\pgfpathrectangle{\pgfqpoint{0.550713in}{2.154633in}}{\pgfqpoint{4.791200in}{1.332187in}}%
\pgfusepath{clip}%
\pgfsetbuttcap%
\pgfsetroundjoin%
\definecolor{currentfill}{rgb}{0.862668,0.899485,0.932211}%
\pgfsetfillcolor{currentfill}%
\pgfsetlinewidth{0.752812pt}%
\definecolor{currentstroke}{rgb}{0.188235,0.188235,0.188235}%
\pgfsetstrokecolor{currentstroke}%
\pgfsetdash{}{0pt}%
\pgfpathmoveto{\pgfqpoint{1.589125in}{2.279341in}}%
\pgfpathlineto{\pgfqpoint{1.620428in}{2.279341in}}%
\pgfpathlineto{\pgfqpoint{1.620428in}{2.346904in}}%
\pgfpathlineto{\pgfqpoint{1.589125in}{2.346904in}}%
\pgfpathclose%
\pgfusepath{stroke,fill}%
\end{pgfscope}%
\begin{pgfscope}%
\pgfpathrectangle{\pgfqpoint{0.550713in}{2.154633in}}{\pgfqpoint{4.791200in}{1.332187in}}%
\pgfusepath{clip}%
\pgfsetbuttcap%
\pgfsetroundjoin%
\definecolor{currentfill}{rgb}{0.472088,0.613613,0.739413}%
\pgfsetfillcolor{currentfill}%
\pgfsetlinewidth{0.752812pt}%
\definecolor{currentstroke}{rgb}{0.188235,0.188235,0.188235}%
\pgfsetstrokecolor{currentstroke}%
\pgfsetdash{}{0pt}%
\pgfpathmoveto{\pgfqpoint{1.573474in}{2.281372in}}%
\pgfpathlineto{\pgfqpoint{1.636079in}{2.281372in}}%
\pgfpathlineto{\pgfqpoint{1.636079in}{2.339984in}}%
\pgfpathlineto{\pgfqpoint{1.573474in}{2.339984in}}%
\pgfpathclose%
\pgfusepath{stroke,fill}%
\end{pgfscope}%
\begin{pgfscope}%
\pgfpathrectangle{\pgfqpoint{0.550713in}{2.154633in}}{\pgfqpoint{4.791200in}{1.332187in}}%
\pgfusepath{clip}%
\pgfsetbuttcap%
\pgfsetroundjoin%
\definecolor{currentfill}{rgb}{0.078431,0.325490,0.545098}%
\pgfsetfillcolor{currentfill}%
\pgfsetlinewidth{0.752812pt}%
\definecolor{currentstroke}{rgb}{0.188235,0.188235,0.188235}%
\pgfsetstrokecolor{currentstroke}%
\pgfsetdash{}{0pt}%
\pgfpathmoveto{\pgfqpoint{1.542172in}{2.285433in}}%
\pgfpathlineto{\pgfqpoint{1.667382in}{2.285433in}}%
\pgfpathlineto{\pgfqpoint{1.667382in}{2.326143in}}%
\pgfpathlineto{\pgfqpoint{1.542172in}{2.326143in}}%
\pgfpathclose%
\pgfusepath{stroke,fill}%
\end{pgfscope}%
\begin{pgfscope}%
\pgfpathrectangle{\pgfqpoint{0.550713in}{2.154633in}}{\pgfqpoint{4.791200in}{1.332187in}}%
\pgfusepath{clip}%
\pgfsetbuttcap%
\pgfsetroundjoin%
\definecolor{currentfill}{rgb}{0.904744,0.941561,0.883122}%
\pgfsetfillcolor{currentfill}%
\pgfsetlinewidth{0.752812pt}%
\definecolor{currentstroke}{rgb}{0.188235,0.188235,0.188235}%
\pgfsetstrokecolor{currentstroke}%
\pgfsetdash{}{0pt}%
\pgfpathmoveto{\pgfqpoint{1.716891in}{2.230444in}}%
\pgfpathlineto{\pgfqpoint{1.748193in}{2.230444in}}%
\pgfpathlineto{\pgfqpoint{1.748193in}{2.261558in}}%
\pgfpathlineto{\pgfqpoint{1.716891in}{2.261558in}}%
\pgfpathclose%
\pgfusepath{stroke,fill}%
\end{pgfscope}%
\begin{pgfscope}%
\pgfpathrectangle{\pgfqpoint{0.550713in}{2.154633in}}{\pgfqpoint{4.791200in}{1.332187in}}%
\pgfusepath{clip}%
\pgfsetbuttcap%
\pgfsetroundjoin%
\definecolor{currentfill}{rgb}{0.633831,0.775356,0.550713}%
\pgfsetfillcolor{currentfill}%
\pgfsetlinewidth{0.752812pt}%
\definecolor{currentstroke}{rgb}{0.188235,0.188235,0.188235}%
\pgfsetstrokecolor{currentstroke}%
\pgfsetdash{}{0pt}%
\pgfpathmoveto{\pgfqpoint{1.701239in}{2.232244in}}%
\pgfpathlineto{\pgfqpoint{1.763844in}{2.232244in}}%
\pgfpathlineto{\pgfqpoint{1.763844in}{2.260032in}}%
\pgfpathlineto{\pgfqpoint{1.701239in}{2.260032in}}%
\pgfpathclose%
\pgfusepath{stroke,fill}%
\end{pgfscope}%
\begin{pgfscope}%
\pgfpathrectangle{\pgfqpoint{0.550713in}{2.154633in}}{\pgfqpoint{4.791200in}{1.332187in}}%
\pgfusepath{clip}%
\pgfsetbuttcap%
\pgfsetroundjoin%
\definecolor{currentfill}{rgb}{0.360784,0.607843,0.215686}%
\pgfsetfillcolor{currentfill}%
\pgfsetlinewidth{0.752812pt}%
\definecolor{currentstroke}{rgb}{0.188235,0.188235,0.188235}%
\pgfsetstrokecolor{currentstroke}%
\pgfsetdash{}{0pt}%
\pgfpathmoveto{\pgfqpoint{1.669937in}{2.235844in}}%
\pgfpathlineto{\pgfqpoint{1.795147in}{2.235844in}}%
\pgfpathlineto{\pgfqpoint{1.795147in}{2.256981in}}%
\pgfpathlineto{\pgfqpoint{1.669937in}{2.256981in}}%
\pgfpathclose%
\pgfusepath{stroke,fill}%
\end{pgfscope}%
\begin{pgfscope}%
\pgfpathrectangle{\pgfqpoint{0.550713in}{2.154633in}}{\pgfqpoint{4.791200in}{1.332187in}}%
\pgfusepath{clip}%
\pgfsetbuttcap%
\pgfsetroundjoin%
\definecolor{currentfill}{rgb}{0.862668,0.899485,0.932211}%
\pgfsetfillcolor{currentfill}%
\pgfsetlinewidth{0.752812pt}%
\definecolor{currentstroke}{rgb}{0.188235,0.188235,0.188235}%
\pgfsetstrokecolor{currentstroke}%
\pgfsetdash{}{0pt}%
\pgfpathmoveto{\pgfqpoint{1.908539in}{2.284976in}}%
\pgfpathlineto{\pgfqpoint{1.939841in}{2.284976in}}%
\pgfpathlineto{\pgfqpoint{1.939841in}{2.399521in}}%
\pgfpathlineto{\pgfqpoint{1.908539in}{2.399521in}}%
\pgfpathclose%
\pgfusepath{stroke,fill}%
\end{pgfscope}%
\begin{pgfscope}%
\pgfpathrectangle{\pgfqpoint{0.550713in}{2.154633in}}{\pgfqpoint{4.791200in}{1.332187in}}%
\pgfusepath{clip}%
\pgfsetbuttcap%
\pgfsetroundjoin%
\definecolor{currentfill}{rgb}{0.472088,0.613613,0.739413}%
\pgfsetfillcolor{currentfill}%
\pgfsetlinewidth{0.752812pt}%
\definecolor{currentstroke}{rgb}{0.188235,0.188235,0.188235}%
\pgfsetstrokecolor{currentstroke}%
\pgfsetdash{}{0pt}%
\pgfpathmoveto{\pgfqpoint{1.892887in}{2.285634in}}%
\pgfpathlineto{\pgfqpoint{1.955492in}{2.285634in}}%
\pgfpathlineto{\pgfqpoint{1.955492in}{2.378892in}}%
\pgfpathlineto{\pgfqpoint{1.892887in}{2.378892in}}%
\pgfpathclose%
\pgfusepath{stroke,fill}%
\end{pgfscope}%
\begin{pgfscope}%
\pgfpathrectangle{\pgfqpoint{0.550713in}{2.154633in}}{\pgfqpoint{4.791200in}{1.332187in}}%
\pgfusepath{clip}%
\pgfsetbuttcap%
\pgfsetroundjoin%
\definecolor{currentfill}{rgb}{0.078431,0.325490,0.545098}%
\pgfsetfillcolor{currentfill}%
\pgfsetlinewidth{0.752812pt}%
\definecolor{currentstroke}{rgb}{0.188235,0.188235,0.188235}%
\pgfsetstrokecolor{currentstroke}%
\pgfsetdash{}{0pt}%
\pgfpathmoveto{\pgfqpoint{1.861585in}{2.286949in}}%
\pgfpathlineto{\pgfqpoint{1.986795in}{2.286949in}}%
\pgfpathlineto{\pgfqpoint{1.986795in}{2.337636in}}%
\pgfpathlineto{\pgfqpoint{1.861585in}{2.337636in}}%
\pgfpathclose%
\pgfusepath{stroke,fill}%
\end{pgfscope}%
\begin{pgfscope}%
\pgfpathrectangle{\pgfqpoint{0.550713in}{2.154633in}}{\pgfqpoint{4.791200in}{1.332187in}}%
\pgfusepath{clip}%
\pgfsetbuttcap%
\pgfsetroundjoin%
\definecolor{currentfill}{rgb}{0.904744,0.941561,0.883122}%
\pgfsetfillcolor{currentfill}%
\pgfsetlinewidth{0.752812pt}%
\definecolor{currentstroke}{rgb}{0.188235,0.188235,0.188235}%
\pgfsetstrokecolor{currentstroke}%
\pgfsetdash{}{0pt}%
\pgfpathmoveto{\pgfqpoint{2.036304in}{2.237651in}}%
\pgfpathlineto{\pgfqpoint{2.067606in}{2.237651in}}%
\pgfpathlineto{\pgfqpoint{2.067606in}{2.270875in}}%
\pgfpathlineto{\pgfqpoint{2.036304in}{2.270875in}}%
\pgfpathclose%
\pgfusepath{stroke,fill}%
\end{pgfscope}%
\begin{pgfscope}%
\pgfpathrectangle{\pgfqpoint{0.550713in}{2.154633in}}{\pgfqpoint{4.791200in}{1.332187in}}%
\pgfusepath{clip}%
\pgfsetbuttcap%
\pgfsetroundjoin%
\definecolor{currentfill}{rgb}{0.633831,0.775356,0.550713}%
\pgfsetfillcolor{currentfill}%
\pgfsetlinewidth{0.752812pt}%
\definecolor{currentstroke}{rgb}{0.188235,0.188235,0.188235}%
\pgfsetstrokecolor{currentstroke}%
\pgfsetdash{}{0pt}%
\pgfpathmoveto{\pgfqpoint{2.020653in}{2.237937in}}%
\pgfpathlineto{\pgfqpoint{2.083258in}{2.237937in}}%
\pgfpathlineto{\pgfqpoint{2.083258in}{2.270339in}}%
\pgfpathlineto{\pgfqpoint{2.020653in}{2.270339in}}%
\pgfpathclose%
\pgfusepath{stroke,fill}%
\end{pgfscope}%
\begin{pgfscope}%
\pgfpathrectangle{\pgfqpoint{0.550713in}{2.154633in}}{\pgfqpoint{4.791200in}{1.332187in}}%
\pgfusepath{clip}%
\pgfsetbuttcap%
\pgfsetroundjoin%
\definecolor{currentfill}{rgb}{0.360784,0.607843,0.215686}%
\pgfsetfillcolor{currentfill}%
\pgfsetlinewidth{0.752812pt}%
\definecolor{currentstroke}{rgb}{0.188235,0.188235,0.188235}%
\pgfsetstrokecolor{currentstroke}%
\pgfsetdash{}{0pt}%
\pgfpathmoveto{\pgfqpoint{1.989350in}{2.238509in}}%
\pgfpathlineto{\pgfqpoint{2.114560in}{2.238509in}}%
\pgfpathlineto{\pgfqpoint{2.114560in}{2.269266in}}%
\pgfpathlineto{\pgfqpoint{1.989350in}{2.269266in}}%
\pgfpathclose%
\pgfusepath{stroke,fill}%
\end{pgfscope}%
\begin{pgfscope}%
\pgfpathrectangle{\pgfqpoint{0.550713in}{2.154633in}}{\pgfqpoint{4.791200in}{1.332187in}}%
\pgfusepath{clip}%
\pgfsetbuttcap%
\pgfsetroundjoin%
\definecolor{currentfill}{rgb}{0.862668,0.899485,0.932211}%
\pgfsetfillcolor{currentfill}%
\pgfsetlinewidth{0.752812pt}%
\definecolor{currentstroke}{rgb}{0.188235,0.188235,0.188235}%
\pgfsetstrokecolor{currentstroke}%
\pgfsetdash{}{0pt}%
\pgfpathmoveto{\pgfqpoint{2.227952in}{2.289746in}}%
\pgfpathlineto{\pgfqpoint{2.259254in}{2.289746in}}%
\pgfpathlineto{\pgfqpoint{2.259254in}{2.582106in}}%
\pgfpathlineto{\pgfqpoint{2.227952in}{2.582106in}}%
\pgfpathclose%
\pgfusepath{stroke,fill}%
\end{pgfscope}%
\begin{pgfscope}%
\pgfpathrectangle{\pgfqpoint{0.550713in}{2.154633in}}{\pgfqpoint{4.791200in}{1.332187in}}%
\pgfusepath{clip}%
\pgfsetbuttcap%
\pgfsetroundjoin%
\definecolor{currentfill}{rgb}{0.472088,0.613613,0.739413}%
\pgfsetfillcolor{currentfill}%
\pgfsetlinewidth{0.752812pt}%
\definecolor{currentstroke}{rgb}{0.188235,0.188235,0.188235}%
\pgfsetstrokecolor{currentstroke}%
\pgfsetdash{}{0pt}%
\pgfpathmoveto{\pgfqpoint{2.212301in}{2.290159in}}%
\pgfpathlineto{\pgfqpoint{2.274906in}{2.290159in}}%
\pgfpathlineto{\pgfqpoint{2.274906in}{2.531438in}}%
\pgfpathlineto{\pgfqpoint{2.212301in}{2.531438in}}%
\pgfpathclose%
\pgfusepath{stroke,fill}%
\end{pgfscope}%
\begin{pgfscope}%
\pgfpathrectangle{\pgfqpoint{0.550713in}{2.154633in}}{\pgfqpoint{4.791200in}{1.332187in}}%
\pgfusepath{clip}%
\pgfsetbuttcap%
\pgfsetroundjoin%
\definecolor{currentfill}{rgb}{0.078431,0.325490,0.545098}%
\pgfsetfillcolor{currentfill}%
\pgfsetlinewidth{0.752812pt}%
\definecolor{currentstroke}{rgb}{0.188235,0.188235,0.188235}%
\pgfsetstrokecolor{currentstroke}%
\pgfsetdash{}{0pt}%
\pgfpathmoveto{\pgfqpoint{2.180998in}{2.290984in}}%
\pgfpathlineto{\pgfqpoint{2.306208in}{2.290984in}}%
\pgfpathlineto{\pgfqpoint{2.306208in}{2.430102in}}%
\pgfpathlineto{\pgfqpoint{2.180998in}{2.430102in}}%
\pgfpathclose%
\pgfusepath{stroke,fill}%
\end{pgfscope}%
\begin{pgfscope}%
\pgfpathrectangle{\pgfqpoint{0.550713in}{2.154633in}}{\pgfqpoint{4.791200in}{1.332187in}}%
\pgfusepath{clip}%
\pgfsetbuttcap%
\pgfsetroundjoin%
\definecolor{currentfill}{rgb}{0.904744,0.941561,0.883122}%
\pgfsetfillcolor{currentfill}%
\pgfsetlinewidth{0.752812pt}%
\definecolor{currentstroke}{rgb}{0.188235,0.188235,0.188235}%
\pgfsetstrokecolor{currentstroke}%
\pgfsetdash{}{0pt}%
\pgfpathmoveto{\pgfqpoint{2.355717in}{2.221286in}}%
\pgfpathlineto{\pgfqpoint{2.387020in}{2.221286in}}%
\pgfpathlineto{\pgfqpoint{2.387020in}{2.260897in}}%
\pgfpathlineto{\pgfqpoint{2.355717in}{2.260897in}}%
\pgfpathclose%
\pgfusepath{stroke,fill}%
\end{pgfscope}%
\begin{pgfscope}%
\pgfpathrectangle{\pgfqpoint{0.550713in}{2.154633in}}{\pgfqpoint{4.791200in}{1.332187in}}%
\pgfusepath{clip}%
\pgfsetbuttcap%
\pgfsetroundjoin%
\definecolor{currentfill}{rgb}{0.633831,0.775356,0.550713}%
\pgfsetfillcolor{currentfill}%
\pgfsetlinewidth{0.752812pt}%
\definecolor{currentstroke}{rgb}{0.188235,0.188235,0.188235}%
\pgfsetstrokecolor{currentstroke}%
\pgfsetdash{}{0pt}%
\pgfpathmoveto{\pgfqpoint{2.340066in}{2.228269in}}%
\pgfpathlineto{\pgfqpoint{2.402671in}{2.228269in}}%
\pgfpathlineto{\pgfqpoint{2.402671in}{2.259356in}}%
\pgfpathlineto{\pgfqpoint{2.340066in}{2.259356in}}%
\pgfpathclose%
\pgfusepath{stroke,fill}%
\end{pgfscope}%
\begin{pgfscope}%
\pgfpathrectangle{\pgfqpoint{0.550713in}{2.154633in}}{\pgfqpoint{4.791200in}{1.332187in}}%
\pgfusepath{clip}%
\pgfsetbuttcap%
\pgfsetroundjoin%
\definecolor{currentfill}{rgb}{0.360784,0.607843,0.215686}%
\pgfsetfillcolor{currentfill}%
\pgfsetlinewidth{0.752812pt}%
\definecolor{currentstroke}{rgb}{0.188235,0.188235,0.188235}%
\pgfsetstrokecolor{currentstroke}%
\pgfsetdash{}{0pt}%
\pgfpathmoveto{\pgfqpoint{2.308763in}{2.242234in}}%
\pgfpathlineto{\pgfqpoint{2.433973in}{2.242234in}}%
\pgfpathlineto{\pgfqpoint{2.433973in}{2.256275in}}%
\pgfpathlineto{\pgfqpoint{2.308763in}{2.256275in}}%
\pgfpathclose%
\pgfusepath{stroke,fill}%
\end{pgfscope}%
\begin{pgfscope}%
\pgfpathrectangle{\pgfqpoint{0.550713in}{2.154633in}}{\pgfqpoint{4.791200in}{1.332187in}}%
\pgfusepath{clip}%
\pgfsetbuttcap%
\pgfsetroundjoin%
\definecolor{currentfill}{rgb}{0.862668,0.899485,0.932211}%
\pgfsetfillcolor{currentfill}%
\pgfsetlinewidth{0.752812pt}%
\definecolor{currentstroke}{rgb}{0.188235,0.188235,0.188235}%
\pgfsetstrokecolor{currentstroke}%
\pgfsetdash{}{0pt}%
\pgfpathmoveto{\pgfqpoint{2.547365in}{2.330636in}}%
\pgfpathlineto{\pgfqpoint{2.578668in}{2.330636in}}%
\pgfpathlineto{\pgfqpoint{2.578668in}{2.444992in}}%
\pgfpathlineto{\pgfqpoint{2.547365in}{2.444992in}}%
\pgfpathclose%
\pgfusepath{stroke,fill}%
\end{pgfscope}%
\begin{pgfscope}%
\pgfpathrectangle{\pgfqpoint{0.550713in}{2.154633in}}{\pgfqpoint{4.791200in}{1.332187in}}%
\pgfusepath{clip}%
\pgfsetbuttcap%
\pgfsetroundjoin%
\definecolor{currentfill}{rgb}{0.472088,0.613613,0.739413}%
\pgfsetfillcolor{currentfill}%
\pgfsetlinewidth{0.752812pt}%
\definecolor{currentstroke}{rgb}{0.188235,0.188235,0.188235}%
\pgfsetstrokecolor{currentstroke}%
\pgfsetdash{}{0pt}%
\pgfpathmoveto{\pgfqpoint{2.531714in}{2.343916in}}%
\pgfpathlineto{\pgfqpoint{2.594319in}{2.343916in}}%
\pgfpathlineto{\pgfqpoint{2.594319in}{2.440969in}}%
\pgfpathlineto{\pgfqpoint{2.531714in}{2.440969in}}%
\pgfpathclose%
\pgfusepath{stroke,fill}%
\end{pgfscope}%
\begin{pgfscope}%
\pgfpathrectangle{\pgfqpoint{0.550713in}{2.154633in}}{\pgfqpoint{4.791200in}{1.332187in}}%
\pgfusepath{clip}%
\pgfsetbuttcap%
\pgfsetroundjoin%
\definecolor{currentfill}{rgb}{0.078431,0.325490,0.545098}%
\pgfsetfillcolor{currentfill}%
\pgfsetlinewidth{0.752812pt}%
\definecolor{currentstroke}{rgb}{0.188235,0.188235,0.188235}%
\pgfsetstrokecolor{currentstroke}%
\pgfsetdash{}{0pt}%
\pgfpathmoveto{\pgfqpoint{2.500411in}{2.370476in}}%
\pgfpathlineto{\pgfqpoint{2.625621in}{2.370476in}}%
\pgfpathlineto{\pgfqpoint{2.625621in}{2.432925in}}%
\pgfpathlineto{\pgfqpoint{2.500411in}{2.432925in}}%
\pgfpathclose%
\pgfusepath{stroke,fill}%
\end{pgfscope}%
\begin{pgfscope}%
\pgfpathrectangle{\pgfqpoint{0.550713in}{2.154633in}}{\pgfqpoint{4.791200in}{1.332187in}}%
\pgfusepath{clip}%
\pgfsetbuttcap%
\pgfsetroundjoin%
\definecolor{currentfill}{rgb}{0.904744,0.941561,0.883122}%
\pgfsetfillcolor{currentfill}%
\pgfsetlinewidth{0.752812pt}%
\definecolor{currentstroke}{rgb}{0.188235,0.188235,0.188235}%
\pgfsetstrokecolor{currentstroke}%
\pgfsetdash{}{0pt}%
\pgfpathmoveto{\pgfqpoint{2.675131in}{2.228407in}}%
\pgfpathlineto{\pgfqpoint{2.706433in}{2.228407in}}%
\pgfpathlineto{\pgfqpoint{2.706433in}{2.295924in}}%
\pgfpathlineto{\pgfqpoint{2.675131in}{2.295924in}}%
\pgfpathclose%
\pgfusepath{stroke,fill}%
\end{pgfscope}%
\begin{pgfscope}%
\pgfpathrectangle{\pgfqpoint{0.550713in}{2.154633in}}{\pgfqpoint{4.791200in}{1.332187in}}%
\pgfusepath{clip}%
\pgfsetbuttcap%
\pgfsetroundjoin%
\definecolor{currentfill}{rgb}{0.633831,0.775356,0.550713}%
\pgfsetfillcolor{currentfill}%
\pgfsetlinewidth{0.752812pt}%
\definecolor{currentstroke}{rgb}{0.188235,0.188235,0.188235}%
\pgfsetstrokecolor{currentstroke}%
\pgfsetdash{}{0pt}%
\pgfpathmoveto{\pgfqpoint{2.659479in}{2.236639in}}%
\pgfpathlineto{\pgfqpoint{2.722084in}{2.236639in}}%
\pgfpathlineto{\pgfqpoint{2.722084in}{2.291320in}}%
\pgfpathlineto{\pgfqpoint{2.659479in}{2.291320in}}%
\pgfpathclose%
\pgfusepath{stroke,fill}%
\end{pgfscope}%
\begin{pgfscope}%
\pgfpathrectangle{\pgfqpoint{0.550713in}{2.154633in}}{\pgfqpoint{4.791200in}{1.332187in}}%
\pgfusepath{clip}%
\pgfsetbuttcap%
\pgfsetroundjoin%
\definecolor{currentfill}{rgb}{0.360784,0.607843,0.215686}%
\pgfsetfillcolor{currentfill}%
\pgfsetlinewidth{0.752812pt}%
\definecolor{currentstroke}{rgb}{0.188235,0.188235,0.188235}%
\pgfsetstrokecolor{currentstroke}%
\pgfsetdash{}{0pt}%
\pgfpathmoveto{\pgfqpoint{2.628177in}{2.253102in}}%
\pgfpathlineto{\pgfqpoint{2.753387in}{2.253102in}}%
\pgfpathlineto{\pgfqpoint{2.753387in}{2.282112in}}%
\pgfpathlineto{\pgfqpoint{2.628177in}{2.282112in}}%
\pgfpathclose%
\pgfusepath{stroke,fill}%
\end{pgfscope}%
\begin{pgfscope}%
\pgfpathrectangle{\pgfqpoint{0.550713in}{2.154633in}}{\pgfqpoint{4.791200in}{1.332187in}}%
\pgfusepath{clip}%
\pgfsetbuttcap%
\pgfsetroundjoin%
\definecolor{currentfill}{rgb}{0.862668,0.899485,0.932211}%
\pgfsetfillcolor{currentfill}%
\pgfsetlinewidth{0.752812pt}%
\definecolor{currentstroke}{rgb}{0.188235,0.188235,0.188235}%
\pgfsetstrokecolor{currentstroke}%
\pgfsetdash{}{0pt}%
\pgfpathmoveto{\pgfqpoint{2.866779in}{2.308101in}}%
\pgfpathlineto{\pgfqpoint{2.898081in}{2.308101in}}%
\pgfpathlineto{\pgfqpoint{2.898081in}{2.451220in}}%
\pgfpathlineto{\pgfqpoint{2.866779in}{2.451220in}}%
\pgfpathclose%
\pgfusepath{stroke,fill}%
\end{pgfscope}%
\begin{pgfscope}%
\pgfpathrectangle{\pgfqpoint{0.550713in}{2.154633in}}{\pgfqpoint{4.791200in}{1.332187in}}%
\pgfusepath{clip}%
\pgfsetbuttcap%
\pgfsetroundjoin%
\definecolor{currentfill}{rgb}{0.472088,0.613613,0.739413}%
\pgfsetfillcolor{currentfill}%
\pgfsetlinewidth{0.752812pt}%
\definecolor{currentstroke}{rgb}{0.188235,0.188235,0.188235}%
\pgfsetstrokecolor{currentstroke}%
\pgfsetdash{}{0pt}%
\pgfpathmoveto{\pgfqpoint{2.851127in}{2.325554in}}%
\pgfpathlineto{\pgfqpoint{2.913732in}{2.325554in}}%
\pgfpathlineto{\pgfqpoint{2.913732in}{2.450655in}}%
\pgfpathlineto{\pgfqpoint{2.851127in}{2.450655in}}%
\pgfpathclose%
\pgfusepath{stroke,fill}%
\end{pgfscope}%
\begin{pgfscope}%
\pgfpathrectangle{\pgfqpoint{0.550713in}{2.154633in}}{\pgfqpoint{4.791200in}{1.332187in}}%
\pgfusepath{clip}%
\pgfsetbuttcap%
\pgfsetroundjoin%
\definecolor{currentfill}{rgb}{0.078431,0.325490,0.545098}%
\pgfsetfillcolor{currentfill}%
\pgfsetlinewidth{0.752812pt}%
\definecolor{currentstroke}{rgb}{0.188235,0.188235,0.188235}%
\pgfsetstrokecolor{currentstroke}%
\pgfsetdash{}{0pt}%
\pgfpathmoveto{\pgfqpoint{2.819825in}{2.360459in}}%
\pgfpathlineto{\pgfqpoint{2.945035in}{2.360459in}}%
\pgfpathlineto{\pgfqpoint{2.945035in}{2.449527in}}%
\pgfpathlineto{\pgfqpoint{2.819825in}{2.449527in}}%
\pgfpathclose%
\pgfusepath{stroke,fill}%
\end{pgfscope}%
\begin{pgfscope}%
\pgfpathrectangle{\pgfqpoint{0.550713in}{2.154633in}}{\pgfqpoint{4.791200in}{1.332187in}}%
\pgfusepath{clip}%
\pgfsetbuttcap%
\pgfsetroundjoin%
\definecolor{currentfill}{rgb}{0.904744,0.941561,0.883122}%
\pgfsetfillcolor{currentfill}%
\pgfsetlinewidth{0.752812pt}%
\definecolor{currentstroke}{rgb}{0.188235,0.188235,0.188235}%
\pgfsetstrokecolor{currentstroke}%
\pgfsetdash{}{0pt}%
\pgfpathmoveto{\pgfqpoint{2.994544in}{2.271272in}}%
\pgfpathlineto{\pgfqpoint{3.025846in}{2.271272in}}%
\pgfpathlineto{\pgfqpoint{3.025846in}{2.310056in}}%
\pgfpathlineto{\pgfqpoint{2.994544in}{2.310056in}}%
\pgfpathclose%
\pgfusepath{stroke,fill}%
\end{pgfscope}%
\begin{pgfscope}%
\pgfpathrectangle{\pgfqpoint{0.550713in}{2.154633in}}{\pgfqpoint{4.791200in}{1.332187in}}%
\pgfusepath{clip}%
\pgfsetbuttcap%
\pgfsetroundjoin%
\definecolor{currentfill}{rgb}{0.633831,0.775356,0.550713}%
\pgfsetfillcolor{currentfill}%
\pgfsetlinewidth{0.752812pt}%
\definecolor{currentstroke}{rgb}{0.188235,0.188235,0.188235}%
\pgfsetstrokecolor{currentstroke}%
\pgfsetdash{}{0pt}%
\pgfpathmoveto{\pgfqpoint{2.978893in}{2.271990in}}%
\pgfpathlineto{\pgfqpoint{3.041498in}{2.271990in}}%
\pgfpathlineto{\pgfqpoint{3.041498in}{2.307838in}}%
\pgfpathlineto{\pgfqpoint{2.978893in}{2.307838in}}%
\pgfpathclose%
\pgfusepath{stroke,fill}%
\end{pgfscope}%
\begin{pgfscope}%
\pgfpathrectangle{\pgfqpoint{0.550713in}{2.154633in}}{\pgfqpoint{4.791200in}{1.332187in}}%
\pgfusepath{clip}%
\pgfsetbuttcap%
\pgfsetroundjoin%
\definecolor{currentfill}{rgb}{0.360784,0.607843,0.215686}%
\pgfsetfillcolor{currentfill}%
\pgfsetlinewidth{0.752812pt}%
\definecolor{currentstroke}{rgb}{0.188235,0.188235,0.188235}%
\pgfsetstrokecolor{currentstroke}%
\pgfsetdash{}{0pt}%
\pgfpathmoveto{\pgfqpoint{2.947590in}{2.273425in}}%
\pgfpathlineto{\pgfqpoint{3.072800in}{2.273425in}}%
\pgfpathlineto{\pgfqpoint{3.072800in}{2.303404in}}%
\pgfpathlineto{\pgfqpoint{2.947590in}{2.303404in}}%
\pgfpathclose%
\pgfusepath{stroke,fill}%
\end{pgfscope}%
\begin{pgfscope}%
\pgfpathrectangle{\pgfqpoint{0.550713in}{2.154633in}}{\pgfqpoint{4.791200in}{1.332187in}}%
\pgfusepath{clip}%
\pgfsetbuttcap%
\pgfsetroundjoin%
\definecolor{currentfill}{rgb}{0.862668,0.899485,0.932211}%
\pgfsetfillcolor{currentfill}%
\pgfsetlinewidth{0.752812pt}%
\definecolor{currentstroke}{rgb}{0.188235,0.188235,0.188235}%
\pgfsetstrokecolor{currentstroke}%
\pgfsetdash{}{0pt}%
\pgfpathmoveto{\pgfqpoint{3.186192in}{2.323309in}}%
\pgfpathlineto{\pgfqpoint{3.217494in}{2.323309in}}%
\pgfpathlineto{\pgfqpoint{3.217494in}{2.396440in}}%
\pgfpathlineto{\pgfqpoint{3.186192in}{2.396440in}}%
\pgfpathclose%
\pgfusepath{stroke,fill}%
\end{pgfscope}%
\begin{pgfscope}%
\pgfpathrectangle{\pgfqpoint{0.550713in}{2.154633in}}{\pgfqpoint{4.791200in}{1.332187in}}%
\pgfusepath{clip}%
\pgfsetbuttcap%
\pgfsetroundjoin%
\definecolor{currentfill}{rgb}{0.472088,0.613613,0.739413}%
\pgfsetfillcolor{currentfill}%
\pgfsetlinewidth{0.752812pt}%
\definecolor{currentstroke}{rgb}{0.188235,0.188235,0.188235}%
\pgfsetstrokecolor{currentstroke}%
\pgfsetdash{}{0pt}%
\pgfpathmoveto{\pgfqpoint{3.170541in}{2.333728in}}%
\pgfpathlineto{\pgfqpoint{3.233146in}{2.333728in}}%
\pgfpathlineto{\pgfqpoint{3.233146in}{2.389838in}}%
\pgfpathlineto{\pgfqpoint{3.170541in}{2.389838in}}%
\pgfpathclose%
\pgfusepath{stroke,fill}%
\end{pgfscope}%
\begin{pgfscope}%
\pgfpathrectangle{\pgfqpoint{0.550713in}{2.154633in}}{\pgfqpoint{4.791200in}{1.332187in}}%
\pgfusepath{clip}%
\pgfsetbuttcap%
\pgfsetroundjoin%
\definecolor{currentfill}{rgb}{0.078431,0.325490,0.545098}%
\pgfsetfillcolor{currentfill}%
\pgfsetlinewidth{0.752812pt}%
\definecolor{currentstroke}{rgb}{0.188235,0.188235,0.188235}%
\pgfsetstrokecolor{currentstroke}%
\pgfsetdash{}{0pt}%
\pgfpathmoveto{\pgfqpoint{3.139238in}{2.354564in}}%
\pgfpathlineto{\pgfqpoint{3.264448in}{2.354564in}}%
\pgfpathlineto{\pgfqpoint{3.264448in}{2.376635in}}%
\pgfpathlineto{\pgfqpoint{3.139238in}{2.376635in}}%
\pgfpathclose%
\pgfusepath{stroke,fill}%
\end{pgfscope}%
\begin{pgfscope}%
\pgfpathrectangle{\pgfqpoint{0.550713in}{2.154633in}}{\pgfqpoint{4.791200in}{1.332187in}}%
\pgfusepath{clip}%
\pgfsetbuttcap%
\pgfsetroundjoin%
\definecolor{currentfill}{rgb}{0.904744,0.941561,0.883122}%
\pgfsetfillcolor{currentfill}%
\pgfsetlinewidth{0.752812pt}%
\definecolor{currentstroke}{rgb}{0.188235,0.188235,0.188235}%
\pgfsetstrokecolor{currentstroke}%
\pgfsetdash{}{0pt}%
\pgfpathmoveto{\pgfqpoint{3.313957in}{2.238126in}}%
\pgfpathlineto{\pgfqpoint{3.345260in}{2.238126in}}%
\pgfpathlineto{\pgfqpoint{3.345260in}{2.314787in}}%
\pgfpathlineto{\pgfqpoint{3.313957in}{2.314787in}}%
\pgfpathclose%
\pgfusepath{stroke,fill}%
\end{pgfscope}%
\begin{pgfscope}%
\pgfpathrectangle{\pgfqpoint{0.550713in}{2.154633in}}{\pgfqpoint{4.791200in}{1.332187in}}%
\pgfusepath{clip}%
\pgfsetbuttcap%
\pgfsetroundjoin%
\definecolor{currentfill}{rgb}{0.633831,0.775356,0.550713}%
\pgfsetfillcolor{currentfill}%
\pgfsetlinewidth{0.752812pt}%
\definecolor{currentstroke}{rgb}{0.188235,0.188235,0.188235}%
\pgfsetstrokecolor{currentstroke}%
\pgfsetdash{}{0pt}%
\pgfpathmoveto{\pgfqpoint{3.298306in}{2.253848in}}%
\pgfpathlineto{\pgfqpoint{3.360911in}{2.253848in}}%
\pgfpathlineto{\pgfqpoint{3.360911in}{2.311754in}}%
\pgfpathlineto{\pgfqpoint{3.298306in}{2.311754in}}%
\pgfpathclose%
\pgfusepath{stroke,fill}%
\end{pgfscope}%
\begin{pgfscope}%
\pgfpathrectangle{\pgfqpoint{0.550713in}{2.154633in}}{\pgfqpoint{4.791200in}{1.332187in}}%
\pgfusepath{clip}%
\pgfsetbuttcap%
\pgfsetroundjoin%
\definecolor{currentfill}{rgb}{0.360784,0.607843,0.215686}%
\pgfsetfillcolor{currentfill}%
\pgfsetlinewidth{0.752812pt}%
\definecolor{currentstroke}{rgb}{0.188235,0.188235,0.188235}%
\pgfsetstrokecolor{currentstroke}%
\pgfsetdash{}{0pt}%
\pgfpathmoveto{\pgfqpoint{3.267003in}{2.285291in}}%
\pgfpathlineto{\pgfqpoint{3.392213in}{2.285291in}}%
\pgfpathlineto{\pgfqpoint{3.392213in}{2.305688in}}%
\pgfpathlineto{\pgfqpoint{3.267003in}{2.305688in}}%
\pgfpathclose%
\pgfusepath{stroke,fill}%
\end{pgfscope}%
\begin{pgfscope}%
\pgfpathrectangle{\pgfqpoint{0.550713in}{2.154633in}}{\pgfqpoint{4.791200in}{1.332187in}}%
\pgfusepath{clip}%
\pgfsetbuttcap%
\pgfsetroundjoin%
\definecolor{currentfill}{rgb}{0.862668,0.899485,0.932211}%
\pgfsetfillcolor{currentfill}%
\pgfsetlinewidth{0.752812pt}%
\definecolor{currentstroke}{rgb}{0.188235,0.188235,0.188235}%
\pgfsetstrokecolor{currentstroke}%
\pgfsetdash{}{0pt}%
\pgfpathmoveto{\pgfqpoint{3.505605in}{2.313520in}}%
\pgfpathlineto{\pgfqpoint{3.536908in}{2.313520in}}%
\pgfpathlineto{\pgfqpoint{3.536908in}{2.368719in}}%
\pgfpathlineto{\pgfqpoint{3.505605in}{2.368719in}}%
\pgfpathclose%
\pgfusepath{stroke,fill}%
\end{pgfscope}%
\begin{pgfscope}%
\pgfpathrectangle{\pgfqpoint{0.550713in}{2.154633in}}{\pgfqpoint{4.791200in}{1.332187in}}%
\pgfusepath{clip}%
\pgfsetbuttcap%
\pgfsetroundjoin%
\definecolor{currentfill}{rgb}{0.472088,0.613613,0.739413}%
\pgfsetfillcolor{currentfill}%
\pgfsetlinewidth{0.752812pt}%
\definecolor{currentstroke}{rgb}{0.188235,0.188235,0.188235}%
\pgfsetstrokecolor{currentstroke}%
\pgfsetdash{}{0pt}%
\pgfpathmoveto{\pgfqpoint{3.489954in}{2.315271in}}%
\pgfpathlineto{\pgfqpoint{3.552559in}{2.315271in}}%
\pgfpathlineto{\pgfqpoint{3.552559in}{2.359633in}}%
\pgfpathlineto{\pgfqpoint{3.489954in}{2.359633in}}%
\pgfpathclose%
\pgfusepath{stroke,fill}%
\end{pgfscope}%
\begin{pgfscope}%
\pgfpathrectangle{\pgfqpoint{0.550713in}{2.154633in}}{\pgfqpoint{4.791200in}{1.332187in}}%
\pgfusepath{clip}%
\pgfsetbuttcap%
\pgfsetroundjoin%
\definecolor{currentfill}{rgb}{0.078431,0.325490,0.545098}%
\pgfsetfillcolor{currentfill}%
\pgfsetlinewidth{0.752812pt}%
\definecolor{currentstroke}{rgb}{0.188235,0.188235,0.188235}%
\pgfsetstrokecolor{currentstroke}%
\pgfsetdash{}{0pt}%
\pgfpathmoveto{\pgfqpoint{3.458651in}{2.318773in}}%
\pgfpathlineto{\pgfqpoint{3.583861in}{2.318773in}}%
\pgfpathlineto{\pgfqpoint{3.583861in}{2.341461in}}%
\pgfpathlineto{\pgfqpoint{3.458651in}{2.341461in}}%
\pgfpathclose%
\pgfusepath{stroke,fill}%
\end{pgfscope}%
\begin{pgfscope}%
\pgfpathrectangle{\pgfqpoint{0.550713in}{2.154633in}}{\pgfqpoint{4.791200in}{1.332187in}}%
\pgfusepath{clip}%
\pgfsetbuttcap%
\pgfsetroundjoin%
\definecolor{currentfill}{rgb}{0.904744,0.941561,0.883122}%
\pgfsetfillcolor{currentfill}%
\pgfsetlinewidth{0.752812pt}%
\definecolor{currentstroke}{rgb}{0.188235,0.188235,0.188235}%
\pgfsetstrokecolor{currentstroke}%
\pgfsetdash{}{0pt}%
\pgfpathmoveto{\pgfqpoint{3.633370in}{2.255890in}}%
\pgfpathlineto{\pgfqpoint{3.664673in}{2.255890in}}%
\pgfpathlineto{\pgfqpoint{3.664673in}{2.367715in}}%
\pgfpathlineto{\pgfqpoint{3.633370in}{2.367715in}}%
\pgfpathclose%
\pgfusepath{stroke,fill}%
\end{pgfscope}%
\begin{pgfscope}%
\pgfpathrectangle{\pgfqpoint{0.550713in}{2.154633in}}{\pgfqpoint{4.791200in}{1.332187in}}%
\pgfusepath{clip}%
\pgfsetbuttcap%
\pgfsetroundjoin%
\definecolor{currentfill}{rgb}{0.633831,0.775356,0.550713}%
\pgfsetfillcolor{currentfill}%
\pgfsetlinewidth{0.752812pt}%
\definecolor{currentstroke}{rgb}{0.188235,0.188235,0.188235}%
\pgfsetstrokecolor{currentstroke}%
\pgfsetdash{}{0pt}%
\pgfpathmoveto{\pgfqpoint{3.617719in}{2.272038in}}%
\pgfpathlineto{\pgfqpoint{3.680324in}{2.272038in}}%
\pgfpathlineto{\pgfqpoint{3.680324in}{2.359743in}}%
\pgfpathlineto{\pgfqpoint{3.617719in}{2.359743in}}%
\pgfpathclose%
\pgfusepath{stroke,fill}%
\end{pgfscope}%
\begin{pgfscope}%
\pgfpathrectangle{\pgfqpoint{0.550713in}{2.154633in}}{\pgfqpoint{4.791200in}{1.332187in}}%
\pgfusepath{clip}%
\pgfsetbuttcap%
\pgfsetroundjoin%
\definecolor{currentfill}{rgb}{0.360784,0.607843,0.215686}%
\pgfsetfillcolor{currentfill}%
\pgfsetlinewidth{0.752812pt}%
\definecolor{currentstroke}{rgb}{0.188235,0.188235,0.188235}%
\pgfsetstrokecolor{currentstroke}%
\pgfsetdash{}{0pt}%
\pgfpathmoveto{\pgfqpoint{3.586417in}{2.304335in}}%
\pgfpathlineto{\pgfqpoint{3.711627in}{2.304335in}}%
\pgfpathlineto{\pgfqpoint{3.711627in}{2.343798in}}%
\pgfpathlineto{\pgfqpoint{3.586417in}{2.343798in}}%
\pgfpathclose%
\pgfusepath{stroke,fill}%
\end{pgfscope}%
\begin{pgfscope}%
\pgfpathrectangle{\pgfqpoint{0.550713in}{2.154633in}}{\pgfqpoint{4.791200in}{1.332187in}}%
\pgfusepath{clip}%
\pgfsetbuttcap%
\pgfsetroundjoin%
\definecolor{currentfill}{rgb}{0.862668,0.899485,0.932211}%
\pgfsetfillcolor{currentfill}%
\pgfsetlinewidth{0.752812pt}%
\definecolor{currentstroke}{rgb}{0.188235,0.188235,0.188235}%
\pgfsetstrokecolor{currentstroke}%
\pgfsetdash{}{0pt}%
\pgfpathmoveto{\pgfqpoint{3.825018in}{2.325074in}}%
\pgfpathlineto{\pgfqpoint{3.856321in}{2.325074in}}%
\pgfpathlineto{\pgfqpoint{3.856321in}{2.534630in}}%
\pgfpathlineto{\pgfqpoint{3.825018in}{2.534630in}}%
\pgfpathclose%
\pgfusepath{stroke,fill}%
\end{pgfscope}%
\begin{pgfscope}%
\pgfpathrectangle{\pgfqpoint{0.550713in}{2.154633in}}{\pgfqpoint{4.791200in}{1.332187in}}%
\pgfusepath{clip}%
\pgfsetbuttcap%
\pgfsetroundjoin%
\definecolor{currentfill}{rgb}{0.472088,0.613613,0.739413}%
\pgfsetfillcolor{currentfill}%
\pgfsetlinewidth{0.752812pt}%
\definecolor{currentstroke}{rgb}{0.188235,0.188235,0.188235}%
\pgfsetstrokecolor{currentstroke}%
\pgfsetdash{}{0pt}%
\pgfpathmoveto{\pgfqpoint{3.809367in}{2.325816in}}%
\pgfpathlineto{\pgfqpoint{3.871972in}{2.325816in}}%
\pgfpathlineto{\pgfqpoint{3.871972in}{2.484732in}}%
\pgfpathlineto{\pgfqpoint{3.809367in}{2.484732in}}%
\pgfpathclose%
\pgfusepath{stroke,fill}%
\end{pgfscope}%
\begin{pgfscope}%
\pgfpathrectangle{\pgfqpoint{0.550713in}{2.154633in}}{\pgfqpoint{4.791200in}{1.332187in}}%
\pgfusepath{clip}%
\pgfsetbuttcap%
\pgfsetroundjoin%
\definecolor{currentfill}{rgb}{0.078431,0.325490,0.545098}%
\pgfsetfillcolor{currentfill}%
\pgfsetlinewidth{0.752812pt}%
\definecolor{currentstroke}{rgb}{0.188235,0.188235,0.188235}%
\pgfsetstrokecolor{currentstroke}%
\pgfsetdash{}{0pt}%
\pgfpathmoveto{\pgfqpoint{3.778065in}{2.327300in}}%
\pgfpathlineto{\pgfqpoint{3.903275in}{2.327300in}}%
\pgfpathlineto{\pgfqpoint{3.903275in}{2.384935in}}%
\pgfpathlineto{\pgfqpoint{3.778065in}{2.384935in}}%
\pgfpathclose%
\pgfusepath{stroke,fill}%
\end{pgfscope}%
\begin{pgfscope}%
\pgfpathrectangle{\pgfqpoint{0.550713in}{2.154633in}}{\pgfqpoint{4.791200in}{1.332187in}}%
\pgfusepath{clip}%
\pgfsetbuttcap%
\pgfsetroundjoin%
\definecolor{currentfill}{rgb}{0.904744,0.941561,0.883122}%
\pgfsetfillcolor{currentfill}%
\pgfsetlinewidth{0.752812pt}%
\definecolor{currentstroke}{rgb}{0.188235,0.188235,0.188235}%
\pgfsetstrokecolor{currentstroke}%
\pgfsetdash{}{0pt}%
\pgfpathmoveto{\pgfqpoint{3.952784in}{2.266180in}}%
\pgfpathlineto{\pgfqpoint{3.984086in}{2.266180in}}%
\pgfpathlineto{\pgfqpoint{3.984086in}{2.356972in}}%
\pgfpathlineto{\pgfqpoint{3.952784in}{2.356972in}}%
\pgfpathclose%
\pgfusepath{stroke,fill}%
\end{pgfscope}%
\begin{pgfscope}%
\pgfpathrectangle{\pgfqpoint{0.550713in}{2.154633in}}{\pgfqpoint{4.791200in}{1.332187in}}%
\pgfusepath{clip}%
\pgfsetbuttcap%
\pgfsetroundjoin%
\definecolor{currentfill}{rgb}{0.633831,0.775356,0.550713}%
\pgfsetfillcolor{currentfill}%
\pgfsetlinewidth{0.752812pt}%
\definecolor{currentstroke}{rgb}{0.188235,0.188235,0.188235}%
\pgfsetstrokecolor{currentstroke}%
\pgfsetdash{}{0pt}%
\pgfpathmoveto{\pgfqpoint{3.937133in}{2.285479in}}%
\pgfpathlineto{\pgfqpoint{3.999738in}{2.285479in}}%
\pgfpathlineto{\pgfqpoint{3.999738in}{2.352831in}}%
\pgfpathlineto{\pgfqpoint{3.937133in}{2.352831in}}%
\pgfpathclose%
\pgfusepath{stroke,fill}%
\end{pgfscope}%
\begin{pgfscope}%
\pgfpathrectangle{\pgfqpoint{0.550713in}{2.154633in}}{\pgfqpoint{4.791200in}{1.332187in}}%
\pgfusepath{clip}%
\pgfsetbuttcap%
\pgfsetroundjoin%
\definecolor{currentfill}{rgb}{0.360784,0.607843,0.215686}%
\pgfsetfillcolor{currentfill}%
\pgfsetlinewidth{0.752812pt}%
\definecolor{currentstroke}{rgb}{0.188235,0.188235,0.188235}%
\pgfsetstrokecolor{currentstroke}%
\pgfsetdash{}{0pt}%
\pgfpathmoveto{\pgfqpoint{3.905830in}{2.324077in}}%
\pgfpathlineto{\pgfqpoint{4.031040in}{2.324077in}}%
\pgfpathlineto{\pgfqpoint{4.031040in}{2.344548in}}%
\pgfpathlineto{\pgfqpoint{3.905830in}{2.344548in}}%
\pgfpathclose%
\pgfusepath{stroke,fill}%
\end{pgfscope}%
\begin{pgfscope}%
\pgfpathrectangle{\pgfqpoint{0.550713in}{2.154633in}}{\pgfqpoint{4.791200in}{1.332187in}}%
\pgfusepath{clip}%
\pgfsetbuttcap%
\pgfsetroundjoin%
\definecolor{currentfill}{rgb}{0.862668,0.899485,0.932211}%
\pgfsetfillcolor{currentfill}%
\pgfsetlinewidth{0.752812pt}%
\definecolor{currentstroke}{rgb}{0.188235,0.188235,0.188235}%
\pgfsetstrokecolor{currentstroke}%
\pgfsetdash{}{0pt}%
\pgfpathmoveto{\pgfqpoint{4.144432in}{2.356227in}}%
\pgfpathlineto{\pgfqpoint{4.175734in}{2.356227in}}%
\pgfpathlineto{\pgfqpoint{4.175734in}{2.416467in}}%
\pgfpathlineto{\pgfqpoint{4.144432in}{2.416467in}}%
\pgfpathclose%
\pgfusepath{stroke,fill}%
\end{pgfscope}%
\begin{pgfscope}%
\pgfpathrectangle{\pgfqpoint{0.550713in}{2.154633in}}{\pgfqpoint{4.791200in}{1.332187in}}%
\pgfusepath{clip}%
\pgfsetbuttcap%
\pgfsetroundjoin%
\definecolor{currentfill}{rgb}{0.472088,0.613613,0.739413}%
\pgfsetfillcolor{currentfill}%
\pgfsetlinewidth{0.752812pt}%
\definecolor{currentstroke}{rgb}{0.188235,0.188235,0.188235}%
\pgfsetstrokecolor{currentstroke}%
\pgfsetdash{}{0pt}%
\pgfpathmoveto{\pgfqpoint{4.128781in}{2.367388in}}%
\pgfpathlineto{\pgfqpoint{4.191386in}{2.367388in}}%
\pgfpathlineto{\pgfqpoint{4.191386in}{2.409301in}}%
\pgfpathlineto{\pgfqpoint{4.128781in}{2.409301in}}%
\pgfpathclose%
\pgfusepath{stroke,fill}%
\end{pgfscope}%
\begin{pgfscope}%
\pgfpathrectangle{\pgfqpoint{0.550713in}{2.154633in}}{\pgfqpoint{4.791200in}{1.332187in}}%
\pgfusepath{clip}%
\pgfsetbuttcap%
\pgfsetroundjoin%
\definecolor{currentfill}{rgb}{0.078431,0.325490,0.545098}%
\pgfsetfillcolor{currentfill}%
\pgfsetlinewidth{0.752812pt}%
\definecolor{currentstroke}{rgb}{0.188235,0.188235,0.188235}%
\pgfsetstrokecolor{currentstroke}%
\pgfsetdash{}{0pt}%
\pgfpathmoveto{\pgfqpoint{4.097478in}{2.389710in}}%
\pgfpathlineto{\pgfqpoint{4.222688in}{2.389710in}}%
\pgfpathlineto{\pgfqpoint{4.222688in}{2.394969in}}%
\pgfpathlineto{\pgfqpoint{4.097478in}{2.394969in}}%
\pgfpathclose%
\pgfusepath{stroke,fill}%
\end{pgfscope}%
\begin{pgfscope}%
\pgfpathrectangle{\pgfqpoint{0.550713in}{2.154633in}}{\pgfqpoint{4.791200in}{1.332187in}}%
\pgfusepath{clip}%
\pgfsetbuttcap%
\pgfsetroundjoin%
\definecolor{currentfill}{rgb}{0.904744,0.941561,0.883122}%
\pgfsetfillcolor{currentfill}%
\pgfsetlinewidth{0.752812pt}%
\definecolor{currentstroke}{rgb}{0.188235,0.188235,0.188235}%
\pgfsetstrokecolor{currentstroke}%
\pgfsetdash{}{0pt}%
\pgfpathmoveto{\pgfqpoint{4.272197in}{2.275362in}}%
\pgfpathlineto{\pgfqpoint{4.303500in}{2.275362in}}%
\pgfpathlineto{\pgfqpoint{4.303500in}{2.381723in}}%
\pgfpathlineto{\pgfqpoint{4.272197in}{2.381723in}}%
\pgfpathclose%
\pgfusepath{stroke,fill}%
\end{pgfscope}%
\begin{pgfscope}%
\pgfpathrectangle{\pgfqpoint{0.550713in}{2.154633in}}{\pgfqpoint{4.791200in}{1.332187in}}%
\pgfusepath{clip}%
\pgfsetbuttcap%
\pgfsetroundjoin%
\definecolor{currentfill}{rgb}{0.633831,0.775356,0.550713}%
\pgfsetfillcolor{currentfill}%
\pgfsetlinewidth{0.752812pt}%
\definecolor{currentstroke}{rgb}{0.188235,0.188235,0.188235}%
\pgfsetstrokecolor{currentstroke}%
\pgfsetdash{}{0pt}%
\pgfpathmoveto{\pgfqpoint{4.256546in}{2.298046in}}%
\pgfpathlineto{\pgfqpoint{4.319151in}{2.298046in}}%
\pgfpathlineto{\pgfqpoint{4.319151in}{2.376806in}}%
\pgfpathlineto{\pgfqpoint{4.256546in}{2.376806in}}%
\pgfpathclose%
\pgfusepath{stroke,fill}%
\end{pgfscope}%
\begin{pgfscope}%
\pgfpathrectangle{\pgfqpoint{0.550713in}{2.154633in}}{\pgfqpoint{4.791200in}{1.332187in}}%
\pgfusepath{clip}%
\pgfsetbuttcap%
\pgfsetroundjoin%
\definecolor{currentfill}{rgb}{0.360784,0.607843,0.215686}%
\pgfsetfillcolor{currentfill}%
\pgfsetlinewidth{0.752812pt}%
\definecolor{currentstroke}{rgb}{0.188235,0.188235,0.188235}%
\pgfsetstrokecolor{currentstroke}%
\pgfsetdash{}{0pt}%
\pgfpathmoveto{\pgfqpoint{4.225243in}{2.343412in}}%
\pgfpathlineto{\pgfqpoint{4.350453in}{2.343412in}}%
\pgfpathlineto{\pgfqpoint{4.350453in}{2.366972in}}%
\pgfpathlineto{\pgfqpoint{4.225243in}{2.366972in}}%
\pgfpathclose%
\pgfusepath{stroke,fill}%
\end{pgfscope}%
\begin{pgfscope}%
\pgfpathrectangle{\pgfqpoint{0.550713in}{2.154633in}}{\pgfqpoint{4.791200in}{1.332187in}}%
\pgfusepath{clip}%
\pgfsetbuttcap%
\pgfsetroundjoin%
\definecolor{currentfill}{rgb}{0.862668,0.899485,0.932211}%
\pgfsetfillcolor{currentfill}%
\pgfsetlinewidth{0.752812pt}%
\definecolor{currentstroke}{rgb}{0.188235,0.188235,0.188235}%
\pgfsetstrokecolor{currentstroke}%
\pgfsetdash{}{0pt}%
\pgfpathmoveto{\pgfqpoint{4.463845in}{2.345199in}}%
\pgfpathlineto{\pgfqpoint{4.495148in}{2.345199in}}%
\pgfpathlineto{\pgfqpoint{4.495148in}{2.440003in}}%
\pgfpathlineto{\pgfqpoint{4.463845in}{2.440003in}}%
\pgfpathclose%
\pgfusepath{stroke,fill}%
\end{pgfscope}%
\begin{pgfscope}%
\pgfpathrectangle{\pgfqpoint{0.550713in}{2.154633in}}{\pgfqpoint{4.791200in}{1.332187in}}%
\pgfusepath{clip}%
\pgfsetbuttcap%
\pgfsetroundjoin%
\definecolor{currentfill}{rgb}{0.472088,0.613613,0.739413}%
\pgfsetfillcolor{currentfill}%
\pgfsetlinewidth{0.752812pt}%
\definecolor{currentstroke}{rgb}{0.188235,0.188235,0.188235}%
\pgfsetstrokecolor{currentstroke}%
\pgfsetdash{}{0pt}%
\pgfpathmoveto{\pgfqpoint{4.448194in}{2.359688in}}%
\pgfpathlineto{\pgfqpoint{4.510799in}{2.359688in}}%
\pgfpathlineto{\pgfqpoint{4.510799in}{2.435443in}}%
\pgfpathlineto{\pgfqpoint{4.448194in}{2.435443in}}%
\pgfpathclose%
\pgfusepath{stroke,fill}%
\end{pgfscope}%
\begin{pgfscope}%
\pgfpathrectangle{\pgfqpoint{0.550713in}{2.154633in}}{\pgfqpoint{4.791200in}{1.332187in}}%
\pgfusepath{clip}%
\pgfsetbuttcap%
\pgfsetroundjoin%
\definecolor{currentfill}{rgb}{0.078431,0.325490,0.545098}%
\pgfsetfillcolor{currentfill}%
\pgfsetlinewidth{0.752812pt}%
\definecolor{currentstroke}{rgb}{0.188235,0.188235,0.188235}%
\pgfsetstrokecolor{currentstroke}%
\pgfsetdash{}{0pt}%
\pgfpathmoveto{\pgfqpoint{4.416891in}{2.388668in}}%
\pgfpathlineto{\pgfqpoint{4.542101in}{2.388668in}}%
\pgfpathlineto{\pgfqpoint{4.542101in}{2.426323in}}%
\pgfpathlineto{\pgfqpoint{4.416891in}{2.426323in}}%
\pgfpathclose%
\pgfusepath{stroke,fill}%
\end{pgfscope}%
\begin{pgfscope}%
\pgfpathrectangle{\pgfqpoint{0.550713in}{2.154633in}}{\pgfqpoint{4.791200in}{1.332187in}}%
\pgfusepath{clip}%
\pgfsetbuttcap%
\pgfsetroundjoin%
\definecolor{currentfill}{rgb}{0.904744,0.941561,0.883122}%
\pgfsetfillcolor{currentfill}%
\pgfsetlinewidth{0.752812pt}%
\definecolor{currentstroke}{rgb}{0.188235,0.188235,0.188235}%
\pgfsetstrokecolor{currentstroke}%
\pgfsetdash{}{0pt}%
\pgfpathmoveto{\pgfqpoint{4.591610in}{2.302101in}}%
\pgfpathlineto{\pgfqpoint{4.622913in}{2.302101in}}%
\pgfpathlineto{\pgfqpoint{4.622913in}{2.404753in}}%
\pgfpathlineto{\pgfqpoint{4.591610in}{2.404753in}}%
\pgfpathclose%
\pgfusepath{stroke,fill}%
\end{pgfscope}%
\begin{pgfscope}%
\pgfpathrectangle{\pgfqpoint{0.550713in}{2.154633in}}{\pgfqpoint{4.791200in}{1.332187in}}%
\pgfusepath{clip}%
\pgfsetbuttcap%
\pgfsetroundjoin%
\definecolor{currentfill}{rgb}{0.633831,0.775356,0.550713}%
\pgfsetfillcolor{currentfill}%
\pgfsetlinewidth{0.752812pt}%
\definecolor{currentstroke}{rgb}{0.188235,0.188235,0.188235}%
\pgfsetstrokecolor{currentstroke}%
\pgfsetdash{}{0pt}%
\pgfpathmoveto{\pgfqpoint{4.575959in}{2.325851in}}%
\pgfpathlineto{\pgfqpoint{4.638564in}{2.325851in}}%
\pgfpathlineto{\pgfqpoint{4.638564in}{2.399718in}}%
\pgfpathlineto{\pgfqpoint{4.575959in}{2.399718in}}%
\pgfpathclose%
\pgfusepath{stroke,fill}%
\end{pgfscope}%
\begin{pgfscope}%
\pgfpathrectangle{\pgfqpoint{0.550713in}{2.154633in}}{\pgfqpoint{4.791200in}{1.332187in}}%
\pgfusepath{clip}%
\pgfsetbuttcap%
\pgfsetroundjoin%
\definecolor{currentfill}{rgb}{0.360784,0.607843,0.215686}%
\pgfsetfillcolor{currentfill}%
\pgfsetlinewidth{0.752812pt}%
\definecolor{currentstroke}{rgb}{0.188235,0.188235,0.188235}%
\pgfsetstrokecolor{currentstroke}%
\pgfsetdash{}{0pt}%
\pgfpathmoveto{\pgfqpoint{4.544657in}{2.373352in}}%
\pgfpathlineto{\pgfqpoint{4.669867in}{2.373352in}}%
\pgfpathlineto{\pgfqpoint{4.669867in}{2.389646in}}%
\pgfpathlineto{\pgfqpoint{4.544657in}{2.389646in}}%
\pgfpathclose%
\pgfusepath{stroke,fill}%
\end{pgfscope}%
\begin{pgfscope}%
\pgfpathrectangle{\pgfqpoint{0.550713in}{2.154633in}}{\pgfqpoint{4.791200in}{1.332187in}}%
\pgfusepath{clip}%
\pgfsetbuttcap%
\pgfsetroundjoin%
\definecolor{currentfill}{rgb}{0.862668,0.899485,0.932211}%
\pgfsetfillcolor{currentfill}%
\pgfsetlinewidth{0.752812pt}%
\definecolor{currentstroke}{rgb}{0.188235,0.188235,0.188235}%
\pgfsetstrokecolor{currentstroke}%
\pgfsetdash{}{0pt}%
\pgfpathmoveto{\pgfqpoint{4.783258in}{2.417031in}}%
\pgfpathlineto{\pgfqpoint{4.814561in}{2.417031in}}%
\pgfpathlineto{\pgfqpoint{4.814561in}{2.661040in}}%
\pgfpathlineto{\pgfqpoint{4.783258in}{2.661040in}}%
\pgfpathclose%
\pgfusepath{stroke,fill}%
\end{pgfscope}%
\begin{pgfscope}%
\pgfpathrectangle{\pgfqpoint{0.550713in}{2.154633in}}{\pgfqpoint{4.791200in}{1.332187in}}%
\pgfusepath{clip}%
\pgfsetbuttcap%
\pgfsetroundjoin%
\definecolor{currentfill}{rgb}{0.472088,0.613613,0.739413}%
\pgfsetfillcolor{currentfill}%
\pgfsetlinewidth{0.752812pt}%
\definecolor{currentstroke}{rgb}{0.188235,0.188235,0.188235}%
\pgfsetstrokecolor{currentstroke}%
\pgfsetdash{}{0pt}%
\pgfpathmoveto{\pgfqpoint{4.767607in}{2.477498in}}%
\pgfpathlineto{\pgfqpoint{4.830212in}{2.477498in}}%
\pgfpathlineto{\pgfqpoint{4.830212in}{2.644065in}}%
\pgfpathlineto{\pgfqpoint{4.767607in}{2.644065in}}%
\pgfpathclose%
\pgfusepath{stroke,fill}%
\end{pgfscope}%
\begin{pgfscope}%
\pgfpathrectangle{\pgfqpoint{0.550713in}{2.154633in}}{\pgfqpoint{4.791200in}{1.332187in}}%
\pgfusepath{clip}%
\pgfsetbuttcap%
\pgfsetroundjoin%
\definecolor{currentfill}{rgb}{0.078431,0.325490,0.545098}%
\pgfsetfillcolor{currentfill}%
\pgfsetlinewidth{0.752812pt}%
\definecolor{currentstroke}{rgb}{0.188235,0.188235,0.188235}%
\pgfsetstrokecolor{currentstroke}%
\pgfsetdash{}{0pt}%
\pgfpathmoveto{\pgfqpoint{4.736305in}{2.598430in}}%
\pgfpathlineto{\pgfqpoint{4.861515in}{2.598430in}}%
\pgfpathlineto{\pgfqpoint{4.861515in}{2.610115in}}%
\pgfpathlineto{\pgfqpoint{4.736305in}{2.610115in}}%
\pgfpathclose%
\pgfusepath{stroke,fill}%
\end{pgfscope}%
\begin{pgfscope}%
\pgfpathrectangle{\pgfqpoint{0.550713in}{2.154633in}}{\pgfqpoint{4.791200in}{1.332187in}}%
\pgfusepath{clip}%
\pgfsetbuttcap%
\pgfsetroundjoin%
\definecolor{currentfill}{rgb}{0.904744,0.941561,0.883122}%
\pgfsetfillcolor{currentfill}%
\pgfsetlinewidth{0.752812pt}%
\definecolor{currentstroke}{rgb}{0.188235,0.188235,0.188235}%
\pgfsetstrokecolor{currentstroke}%
\pgfsetdash{}{0pt}%
\pgfpathmoveto{\pgfqpoint{4.911024in}{2.373666in}}%
\pgfpathlineto{\pgfqpoint{4.942326in}{2.373666in}}%
\pgfpathlineto{\pgfqpoint{4.942326in}{2.490295in}}%
\pgfpathlineto{\pgfqpoint{4.911024in}{2.490295in}}%
\pgfpathclose%
\pgfusepath{stroke,fill}%
\end{pgfscope}%
\begin{pgfscope}%
\pgfpathrectangle{\pgfqpoint{0.550713in}{2.154633in}}{\pgfqpoint{4.791200in}{1.332187in}}%
\pgfusepath{clip}%
\pgfsetbuttcap%
\pgfsetroundjoin%
\definecolor{currentfill}{rgb}{0.633831,0.775356,0.550713}%
\pgfsetfillcolor{currentfill}%
\pgfsetlinewidth{0.752812pt}%
\definecolor{currentstroke}{rgb}{0.188235,0.188235,0.188235}%
\pgfsetstrokecolor{currentstroke}%
\pgfsetdash{}{0pt}%
\pgfpathmoveto{\pgfqpoint{4.895372in}{2.399730in}}%
\pgfpathlineto{\pgfqpoint{4.957977in}{2.399730in}}%
\pgfpathlineto{\pgfqpoint{4.957977in}{2.486399in}}%
\pgfpathlineto{\pgfqpoint{4.895372in}{2.486399in}}%
\pgfpathclose%
\pgfusepath{stroke,fill}%
\end{pgfscope}%
\begin{pgfscope}%
\pgfpathrectangle{\pgfqpoint{0.550713in}{2.154633in}}{\pgfqpoint{4.791200in}{1.332187in}}%
\pgfusepath{clip}%
\pgfsetbuttcap%
\pgfsetroundjoin%
\definecolor{currentfill}{rgb}{0.360784,0.607843,0.215686}%
\pgfsetfillcolor{currentfill}%
\pgfsetlinewidth{0.752812pt}%
\definecolor{currentstroke}{rgb}{0.188235,0.188235,0.188235}%
\pgfsetstrokecolor{currentstroke}%
\pgfsetdash{}{0pt}%
\pgfpathmoveto{\pgfqpoint{4.864070in}{2.451859in}}%
\pgfpathlineto{\pgfqpoint{4.989280in}{2.451859in}}%
\pgfpathlineto{\pgfqpoint{4.989280in}{2.478608in}}%
\pgfpathlineto{\pgfqpoint{4.864070in}{2.478608in}}%
\pgfpathclose%
\pgfusepath{stroke,fill}%
\end{pgfscope}%
\begin{pgfscope}%
\pgfpathrectangle{\pgfqpoint{0.550713in}{2.154633in}}{\pgfqpoint{4.791200in}{1.332187in}}%
\pgfusepath{clip}%
\pgfsetbuttcap%
\pgfsetroundjoin%
\definecolor{currentfill}{rgb}{0.862668,0.899485,0.932211}%
\pgfsetfillcolor{currentfill}%
\pgfsetlinewidth{0.752812pt}%
\definecolor{currentstroke}{rgb}{0.188235,0.188235,0.188235}%
\pgfsetstrokecolor{currentstroke}%
\pgfsetdash{}{0pt}%
\pgfpathmoveto{\pgfqpoint{5.102672in}{2.499187in}}%
\pgfpathlineto{\pgfqpoint{5.133974in}{2.499187in}}%
\pgfpathlineto{\pgfqpoint{5.133974in}{2.871918in}}%
\pgfpathlineto{\pgfqpoint{5.102672in}{2.871918in}}%
\pgfpathclose%
\pgfusepath{stroke,fill}%
\end{pgfscope}%
\begin{pgfscope}%
\pgfpathrectangle{\pgfqpoint{0.550713in}{2.154633in}}{\pgfqpoint{4.791200in}{1.332187in}}%
\pgfusepath{clip}%
\pgfsetbuttcap%
\pgfsetroundjoin%
\definecolor{currentfill}{rgb}{0.472088,0.613613,0.739413}%
\pgfsetfillcolor{currentfill}%
\pgfsetlinewidth{0.752812pt}%
\definecolor{currentstroke}{rgb}{0.188235,0.188235,0.188235}%
\pgfsetstrokecolor{currentstroke}%
\pgfsetdash{}{0pt}%
\pgfpathmoveto{\pgfqpoint{5.087020in}{2.571380in}}%
\pgfpathlineto{\pgfqpoint{5.149625in}{2.571380in}}%
\pgfpathlineto{\pgfqpoint{5.149625in}{2.847390in}}%
\pgfpathlineto{\pgfqpoint{5.087020in}{2.847390in}}%
\pgfpathclose%
\pgfusepath{stroke,fill}%
\end{pgfscope}%
\begin{pgfscope}%
\pgfpathrectangle{\pgfqpoint{0.550713in}{2.154633in}}{\pgfqpoint{4.791200in}{1.332187in}}%
\pgfusepath{clip}%
\pgfsetbuttcap%
\pgfsetroundjoin%
\definecolor{currentfill}{rgb}{0.078431,0.325490,0.545098}%
\pgfsetfillcolor{currentfill}%
\pgfsetlinewidth{0.752812pt}%
\definecolor{currentstroke}{rgb}{0.188235,0.188235,0.188235}%
\pgfsetstrokecolor{currentstroke}%
\pgfsetdash{}{0pt}%
\pgfpathmoveto{\pgfqpoint{5.055718in}{2.715766in}}%
\pgfpathlineto{\pgfqpoint{5.180928in}{2.715766in}}%
\pgfpathlineto{\pgfqpoint{5.180928in}{2.798333in}}%
\pgfpathlineto{\pgfqpoint{5.055718in}{2.798333in}}%
\pgfpathclose%
\pgfusepath{stroke,fill}%
\end{pgfscope}%
\begin{pgfscope}%
\pgfpathrectangle{\pgfqpoint{0.550713in}{2.154633in}}{\pgfqpoint{4.791200in}{1.332187in}}%
\pgfusepath{clip}%
\pgfsetbuttcap%
\pgfsetroundjoin%
\definecolor{currentfill}{rgb}{0.904744,0.941561,0.883122}%
\pgfsetfillcolor{currentfill}%
\pgfsetlinewidth{0.752812pt}%
\definecolor{currentstroke}{rgb}{0.188235,0.188235,0.188235}%
\pgfsetstrokecolor{currentstroke}%
\pgfsetdash{}{0pt}%
\pgfpathmoveto{\pgfqpoint{5.230437in}{2.413605in}}%
\pgfpathlineto{\pgfqpoint{5.261740in}{2.413605in}}%
\pgfpathlineto{\pgfqpoint{5.261740in}{2.513493in}}%
\pgfpathlineto{\pgfqpoint{5.230437in}{2.513493in}}%
\pgfpathclose%
\pgfusepath{stroke,fill}%
\end{pgfscope}%
\begin{pgfscope}%
\pgfpathrectangle{\pgfqpoint{0.550713in}{2.154633in}}{\pgfqpoint{4.791200in}{1.332187in}}%
\pgfusepath{clip}%
\pgfsetbuttcap%
\pgfsetroundjoin%
\definecolor{currentfill}{rgb}{0.633831,0.775356,0.550713}%
\pgfsetfillcolor{currentfill}%
\pgfsetlinewidth{0.752812pt}%
\definecolor{currentstroke}{rgb}{0.188235,0.188235,0.188235}%
\pgfsetstrokecolor{currentstroke}%
\pgfsetdash{}{0pt}%
\pgfpathmoveto{\pgfqpoint{5.214786in}{2.433684in}}%
\pgfpathlineto{\pgfqpoint{5.277391in}{2.433684in}}%
\pgfpathlineto{\pgfqpoint{5.277391in}{2.512290in}}%
\pgfpathlineto{\pgfqpoint{5.214786in}{2.512290in}}%
\pgfpathclose%
\pgfusepath{stroke,fill}%
\end{pgfscope}%
\begin{pgfscope}%
\pgfpathrectangle{\pgfqpoint{0.550713in}{2.154633in}}{\pgfqpoint{4.791200in}{1.332187in}}%
\pgfusepath{clip}%
\pgfsetbuttcap%
\pgfsetroundjoin%
\definecolor{currentfill}{rgb}{0.360784,0.607843,0.215686}%
\pgfsetfillcolor{currentfill}%
\pgfsetlinewidth{0.752812pt}%
\definecolor{currentstroke}{rgb}{0.188235,0.188235,0.188235}%
\pgfsetstrokecolor{currentstroke}%
\pgfsetdash{}{0pt}%
\pgfpathmoveto{\pgfqpoint{5.183483in}{2.473841in}}%
\pgfpathlineto{\pgfqpoint{5.308693in}{2.473841in}}%
\pgfpathlineto{\pgfqpoint{5.308693in}{2.509884in}}%
\pgfpathlineto{\pgfqpoint{5.183483in}{2.509884in}}%
\pgfpathclose%
\pgfusepath{stroke,fill}%
\end{pgfscope}%
\begin{pgfscope}%
\pgfpathrectangle{\pgfqpoint{0.550713in}{2.154633in}}{\pgfqpoint{4.791200in}{1.332187in}}%
\pgfusepath{clip}%
\pgfsetbuttcap%
\pgfsetmiterjoin%
\definecolor{currentfill}{rgb}{0.077941,0.325000,0.545588}%
\pgfsetfillcolor{currentfill}%
\pgfsetlinewidth{0.376406pt}%
\definecolor{currentstroke}{rgb}{0.188235,0.188235,0.188235}%
\pgfsetstrokecolor{currentstroke}%
\pgfsetdash{}{0pt}%
\pgfpathmoveto{\pgfqpoint{0.710419in}{2.154633in}}%
\pgfpathlineto{\pgfqpoint{0.710419in}{2.154633in}}%
\pgfpathlineto{\pgfqpoint{0.710419in}{2.154633in}}%
\pgfpathlineto{\pgfqpoint{0.710419in}{2.154633in}}%
\pgfpathclose%
\pgfusepath{stroke,fill}%
\end{pgfscope}%
\begin{pgfscope}%
\pgfpathrectangle{\pgfqpoint{0.550713in}{2.154633in}}{\pgfqpoint{4.791200in}{1.332187in}}%
\pgfusepath{clip}%
\pgfsetbuttcap%
\pgfsetmiterjoin%
\definecolor{currentfill}{rgb}{0.358824,0.605882,0.217647}%
\pgfsetfillcolor{currentfill}%
\pgfsetlinewidth{0.376406pt}%
\definecolor{currentstroke}{rgb}{0.188235,0.188235,0.188235}%
\pgfsetstrokecolor{currentstroke}%
\pgfsetdash{}{0pt}%
\pgfpathmoveto{\pgfqpoint{0.710419in}{2.154633in}}%
\pgfpathlineto{\pgfqpoint{0.710419in}{2.154633in}}%
\pgfpathlineto{\pgfqpoint{0.710419in}{2.154633in}}%
\pgfpathlineto{\pgfqpoint{0.710419in}{2.154633in}}%
\pgfpathclose%
\pgfusepath{stroke,fill}%
\end{pgfscope}%
\begin{pgfscope}%
\pgfsetbuttcap%
\pgfsetroundjoin%
\definecolor{currentfill}{rgb}{0.000000,0.000000,0.000000}%
\pgfsetfillcolor{currentfill}%
\pgfsetlinewidth{0.803000pt}%
\definecolor{currentstroke}{rgb}{0.000000,0.000000,0.000000}%
\pgfsetstrokecolor{currentstroke}%
\pgfsetdash{}{0pt}%
\pgfsys@defobject{currentmarker}{\pgfqpoint{0.000000in}{-0.048611in}}{\pgfqpoint{0.000000in}{0.000000in}}{%
\pgfpathmoveto{\pgfqpoint{0.000000in}{0.000000in}}%
\pgfpathlineto{\pgfqpoint{0.000000in}{-0.048611in}}%
\pgfusepath{stroke,fill}%
}%
\begin{pgfscope}%
\pgfsys@transformshift{0.710419in}{2.154633in}%
\pgfsys@useobject{currentmarker}{}%
\end{pgfscope}%
\end{pgfscope}%
\begin{pgfscope}%
\pgfsetbuttcap%
\pgfsetroundjoin%
\definecolor{currentfill}{rgb}{0.000000,0.000000,0.000000}%
\pgfsetfillcolor{currentfill}%
\pgfsetlinewidth{0.803000pt}%
\definecolor{currentstroke}{rgb}{0.000000,0.000000,0.000000}%
\pgfsetstrokecolor{currentstroke}%
\pgfsetdash{}{0pt}%
\pgfsys@defobject{currentmarker}{\pgfqpoint{0.000000in}{-0.048611in}}{\pgfqpoint{0.000000in}{0.000000in}}{%
\pgfpathmoveto{\pgfqpoint{0.000000in}{0.000000in}}%
\pgfpathlineto{\pgfqpoint{0.000000in}{-0.048611in}}%
\pgfusepath{stroke,fill}%
}%
\begin{pgfscope}%
\pgfsys@transformshift{1.029833in}{2.154633in}%
\pgfsys@useobject{currentmarker}{}%
\end{pgfscope}%
\end{pgfscope}%
\begin{pgfscope}%
\pgfsetbuttcap%
\pgfsetroundjoin%
\definecolor{currentfill}{rgb}{0.000000,0.000000,0.000000}%
\pgfsetfillcolor{currentfill}%
\pgfsetlinewidth{0.803000pt}%
\definecolor{currentstroke}{rgb}{0.000000,0.000000,0.000000}%
\pgfsetstrokecolor{currentstroke}%
\pgfsetdash{}{0pt}%
\pgfsys@defobject{currentmarker}{\pgfqpoint{0.000000in}{-0.048611in}}{\pgfqpoint{0.000000in}{0.000000in}}{%
\pgfpathmoveto{\pgfqpoint{0.000000in}{0.000000in}}%
\pgfpathlineto{\pgfqpoint{0.000000in}{-0.048611in}}%
\pgfusepath{stroke,fill}%
}%
\begin{pgfscope}%
\pgfsys@transformshift{1.349246in}{2.154633in}%
\pgfsys@useobject{currentmarker}{}%
\end{pgfscope}%
\end{pgfscope}%
\begin{pgfscope}%
\pgfsetbuttcap%
\pgfsetroundjoin%
\definecolor{currentfill}{rgb}{0.000000,0.000000,0.000000}%
\pgfsetfillcolor{currentfill}%
\pgfsetlinewidth{0.803000pt}%
\definecolor{currentstroke}{rgb}{0.000000,0.000000,0.000000}%
\pgfsetstrokecolor{currentstroke}%
\pgfsetdash{}{0pt}%
\pgfsys@defobject{currentmarker}{\pgfqpoint{0.000000in}{-0.048611in}}{\pgfqpoint{0.000000in}{0.000000in}}{%
\pgfpathmoveto{\pgfqpoint{0.000000in}{0.000000in}}%
\pgfpathlineto{\pgfqpoint{0.000000in}{-0.048611in}}%
\pgfusepath{stroke,fill}%
}%
\begin{pgfscope}%
\pgfsys@transformshift{1.668659in}{2.154633in}%
\pgfsys@useobject{currentmarker}{}%
\end{pgfscope}%
\end{pgfscope}%
\begin{pgfscope}%
\pgfsetbuttcap%
\pgfsetroundjoin%
\definecolor{currentfill}{rgb}{0.000000,0.000000,0.000000}%
\pgfsetfillcolor{currentfill}%
\pgfsetlinewidth{0.803000pt}%
\definecolor{currentstroke}{rgb}{0.000000,0.000000,0.000000}%
\pgfsetstrokecolor{currentstroke}%
\pgfsetdash{}{0pt}%
\pgfsys@defobject{currentmarker}{\pgfqpoint{0.000000in}{-0.048611in}}{\pgfqpoint{0.000000in}{0.000000in}}{%
\pgfpathmoveto{\pgfqpoint{0.000000in}{0.000000in}}%
\pgfpathlineto{\pgfqpoint{0.000000in}{-0.048611in}}%
\pgfusepath{stroke,fill}%
}%
\begin{pgfscope}%
\pgfsys@transformshift{1.988073in}{2.154633in}%
\pgfsys@useobject{currentmarker}{}%
\end{pgfscope}%
\end{pgfscope}%
\begin{pgfscope}%
\pgfsetbuttcap%
\pgfsetroundjoin%
\definecolor{currentfill}{rgb}{0.000000,0.000000,0.000000}%
\pgfsetfillcolor{currentfill}%
\pgfsetlinewidth{0.803000pt}%
\definecolor{currentstroke}{rgb}{0.000000,0.000000,0.000000}%
\pgfsetstrokecolor{currentstroke}%
\pgfsetdash{}{0pt}%
\pgfsys@defobject{currentmarker}{\pgfqpoint{0.000000in}{-0.048611in}}{\pgfqpoint{0.000000in}{0.000000in}}{%
\pgfpathmoveto{\pgfqpoint{0.000000in}{0.000000in}}%
\pgfpathlineto{\pgfqpoint{0.000000in}{-0.048611in}}%
\pgfusepath{stroke,fill}%
}%
\begin{pgfscope}%
\pgfsys@transformshift{2.307486in}{2.154633in}%
\pgfsys@useobject{currentmarker}{}%
\end{pgfscope}%
\end{pgfscope}%
\begin{pgfscope}%
\pgfsetbuttcap%
\pgfsetroundjoin%
\definecolor{currentfill}{rgb}{0.000000,0.000000,0.000000}%
\pgfsetfillcolor{currentfill}%
\pgfsetlinewidth{0.803000pt}%
\definecolor{currentstroke}{rgb}{0.000000,0.000000,0.000000}%
\pgfsetstrokecolor{currentstroke}%
\pgfsetdash{}{0pt}%
\pgfsys@defobject{currentmarker}{\pgfqpoint{0.000000in}{-0.048611in}}{\pgfqpoint{0.000000in}{0.000000in}}{%
\pgfpathmoveto{\pgfqpoint{0.000000in}{0.000000in}}%
\pgfpathlineto{\pgfqpoint{0.000000in}{-0.048611in}}%
\pgfusepath{stroke,fill}%
}%
\begin{pgfscope}%
\pgfsys@transformshift{2.626899in}{2.154633in}%
\pgfsys@useobject{currentmarker}{}%
\end{pgfscope}%
\end{pgfscope}%
\begin{pgfscope}%
\pgfsetbuttcap%
\pgfsetroundjoin%
\definecolor{currentfill}{rgb}{0.000000,0.000000,0.000000}%
\pgfsetfillcolor{currentfill}%
\pgfsetlinewidth{0.803000pt}%
\definecolor{currentstroke}{rgb}{0.000000,0.000000,0.000000}%
\pgfsetstrokecolor{currentstroke}%
\pgfsetdash{}{0pt}%
\pgfsys@defobject{currentmarker}{\pgfqpoint{0.000000in}{-0.048611in}}{\pgfqpoint{0.000000in}{0.000000in}}{%
\pgfpathmoveto{\pgfqpoint{0.000000in}{0.000000in}}%
\pgfpathlineto{\pgfqpoint{0.000000in}{-0.048611in}}%
\pgfusepath{stroke,fill}%
}%
\begin{pgfscope}%
\pgfsys@transformshift{2.946312in}{2.154633in}%
\pgfsys@useobject{currentmarker}{}%
\end{pgfscope}%
\end{pgfscope}%
\begin{pgfscope}%
\pgfsetbuttcap%
\pgfsetroundjoin%
\definecolor{currentfill}{rgb}{0.000000,0.000000,0.000000}%
\pgfsetfillcolor{currentfill}%
\pgfsetlinewidth{0.803000pt}%
\definecolor{currentstroke}{rgb}{0.000000,0.000000,0.000000}%
\pgfsetstrokecolor{currentstroke}%
\pgfsetdash{}{0pt}%
\pgfsys@defobject{currentmarker}{\pgfqpoint{0.000000in}{-0.048611in}}{\pgfqpoint{0.000000in}{0.000000in}}{%
\pgfpathmoveto{\pgfqpoint{0.000000in}{0.000000in}}%
\pgfpathlineto{\pgfqpoint{0.000000in}{-0.048611in}}%
\pgfusepath{stroke,fill}%
}%
\begin{pgfscope}%
\pgfsys@transformshift{3.265726in}{2.154633in}%
\pgfsys@useobject{currentmarker}{}%
\end{pgfscope}%
\end{pgfscope}%
\begin{pgfscope}%
\pgfsetbuttcap%
\pgfsetroundjoin%
\definecolor{currentfill}{rgb}{0.000000,0.000000,0.000000}%
\pgfsetfillcolor{currentfill}%
\pgfsetlinewidth{0.803000pt}%
\definecolor{currentstroke}{rgb}{0.000000,0.000000,0.000000}%
\pgfsetstrokecolor{currentstroke}%
\pgfsetdash{}{0pt}%
\pgfsys@defobject{currentmarker}{\pgfqpoint{0.000000in}{-0.048611in}}{\pgfqpoint{0.000000in}{0.000000in}}{%
\pgfpathmoveto{\pgfqpoint{0.000000in}{0.000000in}}%
\pgfpathlineto{\pgfqpoint{0.000000in}{-0.048611in}}%
\pgfusepath{stroke,fill}%
}%
\begin{pgfscope}%
\pgfsys@transformshift{3.585139in}{2.154633in}%
\pgfsys@useobject{currentmarker}{}%
\end{pgfscope}%
\end{pgfscope}%
\begin{pgfscope}%
\pgfsetbuttcap%
\pgfsetroundjoin%
\definecolor{currentfill}{rgb}{0.000000,0.000000,0.000000}%
\pgfsetfillcolor{currentfill}%
\pgfsetlinewidth{0.803000pt}%
\definecolor{currentstroke}{rgb}{0.000000,0.000000,0.000000}%
\pgfsetstrokecolor{currentstroke}%
\pgfsetdash{}{0pt}%
\pgfsys@defobject{currentmarker}{\pgfqpoint{0.000000in}{-0.048611in}}{\pgfqpoint{0.000000in}{0.000000in}}{%
\pgfpathmoveto{\pgfqpoint{0.000000in}{0.000000in}}%
\pgfpathlineto{\pgfqpoint{0.000000in}{-0.048611in}}%
\pgfusepath{stroke,fill}%
}%
\begin{pgfscope}%
\pgfsys@transformshift{3.904552in}{2.154633in}%
\pgfsys@useobject{currentmarker}{}%
\end{pgfscope}%
\end{pgfscope}%
\begin{pgfscope}%
\pgfsetbuttcap%
\pgfsetroundjoin%
\definecolor{currentfill}{rgb}{0.000000,0.000000,0.000000}%
\pgfsetfillcolor{currentfill}%
\pgfsetlinewidth{0.803000pt}%
\definecolor{currentstroke}{rgb}{0.000000,0.000000,0.000000}%
\pgfsetstrokecolor{currentstroke}%
\pgfsetdash{}{0pt}%
\pgfsys@defobject{currentmarker}{\pgfqpoint{0.000000in}{-0.048611in}}{\pgfqpoint{0.000000in}{0.000000in}}{%
\pgfpathmoveto{\pgfqpoint{0.000000in}{0.000000in}}%
\pgfpathlineto{\pgfqpoint{0.000000in}{-0.048611in}}%
\pgfusepath{stroke,fill}%
}%
\begin{pgfscope}%
\pgfsys@transformshift{4.223966in}{2.154633in}%
\pgfsys@useobject{currentmarker}{}%
\end{pgfscope}%
\end{pgfscope}%
\begin{pgfscope}%
\pgfsetbuttcap%
\pgfsetroundjoin%
\definecolor{currentfill}{rgb}{0.000000,0.000000,0.000000}%
\pgfsetfillcolor{currentfill}%
\pgfsetlinewidth{0.803000pt}%
\definecolor{currentstroke}{rgb}{0.000000,0.000000,0.000000}%
\pgfsetstrokecolor{currentstroke}%
\pgfsetdash{}{0pt}%
\pgfsys@defobject{currentmarker}{\pgfqpoint{0.000000in}{-0.048611in}}{\pgfqpoint{0.000000in}{0.000000in}}{%
\pgfpathmoveto{\pgfqpoint{0.000000in}{0.000000in}}%
\pgfpathlineto{\pgfqpoint{0.000000in}{-0.048611in}}%
\pgfusepath{stroke,fill}%
}%
\begin{pgfscope}%
\pgfsys@transformshift{4.543379in}{2.154633in}%
\pgfsys@useobject{currentmarker}{}%
\end{pgfscope}%
\end{pgfscope}%
\begin{pgfscope}%
\pgfsetbuttcap%
\pgfsetroundjoin%
\definecolor{currentfill}{rgb}{0.000000,0.000000,0.000000}%
\pgfsetfillcolor{currentfill}%
\pgfsetlinewidth{0.803000pt}%
\definecolor{currentstroke}{rgb}{0.000000,0.000000,0.000000}%
\pgfsetstrokecolor{currentstroke}%
\pgfsetdash{}{0pt}%
\pgfsys@defobject{currentmarker}{\pgfqpoint{0.000000in}{-0.048611in}}{\pgfqpoint{0.000000in}{0.000000in}}{%
\pgfpathmoveto{\pgfqpoint{0.000000in}{0.000000in}}%
\pgfpathlineto{\pgfqpoint{0.000000in}{-0.048611in}}%
\pgfusepath{stroke,fill}%
}%
\begin{pgfscope}%
\pgfsys@transformshift{4.862792in}{2.154633in}%
\pgfsys@useobject{currentmarker}{}%
\end{pgfscope}%
\end{pgfscope}%
\begin{pgfscope}%
\pgfsetbuttcap%
\pgfsetroundjoin%
\definecolor{currentfill}{rgb}{0.000000,0.000000,0.000000}%
\pgfsetfillcolor{currentfill}%
\pgfsetlinewidth{0.803000pt}%
\definecolor{currentstroke}{rgb}{0.000000,0.000000,0.000000}%
\pgfsetstrokecolor{currentstroke}%
\pgfsetdash{}{0pt}%
\pgfsys@defobject{currentmarker}{\pgfqpoint{0.000000in}{-0.048611in}}{\pgfqpoint{0.000000in}{0.000000in}}{%
\pgfpathmoveto{\pgfqpoint{0.000000in}{0.000000in}}%
\pgfpathlineto{\pgfqpoint{0.000000in}{-0.048611in}}%
\pgfusepath{stroke,fill}%
}%
\begin{pgfscope}%
\pgfsys@transformshift{5.182206in}{2.154633in}%
\pgfsys@useobject{currentmarker}{}%
\end{pgfscope}%
\end{pgfscope}%
\begin{pgfscope}%
\pgfsetbuttcap%
\pgfsetroundjoin%
\definecolor{currentfill}{rgb}{0.000000,0.000000,0.000000}%
\pgfsetfillcolor{currentfill}%
\pgfsetlinewidth{0.803000pt}%
\definecolor{currentstroke}{rgb}{0.000000,0.000000,0.000000}%
\pgfsetstrokecolor{currentstroke}%
\pgfsetdash{}{0pt}%
\pgfsys@defobject{currentmarker}{\pgfqpoint{-0.048611in}{0.000000in}}{\pgfqpoint{-0.000000in}{0.000000in}}{%
\pgfpathmoveto{\pgfqpoint{-0.000000in}{0.000000in}}%
\pgfpathlineto{\pgfqpoint{-0.048611in}{0.000000in}}%
\pgfusepath{stroke,fill}%
}%
\begin{pgfscope}%
\pgfsys@transformshift{0.550713in}{2.154633in}%
\pgfsys@useobject{currentmarker}{}%
\end{pgfscope}%
\end{pgfscope}%
\begin{pgfscope}%
\definecolor{textcolor}{rgb}{0.000000,0.000000,0.000000}%
\pgfsetstrokecolor{textcolor}%
\pgfsetfillcolor{textcolor}%
\pgftext[x=0.384046in, y=2.106438in, left, base]{\color{textcolor}\rmfamily\fontsize{10.000000}{12.000000}\selectfont \(\displaystyle {0}\)}%
\end{pgfscope}%
\begin{pgfscope}%
\pgfsetbuttcap%
\pgfsetroundjoin%
\definecolor{currentfill}{rgb}{0.000000,0.000000,0.000000}%
\pgfsetfillcolor{currentfill}%
\pgfsetlinewidth{0.803000pt}%
\definecolor{currentstroke}{rgb}{0.000000,0.000000,0.000000}%
\pgfsetstrokecolor{currentstroke}%
\pgfsetdash{}{0pt}%
\pgfsys@defobject{currentmarker}{\pgfqpoint{-0.048611in}{0.000000in}}{\pgfqpoint{-0.000000in}{0.000000in}}{%
\pgfpathmoveto{\pgfqpoint{-0.000000in}{0.000000in}}%
\pgfpathlineto{\pgfqpoint{-0.048611in}{0.000000in}}%
\pgfusepath{stroke,fill}%
}%
\begin{pgfscope}%
\pgfsys@transformshift{0.550713in}{2.487679in}%
\pgfsys@useobject{currentmarker}{}%
\end{pgfscope}%
\end{pgfscope}%
\begin{pgfscope}%
\definecolor{textcolor}{rgb}{0.000000,0.000000,0.000000}%
\pgfsetstrokecolor{textcolor}%
\pgfsetfillcolor{textcolor}%
\pgftext[x=0.245156in, y=2.439485in, left, base]{\color{textcolor}\rmfamily\fontsize{10.000000}{12.000000}\selectfont \(\displaystyle {200}\)}%
\end{pgfscope}%
\begin{pgfscope}%
\pgfsetbuttcap%
\pgfsetroundjoin%
\definecolor{currentfill}{rgb}{0.000000,0.000000,0.000000}%
\pgfsetfillcolor{currentfill}%
\pgfsetlinewidth{0.803000pt}%
\definecolor{currentstroke}{rgb}{0.000000,0.000000,0.000000}%
\pgfsetstrokecolor{currentstroke}%
\pgfsetdash{}{0pt}%
\pgfsys@defobject{currentmarker}{\pgfqpoint{-0.048611in}{0.000000in}}{\pgfqpoint{-0.000000in}{0.000000in}}{%
\pgfpathmoveto{\pgfqpoint{-0.000000in}{0.000000in}}%
\pgfpathlineto{\pgfqpoint{-0.048611in}{0.000000in}}%
\pgfusepath{stroke,fill}%
}%
\begin{pgfscope}%
\pgfsys@transformshift{0.550713in}{2.820726in}%
\pgfsys@useobject{currentmarker}{}%
\end{pgfscope}%
\end{pgfscope}%
\begin{pgfscope}%
\definecolor{textcolor}{rgb}{0.000000,0.000000,0.000000}%
\pgfsetstrokecolor{textcolor}%
\pgfsetfillcolor{textcolor}%
\pgftext[x=0.245156in, y=2.772532in, left, base]{\color{textcolor}\rmfamily\fontsize{10.000000}{12.000000}\selectfont \(\displaystyle {400}\)}%
\end{pgfscope}%
\begin{pgfscope}%
\pgfsetbuttcap%
\pgfsetroundjoin%
\definecolor{currentfill}{rgb}{0.000000,0.000000,0.000000}%
\pgfsetfillcolor{currentfill}%
\pgfsetlinewidth{0.803000pt}%
\definecolor{currentstroke}{rgb}{0.000000,0.000000,0.000000}%
\pgfsetstrokecolor{currentstroke}%
\pgfsetdash{}{0pt}%
\pgfsys@defobject{currentmarker}{\pgfqpoint{-0.048611in}{0.000000in}}{\pgfqpoint{-0.000000in}{0.000000in}}{%
\pgfpathmoveto{\pgfqpoint{-0.000000in}{0.000000in}}%
\pgfpathlineto{\pgfqpoint{-0.048611in}{0.000000in}}%
\pgfusepath{stroke,fill}%
}%
\begin{pgfscope}%
\pgfsys@transformshift{0.550713in}{3.153773in}%
\pgfsys@useobject{currentmarker}{}%
\end{pgfscope}%
\end{pgfscope}%
\begin{pgfscope}%
\definecolor{textcolor}{rgb}{0.000000,0.000000,0.000000}%
\pgfsetstrokecolor{textcolor}%
\pgfsetfillcolor{textcolor}%
\pgftext[x=0.245156in, y=3.105578in, left, base]{\color{textcolor}\rmfamily\fontsize{10.000000}{12.000000}\selectfont \(\displaystyle {600}\)}%
\end{pgfscope}%
\begin{pgfscope}%
\pgfsetbuttcap%
\pgfsetroundjoin%
\definecolor{currentfill}{rgb}{0.000000,0.000000,0.000000}%
\pgfsetfillcolor{currentfill}%
\pgfsetlinewidth{0.803000pt}%
\definecolor{currentstroke}{rgb}{0.000000,0.000000,0.000000}%
\pgfsetstrokecolor{currentstroke}%
\pgfsetdash{}{0pt}%
\pgfsys@defobject{currentmarker}{\pgfqpoint{-0.048611in}{0.000000in}}{\pgfqpoint{-0.000000in}{0.000000in}}{%
\pgfpathmoveto{\pgfqpoint{-0.000000in}{0.000000in}}%
\pgfpathlineto{\pgfqpoint{-0.048611in}{0.000000in}}%
\pgfusepath{stroke,fill}%
}%
\begin{pgfscope}%
\pgfsys@transformshift{0.550713in}{3.486820in}%
\pgfsys@useobject{currentmarker}{}%
\end{pgfscope}%
\end{pgfscope}%
\begin{pgfscope}%
\definecolor{textcolor}{rgb}{0.000000,0.000000,0.000000}%
\pgfsetstrokecolor{textcolor}%
\pgfsetfillcolor{textcolor}%
\pgftext[x=0.245156in, y=3.438625in, left, base]{\color{textcolor}\rmfamily\fontsize{10.000000}{12.000000}\selectfont \(\displaystyle {800}\)}%
\end{pgfscope}%
\begin{pgfscope}%
\definecolor{textcolor}{rgb}{0.000000,0.000000,0.000000}%
\pgfsetstrokecolor{textcolor}%
\pgfsetfillcolor{textcolor}%
\pgftext[x=0.189601in,y=2.820726in,,bottom,rotate=90.000000]{\color{textcolor}\rmfamily\fontsize{10.000000}{12.000000}\selectfont \(\displaystyle R_T\)}%
\end{pgfscope}%
\begin{pgfscope}%
\pgfpathrectangle{\pgfqpoint{0.550713in}{2.154633in}}{\pgfqpoint{4.791200in}{1.332187in}}%
\pgfusepath{clip}%
\pgfsetbuttcap%
\pgfsetroundjoin%
\pgfsetlinewidth{0.501875pt}%
\definecolor{currentstroke}{rgb}{0.392157,0.396078,0.403922}%
\pgfsetstrokecolor{currentstroke}%
\pgfsetdash{}{0pt}%
\pgfpathmoveto{\pgfqpoint{0.870126in}{2.154633in}}%
\pgfpathlineto{\pgfqpoint{0.870126in}{3.486820in}}%
\pgfusepath{stroke}%
\end{pgfscope}%
\begin{pgfscope}%
\pgfpathrectangle{\pgfqpoint{0.550713in}{2.154633in}}{\pgfqpoint{4.791200in}{1.332187in}}%
\pgfusepath{clip}%
\pgfsetbuttcap%
\pgfsetroundjoin%
\pgfsetlinewidth{0.501875pt}%
\definecolor{currentstroke}{rgb}{0.392157,0.396078,0.403922}%
\pgfsetstrokecolor{currentstroke}%
\pgfsetdash{}{0pt}%
\pgfpathmoveto{\pgfqpoint{1.189539in}{2.154633in}}%
\pgfpathlineto{\pgfqpoint{1.189539in}{3.486820in}}%
\pgfusepath{stroke}%
\end{pgfscope}%
\begin{pgfscope}%
\pgfpathrectangle{\pgfqpoint{0.550713in}{2.154633in}}{\pgfqpoint{4.791200in}{1.332187in}}%
\pgfusepath{clip}%
\pgfsetbuttcap%
\pgfsetroundjoin%
\pgfsetlinewidth{0.501875pt}%
\definecolor{currentstroke}{rgb}{0.392157,0.396078,0.403922}%
\pgfsetstrokecolor{currentstroke}%
\pgfsetdash{}{0pt}%
\pgfpathmoveto{\pgfqpoint{1.508953in}{2.154633in}}%
\pgfpathlineto{\pgfqpoint{1.508953in}{3.486820in}}%
\pgfusepath{stroke}%
\end{pgfscope}%
\begin{pgfscope}%
\pgfpathrectangle{\pgfqpoint{0.550713in}{2.154633in}}{\pgfqpoint{4.791200in}{1.332187in}}%
\pgfusepath{clip}%
\pgfsetbuttcap%
\pgfsetroundjoin%
\pgfsetlinewidth{0.501875pt}%
\definecolor{currentstroke}{rgb}{0.392157,0.396078,0.403922}%
\pgfsetstrokecolor{currentstroke}%
\pgfsetdash{}{0pt}%
\pgfpathmoveto{\pgfqpoint{1.828366in}{2.154633in}}%
\pgfpathlineto{\pgfqpoint{1.828366in}{3.486820in}}%
\pgfusepath{stroke}%
\end{pgfscope}%
\begin{pgfscope}%
\pgfpathrectangle{\pgfqpoint{0.550713in}{2.154633in}}{\pgfqpoint{4.791200in}{1.332187in}}%
\pgfusepath{clip}%
\pgfsetbuttcap%
\pgfsetroundjoin%
\pgfsetlinewidth{0.501875pt}%
\definecolor{currentstroke}{rgb}{0.392157,0.396078,0.403922}%
\pgfsetstrokecolor{currentstroke}%
\pgfsetdash{}{0pt}%
\pgfpathmoveto{\pgfqpoint{2.147779in}{2.154633in}}%
\pgfpathlineto{\pgfqpoint{2.147779in}{3.486820in}}%
\pgfusepath{stroke}%
\end{pgfscope}%
\begin{pgfscope}%
\pgfpathrectangle{\pgfqpoint{0.550713in}{2.154633in}}{\pgfqpoint{4.791200in}{1.332187in}}%
\pgfusepath{clip}%
\pgfsetbuttcap%
\pgfsetroundjoin%
\pgfsetlinewidth{0.501875pt}%
\definecolor{currentstroke}{rgb}{0.392157,0.396078,0.403922}%
\pgfsetstrokecolor{currentstroke}%
\pgfsetdash{}{0pt}%
\pgfpathmoveto{\pgfqpoint{2.467192in}{2.154633in}}%
\pgfpathlineto{\pgfqpoint{2.467192in}{3.486820in}}%
\pgfusepath{stroke}%
\end{pgfscope}%
\begin{pgfscope}%
\pgfpathrectangle{\pgfqpoint{0.550713in}{2.154633in}}{\pgfqpoint{4.791200in}{1.332187in}}%
\pgfusepath{clip}%
\pgfsetbuttcap%
\pgfsetroundjoin%
\pgfsetlinewidth{0.501875pt}%
\definecolor{currentstroke}{rgb}{0.392157,0.396078,0.403922}%
\pgfsetstrokecolor{currentstroke}%
\pgfsetdash{}{0pt}%
\pgfpathmoveto{\pgfqpoint{2.786606in}{2.154633in}}%
\pgfpathlineto{\pgfqpoint{2.786606in}{3.486820in}}%
\pgfusepath{stroke}%
\end{pgfscope}%
\begin{pgfscope}%
\pgfpathrectangle{\pgfqpoint{0.550713in}{2.154633in}}{\pgfqpoint{4.791200in}{1.332187in}}%
\pgfusepath{clip}%
\pgfsetbuttcap%
\pgfsetroundjoin%
\pgfsetlinewidth{0.501875pt}%
\definecolor{currentstroke}{rgb}{0.392157,0.396078,0.403922}%
\pgfsetstrokecolor{currentstroke}%
\pgfsetdash{}{0pt}%
\pgfpathmoveto{\pgfqpoint{3.106019in}{2.154633in}}%
\pgfpathlineto{\pgfqpoint{3.106019in}{3.486820in}}%
\pgfusepath{stroke}%
\end{pgfscope}%
\begin{pgfscope}%
\pgfpathrectangle{\pgfqpoint{0.550713in}{2.154633in}}{\pgfqpoint{4.791200in}{1.332187in}}%
\pgfusepath{clip}%
\pgfsetbuttcap%
\pgfsetroundjoin%
\pgfsetlinewidth{0.501875pt}%
\definecolor{currentstroke}{rgb}{0.392157,0.396078,0.403922}%
\pgfsetstrokecolor{currentstroke}%
\pgfsetdash{}{0pt}%
\pgfpathmoveto{\pgfqpoint{3.425432in}{2.154633in}}%
\pgfpathlineto{\pgfqpoint{3.425432in}{3.486820in}}%
\pgfusepath{stroke}%
\end{pgfscope}%
\begin{pgfscope}%
\pgfpathrectangle{\pgfqpoint{0.550713in}{2.154633in}}{\pgfqpoint{4.791200in}{1.332187in}}%
\pgfusepath{clip}%
\pgfsetbuttcap%
\pgfsetroundjoin%
\pgfsetlinewidth{0.501875pt}%
\definecolor{currentstroke}{rgb}{0.392157,0.396078,0.403922}%
\pgfsetstrokecolor{currentstroke}%
\pgfsetdash{}{0pt}%
\pgfpathmoveto{\pgfqpoint{3.744846in}{2.154633in}}%
\pgfpathlineto{\pgfqpoint{3.744846in}{3.486820in}}%
\pgfusepath{stroke}%
\end{pgfscope}%
\begin{pgfscope}%
\pgfpathrectangle{\pgfqpoint{0.550713in}{2.154633in}}{\pgfqpoint{4.791200in}{1.332187in}}%
\pgfusepath{clip}%
\pgfsetbuttcap%
\pgfsetroundjoin%
\pgfsetlinewidth{0.501875pt}%
\definecolor{currentstroke}{rgb}{0.392157,0.396078,0.403922}%
\pgfsetstrokecolor{currentstroke}%
\pgfsetdash{}{0pt}%
\pgfpathmoveto{\pgfqpoint{4.064259in}{2.154633in}}%
\pgfpathlineto{\pgfqpoint{4.064259in}{3.486820in}}%
\pgfusepath{stroke}%
\end{pgfscope}%
\begin{pgfscope}%
\pgfpathrectangle{\pgfqpoint{0.550713in}{2.154633in}}{\pgfqpoint{4.791200in}{1.332187in}}%
\pgfusepath{clip}%
\pgfsetbuttcap%
\pgfsetroundjoin%
\pgfsetlinewidth{0.501875pt}%
\definecolor{currentstroke}{rgb}{0.392157,0.396078,0.403922}%
\pgfsetstrokecolor{currentstroke}%
\pgfsetdash{}{0pt}%
\pgfpathmoveto{\pgfqpoint{4.383672in}{2.154633in}}%
\pgfpathlineto{\pgfqpoint{4.383672in}{3.486820in}}%
\pgfusepath{stroke}%
\end{pgfscope}%
\begin{pgfscope}%
\pgfpathrectangle{\pgfqpoint{0.550713in}{2.154633in}}{\pgfqpoint{4.791200in}{1.332187in}}%
\pgfusepath{clip}%
\pgfsetbuttcap%
\pgfsetroundjoin%
\pgfsetlinewidth{0.501875pt}%
\definecolor{currentstroke}{rgb}{0.392157,0.396078,0.403922}%
\pgfsetstrokecolor{currentstroke}%
\pgfsetdash{}{0pt}%
\pgfpathmoveto{\pgfqpoint{4.703086in}{2.154633in}}%
\pgfpathlineto{\pgfqpoint{4.703086in}{3.486820in}}%
\pgfusepath{stroke}%
\end{pgfscope}%
\begin{pgfscope}%
\pgfpathrectangle{\pgfqpoint{0.550713in}{2.154633in}}{\pgfqpoint{4.791200in}{1.332187in}}%
\pgfusepath{clip}%
\pgfsetbuttcap%
\pgfsetroundjoin%
\pgfsetlinewidth{0.501875pt}%
\definecolor{currentstroke}{rgb}{0.392157,0.396078,0.403922}%
\pgfsetstrokecolor{currentstroke}%
\pgfsetdash{}{0pt}%
\pgfpathmoveto{\pgfqpoint{5.022499in}{2.154633in}}%
\pgfpathlineto{\pgfqpoint{5.022499in}{3.486820in}}%
\pgfusepath{stroke}%
\end{pgfscope}%
\begin{pgfscope}%
\pgfpathrectangle{\pgfqpoint{0.550713in}{2.154633in}}{\pgfqpoint{4.791200in}{1.332187in}}%
\pgfusepath{clip}%
\pgfsetbuttcap%
\pgfsetroundjoin%
\pgfsetlinewidth{0.752812pt}%
\definecolor{currentstroke}{rgb}{0.150000,0.150000,0.150000}%
\pgfsetstrokecolor{currentstroke}%
\pgfsetstrokeopacity{0.450000}%
\pgfsetdash{}{0pt}%
\pgfpathmoveto{\pgfqpoint{0.583932in}{2.554019in}}%
\pgfpathlineto{\pgfqpoint{0.709142in}{2.554019in}}%
\pgfusepath{stroke}%
\end{pgfscope}%
\begin{pgfscope}%
\pgfpathrectangle{\pgfqpoint{0.550713in}{2.154633in}}{\pgfqpoint{4.791200in}{1.332187in}}%
\pgfusepath{clip}%
\pgfsetbuttcap%
\pgfsetroundjoin%
\pgfsetlinewidth{0.752812pt}%
\definecolor{currentstroke}{rgb}{0.150000,0.150000,0.150000}%
\pgfsetstrokecolor{currentstroke}%
\pgfsetstrokeopacity{0.450000}%
\pgfsetdash{}{0pt}%
\pgfpathmoveto{\pgfqpoint{0.711697in}{2.260746in}}%
\pgfpathlineto{\pgfqpoint{0.836907in}{2.260746in}}%
\pgfusepath{stroke}%
\end{pgfscope}%
\begin{pgfscope}%
\pgfpathrectangle{\pgfqpoint{0.550713in}{2.154633in}}{\pgfqpoint{4.791200in}{1.332187in}}%
\pgfusepath{clip}%
\pgfsetbuttcap%
\pgfsetroundjoin%
\pgfsetlinewidth{0.752812pt}%
\definecolor{currentstroke}{rgb}{0.150000,0.150000,0.150000}%
\pgfsetstrokecolor{currentstroke}%
\pgfsetstrokeopacity{0.450000}%
\pgfsetdash{}{0pt}%
\pgfpathmoveto{\pgfqpoint{0.903345in}{2.318039in}}%
\pgfpathlineto{\pgfqpoint{1.028555in}{2.318039in}}%
\pgfusepath{stroke}%
\end{pgfscope}%
\begin{pgfscope}%
\pgfpathrectangle{\pgfqpoint{0.550713in}{2.154633in}}{\pgfqpoint{4.791200in}{1.332187in}}%
\pgfusepath{clip}%
\pgfsetbuttcap%
\pgfsetroundjoin%
\pgfsetlinewidth{0.752812pt}%
\definecolor{currentstroke}{rgb}{0.150000,0.150000,0.150000}%
\pgfsetstrokecolor{currentstroke}%
\pgfsetstrokeopacity{0.450000}%
\pgfsetdash{}{0pt}%
\pgfpathmoveto{\pgfqpoint{1.031110in}{2.240685in}}%
\pgfpathlineto{\pgfqpoint{1.156320in}{2.240685in}}%
\pgfusepath{stroke}%
\end{pgfscope}%
\begin{pgfscope}%
\pgfpathrectangle{\pgfqpoint{0.550713in}{2.154633in}}{\pgfqpoint{4.791200in}{1.332187in}}%
\pgfusepath{clip}%
\pgfsetbuttcap%
\pgfsetroundjoin%
\pgfsetlinewidth{0.752812pt}%
\definecolor{currentstroke}{rgb}{0.150000,0.150000,0.150000}%
\pgfsetstrokecolor{currentstroke}%
\pgfsetstrokeopacity{0.450000}%
\pgfsetdash{}{0pt}%
\pgfpathmoveto{\pgfqpoint{1.222758in}{2.335604in}}%
\pgfpathlineto{\pgfqpoint{1.347968in}{2.335604in}}%
\pgfusepath{stroke}%
\end{pgfscope}%
\begin{pgfscope}%
\pgfpathrectangle{\pgfqpoint{0.550713in}{2.154633in}}{\pgfqpoint{4.791200in}{1.332187in}}%
\pgfusepath{clip}%
\pgfsetbuttcap%
\pgfsetroundjoin%
\pgfsetlinewidth{0.752812pt}%
\definecolor{currentstroke}{rgb}{0.150000,0.150000,0.150000}%
\pgfsetstrokecolor{currentstroke}%
\pgfsetstrokeopacity{0.450000}%
\pgfsetdash{}{0pt}%
\pgfpathmoveto{\pgfqpoint{1.350524in}{2.234903in}}%
\pgfpathlineto{\pgfqpoint{1.475734in}{2.234903in}}%
\pgfusepath{stroke}%
\end{pgfscope}%
\begin{pgfscope}%
\pgfpathrectangle{\pgfqpoint{0.550713in}{2.154633in}}{\pgfqpoint{4.791200in}{1.332187in}}%
\pgfusepath{clip}%
\pgfsetbuttcap%
\pgfsetroundjoin%
\pgfsetlinewidth{0.752812pt}%
\definecolor{currentstroke}{rgb}{0.150000,0.150000,0.150000}%
\pgfsetstrokecolor{currentstroke}%
\pgfsetstrokeopacity{0.450000}%
\pgfsetdash{}{0pt}%
\pgfpathmoveto{\pgfqpoint{1.542172in}{2.309311in}}%
\pgfpathlineto{\pgfqpoint{1.667382in}{2.309311in}}%
\pgfusepath{stroke}%
\end{pgfscope}%
\begin{pgfscope}%
\pgfpathrectangle{\pgfqpoint{0.550713in}{2.154633in}}{\pgfqpoint{4.791200in}{1.332187in}}%
\pgfusepath{clip}%
\pgfsetbuttcap%
\pgfsetroundjoin%
\pgfsetlinewidth{0.752812pt}%
\definecolor{currentstroke}{rgb}{0.150000,0.150000,0.150000}%
\pgfsetstrokecolor{currentstroke}%
\pgfsetstrokeopacity{0.450000}%
\pgfsetdash{}{0pt}%
\pgfpathmoveto{\pgfqpoint{1.669937in}{2.248496in}}%
\pgfpathlineto{\pgfqpoint{1.795147in}{2.248496in}}%
\pgfusepath{stroke}%
\end{pgfscope}%
\begin{pgfscope}%
\pgfpathrectangle{\pgfqpoint{0.550713in}{2.154633in}}{\pgfqpoint{4.791200in}{1.332187in}}%
\pgfusepath{clip}%
\pgfsetbuttcap%
\pgfsetroundjoin%
\pgfsetlinewidth{0.752812pt}%
\definecolor{currentstroke}{rgb}{0.150000,0.150000,0.150000}%
\pgfsetstrokecolor{currentstroke}%
\pgfsetstrokeopacity{0.450000}%
\pgfsetdash{}{0pt}%
\pgfpathmoveto{\pgfqpoint{1.861585in}{2.334527in}}%
\pgfpathlineto{\pgfqpoint{1.986795in}{2.334527in}}%
\pgfusepath{stroke}%
\end{pgfscope}%
\begin{pgfscope}%
\pgfpathrectangle{\pgfqpoint{0.550713in}{2.154633in}}{\pgfqpoint{4.791200in}{1.332187in}}%
\pgfusepath{clip}%
\pgfsetbuttcap%
\pgfsetroundjoin%
\pgfsetlinewidth{0.752812pt}%
\definecolor{currentstroke}{rgb}{0.150000,0.150000,0.150000}%
\pgfsetstrokecolor{currentstroke}%
\pgfsetstrokeopacity{0.450000}%
\pgfsetdash{}{0pt}%
\pgfpathmoveto{\pgfqpoint{1.989350in}{2.255544in}}%
\pgfpathlineto{\pgfqpoint{2.114560in}{2.255544in}}%
\pgfusepath{stroke}%
\end{pgfscope}%
\begin{pgfscope}%
\pgfpathrectangle{\pgfqpoint{0.550713in}{2.154633in}}{\pgfqpoint{4.791200in}{1.332187in}}%
\pgfusepath{clip}%
\pgfsetbuttcap%
\pgfsetroundjoin%
\pgfsetlinewidth{0.752812pt}%
\definecolor{currentstroke}{rgb}{0.150000,0.150000,0.150000}%
\pgfsetstrokecolor{currentstroke}%
\pgfsetstrokeopacity{0.450000}%
\pgfsetdash{}{0pt}%
\pgfpathmoveto{\pgfqpoint{2.180998in}{2.298241in}}%
\pgfpathlineto{\pgfqpoint{2.306208in}{2.298241in}}%
\pgfusepath{stroke}%
\end{pgfscope}%
\begin{pgfscope}%
\pgfpathrectangle{\pgfqpoint{0.550713in}{2.154633in}}{\pgfqpoint{4.791200in}{1.332187in}}%
\pgfusepath{clip}%
\pgfsetbuttcap%
\pgfsetroundjoin%
\pgfsetlinewidth{0.752812pt}%
\definecolor{currentstroke}{rgb}{0.150000,0.150000,0.150000}%
\pgfsetstrokecolor{currentstroke}%
\pgfsetstrokeopacity{0.450000}%
\pgfsetdash{}{0pt}%
\pgfpathmoveto{\pgfqpoint{2.308763in}{2.245269in}}%
\pgfpathlineto{\pgfqpoint{2.433973in}{2.245269in}}%
\pgfusepath{stroke}%
\end{pgfscope}%
\begin{pgfscope}%
\pgfpathrectangle{\pgfqpoint{0.550713in}{2.154633in}}{\pgfqpoint{4.791200in}{1.332187in}}%
\pgfusepath{clip}%
\pgfsetbuttcap%
\pgfsetroundjoin%
\pgfsetlinewidth{0.752812pt}%
\definecolor{currentstroke}{rgb}{0.150000,0.150000,0.150000}%
\pgfsetstrokecolor{currentstroke}%
\pgfsetstrokeopacity{0.450000}%
\pgfsetdash{}{0pt}%
\pgfpathmoveto{\pgfqpoint{2.500411in}{2.392315in}}%
\pgfpathlineto{\pgfqpoint{2.625621in}{2.392315in}}%
\pgfusepath{stroke}%
\end{pgfscope}%
\begin{pgfscope}%
\pgfpathrectangle{\pgfqpoint{0.550713in}{2.154633in}}{\pgfqpoint{4.791200in}{1.332187in}}%
\pgfusepath{clip}%
\pgfsetbuttcap%
\pgfsetroundjoin%
\pgfsetlinewidth{0.752812pt}%
\definecolor{currentstroke}{rgb}{0.150000,0.150000,0.150000}%
\pgfsetstrokecolor{currentstroke}%
\pgfsetstrokeopacity{0.450000}%
\pgfsetdash{}{0pt}%
\pgfpathmoveto{\pgfqpoint{2.628177in}{2.254847in}}%
\pgfpathlineto{\pgfqpoint{2.753387in}{2.254847in}}%
\pgfusepath{stroke}%
\end{pgfscope}%
\begin{pgfscope}%
\pgfpathrectangle{\pgfqpoint{0.550713in}{2.154633in}}{\pgfqpoint{4.791200in}{1.332187in}}%
\pgfusepath{clip}%
\pgfsetbuttcap%
\pgfsetroundjoin%
\pgfsetlinewidth{0.752812pt}%
\definecolor{currentstroke}{rgb}{0.150000,0.150000,0.150000}%
\pgfsetstrokecolor{currentstroke}%
\pgfsetstrokeopacity{0.450000}%
\pgfsetdash{}{0pt}%
\pgfpathmoveto{\pgfqpoint{2.819825in}{2.418155in}}%
\pgfpathlineto{\pgfqpoint{2.945035in}{2.418155in}}%
\pgfusepath{stroke}%
\end{pgfscope}%
\begin{pgfscope}%
\pgfpathrectangle{\pgfqpoint{0.550713in}{2.154633in}}{\pgfqpoint{4.791200in}{1.332187in}}%
\pgfusepath{clip}%
\pgfsetbuttcap%
\pgfsetroundjoin%
\pgfsetlinewidth{0.752812pt}%
\definecolor{currentstroke}{rgb}{0.150000,0.150000,0.150000}%
\pgfsetstrokecolor{currentstroke}%
\pgfsetstrokeopacity{0.450000}%
\pgfsetdash{}{0pt}%
\pgfpathmoveto{\pgfqpoint{2.947590in}{2.274694in}}%
\pgfpathlineto{\pgfqpoint{3.072800in}{2.274694in}}%
\pgfusepath{stroke}%
\end{pgfscope}%
\begin{pgfscope}%
\pgfpathrectangle{\pgfqpoint{0.550713in}{2.154633in}}{\pgfqpoint{4.791200in}{1.332187in}}%
\pgfusepath{clip}%
\pgfsetbuttcap%
\pgfsetroundjoin%
\pgfsetlinewidth{0.752812pt}%
\definecolor{currentstroke}{rgb}{0.150000,0.150000,0.150000}%
\pgfsetstrokecolor{currentstroke}%
\pgfsetstrokeopacity{0.450000}%
\pgfsetdash{}{0pt}%
\pgfpathmoveto{\pgfqpoint{3.139238in}{2.372558in}}%
\pgfpathlineto{\pgfqpoint{3.264448in}{2.372558in}}%
\pgfusepath{stroke}%
\end{pgfscope}%
\begin{pgfscope}%
\pgfpathrectangle{\pgfqpoint{0.550713in}{2.154633in}}{\pgfqpoint{4.791200in}{1.332187in}}%
\pgfusepath{clip}%
\pgfsetbuttcap%
\pgfsetroundjoin%
\pgfsetlinewidth{0.752812pt}%
\definecolor{currentstroke}{rgb}{0.150000,0.150000,0.150000}%
\pgfsetstrokecolor{currentstroke}%
\pgfsetstrokeopacity{0.450000}%
\pgfsetdash{}{0pt}%
\pgfpathmoveto{\pgfqpoint{3.267003in}{2.296313in}}%
\pgfpathlineto{\pgfqpoint{3.392213in}{2.296313in}}%
\pgfusepath{stroke}%
\end{pgfscope}%
\begin{pgfscope}%
\pgfpathrectangle{\pgfqpoint{0.550713in}{2.154633in}}{\pgfqpoint{4.791200in}{1.332187in}}%
\pgfusepath{clip}%
\pgfsetbuttcap%
\pgfsetroundjoin%
\pgfsetlinewidth{0.752812pt}%
\definecolor{currentstroke}{rgb}{0.150000,0.150000,0.150000}%
\pgfsetstrokecolor{currentstroke}%
\pgfsetstrokeopacity{0.450000}%
\pgfsetdash{}{0pt}%
\pgfpathmoveto{\pgfqpoint{3.458651in}{2.322069in}}%
\pgfpathlineto{\pgfqpoint{3.583861in}{2.322069in}}%
\pgfusepath{stroke}%
\end{pgfscope}%
\begin{pgfscope}%
\pgfpathrectangle{\pgfqpoint{0.550713in}{2.154633in}}{\pgfqpoint{4.791200in}{1.332187in}}%
\pgfusepath{clip}%
\pgfsetbuttcap%
\pgfsetroundjoin%
\pgfsetlinewidth{0.752812pt}%
\definecolor{currentstroke}{rgb}{0.150000,0.150000,0.150000}%
\pgfsetstrokecolor{currentstroke}%
\pgfsetstrokeopacity{0.450000}%
\pgfsetdash{}{0pt}%
\pgfpathmoveto{\pgfqpoint{3.586417in}{2.329613in}}%
\pgfpathlineto{\pgfqpoint{3.711627in}{2.329613in}}%
\pgfusepath{stroke}%
\end{pgfscope}%
\begin{pgfscope}%
\pgfpathrectangle{\pgfqpoint{0.550713in}{2.154633in}}{\pgfqpoint{4.791200in}{1.332187in}}%
\pgfusepath{clip}%
\pgfsetbuttcap%
\pgfsetroundjoin%
\pgfsetlinewidth{0.752812pt}%
\definecolor{currentstroke}{rgb}{0.150000,0.150000,0.150000}%
\pgfsetstrokecolor{currentstroke}%
\pgfsetstrokeopacity{0.450000}%
\pgfsetdash{}{0pt}%
\pgfpathmoveto{\pgfqpoint{3.778065in}{2.327450in}}%
\pgfpathlineto{\pgfqpoint{3.903275in}{2.327450in}}%
\pgfusepath{stroke}%
\end{pgfscope}%
\begin{pgfscope}%
\pgfpathrectangle{\pgfqpoint{0.550713in}{2.154633in}}{\pgfqpoint{4.791200in}{1.332187in}}%
\pgfusepath{clip}%
\pgfsetbuttcap%
\pgfsetroundjoin%
\pgfsetlinewidth{0.752812pt}%
\definecolor{currentstroke}{rgb}{0.150000,0.150000,0.150000}%
\pgfsetstrokecolor{currentstroke}%
\pgfsetstrokeopacity{0.450000}%
\pgfsetdash{}{0pt}%
\pgfpathmoveto{\pgfqpoint{3.905830in}{2.336166in}}%
\pgfpathlineto{\pgfqpoint{4.031040in}{2.336166in}}%
\pgfusepath{stroke}%
\end{pgfscope}%
\begin{pgfscope}%
\pgfpathrectangle{\pgfqpoint{0.550713in}{2.154633in}}{\pgfqpoint{4.791200in}{1.332187in}}%
\pgfusepath{clip}%
\pgfsetbuttcap%
\pgfsetroundjoin%
\pgfsetlinewidth{0.752812pt}%
\definecolor{currentstroke}{rgb}{0.150000,0.150000,0.150000}%
\pgfsetstrokecolor{currentstroke}%
\pgfsetstrokeopacity{0.450000}%
\pgfsetdash{}{0pt}%
\pgfpathmoveto{\pgfqpoint{4.097478in}{2.390762in}}%
\pgfpathlineto{\pgfqpoint{4.222688in}{2.390762in}}%
\pgfusepath{stroke}%
\end{pgfscope}%
\begin{pgfscope}%
\pgfpathrectangle{\pgfqpoint{0.550713in}{2.154633in}}{\pgfqpoint{4.791200in}{1.332187in}}%
\pgfusepath{clip}%
\pgfsetbuttcap%
\pgfsetroundjoin%
\pgfsetlinewidth{0.752812pt}%
\definecolor{currentstroke}{rgb}{0.150000,0.150000,0.150000}%
\pgfsetstrokecolor{currentstroke}%
\pgfsetstrokeopacity{0.450000}%
\pgfsetdash{}{0pt}%
\pgfpathmoveto{\pgfqpoint{4.225243in}{2.358257in}}%
\pgfpathlineto{\pgfqpoint{4.350453in}{2.358257in}}%
\pgfusepath{stroke}%
\end{pgfscope}%
\begin{pgfscope}%
\pgfpathrectangle{\pgfqpoint{0.550713in}{2.154633in}}{\pgfqpoint{4.791200in}{1.332187in}}%
\pgfusepath{clip}%
\pgfsetbuttcap%
\pgfsetroundjoin%
\pgfsetlinewidth{0.752812pt}%
\definecolor{currentstroke}{rgb}{0.150000,0.150000,0.150000}%
\pgfsetstrokecolor{currentstroke}%
\pgfsetstrokeopacity{0.450000}%
\pgfsetdash{}{0pt}%
\pgfpathmoveto{\pgfqpoint{4.416891in}{2.419235in}}%
\pgfpathlineto{\pgfqpoint{4.542101in}{2.419235in}}%
\pgfusepath{stroke}%
\end{pgfscope}%
\begin{pgfscope}%
\pgfpathrectangle{\pgfqpoint{0.550713in}{2.154633in}}{\pgfqpoint{4.791200in}{1.332187in}}%
\pgfusepath{clip}%
\pgfsetbuttcap%
\pgfsetroundjoin%
\pgfsetlinewidth{0.752812pt}%
\definecolor{currentstroke}{rgb}{0.150000,0.150000,0.150000}%
\pgfsetstrokecolor{currentstroke}%
\pgfsetstrokeopacity{0.450000}%
\pgfsetdash{}{0pt}%
\pgfpathmoveto{\pgfqpoint{4.544657in}{2.378461in}}%
\pgfpathlineto{\pgfqpoint{4.669867in}{2.378461in}}%
\pgfusepath{stroke}%
\end{pgfscope}%
\begin{pgfscope}%
\pgfpathrectangle{\pgfqpoint{0.550713in}{2.154633in}}{\pgfqpoint{4.791200in}{1.332187in}}%
\pgfusepath{clip}%
\pgfsetbuttcap%
\pgfsetroundjoin%
\pgfsetlinewidth{0.752812pt}%
\definecolor{currentstroke}{rgb}{0.150000,0.150000,0.150000}%
\pgfsetstrokecolor{currentstroke}%
\pgfsetstrokeopacity{0.450000}%
\pgfsetdash{}{0pt}%
\pgfpathmoveto{\pgfqpoint{4.736305in}{2.602214in}}%
\pgfpathlineto{\pgfqpoint{4.861515in}{2.602214in}}%
\pgfusepath{stroke}%
\end{pgfscope}%
\begin{pgfscope}%
\pgfpathrectangle{\pgfqpoint{0.550713in}{2.154633in}}{\pgfqpoint{4.791200in}{1.332187in}}%
\pgfusepath{clip}%
\pgfsetbuttcap%
\pgfsetroundjoin%
\pgfsetlinewidth{0.752812pt}%
\definecolor{currentstroke}{rgb}{0.150000,0.150000,0.150000}%
\pgfsetstrokecolor{currentstroke}%
\pgfsetstrokeopacity{0.450000}%
\pgfsetdash{}{0pt}%
\pgfpathmoveto{\pgfqpoint{4.864070in}{2.456454in}}%
\pgfpathlineto{\pgfqpoint{4.989280in}{2.456454in}}%
\pgfusepath{stroke}%
\end{pgfscope}%
\begin{pgfscope}%
\pgfpathrectangle{\pgfqpoint{0.550713in}{2.154633in}}{\pgfqpoint{4.791200in}{1.332187in}}%
\pgfusepath{clip}%
\pgfsetbuttcap%
\pgfsetroundjoin%
\pgfsetlinewidth{0.752812pt}%
\definecolor{currentstroke}{rgb}{0.150000,0.150000,0.150000}%
\pgfsetstrokecolor{currentstroke}%
\pgfsetstrokeopacity{0.450000}%
\pgfsetdash{}{0pt}%
\pgfpathmoveto{\pgfqpoint{5.055718in}{2.789010in}}%
\pgfpathlineto{\pgfqpoint{5.180928in}{2.789010in}}%
\pgfusepath{stroke}%
\end{pgfscope}%
\begin{pgfscope}%
\pgfpathrectangle{\pgfqpoint{0.550713in}{2.154633in}}{\pgfqpoint{4.791200in}{1.332187in}}%
\pgfusepath{clip}%
\pgfsetbuttcap%
\pgfsetroundjoin%
\pgfsetlinewidth{0.752812pt}%
\definecolor{currentstroke}{rgb}{0.150000,0.150000,0.150000}%
\pgfsetstrokecolor{currentstroke}%
\pgfsetstrokeopacity{0.450000}%
\pgfsetdash{}{0pt}%
\pgfpathmoveto{\pgfqpoint{5.183483in}{2.474929in}}%
\pgfpathlineto{\pgfqpoint{5.308693in}{2.474929in}}%
\pgfusepath{stroke}%
\end{pgfscope}%
\begin{pgfscope}%
\pgfpathrectangle{\pgfqpoint{0.550713in}{2.154633in}}{\pgfqpoint{4.791200in}{1.332187in}}%
\pgfusepath{clip}%
\pgfsetbuttcap%
\pgfsetroundjoin%
\pgfsetlinewidth{0.853187pt}%
\definecolor{currentstroke}{rgb}{0.341176,0.670588,0.152941}%
\pgfsetstrokecolor{currentstroke}%
\pgfsetdash{{3.145000pt}{1.360000pt}}{0.000000pt}%
\pgfpathmoveto{\pgfqpoint{0.774302in}{2.285250in}}%
\pgfpathlineto{\pgfqpoint{1.093715in}{2.244604in}}%
\pgfpathlineto{\pgfqpoint{1.413129in}{2.236403in}}%
\pgfpathlineto{\pgfqpoint{1.732542in}{2.246609in}}%
\pgfpathlineto{\pgfqpoint{2.051955in}{2.254419in}}%
\pgfpathlineto{\pgfqpoint{2.371368in}{2.244104in}}%
\pgfpathlineto{\pgfqpoint{2.690782in}{2.262153in}}%
\pgfpathlineto{\pgfqpoint{3.010195in}{2.286870in}}%
\pgfpathlineto{\pgfqpoint{3.329608in}{2.285503in}}%
\pgfpathlineto{\pgfqpoint{3.649022in}{2.318635in}}%
\pgfpathlineto{\pgfqpoint{3.968435in}{2.322557in}}%
\pgfpathlineto{\pgfqpoint{4.287848in}{2.341592in}}%
\pgfpathlineto{\pgfqpoint{4.607262in}{2.365920in}}%
\pgfpathlineto{\pgfqpoint{4.926675in}{2.445742in}}%
\pgfpathlineto{\pgfqpoint{5.246088in}{2.473375in}}%
\pgfusepath{stroke}%
\end{pgfscope}%
\begin{pgfscope}%
\pgfpathrectangle{\pgfqpoint{0.550713in}{2.154633in}}{\pgfqpoint{4.791200in}{1.332187in}}%
\pgfusepath{clip}%
\pgfsetbuttcap%
\pgfsetroundjoin%
\definecolor{currentfill}{rgb}{0.341176,0.670588,0.152941}%
\pgfsetfillcolor{currentfill}%
\pgfsetlinewidth{0.752812pt}%
\definecolor{currentstroke}{rgb}{1.000000,1.000000,1.000000}%
\pgfsetstrokecolor{currentstroke}%
\pgfsetdash{}{0pt}%
\pgfsys@defobject{currentmarker}{\pgfqpoint{-0.027778in}{-0.027778in}}{\pgfqpoint{0.027778in}{0.027778in}}{%
\pgfpathmoveto{\pgfqpoint{0.000000in}{-0.027778in}}%
\pgfpathcurveto{\pgfqpoint{0.007367in}{-0.027778in}}{\pgfqpoint{0.014433in}{-0.024851in}}{\pgfqpoint{0.019642in}{-0.019642in}}%
\pgfpathcurveto{\pgfqpoint{0.024851in}{-0.014433in}}{\pgfqpoint{0.027778in}{-0.007367in}}{\pgfqpoint{0.027778in}{0.000000in}}%
\pgfpathcurveto{\pgfqpoint{0.027778in}{0.007367in}}{\pgfqpoint{0.024851in}{0.014433in}}{\pgfqpoint{0.019642in}{0.019642in}}%
\pgfpathcurveto{\pgfqpoint{0.014433in}{0.024851in}}{\pgfqpoint{0.007367in}{0.027778in}}{\pgfqpoint{0.000000in}{0.027778in}}%
\pgfpathcurveto{\pgfqpoint{-0.007367in}{0.027778in}}{\pgfqpoint{-0.014433in}{0.024851in}}{\pgfqpoint{-0.019642in}{0.019642in}}%
\pgfpathcurveto{\pgfqpoint{-0.024851in}{0.014433in}}{\pgfqpoint{-0.027778in}{0.007367in}}{\pgfqpoint{-0.027778in}{0.000000in}}%
\pgfpathcurveto{\pgfqpoint{-0.027778in}{-0.007367in}}{\pgfqpoint{-0.024851in}{-0.014433in}}{\pgfqpoint{-0.019642in}{-0.019642in}}%
\pgfpathcurveto{\pgfqpoint{-0.014433in}{-0.024851in}}{\pgfqpoint{-0.007367in}{-0.027778in}}{\pgfqpoint{0.000000in}{-0.027778in}}%
\pgfpathclose%
\pgfusepath{stroke,fill}%
}%
\begin{pgfscope}%
\pgfsys@transformshift{0.774302in}{2.285250in}%
\pgfsys@useobject{currentmarker}{}%
\end{pgfscope}%
\begin{pgfscope}%
\pgfsys@transformshift{1.093715in}{2.244604in}%
\pgfsys@useobject{currentmarker}{}%
\end{pgfscope}%
\begin{pgfscope}%
\pgfsys@transformshift{1.413129in}{2.236403in}%
\pgfsys@useobject{currentmarker}{}%
\end{pgfscope}%
\begin{pgfscope}%
\pgfsys@transformshift{1.732542in}{2.246609in}%
\pgfsys@useobject{currentmarker}{}%
\end{pgfscope}%
\begin{pgfscope}%
\pgfsys@transformshift{2.051955in}{2.254419in}%
\pgfsys@useobject{currentmarker}{}%
\end{pgfscope}%
\begin{pgfscope}%
\pgfsys@transformshift{2.371368in}{2.244104in}%
\pgfsys@useobject{currentmarker}{}%
\end{pgfscope}%
\begin{pgfscope}%
\pgfsys@transformshift{2.690782in}{2.262153in}%
\pgfsys@useobject{currentmarker}{}%
\end{pgfscope}%
\begin{pgfscope}%
\pgfsys@transformshift{3.010195in}{2.286870in}%
\pgfsys@useobject{currentmarker}{}%
\end{pgfscope}%
\begin{pgfscope}%
\pgfsys@transformshift{3.329608in}{2.285503in}%
\pgfsys@useobject{currentmarker}{}%
\end{pgfscope}%
\begin{pgfscope}%
\pgfsys@transformshift{3.649022in}{2.318635in}%
\pgfsys@useobject{currentmarker}{}%
\end{pgfscope}%
\begin{pgfscope}%
\pgfsys@transformshift{3.968435in}{2.322557in}%
\pgfsys@useobject{currentmarker}{}%
\end{pgfscope}%
\begin{pgfscope}%
\pgfsys@transformshift{4.287848in}{2.341592in}%
\pgfsys@useobject{currentmarker}{}%
\end{pgfscope}%
\begin{pgfscope}%
\pgfsys@transformshift{4.607262in}{2.365920in}%
\pgfsys@useobject{currentmarker}{}%
\end{pgfscope}%
\begin{pgfscope}%
\pgfsys@transformshift{4.926675in}{2.445742in}%
\pgfsys@useobject{currentmarker}{}%
\end{pgfscope}%
\begin{pgfscope}%
\pgfsys@transformshift{5.246088in}{2.473375in}%
\pgfsys@useobject{currentmarker}{}%
\end{pgfscope}%
\end{pgfscope}%
\begin{pgfscope}%
\pgfpathrectangle{\pgfqpoint{0.550713in}{2.154633in}}{\pgfqpoint{4.791200in}{1.332187in}}%
\pgfusepath{clip}%
\pgfsetbuttcap%
\pgfsetroundjoin%
\pgfsetlinewidth{0.853187pt}%
\definecolor{currentstroke}{rgb}{0.000000,0.329412,0.623529}%
\pgfsetstrokecolor{currentstroke}%
\pgfsetdash{{3.145000pt}{1.360000pt}}{0.000000pt}%
\pgfpathmoveto{\pgfqpoint{0.646537in}{2.529479in}}%
\pgfpathlineto{\pgfqpoint{0.965950in}{2.388785in}}%
\pgfpathlineto{\pgfqpoint{1.285363in}{2.399859in}}%
\pgfpathlineto{\pgfqpoint{1.604777in}{2.310404in}}%
\pgfpathlineto{\pgfqpoint{1.924190in}{2.332716in}}%
\pgfpathlineto{\pgfqpoint{2.243603in}{2.388287in}}%
\pgfpathlineto{\pgfqpoint{2.563016in}{2.392417in}}%
\pgfpathlineto{\pgfqpoint{2.882430in}{2.394115in}}%
\pgfpathlineto{\pgfqpoint{3.201843in}{2.363938in}}%
\pgfpathlineto{\pgfqpoint{3.521256in}{2.334375in}}%
\pgfpathlineto{\pgfqpoint{3.840670in}{2.389709in}}%
\pgfpathlineto{\pgfqpoint{4.160083in}{2.388828in}}%
\pgfpathlineto{\pgfqpoint{4.479496in}{2.401899in}}%
\pgfpathlineto{\pgfqpoint{4.798910in}{2.569068in}}%
\pgfpathlineto{\pgfqpoint{5.118323in}{2.725310in}}%
\pgfusepath{stroke}%
\end{pgfscope}%
\begin{pgfscope}%
\pgfpathrectangle{\pgfqpoint{0.550713in}{2.154633in}}{\pgfqpoint{4.791200in}{1.332187in}}%
\pgfusepath{clip}%
\pgfsetbuttcap%
\pgfsetroundjoin%
\definecolor{currentfill}{rgb}{0.000000,0.329412,0.623529}%
\pgfsetfillcolor{currentfill}%
\pgfsetlinewidth{0.752812pt}%
\definecolor{currentstroke}{rgb}{1.000000,1.000000,1.000000}%
\pgfsetstrokecolor{currentstroke}%
\pgfsetdash{}{0pt}%
\pgfsys@defobject{currentmarker}{\pgfqpoint{-0.027778in}{-0.027778in}}{\pgfqpoint{0.027778in}{0.027778in}}{%
\pgfpathmoveto{\pgfqpoint{0.000000in}{-0.027778in}}%
\pgfpathcurveto{\pgfqpoint{0.007367in}{-0.027778in}}{\pgfqpoint{0.014433in}{-0.024851in}}{\pgfqpoint{0.019642in}{-0.019642in}}%
\pgfpathcurveto{\pgfqpoint{0.024851in}{-0.014433in}}{\pgfqpoint{0.027778in}{-0.007367in}}{\pgfqpoint{0.027778in}{0.000000in}}%
\pgfpathcurveto{\pgfqpoint{0.027778in}{0.007367in}}{\pgfqpoint{0.024851in}{0.014433in}}{\pgfqpoint{0.019642in}{0.019642in}}%
\pgfpathcurveto{\pgfqpoint{0.014433in}{0.024851in}}{\pgfqpoint{0.007367in}{0.027778in}}{\pgfqpoint{0.000000in}{0.027778in}}%
\pgfpathcurveto{\pgfqpoint{-0.007367in}{0.027778in}}{\pgfqpoint{-0.014433in}{0.024851in}}{\pgfqpoint{-0.019642in}{0.019642in}}%
\pgfpathcurveto{\pgfqpoint{-0.024851in}{0.014433in}}{\pgfqpoint{-0.027778in}{0.007367in}}{\pgfqpoint{-0.027778in}{0.000000in}}%
\pgfpathcurveto{\pgfqpoint{-0.027778in}{-0.007367in}}{\pgfqpoint{-0.024851in}{-0.014433in}}{\pgfqpoint{-0.019642in}{-0.019642in}}%
\pgfpathcurveto{\pgfqpoint{-0.014433in}{-0.024851in}}{\pgfqpoint{-0.007367in}{-0.027778in}}{\pgfqpoint{0.000000in}{-0.027778in}}%
\pgfpathclose%
\pgfusepath{stroke,fill}%
}%
\begin{pgfscope}%
\pgfsys@transformshift{0.646537in}{2.529479in}%
\pgfsys@useobject{currentmarker}{}%
\end{pgfscope}%
\begin{pgfscope}%
\pgfsys@transformshift{0.965950in}{2.388785in}%
\pgfsys@useobject{currentmarker}{}%
\end{pgfscope}%
\begin{pgfscope}%
\pgfsys@transformshift{1.285363in}{2.399859in}%
\pgfsys@useobject{currentmarker}{}%
\end{pgfscope}%
\begin{pgfscope}%
\pgfsys@transformshift{1.604777in}{2.310404in}%
\pgfsys@useobject{currentmarker}{}%
\end{pgfscope}%
\begin{pgfscope}%
\pgfsys@transformshift{1.924190in}{2.332716in}%
\pgfsys@useobject{currentmarker}{}%
\end{pgfscope}%
\begin{pgfscope}%
\pgfsys@transformshift{2.243603in}{2.388287in}%
\pgfsys@useobject{currentmarker}{}%
\end{pgfscope}%
\begin{pgfscope}%
\pgfsys@transformshift{2.563016in}{2.392417in}%
\pgfsys@useobject{currentmarker}{}%
\end{pgfscope}%
\begin{pgfscope}%
\pgfsys@transformshift{2.882430in}{2.394115in}%
\pgfsys@useobject{currentmarker}{}%
\end{pgfscope}%
\begin{pgfscope}%
\pgfsys@transformshift{3.201843in}{2.363938in}%
\pgfsys@useobject{currentmarker}{}%
\end{pgfscope}%
\begin{pgfscope}%
\pgfsys@transformshift{3.521256in}{2.334375in}%
\pgfsys@useobject{currentmarker}{}%
\end{pgfscope}%
\begin{pgfscope}%
\pgfsys@transformshift{3.840670in}{2.389709in}%
\pgfsys@useobject{currentmarker}{}%
\end{pgfscope}%
\begin{pgfscope}%
\pgfsys@transformshift{4.160083in}{2.388828in}%
\pgfsys@useobject{currentmarker}{}%
\end{pgfscope}%
\begin{pgfscope}%
\pgfsys@transformshift{4.479496in}{2.401899in}%
\pgfsys@useobject{currentmarker}{}%
\end{pgfscope}%
\begin{pgfscope}%
\pgfsys@transformshift{4.798910in}{2.569068in}%
\pgfsys@useobject{currentmarker}{}%
\end{pgfscope}%
\begin{pgfscope}%
\pgfsys@transformshift{5.118323in}{2.725310in}%
\pgfsys@useobject{currentmarker}{}%
\end{pgfscope}%
\end{pgfscope}%
\begin{pgfscope}%
\pgfsetrectcap%
\pgfsetmiterjoin%
\pgfsetlinewidth{0.803000pt}%
\definecolor{currentstroke}{rgb}{0.000000,0.000000,0.000000}%
\pgfsetstrokecolor{currentstroke}%
\pgfsetdash{}{0pt}%
\pgfpathmoveto{\pgfqpoint{0.550713in}{2.154633in}}%
\pgfpathlineto{\pgfqpoint{0.550713in}{3.486820in}}%
\pgfusepath{stroke}%
\end{pgfscope}%
\begin{pgfscope}%
\pgfsetrectcap%
\pgfsetmiterjoin%
\pgfsetlinewidth{0.803000pt}%
\definecolor{currentstroke}{rgb}{0.000000,0.000000,0.000000}%
\pgfsetstrokecolor{currentstroke}%
\pgfsetdash{}{0pt}%
\pgfpathmoveto{\pgfqpoint{5.341912in}{2.154633in}}%
\pgfpathlineto{\pgfqpoint{5.341912in}{3.486820in}}%
\pgfusepath{stroke}%
\end{pgfscope}%
\begin{pgfscope}%
\pgfsetrectcap%
\pgfsetmiterjoin%
\pgfsetlinewidth{0.803000pt}%
\definecolor{currentstroke}{rgb}{0.000000,0.000000,0.000000}%
\pgfsetstrokecolor{currentstroke}%
\pgfsetdash{}{0pt}%
\pgfpathmoveto{\pgfqpoint{0.550713in}{2.154633in}}%
\pgfpathlineto{\pgfqpoint{5.341912in}{2.154633in}}%
\pgfusepath{stroke}%
\end{pgfscope}%
\begin{pgfscope}%
\pgfsetrectcap%
\pgfsetmiterjoin%
\pgfsetlinewidth{0.803000pt}%
\definecolor{currentstroke}{rgb}{0.000000,0.000000,0.000000}%
\pgfsetstrokecolor{currentstroke}%
\pgfsetdash{}{0pt}%
\pgfpathmoveto{\pgfqpoint{0.550713in}{3.486820in}}%
\pgfpathlineto{\pgfqpoint{5.341912in}{3.486820in}}%
\pgfusepath{stroke}%
\end{pgfscope}%
\begin{pgfscope}%
\pgfsetbuttcap%
\pgfsetmiterjoin%
\definecolor{currentfill}{rgb}{1.000000,1.000000,1.000000}%
\pgfsetfillcolor{currentfill}%
\pgfsetlinewidth{1.003750pt}%
\definecolor{currentstroke}{rgb}{1.000000,1.000000,1.000000}%
\pgfsetstrokecolor{currentstroke}%
\pgfsetdash{}{0pt}%
\pgfpathmoveto{\pgfqpoint{0.654864in}{3.201477in}}%
\pgfpathlineto{\pgfqpoint{1.557364in}{3.201477in}}%
\pgfpathlineto{\pgfqpoint{1.557364in}{3.439115in}}%
\pgfpathlineto{\pgfqpoint{0.654864in}{3.439115in}}%
\pgfpathclose%
\pgfusepath{stroke,fill}%
\end{pgfscope}%
\begin{pgfscope}%
\definecolor{textcolor}{rgb}{0.000000,0.000000,0.000000}%
\pgfsetstrokecolor{textcolor}%
\pgfsetfillcolor{textcolor}%
\pgftext[x=0.710419in,y=3.320296in,left,]{\color{textcolor}\rmfamily\fontsize{10.000000}{12.000000}\selectfont UI forgetting}%
\end{pgfscope}%
\begin{pgfscope}%
\pgfsetbuttcap%
\pgfsetmiterjoin%
\definecolor{currentfill}{rgb}{1.000000,1.000000,1.000000}%
\pgfsetfillcolor{currentfill}%
\pgfsetlinewidth{1.003750pt}%
\definecolor{currentstroke}{rgb}{1.000000,1.000000,1.000000}%
\pgfsetstrokecolor{currentstroke}%
\pgfsetdash{}{0pt}%
\pgfpathmoveto{\pgfqpoint{2.551382in}{2.795709in}}%
\pgfpathlineto{\pgfqpoint{3.341243in}{2.795709in}}%
\pgfpathquadraticcurveto{\pgfqpoint{3.369021in}{2.795709in}}{\pgfqpoint{3.369021in}{2.823487in}}%
\pgfpathlineto{\pgfqpoint{3.369021in}{3.389597in}}%
\pgfpathquadraticcurveto{\pgfqpoint{3.369021in}{3.417375in}}{\pgfqpoint{3.341243in}{3.417375in}}%
\pgfpathlineto{\pgfqpoint{2.551382in}{3.417375in}}%
\pgfpathquadraticcurveto{\pgfqpoint{2.523604in}{3.417375in}}{\pgfqpoint{2.523604in}{3.389597in}}%
\pgfpathlineto{\pgfqpoint{2.523604in}{2.823487in}}%
\pgfpathquadraticcurveto{\pgfqpoint{2.523604in}{2.795709in}}{\pgfqpoint{2.551382in}{2.795709in}}%
\pgfpathclose%
\pgfusepath{stroke,fill}%
\end{pgfscope}%
\begin{pgfscope}%
\definecolor{textcolor}{rgb}{0.000000,0.000000,0.000000}%
\pgfsetstrokecolor{textcolor}%
\pgfsetfillcolor{textcolor}%
\pgftext[x=2.579160in,y=3.265431in,left,base]{\color{textcolor}\rmfamily\fontsize{10.000000}{12.000000}\selectfont Constrained}%
\end{pgfscope}%
\begin{pgfscope}%
\pgfsetbuttcap%
\pgfsetmiterjoin%
\definecolor{currentfill}{rgb}{0.077941,0.325000,0.545588}%
\pgfsetfillcolor{currentfill}%
\pgfsetlinewidth{0.376406pt}%
\definecolor{currentstroke}{rgb}{0.188235,0.188235,0.188235}%
\pgfsetstrokecolor{currentstroke}%
\pgfsetdash{}{0pt}%
\pgfpathmoveto{\pgfqpoint{2.600062in}{3.071820in}}%
\pgfpathlineto{\pgfqpoint{2.877840in}{3.071820in}}%
\pgfpathlineto{\pgfqpoint{2.877840in}{3.169042in}}%
\pgfpathlineto{\pgfqpoint{2.600062in}{3.169042in}}%
\pgfpathclose%
\pgfusepath{stroke,fill}%
\end{pgfscope}%
\begin{pgfscope}%
\definecolor{textcolor}{rgb}{0.000000,0.000000,0.000000}%
\pgfsetstrokecolor{textcolor}%
\pgfsetfillcolor{textcolor}%
\pgftext[x=2.988951in,y=3.071820in,left,base]{\color{textcolor}\rmfamily\fontsize{10.000000}{12.000000}\selectfont False}%
\end{pgfscope}%
\begin{pgfscope}%
\pgfsetbuttcap%
\pgfsetmiterjoin%
\definecolor{currentfill}{rgb}{0.358824,0.605882,0.217647}%
\pgfsetfillcolor{currentfill}%
\pgfsetlinewidth{0.376406pt}%
\definecolor{currentstroke}{rgb}{0.188235,0.188235,0.188235}%
\pgfsetstrokecolor{currentstroke}%
\pgfsetdash{}{0pt}%
\pgfpathmoveto{\pgfqpoint{2.600062in}{2.878209in}}%
\pgfpathlineto{\pgfqpoint{2.877840in}{2.878209in}}%
\pgfpathlineto{\pgfqpoint{2.877840in}{2.975431in}}%
\pgfpathlineto{\pgfqpoint{2.600062in}{2.975431in}}%
\pgfpathclose%
\pgfusepath{stroke,fill}%
\end{pgfscope}%
\begin{pgfscope}%
\definecolor{textcolor}{rgb}{0.000000,0.000000,0.000000}%
\pgfsetstrokecolor{textcolor}%
\pgfsetfillcolor{textcolor}%
\pgftext[x=2.988951in,y=2.878209in,left,base]{\color{textcolor}\rmfamily\fontsize{10.000000}{12.000000}\selectfont True}%
\end{pgfscope}%
\begin{pgfscope}%
\pgfsetbuttcap%
\pgfsetmiterjoin%
\definecolor{currentfill}{rgb}{1.000000,1.000000,1.000000}%
\pgfsetfillcolor{currentfill}%
\pgfsetlinewidth{0.000000pt}%
\definecolor{currentstroke}{rgb}{0.000000,0.000000,0.000000}%
\pgfsetstrokecolor{currentstroke}%
\pgfsetstrokeopacity{0.000000}%
\pgfsetdash{}{0pt}%
\pgfpathmoveto{\pgfqpoint{0.550713in}{0.689227in}}%
\pgfpathlineto{\pgfqpoint{5.341912in}{0.689227in}}%
\pgfpathlineto{\pgfqpoint{5.341912in}{2.021414in}}%
\pgfpathlineto{\pgfqpoint{0.550713in}{2.021414in}}%
\pgfpathclose%
\pgfusepath{fill}%
\end{pgfscope}%
\begin{pgfscope}%
\pgfsetbuttcap%
\pgfsetroundjoin%
\definecolor{currentfill}{rgb}{0.000000,0.000000,0.000000}%
\pgfsetfillcolor{currentfill}%
\pgfsetlinewidth{0.803000pt}%
\definecolor{currentstroke}{rgb}{0.000000,0.000000,0.000000}%
\pgfsetstrokecolor{currentstroke}%
\pgfsetdash{}{0pt}%
\pgfsys@defobject{currentmarker}{\pgfqpoint{0.000000in}{-0.048611in}}{\pgfqpoint{0.000000in}{0.000000in}}{%
\pgfpathmoveto{\pgfqpoint{0.000000in}{0.000000in}}%
\pgfpathlineto{\pgfqpoint{0.000000in}{-0.048611in}}%
\pgfusepath{stroke,fill}%
}%
\begin{pgfscope}%
\pgfsys@transformshift{0.710419in}{0.689227in}%
\pgfsys@useobject{currentmarker}{}%
\end{pgfscope}%
\end{pgfscope}%
\begin{pgfscope}%
\definecolor{textcolor}{rgb}{0.000000,0.000000,0.000000}%
\pgfsetstrokecolor{textcolor}%
\pgfsetfillcolor{textcolor}%
\pgftext[x=0.623111in, y=0.300127in, left, base,rotate=45.000000]{\color{textcolor}\rmfamily\fontsize{10.000000}{12.000000}\selectfont 0.005}%
\end{pgfscope}%
\begin{pgfscope}%
\pgfsetbuttcap%
\pgfsetroundjoin%
\definecolor{currentfill}{rgb}{0.000000,0.000000,0.000000}%
\pgfsetfillcolor{currentfill}%
\pgfsetlinewidth{0.803000pt}%
\definecolor{currentstroke}{rgb}{0.000000,0.000000,0.000000}%
\pgfsetstrokecolor{currentstroke}%
\pgfsetdash{}{0pt}%
\pgfsys@defobject{currentmarker}{\pgfqpoint{0.000000in}{-0.048611in}}{\pgfqpoint{0.000000in}{0.000000in}}{%
\pgfpathmoveto{\pgfqpoint{0.000000in}{0.000000in}}%
\pgfpathlineto{\pgfqpoint{0.000000in}{-0.048611in}}%
\pgfusepath{stroke,fill}%
}%
\begin{pgfscope}%
\pgfsys@transformshift{1.029833in}{0.689227in}%
\pgfsys@useobject{currentmarker}{}%
\end{pgfscope}%
\end{pgfscope}%
\begin{pgfscope}%
\definecolor{textcolor}{rgb}{0.000000,0.000000,0.000000}%
\pgfsetstrokecolor{textcolor}%
\pgfsetfillcolor{textcolor}%
\pgftext[x=0.942525in, y=0.300127in, left, base,rotate=45.000000]{\color{textcolor}\rmfamily\fontsize{10.000000}{12.000000}\selectfont 0.007}%
\end{pgfscope}%
\begin{pgfscope}%
\pgfsetbuttcap%
\pgfsetroundjoin%
\definecolor{currentfill}{rgb}{0.000000,0.000000,0.000000}%
\pgfsetfillcolor{currentfill}%
\pgfsetlinewidth{0.803000pt}%
\definecolor{currentstroke}{rgb}{0.000000,0.000000,0.000000}%
\pgfsetstrokecolor{currentstroke}%
\pgfsetdash{}{0pt}%
\pgfsys@defobject{currentmarker}{\pgfqpoint{0.000000in}{-0.048611in}}{\pgfqpoint{0.000000in}{0.000000in}}{%
\pgfpathmoveto{\pgfqpoint{0.000000in}{0.000000in}}%
\pgfpathlineto{\pgfqpoint{0.000000in}{-0.048611in}}%
\pgfusepath{stroke,fill}%
}%
\begin{pgfscope}%
\pgfsys@transformshift{1.349246in}{0.689227in}%
\pgfsys@useobject{currentmarker}{}%
\end{pgfscope}%
\end{pgfscope}%
\begin{pgfscope}%
\definecolor{textcolor}{rgb}{0.000000,0.000000,0.000000}%
\pgfsetstrokecolor{textcolor}%
\pgfsetfillcolor{textcolor}%
\pgftext[x=1.261938in, y=0.300127in, left, base,rotate=45.000000]{\color{textcolor}\rmfamily\fontsize{10.000000}{12.000000}\selectfont 0.009}%
\end{pgfscope}%
\begin{pgfscope}%
\pgfsetbuttcap%
\pgfsetroundjoin%
\definecolor{currentfill}{rgb}{0.000000,0.000000,0.000000}%
\pgfsetfillcolor{currentfill}%
\pgfsetlinewidth{0.803000pt}%
\definecolor{currentstroke}{rgb}{0.000000,0.000000,0.000000}%
\pgfsetstrokecolor{currentstroke}%
\pgfsetdash{}{0pt}%
\pgfsys@defobject{currentmarker}{\pgfqpoint{0.000000in}{-0.048611in}}{\pgfqpoint{0.000000in}{0.000000in}}{%
\pgfpathmoveto{\pgfqpoint{0.000000in}{0.000000in}}%
\pgfpathlineto{\pgfqpoint{0.000000in}{-0.048611in}}%
\pgfusepath{stroke,fill}%
}%
\begin{pgfscope}%
\pgfsys@transformshift{1.668659in}{0.689227in}%
\pgfsys@useobject{currentmarker}{}%
\end{pgfscope}%
\end{pgfscope}%
\begin{pgfscope}%
\definecolor{textcolor}{rgb}{0.000000,0.000000,0.000000}%
\pgfsetstrokecolor{textcolor}%
\pgfsetfillcolor{textcolor}%
\pgftext[x=1.605904in, y=0.349232in, left, base,rotate=45.000000]{\color{textcolor}\rmfamily\fontsize{10.000000}{12.000000}\selectfont 0.01}%
\end{pgfscope}%
\begin{pgfscope}%
\pgfsetbuttcap%
\pgfsetroundjoin%
\definecolor{currentfill}{rgb}{0.000000,0.000000,0.000000}%
\pgfsetfillcolor{currentfill}%
\pgfsetlinewidth{0.803000pt}%
\definecolor{currentstroke}{rgb}{0.000000,0.000000,0.000000}%
\pgfsetstrokecolor{currentstroke}%
\pgfsetdash{}{0pt}%
\pgfsys@defobject{currentmarker}{\pgfqpoint{0.000000in}{-0.048611in}}{\pgfqpoint{0.000000in}{0.000000in}}{%
\pgfpathmoveto{\pgfqpoint{0.000000in}{0.000000in}}%
\pgfpathlineto{\pgfqpoint{0.000000in}{-0.048611in}}%
\pgfusepath{stroke,fill}%
}%
\begin{pgfscope}%
\pgfsys@transformshift{1.988073in}{0.689227in}%
\pgfsys@useobject{currentmarker}{}%
\end{pgfscope}%
\end{pgfscope}%
\begin{pgfscope}%
\definecolor{textcolor}{rgb}{0.000000,0.000000,0.000000}%
\pgfsetstrokecolor{textcolor}%
\pgfsetfillcolor{textcolor}%
\pgftext[x=1.900765in, y=0.300127in, left, base,rotate=45.000000]{\color{textcolor}\rmfamily\fontsize{10.000000}{12.000000}\selectfont 0.013}%
\end{pgfscope}%
\begin{pgfscope}%
\pgfsetbuttcap%
\pgfsetroundjoin%
\definecolor{currentfill}{rgb}{0.000000,0.000000,0.000000}%
\pgfsetfillcolor{currentfill}%
\pgfsetlinewidth{0.803000pt}%
\definecolor{currentstroke}{rgb}{0.000000,0.000000,0.000000}%
\pgfsetstrokecolor{currentstroke}%
\pgfsetdash{}{0pt}%
\pgfsys@defobject{currentmarker}{\pgfqpoint{0.000000in}{-0.048611in}}{\pgfqpoint{0.000000in}{0.000000in}}{%
\pgfpathmoveto{\pgfqpoint{0.000000in}{0.000000in}}%
\pgfpathlineto{\pgfqpoint{0.000000in}{-0.048611in}}%
\pgfusepath{stroke,fill}%
}%
\begin{pgfscope}%
\pgfsys@transformshift{2.307486in}{0.689227in}%
\pgfsys@useobject{currentmarker}{}%
\end{pgfscope}%
\end{pgfscope}%
\begin{pgfscope}%
\definecolor{textcolor}{rgb}{0.000000,0.000000,0.000000}%
\pgfsetstrokecolor{textcolor}%
\pgfsetfillcolor{textcolor}%
\pgftext[x=2.195626in, y=0.251022in, left, base,rotate=45.000000]{\color{textcolor}\rmfamily\fontsize{10.000000}{12.000000}\selectfont 0.0165}%
\end{pgfscope}%
\begin{pgfscope}%
\pgfsetbuttcap%
\pgfsetroundjoin%
\definecolor{currentfill}{rgb}{0.000000,0.000000,0.000000}%
\pgfsetfillcolor{currentfill}%
\pgfsetlinewidth{0.803000pt}%
\definecolor{currentstroke}{rgb}{0.000000,0.000000,0.000000}%
\pgfsetstrokecolor{currentstroke}%
\pgfsetdash{}{0pt}%
\pgfsys@defobject{currentmarker}{\pgfqpoint{0.000000in}{-0.048611in}}{\pgfqpoint{0.000000in}{0.000000in}}{%
\pgfpathmoveto{\pgfqpoint{0.000000in}{0.000000in}}%
\pgfpathlineto{\pgfqpoint{0.000000in}{-0.048611in}}%
\pgfusepath{stroke,fill}%
}%
\begin{pgfscope}%
\pgfsys@transformshift{2.626899in}{0.689227in}%
\pgfsys@useobject{currentmarker}{}%
\end{pgfscope}%
\end{pgfscope}%
\begin{pgfscope}%
\definecolor{textcolor}{rgb}{0.000000,0.000000,0.000000}%
\pgfsetstrokecolor{textcolor}%
\pgfsetfillcolor{textcolor}%
\pgftext[x=2.515039in, y=0.251022in, left, base,rotate=45.000000]{\color{textcolor}\rmfamily\fontsize{10.000000}{12.000000}\selectfont 0.0215}%
\end{pgfscope}%
\begin{pgfscope}%
\pgfsetbuttcap%
\pgfsetroundjoin%
\definecolor{currentfill}{rgb}{0.000000,0.000000,0.000000}%
\pgfsetfillcolor{currentfill}%
\pgfsetlinewidth{0.803000pt}%
\definecolor{currentstroke}{rgb}{0.000000,0.000000,0.000000}%
\pgfsetstrokecolor{currentstroke}%
\pgfsetdash{}{0pt}%
\pgfsys@defobject{currentmarker}{\pgfqpoint{0.000000in}{-0.048611in}}{\pgfqpoint{0.000000in}{0.000000in}}{%
\pgfpathmoveto{\pgfqpoint{0.000000in}{0.000000in}}%
\pgfpathlineto{\pgfqpoint{0.000000in}{-0.048611in}}%
\pgfusepath{stroke,fill}%
}%
\begin{pgfscope}%
\pgfsys@transformshift{2.946312in}{0.689227in}%
\pgfsys@useobject{currentmarker}{}%
\end{pgfscope}%
\end{pgfscope}%
\begin{pgfscope}%
\definecolor{textcolor}{rgb}{0.000000,0.000000,0.000000}%
\pgfsetstrokecolor{textcolor}%
\pgfsetfillcolor{textcolor}%
\pgftext[x=2.859004in, y=0.300127in, left, base,rotate=45.000000]{\color{textcolor}\rmfamily\fontsize{10.000000}{12.000000}\selectfont 0.028}%
\end{pgfscope}%
\begin{pgfscope}%
\pgfsetbuttcap%
\pgfsetroundjoin%
\definecolor{currentfill}{rgb}{0.000000,0.000000,0.000000}%
\pgfsetfillcolor{currentfill}%
\pgfsetlinewidth{0.803000pt}%
\definecolor{currentstroke}{rgb}{0.000000,0.000000,0.000000}%
\pgfsetstrokecolor{currentstroke}%
\pgfsetdash{}{0pt}%
\pgfsys@defobject{currentmarker}{\pgfqpoint{0.000000in}{-0.048611in}}{\pgfqpoint{0.000000in}{0.000000in}}{%
\pgfpathmoveto{\pgfqpoint{0.000000in}{0.000000in}}%
\pgfpathlineto{\pgfqpoint{0.000000in}{-0.048611in}}%
\pgfusepath{stroke,fill}%
}%
\begin{pgfscope}%
\pgfsys@transformshift{3.265726in}{0.689227in}%
\pgfsys@useobject{currentmarker}{}%
\end{pgfscope}%
\end{pgfscope}%
\begin{pgfscope}%
\definecolor{textcolor}{rgb}{0.000000,0.000000,0.000000}%
\pgfsetstrokecolor{textcolor}%
\pgfsetfillcolor{textcolor}%
\pgftext[x=3.178418in, y=0.300127in, left, base,rotate=45.000000]{\color{textcolor}\rmfamily\fontsize{10.000000}{12.000000}\selectfont 0.036}%
\end{pgfscope}%
\begin{pgfscope}%
\pgfsetbuttcap%
\pgfsetroundjoin%
\definecolor{currentfill}{rgb}{0.000000,0.000000,0.000000}%
\pgfsetfillcolor{currentfill}%
\pgfsetlinewidth{0.803000pt}%
\definecolor{currentstroke}{rgb}{0.000000,0.000000,0.000000}%
\pgfsetstrokecolor{currentstroke}%
\pgfsetdash{}{0pt}%
\pgfsys@defobject{currentmarker}{\pgfqpoint{0.000000in}{-0.048611in}}{\pgfqpoint{0.000000in}{0.000000in}}{%
\pgfpathmoveto{\pgfqpoint{0.000000in}{0.000000in}}%
\pgfpathlineto{\pgfqpoint{0.000000in}{-0.048611in}}%
\pgfusepath{stroke,fill}%
}%
\begin{pgfscope}%
\pgfsys@transformshift{3.585139in}{0.689227in}%
\pgfsys@useobject{currentmarker}{}%
\end{pgfscope}%
\end{pgfscope}%
\begin{pgfscope}%
\definecolor{textcolor}{rgb}{0.000000,0.000000,0.000000}%
\pgfsetstrokecolor{textcolor}%
\pgfsetfillcolor{textcolor}%
\pgftext[x=3.473279in, y=0.251022in, left, base,rotate=45.000000]{\color{textcolor}\rmfamily\fontsize{10.000000}{12.000000}\selectfont 0.0465}%
\end{pgfscope}%
\begin{pgfscope}%
\pgfsetbuttcap%
\pgfsetroundjoin%
\definecolor{currentfill}{rgb}{0.000000,0.000000,0.000000}%
\pgfsetfillcolor{currentfill}%
\pgfsetlinewidth{0.803000pt}%
\definecolor{currentstroke}{rgb}{0.000000,0.000000,0.000000}%
\pgfsetstrokecolor{currentstroke}%
\pgfsetdash{}{0pt}%
\pgfsys@defobject{currentmarker}{\pgfqpoint{0.000000in}{-0.048611in}}{\pgfqpoint{0.000000in}{0.000000in}}{%
\pgfpathmoveto{\pgfqpoint{0.000000in}{0.000000in}}%
\pgfpathlineto{\pgfqpoint{0.000000in}{-0.048611in}}%
\pgfusepath{stroke,fill}%
}%
\begin{pgfscope}%
\pgfsys@transformshift{3.904552in}{0.689227in}%
\pgfsys@useobject{currentmarker}{}%
\end{pgfscope}%
\end{pgfscope}%
\begin{pgfscope}%
\definecolor{textcolor}{rgb}{0.000000,0.000000,0.000000}%
\pgfsetstrokecolor{textcolor}%
\pgfsetfillcolor{textcolor}%
\pgftext[x=3.841797in, y=0.349232in, left, base,rotate=45.000000]{\color{textcolor}\rmfamily\fontsize{10.000000}{12.000000}\selectfont 0.06}%
\end{pgfscope}%
\begin{pgfscope}%
\pgfsetbuttcap%
\pgfsetroundjoin%
\definecolor{currentfill}{rgb}{0.000000,0.000000,0.000000}%
\pgfsetfillcolor{currentfill}%
\pgfsetlinewidth{0.803000pt}%
\definecolor{currentstroke}{rgb}{0.000000,0.000000,0.000000}%
\pgfsetstrokecolor{currentstroke}%
\pgfsetdash{}{0pt}%
\pgfsys@defobject{currentmarker}{\pgfqpoint{0.000000in}{-0.048611in}}{\pgfqpoint{0.000000in}{0.000000in}}{%
\pgfpathmoveto{\pgfqpoint{0.000000in}{0.000000in}}%
\pgfpathlineto{\pgfqpoint{0.000000in}{-0.048611in}}%
\pgfusepath{stroke,fill}%
}%
\begin{pgfscope}%
\pgfsys@transformshift{4.223966in}{0.689227in}%
\pgfsys@useobject{currentmarker}{}%
\end{pgfscope}%
\end{pgfscope}%
\begin{pgfscope}%
\definecolor{textcolor}{rgb}{0.000000,0.000000,0.000000}%
\pgfsetstrokecolor{textcolor}%
\pgfsetfillcolor{textcolor}%
\pgftext[x=4.136658in, y=0.300127in, left, base,rotate=45.000000]{\color{textcolor}\rmfamily\fontsize{10.000000}{12.000000}\selectfont 0.077}%
\end{pgfscope}%
\begin{pgfscope}%
\pgfsetbuttcap%
\pgfsetroundjoin%
\definecolor{currentfill}{rgb}{0.000000,0.000000,0.000000}%
\pgfsetfillcolor{currentfill}%
\pgfsetlinewidth{0.803000pt}%
\definecolor{currentstroke}{rgb}{0.000000,0.000000,0.000000}%
\pgfsetstrokecolor{currentstroke}%
\pgfsetdash{}{0pt}%
\pgfsys@defobject{currentmarker}{\pgfqpoint{0.000000in}{-0.048611in}}{\pgfqpoint{0.000000in}{0.000000in}}{%
\pgfpathmoveto{\pgfqpoint{0.000000in}{0.000000in}}%
\pgfpathlineto{\pgfqpoint{0.000000in}{-0.048611in}}%
\pgfusepath{stroke,fill}%
}%
\begin{pgfscope}%
\pgfsys@transformshift{4.543379in}{0.689227in}%
\pgfsys@useobject{currentmarker}{}%
\end{pgfscope}%
\end{pgfscope}%
\begin{pgfscope}%
\definecolor{textcolor}{rgb}{0.000000,0.000000,0.000000}%
\pgfsetstrokecolor{textcolor}%
\pgfsetfillcolor{textcolor}%
\pgftext[x=4.505176in, y=0.398336in, left, base,rotate=45.000000]{\color{textcolor}\rmfamily\fontsize{10.000000}{12.000000}\selectfont 0.1}%
\end{pgfscope}%
\begin{pgfscope}%
\pgfsetbuttcap%
\pgfsetroundjoin%
\definecolor{currentfill}{rgb}{0.000000,0.000000,0.000000}%
\pgfsetfillcolor{currentfill}%
\pgfsetlinewidth{0.803000pt}%
\definecolor{currentstroke}{rgb}{0.000000,0.000000,0.000000}%
\pgfsetstrokecolor{currentstroke}%
\pgfsetdash{}{0pt}%
\pgfsys@defobject{currentmarker}{\pgfqpoint{0.000000in}{-0.048611in}}{\pgfqpoint{0.000000in}{0.000000in}}{%
\pgfpathmoveto{\pgfqpoint{0.000000in}{0.000000in}}%
\pgfpathlineto{\pgfqpoint{0.000000in}{-0.048611in}}%
\pgfusepath{stroke,fill}%
}%
\begin{pgfscope}%
\pgfsys@transformshift{4.862792in}{0.689227in}%
\pgfsys@useobject{currentmarker}{}%
\end{pgfscope}%
\end{pgfscope}%
\begin{pgfscope}%
\definecolor{textcolor}{rgb}{0.000000,0.000000,0.000000}%
\pgfsetstrokecolor{textcolor}%
\pgfsetfillcolor{textcolor}%
\pgftext[x=4.824589in, y=0.398336in, left, base,rotate=45.000000]{\color{textcolor}\rmfamily\fontsize{10.000000}{12.000000}\selectfont 0.3}%
\end{pgfscope}%
\begin{pgfscope}%
\pgfsetbuttcap%
\pgfsetroundjoin%
\definecolor{currentfill}{rgb}{0.000000,0.000000,0.000000}%
\pgfsetfillcolor{currentfill}%
\pgfsetlinewidth{0.803000pt}%
\definecolor{currentstroke}{rgb}{0.000000,0.000000,0.000000}%
\pgfsetstrokecolor{currentstroke}%
\pgfsetdash{}{0pt}%
\pgfsys@defobject{currentmarker}{\pgfqpoint{0.000000in}{-0.048611in}}{\pgfqpoint{0.000000in}{0.000000in}}{%
\pgfpathmoveto{\pgfqpoint{0.000000in}{0.000000in}}%
\pgfpathlineto{\pgfqpoint{0.000000in}{-0.048611in}}%
\pgfusepath{stroke,fill}%
}%
\begin{pgfscope}%
\pgfsys@transformshift{5.182206in}{0.689227in}%
\pgfsys@useobject{currentmarker}{}%
\end{pgfscope}%
\end{pgfscope}%
\begin{pgfscope}%
\definecolor{textcolor}{rgb}{0.000000,0.000000,0.000000}%
\pgfsetstrokecolor{textcolor}%
\pgfsetfillcolor{textcolor}%
\pgftext[x=5.144002in, y=0.398336in, left, base,rotate=45.000000]{\color{textcolor}\rmfamily\fontsize{10.000000}{12.000000}\selectfont 0.5}%
\end{pgfscope}%
\begin{pgfscope}%
\definecolor{textcolor}{rgb}{0.000000,0.000000,0.000000}%
\pgfsetstrokecolor{textcolor}%
\pgfsetfillcolor{textcolor}%
\pgftext[x=2.946312in,y=0.176414in,,top]{\color{textcolor}\rmfamily\fontsize{10.000000}{12.000000}\selectfont Forgetting Factor (\(\displaystyle \epsilon\)/\(\displaystyle \hat{\sigma}_w^2\))}%
\end{pgfscope}%
\begin{pgfscope}%
\pgfsetbuttcap%
\pgfsetroundjoin%
\definecolor{currentfill}{rgb}{0.000000,0.000000,0.000000}%
\pgfsetfillcolor{currentfill}%
\pgfsetlinewidth{0.803000pt}%
\definecolor{currentstroke}{rgb}{0.000000,0.000000,0.000000}%
\pgfsetstrokecolor{currentstroke}%
\pgfsetdash{}{0pt}%
\pgfsys@defobject{currentmarker}{\pgfqpoint{-0.048611in}{0.000000in}}{\pgfqpoint{-0.000000in}{0.000000in}}{%
\pgfpathmoveto{\pgfqpoint{-0.000000in}{0.000000in}}%
\pgfpathlineto{\pgfqpoint{-0.048611in}{0.000000in}}%
\pgfusepath{stroke,fill}%
}%
\begin{pgfscope}%
\pgfsys@transformshift{0.550713in}{0.689227in}%
\pgfsys@useobject{currentmarker}{}%
\end{pgfscope}%
\end{pgfscope}%
\begin{pgfscope}%
\definecolor{textcolor}{rgb}{0.000000,0.000000,0.000000}%
\pgfsetstrokecolor{textcolor}%
\pgfsetfillcolor{textcolor}%
\pgftext[x=0.384046in, y=0.641033in, left, base]{\color{textcolor}\rmfamily\fontsize{10.000000}{12.000000}\selectfont \(\displaystyle {0}\)}%
\end{pgfscope}%
\begin{pgfscope}%
\pgfsetbuttcap%
\pgfsetroundjoin%
\definecolor{currentfill}{rgb}{0.000000,0.000000,0.000000}%
\pgfsetfillcolor{currentfill}%
\pgfsetlinewidth{0.803000pt}%
\definecolor{currentstroke}{rgb}{0.000000,0.000000,0.000000}%
\pgfsetstrokecolor{currentstroke}%
\pgfsetdash{}{0pt}%
\pgfsys@defobject{currentmarker}{\pgfqpoint{-0.048611in}{0.000000in}}{\pgfqpoint{-0.000000in}{0.000000in}}{%
\pgfpathmoveto{\pgfqpoint{-0.000000in}{0.000000in}}%
\pgfpathlineto{\pgfqpoint{-0.048611in}{0.000000in}}%
\pgfusepath{stroke,fill}%
}%
\begin{pgfscope}%
\pgfsys@transformshift{0.550713in}{1.133289in}%
\pgfsys@useobject{currentmarker}{}%
\end{pgfscope}%
\end{pgfscope}%
\begin{pgfscope}%
\definecolor{textcolor}{rgb}{0.000000,0.000000,0.000000}%
\pgfsetstrokecolor{textcolor}%
\pgfsetfillcolor{textcolor}%
\pgftext[x=0.245156in, y=1.085095in, left, base]{\color{textcolor}\rmfamily\fontsize{10.000000}{12.000000}\selectfont \(\displaystyle {100}\)}%
\end{pgfscope}%
\begin{pgfscope}%
\pgfsetbuttcap%
\pgfsetroundjoin%
\definecolor{currentfill}{rgb}{0.000000,0.000000,0.000000}%
\pgfsetfillcolor{currentfill}%
\pgfsetlinewidth{0.803000pt}%
\definecolor{currentstroke}{rgb}{0.000000,0.000000,0.000000}%
\pgfsetstrokecolor{currentstroke}%
\pgfsetdash{}{0pt}%
\pgfsys@defobject{currentmarker}{\pgfqpoint{-0.048611in}{0.000000in}}{\pgfqpoint{-0.000000in}{0.000000in}}{%
\pgfpathmoveto{\pgfqpoint{-0.000000in}{0.000000in}}%
\pgfpathlineto{\pgfqpoint{-0.048611in}{0.000000in}}%
\pgfusepath{stroke,fill}%
}%
\begin{pgfscope}%
\pgfsys@transformshift{0.550713in}{1.577352in}%
\pgfsys@useobject{currentmarker}{}%
\end{pgfscope}%
\end{pgfscope}%
\begin{pgfscope}%
\definecolor{textcolor}{rgb}{0.000000,0.000000,0.000000}%
\pgfsetstrokecolor{textcolor}%
\pgfsetfillcolor{textcolor}%
\pgftext[x=0.245156in, y=1.529157in, left, base]{\color{textcolor}\rmfamily\fontsize{10.000000}{12.000000}\selectfont \(\displaystyle {200}\)}%
\end{pgfscope}%
\begin{pgfscope}%
\pgfsetbuttcap%
\pgfsetroundjoin%
\definecolor{currentfill}{rgb}{0.000000,0.000000,0.000000}%
\pgfsetfillcolor{currentfill}%
\pgfsetlinewidth{0.803000pt}%
\definecolor{currentstroke}{rgb}{0.000000,0.000000,0.000000}%
\pgfsetstrokecolor{currentstroke}%
\pgfsetdash{}{0pt}%
\pgfsys@defobject{currentmarker}{\pgfqpoint{-0.048611in}{0.000000in}}{\pgfqpoint{-0.000000in}{0.000000in}}{%
\pgfpathmoveto{\pgfqpoint{-0.000000in}{0.000000in}}%
\pgfpathlineto{\pgfqpoint{-0.048611in}{0.000000in}}%
\pgfusepath{stroke,fill}%
}%
\begin{pgfscope}%
\pgfsys@transformshift{0.550713in}{2.021414in}%
\pgfsys@useobject{currentmarker}{}%
\end{pgfscope}%
\end{pgfscope}%
\begin{pgfscope}%
\definecolor{textcolor}{rgb}{0.000000,0.000000,0.000000}%
\pgfsetstrokecolor{textcolor}%
\pgfsetfillcolor{textcolor}%
\pgftext[x=0.245156in, y=1.973220in, left, base]{\color{textcolor}\rmfamily\fontsize{10.000000}{12.000000}\selectfont \(\displaystyle {300}\)}%
\end{pgfscope}%
\begin{pgfscope}%
\definecolor{textcolor}{rgb}{0.000000,0.000000,0.000000}%
\pgfsetstrokecolor{textcolor}%
\pgfsetfillcolor{textcolor}%
\pgftext[x=0.189601in,y=1.355321in,,bottom,rotate=90.000000]{\color{textcolor}\rmfamily\fontsize{10.000000}{12.000000}\selectfont \(\displaystyle R_T\)}%
\end{pgfscope}%
\begin{pgfscope}%
\pgfpathrectangle{\pgfqpoint{0.550713in}{0.689227in}}{\pgfqpoint{4.791200in}{1.332187in}}%
\pgfusepath{clip}%
\pgfsetbuttcap%
\pgfsetroundjoin%
\pgfsetlinewidth{0.501875pt}%
\definecolor{currentstroke}{rgb}{0.392157,0.396078,0.403922}%
\pgfsetstrokecolor{currentstroke}%
\pgfsetdash{}{0pt}%
\pgfpathmoveto{\pgfqpoint{0.870126in}{0.689227in}}%
\pgfpathlineto{\pgfqpoint{0.870126in}{2.021414in}}%
\pgfusepath{stroke}%
\end{pgfscope}%
\begin{pgfscope}%
\pgfpathrectangle{\pgfqpoint{0.550713in}{0.689227in}}{\pgfqpoint{4.791200in}{1.332187in}}%
\pgfusepath{clip}%
\pgfsetbuttcap%
\pgfsetroundjoin%
\pgfsetlinewidth{0.501875pt}%
\definecolor{currentstroke}{rgb}{0.392157,0.396078,0.403922}%
\pgfsetstrokecolor{currentstroke}%
\pgfsetdash{}{0pt}%
\pgfpathmoveto{\pgfqpoint{1.189539in}{0.689227in}}%
\pgfpathlineto{\pgfqpoint{1.189539in}{2.021414in}}%
\pgfusepath{stroke}%
\end{pgfscope}%
\begin{pgfscope}%
\pgfpathrectangle{\pgfqpoint{0.550713in}{0.689227in}}{\pgfqpoint{4.791200in}{1.332187in}}%
\pgfusepath{clip}%
\pgfsetbuttcap%
\pgfsetroundjoin%
\pgfsetlinewidth{0.501875pt}%
\definecolor{currentstroke}{rgb}{0.392157,0.396078,0.403922}%
\pgfsetstrokecolor{currentstroke}%
\pgfsetdash{}{0pt}%
\pgfpathmoveto{\pgfqpoint{1.508953in}{0.689227in}}%
\pgfpathlineto{\pgfqpoint{1.508953in}{2.021414in}}%
\pgfusepath{stroke}%
\end{pgfscope}%
\begin{pgfscope}%
\pgfpathrectangle{\pgfqpoint{0.550713in}{0.689227in}}{\pgfqpoint{4.791200in}{1.332187in}}%
\pgfusepath{clip}%
\pgfsetbuttcap%
\pgfsetroundjoin%
\pgfsetlinewidth{0.501875pt}%
\definecolor{currentstroke}{rgb}{0.392157,0.396078,0.403922}%
\pgfsetstrokecolor{currentstroke}%
\pgfsetdash{}{0pt}%
\pgfpathmoveto{\pgfqpoint{1.828366in}{0.689227in}}%
\pgfpathlineto{\pgfqpoint{1.828366in}{2.021414in}}%
\pgfusepath{stroke}%
\end{pgfscope}%
\begin{pgfscope}%
\pgfpathrectangle{\pgfqpoint{0.550713in}{0.689227in}}{\pgfqpoint{4.791200in}{1.332187in}}%
\pgfusepath{clip}%
\pgfsetbuttcap%
\pgfsetroundjoin%
\pgfsetlinewidth{0.501875pt}%
\definecolor{currentstroke}{rgb}{0.392157,0.396078,0.403922}%
\pgfsetstrokecolor{currentstroke}%
\pgfsetdash{}{0pt}%
\pgfpathmoveto{\pgfqpoint{2.147779in}{0.689227in}}%
\pgfpathlineto{\pgfqpoint{2.147779in}{2.021414in}}%
\pgfusepath{stroke}%
\end{pgfscope}%
\begin{pgfscope}%
\pgfpathrectangle{\pgfqpoint{0.550713in}{0.689227in}}{\pgfqpoint{4.791200in}{1.332187in}}%
\pgfusepath{clip}%
\pgfsetbuttcap%
\pgfsetroundjoin%
\pgfsetlinewidth{0.501875pt}%
\definecolor{currentstroke}{rgb}{0.392157,0.396078,0.403922}%
\pgfsetstrokecolor{currentstroke}%
\pgfsetdash{}{0pt}%
\pgfpathmoveto{\pgfqpoint{2.467192in}{0.689227in}}%
\pgfpathlineto{\pgfqpoint{2.467192in}{2.021414in}}%
\pgfusepath{stroke}%
\end{pgfscope}%
\begin{pgfscope}%
\pgfpathrectangle{\pgfqpoint{0.550713in}{0.689227in}}{\pgfqpoint{4.791200in}{1.332187in}}%
\pgfusepath{clip}%
\pgfsetbuttcap%
\pgfsetroundjoin%
\pgfsetlinewidth{0.501875pt}%
\definecolor{currentstroke}{rgb}{0.392157,0.396078,0.403922}%
\pgfsetstrokecolor{currentstroke}%
\pgfsetdash{}{0pt}%
\pgfpathmoveto{\pgfqpoint{2.786606in}{0.689227in}}%
\pgfpathlineto{\pgfqpoint{2.786606in}{2.021414in}}%
\pgfusepath{stroke}%
\end{pgfscope}%
\begin{pgfscope}%
\pgfpathrectangle{\pgfqpoint{0.550713in}{0.689227in}}{\pgfqpoint{4.791200in}{1.332187in}}%
\pgfusepath{clip}%
\pgfsetbuttcap%
\pgfsetroundjoin%
\pgfsetlinewidth{0.501875pt}%
\definecolor{currentstroke}{rgb}{0.392157,0.396078,0.403922}%
\pgfsetstrokecolor{currentstroke}%
\pgfsetdash{}{0pt}%
\pgfpathmoveto{\pgfqpoint{3.106019in}{0.689227in}}%
\pgfpathlineto{\pgfqpoint{3.106019in}{2.021414in}}%
\pgfusepath{stroke}%
\end{pgfscope}%
\begin{pgfscope}%
\pgfpathrectangle{\pgfqpoint{0.550713in}{0.689227in}}{\pgfqpoint{4.791200in}{1.332187in}}%
\pgfusepath{clip}%
\pgfsetbuttcap%
\pgfsetroundjoin%
\pgfsetlinewidth{0.501875pt}%
\definecolor{currentstroke}{rgb}{0.392157,0.396078,0.403922}%
\pgfsetstrokecolor{currentstroke}%
\pgfsetdash{}{0pt}%
\pgfpathmoveto{\pgfqpoint{3.425432in}{0.689227in}}%
\pgfpathlineto{\pgfqpoint{3.425432in}{2.021414in}}%
\pgfusepath{stroke}%
\end{pgfscope}%
\begin{pgfscope}%
\pgfpathrectangle{\pgfqpoint{0.550713in}{0.689227in}}{\pgfqpoint{4.791200in}{1.332187in}}%
\pgfusepath{clip}%
\pgfsetbuttcap%
\pgfsetroundjoin%
\pgfsetlinewidth{0.501875pt}%
\definecolor{currentstroke}{rgb}{0.392157,0.396078,0.403922}%
\pgfsetstrokecolor{currentstroke}%
\pgfsetdash{}{0pt}%
\pgfpathmoveto{\pgfqpoint{3.744846in}{0.689227in}}%
\pgfpathlineto{\pgfqpoint{3.744846in}{2.021414in}}%
\pgfusepath{stroke}%
\end{pgfscope}%
\begin{pgfscope}%
\pgfpathrectangle{\pgfqpoint{0.550713in}{0.689227in}}{\pgfqpoint{4.791200in}{1.332187in}}%
\pgfusepath{clip}%
\pgfsetbuttcap%
\pgfsetroundjoin%
\pgfsetlinewidth{0.501875pt}%
\definecolor{currentstroke}{rgb}{0.392157,0.396078,0.403922}%
\pgfsetstrokecolor{currentstroke}%
\pgfsetdash{}{0pt}%
\pgfpathmoveto{\pgfqpoint{4.064259in}{0.689227in}}%
\pgfpathlineto{\pgfqpoint{4.064259in}{2.021414in}}%
\pgfusepath{stroke}%
\end{pgfscope}%
\begin{pgfscope}%
\pgfpathrectangle{\pgfqpoint{0.550713in}{0.689227in}}{\pgfqpoint{4.791200in}{1.332187in}}%
\pgfusepath{clip}%
\pgfsetbuttcap%
\pgfsetroundjoin%
\pgfsetlinewidth{0.501875pt}%
\definecolor{currentstroke}{rgb}{0.392157,0.396078,0.403922}%
\pgfsetstrokecolor{currentstroke}%
\pgfsetdash{}{0pt}%
\pgfpathmoveto{\pgfqpoint{4.383672in}{0.689227in}}%
\pgfpathlineto{\pgfqpoint{4.383672in}{2.021414in}}%
\pgfusepath{stroke}%
\end{pgfscope}%
\begin{pgfscope}%
\pgfpathrectangle{\pgfqpoint{0.550713in}{0.689227in}}{\pgfqpoint{4.791200in}{1.332187in}}%
\pgfusepath{clip}%
\pgfsetbuttcap%
\pgfsetroundjoin%
\pgfsetlinewidth{0.501875pt}%
\definecolor{currentstroke}{rgb}{0.392157,0.396078,0.403922}%
\pgfsetstrokecolor{currentstroke}%
\pgfsetdash{}{0pt}%
\pgfpathmoveto{\pgfqpoint{4.703086in}{0.689227in}}%
\pgfpathlineto{\pgfqpoint{4.703086in}{2.021414in}}%
\pgfusepath{stroke}%
\end{pgfscope}%
\begin{pgfscope}%
\pgfpathrectangle{\pgfqpoint{0.550713in}{0.689227in}}{\pgfqpoint{4.791200in}{1.332187in}}%
\pgfusepath{clip}%
\pgfsetbuttcap%
\pgfsetroundjoin%
\pgfsetlinewidth{0.501875pt}%
\definecolor{currentstroke}{rgb}{0.392157,0.396078,0.403922}%
\pgfsetstrokecolor{currentstroke}%
\pgfsetdash{}{0pt}%
\pgfpathmoveto{\pgfqpoint{5.022499in}{0.689227in}}%
\pgfpathlineto{\pgfqpoint{5.022499in}{2.021414in}}%
\pgfusepath{stroke}%
\end{pgfscope}%
\begin{pgfscope}%
\pgfpathrectangle{\pgfqpoint{0.550713in}{0.689227in}}{\pgfqpoint{4.791200in}{1.332187in}}%
\pgfusepath{clip}%
\pgfsetrectcap%
\pgfsetroundjoin%
\pgfsetlinewidth{1.505625pt}%
\definecolor{currentstroke}{rgb}{0.631373,0.062745,0.207843}%
\pgfsetstrokecolor{currentstroke}%
\pgfsetdash{}{0pt}%
\pgfpathmoveto{\pgfqpoint{0.710419in}{1.462491in}}%
\pgfpathlineto{\pgfqpoint{1.029833in}{1.540432in}}%
\pgfpathlineto{\pgfqpoint{1.349246in}{1.160010in}}%
\pgfpathlineto{\pgfqpoint{1.668659in}{1.225664in}}%
\pgfpathlineto{\pgfqpoint{1.988073in}{1.178048in}}%
\pgfpathlineto{\pgfqpoint{2.307486in}{1.181597in}}%
\pgfpathlineto{\pgfqpoint{2.626899in}{1.157778in}}%
\pgfpathlineto{\pgfqpoint{2.946312in}{1.131424in}}%
\pgfpathlineto{\pgfqpoint{3.265726in}{1.215873in}}%
\pgfpathlineto{\pgfqpoint{3.585139in}{1.317877in}}%
\pgfpathlineto{\pgfqpoint{3.904552in}{1.383963in}}%
\pgfpathlineto{\pgfqpoint{4.223966in}{1.460051in}}%
\pgfpathlineto{\pgfqpoint{4.543379in}{1.634682in}}%
\pgfpathlineto{\pgfqpoint{4.706922in}{2.031414in}}%
\pgfusepath{stroke}%
\end{pgfscope}%
\begin{pgfscope}%
\pgfpathrectangle{\pgfqpoint{0.550713in}{0.689227in}}{\pgfqpoint{4.791200in}{1.332187in}}%
\pgfusepath{clip}%
\pgfsetbuttcap%
\pgfsetroundjoin%
\definecolor{currentfill}{rgb}{0.631373,0.062745,0.207843}%
\pgfsetfillcolor{currentfill}%
\pgfsetlinewidth{0.752812pt}%
\definecolor{currentstroke}{rgb}{0.000000,0.000000,0.000000}%
\pgfsetstrokecolor{currentstroke}%
\pgfsetdash{}{0pt}%
\pgfsys@defobject{currentmarker}{\pgfqpoint{-0.027778in}{-0.027778in}}{\pgfqpoint{0.027778in}{0.027778in}}{%
\pgfpathmoveto{\pgfqpoint{0.000000in}{-0.027778in}}%
\pgfpathcurveto{\pgfqpoint{0.007367in}{-0.027778in}}{\pgfqpoint{0.014433in}{-0.024851in}}{\pgfqpoint{0.019642in}{-0.019642in}}%
\pgfpathcurveto{\pgfqpoint{0.024851in}{-0.014433in}}{\pgfqpoint{0.027778in}{-0.007367in}}{\pgfqpoint{0.027778in}{0.000000in}}%
\pgfpathcurveto{\pgfqpoint{0.027778in}{0.007367in}}{\pgfqpoint{0.024851in}{0.014433in}}{\pgfqpoint{0.019642in}{0.019642in}}%
\pgfpathcurveto{\pgfqpoint{0.014433in}{0.024851in}}{\pgfqpoint{0.007367in}{0.027778in}}{\pgfqpoint{0.000000in}{0.027778in}}%
\pgfpathcurveto{\pgfqpoint{-0.007367in}{0.027778in}}{\pgfqpoint{-0.014433in}{0.024851in}}{\pgfqpoint{-0.019642in}{0.019642in}}%
\pgfpathcurveto{\pgfqpoint{-0.024851in}{0.014433in}}{\pgfqpoint{-0.027778in}{0.007367in}}{\pgfqpoint{-0.027778in}{0.000000in}}%
\pgfpathcurveto{\pgfqpoint{-0.027778in}{-0.007367in}}{\pgfqpoint{-0.024851in}{-0.014433in}}{\pgfqpoint{-0.019642in}{-0.019642in}}%
\pgfpathcurveto{\pgfqpoint{-0.014433in}{-0.024851in}}{\pgfqpoint{-0.007367in}{-0.027778in}}{\pgfqpoint{0.000000in}{-0.027778in}}%
\pgfpathclose%
\pgfusepath{stroke,fill}%
}%
\begin{pgfscope}%
\pgfsys@transformshift{0.710419in}{1.462491in}%
\pgfsys@useobject{currentmarker}{}%
\end{pgfscope}%
\begin{pgfscope}%
\pgfsys@transformshift{1.029833in}{1.540432in}%
\pgfsys@useobject{currentmarker}{}%
\end{pgfscope}%
\begin{pgfscope}%
\pgfsys@transformshift{1.349246in}{1.160010in}%
\pgfsys@useobject{currentmarker}{}%
\end{pgfscope}%
\begin{pgfscope}%
\pgfsys@transformshift{1.668659in}{1.225664in}%
\pgfsys@useobject{currentmarker}{}%
\end{pgfscope}%
\begin{pgfscope}%
\pgfsys@transformshift{1.988073in}{1.178048in}%
\pgfsys@useobject{currentmarker}{}%
\end{pgfscope}%
\begin{pgfscope}%
\pgfsys@transformshift{2.307486in}{1.181597in}%
\pgfsys@useobject{currentmarker}{}%
\end{pgfscope}%
\begin{pgfscope}%
\pgfsys@transformshift{2.626899in}{1.157778in}%
\pgfsys@useobject{currentmarker}{}%
\end{pgfscope}%
\begin{pgfscope}%
\pgfsys@transformshift{2.946312in}{1.131424in}%
\pgfsys@useobject{currentmarker}{}%
\end{pgfscope}%
\begin{pgfscope}%
\pgfsys@transformshift{3.265726in}{1.215873in}%
\pgfsys@useobject{currentmarker}{}%
\end{pgfscope}%
\begin{pgfscope}%
\pgfsys@transformshift{3.585139in}{1.317877in}%
\pgfsys@useobject{currentmarker}{}%
\end{pgfscope}%
\begin{pgfscope}%
\pgfsys@transformshift{3.904552in}{1.383963in}%
\pgfsys@useobject{currentmarker}{}%
\end{pgfscope}%
\begin{pgfscope}%
\pgfsys@transformshift{4.223966in}{1.460051in}%
\pgfsys@useobject{currentmarker}{}%
\end{pgfscope}%
\begin{pgfscope}%
\pgfsys@transformshift{4.543379in}{1.634682in}%
\pgfsys@useobject{currentmarker}{}%
\end{pgfscope}%
\begin{pgfscope}%
\pgfsys@transformshift{4.862792in}{2.409535in}%
\pgfsys@useobject{currentmarker}{}%
\end{pgfscope}%
\begin{pgfscope}%
\pgfsys@transformshift{5.182206in}{2.840723in}%
\pgfsys@useobject{currentmarker}{}%
\end{pgfscope}%
\end{pgfscope}%
\begin{pgfscope}%
\pgfpathrectangle{\pgfqpoint{0.550713in}{0.689227in}}{\pgfqpoint{4.791200in}{1.332187in}}%
\pgfusepath{clip}%
\pgfsetrectcap%
\pgfsetroundjoin%
\pgfsetlinewidth{1.505625pt}%
\definecolor{currentstroke}{rgb}{0.380392,0.129412,0.345098}%
\pgfsetstrokecolor{currentstroke}%
\pgfsetdash{}{0pt}%
\pgfpathmoveto{\pgfqpoint{0.710419in}{0.937933in}}%
\pgfpathlineto{\pgfqpoint{1.029833in}{0.949702in}}%
\pgfpathlineto{\pgfqpoint{1.349246in}{0.928951in}}%
\pgfpathlineto{\pgfqpoint{1.668659in}{0.930561in}}%
\pgfpathlineto{\pgfqpoint{1.988073in}{0.959011in}}%
\pgfpathlineto{\pgfqpoint{2.307486in}{1.044885in}}%
\pgfpathlineto{\pgfqpoint{2.626899in}{1.118617in}}%
\pgfpathlineto{\pgfqpoint{2.946312in}{1.222375in}}%
\pgfpathlineto{\pgfqpoint{3.265726in}{1.327096in}}%
\pgfpathlineto{\pgfqpoint{3.585139in}{1.480945in}}%
\pgfpathlineto{\pgfqpoint{3.904552in}{1.543973in}}%
\pgfpathlineto{\pgfqpoint{4.223966in}{1.726215in}}%
\pgfpathlineto{\pgfqpoint{4.543379in}{1.743237in}}%
\pgfpathlineto{\pgfqpoint{4.862792in}{1.222283in}}%
\pgfpathlineto{\pgfqpoint{4.958157in}{2.031414in}}%
\pgfusepath{stroke}%
\end{pgfscope}%
\begin{pgfscope}%
\pgfpathrectangle{\pgfqpoint{0.550713in}{0.689227in}}{\pgfqpoint{4.791200in}{1.332187in}}%
\pgfusepath{clip}%
\pgfsetbuttcap%
\pgfsetroundjoin%
\definecolor{currentfill}{rgb}{0.380392,0.129412,0.345098}%
\pgfsetfillcolor{currentfill}%
\pgfsetlinewidth{0.752812pt}%
\definecolor{currentstroke}{rgb}{0.000000,0.000000,0.000000}%
\pgfsetstrokecolor{currentstroke}%
\pgfsetdash{}{0pt}%
\pgfsys@defobject{currentmarker}{\pgfqpoint{-0.027778in}{-0.027778in}}{\pgfqpoint{0.027778in}{0.027778in}}{%
\pgfpathmoveto{\pgfqpoint{0.000000in}{-0.027778in}}%
\pgfpathcurveto{\pgfqpoint{0.007367in}{-0.027778in}}{\pgfqpoint{0.014433in}{-0.024851in}}{\pgfqpoint{0.019642in}{-0.019642in}}%
\pgfpathcurveto{\pgfqpoint{0.024851in}{-0.014433in}}{\pgfqpoint{0.027778in}{-0.007367in}}{\pgfqpoint{0.027778in}{0.000000in}}%
\pgfpathcurveto{\pgfqpoint{0.027778in}{0.007367in}}{\pgfqpoint{0.024851in}{0.014433in}}{\pgfqpoint{0.019642in}{0.019642in}}%
\pgfpathcurveto{\pgfqpoint{0.014433in}{0.024851in}}{\pgfqpoint{0.007367in}{0.027778in}}{\pgfqpoint{0.000000in}{0.027778in}}%
\pgfpathcurveto{\pgfqpoint{-0.007367in}{0.027778in}}{\pgfqpoint{-0.014433in}{0.024851in}}{\pgfqpoint{-0.019642in}{0.019642in}}%
\pgfpathcurveto{\pgfqpoint{-0.024851in}{0.014433in}}{\pgfqpoint{-0.027778in}{0.007367in}}{\pgfqpoint{-0.027778in}{0.000000in}}%
\pgfpathcurveto{\pgfqpoint{-0.027778in}{-0.007367in}}{\pgfqpoint{-0.024851in}{-0.014433in}}{\pgfqpoint{-0.019642in}{-0.019642in}}%
\pgfpathcurveto{\pgfqpoint{-0.014433in}{-0.024851in}}{\pgfqpoint{-0.007367in}{-0.027778in}}{\pgfqpoint{0.000000in}{-0.027778in}}%
\pgfpathclose%
\pgfusepath{stroke,fill}%
}%
\begin{pgfscope}%
\pgfsys@transformshift{0.710419in}{0.937933in}%
\pgfsys@useobject{currentmarker}{}%
\end{pgfscope}%
\begin{pgfscope}%
\pgfsys@transformshift{1.029833in}{0.949702in}%
\pgfsys@useobject{currentmarker}{}%
\end{pgfscope}%
\begin{pgfscope}%
\pgfsys@transformshift{1.349246in}{0.928951in}%
\pgfsys@useobject{currentmarker}{}%
\end{pgfscope}%
\begin{pgfscope}%
\pgfsys@transformshift{1.668659in}{0.930561in}%
\pgfsys@useobject{currentmarker}{}%
\end{pgfscope}%
\begin{pgfscope}%
\pgfsys@transformshift{1.988073in}{0.959011in}%
\pgfsys@useobject{currentmarker}{}%
\end{pgfscope}%
\begin{pgfscope}%
\pgfsys@transformshift{2.307486in}{1.044885in}%
\pgfsys@useobject{currentmarker}{}%
\end{pgfscope}%
\begin{pgfscope}%
\pgfsys@transformshift{2.626899in}{1.118617in}%
\pgfsys@useobject{currentmarker}{}%
\end{pgfscope}%
\begin{pgfscope}%
\pgfsys@transformshift{2.946312in}{1.222375in}%
\pgfsys@useobject{currentmarker}{}%
\end{pgfscope}%
\begin{pgfscope}%
\pgfsys@transformshift{3.265726in}{1.327096in}%
\pgfsys@useobject{currentmarker}{}%
\end{pgfscope}%
\begin{pgfscope}%
\pgfsys@transformshift{3.585139in}{1.480945in}%
\pgfsys@useobject{currentmarker}{}%
\end{pgfscope}%
\begin{pgfscope}%
\pgfsys@transformshift{3.904552in}{1.543973in}%
\pgfsys@useobject{currentmarker}{}%
\end{pgfscope}%
\begin{pgfscope}%
\pgfsys@transformshift{4.223966in}{1.726215in}%
\pgfsys@useobject{currentmarker}{}%
\end{pgfscope}%
\begin{pgfscope}%
\pgfsys@transformshift{4.543379in}{1.743237in}%
\pgfsys@useobject{currentmarker}{}%
\end{pgfscope}%
\begin{pgfscope}%
\pgfsys@transformshift{4.862792in}{1.222283in}%
\pgfsys@useobject{currentmarker}{}%
\end{pgfscope}%
\begin{pgfscope}%
\pgfsys@transformshift{5.182206in}{3.932364in}%
\pgfsys@useobject{currentmarker}{}%
\end{pgfscope}%
\end{pgfscope}%
\begin{pgfscope}%
\pgfpathrectangle{\pgfqpoint{0.550713in}{0.689227in}}{\pgfqpoint{4.791200in}{1.332187in}}%
\pgfusepath{clip}%
\pgfsetrectcap%
\pgfsetroundjoin%
\pgfsetlinewidth{1.505625pt}%
\definecolor{currentstroke}{rgb}{0.341176,0.670588,0.152941}%
\pgfsetstrokecolor{currentstroke}%
\pgfsetdash{}{0pt}%
\pgfpathmoveto{\pgfqpoint{0.710419in}{1.037541in}}%
\pgfpathlineto{\pgfqpoint{1.029833in}{0.929151in}}%
\pgfpathlineto{\pgfqpoint{1.349246in}{0.907282in}}%
\pgfpathlineto{\pgfqpoint{1.668659in}{0.934498in}}%
\pgfpathlineto{\pgfqpoint{1.988073in}{0.955324in}}%
\pgfpathlineto{\pgfqpoint{2.307486in}{0.927817in}}%
\pgfpathlineto{\pgfqpoint{2.626899in}{0.975948in}}%
\pgfpathlineto{\pgfqpoint{2.946312in}{1.041860in}}%
\pgfpathlineto{\pgfqpoint{3.265726in}{1.038215in}}%
\pgfpathlineto{\pgfqpoint{3.585139in}{1.126566in}}%
\pgfpathlineto{\pgfqpoint{3.904552in}{1.137026in}}%
\pgfpathlineto{\pgfqpoint{4.223966in}{1.187785in}}%
\pgfpathlineto{\pgfqpoint{4.543379in}{1.252660in}}%
\pgfpathlineto{\pgfqpoint{4.862792in}{1.465520in}}%
\pgfpathlineto{\pgfqpoint{5.182206in}{1.539208in}}%
\pgfusepath{stroke}%
\end{pgfscope}%
\begin{pgfscope}%
\pgfpathrectangle{\pgfqpoint{0.550713in}{0.689227in}}{\pgfqpoint{4.791200in}{1.332187in}}%
\pgfusepath{clip}%
\pgfsetbuttcap%
\pgfsetroundjoin%
\definecolor{currentfill}{rgb}{0.341176,0.670588,0.152941}%
\pgfsetfillcolor{currentfill}%
\pgfsetlinewidth{0.752812pt}%
\definecolor{currentstroke}{rgb}{0.000000,0.000000,0.000000}%
\pgfsetstrokecolor{currentstroke}%
\pgfsetdash{}{0pt}%
\pgfsys@defobject{currentmarker}{\pgfqpoint{-0.027778in}{-0.027778in}}{\pgfqpoint{0.027778in}{0.027778in}}{%
\pgfpathmoveto{\pgfqpoint{0.000000in}{-0.027778in}}%
\pgfpathcurveto{\pgfqpoint{0.007367in}{-0.027778in}}{\pgfqpoint{0.014433in}{-0.024851in}}{\pgfqpoint{0.019642in}{-0.019642in}}%
\pgfpathcurveto{\pgfqpoint{0.024851in}{-0.014433in}}{\pgfqpoint{0.027778in}{-0.007367in}}{\pgfqpoint{0.027778in}{0.000000in}}%
\pgfpathcurveto{\pgfqpoint{0.027778in}{0.007367in}}{\pgfqpoint{0.024851in}{0.014433in}}{\pgfqpoint{0.019642in}{0.019642in}}%
\pgfpathcurveto{\pgfqpoint{0.014433in}{0.024851in}}{\pgfqpoint{0.007367in}{0.027778in}}{\pgfqpoint{0.000000in}{0.027778in}}%
\pgfpathcurveto{\pgfqpoint{-0.007367in}{0.027778in}}{\pgfqpoint{-0.014433in}{0.024851in}}{\pgfqpoint{-0.019642in}{0.019642in}}%
\pgfpathcurveto{\pgfqpoint{-0.024851in}{0.014433in}}{\pgfqpoint{-0.027778in}{0.007367in}}{\pgfqpoint{-0.027778in}{0.000000in}}%
\pgfpathcurveto{\pgfqpoint{-0.027778in}{-0.007367in}}{\pgfqpoint{-0.024851in}{-0.014433in}}{\pgfqpoint{-0.019642in}{-0.019642in}}%
\pgfpathcurveto{\pgfqpoint{-0.014433in}{-0.024851in}}{\pgfqpoint{-0.007367in}{-0.027778in}}{\pgfqpoint{0.000000in}{-0.027778in}}%
\pgfpathclose%
\pgfusepath{stroke,fill}%
}%
\begin{pgfscope}%
\pgfsys@transformshift{0.710419in}{1.037541in}%
\pgfsys@useobject{currentmarker}{}%
\end{pgfscope}%
\begin{pgfscope}%
\pgfsys@transformshift{1.029833in}{0.929151in}%
\pgfsys@useobject{currentmarker}{}%
\end{pgfscope}%
\begin{pgfscope}%
\pgfsys@transformshift{1.349246in}{0.907282in}%
\pgfsys@useobject{currentmarker}{}%
\end{pgfscope}%
\begin{pgfscope}%
\pgfsys@transformshift{1.668659in}{0.934498in}%
\pgfsys@useobject{currentmarker}{}%
\end{pgfscope}%
\begin{pgfscope}%
\pgfsys@transformshift{1.988073in}{0.955324in}%
\pgfsys@useobject{currentmarker}{}%
\end{pgfscope}%
\begin{pgfscope}%
\pgfsys@transformshift{2.307486in}{0.927817in}%
\pgfsys@useobject{currentmarker}{}%
\end{pgfscope}%
\begin{pgfscope}%
\pgfsys@transformshift{2.626899in}{0.975948in}%
\pgfsys@useobject{currentmarker}{}%
\end{pgfscope}%
\begin{pgfscope}%
\pgfsys@transformshift{2.946312in}{1.041860in}%
\pgfsys@useobject{currentmarker}{}%
\end{pgfscope}%
\begin{pgfscope}%
\pgfsys@transformshift{3.265726in}{1.038215in}%
\pgfsys@useobject{currentmarker}{}%
\end{pgfscope}%
\begin{pgfscope}%
\pgfsys@transformshift{3.585139in}{1.126566in}%
\pgfsys@useobject{currentmarker}{}%
\end{pgfscope}%
\begin{pgfscope}%
\pgfsys@transformshift{3.904552in}{1.137026in}%
\pgfsys@useobject{currentmarker}{}%
\end{pgfscope}%
\begin{pgfscope}%
\pgfsys@transformshift{4.223966in}{1.187785in}%
\pgfsys@useobject{currentmarker}{}%
\end{pgfscope}%
\begin{pgfscope}%
\pgfsys@transformshift{4.543379in}{1.252660in}%
\pgfsys@useobject{currentmarker}{}%
\end{pgfscope}%
\begin{pgfscope}%
\pgfsys@transformshift{4.862792in}{1.465520in}%
\pgfsys@useobject{currentmarker}{}%
\end{pgfscope}%
\begin{pgfscope}%
\pgfsys@transformshift{5.182206in}{1.539208in}%
\pgfsys@useobject{currentmarker}{}%
\end{pgfscope}%
\end{pgfscope}%
\begin{pgfscope}%
\pgfpathrectangle{\pgfqpoint{0.550713in}{0.689227in}}{\pgfqpoint{4.791200in}{1.332187in}}%
\pgfusepath{clip}%
\pgfsetrectcap%
\pgfsetroundjoin%
\pgfsetlinewidth{1.505625pt}%
\definecolor{currentstroke}{rgb}{0.000000,0.329412,0.623529}%
\pgfsetstrokecolor{currentstroke}%
\pgfsetdash{}{0pt}%
\pgfpathmoveto{\pgfqpoint{0.710419in}{1.688818in}}%
\pgfpathlineto{\pgfqpoint{1.029833in}{1.313633in}}%
\pgfpathlineto{\pgfqpoint{1.349246in}{1.343164in}}%
\pgfpathlineto{\pgfqpoint{1.668659in}{1.104618in}}%
\pgfpathlineto{\pgfqpoint{1.988073in}{1.164116in}}%
\pgfpathlineto{\pgfqpoint{2.307486in}{1.312305in}}%
\pgfpathlineto{\pgfqpoint{2.626899in}{1.323319in}}%
\pgfpathlineto{\pgfqpoint{2.946312in}{1.327846in}}%
\pgfpathlineto{\pgfqpoint{3.265726in}{1.247375in}}%
\pgfpathlineto{\pgfqpoint{3.585139in}{1.168541in}}%
\pgfpathlineto{\pgfqpoint{3.904552in}{1.316097in}}%
\pgfpathlineto{\pgfqpoint{4.223966in}{1.313748in}}%
\pgfpathlineto{\pgfqpoint{4.543379in}{1.348605in}}%
\pgfpathlineto{\pgfqpoint{4.862792in}{1.794388in}}%
\pgfpathlineto{\pgfqpoint{5.044504in}{2.031414in}}%
\pgfusepath{stroke}%
\end{pgfscope}%
\begin{pgfscope}%
\pgfpathrectangle{\pgfqpoint{0.550713in}{0.689227in}}{\pgfqpoint{4.791200in}{1.332187in}}%
\pgfusepath{clip}%
\pgfsetbuttcap%
\pgfsetroundjoin%
\definecolor{currentfill}{rgb}{0.000000,0.329412,0.623529}%
\pgfsetfillcolor{currentfill}%
\pgfsetlinewidth{0.752812pt}%
\definecolor{currentstroke}{rgb}{0.000000,0.000000,0.000000}%
\pgfsetstrokecolor{currentstroke}%
\pgfsetdash{}{0pt}%
\pgfsys@defobject{currentmarker}{\pgfqpoint{-0.027778in}{-0.027778in}}{\pgfqpoint{0.027778in}{0.027778in}}{%
\pgfpathmoveto{\pgfqpoint{0.000000in}{-0.027778in}}%
\pgfpathcurveto{\pgfqpoint{0.007367in}{-0.027778in}}{\pgfqpoint{0.014433in}{-0.024851in}}{\pgfqpoint{0.019642in}{-0.019642in}}%
\pgfpathcurveto{\pgfqpoint{0.024851in}{-0.014433in}}{\pgfqpoint{0.027778in}{-0.007367in}}{\pgfqpoint{0.027778in}{0.000000in}}%
\pgfpathcurveto{\pgfqpoint{0.027778in}{0.007367in}}{\pgfqpoint{0.024851in}{0.014433in}}{\pgfqpoint{0.019642in}{0.019642in}}%
\pgfpathcurveto{\pgfqpoint{0.014433in}{0.024851in}}{\pgfqpoint{0.007367in}{0.027778in}}{\pgfqpoint{0.000000in}{0.027778in}}%
\pgfpathcurveto{\pgfqpoint{-0.007367in}{0.027778in}}{\pgfqpoint{-0.014433in}{0.024851in}}{\pgfqpoint{-0.019642in}{0.019642in}}%
\pgfpathcurveto{\pgfqpoint{-0.024851in}{0.014433in}}{\pgfqpoint{-0.027778in}{0.007367in}}{\pgfqpoint{-0.027778in}{0.000000in}}%
\pgfpathcurveto{\pgfqpoint{-0.027778in}{-0.007367in}}{\pgfqpoint{-0.024851in}{-0.014433in}}{\pgfqpoint{-0.019642in}{-0.019642in}}%
\pgfpathcurveto{\pgfqpoint{-0.014433in}{-0.024851in}}{\pgfqpoint{-0.007367in}{-0.027778in}}{\pgfqpoint{0.000000in}{-0.027778in}}%
\pgfpathclose%
\pgfusepath{stroke,fill}%
}%
\begin{pgfscope}%
\pgfsys@transformshift{0.710419in}{1.688818in}%
\pgfsys@useobject{currentmarker}{}%
\end{pgfscope}%
\begin{pgfscope}%
\pgfsys@transformshift{1.029833in}{1.313633in}%
\pgfsys@useobject{currentmarker}{}%
\end{pgfscope}%
\begin{pgfscope}%
\pgfsys@transformshift{1.349246in}{1.343164in}%
\pgfsys@useobject{currentmarker}{}%
\end{pgfscope}%
\begin{pgfscope}%
\pgfsys@transformshift{1.668659in}{1.104618in}%
\pgfsys@useobject{currentmarker}{}%
\end{pgfscope}%
\begin{pgfscope}%
\pgfsys@transformshift{1.988073in}{1.164116in}%
\pgfsys@useobject{currentmarker}{}%
\end{pgfscope}%
\begin{pgfscope}%
\pgfsys@transformshift{2.307486in}{1.312305in}%
\pgfsys@useobject{currentmarker}{}%
\end{pgfscope}%
\begin{pgfscope}%
\pgfsys@transformshift{2.626899in}{1.323319in}%
\pgfsys@useobject{currentmarker}{}%
\end{pgfscope}%
\begin{pgfscope}%
\pgfsys@transformshift{2.946312in}{1.327846in}%
\pgfsys@useobject{currentmarker}{}%
\end{pgfscope}%
\begin{pgfscope}%
\pgfsys@transformshift{3.265726in}{1.247375in}%
\pgfsys@useobject{currentmarker}{}%
\end{pgfscope}%
\begin{pgfscope}%
\pgfsys@transformshift{3.585139in}{1.168541in}%
\pgfsys@useobject{currentmarker}{}%
\end{pgfscope}%
\begin{pgfscope}%
\pgfsys@transformshift{3.904552in}{1.316097in}%
\pgfsys@useobject{currentmarker}{}%
\end{pgfscope}%
\begin{pgfscope}%
\pgfsys@transformshift{4.223966in}{1.313748in}%
\pgfsys@useobject{currentmarker}{}%
\end{pgfscope}%
\begin{pgfscope}%
\pgfsys@transformshift{4.543379in}{1.348605in}%
\pgfsys@useobject{currentmarker}{}%
\end{pgfscope}%
\begin{pgfscope}%
\pgfsys@transformshift{4.862792in}{1.794388in}%
\pgfsys@useobject{currentmarker}{}%
\end{pgfscope}%
\begin{pgfscope}%
\pgfsys@transformshift{5.182206in}{2.211033in}%
\pgfsys@useobject{currentmarker}{}%
\end{pgfscope}%
\end{pgfscope}%
\begin{pgfscope}%
\pgfsetrectcap%
\pgfsetmiterjoin%
\pgfsetlinewidth{0.803000pt}%
\definecolor{currentstroke}{rgb}{0.000000,0.000000,0.000000}%
\pgfsetstrokecolor{currentstroke}%
\pgfsetdash{}{0pt}%
\pgfpathmoveto{\pgfqpoint{0.550713in}{0.689227in}}%
\pgfpathlineto{\pgfqpoint{0.550713in}{2.021414in}}%
\pgfusepath{stroke}%
\end{pgfscope}%
\begin{pgfscope}%
\pgfsetrectcap%
\pgfsetmiterjoin%
\pgfsetlinewidth{0.803000pt}%
\definecolor{currentstroke}{rgb}{0.000000,0.000000,0.000000}%
\pgfsetstrokecolor{currentstroke}%
\pgfsetdash{}{0pt}%
\pgfpathmoveto{\pgfqpoint{5.341912in}{0.689227in}}%
\pgfpathlineto{\pgfqpoint{5.341912in}{2.021414in}}%
\pgfusepath{stroke}%
\end{pgfscope}%
\begin{pgfscope}%
\pgfsetrectcap%
\pgfsetmiterjoin%
\pgfsetlinewidth{0.803000pt}%
\definecolor{currentstroke}{rgb}{0.000000,0.000000,0.000000}%
\pgfsetstrokecolor{currentstroke}%
\pgfsetdash{}{0pt}%
\pgfpathmoveto{\pgfqpoint{0.550713in}{0.689227in}}%
\pgfpathlineto{\pgfqpoint{5.341912in}{0.689227in}}%
\pgfusepath{stroke}%
\end{pgfscope}%
\begin{pgfscope}%
\pgfsetrectcap%
\pgfsetmiterjoin%
\pgfsetlinewidth{0.803000pt}%
\definecolor{currentstroke}{rgb}{0.000000,0.000000,0.000000}%
\pgfsetstrokecolor{currentstroke}%
\pgfsetdash{}{0pt}%
\pgfpathmoveto{\pgfqpoint{0.550713in}{2.021414in}}%
\pgfpathlineto{\pgfqpoint{5.341912in}{2.021414in}}%
\pgfusepath{stroke}%
\end{pgfscope}%
\end{pgfpicture}%
\makeatother%
\endgroup%

    \caption[Sensitivity analysis of the forgetting factors of \gls{ui} and \gls{b2p} forgetting.]{Sensitivity analysis of the \gls{ui} and \gls{b2p} forgetting factors on the dynamic cumulative regret with the example of the moving parabola. The means (white markers) are compared in the graph at the bottom. \gls{ctvbo} with \gls{ui} forgetting results in the lowest regret for almost all forgetting factors.}
    \label{fig:Parabola1D_forgetting_factors}
\end{figure}

For \gls{ui} forgetting, applying \gls{ctvbo} reduces the regret for all forgetting factors as $\EX[f_t(\mathbf{\hat{x}}_t^*]$ does not propagate towards the prior mean supporting  Hypothesis~\ref{hyp:ctvbo}. Furthermore, \gls{ctvbo} reduces the regret's variance. Comparing the mean regret of \gls{ui} forgetting and \gls{b2p} forgetting, the bottom graph of Figure~\ref{fig:Parabola1D_forgetting_factors} shows that the combination of the proposed methods results in the lowest regret, expect for $\epsilon=\hat{\sigma}_w^2 = 0.005$ and $\epsilon=\hat{\sigma}_w^2 = 0.3$. Furthermore, the regret of \gls{ui} forgetting and \gls{b2p} forgetting at low forgetting factors is very similar, supporting  Hypothesis~\ref{hyp:ui_good_mean}.

To further test Hypothesis~\ref{hyp:ui_structural_information}, the forgetting factors with the lowest mean regret in the sensitivity analysis in Figure~\ref{fig:Parabola1D_forgetting_factors} are listed in Table~\ref{tab:1d} and further compared by also considering an optimistic prior mean of $\mu_0 = -1$. These trajectories are shown in Appendix~\ref{apx:trajectories_1D_parabola}.
\bgroup
\def\arraystretch{1.2}
\begin{table}[h]
\begin{center}
\begin{tabular}{ c || c}
\textbf{Variation} & \textbf{Forgetting Factor $\epsilon$ / $\hat{\sigma}_w^2$} \\\hline\hline
\gls{uitvbo}, B \gls{uitvbo}& $0.01$\\
TV-GP-UCB, SW TV-GP-UCB & $0.028$\\
C-\gls{uitvbo}, B C-\gls{uitvbo} & $0.009$ \\ 
C-TV-GP-UCB, SW C-TV-GP-UCB & $0.009$
\end{tabular}
\end{center}
\caption{Forgetting factors with the lowest regret in the sensitivity analysis.}
\label{tab:1d}
\end{table}
\egroup

For the variations with data selection strategy, the same forgetting factors are chosen as without data selection strategy. The results of the direct comparison the well-defined prior mean $\mu_0=0$ and the optimistic prior mean $\mu_0 = -1$ are shown in Figure~\ref{fig:Parabola1D_cumulative_regret}.
\begin{figure}[h!]
    \centering
    %% Creator: Matplotlib, PGF backend
%%
%% To include the figure in your LaTeX document, write
%%   \input{<filename>.pgf}
%%
%% Make sure the required packages are loaded in your preamble
%%   \usepackage{pgf}
%%
%% Figures using additional raster images can only be included by \input if
%% they are in the same directory as the main LaTeX file. For loading figures
%% from other directories you can use the `import` package
%%   \usepackage{import}
%%
%% and then include the figures with
%%   \import{<path to file>}{<filename>.pgf}
%%
%% Matplotlib used the following preamble
%%   \usepackage{fontspec}
%%
\begingroup%
\makeatletter%
\begin{pgfpicture}%
\pgfpathrectangle{\pgfpointorigin}{\pgfqpoint{5.507126in}{2.552693in}}%
\pgfusepath{use as bounding box, clip}%
\begin{pgfscope}%
\pgfsetbuttcap%
\pgfsetmiterjoin%
\definecolor{currentfill}{rgb}{1.000000,1.000000,1.000000}%
\pgfsetfillcolor{currentfill}%
\pgfsetlinewidth{0.000000pt}%
\definecolor{currentstroke}{rgb}{1.000000,1.000000,1.000000}%
\pgfsetstrokecolor{currentstroke}%
\pgfsetdash{}{0pt}%
\pgfpathmoveto{\pgfqpoint{0.000000in}{0.000000in}}%
\pgfpathlineto{\pgfqpoint{5.507126in}{0.000000in}}%
\pgfpathlineto{\pgfqpoint{5.507126in}{2.552693in}}%
\pgfpathlineto{\pgfqpoint{0.000000in}{2.552693in}}%
\pgfpathclose%
\pgfusepath{fill}%
\end{pgfscope}%
\begin{pgfscope}%
\pgfsetbuttcap%
\pgfsetmiterjoin%
\definecolor{currentfill}{rgb}{1.000000,1.000000,1.000000}%
\pgfsetfillcolor{currentfill}%
\pgfsetlinewidth{0.000000pt}%
\definecolor{currentstroke}{rgb}{0.000000,0.000000,0.000000}%
\pgfsetstrokecolor{currentstroke}%
\pgfsetstrokeopacity{0.000000}%
\pgfsetdash{}{0pt}%
\pgfpathmoveto{\pgfqpoint{0.550713in}{0.127635in}}%
\pgfpathlineto{\pgfqpoint{3.744846in}{0.127635in}}%
\pgfpathlineto{\pgfqpoint{3.744846in}{2.425059in}}%
\pgfpathlineto{\pgfqpoint{0.550713in}{2.425059in}}%
\pgfpathclose%
\pgfusepath{fill}%
\end{pgfscope}%
\begin{pgfscope}%
\pgfpathrectangle{\pgfqpoint{0.550713in}{0.127635in}}{\pgfqpoint{3.194133in}{2.297424in}}%
\pgfusepath{clip}%
\pgfsetbuttcap%
\pgfsetmiterjoin%
\definecolor{currentfill}{rgb}{0.631373,0.062745,0.207843}%
\pgfsetfillcolor{currentfill}%
\pgfsetlinewidth{0.752812pt}%
\definecolor{currentstroke}{rgb}{0.000000,0.000000,0.000000}%
\pgfsetstrokecolor{currentstroke}%
\pgfsetdash{}{0pt}%
\pgfpathmoveto{\pgfqpoint{0.592236in}{0.725996in}}%
\pgfpathlineto{\pgfqpoint{0.748749in}{0.725996in}}%
\pgfpathlineto{\pgfqpoint{0.748749in}{0.874892in}}%
\pgfpathlineto{\pgfqpoint{0.592236in}{0.874892in}}%
\pgfpathlineto{\pgfqpoint{0.592236in}{0.725996in}}%
\pgfpathclose%
\pgfusepath{stroke,fill}%
\end{pgfscope}%
\begin{pgfscope}%
\pgfpathrectangle{\pgfqpoint{0.550713in}{0.127635in}}{\pgfqpoint{3.194133in}{2.297424in}}%
\pgfusepath{clip}%
\pgfsetbuttcap%
\pgfsetmiterjoin%
\definecolor{currentfill}{rgb}{0.898039,0.772549,0.752941}%
\pgfsetfillcolor{currentfill}%
\pgfsetlinewidth{0.752812pt}%
\definecolor{currentstroke}{rgb}{0.000000,0.000000,0.000000}%
\pgfsetstrokecolor{currentstroke}%
\pgfsetdash{}{0pt}%
\pgfpathmoveto{\pgfqpoint{0.751943in}{1.046819in}}%
\pgfpathlineto{\pgfqpoint{0.908456in}{1.046819in}}%
\pgfpathlineto{\pgfqpoint{0.908456in}{1.360423in}}%
\pgfpathlineto{\pgfqpoint{0.751943in}{1.360423in}}%
\pgfpathlineto{\pgfqpoint{0.751943in}{1.046819in}}%
\pgfpathclose%
\pgfusepath{stroke,fill}%
\end{pgfscope}%
\begin{pgfscope}%
\pgfpathrectangle{\pgfqpoint{0.550713in}{0.127635in}}{\pgfqpoint{3.194133in}{2.297424in}}%
\pgfusepath{clip}%
\pgfsetbuttcap%
\pgfsetmiterjoin%
\definecolor{currentfill}{rgb}{0.890196,0.000000,0.400000}%
\pgfsetfillcolor{currentfill}%
\pgfsetlinewidth{0.752812pt}%
\definecolor{currentstroke}{rgb}{0.000000,0.000000,0.000000}%
\pgfsetstrokecolor{currentstroke}%
\pgfsetdash{}{0pt}%
\pgfpathmoveto{\pgfqpoint{0.991503in}{1.901825in}}%
\pgfpathlineto{\pgfqpoint{1.148015in}{1.901825in}}%
\pgfpathlineto{\pgfqpoint{1.148015in}{2.161223in}}%
\pgfpathlineto{\pgfqpoint{0.991503in}{2.161223in}}%
\pgfpathlineto{\pgfqpoint{0.991503in}{1.901825in}}%
\pgfpathclose%
\pgfusepath{stroke,fill}%
\end{pgfscope}%
\begin{pgfscope}%
\pgfpathrectangle{\pgfqpoint{0.550713in}{0.127635in}}{\pgfqpoint{3.194133in}{2.297424in}}%
\pgfusepath{clip}%
\pgfsetbuttcap%
\pgfsetmiterjoin%
\definecolor{currentfill}{rgb}{0.976471,0.823529,0.854902}%
\pgfsetfillcolor{currentfill}%
\pgfsetlinewidth{0.752812pt}%
\definecolor{currentstroke}{rgb}{0.000000,0.000000,0.000000}%
\pgfsetstrokecolor{currentstroke}%
\pgfsetdash{}{0pt}%
\pgfpathmoveto{\pgfqpoint{1.151210in}{1.910175in}}%
\pgfpathlineto{\pgfqpoint{1.307722in}{1.910175in}}%
\pgfpathlineto{\pgfqpoint{1.307722in}{2.166913in}}%
\pgfpathlineto{\pgfqpoint{1.151210in}{2.166913in}}%
\pgfpathlineto{\pgfqpoint{1.151210in}{1.910175in}}%
\pgfpathclose%
\pgfusepath{stroke,fill}%
\end{pgfscope}%
\begin{pgfscope}%
\pgfpathrectangle{\pgfqpoint{0.550713in}{0.127635in}}{\pgfqpoint{3.194133in}{2.297424in}}%
\pgfusepath{clip}%
\pgfsetbuttcap%
\pgfsetmiterjoin%
\definecolor{currentfill}{rgb}{0.000000,0.329412,0.623529}%
\pgfsetfillcolor{currentfill}%
\pgfsetlinewidth{0.752812pt}%
\definecolor{currentstroke}{rgb}{0.000000,0.000000,0.000000}%
\pgfsetstrokecolor{currentstroke}%
\pgfsetdash{}{0pt}%
\pgfpathmoveto{\pgfqpoint{1.390770in}{0.643228in}}%
\pgfpathlineto{\pgfqpoint{1.547282in}{0.643228in}}%
\pgfpathlineto{\pgfqpoint{1.547282in}{0.803699in}}%
\pgfpathlineto{\pgfqpoint{1.390770in}{0.803699in}}%
\pgfpathlineto{\pgfqpoint{1.390770in}{0.643228in}}%
\pgfpathclose%
\pgfusepath{stroke,fill}%
\end{pgfscope}%
\begin{pgfscope}%
\pgfpathrectangle{\pgfqpoint{0.550713in}{0.127635in}}{\pgfqpoint{3.194133in}{2.297424in}}%
\pgfusepath{clip}%
\pgfsetbuttcap%
\pgfsetmiterjoin%
\definecolor{currentfill}{rgb}{0.780392,0.866667,0.949020}%
\pgfsetfillcolor{currentfill}%
\pgfsetlinewidth{0.752812pt}%
\definecolor{currentstroke}{rgb}{0.000000,0.000000,0.000000}%
\pgfsetstrokecolor{currentstroke}%
\pgfsetdash{}{0pt}%
\pgfpathmoveto{\pgfqpoint{1.550476in}{0.637753in}}%
\pgfpathlineto{\pgfqpoint{1.706989in}{0.637753in}}%
\pgfpathlineto{\pgfqpoint{1.706989in}{0.956461in}}%
\pgfpathlineto{\pgfqpoint{1.550476in}{0.956461in}}%
\pgfpathlineto{\pgfqpoint{1.550476in}{0.637753in}}%
\pgfpathclose%
\pgfusepath{stroke,fill}%
\end{pgfscope}%
\begin{pgfscope}%
\pgfpathrectangle{\pgfqpoint{0.550713in}{0.127635in}}{\pgfqpoint{3.194133in}{2.297424in}}%
\pgfusepath{clip}%
\pgfsetbuttcap%
\pgfsetmiterjoin%
\definecolor{currentfill}{rgb}{0.000000,0.380392,0.396078}%
\pgfsetfillcolor{currentfill}%
\pgfsetlinewidth{0.752812pt}%
\definecolor{currentstroke}{rgb}{0.000000,0.000000,0.000000}%
\pgfsetstrokecolor{currentstroke}%
\pgfsetdash{}{0pt}%
\pgfpathmoveto{\pgfqpoint{1.790036in}{1.032313in}}%
\pgfpathlineto{\pgfqpoint{1.946549in}{1.032313in}}%
\pgfpathlineto{\pgfqpoint{1.946549in}{1.158237in}}%
\pgfpathlineto{\pgfqpoint{1.790036in}{1.158237in}}%
\pgfpathlineto{\pgfqpoint{1.790036in}{1.032313in}}%
\pgfpathclose%
\pgfusepath{stroke,fill}%
\end{pgfscope}%
\begin{pgfscope}%
\pgfpathrectangle{\pgfqpoint{0.550713in}{0.127635in}}{\pgfqpoint{3.194133in}{2.297424in}}%
\pgfusepath{clip}%
\pgfsetbuttcap%
\pgfsetmiterjoin%
\definecolor{currentfill}{rgb}{0.749020,0.815686,0.819608}%
\pgfsetfillcolor{currentfill}%
\pgfsetlinewidth{0.752812pt}%
\definecolor{currentstroke}{rgb}{0.000000,0.000000,0.000000}%
\pgfsetstrokecolor{currentstroke}%
\pgfsetdash{}{0pt}%
\pgfpathmoveto{\pgfqpoint{1.949743in}{0.826105in}}%
\pgfpathlineto{\pgfqpoint{2.106255in}{0.826105in}}%
\pgfpathlineto{\pgfqpoint{2.106255in}{1.154033in}}%
\pgfpathlineto{\pgfqpoint{1.949743in}{1.154033in}}%
\pgfpathlineto{\pgfqpoint{1.949743in}{0.826105in}}%
\pgfpathclose%
\pgfusepath{stroke,fill}%
\end{pgfscope}%
\begin{pgfscope}%
\pgfpathrectangle{\pgfqpoint{0.550713in}{0.127635in}}{\pgfqpoint{3.194133in}{2.297424in}}%
\pgfusepath{clip}%
\pgfsetbuttcap%
\pgfsetmiterjoin%
\definecolor{currentfill}{rgb}{0.380392,0.129412,0.345098}%
\pgfsetfillcolor{currentfill}%
\pgfsetlinewidth{0.752812pt}%
\definecolor{currentstroke}{rgb}{0.000000,0.000000,0.000000}%
\pgfsetstrokecolor{currentstroke}%
\pgfsetdash{}{0pt}%
\pgfpathmoveto{\pgfqpoint{2.189303in}{0.485039in}}%
\pgfpathlineto{\pgfqpoint{2.345815in}{0.485039in}}%
\pgfpathlineto{\pgfqpoint{2.345815in}{0.541355in}}%
\pgfpathlineto{\pgfqpoint{2.189303in}{0.541355in}}%
\pgfpathlineto{\pgfqpoint{2.189303in}{0.485039in}}%
\pgfpathclose%
\pgfusepath{stroke,fill}%
\end{pgfscope}%
\begin{pgfscope}%
\pgfpathrectangle{\pgfqpoint{0.550713in}{0.127635in}}{\pgfqpoint{3.194133in}{2.297424in}}%
\pgfusepath{clip}%
\pgfsetbuttcap%
\pgfsetmiterjoin%
\definecolor{currentfill}{rgb}{0.823529,0.752941,0.803922}%
\pgfsetfillcolor{currentfill}%
\pgfsetlinewidth{0.752812pt}%
\definecolor{currentstroke}{rgb}{0.000000,0.000000,0.000000}%
\pgfsetstrokecolor{currentstroke}%
\pgfsetdash{}{0pt}%
\pgfpathmoveto{\pgfqpoint{2.349010in}{0.522259in}}%
\pgfpathlineto{\pgfqpoint{2.505522in}{0.522259in}}%
\pgfpathlineto{\pgfqpoint{2.505522in}{0.604354in}}%
\pgfpathlineto{\pgfqpoint{2.349010in}{0.604354in}}%
\pgfpathlineto{\pgfqpoint{2.349010in}{0.522259in}}%
\pgfpathclose%
\pgfusepath{stroke,fill}%
\end{pgfscope}%
\begin{pgfscope}%
\pgfpathrectangle{\pgfqpoint{0.550713in}{0.127635in}}{\pgfqpoint{3.194133in}{2.297424in}}%
\pgfusepath{clip}%
\pgfsetbuttcap%
\pgfsetmiterjoin%
\definecolor{currentfill}{rgb}{0.964706,0.658824,0.000000}%
\pgfsetfillcolor{currentfill}%
\pgfsetlinewidth{0.752812pt}%
\definecolor{currentstroke}{rgb}{0.000000,0.000000,0.000000}%
\pgfsetstrokecolor{currentstroke}%
\pgfsetdash{}{0pt}%
\pgfpathmoveto{\pgfqpoint{2.588570in}{0.636402in}}%
\pgfpathlineto{\pgfqpoint{2.745082in}{0.636402in}}%
\pgfpathlineto{\pgfqpoint{2.745082in}{0.674544in}}%
\pgfpathlineto{\pgfqpoint{2.588570in}{0.674544in}}%
\pgfpathlineto{\pgfqpoint{2.588570in}{0.636402in}}%
\pgfpathclose%
\pgfusepath{stroke,fill}%
\end{pgfscope}%
\begin{pgfscope}%
\pgfpathrectangle{\pgfqpoint{0.550713in}{0.127635in}}{\pgfqpoint{3.194133in}{2.297424in}}%
\pgfusepath{clip}%
\pgfsetbuttcap%
\pgfsetmiterjoin%
\definecolor{currentfill}{rgb}{0.996078,0.917647,0.788235}%
\pgfsetfillcolor{currentfill}%
\pgfsetlinewidth{0.752812pt}%
\definecolor{currentstroke}{rgb}{0.000000,0.000000,0.000000}%
\pgfsetstrokecolor{currentstroke}%
\pgfsetdash{}{0pt}%
\pgfpathmoveto{\pgfqpoint{2.748276in}{0.610462in}}%
\pgfpathlineto{\pgfqpoint{2.904789in}{0.610462in}}%
\pgfpathlineto{\pgfqpoint{2.904789in}{0.772525in}}%
\pgfpathlineto{\pgfqpoint{2.748276in}{0.772525in}}%
\pgfpathlineto{\pgfqpoint{2.748276in}{0.610462in}}%
\pgfpathclose%
\pgfusepath{stroke,fill}%
\end{pgfscope}%
\begin{pgfscope}%
\pgfpathrectangle{\pgfqpoint{0.550713in}{0.127635in}}{\pgfqpoint{3.194133in}{2.297424in}}%
\pgfusepath{clip}%
\pgfsetbuttcap%
\pgfsetmiterjoin%
\definecolor{currentfill}{rgb}{0.341176,0.670588,0.152941}%
\pgfsetfillcolor{currentfill}%
\pgfsetlinewidth{0.752812pt}%
\definecolor{currentstroke}{rgb}{0.000000,0.000000,0.000000}%
\pgfsetstrokecolor{currentstroke}%
\pgfsetdash{}{0pt}%
\pgfpathmoveto{\pgfqpoint{2.987836in}{0.419047in}}%
\pgfpathlineto{\pgfqpoint{3.144349in}{0.419047in}}%
\pgfpathlineto{\pgfqpoint{3.144349in}{0.556683in}}%
\pgfpathlineto{\pgfqpoint{2.987836in}{0.556683in}}%
\pgfpathlineto{\pgfqpoint{2.987836in}{0.419047in}}%
\pgfpathclose%
\pgfusepath{stroke,fill}%
\end{pgfscope}%
\begin{pgfscope}%
\pgfpathrectangle{\pgfqpoint{0.550713in}{0.127635in}}{\pgfqpoint{3.194133in}{2.297424in}}%
\pgfusepath{clip}%
\pgfsetbuttcap%
\pgfsetmiterjoin%
\definecolor{currentfill}{rgb}{0.866667,0.921569,0.807843}%
\pgfsetfillcolor{currentfill}%
\pgfsetlinewidth{0.752812pt}%
\definecolor{currentstroke}{rgb}{0.000000,0.000000,0.000000}%
\pgfsetstrokecolor{currentstroke}%
\pgfsetdash{}{0pt}%
\pgfpathmoveto{\pgfqpoint{3.147543in}{0.428743in}}%
\pgfpathlineto{\pgfqpoint{3.304055in}{0.428743in}}%
\pgfpathlineto{\pgfqpoint{3.304055in}{0.495873in}}%
\pgfpathlineto{\pgfqpoint{3.147543in}{0.495873in}}%
\pgfpathlineto{\pgfqpoint{3.147543in}{0.428743in}}%
\pgfpathclose%
\pgfusepath{stroke,fill}%
\end{pgfscope}%
\begin{pgfscope}%
\pgfpathrectangle{\pgfqpoint{0.550713in}{0.127635in}}{\pgfqpoint{3.194133in}{2.297424in}}%
\pgfusepath{clip}%
\pgfsetbuttcap%
\pgfsetmiterjoin%
\definecolor{currentfill}{rgb}{0.478431,0.435294,0.674510}%
\pgfsetfillcolor{currentfill}%
\pgfsetlinewidth{0.752812pt}%
\definecolor{currentstroke}{rgb}{0.000000,0.000000,0.000000}%
\pgfsetstrokecolor{currentstroke}%
\pgfsetdash{}{0pt}%
\pgfpathmoveto{\pgfqpoint{3.387103in}{0.631672in}}%
\pgfpathlineto{\pgfqpoint{3.543615in}{0.631672in}}%
\pgfpathlineto{\pgfqpoint{3.543615in}{0.673311in}}%
\pgfpathlineto{\pgfqpoint{3.387103in}{0.673311in}}%
\pgfpathlineto{\pgfqpoint{3.387103in}{0.631672in}}%
\pgfpathclose%
\pgfusepath{stroke,fill}%
\end{pgfscope}%
\begin{pgfscope}%
\pgfpathrectangle{\pgfqpoint{0.550713in}{0.127635in}}{\pgfqpoint{3.194133in}{2.297424in}}%
\pgfusepath{clip}%
\pgfsetbuttcap%
\pgfsetmiterjoin%
\definecolor{currentfill}{rgb}{0.870588,0.854902,0.921569}%
\pgfsetfillcolor{currentfill}%
\pgfsetlinewidth{0.752812pt}%
\definecolor{currentstroke}{rgb}{0.000000,0.000000,0.000000}%
\pgfsetstrokecolor{currentstroke}%
\pgfsetdash{}{0pt}%
\pgfpathmoveto{\pgfqpoint{3.546809in}{0.590000in}}%
\pgfpathlineto{\pgfqpoint{3.703322in}{0.590000in}}%
\pgfpathlineto{\pgfqpoint{3.703322in}{0.714501in}}%
\pgfpathlineto{\pgfqpoint{3.546809in}{0.714501in}}%
\pgfpathlineto{\pgfqpoint{3.546809in}{0.590000in}}%
\pgfpathclose%
\pgfusepath{stroke,fill}%
\end{pgfscope}%
\begin{pgfscope}%
\pgfpathrectangle{\pgfqpoint{0.550713in}{0.127635in}}{\pgfqpoint{3.194133in}{2.297424in}}%
\pgfusepath{clip}%
\pgfsetbuttcap%
\pgfsetmiterjoin%
\definecolor{currentfill}{rgb}{0.000000,0.000000,0.000000}%
\pgfsetfillcolor{currentfill}%
\pgfsetlinewidth{0.376406pt}%
\definecolor{currentstroke}{rgb}{0.000000,0.000000,0.000000}%
\pgfsetstrokecolor{currentstroke}%
\pgfsetdash{}{0pt}%
\pgfpathmoveto{\pgfqpoint{0.750346in}{0.127635in}}%
\pgfpathlineto{\pgfqpoint{0.750346in}{0.127635in}}%
\pgfpathlineto{\pgfqpoint{0.750346in}{0.127635in}}%
\pgfpathlineto{\pgfqpoint{0.750346in}{0.127635in}}%
\pgfpathclose%
\pgfusepath{stroke,fill}%
\end{pgfscope}%
\begin{pgfscope}%
\pgfpathrectangle{\pgfqpoint{0.550713in}{0.127635in}}{\pgfqpoint{3.194133in}{2.297424in}}%
\pgfusepath{clip}%
\pgfsetbuttcap%
\pgfsetmiterjoin%
\definecolor{currentfill}{rgb}{0.813235,0.819118,0.822059}%
\pgfsetfillcolor{currentfill}%
\pgfsetlinewidth{0.376406pt}%
\definecolor{currentstroke}{rgb}{0.000000,0.000000,0.000000}%
\pgfsetstrokecolor{currentstroke}%
\pgfsetdash{}{0pt}%
\pgfpathmoveto{\pgfqpoint{0.750346in}{0.127635in}}%
\pgfpathlineto{\pgfqpoint{0.750346in}{0.127635in}}%
\pgfpathlineto{\pgfqpoint{0.750346in}{0.127635in}}%
\pgfpathlineto{\pgfqpoint{0.750346in}{0.127635in}}%
\pgfpathclose%
\pgfusepath{stroke,fill}%
\end{pgfscope}%
\begin{pgfscope}%
\pgfsetbuttcap%
\pgfsetroundjoin%
\definecolor{currentfill}{rgb}{0.000000,0.000000,0.000000}%
\pgfsetfillcolor{currentfill}%
\pgfsetlinewidth{0.803000pt}%
\definecolor{currentstroke}{rgb}{0.000000,0.000000,0.000000}%
\pgfsetstrokecolor{currentstroke}%
\pgfsetdash{}{0pt}%
\pgfsys@defobject{currentmarker}{\pgfqpoint{-0.048611in}{0.000000in}}{\pgfqpoint{-0.000000in}{0.000000in}}{%
\pgfpathmoveto{\pgfqpoint{-0.000000in}{0.000000in}}%
\pgfpathlineto{\pgfqpoint{-0.048611in}{0.000000in}}%
\pgfusepath{stroke,fill}%
}%
\begin{pgfscope}%
\pgfsys@transformshift{0.550713in}{0.127635in}%
\pgfsys@useobject{currentmarker}{}%
\end{pgfscope}%
\end{pgfscope}%
\begin{pgfscope}%
\definecolor{textcolor}{rgb}{0.000000,0.000000,0.000000}%
\pgfsetstrokecolor{textcolor}%
\pgfsetfillcolor{textcolor}%
\pgftext[x=0.384046in, y=0.079440in, left, base]{\color{textcolor}\rmfamily\fontsize{10.000000}{12.000000}\selectfont \(\displaystyle {0}\)}%
\end{pgfscope}%
\begin{pgfscope}%
\pgfsetbuttcap%
\pgfsetroundjoin%
\definecolor{currentfill}{rgb}{0.000000,0.000000,0.000000}%
\pgfsetfillcolor{currentfill}%
\pgfsetlinewidth{0.803000pt}%
\definecolor{currentstroke}{rgb}{0.000000,0.000000,0.000000}%
\pgfsetstrokecolor{currentstroke}%
\pgfsetdash{}{0pt}%
\pgfsys@defobject{currentmarker}{\pgfqpoint{-0.048611in}{0.000000in}}{\pgfqpoint{-0.000000in}{0.000000in}}{%
\pgfpathmoveto{\pgfqpoint{-0.000000in}{0.000000in}}%
\pgfpathlineto{\pgfqpoint{-0.048611in}{0.000000in}}%
\pgfusepath{stroke,fill}%
}%
\begin{pgfscope}%
\pgfsys@transformshift{0.550713in}{0.455838in}%
\pgfsys@useobject{currentmarker}{}%
\end{pgfscope}%
\end{pgfscope}%
\begin{pgfscope}%
\definecolor{textcolor}{rgb}{0.000000,0.000000,0.000000}%
\pgfsetstrokecolor{textcolor}%
\pgfsetfillcolor{textcolor}%
\pgftext[x=0.314601in, y=0.407644in, left, base]{\color{textcolor}\rmfamily\fontsize{10.000000}{12.000000}\selectfont \(\displaystyle {50}\)}%
\end{pgfscope}%
\begin{pgfscope}%
\pgfsetbuttcap%
\pgfsetroundjoin%
\definecolor{currentfill}{rgb}{0.000000,0.000000,0.000000}%
\pgfsetfillcolor{currentfill}%
\pgfsetlinewidth{0.803000pt}%
\definecolor{currentstroke}{rgb}{0.000000,0.000000,0.000000}%
\pgfsetstrokecolor{currentstroke}%
\pgfsetdash{}{0pt}%
\pgfsys@defobject{currentmarker}{\pgfqpoint{-0.048611in}{0.000000in}}{\pgfqpoint{-0.000000in}{0.000000in}}{%
\pgfpathmoveto{\pgfqpoint{-0.000000in}{0.000000in}}%
\pgfpathlineto{\pgfqpoint{-0.048611in}{0.000000in}}%
\pgfusepath{stroke,fill}%
}%
\begin{pgfscope}%
\pgfsys@transformshift{0.550713in}{0.784042in}%
\pgfsys@useobject{currentmarker}{}%
\end{pgfscope}%
\end{pgfscope}%
\begin{pgfscope}%
\definecolor{textcolor}{rgb}{0.000000,0.000000,0.000000}%
\pgfsetstrokecolor{textcolor}%
\pgfsetfillcolor{textcolor}%
\pgftext[x=0.245156in, y=0.735847in, left, base]{\color{textcolor}\rmfamily\fontsize{10.000000}{12.000000}\selectfont \(\displaystyle {100}\)}%
\end{pgfscope}%
\begin{pgfscope}%
\pgfsetbuttcap%
\pgfsetroundjoin%
\definecolor{currentfill}{rgb}{0.000000,0.000000,0.000000}%
\pgfsetfillcolor{currentfill}%
\pgfsetlinewidth{0.803000pt}%
\definecolor{currentstroke}{rgb}{0.000000,0.000000,0.000000}%
\pgfsetstrokecolor{currentstroke}%
\pgfsetdash{}{0pt}%
\pgfsys@defobject{currentmarker}{\pgfqpoint{-0.048611in}{0.000000in}}{\pgfqpoint{-0.000000in}{0.000000in}}{%
\pgfpathmoveto{\pgfqpoint{-0.000000in}{0.000000in}}%
\pgfpathlineto{\pgfqpoint{-0.048611in}{0.000000in}}%
\pgfusepath{stroke,fill}%
}%
\begin{pgfscope}%
\pgfsys@transformshift{0.550713in}{1.112245in}%
\pgfsys@useobject{currentmarker}{}%
\end{pgfscope}%
\end{pgfscope}%
\begin{pgfscope}%
\definecolor{textcolor}{rgb}{0.000000,0.000000,0.000000}%
\pgfsetstrokecolor{textcolor}%
\pgfsetfillcolor{textcolor}%
\pgftext[x=0.245156in, y=1.064050in, left, base]{\color{textcolor}\rmfamily\fontsize{10.000000}{12.000000}\selectfont \(\displaystyle {150}\)}%
\end{pgfscope}%
\begin{pgfscope}%
\pgfsetbuttcap%
\pgfsetroundjoin%
\definecolor{currentfill}{rgb}{0.000000,0.000000,0.000000}%
\pgfsetfillcolor{currentfill}%
\pgfsetlinewidth{0.803000pt}%
\definecolor{currentstroke}{rgb}{0.000000,0.000000,0.000000}%
\pgfsetstrokecolor{currentstroke}%
\pgfsetdash{}{0pt}%
\pgfsys@defobject{currentmarker}{\pgfqpoint{-0.048611in}{0.000000in}}{\pgfqpoint{-0.000000in}{0.000000in}}{%
\pgfpathmoveto{\pgfqpoint{-0.000000in}{0.000000in}}%
\pgfpathlineto{\pgfqpoint{-0.048611in}{0.000000in}}%
\pgfusepath{stroke,fill}%
}%
\begin{pgfscope}%
\pgfsys@transformshift{0.550713in}{1.440448in}%
\pgfsys@useobject{currentmarker}{}%
\end{pgfscope}%
\end{pgfscope}%
\begin{pgfscope}%
\definecolor{textcolor}{rgb}{0.000000,0.000000,0.000000}%
\pgfsetstrokecolor{textcolor}%
\pgfsetfillcolor{textcolor}%
\pgftext[x=0.245156in, y=1.392254in, left, base]{\color{textcolor}\rmfamily\fontsize{10.000000}{12.000000}\selectfont \(\displaystyle {200}\)}%
\end{pgfscope}%
\begin{pgfscope}%
\pgfsetbuttcap%
\pgfsetroundjoin%
\definecolor{currentfill}{rgb}{0.000000,0.000000,0.000000}%
\pgfsetfillcolor{currentfill}%
\pgfsetlinewidth{0.803000pt}%
\definecolor{currentstroke}{rgb}{0.000000,0.000000,0.000000}%
\pgfsetstrokecolor{currentstroke}%
\pgfsetdash{}{0pt}%
\pgfsys@defobject{currentmarker}{\pgfqpoint{-0.048611in}{0.000000in}}{\pgfqpoint{-0.000000in}{0.000000in}}{%
\pgfpathmoveto{\pgfqpoint{-0.000000in}{0.000000in}}%
\pgfpathlineto{\pgfqpoint{-0.048611in}{0.000000in}}%
\pgfusepath{stroke,fill}%
}%
\begin{pgfscope}%
\pgfsys@transformshift{0.550713in}{1.768652in}%
\pgfsys@useobject{currentmarker}{}%
\end{pgfscope}%
\end{pgfscope}%
\begin{pgfscope}%
\definecolor{textcolor}{rgb}{0.000000,0.000000,0.000000}%
\pgfsetstrokecolor{textcolor}%
\pgfsetfillcolor{textcolor}%
\pgftext[x=0.245156in, y=1.720457in, left, base]{\color{textcolor}\rmfamily\fontsize{10.000000}{12.000000}\selectfont \(\displaystyle {250}\)}%
\end{pgfscope}%
\begin{pgfscope}%
\pgfsetbuttcap%
\pgfsetroundjoin%
\definecolor{currentfill}{rgb}{0.000000,0.000000,0.000000}%
\pgfsetfillcolor{currentfill}%
\pgfsetlinewidth{0.803000pt}%
\definecolor{currentstroke}{rgb}{0.000000,0.000000,0.000000}%
\pgfsetstrokecolor{currentstroke}%
\pgfsetdash{}{0pt}%
\pgfsys@defobject{currentmarker}{\pgfqpoint{-0.048611in}{0.000000in}}{\pgfqpoint{-0.000000in}{0.000000in}}{%
\pgfpathmoveto{\pgfqpoint{-0.000000in}{0.000000in}}%
\pgfpathlineto{\pgfqpoint{-0.048611in}{0.000000in}}%
\pgfusepath{stroke,fill}%
}%
\begin{pgfscope}%
\pgfsys@transformshift{0.550713in}{2.096855in}%
\pgfsys@useobject{currentmarker}{}%
\end{pgfscope}%
\end{pgfscope}%
\begin{pgfscope}%
\definecolor{textcolor}{rgb}{0.000000,0.000000,0.000000}%
\pgfsetstrokecolor{textcolor}%
\pgfsetfillcolor{textcolor}%
\pgftext[x=0.245156in, y=2.048661in, left, base]{\color{textcolor}\rmfamily\fontsize{10.000000}{12.000000}\selectfont \(\displaystyle {300}\)}%
\end{pgfscope}%
\begin{pgfscope}%
\pgfsetbuttcap%
\pgfsetroundjoin%
\definecolor{currentfill}{rgb}{0.000000,0.000000,0.000000}%
\pgfsetfillcolor{currentfill}%
\pgfsetlinewidth{0.803000pt}%
\definecolor{currentstroke}{rgb}{0.000000,0.000000,0.000000}%
\pgfsetstrokecolor{currentstroke}%
\pgfsetdash{}{0pt}%
\pgfsys@defobject{currentmarker}{\pgfqpoint{-0.048611in}{0.000000in}}{\pgfqpoint{-0.000000in}{0.000000in}}{%
\pgfpathmoveto{\pgfqpoint{-0.000000in}{0.000000in}}%
\pgfpathlineto{\pgfqpoint{-0.048611in}{0.000000in}}%
\pgfusepath{stroke,fill}%
}%
\begin{pgfscope}%
\pgfsys@transformshift{0.550713in}{2.425059in}%
\pgfsys@useobject{currentmarker}{}%
\end{pgfscope}%
\end{pgfscope}%
\begin{pgfscope}%
\definecolor{textcolor}{rgb}{0.000000,0.000000,0.000000}%
\pgfsetstrokecolor{textcolor}%
\pgfsetfillcolor{textcolor}%
\pgftext[x=0.245156in, y=2.376864in, left, base]{\color{textcolor}\rmfamily\fontsize{10.000000}{12.000000}\selectfont \(\displaystyle {350}\)}%
\end{pgfscope}%
\begin{pgfscope}%
\definecolor{textcolor}{rgb}{0.000000,0.000000,0.000000}%
\pgfsetstrokecolor{textcolor}%
\pgfsetfillcolor{textcolor}%
\pgftext[x=0.189601in,y=1.276347in,,bottom,rotate=90.000000]{\color{textcolor}\rmfamily\fontsize{10.000000}{12.000000}\selectfont \(\displaystyle R_T\)}%
\end{pgfscope}%
\begin{pgfscope}%
\pgfpathrectangle{\pgfqpoint{0.550713in}{0.127635in}}{\pgfqpoint{3.194133in}{2.297424in}}%
\pgfusepath{clip}%
\pgfsetbuttcap%
\pgfsetroundjoin%
\pgfsetlinewidth{0.501875pt}%
\definecolor{currentstroke}{rgb}{0.392157,0.396078,0.403922}%
\pgfsetstrokecolor{currentstroke}%
\pgfsetdash{}{0pt}%
\pgfpathmoveto{\pgfqpoint{0.949979in}{0.127635in}}%
\pgfpathlineto{\pgfqpoint{0.949979in}{2.425059in}}%
\pgfusepath{stroke}%
\end{pgfscope}%
\begin{pgfscope}%
\pgfpathrectangle{\pgfqpoint{0.550713in}{0.127635in}}{\pgfqpoint{3.194133in}{2.297424in}}%
\pgfusepath{clip}%
\pgfsetbuttcap%
\pgfsetroundjoin%
\pgfsetlinewidth{0.501875pt}%
\definecolor{currentstroke}{rgb}{0.392157,0.396078,0.403922}%
\pgfsetstrokecolor{currentstroke}%
\pgfsetdash{}{0pt}%
\pgfpathmoveto{\pgfqpoint{1.349246in}{0.127635in}}%
\pgfpathlineto{\pgfqpoint{1.349246in}{2.425059in}}%
\pgfusepath{stroke}%
\end{pgfscope}%
\begin{pgfscope}%
\pgfpathrectangle{\pgfqpoint{0.550713in}{0.127635in}}{\pgfqpoint{3.194133in}{2.297424in}}%
\pgfusepath{clip}%
\pgfsetbuttcap%
\pgfsetroundjoin%
\pgfsetlinewidth{0.501875pt}%
\definecolor{currentstroke}{rgb}{0.392157,0.396078,0.403922}%
\pgfsetstrokecolor{currentstroke}%
\pgfsetdash{}{0pt}%
\pgfpathmoveto{\pgfqpoint{1.748513in}{0.127635in}}%
\pgfpathlineto{\pgfqpoint{1.748513in}{2.425059in}}%
\pgfusepath{stroke}%
\end{pgfscope}%
\begin{pgfscope}%
\pgfpathrectangle{\pgfqpoint{0.550713in}{0.127635in}}{\pgfqpoint{3.194133in}{2.297424in}}%
\pgfusepath{clip}%
\pgfsetbuttcap%
\pgfsetroundjoin%
\pgfsetlinewidth{0.501875pt}%
\definecolor{currentstroke}{rgb}{0.392157,0.396078,0.403922}%
\pgfsetstrokecolor{currentstroke}%
\pgfsetdash{}{0pt}%
\pgfpathmoveto{\pgfqpoint{2.147779in}{0.127635in}}%
\pgfpathlineto{\pgfqpoint{2.147779in}{2.425059in}}%
\pgfusepath{stroke}%
\end{pgfscope}%
\begin{pgfscope}%
\pgfpathrectangle{\pgfqpoint{0.550713in}{0.127635in}}{\pgfqpoint{3.194133in}{2.297424in}}%
\pgfusepath{clip}%
\pgfsetbuttcap%
\pgfsetroundjoin%
\pgfsetlinewidth{0.501875pt}%
\definecolor{currentstroke}{rgb}{0.392157,0.396078,0.403922}%
\pgfsetstrokecolor{currentstroke}%
\pgfsetdash{}{0pt}%
\pgfpathmoveto{\pgfqpoint{2.547046in}{0.127635in}}%
\pgfpathlineto{\pgfqpoint{2.547046in}{2.425059in}}%
\pgfusepath{stroke}%
\end{pgfscope}%
\begin{pgfscope}%
\pgfpathrectangle{\pgfqpoint{0.550713in}{0.127635in}}{\pgfqpoint{3.194133in}{2.297424in}}%
\pgfusepath{clip}%
\pgfsetbuttcap%
\pgfsetroundjoin%
\pgfsetlinewidth{0.501875pt}%
\definecolor{currentstroke}{rgb}{0.392157,0.396078,0.403922}%
\pgfsetstrokecolor{currentstroke}%
\pgfsetdash{}{0pt}%
\pgfpathmoveto{\pgfqpoint{2.946312in}{0.127635in}}%
\pgfpathlineto{\pgfqpoint{2.946312in}{2.425059in}}%
\pgfusepath{stroke}%
\end{pgfscope}%
\begin{pgfscope}%
\pgfpathrectangle{\pgfqpoint{0.550713in}{0.127635in}}{\pgfqpoint{3.194133in}{2.297424in}}%
\pgfusepath{clip}%
\pgfsetbuttcap%
\pgfsetroundjoin%
\pgfsetlinewidth{0.501875pt}%
\definecolor{currentstroke}{rgb}{0.392157,0.396078,0.403922}%
\pgfsetstrokecolor{currentstroke}%
\pgfsetdash{}{0pt}%
\pgfpathmoveto{\pgfqpoint{3.345579in}{0.127635in}}%
\pgfpathlineto{\pgfqpoint{3.345579in}{2.425059in}}%
\pgfusepath{stroke}%
\end{pgfscope}%
\begin{pgfscope}%
\pgfpathrectangle{\pgfqpoint{0.550713in}{0.127635in}}{\pgfqpoint{3.194133in}{2.297424in}}%
\pgfusepath{clip}%
\pgfsetbuttcap%
\pgfsetroundjoin%
\pgfsetlinewidth{0.853187pt}%
\definecolor{currentstroke}{rgb}{0.392157,0.396078,0.403922}%
\pgfsetstrokecolor{currentstroke}%
\pgfsetdash{{3.145000pt}{1.360000pt}}{0.000000pt}%
\pgfusepath{stroke}%
\end{pgfscope}%
\begin{pgfscope}%
\pgfpathrectangle{\pgfqpoint{0.550713in}{0.127635in}}{\pgfqpoint{3.194133in}{2.297424in}}%
\pgfusepath{clip}%
\pgfsetrectcap%
\pgfsetroundjoin%
\pgfsetlinewidth{0.752812pt}%
\definecolor{currentstroke}{rgb}{0.000000,0.000000,0.000000}%
\pgfsetstrokecolor{currentstroke}%
\pgfsetdash{}{0pt}%
\pgfpathmoveto{\pgfqpoint{0.670493in}{0.725996in}}%
\pgfpathlineto{\pgfqpoint{0.670493in}{0.629439in}}%
\pgfusepath{stroke}%
\end{pgfscope}%
\begin{pgfscope}%
\pgfpathrectangle{\pgfqpoint{0.550713in}{0.127635in}}{\pgfqpoint{3.194133in}{2.297424in}}%
\pgfusepath{clip}%
\pgfsetrectcap%
\pgfsetroundjoin%
\pgfsetlinewidth{0.752812pt}%
\definecolor{currentstroke}{rgb}{0.000000,0.000000,0.000000}%
\pgfsetstrokecolor{currentstroke}%
\pgfsetdash{}{0pt}%
\pgfpathmoveto{\pgfqpoint{0.670493in}{0.874892in}}%
\pgfpathlineto{\pgfqpoint{0.670493in}{1.026271in}}%
\pgfusepath{stroke}%
\end{pgfscope}%
\begin{pgfscope}%
\pgfpathrectangle{\pgfqpoint{0.550713in}{0.127635in}}{\pgfqpoint{3.194133in}{2.297424in}}%
\pgfusepath{clip}%
\pgfsetrectcap%
\pgfsetroundjoin%
\pgfsetlinewidth{0.752812pt}%
\definecolor{currentstroke}{rgb}{0.000000,0.000000,0.000000}%
\pgfsetstrokecolor{currentstroke}%
\pgfsetdash{}{0pt}%
\pgfpathmoveto{\pgfqpoint{0.631364in}{0.629439in}}%
\pgfpathlineto{\pgfqpoint{0.709621in}{0.629439in}}%
\pgfusepath{stroke}%
\end{pgfscope}%
\begin{pgfscope}%
\pgfpathrectangle{\pgfqpoint{0.550713in}{0.127635in}}{\pgfqpoint{3.194133in}{2.297424in}}%
\pgfusepath{clip}%
\pgfsetrectcap%
\pgfsetroundjoin%
\pgfsetlinewidth{0.752812pt}%
\definecolor{currentstroke}{rgb}{0.000000,0.000000,0.000000}%
\pgfsetstrokecolor{currentstroke}%
\pgfsetdash{}{0pt}%
\pgfpathmoveto{\pgfqpoint{0.631364in}{1.026271in}}%
\pgfpathlineto{\pgfqpoint{0.709621in}{1.026271in}}%
\pgfusepath{stroke}%
\end{pgfscope}%
\begin{pgfscope}%
\pgfpathrectangle{\pgfqpoint{0.550713in}{0.127635in}}{\pgfqpoint{3.194133in}{2.297424in}}%
\pgfusepath{clip}%
\pgfsetrectcap%
\pgfsetroundjoin%
\pgfsetlinewidth{0.752812pt}%
\definecolor{currentstroke}{rgb}{0.000000,0.000000,0.000000}%
\pgfsetstrokecolor{currentstroke}%
\pgfsetdash{}{0pt}%
\pgfpathmoveto{\pgfqpoint{0.830199in}{1.046819in}}%
\pgfpathlineto{\pgfqpoint{0.830199in}{0.946740in}}%
\pgfusepath{stroke}%
\end{pgfscope}%
\begin{pgfscope}%
\pgfpathrectangle{\pgfqpoint{0.550713in}{0.127635in}}{\pgfqpoint{3.194133in}{2.297424in}}%
\pgfusepath{clip}%
\pgfsetrectcap%
\pgfsetroundjoin%
\pgfsetlinewidth{0.752812pt}%
\definecolor{currentstroke}{rgb}{0.000000,0.000000,0.000000}%
\pgfsetstrokecolor{currentstroke}%
\pgfsetdash{}{0pt}%
\pgfpathmoveto{\pgfqpoint{0.830199in}{1.360423in}}%
\pgfpathlineto{\pgfqpoint{0.830199in}{1.503568in}}%
\pgfusepath{stroke}%
\end{pgfscope}%
\begin{pgfscope}%
\pgfpathrectangle{\pgfqpoint{0.550713in}{0.127635in}}{\pgfqpoint{3.194133in}{2.297424in}}%
\pgfusepath{clip}%
\pgfsetrectcap%
\pgfsetroundjoin%
\pgfsetlinewidth{0.752812pt}%
\definecolor{currentstroke}{rgb}{0.000000,0.000000,0.000000}%
\pgfsetstrokecolor{currentstroke}%
\pgfsetdash{}{0pt}%
\pgfpathmoveto{\pgfqpoint{0.791071in}{0.946740in}}%
\pgfpathlineto{\pgfqpoint{0.869327in}{0.946740in}}%
\pgfusepath{stroke}%
\end{pgfscope}%
\begin{pgfscope}%
\pgfpathrectangle{\pgfqpoint{0.550713in}{0.127635in}}{\pgfqpoint{3.194133in}{2.297424in}}%
\pgfusepath{clip}%
\pgfsetrectcap%
\pgfsetroundjoin%
\pgfsetlinewidth{0.752812pt}%
\definecolor{currentstroke}{rgb}{0.000000,0.000000,0.000000}%
\pgfsetstrokecolor{currentstroke}%
\pgfsetdash{}{0pt}%
\pgfpathmoveto{\pgfqpoint{0.791071in}{1.503568in}}%
\pgfpathlineto{\pgfqpoint{0.869327in}{1.503568in}}%
\pgfusepath{stroke}%
\end{pgfscope}%
\begin{pgfscope}%
\pgfpathrectangle{\pgfqpoint{0.550713in}{0.127635in}}{\pgfqpoint{3.194133in}{2.297424in}}%
\pgfusepath{clip}%
\pgfsetrectcap%
\pgfsetroundjoin%
\pgfsetlinewidth{0.752812pt}%
\definecolor{currentstroke}{rgb}{0.000000,0.000000,0.000000}%
\pgfsetstrokecolor{currentstroke}%
\pgfsetdash{}{0pt}%
\pgfpathmoveto{\pgfqpoint{1.069759in}{1.901825in}}%
\pgfpathlineto{\pgfqpoint{1.069759in}{1.778953in}}%
\pgfusepath{stroke}%
\end{pgfscope}%
\begin{pgfscope}%
\pgfpathrectangle{\pgfqpoint{0.550713in}{0.127635in}}{\pgfqpoint{3.194133in}{2.297424in}}%
\pgfusepath{clip}%
\pgfsetrectcap%
\pgfsetroundjoin%
\pgfsetlinewidth{0.752812pt}%
\definecolor{currentstroke}{rgb}{0.000000,0.000000,0.000000}%
\pgfsetstrokecolor{currentstroke}%
\pgfsetdash{}{0pt}%
\pgfpathmoveto{\pgfqpoint{1.069759in}{2.161223in}}%
\pgfpathlineto{\pgfqpoint{1.069759in}{2.255558in}}%
\pgfusepath{stroke}%
\end{pgfscope}%
\begin{pgfscope}%
\pgfpathrectangle{\pgfqpoint{0.550713in}{0.127635in}}{\pgfqpoint{3.194133in}{2.297424in}}%
\pgfusepath{clip}%
\pgfsetrectcap%
\pgfsetroundjoin%
\pgfsetlinewidth{0.752812pt}%
\definecolor{currentstroke}{rgb}{0.000000,0.000000,0.000000}%
\pgfsetstrokecolor{currentstroke}%
\pgfsetdash{}{0pt}%
\pgfpathmoveto{\pgfqpoint{1.030631in}{1.778953in}}%
\pgfpathlineto{\pgfqpoint{1.108887in}{1.778953in}}%
\pgfusepath{stroke}%
\end{pgfscope}%
\begin{pgfscope}%
\pgfpathrectangle{\pgfqpoint{0.550713in}{0.127635in}}{\pgfqpoint{3.194133in}{2.297424in}}%
\pgfusepath{clip}%
\pgfsetrectcap%
\pgfsetroundjoin%
\pgfsetlinewidth{0.752812pt}%
\definecolor{currentstroke}{rgb}{0.000000,0.000000,0.000000}%
\pgfsetstrokecolor{currentstroke}%
\pgfsetdash{}{0pt}%
\pgfpathmoveto{\pgfqpoint{1.030631in}{2.255558in}}%
\pgfpathlineto{\pgfqpoint{1.108887in}{2.255558in}}%
\pgfusepath{stroke}%
\end{pgfscope}%
\begin{pgfscope}%
\pgfpathrectangle{\pgfqpoint{0.550713in}{0.127635in}}{\pgfqpoint{3.194133in}{2.297424in}}%
\pgfusepath{clip}%
\pgfsetrectcap%
\pgfsetroundjoin%
\pgfsetlinewidth{0.752812pt}%
\definecolor{currentstroke}{rgb}{0.000000,0.000000,0.000000}%
\pgfsetstrokecolor{currentstroke}%
\pgfsetdash{}{0pt}%
\pgfpathmoveto{\pgfqpoint{1.229466in}{1.910175in}}%
\pgfpathlineto{\pgfqpoint{1.229466in}{1.886592in}}%
\pgfusepath{stroke}%
\end{pgfscope}%
\begin{pgfscope}%
\pgfpathrectangle{\pgfqpoint{0.550713in}{0.127635in}}{\pgfqpoint{3.194133in}{2.297424in}}%
\pgfusepath{clip}%
\pgfsetrectcap%
\pgfsetroundjoin%
\pgfsetlinewidth{0.752812pt}%
\definecolor{currentstroke}{rgb}{0.000000,0.000000,0.000000}%
\pgfsetstrokecolor{currentstroke}%
\pgfsetdash{}{0pt}%
\pgfpathmoveto{\pgfqpoint{1.229466in}{2.166913in}}%
\pgfpathlineto{\pgfqpoint{1.229466in}{2.187353in}}%
\pgfusepath{stroke}%
\end{pgfscope}%
\begin{pgfscope}%
\pgfpathrectangle{\pgfqpoint{0.550713in}{0.127635in}}{\pgfqpoint{3.194133in}{2.297424in}}%
\pgfusepath{clip}%
\pgfsetrectcap%
\pgfsetroundjoin%
\pgfsetlinewidth{0.752812pt}%
\definecolor{currentstroke}{rgb}{0.000000,0.000000,0.000000}%
\pgfsetstrokecolor{currentstroke}%
\pgfsetdash{}{0pt}%
\pgfpathmoveto{\pgfqpoint{1.190338in}{1.886592in}}%
\pgfpathlineto{\pgfqpoint{1.268594in}{1.886592in}}%
\pgfusepath{stroke}%
\end{pgfscope}%
\begin{pgfscope}%
\pgfpathrectangle{\pgfqpoint{0.550713in}{0.127635in}}{\pgfqpoint{3.194133in}{2.297424in}}%
\pgfusepath{clip}%
\pgfsetrectcap%
\pgfsetroundjoin%
\pgfsetlinewidth{0.752812pt}%
\definecolor{currentstroke}{rgb}{0.000000,0.000000,0.000000}%
\pgfsetstrokecolor{currentstroke}%
\pgfsetdash{}{0pt}%
\pgfpathmoveto{\pgfqpoint{1.190338in}{2.187353in}}%
\pgfpathlineto{\pgfqpoint{1.268594in}{2.187353in}}%
\pgfusepath{stroke}%
\end{pgfscope}%
\begin{pgfscope}%
\pgfpathrectangle{\pgfqpoint{0.550713in}{0.127635in}}{\pgfqpoint{3.194133in}{2.297424in}}%
\pgfusepath{clip}%
\pgfsetrectcap%
\pgfsetroundjoin%
\pgfsetlinewidth{0.752812pt}%
\definecolor{currentstroke}{rgb}{0.000000,0.000000,0.000000}%
\pgfsetstrokecolor{currentstroke}%
\pgfsetdash{}{0pt}%
\pgfpathmoveto{\pgfqpoint{1.469026in}{0.643228in}}%
\pgfpathlineto{\pgfqpoint{1.469026in}{0.611207in}}%
\pgfusepath{stroke}%
\end{pgfscope}%
\begin{pgfscope}%
\pgfpathrectangle{\pgfqpoint{0.550713in}{0.127635in}}{\pgfqpoint{3.194133in}{2.297424in}}%
\pgfusepath{clip}%
\pgfsetrectcap%
\pgfsetroundjoin%
\pgfsetlinewidth{0.752812pt}%
\definecolor{currentstroke}{rgb}{0.000000,0.000000,0.000000}%
\pgfsetstrokecolor{currentstroke}%
\pgfsetdash{}{0pt}%
\pgfpathmoveto{\pgfqpoint{1.469026in}{0.803699in}}%
\pgfpathlineto{\pgfqpoint{1.469026in}{0.912815in}}%
\pgfusepath{stroke}%
\end{pgfscope}%
\begin{pgfscope}%
\pgfpathrectangle{\pgfqpoint{0.550713in}{0.127635in}}{\pgfqpoint{3.194133in}{2.297424in}}%
\pgfusepath{clip}%
\pgfsetrectcap%
\pgfsetroundjoin%
\pgfsetlinewidth{0.752812pt}%
\definecolor{currentstroke}{rgb}{0.000000,0.000000,0.000000}%
\pgfsetstrokecolor{currentstroke}%
\pgfsetdash{}{0pt}%
\pgfpathmoveto{\pgfqpoint{1.429898in}{0.611207in}}%
\pgfpathlineto{\pgfqpoint{1.508154in}{0.611207in}}%
\pgfusepath{stroke}%
\end{pgfscope}%
\begin{pgfscope}%
\pgfpathrectangle{\pgfqpoint{0.550713in}{0.127635in}}{\pgfqpoint{3.194133in}{2.297424in}}%
\pgfusepath{clip}%
\pgfsetrectcap%
\pgfsetroundjoin%
\pgfsetlinewidth{0.752812pt}%
\definecolor{currentstroke}{rgb}{0.000000,0.000000,0.000000}%
\pgfsetstrokecolor{currentstroke}%
\pgfsetdash{}{0pt}%
\pgfpathmoveto{\pgfqpoint{1.429898in}{0.912815in}}%
\pgfpathlineto{\pgfqpoint{1.508154in}{0.912815in}}%
\pgfusepath{stroke}%
\end{pgfscope}%
\begin{pgfscope}%
\pgfpathrectangle{\pgfqpoint{0.550713in}{0.127635in}}{\pgfqpoint{3.194133in}{2.297424in}}%
\pgfusepath{clip}%
\pgfsetrectcap%
\pgfsetroundjoin%
\pgfsetlinewidth{0.752812pt}%
\definecolor{currentstroke}{rgb}{0.000000,0.000000,0.000000}%
\pgfsetstrokecolor{currentstroke}%
\pgfsetdash{}{0pt}%
\pgfpathmoveto{\pgfqpoint{1.628733in}{0.637753in}}%
\pgfpathlineto{\pgfqpoint{1.628733in}{0.581194in}}%
\pgfusepath{stroke}%
\end{pgfscope}%
\begin{pgfscope}%
\pgfpathrectangle{\pgfqpoint{0.550713in}{0.127635in}}{\pgfqpoint{3.194133in}{2.297424in}}%
\pgfusepath{clip}%
\pgfsetrectcap%
\pgfsetroundjoin%
\pgfsetlinewidth{0.752812pt}%
\definecolor{currentstroke}{rgb}{0.000000,0.000000,0.000000}%
\pgfsetstrokecolor{currentstroke}%
\pgfsetdash{}{0pt}%
\pgfpathmoveto{\pgfqpoint{1.628733in}{0.956461in}}%
\pgfpathlineto{\pgfqpoint{1.628733in}{1.107385in}}%
\pgfusepath{stroke}%
\end{pgfscope}%
\begin{pgfscope}%
\pgfpathrectangle{\pgfqpoint{0.550713in}{0.127635in}}{\pgfqpoint{3.194133in}{2.297424in}}%
\pgfusepath{clip}%
\pgfsetrectcap%
\pgfsetroundjoin%
\pgfsetlinewidth{0.752812pt}%
\definecolor{currentstroke}{rgb}{0.000000,0.000000,0.000000}%
\pgfsetstrokecolor{currentstroke}%
\pgfsetdash{}{0pt}%
\pgfpathmoveto{\pgfqpoint{1.589604in}{0.581194in}}%
\pgfpathlineto{\pgfqpoint{1.667861in}{0.581194in}}%
\pgfusepath{stroke}%
\end{pgfscope}%
\begin{pgfscope}%
\pgfpathrectangle{\pgfqpoint{0.550713in}{0.127635in}}{\pgfqpoint{3.194133in}{2.297424in}}%
\pgfusepath{clip}%
\pgfsetrectcap%
\pgfsetroundjoin%
\pgfsetlinewidth{0.752812pt}%
\definecolor{currentstroke}{rgb}{0.000000,0.000000,0.000000}%
\pgfsetstrokecolor{currentstroke}%
\pgfsetdash{}{0pt}%
\pgfpathmoveto{\pgfqpoint{1.589604in}{1.107385in}}%
\pgfpathlineto{\pgfqpoint{1.667861in}{1.107385in}}%
\pgfusepath{stroke}%
\end{pgfscope}%
\begin{pgfscope}%
\pgfpathrectangle{\pgfqpoint{0.550713in}{0.127635in}}{\pgfqpoint{3.194133in}{2.297424in}}%
\pgfusepath{clip}%
\pgfsetrectcap%
\pgfsetroundjoin%
\pgfsetlinewidth{0.752812pt}%
\definecolor{currentstroke}{rgb}{0.000000,0.000000,0.000000}%
\pgfsetstrokecolor{currentstroke}%
\pgfsetdash{}{0pt}%
\pgfpathmoveto{\pgfqpoint{1.868293in}{1.032313in}}%
\pgfpathlineto{\pgfqpoint{1.868293in}{1.032313in}}%
\pgfusepath{stroke}%
\end{pgfscope}%
\begin{pgfscope}%
\pgfpathrectangle{\pgfqpoint{0.550713in}{0.127635in}}{\pgfqpoint{3.194133in}{2.297424in}}%
\pgfusepath{clip}%
\pgfsetrectcap%
\pgfsetroundjoin%
\pgfsetlinewidth{0.752812pt}%
\definecolor{currentstroke}{rgb}{0.000000,0.000000,0.000000}%
\pgfsetstrokecolor{currentstroke}%
\pgfsetdash{}{0pt}%
\pgfpathmoveto{\pgfqpoint{1.868293in}{1.158237in}}%
\pgfpathlineto{\pgfqpoint{1.868293in}{1.346954in}}%
\pgfusepath{stroke}%
\end{pgfscope}%
\begin{pgfscope}%
\pgfpathrectangle{\pgfqpoint{0.550713in}{0.127635in}}{\pgfqpoint{3.194133in}{2.297424in}}%
\pgfusepath{clip}%
\pgfsetrectcap%
\pgfsetroundjoin%
\pgfsetlinewidth{0.752812pt}%
\definecolor{currentstroke}{rgb}{0.000000,0.000000,0.000000}%
\pgfsetstrokecolor{currentstroke}%
\pgfsetdash{}{0pt}%
\pgfpathmoveto{\pgfqpoint{1.829164in}{1.032313in}}%
\pgfpathlineto{\pgfqpoint{1.907421in}{1.032313in}}%
\pgfusepath{stroke}%
\end{pgfscope}%
\begin{pgfscope}%
\pgfpathrectangle{\pgfqpoint{0.550713in}{0.127635in}}{\pgfqpoint{3.194133in}{2.297424in}}%
\pgfusepath{clip}%
\pgfsetrectcap%
\pgfsetroundjoin%
\pgfsetlinewidth{0.752812pt}%
\definecolor{currentstroke}{rgb}{0.000000,0.000000,0.000000}%
\pgfsetstrokecolor{currentstroke}%
\pgfsetdash{}{0pt}%
\pgfpathmoveto{\pgfqpoint{1.829164in}{1.346954in}}%
\pgfpathlineto{\pgfqpoint{1.907421in}{1.346954in}}%
\pgfusepath{stroke}%
\end{pgfscope}%
\begin{pgfscope}%
\pgfpathrectangle{\pgfqpoint{0.550713in}{0.127635in}}{\pgfqpoint{3.194133in}{2.297424in}}%
\pgfusepath{clip}%
\pgfsetbuttcap%
\pgfsetmiterjoin%
\definecolor{currentfill}{rgb}{0.000000,0.000000,0.000000}%
\pgfsetfillcolor{currentfill}%
\pgfsetlinewidth{1.003750pt}%
\definecolor{currentstroke}{rgb}{0.000000,0.000000,0.000000}%
\pgfsetstrokecolor{currentstroke}%
\pgfsetdash{}{0pt}%
\pgfsys@defobject{currentmarker}{\pgfqpoint{-0.011785in}{-0.019642in}}{\pgfqpoint{0.011785in}{0.019642in}}{%
\pgfpathmoveto{\pgfqpoint{-0.000000in}{-0.019642in}}%
\pgfpathlineto{\pgfqpoint{0.011785in}{0.000000in}}%
\pgfpathlineto{\pgfqpoint{0.000000in}{0.019642in}}%
\pgfpathlineto{\pgfqpoint{-0.011785in}{0.000000in}}%
\pgfpathclose%
\pgfusepath{stroke,fill}%
}%
\begin{pgfscope}%
\pgfsys@transformshift{1.868293in}{0.772508in}%
\pgfsys@useobject{currentmarker}{}%
\end{pgfscope}%
\end{pgfscope}%
\begin{pgfscope}%
\pgfpathrectangle{\pgfqpoint{0.550713in}{0.127635in}}{\pgfqpoint{3.194133in}{2.297424in}}%
\pgfusepath{clip}%
\pgfsetrectcap%
\pgfsetroundjoin%
\pgfsetlinewidth{0.752812pt}%
\definecolor{currentstroke}{rgb}{0.000000,0.000000,0.000000}%
\pgfsetstrokecolor{currentstroke}%
\pgfsetdash{}{0pt}%
\pgfpathmoveto{\pgfqpoint{2.027999in}{0.826105in}}%
\pgfpathlineto{\pgfqpoint{2.027999in}{0.772570in}}%
\pgfusepath{stroke}%
\end{pgfscope}%
\begin{pgfscope}%
\pgfpathrectangle{\pgfqpoint{0.550713in}{0.127635in}}{\pgfqpoint{3.194133in}{2.297424in}}%
\pgfusepath{clip}%
\pgfsetrectcap%
\pgfsetroundjoin%
\pgfsetlinewidth{0.752812pt}%
\definecolor{currentstroke}{rgb}{0.000000,0.000000,0.000000}%
\pgfsetstrokecolor{currentstroke}%
\pgfsetdash{}{0pt}%
\pgfpathmoveto{\pgfqpoint{2.027999in}{1.154033in}}%
\pgfpathlineto{\pgfqpoint{2.027999in}{1.227187in}}%
\pgfusepath{stroke}%
\end{pgfscope}%
\begin{pgfscope}%
\pgfpathrectangle{\pgfqpoint{0.550713in}{0.127635in}}{\pgfqpoint{3.194133in}{2.297424in}}%
\pgfusepath{clip}%
\pgfsetrectcap%
\pgfsetroundjoin%
\pgfsetlinewidth{0.752812pt}%
\definecolor{currentstroke}{rgb}{0.000000,0.000000,0.000000}%
\pgfsetstrokecolor{currentstroke}%
\pgfsetdash{}{0pt}%
\pgfpathmoveto{\pgfqpoint{1.988871in}{0.772570in}}%
\pgfpathlineto{\pgfqpoint{2.067127in}{0.772570in}}%
\pgfusepath{stroke}%
\end{pgfscope}%
\begin{pgfscope}%
\pgfpathrectangle{\pgfqpoint{0.550713in}{0.127635in}}{\pgfqpoint{3.194133in}{2.297424in}}%
\pgfusepath{clip}%
\pgfsetrectcap%
\pgfsetroundjoin%
\pgfsetlinewidth{0.752812pt}%
\definecolor{currentstroke}{rgb}{0.000000,0.000000,0.000000}%
\pgfsetstrokecolor{currentstroke}%
\pgfsetdash{}{0pt}%
\pgfpathmoveto{\pgfqpoint{1.988871in}{1.227187in}}%
\pgfpathlineto{\pgfqpoint{2.067127in}{1.227187in}}%
\pgfusepath{stroke}%
\end{pgfscope}%
\begin{pgfscope}%
\pgfpathrectangle{\pgfqpoint{0.550713in}{0.127635in}}{\pgfqpoint{3.194133in}{2.297424in}}%
\pgfusepath{clip}%
\pgfsetrectcap%
\pgfsetroundjoin%
\pgfsetlinewidth{0.752812pt}%
\definecolor{currentstroke}{rgb}{0.000000,0.000000,0.000000}%
\pgfsetstrokecolor{currentstroke}%
\pgfsetdash{}{0pt}%
\pgfpathmoveto{\pgfqpoint{2.267559in}{0.485039in}}%
\pgfpathlineto{\pgfqpoint{2.267559in}{0.463779in}}%
\pgfusepath{stroke}%
\end{pgfscope}%
\begin{pgfscope}%
\pgfpathrectangle{\pgfqpoint{0.550713in}{0.127635in}}{\pgfqpoint{3.194133in}{2.297424in}}%
\pgfusepath{clip}%
\pgfsetrectcap%
\pgfsetroundjoin%
\pgfsetlinewidth{0.752812pt}%
\definecolor{currentstroke}{rgb}{0.000000,0.000000,0.000000}%
\pgfsetstrokecolor{currentstroke}%
\pgfsetdash{}{0pt}%
\pgfpathmoveto{\pgfqpoint{2.267559in}{0.541355in}}%
\pgfpathlineto{\pgfqpoint{2.267559in}{0.541355in}}%
\pgfusepath{stroke}%
\end{pgfscope}%
\begin{pgfscope}%
\pgfpathrectangle{\pgfqpoint{0.550713in}{0.127635in}}{\pgfqpoint{3.194133in}{2.297424in}}%
\pgfusepath{clip}%
\pgfsetrectcap%
\pgfsetroundjoin%
\pgfsetlinewidth{0.752812pt}%
\definecolor{currentstroke}{rgb}{0.000000,0.000000,0.000000}%
\pgfsetstrokecolor{currentstroke}%
\pgfsetdash{}{0pt}%
\pgfpathmoveto{\pgfqpoint{2.228431in}{0.463779in}}%
\pgfpathlineto{\pgfqpoint{2.306687in}{0.463779in}}%
\pgfusepath{stroke}%
\end{pgfscope}%
\begin{pgfscope}%
\pgfpathrectangle{\pgfqpoint{0.550713in}{0.127635in}}{\pgfqpoint{3.194133in}{2.297424in}}%
\pgfusepath{clip}%
\pgfsetrectcap%
\pgfsetroundjoin%
\pgfsetlinewidth{0.752812pt}%
\definecolor{currentstroke}{rgb}{0.000000,0.000000,0.000000}%
\pgfsetstrokecolor{currentstroke}%
\pgfsetdash{}{0pt}%
\pgfpathmoveto{\pgfqpoint{2.228431in}{0.541355in}}%
\pgfpathlineto{\pgfqpoint{2.306687in}{0.541355in}}%
\pgfusepath{stroke}%
\end{pgfscope}%
\begin{pgfscope}%
\pgfpathrectangle{\pgfqpoint{0.550713in}{0.127635in}}{\pgfqpoint{3.194133in}{2.297424in}}%
\pgfusepath{clip}%
\pgfsetbuttcap%
\pgfsetmiterjoin%
\definecolor{currentfill}{rgb}{0.000000,0.000000,0.000000}%
\pgfsetfillcolor{currentfill}%
\pgfsetlinewidth{1.003750pt}%
\definecolor{currentstroke}{rgb}{0.000000,0.000000,0.000000}%
\pgfsetstrokecolor{currentstroke}%
\pgfsetdash{}{0pt}%
\pgfsys@defobject{currentmarker}{\pgfqpoint{-0.011785in}{-0.019642in}}{\pgfqpoint{0.011785in}{0.019642in}}{%
\pgfpathmoveto{\pgfqpoint{-0.000000in}{-0.019642in}}%
\pgfpathlineto{\pgfqpoint{0.011785in}{0.000000in}}%
\pgfpathlineto{\pgfqpoint{0.000000in}{0.019642in}}%
\pgfpathlineto{\pgfqpoint{-0.011785in}{0.000000in}}%
\pgfpathclose%
\pgfusepath{stroke,fill}%
}%
\begin{pgfscope}%
\pgfsys@transformshift{2.267559in}{0.692873in}%
\pgfsys@useobject{currentmarker}{}%
\end{pgfscope}%
\end{pgfscope}%
\begin{pgfscope}%
\pgfpathrectangle{\pgfqpoint{0.550713in}{0.127635in}}{\pgfqpoint{3.194133in}{2.297424in}}%
\pgfusepath{clip}%
\pgfsetrectcap%
\pgfsetroundjoin%
\pgfsetlinewidth{0.752812pt}%
\definecolor{currentstroke}{rgb}{0.000000,0.000000,0.000000}%
\pgfsetstrokecolor{currentstroke}%
\pgfsetdash{}{0pt}%
\pgfpathmoveto{\pgfqpoint{2.427266in}{0.522259in}}%
\pgfpathlineto{\pgfqpoint{2.427266in}{0.468674in}}%
\pgfusepath{stroke}%
\end{pgfscope}%
\begin{pgfscope}%
\pgfpathrectangle{\pgfqpoint{0.550713in}{0.127635in}}{\pgfqpoint{3.194133in}{2.297424in}}%
\pgfusepath{clip}%
\pgfsetrectcap%
\pgfsetroundjoin%
\pgfsetlinewidth{0.752812pt}%
\definecolor{currentstroke}{rgb}{0.000000,0.000000,0.000000}%
\pgfsetstrokecolor{currentstroke}%
\pgfsetdash{}{0pt}%
\pgfpathmoveto{\pgfqpoint{2.427266in}{0.604354in}}%
\pgfpathlineto{\pgfqpoint{2.427266in}{0.605821in}}%
\pgfusepath{stroke}%
\end{pgfscope}%
\begin{pgfscope}%
\pgfpathrectangle{\pgfqpoint{0.550713in}{0.127635in}}{\pgfqpoint{3.194133in}{2.297424in}}%
\pgfusepath{clip}%
\pgfsetrectcap%
\pgfsetroundjoin%
\pgfsetlinewidth{0.752812pt}%
\definecolor{currentstroke}{rgb}{0.000000,0.000000,0.000000}%
\pgfsetstrokecolor{currentstroke}%
\pgfsetdash{}{0pt}%
\pgfpathmoveto{\pgfqpoint{2.388138in}{0.468674in}}%
\pgfpathlineto{\pgfqpoint{2.466394in}{0.468674in}}%
\pgfusepath{stroke}%
\end{pgfscope}%
\begin{pgfscope}%
\pgfpathrectangle{\pgfqpoint{0.550713in}{0.127635in}}{\pgfqpoint{3.194133in}{2.297424in}}%
\pgfusepath{clip}%
\pgfsetrectcap%
\pgfsetroundjoin%
\pgfsetlinewidth{0.752812pt}%
\definecolor{currentstroke}{rgb}{0.000000,0.000000,0.000000}%
\pgfsetstrokecolor{currentstroke}%
\pgfsetdash{}{0pt}%
\pgfpathmoveto{\pgfqpoint{2.388138in}{0.605821in}}%
\pgfpathlineto{\pgfqpoint{2.466394in}{0.605821in}}%
\pgfusepath{stroke}%
\end{pgfscope}%
\begin{pgfscope}%
\pgfpathrectangle{\pgfqpoint{0.550713in}{0.127635in}}{\pgfqpoint{3.194133in}{2.297424in}}%
\pgfusepath{clip}%
\pgfsetrectcap%
\pgfsetroundjoin%
\pgfsetlinewidth{0.752812pt}%
\definecolor{currentstroke}{rgb}{0.000000,0.000000,0.000000}%
\pgfsetstrokecolor{currentstroke}%
\pgfsetdash{}{0pt}%
\pgfpathmoveto{\pgfqpoint{2.666826in}{0.636402in}}%
\pgfpathlineto{\pgfqpoint{2.666826in}{0.636402in}}%
\pgfusepath{stroke}%
\end{pgfscope}%
\begin{pgfscope}%
\pgfpathrectangle{\pgfqpoint{0.550713in}{0.127635in}}{\pgfqpoint{3.194133in}{2.297424in}}%
\pgfusepath{clip}%
\pgfsetrectcap%
\pgfsetroundjoin%
\pgfsetlinewidth{0.752812pt}%
\definecolor{currentstroke}{rgb}{0.000000,0.000000,0.000000}%
\pgfsetstrokecolor{currentstroke}%
\pgfsetdash{}{0pt}%
\pgfpathmoveto{\pgfqpoint{2.666826in}{0.674544in}}%
\pgfpathlineto{\pgfqpoint{2.666826in}{0.695529in}}%
\pgfusepath{stroke}%
\end{pgfscope}%
\begin{pgfscope}%
\pgfpathrectangle{\pgfqpoint{0.550713in}{0.127635in}}{\pgfqpoint{3.194133in}{2.297424in}}%
\pgfusepath{clip}%
\pgfsetrectcap%
\pgfsetroundjoin%
\pgfsetlinewidth{0.752812pt}%
\definecolor{currentstroke}{rgb}{0.000000,0.000000,0.000000}%
\pgfsetstrokecolor{currentstroke}%
\pgfsetdash{}{0pt}%
\pgfpathmoveto{\pgfqpoint{2.627698in}{0.636402in}}%
\pgfpathlineto{\pgfqpoint{2.705954in}{0.636402in}}%
\pgfusepath{stroke}%
\end{pgfscope}%
\begin{pgfscope}%
\pgfpathrectangle{\pgfqpoint{0.550713in}{0.127635in}}{\pgfqpoint{3.194133in}{2.297424in}}%
\pgfusepath{clip}%
\pgfsetrectcap%
\pgfsetroundjoin%
\pgfsetlinewidth{0.752812pt}%
\definecolor{currentstroke}{rgb}{0.000000,0.000000,0.000000}%
\pgfsetstrokecolor{currentstroke}%
\pgfsetdash{}{0pt}%
\pgfpathmoveto{\pgfqpoint{2.627698in}{0.695529in}}%
\pgfpathlineto{\pgfqpoint{2.705954in}{0.695529in}}%
\pgfusepath{stroke}%
\end{pgfscope}%
\begin{pgfscope}%
\pgfpathrectangle{\pgfqpoint{0.550713in}{0.127635in}}{\pgfqpoint{3.194133in}{2.297424in}}%
\pgfusepath{clip}%
\pgfsetbuttcap%
\pgfsetmiterjoin%
\definecolor{currentfill}{rgb}{0.000000,0.000000,0.000000}%
\pgfsetfillcolor{currentfill}%
\pgfsetlinewidth{1.003750pt}%
\definecolor{currentstroke}{rgb}{0.000000,0.000000,0.000000}%
\pgfsetstrokecolor{currentstroke}%
\pgfsetdash{}{0pt}%
\pgfsys@defobject{currentmarker}{\pgfqpoint{-0.011785in}{-0.019642in}}{\pgfqpoint{0.011785in}{0.019642in}}{%
\pgfpathmoveto{\pgfqpoint{-0.000000in}{-0.019642in}}%
\pgfpathlineto{\pgfqpoint{0.011785in}{0.000000in}}%
\pgfpathlineto{\pgfqpoint{0.000000in}{0.019642in}}%
\pgfpathlineto{\pgfqpoint{-0.011785in}{0.000000in}}%
\pgfpathclose%
\pgfusepath{stroke,fill}%
}%
\begin{pgfscope}%
\pgfsys@transformshift{2.666826in}{0.513288in}%
\pgfsys@useobject{currentmarker}{}%
\end{pgfscope}%
\end{pgfscope}%
\begin{pgfscope}%
\pgfpathrectangle{\pgfqpoint{0.550713in}{0.127635in}}{\pgfqpoint{3.194133in}{2.297424in}}%
\pgfusepath{clip}%
\pgfsetrectcap%
\pgfsetroundjoin%
\pgfsetlinewidth{0.752812pt}%
\definecolor{currentstroke}{rgb}{0.000000,0.000000,0.000000}%
\pgfsetstrokecolor{currentstroke}%
\pgfsetdash{}{0pt}%
\pgfpathmoveto{\pgfqpoint{2.826532in}{0.610462in}}%
\pgfpathlineto{\pgfqpoint{2.826532in}{0.601850in}}%
\pgfusepath{stroke}%
\end{pgfscope}%
\begin{pgfscope}%
\pgfpathrectangle{\pgfqpoint{0.550713in}{0.127635in}}{\pgfqpoint{3.194133in}{2.297424in}}%
\pgfusepath{clip}%
\pgfsetrectcap%
\pgfsetroundjoin%
\pgfsetlinewidth{0.752812pt}%
\definecolor{currentstroke}{rgb}{0.000000,0.000000,0.000000}%
\pgfsetstrokecolor{currentstroke}%
\pgfsetdash{}{0pt}%
\pgfpathmoveto{\pgfqpoint{2.826532in}{0.772525in}}%
\pgfpathlineto{\pgfqpoint{2.826532in}{0.774809in}}%
\pgfusepath{stroke}%
\end{pgfscope}%
\begin{pgfscope}%
\pgfpathrectangle{\pgfqpoint{0.550713in}{0.127635in}}{\pgfqpoint{3.194133in}{2.297424in}}%
\pgfusepath{clip}%
\pgfsetrectcap%
\pgfsetroundjoin%
\pgfsetlinewidth{0.752812pt}%
\definecolor{currentstroke}{rgb}{0.000000,0.000000,0.000000}%
\pgfsetstrokecolor{currentstroke}%
\pgfsetdash{}{0pt}%
\pgfpathmoveto{\pgfqpoint{2.787404in}{0.601850in}}%
\pgfpathlineto{\pgfqpoint{2.865661in}{0.601850in}}%
\pgfusepath{stroke}%
\end{pgfscope}%
\begin{pgfscope}%
\pgfpathrectangle{\pgfqpoint{0.550713in}{0.127635in}}{\pgfqpoint{3.194133in}{2.297424in}}%
\pgfusepath{clip}%
\pgfsetrectcap%
\pgfsetroundjoin%
\pgfsetlinewidth{0.752812pt}%
\definecolor{currentstroke}{rgb}{0.000000,0.000000,0.000000}%
\pgfsetstrokecolor{currentstroke}%
\pgfsetdash{}{0pt}%
\pgfpathmoveto{\pgfqpoint{2.787404in}{0.774809in}}%
\pgfpathlineto{\pgfqpoint{2.865661in}{0.774809in}}%
\pgfusepath{stroke}%
\end{pgfscope}%
\begin{pgfscope}%
\pgfpathrectangle{\pgfqpoint{0.550713in}{0.127635in}}{\pgfqpoint{3.194133in}{2.297424in}}%
\pgfusepath{clip}%
\pgfsetrectcap%
\pgfsetroundjoin%
\pgfsetlinewidth{0.752812pt}%
\definecolor{currentstroke}{rgb}{0.000000,0.000000,0.000000}%
\pgfsetstrokecolor{currentstroke}%
\pgfsetdash{}{0pt}%
\pgfpathmoveto{\pgfqpoint{3.066092in}{0.419047in}}%
\pgfpathlineto{\pgfqpoint{3.066092in}{0.406247in}}%
\pgfusepath{stroke}%
\end{pgfscope}%
\begin{pgfscope}%
\pgfpathrectangle{\pgfqpoint{0.550713in}{0.127635in}}{\pgfqpoint{3.194133in}{2.297424in}}%
\pgfusepath{clip}%
\pgfsetrectcap%
\pgfsetroundjoin%
\pgfsetlinewidth{0.752812pt}%
\definecolor{currentstroke}{rgb}{0.000000,0.000000,0.000000}%
\pgfsetstrokecolor{currentstroke}%
\pgfsetdash{}{0pt}%
\pgfpathmoveto{\pgfqpoint{3.066092in}{0.556683in}}%
\pgfpathlineto{\pgfqpoint{3.066092in}{0.604659in}}%
\pgfusepath{stroke}%
\end{pgfscope}%
\begin{pgfscope}%
\pgfpathrectangle{\pgfqpoint{0.550713in}{0.127635in}}{\pgfqpoint{3.194133in}{2.297424in}}%
\pgfusepath{clip}%
\pgfsetrectcap%
\pgfsetroundjoin%
\pgfsetlinewidth{0.752812pt}%
\definecolor{currentstroke}{rgb}{0.000000,0.000000,0.000000}%
\pgfsetstrokecolor{currentstroke}%
\pgfsetdash{}{0pt}%
\pgfpathmoveto{\pgfqpoint{3.026964in}{0.406247in}}%
\pgfpathlineto{\pgfqpoint{3.105221in}{0.406247in}}%
\pgfusepath{stroke}%
\end{pgfscope}%
\begin{pgfscope}%
\pgfpathrectangle{\pgfqpoint{0.550713in}{0.127635in}}{\pgfqpoint{3.194133in}{2.297424in}}%
\pgfusepath{clip}%
\pgfsetrectcap%
\pgfsetroundjoin%
\pgfsetlinewidth{0.752812pt}%
\definecolor{currentstroke}{rgb}{0.000000,0.000000,0.000000}%
\pgfsetstrokecolor{currentstroke}%
\pgfsetdash{}{0pt}%
\pgfpathmoveto{\pgfqpoint{3.026964in}{0.604659in}}%
\pgfpathlineto{\pgfqpoint{3.105221in}{0.604659in}}%
\pgfusepath{stroke}%
\end{pgfscope}%
\begin{pgfscope}%
\pgfpathrectangle{\pgfqpoint{0.550713in}{0.127635in}}{\pgfqpoint{3.194133in}{2.297424in}}%
\pgfusepath{clip}%
\pgfsetrectcap%
\pgfsetroundjoin%
\pgfsetlinewidth{0.752812pt}%
\definecolor{currentstroke}{rgb}{0.000000,0.000000,0.000000}%
\pgfsetstrokecolor{currentstroke}%
\pgfsetdash{}{0pt}%
\pgfpathmoveto{\pgfqpoint{3.225799in}{0.428743in}}%
\pgfpathlineto{\pgfqpoint{3.225799in}{0.395969in}}%
\pgfusepath{stroke}%
\end{pgfscope}%
\begin{pgfscope}%
\pgfpathrectangle{\pgfqpoint{0.550713in}{0.127635in}}{\pgfqpoint{3.194133in}{2.297424in}}%
\pgfusepath{clip}%
\pgfsetrectcap%
\pgfsetroundjoin%
\pgfsetlinewidth{0.752812pt}%
\definecolor{currentstroke}{rgb}{0.000000,0.000000,0.000000}%
\pgfsetstrokecolor{currentstroke}%
\pgfsetdash{}{0pt}%
\pgfpathmoveto{\pgfqpoint{3.225799in}{0.495873in}}%
\pgfpathlineto{\pgfqpoint{3.225799in}{0.495873in}}%
\pgfusepath{stroke}%
\end{pgfscope}%
\begin{pgfscope}%
\pgfpathrectangle{\pgfqpoint{0.550713in}{0.127635in}}{\pgfqpoint{3.194133in}{2.297424in}}%
\pgfusepath{clip}%
\pgfsetrectcap%
\pgfsetroundjoin%
\pgfsetlinewidth{0.752812pt}%
\definecolor{currentstroke}{rgb}{0.000000,0.000000,0.000000}%
\pgfsetstrokecolor{currentstroke}%
\pgfsetdash{}{0pt}%
\pgfpathmoveto{\pgfqpoint{3.186671in}{0.395969in}}%
\pgfpathlineto{\pgfqpoint{3.264927in}{0.395969in}}%
\pgfusepath{stroke}%
\end{pgfscope}%
\begin{pgfscope}%
\pgfpathrectangle{\pgfqpoint{0.550713in}{0.127635in}}{\pgfqpoint{3.194133in}{2.297424in}}%
\pgfusepath{clip}%
\pgfsetrectcap%
\pgfsetroundjoin%
\pgfsetlinewidth{0.752812pt}%
\definecolor{currentstroke}{rgb}{0.000000,0.000000,0.000000}%
\pgfsetstrokecolor{currentstroke}%
\pgfsetdash{}{0pt}%
\pgfpathmoveto{\pgfqpoint{3.186671in}{0.495873in}}%
\pgfpathlineto{\pgfqpoint{3.264927in}{0.495873in}}%
\pgfusepath{stroke}%
\end{pgfscope}%
\begin{pgfscope}%
\pgfpathrectangle{\pgfqpoint{0.550713in}{0.127635in}}{\pgfqpoint{3.194133in}{2.297424in}}%
\pgfusepath{clip}%
\pgfsetbuttcap%
\pgfsetmiterjoin%
\definecolor{currentfill}{rgb}{0.000000,0.000000,0.000000}%
\pgfsetfillcolor{currentfill}%
\pgfsetlinewidth{1.003750pt}%
\definecolor{currentstroke}{rgb}{0.000000,0.000000,0.000000}%
\pgfsetstrokecolor{currentstroke}%
\pgfsetdash{}{0pt}%
\pgfsys@defobject{currentmarker}{\pgfqpoint{-0.011785in}{-0.019642in}}{\pgfqpoint{0.011785in}{0.019642in}}{%
\pgfpathmoveto{\pgfqpoint{-0.000000in}{-0.019642in}}%
\pgfpathlineto{\pgfqpoint{0.011785in}{0.000000in}}%
\pgfpathlineto{\pgfqpoint{0.000000in}{0.019642in}}%
\pgfpathlineto{\pgfqpoint{-0.011785in}{0.000000in}}%
\pgfpathclose%
\pgfusepath{stroke,fill}%
}%
\begin{pgfscope}%
\pgfsys@transformshift{3.225799in}{0.722922in}%
\pgfsys@useobject{currentmarker}{}%
\end{pgfscope}%
\end{pgfscope}%
\begin{pgfscope}%
\pgfpathrectangle{\pgfqpoint{0.550713in}{0.127635in}}{\pgfqpoint{3.194133in}{2.297424in}}%
\pgfusepath{clip}%
\pgfsetrectcap%
\pgfsetroundjoin%
\pgfsetlinewidth{0.752812pt}%
\definecolor{currentstroke}{rgb}{0.000000,0.000000,0.000000}%
\pgfsetstrokecolor{currentstroke}%
\pgfsetdash{}{0pt}%
\pgfpathmoveto{\pgfqpoint{3.465359in}{0.631672in}}%
\pgfpathlineto{\pgfqpoint{3.465359in}{0.631672in}}%
\pgfusepath{stroke}%
\end{pgfscope}%
\begin{pgfscope}%
\pgfpathrectangle{\pgfqpoint{0.550713in}{0.127635in}}{\pgfqpoint{3.194133in}{2.297424in}}%
\pgfusepath{clip}%
\pgfsetrectcap%
\pgfsetroundjoin%
\pgfsetlinewidth{0.752812pt}%
\definecolor{currentstroke}{rgb}{0.000000,0.000000,0.000000}%
\pgfsetstrokecolor{currentstroke}%
\pgfsetdash{}{0pt}%
\pgfpathmoveto{\pgfqpoint{3.465359in}{0.673311in}}%
\pgfpathlineto{\pgfqpoint{3.465359in}{0.673311in}}%
\pgfusepath{stroke}%
\end{pgfscope}%
\begin{pgfscope}%
\pgfpathrectangle{\pgfqpoint{0.550713in}{0.127635in}}{\pgfqpoint{3.194133in}{2.297424in}}%
\pgfusepath{clip}%
\pgfsetrectcap%
\pgfsetroundjoin%
\pgfsetlinewidth{0.752812pt}%
\definecolor{currentstroke}{rgb}{0.000000,0.000000,0.000000}%
\pgfsetstrokecolor{currentstroke}%
\pgfsetdash{}{0pt}%
\pgfpathmoveto{\pgfqpoint{3.426231in}{0.631672in}}%
\pgfpathlineto{\pgfqpoint{3.504487in}{0.631672in}}%
\pgfusepath{stroke}%
\end{pgfscope}%
\begin{pgfscope}%
\pgfpathrectangle{\pgfqpoint{0.550713in}{0.127635in}}{\pgfqpoint{3.194133in}{2.297424in}}%
\pgfusepath{clip}%
\pgfsetrectcap%
\pgfsetroundjoin%
\pgfsetlinewidth{0.752812pt}%
\definecolor{currentstroke}{rgb}{0.000000,0.000000,0.000000}%
\pgfsetstrokecolor{currentstroke}%
\pgfsetdash{}{0pt}%
\pgfpathmoveto{\pgfqpoint{3.426231in}{0.673311in}}%
\pgfpathlineto{\pgfqpoint{3.504487in}{0.673311in}}%
\pgfusepath{stroke}%
\end{pgfscope}%
\begin{pgfscope}%
\pgfpathrectangle{\pgfqpoint{0.550713in}{0.127635in}}{\pgfqpoint{3.194133in}{2.297424in}}%
\pgfusepath{clip}%
\pgfsetbuttcap%
\pgfsetmiterjoin%
\definecolor{currentfill}{rgb}{0.000000,0.000000,0.000000}%
\pgfsetfillcolor{currentfill}%
\pgfsetlinewidth{1.003750pt}%
\definecolor{currentstroke}{rgb}{0.000000,0.000000,0.000000}%
\pgfsetstrokecolor{currentstroke}%
\pgfsetdash{}{0pt}%
\pgfsys@defobject{currentmarker}{\pgfqpoint{-0.011785in}{-0.019642in}}{\pgfqpoint{0.011785in}{0.019642in}}{%
\pgfpathmoveto{\pgfqpoint{-0.000000in}{-0.019642in}}%
\pgfpathlineto{\pgfqpoint{0.011785in}{0.000000in}}%
\pgfpathlineto{\pgfqpoint{0.000000in}{0.019642in}}%
\pgfpathlineto{\pgfqpoint{-0.011785in}{0.000000in}}%
\pgfpathclose%
\pgfusepath{stroke,fill}%
}%
\begin{pgfscope}%
\pgfsys@transformshift{3.465359in}{0.450242in}%
\pgfsys@useobject{currentmarker}{}%
\end{pgfscope}%
\begin{pgfscope}%
\pgfsys@transformshift{3.465359in}{0.744446in}%
\pgfsys@useobject{currentmarker}{}%
\end{pgfscope}%
\end{pgfscope}%
\begin{pgfscope}%
\pgfpathrectangle{\pgfqpoint{0.550713in}{0.127635in}}{\pgfqpoint{3.194133in}{2.297424in}}%
\pgfusepath{clip}%
\pgfsetrectcap%
\pgfsetroundjoin%
\pgfsetlinewidth{0.752812pt}%
\definecolor{currentstroke}{rgb}{0.000000,0.000000,0.000000}%
\pgfsetstrokecolor{currentstroke}%
\pgfsetdash{}{0pt}%
\pgfpathmoveto{\pgfqpoint{3.625066in}{0.590000in}}%
\pgfpathlineto{\pgfqpoint{3.625066in}{0.565235in}}%
\pgfusepath{stroke}%
\end{pgfscope}%
\begin{pgfscope}%
\pgfpathrectangle{\pgfqpoint{0.550713in}{0.127635in}}{\pgfqpoint{3.194133in}{2.297424in}}%
\pgfusepath{clip}%
\pgfsetrectcap%
\pgfsetroundjoin%
\pgfsetlinewidth{0.752812pt}%
\definecolor{currentstroke}{rgb}{0.000000,0.000000,0.000000}%
\pgfsetstrokecolor{currentstroke}%
\pgfsetdash{}{0pt}%
\pgfpathmoveto{\pgfqpoint{3.625066in}{0.714501in}}%
\pgfpathlineto{\pgfqpoint{3.625066in}{0.775535in}}%
\pgfusepath{stroke}%
\end{pgfscope}%
\begin{pgfscope}%
\pgfpathrectangle{\pgfqpoint{0.550713in}{0.127635in}}{\pgfqpoint{3.194133in}{2.297424in}}%
\pgfusepath{clip}%
\pgfsetrectcap%
\pgfsetroundjoin%
\pgfsetlinewidth{0.752812pt}%
\definecolor{currentstroke}{rgb}{0.000000,0.000000,0.000000}%
\pgfsetstrokecolor{currentstroke}%
\pgfsetdash{}{0pt}%
\pgfpathmoveto{\pgfqpoint{3.585938in}{0.565235in}}%
\pgfpathlineto{\pgfqpoint{3.664194in}{0.565235in}}%
\pgfusepath{stroke}%
\end{pgfscope}%
\begin{pgfscope}%
\pgfpathrectangle{\pgfqpoint{0.550713in}{0.127635in}}{\pgfqpoint{3.194133in}{2.297424in}}%
\pgfusepath{clip}%
\pgfsetrectcap%
\pgfsetroundjoin%
\pgfsetlinewidth{0.752812pt}%
\definecolor{currentstroke}{rgb}{0.000000,0.000000,0.000000}%
\pgfsetstrokecolor{currentstroke}%
\pgfsetdash{}{0pt}%
\pgfpathmoveto{\pgfqpoint{3.585938in}{0.775535in}}%
\pgfpathlineto{\pgfqpoint{3.664194in}{0.775535in}}%
\pgfusepath{stroke}%
\end{pgfscope}%
\begin{pgfscope}%
\pgfpathrectangle{\pgfqpoint{0.550713in}{0.127635in}}{\pgfqpoint{3.194133in}{2.297424in}}%
\pgfusepath{clip}%
\pgfsetrectcap%
\pgfsetroundjoin%
\pgfsetlinewidth{0.752812pt}%
\definecolor{currentstroke}{rgb}{0.000000,0.000000,0.000000}%
\pgfsetstrokecolor{currentstroke}%
\pgfsetdash{}{0pt}%
\pgfpathmoveto{\pgfqpoint{0.592236in}{0.727280in}}%
\pgfpathlineto{\pgfqpoint{0.748749in}{0.727280in}}%
\pgfusepath{stroke}%
\end{pgfscope}%
\begin{pgfscope}%
\pgfpathrectangle{\pgfqpoint{0.550713in}{0.127635in}}{\pgfqpoint{3.194133in}{2.297424in}}%
\pgfusepath{clip}%
\pgfsetbuttcap%
\pgfsetroundjoin%
\definecolor{currentfill}{rgb}{1.000000,1.000000,1.000000}%
\pgfsetfillcolor{currentfill}%
\pgfsetlinewidth{1.003750pt}%
\definecolor{currentstroke}{rgb}{0.000000,0.000000,0.000000}%
\pgfsetstrokecolor{currentstroke}%
\pgfsetdash{}{0pt}%
\pgfsys@defobject{currentmarker}{\pgfqpoint{-0.027778in}{-0.027778in}}{\pgfqpoint{0.027778in}{0.027778in}}{%
\pgfpathmoveto{\pgfqpoint{0.000000in}{-0.027778in}}%
\pgfpathcurveto{\pgfqpoint{0.007367in}{-0.027778in}}{\pgfqpoint{0.014433in}{-0.024851in}}{\pgfqpoint{0.019642in}{-0.019642in}}%
\pgfpathcurveto{\pgfqpoint{0.024851in}{-0.014433in}}{\pgfqpoint{0.027778in}{-0.007367in}}{\pgfqpoint{0.027778in}{0.000000in}}%
\pgfpathcurveto{\pgfqpoint{0.027778in}{0.007367in}}{\pgfqpoint{0.024851in}{0.014433in}}{\pgfqpoint{0.019642in}{0.019642in}}%
\pgfpathcurveto{\pgfqpoint{0.014433in}{0.024851in}}{\pgfqpoint{0.007367in}{0.027778in}}{\pgfqpoint{0.000000in}{0.027778in}}%
\pgfpathcurveto{\pgfqpoint{-0.007367in}{0.027778in}}{\pgfqpoint{-0.014433in}{0.024851in}}{\pgfqpoint{-0.019642in}{0.019642in}}%
\pgfpathcurveto{\pgfqpoint{-0.024851in}{0.014433in}}{\pgfqpoint{-0.027778in}{0.007367in}}{\pgfqpoint{-0.027778in}{0.000000in}}%
\pgfpathcurveto{\pgfqpoint{-0.027778in}{-0.007367in}}{\pgfqpoint{-0.024851in}{-0.014433in}}{\pgfqpoint{-0.019642in}{-0.019642in}}%
\pgfpathcurveto{\pgfqpoint{-0.014433in}{-0.024851in}}{\pgfqpoint{-0.007367in}{-0.027778in}}{\pgfqpoint{0.000000in}{-0.027778in}}%
\pgfpathclose%
\pgfusepath{stroke,fill}%
}%
\begin{pgfscope}%
\pgfsys@transformshift{0.670493in}{0.796776in}%
\pgfsys@useobject{currentmarker}{}%
\end{pgfscope}%
\end{pgfscope}%
\begin{pgfscope}%
\pgfpathrectangle{\pgfqpoint{0.550713in}{0.127635in}}{\pgfqpoint{3.194133in}{2.297424in}}%
\pgfusepath{clip}%
\pgfsetrectcap%
\pgfsetroundjoin%
\pgfsetlinewidth{0.752812pt}%
\definecolor{currentstroke}{rgb}{0.000000,0.000000,0.000000}%
\pgfsetstrokecolor{currentstroke}%
\pgfsetdash{}{0pt}%
\pgfpathmoveto{\pgfqpoint{0.751943in}{1.129784in}}%
\pgfpathlineto{\pgfqpoint{0.908456in}{1.129784in}}%
\pgfusepath{stroke}%
\end{pgfscope}%
\begin{pgfscope}%
\pgfpathrectangle{\pgfqpoint{0.550713in}{0.127635in}}{\pgfqpoint{3.194133in}{2.297424in}}%
\pgfusepath{clip}%
\pgfsetbuttcap%
\pgfsetroundjoin%
\definecolor{currentfill}{rgb}{1.000000,1.000000,1.000000}%
\pgfsetfillcolor{currentfill}%
\pgfsetlinewidth{1.003750pt}%
\definecolor{currentstroke}{rgb}{0.000000,0.000000,0.000000}%
\pgfsetstrokecolor{currentstroke}%
\pgfsetdash{}{0pt}%
\pgfsys@defobject{currentmarker}{\pgfqpoint{-0.027778in}{-0.027778in}}{\pgfqpoint{0.027778in}{0.027778in}}{%
\pgfpathmoveto{\pgfqpoint{0.000000in}{-0.027778in}}%
\pgfpathcurveto{\pgfqpoint{0.007367in}{-0.027778in}}{\pgfqpoint{0.014433in}{-0.024851in}}{\pgfqpoint{0.019642in}{-0.019642in}}%
\pgfpathcurveto{\pgfqpoint{0.024851in}{-0.014433in}}{\pgfqpoint{0.027778in}{-0.007367in}}{\pgfqpoint{0.027778in}{0.000000in}}%
\pgfpathcurveto{\pgfqpoint{0.027778in}{0.007367in}}{\pgfqpoint{0.024851in}{0.014433in}}{\pgfqpoint{0.019642in}{0.019642in}}%
\pgfpathcurveto{\pgfqpoint{0.014433in}{0.024851in}}{\pgfqpoint{0.007367in}{0.027778in}}{\pgfqpoint{0.000000in}{0.027778in}}%
\pgfpathcurveto{\pgfqpoint{-0.007367in}{0.027778in}}{\pgfqpoint{-0.014433in}{0.024851in}}{\pgfqpoint{-0.019642in}{0.019642in}}%
\pgfpathcurveto{\pgfqpoint{-0.024851in}{0.014433in}}{\pgfqpoint{-0.027778in}{0.007367in}}{\pgfqpoint{-0.027778in}{0.000000in}}%
\pgfpathcurveto{\pgfqpoint{-0.027778in}{-0.007367in}}{\pgfqpoint{-0.024851in}{-0.014433in}}{\pgfqpoint{-0.019642in}{-0.019642in}}%
\pgfpathcurveto{\pgfqpoint{-0.014433in}{-0.024851in}}{\pgfqpoint{-0.007367in}{-0.027778in}}{\pgfqpoint{0.000000in}{-0.027778in}}%
\pgfpathclose%
\pgfusepath{stroke,fill}%
}%
\begin{pgfscope}%
\pgfsys@transformshift{0.830199in}{1.197467in}%
\pgfsys@useobject{currentmarker}{}%
\end{pgfscope}%
\end{pgfscope}%
\begin{pgfscope}%
\pgfpathrectangle{\pgfqpoint{0.550713in}{0.127635in}}{\pgfqpoint{3.194133in}{2.297424in}}%
\pgfusepath{clip}%
\pgfsetrectcap%
\pgfsetroundjoin%
\pgfsetlinewidth{0.752812pt}%
\definecolor{currentstroke}{rgb}{0.000000,0.000000,0.000000}%
\pgfsetstrokecolor{currentstroke}%
\pgfsetdash{}{0pt}%
\pgfpathmoveto{\pgfqpoint{0.991503in}{1.994663in}}%
\pgfpathlineto{\pgfqpoint{1.148015in}{1.994663in}}%
\pgfusepath{stroke}%
\end{pgfscope}%
\begin{pgfscope}%
\pgfpathrectangle{\pgfqpoint{0.550713in}{0.127635in}}{\pgfqpoint{3.194133in}{2.297424in}}%
\pgfusepath{clip}%
\pgfsetbuttcap%
\pgfsetroundjoin%
\definecolor{currentfill}{rgb}{1.000000,1.000000,1.000000}%
\pgfsetfillcolor{currentfill}%
\pgfsetlinewidth{1.003750pt}%
\definecolor{currentstroke}{rgb}{0.000000,0.000000,0.000000}%
\pgfsetstrokecolor{currentstroke}%
\pgfsetdash{}{0pt}%
\pgfsys@defobject{currentmarker}{\pgfqpoint{-0.027778in}{-0.027778in}}{\pgfqpoint{0.027778in}{0.027778in}}{%
\pgfpathmoveto{\pgfqpoint{0.000000in}{-0.027778in}}%
\pgfpathcurveto{\pgfqpoint{0.007367in}{-0.027778in}}{\pgfqpoint{0.014433in}{-0.024851in}}{\pgfqpoint{0.019642in}{-0.019642in}}%
\pgfpathcurveto{\pgfqpoint{0.024851in}{-0.014433in}}{\pgfqpoint{0.027778in}{-0.007367in}}{\pgfqpoint{0.027778in}{0.000000in}}%
\pgfpathcurveto{\pgfqpoint{0.027778in}{0.007367in}}{\pgfqpoint{0.024851in}{0.014433in}}{\pgfqpoint{0.019642in}{0.019642in}}%
\pgfpathcurveto{\pgfqpoint{0.014433in}{0.024851in}}{\pgfqpoint{0.007367in}{0.027778in}}{\pgfqpoint{0.000000in}{0.027778in}}%
\pgfpathcurveto{\pgfqpoint{-0.007367in}{0.027778in}}{\pgfqpoint{-0.014433in}{0.024851in}}{\pgfqpoint{-0.019642in}{0.019642in}}%
\pgfpathcurveto{\pgfqpoint{-0.024851in}{0.014433in}}{\pgfqpoint{-0.027778in}{0.007367in}}{\pgfqpoint{-0.027778in}{0.000000in}}%
\pgfpathcurveto{\pgfqpoint{-0.027778in}{-0.007367in}}{\pgfqpoint{-0.024851in}{-0.014433in}}{\pgfqpoint{-0.019642in}{-0.019642in}}%
\pgfpathcurveto{\pgfqpoint{-0.014433in}{-0.024851in}}{\pgfqpoint{-0.007367in}{-0.027778in}}{\pgfqpoint{0.000000in}{-0.027778in}}%
\pgfpathclose%
\pgfusepath{stroke,fill}%
}%
\begin{pgfscope}%
\pgfsys@transformshift{1.069759in}{2.018444in}%
\pgfsys@useobject{currentmarker}{}%
\end{pgfscope}%
\end{pgfscope}%
\begin{pgfscope}%
\pgfpathrectangle{\pgfqpoint{0.550713in}{0.127635in}}{\pgfqpoint{3.194133in}{2.297424in}}%
\pgfusepath{clip}%
\pgfsetrectcap%
\pgfsetroundjoin%
\pgfsetlinewidth{0.752812pt}%
\definecolor{currentstroke}{rgb}{0.000000,0.000000,0.000000}%
\pgfsetstrokecolor{currentstroke}%
\pgfsetdash{}{0pt}%
\pgfpathmoveto{\pgfqpoint{1.151210in}{2.084150in}}%
\pgfpathlineto{\pgfqpoint{1.307722in}{2.084150in}}%
\pgfusepath{stroke}%
\end{pgfscope}%
\begin{pgfscope}%
\pgfpathrectangle{\pgfqpoint{0.550713in}{0.127635in}}{\pgfqpoint{3.194133in}{2.297424in}}%
\pgfusepath{clip}%
\pgfsetbuttcap%
\pgfsetroundjoin%
\definecolor{currentfill}{rgb}{1.000000,1.000000,1.000000}%
\pgfsetfillcolor{currentfill}%
\pgfsetlinewidth{1.003750pt}%
\definecolor{currentstroke}{rgb}{0.000000,0.000000,0.000000}%
\pgfsetstrokecolor{currentstroke}%
\pgfsetdash{}{0pt}%
\pgfsys@defobject{currentmarker}{\pgfqpoint{-0.027778in}{-0.027778in}}{\pgfqpoint{0.027778in}{0.027778in}}{%
\pgfpathmoveto{\pgfqpoint{0.000000in}{-0.027778in}}%
\pgfpathcurveto{\pgfqpoint{0.007367in}{-0.027778in}}{\pgfqpoint{0.014433in}{-0.024851in}}{\pgfqpoint{0.019642in}{-0.019642in}}%
\pgfpathcurveto{\pgfqpoint{0.024851in}{-0.014433in}}{\pgfqpoint{0.027778in}{-0.007367in}}{\pgfqpoint{0.027778in}{0.000000in}}%
\pgfpathcurveto{\pgfqpoint{0.027778in}{0.007367in}}{\pgfqpoint{0.024851in}{0.014433in}}{\pgfqpoint{0.019642in}{0.019642in}}%
\pgfpathcurveto{\pgfqpoint{0.014433in}{0.024851in}}{\pgfqpoint{0.007367in}{0.027778in}}{\pgfqpoint{0.000000in}{0.027778in}}%
\pgfpathcurveto{\pgfqpoint{-0.007367in}{0.027778in}}{\pgfqpoint{-0.014433in}{0.024851in}}{\pgfqpoint{-0.019642in}{0.019642in}}%
\pgfpathcurveto{\pgfqpoint{-0.024851in}{0.014433in}}{\pgfqpoint{-0.027778in}{0.007367in}}{\pgfqpoint{-0.027778in}{0.000000in}}%
\pgfpathcurveto{\pgfqpoint{-0.027778in}{-0.007367in}}{\pgfqpoint{-0.024851in}{-0.014433in}}{\pgfqpoint{-0.019642in}{-0.019642in}}%
\pgfpathcurveto{\pgfqpoint{-0.014433in}{-0.024851in}}{\pgfqpoint{-0.007367in}{-0.027778in}}{\pgfqpoint{0.000000in}{-0.027778in}}%
\pgfpathclose%
\pgfusepath{stroke,fill}%
}%
\begin{pgfscope}%
\pgfsys@transformshift{1.229466in}{2.047037in}%
\pgfsys@useobject{currentmarker}{}%
\end{pgfscope}%
\end{pgfscope}%
\begin{pgfscope}%
\pgfpathrectangle{\pgfqpoint{0.550713in}{0.127635in}}{\pgfqpoint{3.194133in}{2.297424in}}%
\pgfusepath{clip}%
\pgfsetrectcap%
\pgfsetroundjoin%
\pgfsetlinewidth{0.752812pt}%
\definecolor{currentstroke}{rgb}{0.000000,0.000000,0.000000}%
\pgfsetstrokecolor{currentstroke}%
\pgfsetdash{}{0pt}%
\pgfpathmoveto{\pgfqpoint{1.390770in}{0.737352in}}%
\pgfpathlineto{\pgfqpoint{1.547282in}{0.737352in}}%
\pgfusepath{stroke}%
\end{pgfscope}%
\begin{pgfscope}%
\pgfpathrectangle{\pgfqpoint{0.550713in}{0.127635in}}{\pgfqpoint{3.194133in}{2.297424in}}%
\pgfusepath{clip}%
\pgfsetbuttcap%
\pgfsetroundjoin%
\definecolor{currentfill}{rgb}{1.000000,1.000000,1.000000}%
\pgfsetfillcolor{currentfill}%
\pgfsetlinewidth{1.003750pt}%
\definecolor{currentstroke}{rgb}{0.000000,0.000000,0.000000}%
\pgfsetstrokecolor{currentstroke}%
\pgfsetdash{}{0pt}%
\pgfsys@defobject{currentmarker}{\pgfqpoint{-0.027778in}{-0.027778in}}{\pgfqpoint{0.027778in}{0.027778in}}{%
\pgfpathmoveto{\pgfqpoint{0.000000in}{-0.027778in}}%
\pgfpathcurveto{\pgfqpoint{0.007367in}{-0.027778in}}{\pgfqpoint{0.014433in}{-0.024851in}}{\pgfqpoint{0.019642in}{-0.019642in}}%
\pgfpathcurveto{\pgfqpoint{0.024851in}{-0.014433in}}{\pgfqpoint{0.027778in}{-0.007367in}}{\pgfqpoint{0.027778in}{0.000000in}}%
\pgfpathcurveto{\pgfqpoint{0.027778in}{0.007367in}}{\pgfqpoint{0.024851in}{0.014433in}}{\pgfqpoint{0.019642in}{0.019642in}}%
\pgfpathcurveto{\pgfqpoint{0.014433in}{0.024851in}}{\pgfqpoint{0.007367in}{0.027778in}}{\pgfqpoint{0.000000in}{0.027778in}}%
\pgfpathcurveto{\pgfqpoint{-0.007367in}{0.027778in}}{\pgfqpoint{-0.014433in}{0.024851in}}{\pgfqpoint{-0.019642in}{0.019642in}}%
\pgfpathcurveto{\pgfqpoint{-0.024851in}{0.014433in}}{\pgfqpoint{-0.027778in}{0.007367in}}{\pgfqpoint{-0.027778in}{0.000000in}}%
\pgfpathcurveto{\pgfqpoint{-0.027778in}{-0.007367in}}{\pgfqpoint{-0.024851in}{-0.014433in}}{\pgfqpoint{-0.019642in}{-0.019642in}}%
\pgfpathcurveto{\pgfqpoint{-0.014433in}{-0.024851in}}{\pgfqpoint{-0.007367in}{-0.027778in}}{\pgfqpoint{0.000000in}{-0.027778in}}%
\pgfpathclose%
\pgfusepath{stroke,fill}%
}%
\begin{pgfscope}%
\pgfsys@transformshift{1.469026in}{0.741660in}%
\pgfsys@useobject{currentmarker}{}%
\end{pgfscope}%
\end{pgfscope}%
\begin{pgfscope}%
\pgfpathrectangle{\pgfqpoint{0.550713in}{0.127635in}}{\pgfqpoint{3.194133in}{2.297424in}}%
\pgfusepath{clip}%
\pgfsetrectcap%
\pgfsetroundjoin%
\pgfsetlinewidth{0.752812pt}%
\definecolor{currentstroke}{rgb}{0.000000,0.000000,0.000000}%
\pgfsetstrokecolor{currentstroke}%
\pgfsetdash{}{0pt}%
\pgfpathmoveto{\pgfqpoint{1.550476in}{0.752370in}}%
\pgfpathlineto{\pgfqpoint{1.706989in}{0.752370in}}%
\pgfusepath{stroke}%
\end{pgfscope}%
\begin{pgfscope}%
\pgfpathrectangle{\pgfqpoint{0.550713in}{0.127635in}}{\pgfqpoint{3.194133in}{2.297424in}}%
\pgfusepath{clip}%
\pgfsetbuttcap%
\pgfsetroundjoin%
\definecolor{currentfill}{rgb}{1.000000,1.000000,1.000000}%
\pgfsetfillcolor{currentfill}%
\pgfsetlinewidth{1.003750pt}%
\definecolor{currentstroke}{rgb}{0.000000,0.000000,0.000000}%
\pgfsetstrokecolor{currentstroke}%
\pgfsetdash{}{0pt}%
\pgfsys@defobject{currentmarker}{\pgfqpoint{-0.027778in}{-0.027778in}}{\pgfqpoint{0.027778in}{0.027778in}}{%
\pgfpathmoveto{\pgfqpoint{0.000000in}{-0.027778in}}%
\pgfpathcurveto{\pgfqpoint{0.007367in}{-0.027778in}}{\pgfqpoint{0.014433in}{-0.024851in}}{\pgfqpoint{0.019642in}{-0.019642in}}%
\pgfpathcurveto{\pgfqpoint{0.024851in}{-0.014433in}}{\pgfqpoint{0.027778in}{-0.007367in}}{\pgfqpoint{0.027778in}{0.000000in}}%
\pgfpathcurveto{\pgfqpoint{0.027778in}{0.007367in}}{\pgfqpoint{0.024851in}{0.014433in}}{\pgfqpoint{0.019642in}{0.019642in}}%
\pgfpathcurveto{\pgfqpoint{0.014433in}{0.024851in}}{\pgfqpoint{0.007367in}{0.027778in}}{\pgfqpoint{0.000000in}{0.027778in}}%
\pgfpathcurveto{\pgfqpoint{-0.007367in}{0.027778in}}{\pgfqpoint{-0.014433in}{0.024851in}}{\pgfqpoint{-0.019642in}{0.019642in}}%
\pgfpathcurveto{\pgfqpoint{-0.024851in}{0.014433in}}{\pgfqpoint{-0.027778in}{0.007367in}}{\pgfqpoint{-0.027778in}{0.000000in}}%
\pgfpathcurveto{\pgfqpoint{-0.027778in}{-0.007367in}}{\pgfqpoint{-0.024851in}{-0.014433in}}{\pgfqpoint{-0.019642in}{-0.019642in}}%
\pgfpathcurveto{\pgfqpoint{-0.014433in}{-0.024851in}}{\pgfqpoint{-0.007367in}{-0.027778in}}{\pgfqpoint{0.000000in}{-0.027778in}}%
\pgfpathclose%
\pgfusepath{stroke,fill}%
}%
\begin{pgfscope}%
\pgfsys@transformshift{1.628733in}{0.807033in}%
\pgfsys@useobject{currentmarker}{}%
\end{pgfscope}%
\end{pgfscope}%
\begin{pgfscope}%
\pgfpathrectangle{\pgfqpoint{0.550713in}{0.127635in}}{\pgfqpoint{3.194133in}{2.297424in}}%
\pgfusepath{clip}%
\pgfsetrectcap%
\pgfsetroundjoin%
\pgfsetlinewidth{0.752812pt}%
\definecolor{currentstroke}{rgb}{0.000000,0.000000,0.000000}%
\pgfsetstrokecolor{currentstroke}%
\pgfsetdash{}{0pt}%
\pgfpathmoveto{\pgfqpoint{1.790036in}{1.145738in}}%
\pgfpathlineto{\pgfqpoint{1.946549in}{1.145738in}}%
\pgfusepath{stroke}%
\end{pgfscope}%
\begin{pgfscope}%
\pgfpathrectangle{\pgfqpoint{0.550713in}{0.127635in}}{\pgfqpoint{3.194133in}{2.297424in}}%
\pgfusepath{clip}%
\pgfsetbuttcap%
\pgfsetroundjoin%
\definecolor{currentfill}{rgb}{1.000000,1.000000,1.000000}%
\pgfsetfillcolor{currentfill}%
\pgfsetlinewidth{1.003750pt}%
\definecolor{currentstroke}{rgb}{0.000000,0.000000,0.000000}%
\pgfsetstrokecolor{currentstroke}%
\pgfsetdash{}{0pt}%
\pgfsys@defobject{currentmarker}{\pgfqpoint{-0.027778in}{-0.027778in}}{\pgfqpoint{0.027778in}{0.027778in}}{%
\pgfpathmoveto{\pgfqpoint{0.000000in}{-0.027778in}}%
\pgfpathcurveto{\pgfqpoint{0.007367in}{-0.027778in}}{\pgfqpoint{0.014433in}{-0.024851in}}{\pgfqpoint{0.019642in}{-0.019642in}}%
\pgfpathcurveto{\pgfqpoint{0.024851in}{-0.014433in}}{\pgfqpoint{0.027778in}{-0.007367in}}{\pgfqpoint{0.027778in}{0.000000in}}%
\pgfpathcurveto{\pgfqpoint{0.027778in}{0.007367in}}{\pgfqpoint{0.024851in}{0.014433in}}{\pgfqpoint{0.019642in}{0.019642in}}%
\pgfpathcurveto{\pgfqpoint{0.014433in}{0.024851in}}{\pgfqpoint{0.007367in}{0.027778in}}{\pgfqpoint{0.000000in}{0.027778in}}%
\pgfpathcurveto{\pgfqpoint{-0.007367in}{0.027778in}}{\pgfqpoint{-0.014433in}{0.024851in}}{\pgfqpoint{-0.019642in}{0.019642in}}%
\pgfpathcurveto{\pgfqpoint{-0.024851in}{0.014433in}}{\pgfqpoint{-0.027778in}{0.007367in}}{\pgfqpoint{-0.027778in}{0.000000in}}%
\pgfpathcurveto{\pgfqpoint{-0.027778in}{-0.007367in}}{\pgfqpoint{-0.024851in}{-0.014433in}}{\pgfqpoint{-0.019642in}{-0.019642in}}%
\pgfpathcurveto{\pgfqpoint{-0.014433in}{-0.024851in}}{\pgfqpoint{-0.007367in}{-0.027778in}}{\pgfqpoint{0.000000in}{-0.027778in}}%
\pgfpathclose%
\pgfusepath{stroke,fill}%
}%
\begin{pgfscope}%
\pgfsys@transformshift{1.868293in}{1.091150in}%
\pgfsys@useobject{currentmarker}{}%
\end{pgfscope}%
\end{pgfscope}%
\begin{pgfscope}%
\pgfpathrectangle{\pgfqpoint{0.550713in}{0.127635in}}{\pgfqpoint{3.194133in}{2.297424in}}%
\pgfusepath{clip}%
\pgfsetrectcap%
\pgfsetroundjoin%
\pgfsetlinewidth{0.752812pt}%
\definecolor{currentstroke}{rgb}{0.000000,0.000000,0.000000}%
\pgfsetstrokecolor{currentstroke}%
\pgfsetdash{}{0pt}%
\pgfpathmoveto{\pgfqpoint{1.949743in}{0.881393in}}%
\pgfpathlineto{\pgfqpoint{2.106255in}{0.881393in}}%
\pgfusepath{stroke}%
\end{pgfscope}%
\begin{pgfscope}%
\pgfpathrectangle{\pgfqpoint{0.550713in}{0.127635in}}{\pgfqpoint{3.194133in}{2.297424in}}%
\pgfusepath{clip}%
\pgfsetbuttcap%
\pgfsetroundjoin%
\definecolor{currentfill}{rgb}{1.000000,1.000000,1.000000}%
\pgfsetfillcolor{currentfill}%
\pgfsetlinewidth{1.003750pt}%
\definecolor{currentstroke}{rgb}{0.000000,0.000000,0.000000}%
\pgfsetstrokecolor{currentstroke}%
\pgfsetdash{}{0pt}%
\pgfsys@defobject{currentmarker}{\pgfqpoint{-0.027778in}{-0.027778in}}{\pgfqpoint{0.027778in}{0.027778in}}{%
\pgfpathmoveto{\pgfqpoint{0.000000in}{-0.027778in}}%
\pgfpathcurveto{\pgfqpoint{0.007367in}{-0.027778in}}{\pgfqpoint{0.014433in}{-0.024851in}}{\pgfqpoint{0.019642in}{-0.019642in}}%
\pgfpathcurveto{\pgfqpoint{0.024851in}{-0.014433in}}{\pgfqpoint{0.027778in}{-0.007367in}}{\pgfqpoint{0.027778in}{0.000000in}}%
\pgfpathcurveto{\pgfqpoint{0.027778in}{0.007367in}}{\pgfqpoint{0.024851in}{0.014433in}}{\pgfqpoint{0.019642in}{0.019642in}}%
\pgfpathcurveto{\pgfqpoint{0.014433in}{0.024851in}}{\pgfqpoint{0.007367in}{0.027778in}}{\pgfqpoint{0.000000in}{0.027778in}}%
\pgfpathcurveto{\pgfqpoint{-0.007367in}{0.027778in}}{\pgfqpoint{-0.014433in}{0.024851in}}{\pgfqpoint{-0.019642in}{0.019642in}}%
\pgfpathcurveto{\pgfqpoint{-0.024851in}{0.014433in}}{\pgfqpoint{-0.027778in}{0.007367in}}{\pgfqpoint{-0.027778in}{0.000000in}}%
\pgfpathcurveto{\pgfqpoint{-0.027778in}{-0.007367in}}{\pgfqpoint{-0.024851in}{-0.014433in}}{\pgfqpoint{-0.019642in}{-0.019642in}}%
\pgfpathcurveto{\pgfqpoint{-0.014433in}{-0.024851in}}{\pgfqpoint{-0.007367in}{-0.027778in}}{\pgfqpoint{0.000000in}{-0.027778in}}%
\pgfpathclose%
\pgfusepath{stroke,fill}%
}%
\begin{pgfscope}%
\pgfsys@transformshift{2.027999in}{0.972257in}%
\pgfsys@useobject{currentmarker}{}%
\end{pgfscope}%
\end{pgfscope}%
\begin{pgfscope}%
\pgfpathrectangle{\pgfqpoint{0.550713in}{0.127635in}}{\pgfqpoint{3.194133in}{2.297424in}}%
\pgfusepath{clip}%
\pgfsetrectcap%
\pgfsetroundjoin%
\pgfsetlinewidth{0.752812pt}%
\definecolor{currentstroke}{rgb}{0.000000,0.000000,0.000000}%
\pgfsetstrokecolor{currentstroke}%
\pgfsetdash{}{0pt}%
\pgfpathmoveto{\pgfqpoint{2.189303in}{0.538387in}}%
\pgfpathlineto{\pgfqpoint{2.345815in}{0.538387in}}%
\pgfusepath{stroke}%
\end{pgfscope}%
\begin{pgfscope}%
\pgfpathrectangle{\pgfqpoint{0.550713in}{0.127635in}}{\pgfqpoint{3.194133in}{2.297424in}}%
\pgfusepath{clip}%
\pgfsetbuttcap%
\pgfsetroundjoin%
\definecolor{currentfill}{rgb}{1.000000,1.000000,1.000000}%
\pgfsetfillcolor{currentfill}%
\pgfsetlinewidth{1.003750pt}%
\definecolor{currentstroke}{rgb}{0.000000,0.000000,0.000000}%
\pgfsetstrokecolor{currentstroke}%
\pgfsetdash{}{0pt}%
\pgfsys@defobject{currentmarker}{\pgfqpoint{-0.027778in}{-0.027778in}}{\pgfqpoint{0.027778in}{0.027778in}}{%
\pgfpathmoveto{\pgfqpoint{0.000000in}{-0.027778in}}%
\pgfpathcurveto{\pgfqpoint{0.007367in}{-0.027778in}}{\pgfqpoint{0.014433in}{-0.024851in}}{\pgfqpoint{0.019642in}{-0.019642in}}%
\pgfpathcurveto{\pgfqpoint{0.024851in}{-0.014433in}}{\pgfqpoint{0.027778in}{-0.007367in}}{\pgfqpoint{0.027778in}{0.000000in}}%
\pgfpathcurveto{\pgfqpoint{0.027778in}{0.007367in}}{\pgfqpoint{0.024851in}{0.014433in}}{\pgfqpoint{0.019642in}{0.019642in}}%
\pgfpathcurveto{\pgfqpoint{0.014433in}{0.024851in}}{\pgfqpoint{0.007367in}{0.027778in}}{\pgfqpoint{0.000000in}{0.027778in}}%
\pgfpathcurveto{\pgfqpoint{-0.007367in}{0.027778in}}{\pgfqpoint{-0.014433in}{0.024851in}}{\pgfqpoint{-0.019642in}{0.019642in}}%
\pgfpathcurveto{\pgfqpoint{-0.024851in}{0.014433in}}{\pgfqpoint{-0.027778in}{0.007367in}}{\pgfqpoint{-0.027778in}{0.000000in}}%
\pgfpathcurveto{\pgfqpoint{-0.027778in}{-0.007367in}}{\pgfqpoint{-0.024851in}{-0.014433in}}{\pgfqpoint{-0.019642in}{-0.019642in}}%
\pgfpathcurveto{\pgfqpoint{-0.014433in}{-0.024851in}}{\pgfqpoint{-0.007367in}{-0.027778in}}{\pgfqpoint{0.000000in}{-0.027778in}}%
\pgfpathclose%
\pgfusepath{stroke,fill}%
}%
\begin{pgfscope}%
\pgfsys@transformshift{2.267559in}{0.544287in}%
\pgfsys@useobject{currentmarker}{}%
\end{pgfscope}%
\end{pgfscope}%
\begin{pgfscope}%
\pgfpathrectangle{\pgfqpoint{0.550713in}{0.127635in}}{\pgfqpoint{3.194133in}{2.297424in}}%
\pgfusepath{clip}%
\pgfsetrectcap%
\pgfsetroundjoin%
\pgfsetlinewidth{0.752812pt}%
\definecolor{currentstroke}{rgb}{0.000000,0.000000,0.000000}%
\pgfsetstrokecolor{currentstroke}%
\pgfsetdash{}{0pt}%
\pgfpathmoveto{\pgfqpoint{2.349010in}{0.527044in}}%
\pgfpathlineto{\pgfqpoint{2.505522in}{0.527044in}}%
\pgfusepath{stroke}%
\end{pgfscope}%
\begin{pgfscope}%
\pgfpathrectangle{\pgfqpoint{0.550713in}{0.127635in}}{\pgfqpoint{3.194133in}{2.297424in}}%
\pgfusepath{clip}%
\pgfsetbuttcap%
\pgfsetroundjoin%
\definecolor{currentfill}{rgb}{1.000000,1.000000,1.000000}%
\pgfsetfillcolor{currentfill}%
\pgfsetlinewidth{1.003750pt}%
\definecolor{currentstroke}{rgb}{0.000000,0.000000,0.000000}%
\pgfsetstrokecolor{currentstroke}%
\pgfsetdash{}{0pt}%
\pgfsys@defobject{currentmarker}{\pgfqpoint{-0.027778in}{-0.027778in}}{\pgfqpoint{0.027778in}{0.027778in}}{%
\pgfpathmoveto{\pgfqpoint{0.000000in}{-0.027778in}}%
\pgfpathcurveto{\pgfqpoint{0.007367in}{-0.027778in}}{\pgfqpoint{0.014433in}{-0.024851in}}{\pgfqpoint{0.019642in}{-0.019642in}}%
\pgfpathcurveto{\pgfqpoint{0.024851in}{-0.014433in}}{\pgfqpoint{0.027778in}{-0.007367in}}{\pgfqpoint{0.027778in}{0.000000in}}%
\pgfpathcurveto{\pgfqpoint{0.027778in}{0.007367in}}{\pgfqpoint{0.024851in}{0.014433in}}{\pgfqpoint{0.019642in}{0.019642in}}%
\pgfpathcurveto{\pgfqpoint{0.014433in}{0.024851in}}{\pgfqpoint{0.007367in}{0.027778in}}{\pgfqpoint{0.000000in}{0.027778in}}%
\pgfpathcurveto{\pgfqpoint{-0.007367in}{0.027778in}}{\pgfqpoint{-0.014433in}{0.024851in}}{\pgfqpoint{-0.019642in}{0.019642in}}%
\pgfpathcurveto{\pgfqpoint{-0.024851in}{0.014433in}}{\pgfqpoint{-0.027778in}{0.007367in}}{\pgfqpoint{-0.027778in}{0.000000in}}%
\pgfpathcurveto{\pgfqpoint{-0.027778in}{-0.007367in}}{\pgfqpoint{-0.024851in}{-0.014433in}}{\pgfqpoint{-0.019642in}{-0.019642in}}%
\pgfpathcurveto{\pgfqpoint{-0.014433in}{-0.024851in}}{\pgfqpoint{-0.007367in}{-0.027778in}}{\pgfqpoint{0.000000in}{-0.027778in}}%
\pgfpathclose%
\pgfusepath{stroke,fill}%
}%
\begin{pgfscope}%
\pgfsys@transformshift{2.427266in}{0.545630in}%
\pgfsys@useobject{currentmarker}{}%
\end{pgfscope}%
\end{pgfscope}%
\begin{pgfscope}%
\pgfpathrectangle{\pgfqpoint{0.550713in}{0.127635in}}{\pgfqpoint{3.194133in}{2.297424in}}%
\pgfusepath{clip}%
\pgfsetrectcap%
\pgfsetroundjoin%
\pgfsetlinewidth{0.752812pt}%
\definecolor{currentstroke}{rgb}{0.000000,0.000000,0.000000}%
\pgfsetstrokecolor{currentstroke}%
\pgfsetdash{}{0pt}%
\pgfpathmoveto{\pgfqpoint{2.588570in}{0.661207in}}%
\pgfpathlineto{\pgfqpoint{2.745082in}{0.661207in}}%
\pgfusepath{stroke}%
\end{pgfscope}%
\begin{pgfscope}%
\pgfpathrectangle{\pgfqpoint{0.550713in}{0.127635in}}{\pgfqpoint{3.194133in}{2.297424in}}%
\pgfusepath{clip}%
\pgfsetbuttcap%
\pgfsetroundjoin%
\definecolor{currentfill}{rgb}{1.000000,1.000000,1.000000}%
\pgfsetfillcolor{currentfill}%
\pgfsetlinewidth{1.003750pt}%
\definecolor{currentstroke}{rgb}{0.000000,0.000000,0.000000}%
\pgfsetstrokecolor{currentstroke}%
\pgfsetdash{}{0pt}%
\pgfsys@defobject{currentmarker}{\pgfqpoint{-0.027778in}{-0.027778in}}{\pgfqpoint{0.027778in}{0.027778in}}{%
\pgfpathmoveto{\pgfqpoint{0.000000in}{-0.027778in}}%
\pgfpathcurveto{\pgfqpoint{0.007367in}{-0.027778in}}{\pgfqpoint{0.014433in}{-0.024851in}}{\pgfqpoint{0.019642in}{-0.019642in}}%
\pgfpathcurveto{\pgfqpoint{0.024851in}{-0.014433in}}{\pgfqpoint{0.027778in}{-0.007367in}}{\pgfqpoint{0.027778in}{0.000000in}}%
\pgfpathcurveto{\pgfqpoint{0.027778in}{0.007367in}}{\pgfqpoint{0.024851in}{0.014433in}}{\pgfqpoint{0.019642in}{0.019642in}}%
\pgfpathcurveto{\pgfqpoint{0.014433in}{0.024851in}}{\pgfqpoint{0.007367in}{0.027778in}}{\pgfqpoint{0.000000in}{0.027778in}}%
\pgfpathcurveto{\pgfqpoint{-0.007367in}{0.027778in}}{\pgfqpoint{-0.014433in}{0.024851in}}{\pgfqpoint{-0.019642in}{0.019642in}}%
\pgfpathcurveto{\pgfqpoint{-0.024851in}{0.014433in}}{\pgfqpoint{-0.027778in}{0.007367in}}{\pgfqpoint{-0.027778in}{0.000000in}}%
\pgfpathcurveto{\pgfqpoint{-0.027778in}{-0.007367in}}{\pgfqpoint{-0.024851in}{-0.014433in}}{\pgfqpoint{-0.019642in}{-0.019642in}}%
\pgfpathcurveto{\pgfqpoint{-0.014433in}{-0.024851in}}{\pgfqpoint{-0.007367in}{-0.027778in}}{\pgfqpoint{0.000000in}{-0.027778in}}%
\pgfpathclose%
\pgfusepath{stroke,fill}%
}%
\begin{pgfscope}%
\pgfsys@transformshift{2.666826in}{0.636194in}%
\pgfsys@useobject{currentmarker}{}%
\end{pgfscope}%
\end{pgfscope}%
\begin{pgfscope}%
\pgfpathrectangle{\pgfqpoint{0.550713in}{0.127635in}}{\pgfqpoint{3.194133in}{2.297424in}}%
\pgfusepath{clip}%
\pgfsetrectcap%
\pgfsetroundjoin%
\pgfsetlinewidth{0.752812pt}%
\definecolor{currentstroke}{rgb}{0.000000,0.000000,0.000000}%
\pgfsetstrokecolor{currentstroke}%
\pgfsetdash{}{0pt}%
\pgfpathmoveto{\pgfqpoint{2.748276in}{0.691658in}}%
\pgfpathlineto{\pgfqpoint{2.904789in}{0.691658in}}%
\pgfusepath{stroke}%
\end{pgfscope}%
\begin{pgfscope}%
\pgfpathrectangle{\pgfqpoint{0.550713in}{0.127635in}}{\pgfqpoint{3.194133in}{2.297424in}}%
\pgfusepath{clip}%
\pgfsetbuttcap%
\pgfsetroundjoin%
\definecolor{currentfill}{rgb}{1.000000,1.000000,1.000000}%
\pgfsetfillcolor{currentfill}%
\pgfsetlinewidth{1.003750pt}%
\definecolor{currentstroke}{rgb}{0.000000,0.000000,0.000000}%
\pgfsetstrokecolor{currentstroke}%
\pgfsetdash{}{0pt}%
\pgfsys@defobject{currentmarker}{\pgfqpoint{-0.027778in}{-0.027778in}}{\pgfqpoint{0.027778in}{0.027778in}}{%
\pgfpathmoveto{\pgfqpoint{0.000000in}{-0.027778in}}%
\pgfpathcurveto{\pgfqpoint{0.007367in}{-0.027778in}}{\pgfqpoint{0.014433in}{-0.024851in}}{\pgfqpoint{0.019642in}{-0.019642in}}%
\pgfpathcurveto{\pgfqpoint{0.024851in}{-0.014433in}}{\pgfqpoint{0.027778in}{-0.007367in}}{\pgfqpoint{0.027778in}{0.000000in}}%
\pgfpathcurveto{\pgfqpoint{0.027778in}{0.007367in}}{\pgfqpoint{0.024851in}{0.014433in}}{\pgfqpoint{0.019642in}{0.019642in}}%
\pgfpathcurveto{\pgfqpoint{0.014433in}{0.024851in}}{\pgfqpoint{0.007367in}{0.027778in}}{\pgfqpoint{0.000000in}{0.027778in}}%
\pgfpathcurveto{\pgfqpoint{-0.007367in}{0.027778in}}{\pgfqpoint{-0.014433in}{0.024851in}}{\pgfqpoint{-0.019642in}{0.019642in}}%
\pgfpathcurveto{\pgfqpoint{-0.024851in}{0.014433in}}{\pgfqpoint{-0.027778in}{0.007367in}}{\pgfqpoint{-0.027778in}{0.000000in}}%
\pgfpathcurveto{\pgfqpoint{-0.027778in}{-0.007367in}}{\pgfqpoint{-0.024851in}{-0.014433in}}{\pgfqpoint{-0.019642in}{-0.019642in}}%
\pgfpathcurveto{\pgfqpoint{-0.014433in}{-0.024851in}}{\pgfqpoint{-0.007367in}{-0.027778in}}{\pgfqpoint{0.000000in}{-0.027778in}}%
\pgfpathclose%
\pgfusepath{stroke,fill}%
}%
\begin{pgfscope}%
\pgfsys@transformshift{2.826532in}{0.690261in}%
\pgfsys@useobject{currentmarker}{}%
\end{pgfscope}%
\end{pgfscope}%
\begin{pgfscope}%
\pgfpathrectangle{\pgfqpoint{0.550713in}{0.127635in}}{\pgfqpoint{3.194133in}{2.297424in}}%
\pgfusepath{clip}%
\pgfsetrectcap%
\pgfsetroundjoin%
\pgfsetlinewidth{0.752812pt}%
\definecolor{currentstroke}{rgb}{0.000000,0.000000,0.000000}%
\pgfsetstrokecolor{currentstroke}%
\pgfsetdash{}{0pt}%
\pgfpathmoveto{\pgfqpoint{2.987836in}{0.424188in}}%
\pgfpathlineto{\pgfqpoint{3.144349in}{0.424188in}}%
\pgfusepath{stroke}%
\end{pgfscope}%
\begin{pgfscope}%
\pgfpathrectangle{\pgfqpoint{0.550713in}{0.127635in}}{\pgfqpoint{3.194133in}{2.297424in}}%
\pgfusepath{clip}%
\pgfsetbuttcap%
\pgfsetroundjoin%
\definecolor{currentfill}{rgb}{1.000000,1.000000,1.000000}%
\pgfsetfillcolor{currentfill}%
\pgfsetlinewidth{1.003750pt}%
\definecolor{currentstroke}{rgb}{0.000000,0.000000,0.000000}%
\pgfsetstrokecolor{currentstroke}%
\pgfsetdash{}{0pt}%
\pgfsys@defobject{currentmarker}{\pgfqpoint{-0.027778in}{-0.027778in}}{\pgfqpoint{0.027778in}{0.027778in}}{%
\pgfpathmoveto{\pgfqpoint{0.000000in}{-0.027778in}}%
\pgfpathcurveto{\pgfqpoint{0.007367in}{-0.027778in}}{\pgfqpoint{0.014433in}{-0.024851in}}{\pgfqpoint{0.019642in}{-0.019642in}}%
\pgfpathcurveto{\pgfqpoint{0.024851in}{-0.014433in}}{\pgfqpoint{0.027778in}{-0.007367in}}{\pgfqpoint{0.027778in}{0.000000in}}%
\pgfpathcurveto{\pgfqpoint{0.027778in}{0.007367in}}{\pgfqpoint{0.024851in}{0.014433in}}{\pgfqpoint{0.019642in}{0.019642in}}%
\pgfpathcurveto{\pgfqpoint{0.014433in}{0.024851in}}{\pgfqpoint{0.007367in}{0.027778in}}{\pgfqpoint{0.000000in}{0.027778in}}%
\pgfpathcurveto{\pgfqpoint{-0.007367in}{0.027778in}}{\pgfqpoint{-0.014433in}{0.024851in}}{\pgfqpoint{-0.019642in}{0.019642in}}%
\pgfpathcurveto{\pgfqpoint{-0.024851in}{0.014433in}}{\pgfqpoint{-0.027778in}{0.007367in}}{\pgfqpoint{-0.027778in}{0.000000in}}%
\pgfpathcurveto{\pgfqpoint{-0.027778in}{-0.007367in}}{\pgfqpoint{-0.024851in}{-0.014433in}}{\pgfqpoint{-0.019642in}{-0.019642in}}%
\pgfpathcurveto{\pgfqpoint{-0.014433in}{-0.024851in}}{\pgfqpoint{-0.007367in}{-0.027778in}}{\pgfqpoint{0.000000in}{-0.027778in}}%
\pgfpathclose%
\pgfusepath{stroke,fill}%
}%
\begin{pgfscope}%
\pgfsys@transformshift{3.066092in}{0.482165in}%
\pgfsys@useobject{currentmarker}{}%
\end{pgfscope}%
\end{pgfscope}%
\begin{pgfscope}%
\pgfpathrectangle{\pgfqpoint{0.550713in}{0.127635in}}{\pgfqpoint{3.194133in}{2.297424in}}%
\pgfusepath{clip}%
\pgfsetrectcap%
\pgfsetroundjoin%
\pgfsetlinewidth{0.752812pt}%
\definecolor{currentstroke}{rgb}{0.000000,0.000000,0.000000}%
\pgfsetstrokecolor{currentstroke}%
\pgfsetdash{}{0pt}%
\pgfpathmoveto{\pgfqpoint{3.147543in}{0.465931in}}%
\pgfpathlineto{\pgfqpoint{3.304055in}{0.465931in}}%
\pgfusepath{stroke}%
\end{pgfscope}%
\begin{pgfscope}%
\pgfpathrectangle{\pgfqpoint{0.550713in}{0.127635in}}{\pgfqpoint{3.194133in}{2.297424in}}%
\pgfusepath{clip}%
\pgfsetbuttcap%
\pgfsetroundjoin%
\definecolor{currentfill}{rgb}{1.000000,1.000000,1.000000}%
\pgfsetfillcolor{currentfill}%
\pgfsetlinewidth{1.003750pt}%
\definecolor{currentstroke}{rgb}{0.000000,0.000000,0.000000}%
\pgfsetstrokecolor{currentstroke}%
\pgfsetdash{}{0pt}%
\pgfsys@defobject{currentmarker}{\pgfqpoint{-0.027778in}{-0.027778in}}{\pgfqpoint{0.027778in}{0.027778in}}{%
\pgfpathmoveto{\pgfqpoint{0.000000in}{-0.027778in}}%
\pgfpathcurveto{\pgfqpoint{0.007367in}{-0.027778in}}{\pgfqpoint{0.014433in}{-0.024851in}}{\pgfqpoint{0.019642in}{-0.019642in}}%
\pgfpathcurveto{\pgfqpoint{0.024851in}{-0.014433in}}{\pgfqpoint{0.027778in}{-0.007367in}}{\pgfqpoint{0.027778in}{0.000000in}}%
\pgfpathcurveto{\pgfqpoint{0.027778in}{0.007367in}}{\pgfqpoint{0.024851in}{0.014433in}}{\pgfqpoint{0.019642in}{0.019642in}}%
\pgfpathcurveto{\pgfqpoint{0.014433in}{0.024851in}}{\pgfqpoint{0.007367in}{0.027778in}}{\pgfqpoint{0.000000in}{0.027778in}}%
\pgfpathcurveto{\pgfqpoint{-0.007367in}{0.027778in}}{\pgfqpoint{-0.014433in}{0.024851in}}{\pgfqpoint{-0.019642in}{0.019642in}}%
\pgfpathcurveto{\pgfqpoint{-0.024851in}{0.014433in}}{\pgfqpoint{-0.027778in}{0.007367in}}{\pgfqpoint{-0.027778in}{0.000000in}}%
\pgfpathcurveto{\pgfqpoint{-0.027778in}{-0.007367in}}{\pgfqpoint{-0.024851in}{-0.014433in}}{\pgfqpoint{-0.019642in}{-0.019642in}}%
\pgfpathcurveto{\pgfqpoint{-0.014433in}{-0.024851in}}{\pgfqpoint{-0.007367in}{-0.027778in}}{\pgfqpoint{0.000000in}{-0.027778in}}%
\pgfpathclose%
\pgfusepath{stroke,fill}%
}%
\begin{pgfscope}%
\pgfsys@transformshift{3.225799in}{0.501887in}%
\pgfsys@useobject{currentmarker}{}%
\end{pgfscope}%
\end{pgfscope}%
\begin{pgfscope}%
\pgfpathrectangle{\pgfqpoint{0.550713in}{0.127635in}}{\pgfqpoint{3.194133in}{2.297424in}}%
\pgfusepath{clip}%
\pgfsetrectcap%
\pgfsetroundjoin%
\pgfsetlinewidth{0.752812pt}%
\definecolor{currentstroke}{rgb}{0.000000,0.000000,0.000000}%
\pgfsetstrokecolor{currentstroke}%
\pgfsetdash{}{0pt}%
\pgfpathmoveto{\pgfqpoint{3.387103in}{0.666602in}}%
\pgfpathlineto{\pgfqpoint{3.543615in}{0.666602in}}%
\pgfusepath{stroke}%
\end{pgfscope}%
\begin{pgfscope}%
\pgfpathrectangle{\pgfqpoint{0.550713in}{0.127635in}}{\pgfqpoint{3.194133in}{2.297424in}}%
\pgfusepath{clip}%
\pgfsetbuttcap%
\pgfsetroundjoin%
\definecolor{currentfill}{rgb}{1.000000,1.000000,1.000000}%
\pgfsetfillcolor{currentfill}%
\pgfsetlinewidth{1.003750pt}%
\definecolor{currentstroke}{rgb}{0.000000,0.000000,0.000000}%
\pgfsetstrokecolor{currentstroke}%
\pgfsetdash{}{0pt}%
\pgfsys@defobject{currentmarker}{\pgfqpoint{-0.027778in}{-0.027778in}}{\pgfqpoint{0.027778in}{0.027778in}}{%
\pgfpathmoveto{\pgfqpoint{0.000000in}{-0.027778in}}%
\pgfpathcurveto{\pgfqpoint{0.007367in}{-0.027778in}}{\pgfqpoint{0.014433in}{-0.024851in}}{\pgfqpoint{0.019642in}{-0.019642in}}%
\pgfpathcurveto{\pgfqpoint{0.024851in}{-0.014433in}}{\pgfqpoint{0.027778in}{-0.007367in}}{\pgfqpoint{0.027778in}{0.000000in}}%
\pgfpathcurveto{\pgfqpoint{0.027778in}{0.007367in}}{\pgfqpoint{0.024851in}{0.014433in}}{\pgfqpoint{0.019642in}{0.019642in}}%
\pgfpathcurveto{\pgfqpoint{0.014433in}{0.024851in}}{\pgfqpoint{0.007367in}{0.027778in}}{\pgfqpoint{0.000000in}{0.027778in}}%
\pgfpathcurveto{\pgfqpoint{-0.007367in}{0.027778in}}{\pgfqpoint{-0.014433in}{0.024851in}}{\pgfqpoint{-0.019642in}{0.019642in}}%
\pgfpathcurveto{\pgfqpoint{-0.024851in}{0.014433in}}{\pgfqpoint{-0.027778in}{0.007367in}}{\pgfqpoint{-0.027778in}{0.000000in}}%
\pgfpathcurveto{\pgfqpoint{-0.027778in}{-0.007367in}}{\pgfqpoint{-0.024851in}{-0.014433in}}{\pgfqpoint{-0.019642in}{-0.019642in}}%
\pgfpathcurveto{\pgfqpoint{-0.014433in}{-0.024851in}}{\pgfqpoint{-0.007367in}{-0.027778in}}{\pgfqpoint{0.000000in}{-0.027778in}}%
\pgfpathclose%
\pgfusepath{stroke,fill}%
}%
\begin{pgfscope}%
\pgfsys@transformshift{3.465359in}{0.633255in}%
\pgfsys@useobject{currentmarker}{}%
\end{pgfscope}%
\end{pgfscope}%
\begin{pgfscope}%
\pgfpathrectangle{\pgfqpoint{0.550713in}{0.127635in}}{\pgfqpoint{3.194133in}{2.297424in}}%
\pgfusepath{clip}%
\pgfsetrectcap%
\pgfsetroundjoin%
\pgfsetlinewidth{0.752812pt}%
\definecolor{currentstroke}{rgb}{0.000000,0.000000,0.000000}%
\pgfsetstrokecolor{currentstroke}%
\pgfsetdash{}{0pt}%
\pgfpathmoveto{\pgfqpoint{3.546809in}{0.600761in}}%
\pgfpathlineto{\pgfqpoint{3.703322in}{0.600761in}}%
\pgfusepath{stroke}%
\end{pgfscope}%
\begin{pgfscope}%
\pgfpathrectangle{\pgfqpoint{0.550713in}{0.127635in}}{\pgfqpoint{3.194133in}{2.297424in}}%
\pgfusepath{clip}%
\pgfsetbuttcap%
\pgfsetroundjoin%
\definecolor{currentfill}{rgb}{1.000000,1.000000,1.000000}%
\pgfsetfillcolor{currentfill}%
\pgfsetlinewidth{1.003750pt}%
\definecolor{currentstroke}{rgb}{0.000000,0.000000,0.000000}%
\pgfsetstrokecolor{currentstroke}%
\pgfsetdash{}{0pt}%
\pgfsys@defobject{currentmarker}{\pgfqpoint{-0.027778in}{-0.027778in}}{\pgfqpoint{0.027778in}{0.027778in}}{%
\pgfpathmoveto{\pgfqpoint{0.000000in}{-0.027778in}}%
\pgfpathcurveto{\pgfqpoint{0.007367in}{-0.027778in}}{\pgfqpoint{0.014433in}{-0.024851in}}{\pgfqpoint{0.019642in}{-0.019642in}}%
\pgfpathcurveto{\pgfqpoint{0.024851in}{-0.014433in}}{\pgfqpoint{0.027778in}{-0.007367in}}{\pgfqpoint{0.027778in}{0.000000in}}%
\pgfpathcurveto{\pgfqpoint{0.027778in}{0.007367in}}{\pgfqpoint{0.024851in}{0.014433in}}{\pgfqpoint{0.019642in}{0.019642in}}%
\pgfpathcurveto{\pgfqpoint{0.014433in}{0.024851in}}{\pgfqpoint{0.007367in}{0.027778in}}{\pgfqpoint{0.000000in}{0.027778in}}%
\pgfpathcurveto{\pgfqpoint{-0.007367in}{0.027778in}}{\pgfqpoint{-0.014433in}{0.024851in}}{\pgfqpoint{-0.019642in}{0.019642in}}%
\pgfpathcurveto{\pgfqpoint{-0.024851in}{0.014433in}}{\pgfqpoint{-0.027778in}{0.007367in}}{\pgfqpoint{-0.027778in}{0.000000in}}%
\pgfpathcurveto{\pgfqpoint{-0.027778in}{-0.007367in}}{\pgfqpoint{-0.024851in}{-0.014433in}}{\pgfqpoint{-0.019642in}{-0.019642in}}%
\pgfpathcurveto{\pgfqpoint{-0.014433in}{-0.024851in}}{\pgfqpoint{-0.007367in}{-0.027778in}}{\pgfqpoint{0.000000in}{-0.027778in}}%
\pgfpathclose%
\pgfusepath{stroke,fill}%
}%
\begin{pgfscope}%
\pgfsys@transformshift{3.625066in}{0.649206in}%
\pgfsys@useobject{currentmarker}{}%
\end{pgfscope}%
\end{pgfscope}%
\begin{pgfscope}%
\pgfsetrectcap%
\pgfsetmiterjoin%
\pgfsetlinewidth{0.752812pt}%
\definecolor{currentstroke}{rgb}{0.000000,0.000000,0.000000}%
\pgfsetstrokecolor{currentstroke}%
\pgfsetdash{}{0pt}%
\pgfpathmoveto{\pgfqpoint{0.550713in}{0.127635in}}%
\pgfpathlineto{\pgfqpoint{0.550713in}{2.425059in}}%
\pgfusepath{stroke}%
\end{pgfscope}%
\begin{pgfscope}%
\pgfsetrectcap%
\pgfsetmiterjoin%
\pgfsetlinewidth{0.752812pt}%
\definecolor{currentstroke}{rgb}{0.000000,0.000000,0.000000}%
\pgfsetstrokecolor{currentstroke}%
\pgfsetdash{}{0pt}%
\pgfpathmoveto{\pgfqpoint{3.744846in}{0.127635in}}%
\pgfpathlineto{\pgfqpoint{3.744846in}{2.425059in}}%
\pgfusepath{stroke}%
\end{pgfscope}%
\begin{pgfscope}%
\pgfsetrectcap%
\pgfsetmiterjoin%
\pgfsetlinewidth{0.752812pt}%
\definecolor{currentstroke}{rgb}{0.000000,0.000000,0.000000}%
\pgfsetstrokecolor{currentstroke}%
\pgfsetdash{}{0pt}%
\pgfpathmoveto{\pgfqpoint{0.550713in}{0.127635in}}%
\pgfpathlineto{\pgfqpoint{3.744846in}{0.127635in}}%
\pgfusepath{stroke}%
\end{pgfscope}%
\begin{pgfscope}%
\pgfsetrectcap%
\pgfsetmiterjoin%
\pgfsetlinewidth{0.752812pt}%
\definecolor{currentstroke}{rgb}{0.000000,0.000000,0.000000}%
\pgfsetstrokecolor{currentstroke}%
\pgfsetdash{}{0pt}%
\pgfpathmoveto{\pgfqpoint{0.550713in}{2.425059in}}%
\pgfpathlineto{\pgfqpoint{3.744846in}{2.425059in}}%
\pgfusepath{stroke}%
\end{pgfscope}%
\begin{pgfscope}%
\pgfsetbuttcap%
\pgfsetroundjoin%
\pgfsetlinewidth{1.003750pt}%
\definecolor{currentstroke}{rgb}{0.392157,0.396078,0.403922}%
\pgfsetstrokecolor{currentstroke}%
\pgfsetdash{{3.700000pt}{1.600000pt}}{0.000000pt}%
\pgfpathmoveto{\pgfqpoint{3.869846in}{2.051207in}}%
\pgfpathlineto{\pgfqpoint{4.147623in}{2.051207in}}%
\pgfusepath{stroke}%
\end{pgfscope}%
\begin{pgfscope}%
\definecolor{textcolor}{rgb}{0.000000,0.000000,0.000000}%
\pgfsetstrokecolor{textcolor}%
\pgfsetfillcolor{textcolor}%
\pgftext[x=4.258735in, y=2.087374in, left, base]{\color{textcolor}\rmfamily\fontsize{10.000000}{12.000000}\selectfont Only Exploitation:}%
\end{pgfscope}%
\begin{pgfscope}%
\definecolor{textcolor}{rgb}{0.000000,0.000000,0.000000}%
\pgfsetstrokecolor{textcolor}%
\pgfsetfillcolor{textcolor}%
\pgftext[x=4.258735in, y=1.943235in, left, base]{\color{textcolor}\rmfamily\fontsize{10.000000}{12.000000}\selectfont \(\displaystyle R_T=536.55\)}%
\end{pgfscope}%
\begin{pgfscope}%
\pgfsetbuttcap%
\pgfsetmiterjoin%
\definecolor{currentfill}{rgb}{0.631373,0.062745,0.207843}%
\pgfsetfillcolor{currentfill}%
\pgfsetlinewidth{0.000000pt}%
\definecolor{currentstroke}{rgb}{0.000000,0.000000,0.000000}%
\pgfsetstrokecolor{currentstroke}%
\pgfsetstrokeopacity{0.000000}%
\pgfsetdash{}{0pt}%
\pgfpathmoveto{\pgfqpoint{3.869846in}{1.749624in}}%
\pgfpathlineto{\pgfqpoint{4.147623in}{1.749624in}}%
\pgfpathlineto{\pgfqpoint{4.147623in}{1.846846in}}%
\pgfpathlineto{\pgfqpoint{3.869846in}{1.846846in}}%
\pgfpathclose%
\pgfusepath{fill}%
\end{pgfscope}%
\begin{pgfscope}%
\definecolor{textcolor}{rgb}{0.000000,0.000000,0.000000}%
\pgfsetstrokecolor{textcolor}%
\pgfsetfillcolor{textcolor}%
\pgftext[x=4.258735in,y=1.749624in,left,base]{\color{textcolor}\rmfamily\fontsize{10.000000}{12.000000}\selectfont TV-GP-UCB}%
\end{pgfscope}%
\begin{pgfscope}%
\pgfsetbuttcap%
\pgfsetmiterjoin%
\definecolor{currentfill}{rgb}{0.890196,0.000000,0.400000}%
\pgfsetfillcolor{currentfill}%
\pgfsetlinewidth{0.000000pt}%
\definecolor{currentstroke}{rgb}{0.000000,0.000000,0.000000}%
\pgfsetstrokecolor{currentstroke}%
\pgfsetstrokeopacity{0.000000}%
\pgfsetdash{}{0pt}%
\pgfpathmoveto{\pgfqpoint{3.869846in}{1.556013in}}%
\pgfpathlineto{\pgfqpoint{4.147623in}{1.556013in}}%
\pgfpathlineto{\pgfqpoint{4.147623in}{1.653235in}}%
\pgfpathlineto{\pgfqpoint{3.869846in}{1.653235in}}%
\pgfpathclose%
\pgfusepath{fill}%
\end{pgfscope}%
\begin{pgfscope}%
\definecolor{textcolor}{rgb}{0.000000,0.000000,0.000000}%
\pgfsetstrokecolor{textcolor}%
\pgfsetfillcolor{textcolor}%
\pgftext[x=4.258735in,y=1.556013in,left,base]{\color{textcolor}\rmfamily\fontsize{10.000000}{12.000000}\selectfont SW TV-GP-UCB}%
\end{pgfscope}%
\begin{pgfscope}%
\pgfsetbuttcap%
\pgfsetmiterjoin%
\definecolor{currentfill}{rgb}{0.000000,0.329412,0.623529}%
\pgfsetfillcolor{currentfill}%
\pgfsetlinewidth{0.000000pt}%
\definecolor{currentstroke}{rgb}{0.000000,0.000000,0.000000}%
\pgfsetstrokecolor{currentstroke}%
\pgfsetstrokeopacity{0.000000}%
\pgfsetdash{}{0pt}%
\pgfpathmoveto{\pgfqpoint{3.869846in}{1.362402in}}%
\pgfpathlineto{\pgfqpoint{4.147623in}{1.362402in}}%
\pgfpathlineto{\pgfqpoint{4.147623in}{1.459624in}}%
\pgfpathlineto{\pgfqpoint{3.869846in}{1.459624in}}%
\pgfpathclose%
\pgfusepath{fill}%
\end{pgfscope}%
\begin{pgfscope}%
\definecolor{textcolor}{rgb}{0.000000,0.000000,0.000000}%
\pgfsetstrokecolor{textcolor}%
\pgfsetfillcolor{textcolor}%
\pgftext[x=4.258735in,y=1.362402in,left,base]{\color{textcolor}\rmfamily\fontsize{10.000000}{12.000000}\selectfont UI-TVBO}%
\end{pgfscope}%
\begin{pgfscope}%
\pgfsetbuttcap%
\pgfsetmiterjoin%
\definecolor{currentfill}{rgb}{0.000000,0.380392,0.396078}%
\pgfsetfillcolor{currentfill}%
\pgfsetlinewidth{0.000000pt}%
\definecolor{currentstroke}{rgb}{0.000000,0.000000,0.000000}%
\pgfsetstrokecolor{currentstroke}%
\pgfsetstrokeopacity{0.000000}%
\pgfsetdash{}{0pt}%
\pgfpathmoveto{\pgfqpoint{3.869846in}{1.168791in}}%
\pgfpathlineto{\pgfqpoint{4.147623in}{1.168791in}}%
\pgfpathlineto{\pgfqpoint{4.147623in}{1.266013in}}%
\pgfpathlineto{\pgfqpoint{3.869846in}{1.266013in}}%
\pgfpathclose%
\pgfusepath{fill}%
\end{pgfscope}%
\begin{pgfscope}%
\definecolor{textcolor}{rgb}{0.000000,0.000000,0.000000}%
\pgfsetstrokecolor{textcolor}%
\pgfsetfillcolor{textcolor}%
\pgftext[x=4.258735in,y=1.168791in,left,base]{\color{textcolor}\rmfamily\fontsize{10.000000}{12.000000}\selectfont B UI-TVBO}%
\end{pgfscope}%
\begin{pgfscope}%
\pgfsetbuttcap%
\pgfsetmiterjoin%
\definecolor{currentfill}{rgb}{0.380392,0.129412,0.345098}%
\pgfsetfillcolor{currentfill}%
\pgfsetlinewidth{0.000000pt}%
\definecolor{currentstroke}{rgb}{0.000000,0.000000,0.000000}%
\pgfsetstrokecolor{currentstroke}%
\pgfsetstrokeopacity{0.000000}%
\pgfsetdash{}{0pt}%
\pgfpathmoveto{\pgfqpoint{3.869846in}{0.975180in}}%
\pgfpathlineto{\pgfqpoint{4.147623in}{0.975180in}}%
\pgfpathlineto{\pgfqpoint{4.147623in}{1.072402in}}%
\pgfpathlineto{\pgfqpoint{3.869846in}{1.072402in}}%
\pgfpathclose%
\pgfusepath{fill}%
\end{pgfscope}%
\begin{pgfscope}%
\definecolor{textcolor}{rgb}{0.000000,0.000000,0.000000}%
\pgfsetstrokecolor{textcolor}%
\pgfsetfillcolor{textcolor}%
\pgftext[x=4.258735in,y=0.975180in,left,base]{\color{textcolor}\rmfamily\fontsize{10.000000}{12.000000}\selectfont C-TV-GP-UCB}%
\end{pgfscope}%
\begin{pgfscope}%
\pgfsetbuttcap%
\pgfsetmiterjoin%
\definecolor{currentfill}{rgb}{0.964706,0.658824,0.000000}%
\pgfsetfillcolor{currentfill}%
\pgfsetlinewidth{0.000000pt}%
\definecolor{currentstroke}{rgb}{0.000000,0.000000,0.000000}%
\pgfsetstrokecolor{currentstroke}%
\pgfsetstrokeopacity{0.000000}%
\pgfsetdash{}{0pt}%
\pgfpathmoveto{\pgfqpoint{3.869846in}{0.781569in}}%
\pgfpathlineto{\pgfqpoint{4.147623in}{0.781569in}}%
\pgfpathlineto{\pgfqpoint{4.147623in}{0.878791in}}%
\pgfpathlineto{\pgfqpoint{3.869846in}{0.878791in}}%
\pgfpathclose%
\pgfusepath{fill}%
\end{pgfscope}%
\begin{pgfscope}%
\definecolor{textcolor}{rgb}{0.000000,0.000000,0.000000}%
\pgfsetstrokecolor{textcolor}%
\pgfsetfillcolor{textcolor}%
\pgftext[x=4.258735in,y=0.781569in,left,base]{\color{textcolor}\rmfamily\fontsize{10.000000}{12.000000}\selectfont SW C-TV-GP-UCB}%
\end{pgfscope}%
\begin{pgfscope}%
\pgfsetbuttcap%
\pgfsetmiterjoin%
\definecolor{currentfill}{rgb}{0.341176,0.670588,0.152941}%
\pgfsetfillcolor{currentfill}%
\pgfsetlinewidth{0.000000pt}%
\definecolor{currentstroke}{rgb}{0.000000,0.000000,0.000000}%
\pgfsetstrokecolor{currentstroke}%
\pgfsetstrokeopacity{0.000000}%
\pgfsetdash{}{0pt}%
\pgfpathmoveto{\pgfqpoint{3.869846in}{0.587958in}}%
\pgfpathlineto{\pgfqpoint{4.147623in}{0.587958in}}%
\pgfpathlineto{\pgfqpoint{4.147623in}{0.685180in}}%
\pgfpathlineto{\pgfqpoint{3.869846in}{0.685180in}}%
\pgfpathclose%
\pgfusepath{fill}%
\end{pgfscope}%
\begin{pgfscope}%
\definecolor{textcolor}{rgb}{0.000000,0.000000,0.000000}%
\pgfsetstrokecolor{textcolor}%
\pgfsetfillcolor{textcolor}%
\pgftext[x=4.258735in,y=0.587958in,left,base]{\color{textcolor}\rmfamily\fontsize{10.000000}{12.000000}\selectfont C-UI-TVBO}%
\end{pgfscope}%
\begin{pgfscope}%
\pgfsetbuttcap%
\pgfsetmiterjoin%
\definecolor{currentfill}{rgb}{0.478431,0.435294,0.674510}%
\pgfsetfillcolor{currentfill}%
\pgfsetlinewidth{0.000000pt}%
\definecolor{currentstroke}{rgb}{0.000000,0.000000,0.000000}%
\pgfsetstrokecolor{currentstroke}%
\pgfsetstrokeopacity{0.000000}%
\pgfsetdash{}{0pt}%
\pgfpathmoveto{\pgfqpoint{3.869846in}{0.394347in}}%
\pgfpathlineto{\pgfqpoint{4.147623in}{0.394347in}}%
\pgfpathlineto{\pgfqpoint{4.147623in}{0.491569in}}%
\pgfpathlineto{\pgfqpoint{3.869846in}{0.491569in}}%
\pgfpathclose%
\pgfusepath{fill}%
\end{pgfscope}%
\begin{pgfscope}%
\definecolor{textcolor}{rgb}{0.000000,0.000000,0.000000}%
\pgfsetstrokecolor{textcolor}%
\pgfsetfillcolor{textcolor}%
\pgftext[x=4.258735in,y=0.394347in,left,base]{\color{textcolor}\rmfamily\fontsize{10.000000}{12.000000}\selectfont B C-UI-TVBO}%
\end{pgfscope}%
\begin{pgfscope}%
\pgfsetbuttcap%
\pgfsetmiterjoin%
\definecolor{currentfill}{rgb}{1.000000,1.000000,1.000000}%
\pgfsetfillcolor{currentfill}%
\pgfsetlinewidth{1.003750pt}%
\definecolor{currentstroke}{rgb}{1.000000,1.000000,1.000000}%
\pgfsetstrokecolor{currentstroke}%
\pgfsetdash{}{0pt}%
\pgfpathmoveto{\pgfqpoint{2.736198in}{1.968269in}}%
\pgfpathlineto{\pgfqpoint{3.689290in}{1.968269in}}%
\pgfpathquadraticcurveto{\pgfqpoint{3.717068in}{1.968269in}}{\pgfqpoint{3.717068in}{1.996046in}}%
\pgfpathlineto{\pgfqpoint{3.717068in}{2.369503in}}%
\pgfpathquadraticcurveto{\pgfqpoint{3.717068in}{2.397281in}}{\pgfqpoint{3.689290in}{2.397281in}}%
\pgfpathlineto{\pgfqpoint{2.736198in}{2.397281in}}%
\pgfpathquadraticcurveto{\pgfqpoint{2.708420in}{2.397281in}}{\pgfqpoint{2.708420in}{2.369503in}}%
\pgfpathlineto{\pgfqpoint{2.708420in}{1.996046in}}%
\pgfpathquadraticcurveto{\pgfqpoint{2.708420in}{1.968269in}}{\pgfqpoint{2.736198in}{1.968269in}}%
\pgfpathclose%
\pgfusepath{stroke,fill}%
\end{pgfscope}%
\begin{pgfscope}%
\pgfsetbuttcap%
\pgfsetmiterjoin%
\definecolor{currentfill}{rgb}{0.000000,0.000000,0.000000}%
\pgfsetfillcolor{currentfill}%
\pgfsetlinewidth{0.000000pt}%
\definecolor{currentstroke}{rgb}{0.000000,0.000000,0.000000}%
\pgfsetstrokecolor{currentstroke}%
\pgfsetstrokeopacity{0.000000}%
\pgfsetdash{}{0pt}%
\pgfpathmoveto{\pgfqpoint{2.763976in}{2.244503in}}%
\pgfpathlineto{\pgfqpoint{3.041753in}{2.244503in}}%
\pgfpathlineto{\pgfqpoint{3.041753in}{2.341725in}}%
\pgfpathlineto{\pgfqpoint{2.763976in}{2.341725in}}%
\pgfpathclose%
\pgfusepath{fill}%
\end{pgfscope}%
\begin{pgfscope}%
\definecolor{textcolor}{rgb}{0.000000,0.000000,0.000000}%
\pgfsetstrokecolor{textcolor}%
\pgfsetfillcolor{textcolor}%
\pgftext[x=3.152864in,y=2.244503in,left,base]{\color{textcolor}\rmfamily\fontsize{10.000000}{12.000000}\selectfont \(\displaystyle \mu_0=0\)}%
\end{pgfscope}%
\begin{pgfscope}%
\pgfsetbuttcap%
\pgfsetmiterjoin%
\definecolor{currentfill}{rgb}{0.811765,0.819608,0.823529}%
\pgfsetfillcolor{currentfill}%
\pgfsetlinewidth{0.000000pt}%
\definecolor{currentstroke}{rgb}{0.000000,0.000000,0.000000}%
\pgfsetstrokecolor{currentstroke}%
\pgfsetstrokeopacity{0.000000}%
\pgfsetdash{}{0pt}%
\pgfpathmoveto{\pgfqpoint{2.763976in}{2.050830in}}%
\pgfpathlineto{\pgfqpoint{3.041753in}{2.050830in}}%
\pgfpathlineto{\pgfqpoint{3.041753in}{2.148053in}}%
\pgfpathlineto{\pgfqpoint{2.763976in}{2.148053in}}%
\pgfpathclose%
\pgfusepath{fill}%
\end{pgfscope}%
\begin{pgfscope}%
\definecolor{textcolor}{rgb}{0.000000,0.000000,0.000000}%
\pgfsetstrokecolor{textcolor}%
\pgfsetfillcolor{textcolor}%
\pgftext[x=3.152864in,y=2.050830in,left,base]{\color{textcolor}\rmfamily\fontsize{10.000000}{12.000000}\selectfont \(\displaystyle \mu_0=-1\)}%
\end{pgfscope}%
\begin{pgfscope}%
\pgfsetbuttcap%
\pgfsetmiterjoin%
\definecolor{currentfill}{rgb}{1.000000,1.000000,1.000000}%
\pgfsetfillcolor{currentfill}%
\pgfsetlinewidth{1.003750pt}%
\definecolor{currentstroke}{rgb}{1.000000,1.000000,1.000000}%
\pgfsetstrokecolor{currentstroke}%
\pgfsetdash{}{0pt}%
\pgfpathmoveto{\pgfqpoint{2.736198in}{1.968269in}}%
\pgfpathlineto{\pgfqpoint{3.689290in}{1.968269in}}%
\pgfpathquadraticcurveto{\pgfqpoint{3.717068in}{1.968269in}}{\pgfqpoint{3.717068in}{1.996046in}}%
\pgfpathlineto{\pgfqpoint{3.717068in}{2.369503in}}%
\pgfpathquadraticcurveto{\pgfqpoint{3.717068in}{2.397281in}}{\pgfqpoint{3.689290in}{2.397281in}}%
\pgfpathlineto{\pgfqpoint{2.736198in}{2.397281in}}%
\pgfpathquadraticcurveto{\pgfqpoint{2.708420in}{2.397281in}}{\pgfqpoint{2.708420in}{2.369503in}}%
\pgfpathlineto{\pgfqpoint{2.708420in}{1.996046in}}%
\pgfpathquadraticcurveto{\pgfqpoint{2.708420in}{1.968269in}}{\pgfqpoint{2.736198in}{1.968269in}}%
\pgfpathclose%
\pgfusepath{stroke,fill}%
\end{pgfscope}%
\begin{pgfscope}%
\pgfsetbuttcap%
\pgfsetmiterjoin%
\definecolor{currentfill}{rgb}{0.000000,0.000000,0.000000}%
\pgfsetfillcolor{currentfill}%
\pgfsetlinewidth{0.000000pt}%
\definecolor{currentstroke}{rgb}{0.000000,0.000000,0.000000}%
\pgfsetstrokecolor{currentstroke}%
\pgfsetstrokeopacity{0.000000}%
\pgfsetdash{}{0pt}%
\pgfpathmoveto{\pgfqpoint{2.763976in}{2.244503in}}%
\pgfpathlineto{\pgfqpoint{3.041753in}{2.244503in}}%
\pgfpathlineto{\pgfqpoint{3.041753in}{2.341725in}}%
\pgfpathlineto{\pgfqpoint{2.763976in}{2.341725in}}%
\pgfpathclose%
\pgfusepath{fill}%
\end{pgfscope}%
\begin{pgfscope}%
\definecolor{textcolor}{rgb}{0.000000,0.000000,0.000000}%
\pgfsetstrokecolor{textcolor}%
\pgfsetfillcolor{textcolor}%
\pgftext[x=3.152864in,y=2.244503in,left,base]{\color{textcolor}\rmfamily\fontsize{10.000000}{12.000000}\selectfont \(\displaystyle \mu_0=0\)}%
\end{pgfscope}%
\begin{pgfscope}%
\pgfsetbuttcap%
\pgfsetmiterjoin%
\definecolor{currentfill}{rgb}{0.811765,0.819608,0.823529}%
\pgfsetfillcolor{currentfill}%
\pgfsetlinewidth{0.000000pt}%
\definecolor{currentstroke}{rgb}{0.000000,0.000000,0.000000}%
\pgfsetstrokecolor{currentstroke}%
\pgfsetstrokeopacity{0.000000}%
\pgfsetdash{}{0pt}%
\pgfpathmoveto{\pgfqpoint{2.763976in}{2.050830in}}%
\pgfpathlineto{\pgfqpoint{3.041753in}{2.050830in}}%
\pgfpathlineto{\pgfqpoint{3.041753in}{2.148053in}}%
\pgfpathlineto{\pgfqpoint{2.763976in}{2.148053in}}%
\pgfpathclose%
\pgfusepath{fill}%
\end{pgfscope}%
\begin{pgfscope}%
\definecolor{textcolor}{rgb}{0.000000,0.000000,0.000000}%
\pgfsetstrokecolor{textcolor}%
\pgfsetfillcolor{textcolor}%
\pgftext[x=3.152864in,y=2.050830in,left,base]{\color{textcolor}\rmfamily\fontsize{10.000000}{12.000000}\selectfont \(\displaystyle \mu_0=-1\)}%
\end{pgfscope}%
\begin{pgfscope}%
\pgfsetbuttcap%
\pgfsetmiterjoin%
\definecolor{currentfill}{rgb}{1.000000,1.000000,1.000000}%
\pgfsetfillcolor{currentfill}%
\pgfsetlinewidth{1.003750pt}%
\definecolor{currentstroke}{rgb}{1.000000,1.000000,1.000000}%
\pgfsetstrokecolor{currentstroke}%
\pgfsetdash{}{0pt}%
\pgfpathmoveto{\pgfqpoint{2.736198in}{1.968269in}}%
\pgfpathlineto{\pgfqpoint{3.689290in}{1.968269in}}%
\pgfpathquadraticcurveto{\pgfqpoint{3.717068in}{1.968269in}}{\pgfqpoint{3.717068in}{1.996046in}}%
\pgfpathlineto{\pgfqpoint{3.717068in}{2.369503in}}%
\pgfpathquadraticcurveto{\pgfqpoint{3.717068in}{2.397281in}}{\pgfqpoint{3.689290in}{2.397281in}}%
\pgfpathlineto{\pgfqpoint{2.736198in}{2.397281in}}%
\pgfpathquadraticcurveto{\pgfqpoint{2.708420in}{2.397281in}}{\pgfqpoint{2.708420in}{2.369503in}}%
\pgfpathlineto{\pgfqpoint{2.708420in}{1.996046in}}%
\pgfpathquadraticcurveto{\pgfqpoint{2.708420in}{1.968269in}}{\pgfqpoint{2.736198in}{1.968269in}}%
\pgfpathclose%
\pgfusepath{stroke,fill}%
\end{pgfscope}%
\begin{pgfscope}%
\pgfsetbuttcap%
\pgfsetmiterjoin%
\definecolor{currentfill}{rgb}{0.000000,0.000000,0.000000}%
\pgfsetfillcolor{currentfill}%
\pgfsetlinewidth{0.000000pt}%
\definecolor{currentstroke}{rgb}{0.000000,0.000000,0.000000}%
\pgfsetstrokecolor{currentstroke}%
\pgfsetstrokeopacity{0.000000}%
\pgfsetdash{}{0pt}%
\pgfpathmoveto{\pgfqpoint{2.763976in}{2.244503in}}%
\pgfpathlineto{\pgfqpoint{3.041753in}{2.244503in}}%
\pgfpathlineto{\pgfqpoint{3.041753in}{2.341725in}}%
\pgfpathlineto{\pgfqpoint{2.763976in}{2.341725in}}%
\pgfpathclose%
\pgfusepath{fill}%
\end{pgfscope}%
\begin{pgfscope}%
\definecolor{textcolor}{rgb}{0.000000,0.000000,0.000000}%
\pgfsetstrokecolor{textcolor}%
\pgfsetfillcolor{textcolor}%
\pgftext[x=3.152864in,y=2.244503in,left,base]{\color{textcolor}\rmfamily\fontsize{10.000000}{12.000000}\selectfont \(\displaystyle \mu_0=0\)}%
\end{pgfscope}%
\begin{pgfscope}%
\pgfsetbuttcap%
\pgfsetmiterjoin%
\definecolor{currentfill}{rgb}{0.811765,0.819608,0.823529}%
\pgfsetfillcolor{currentfill}%
\pgfsetlinewidth{0.000000pt}%
\definecolor{currentstroke}{rgb}{0.000000,0.000000,0.000000}%
\pgfsetstrokecolor{currentstroke}%
\pgfsetstrokeopacity{0.000000}%
\pgfsetdash{}{0pt}%
\pgfpathmoveto{\pgfqpoint{2.763976in}{2.050830in}}%
\pgfpathlineto{\pgfqpoint{3.041753in}{2.050830in}}%
\pgfpathlineto{\pgfqpoint{3.041753in}{2.148053in}}%
\pgfpathlineto{\pgfqpoint{2.763976in}{2.148053in}}%
\pgfpathclose%
\pgfusepath{fill}%
\end{pgfscope}%
\begin{pgfscope}%
\definecolor{textcolor}{rgb}{0.000000,0.000000,0.000000}%
\pgfsetstrokecolor{textcolor}%
\pgfsetfillcolor{textcolor}%
\pgftext[x=3.152864in,y=2.050830in,left,base]{\color{textcolor}\rmfamily\fontsize{10.000000}{12.000000}\selectfont \(\displaystyle \mu_0=-1\)}%
\end{pgfscope}%
\begin{pgfscope}%
\pgfpathrectangle{\pgfqpoint{0.550713in}{0.127635in}}{\pgfqpoint{3.194133in}{2.297424in}}%
\pgfusepath{clip}%
\pgfsetbuttcap%
\pgfsetmiterjoin%
\definecolor{currentfill}{rgb}{1.000000,1.000000,1.000000}%
\pgfsetfillcolor{currentfill}%
\pgfsetlinewidth{0.000000pt}%
\definecolor{currentstroke}{rgb}{0.000000,0.000000,0.000000}%
\pgfsetstrokecolor{currentstroke}%
\pgfsetstrokeopacity{0.000000}%
\pgfsetdash{}{0pt}%
\pgfpathmoveto{\pgfqpoint{2.227632in}{1.768652in}}%
\pgfpathlineto{\pgfqpoint{3.704919in}{1.768652in}}%
\pgfpathlineto{\pgfqpoint{3.704919in}{1.965574in}}%
\pgfpathlineto{\pgfqpoint{2.227632in}{1.965574in}}%
\pgfpathclose%
\pgfusepath{fill}%
\end{pgfscope}%
\begin{pgfscope}%
\pgfsetbuttcap%
\pgfsetmiterjoin%
\definecolor{currentfill}{rgb}{1.000000,1.000000,1.000000}%
\pgfsetfillcolor{currentfill}%
\pgfsetlinewidth{0.000000pt}%
\definecolor{currentstroke}{rgb}{0.000000,0.000000,0.000000}%
\pgfsetstrokecolor{currentstroke}%
\pgfsetstrokeopacity{0.000000}%
\pgfsetdash{}{0pt}%
\pgfpathmoveto{\pgfqpoint{2.243603in}{0.862810in}}%
\pgfpathlineto{\pgfqpoint{3.649022in}{0.862810in}}%
\pgfpathlineto{\pgfqpoint{3.649022in}{1.781780in}}%
\pgfpathlineto{\pgfqpoint{2.243603in}{1.781780in}}%
\pgfpathclose%
\pgfusepath{fill}%
\end{pgfscope}%
\begin{pgfscope}%
\pgfpathrectangle{\pgfqpoint{2.243603in}{0.862810in}}{\pgfqpoint{1.405419in}{0.918970in}}%
\pgfusepath{clip}%
\pgfsetbuttcap%
\pgfsetmiterjoin%
\definecolor{currentfill}{rgb}{0.380392,0.129412,0.345098}%
\pgfsetfillcolor{currentfill}%
\pgfsetlinewidth{0.752812pt}%
\definecolor{currentstroke}{rgb}{0.000000,0.000000,0.000000}%
\pgfsetstrokecolor{currentstroke}%
\pgfsetdash{}{0pt}%
\pgfpathmoveto{\pgfqpoint{2.280144in}{1.172089in}}%
\pgfpathlineto{\pgfqpoint{2.417875in}{1.172089in}}%
\pgfpathlineto{\pgfqpoint{2.417875in}{1.318840in}}%
\pgfpathlineto{\pgfqpoint{2.280144in}{1.318840in}}%
\pgfpathlineto{\pgfqpoint{2.280144in}{1.172089in}}%
\pgfpathclose%
\pgfusepath{stroke,fill}%
\end{pgfscope}%
\begin{pgfscope}%
\pgfpathrectangle{\pgfqpoint{2.243603in}{0.862810in}}{\pgfqpoint{1.405419in}{0.918970in}}%
\pgfusepath{clip}%
\pgfsetbuttcap%
\pgfsetmiterjoin%
\definecolor{currentfill}{rgb}{0.823529,0.752941,0.803922}%
\pgfsetfillcolor{currentfill}%
\pgfsetlinewidth{0.752812pt}%
\definecolor{currentstroke}{rgb}{0.000000,0.000000,0.000000}%
\pgfsetstrokecolor{currentstroke}%
\pgfsetdash{}{0pt}%
\pgfpathmoveto{\pgfqpoint{2.420686in}{1.271447in}}%
\pgfpathlineto{\pgfqpoint{2.558417in}{1.271447in}}%
\pgfpathlineto{\pgfqpoint{2.558417in}{1.460991in}}%
\pgfpathlineto{\pgfqpoint{2.420686in}{1.460991in}}%
\pgfpathlineto{\pgfqpoint{2.420686in}{1.271447in}}%
\pgfpathclose%
\pgfusepath{stroke,fill}%
\end{pgfscope}%
\begin{pgfscope}%
\pgfpathrectangle{\pgfqpoint{2.243603in}{0.862810in}}{\pgfqpoint{1.405419in}{0.918970in}}%
\pgfusepath{clip}%
\pgfsetbuttcap%
\pgfsetmiterjoin%
\definecolor{currentfill}{rgb}{0.964706,0.658824,0.000000}%
\pgfsetfillcolor{currentfill}%
\pgfsetlinewidth{0.752812pt}%
\definecolor{currentstroke}{rgb}{0.000000,0.000000,0.000000}%
\pgfsetstrokecolor{currentstroke}%
\pgfsetdash{}{0pt}%
\pgfpathmoveto{\pgfqpoint{2.631499in}{1.526244in}}%
\pgfpathlineto{\pgfqpoint{2.769230in}{1.526244in}}%
\pgfpathlineto{\pgfqpoint{2.769230in}{1.598749in}}%
\pgfpathlineto{\pgfqpoint{2.631499in}{1.598749in}}%
\pgfpathlineto{\pgfqpoint{2.631499in}{1.526244in}}%
\pgfpathclose%
\pgfusepath{stroke,fill}%
\end{pgfscope}%
\begin{pgfscope}%
\pgfpathrectangle{\pgfqpoint{2.243603in}{0.862810in}}{\pgfqpoint{1.405419in}{0.918970in}}%
\pgfusepath{clip}%
\pgfsetbuttcap%
\pgfsetmiterjoin%
\definecolor{currentfill}{rgb}{0.996078,0.917647,0.788235}%
\pgfsetfillcolor{currentfill}%
\pgfsetlinewidth{0.752812pt}%
\definecolor{currentstroke}{rgb}{0.000000,0.000000,0.000000}%
\pgfsetstrokecolor{currentstroke}%
\pgfsetdash{}{0pt}%
\pgfpathmoveto{\pgfqpoint{2.772041in}{1.473760in}}%
\pgfpathlineto{\pgfqpoint{2.909772in}{1.473760in}}%
\pgfpathlineto{\pgfqpoint{2.909772in}{1.764028in}}%
\pgfpathlineto{\pgfqpoint{2.772041in}{1.764028in}}%
\pgfpathlineto{\pgfqpoint{2.772041in}{1.473760in}}%
\pgfpathclose%
\pgfusepath{stroke,fill}%
\end{pgfscope}%
\begin{pgfscope}%
\pgfpathrectangle{\pgfqpoint{2.243603in}{0.862810in}}{\pgfqpoint{1.405419in}{0.918970in}}%
\pgfusepath{clip}%
\pgfsetbuttcap%
\pgfsetmiterjoin%
\definecolor{currentfill}{rgb}{0.341176,0.670588,0.152941}%
\pgfsetfillcolor{currentfill}%
\pgfsetlinewidth{0.752812pt}%
\definecolor{currentstroke}{rgb}{0.000000,0.000000,0.000000}%
\pgfsetstrokecolor{currentstroke}%
\pgfsetdash{}{0pt}%
\pgfpathmoveto{\pgfqpoint{2.982853in}{0.967363in}}%
\pgfpathlineto{\pgfqpoint{3.120584in}{0.967363in}}%
\pgfpathlineto{\pgfqpoint{3.120584in}{1.355326in}}%
\pgfpathlineto{\pgfqpoint{2.982853in}{1.355326in}}%
\pgfpathlineto{\pgfqpoint{2.982853in}{0.967363in}}%
\pgfpathclose%
\pgfusepath{stroke,fill}%
\end{pgfscope}%
\begin{pgfscope}%
\pgfpathrectangle{\pgfqpoint{2.243603in}{0.862810in}}{\pgfqpoint{1.405419in}{0.918970in}}%
\pgfusepath{clip}%
\pgfsetbuttcap%
\pgfsetmiterjoin%
\definecolor{currentfill}{rgb}{0.866667,0.921569,0.807843}%
\pgfsetfillcolor{currentfill}%
\pgfsetlinewidth{0.752812pt}%
\definecolor{currentstroke}{rgb}{0.000000,0.000000,0.000000}%
\pgfsetstrokecolor{currentstroke}%
\pgfsetdash{}{0pt}%
\pgfpathmoveto{\pgfqpoint{3.123395in}{1.000192in}}%
\pgfpathlineto{\pgfqpoint{3.261126in}{1.000192in}}%
\pgfpathlineto{\pgfqpoint{3.261126in}{1.202038in}}%
\pgfpathlineto{\pgfqpoint{3.123395in}{1.202038in}}%
\pgfpathlineto{\pgfqpoint{3.123395in}{1.000192in}}%
\pgfpathclose%
\pgfusepath{stroke,fill}%
\end{pgfscope}%
\begin{pgfscope}%
\pgfpathrectangle{\pgfqpoint{2.243603in}{0.862810in}}{\pgfqpoint{1.405419in}{0.918970in}}%
\pgfusepath{clip}%
\pgfsetbuttcap%
\pgfsetmiterjoin%
\definecolor{currentfill}{rgb}{0.478431,0.435294,0.674510}%
\pgfsetfillcolor{currentfill}%
\pgfsetlinewidth{0.752812pt}%
\definecolor{currentstroke}{rgb}{0.000000,0.000000,0.000000}%
\pgfsetstrokecolor{currentstroke}%
\pgfsetdash{}{0pt}%
\pgfpathmoveto{\pgfqpoint{3.334208in}{1.516878in}}%
\pgfpathlineto{\pgfqpoint{3.471939in}{1.516878in}}%
\pgfpathlineto{\pgfqpoint{3.471939in}{1.596485in}}%
\pgfpathlineto{\pgfqpoint{3.334208in}{1.596485in}}%
\pgfpathlineto{\pgfqpoint{3.334208in}{1.516878in}}%
\pgfpathclose%
\pgfusepath{stroke,fill}%
\end{pgfscope}%
\begin{pgfscope}%
\pgfpathrectangle{\pgfqpoint{2.243603in}{0.862810in}}{\pgfqpoint{1.405419in}{0.918970in}}%
\pgfusepath{clip}%
\pgfsetbuttcap%
\pgfsetmiterjoin%
\definecolor{currentfill}{rgb}{0.870588,0.854902,0.921569}%
\pgfsetfillcolor{currentfill}%
\pgfsetlinewidth{0.752812pt}%
\definecolor{currentstroke}{rgb}{0.000000,0.000000,0.000000}%
\pgfsetstrokecolor{currentstroke}%
\pgfsetdash{}{0pt}%
\pgfpathmoveto{\pgfqpoint{3.474750in}{1.430330in}}%
\pgfpathlineto{\pgfqpoint{3.612481in}{1.430330in}}%
\pgfpathlineto{\pgfqpoint{3.612481in}{1.669468in}}%
\pgfpathlineto{\pgfqpoint{3.474750in}{1.669468in}}%
\pgfpathlineto{\pgfqpoint{3.474750in}{1.430330in}}%
\pgfpathclose%
\pgfusepath{stroke,fill}%
\end{pgfscope}%
\begin{pgfscope}%
\pgfpathrectangle{\pgfqpoint{2.243603in}{0.862810in}}{\pgfqpoint{1.405419in}{0.918970in}}%
\pgfusepath{clip}%
\pgfsetbuttcap%
\pgfsetmiterjoin%
\definecolor{currentfill}{rgb}{0.000000,0.000000,0.000000}%
\pgfsetfillcolor{currentfill}%
\pgfsetlinewidth{0.376406pt}%
\definecolor{currentstroke}{rgb}{0.000000,0.000000,0.000000}%
\pgfsetstrokecolor{currentstroke}%
\pgfsetdash{}{0pt}%
\pgfpathmoveto{\pgfqpoint{2.419280in}{-2312.153767in}}%
\pgfpathlineto{\pgfqpoint{2.419280in}{-2312.153767in}}%
\pgfpathlineto{\pgfqpoint{2.419280in}{-2312.153767in}}%
\pgfpathlineto{\pgfqpoint{2.419280in}{-2312.153767in}}%
\pgfpathclose%
\pgfusepath{stroke,fill}%
\end{pgfscope}%
\begin{pgfscope}%
\pgfpathrectangle{\pgfqpoint{2.243603in}{0.862810in}}{\pgfqpoint{1.405419in}{0.918970in}}%
\pgfusepath{clip}%
\pgfsetbuttcap%
\pgfsetmiterjoin%
\definecolor{currentfill}{rgb}{0.813235,0.819118,0.822059}%
\pgfsetfillcolor{currentfill}%
\pgfsetlinewidth{0.376406pt}%
\definecolor{currentstroke}{rgb}{0.000000,0.000000,0.000000}%
\pgfsetstrokecolor{currentstroke}%
\pgfsetdash{}{0pt}%
\pgfpathmoveto{\pgfqpoint{2.419280in}{-2312.153767in}}%
\pgfpathlineto{\pgfqpoint{2.419280in}{-2312.153767in}}%
\pgfpathlineto{\pgfqpoint{2.419280in}{-2312.153767in}}%
\pgfpathlineto{\pgfqpoint{2.419280in}{-2312.153767in}}%
\pgfpathclose%
\pgfusepath{stroke,fill}%
\end{pgfscope}%
\begin{pgfscope}%
\pgfsetbuttcap%
\pgfsetroundjoin%
\definecolor{currentfill}{rgb}{0.000000,0.000000,0.000000}%
\pgfsetfillcolor{currentfill}%
\pgfsetlinewidth{0.803000pt}%
\definecolor{currentstroke}{rgb}{0.000000,0.000000,0.000000}%
\pgfsetstrokecolor{currentstroke}%
\pgfsetdash{}{0pt}%
\pgfsys@defobject{currentmarker}{\pgfqpoint{-0.048611in}{0.000000in}}{\pgfqpoint{-0.000000in}{0.000000in}}{%
\pgfpathmoveto{\pgfqpoint{-0.000000in}{0.000000in}}%
\pgfpathlineto{\pgfqpoint{-0.048611in}{0.000000in}}%
\pgfusepath{stroke,fill}%
}%
\begin{pgfscope}%
\pgfsys@transformshift{2.243603in}{1.781780in}%
\pgfsys@useobject{currentmarker}{}%
\end{pgfscope}%
\end{pgfscope}%
\begin{pgfscope}%
\pgfsetbuttcap%
\pgfsetroundjoin%
\definecolor{currentfill}{rgb}{0.000000,0.000000,0.000000}%
\pgfsetfillcolor{currentfill}%
\pgfsetlinewidth{0.602250pt}%
\definecolor{currentstroke}{rgb}{0.000000,0.000000,0.000000}%
\pgfsetstrokecolor{currentstroke}%
\pgfsetdash{}{0pt}%
\pgfsys@defobject{currentmarker}{\pgfqpoint{-0.027778in}{0.000000in}}{\pgfqpoint{-0.000000in}{0.000000in}}{%
\pgfpathmoveto{\pgfqpoint{-0.000000in}{0.000000in}}%
\pgfpathlineto{\pgfqpoint{-0.027778in}{0.000000in}}%
\pgfusepath{stroke,fill}%
}%
\begin{pgfscope}%
\pgfsys@transformshift{2.243603in}{0.862810in}%
\pgfsys@useobject{currentmarker}{}%
\end{pgfscope}%
\end{pgfscope}%
\begin{pgfscope}%
\pgfsetbuttcap%
\pgfsetroundjoin%
\definecolor{currentfill}{rgb}{0.000000,0.000000,0.000000}%
\pgfsetfillcolor{currentfill}%
\pgfsetlinewidth{0.602250pt}%
\definecolor{currentstroke}{rgb}{0.000000,0.000000,0.000000}%
\pgfsetstrokecolor{currentstroke}%
\pgfsetdash{}{0pt}%
\pgfsys@defobject{currentmarker}{\pgfqpoint{-0.027778in}{0.000000in}}{\pgfqpoint{-0.000000in}{0.000000in}}{%
\pgfpathmoveto{\pgfqpoint{-0.000000in}{0.000000in}}%
\pgfpathlineto{\pgfqpoint{-0.027778in}{0.000000in}}%
\pgfusepath{stroke,fill}%
}%
\begin{pgfscope}%
\pgfsys@transformshift{2.243603in}{1.086606in}%
\pgfsys@useobject{currentmarker}{}%
\end{pgfscope}%
\end{pgfscope}%
\begin{pgfscope}%
\pgfsetbuttcap%
\pgfsetroundjoin%
\definecolor{currentfill}{rgb}{0.000000,0.000000,0.000000}%
\pgfsetfillcolor{currentfill}%
\pgfsetlinewidth{0.602250pt}%
\definecolor{currentstroke}{rgb}{0.000000,0.000000,0.000000}%
\pgfsetstrokecolor{currentstroke}%
\pgfsetdash{}{0pt}%
\pgfsys@defobject{currentmarker}{\pgfqpoint{-0.027778in}{0.000000in}}{\pgfqpoint{-0.000000in}{0.000000in}}{%
\pgfpathmoveto{\pgfqpoint{-0.000000in}{0.000000in}}%
\pgfpathlineto{\pgfqpoint{-0.027778in}{0.000000in}}%
\pgfusepath{stroke,fill}%
}%
\begin{pgfscope}%
\pgfsys@transformshift{2.243603in}{1.269461in}%
\pgfsys@useobject{currentmarker}{}%
\end{pgfscope}%
\end{pgfscope}%
\begin{pgfscope}%
\pgfsetbuttcap%
\pgfsetroundjoin%
\definecolor{currentfill}{rgb}{0.000000,0.000000,0.000000}%
\pgfsetfillcolor{currentfill}%
\pgfsetlinewidth{0.602250pt}%
\definecolor{currentstroke}{rgb}{0.000000,0.000000,0.000000}%
\pgfsetstrokecolor{currentstroke}%
\pgfsetdash{}{0pt}%
\pgfsys@defobject{currentmarker}{\pgfqpoint{-0.027778in}{0.000000in}}{\pgfqpoint{-0.000000in}{0.000000in}}{%
\pgfpathmoveto{\pgfqpoint{-0.000000in}{0.000000in}}%
\pgfpathlineto{\pgfqpoint{-0.027778in}{0.000000in}}%
\pgfusepath{stroke,fill}%
}%
\begin{pgfscope}%
\pgfsys@transformshift{2.243603in}{1.424062in}%
\pgfsys@useobject{currentmarker}{}%
\end{pgfscope}%
\end{pgfscope}%
\begin{pgfscope}%
\pgfsetbuttcap%
\pgfsetroundjoin%
\definecolor{currentfill}{rgb}{0.000000,0.000000,0.000000}%
\pgfsetfillcolor{currentfill}%
\pgfsetlinewidth{0.602250pt}%
\definecolor{currentstroke}{rgb}{0.000000,0.000000,0.000000}%
\pgfsetstrokecolor{currentstroke}%
\pgfsetdash{}{0pt}%
\pgfsys@defobject{currentmarker}{\pgfqpoint{-0.027778in}{0.000000in}}{\pgfqpoint{-0.000000in}{0.000000in}}{%
\pgfpathmoveto{\pgfqpoint{-0.000000in}{0.000000in}}%
\pgfpathlineto{\pgfqpoint{-0.027778in}{0.000000in}}%
\pgfusepath{stroke,fill}%
}%
\begin{pgfscope}%
\pgfsys@transformshift{2.243603in}{1.557984in}%
\pgfsys@useobject{currentmarker}{}%
\end{pgfscope}%
\end{pgfscope}%
\begin{pgfscope}%
\pgfsetbuttcap%
\pgfsetroundjoin%
\definecolor{currentfill}{rgb}{0.000000,0.000000,0.000000}%
\pgfsetfillcolor{currentfill}%
\pgfsetlinewidth{0.602250pt}%
\definecolor{currentstroke}{rgb}{0.000000,0.000000,0.000000}%
\pgfsetstrokecolor{currentstroke}%
\pgfsetdash{}{0pt}%
\pgfsys@defobject{currentmarker}{\pgfqpoint{-0.027778in}{0.000000in}}{\pgfqpoint{-0.000000in}{0.000000in}}{%
\pgfpathmoveto{\pgfqpoint{-0.000000in}{0.000000in}}%
\pgfpathlineto{\pgfqpoint{-0.027778in}{0.000000in}}%
\pgfusepath{stroke,fill}%
}%
\begin{pgfscope}%
\pgfsys@transformshift{2.243603in}{1.676111in}%
\pgfsys@useobject{currentmarker}{}%
\end{pgfscope}%
\end{pgfscope}%
\begin{pgfscope}%
\pgfpathrectangle{\pgfqpoint{2.243603in}{0.862810in}}{\pgfqpoint{1.405419in}{0.918970in}}%
\pgfusepath{clip}%
\pgfsetbuttcap%
\pgfsetroundjoin%
\pgfsetlinewidth{0.501875pt}%
\definecolor{currentstroke}{rgb}{0.392157,0.396078,0.403922}%
\pgfsetstrokecolor{currentstroke}%
\pgfsetdash{}{0pt}%
\pgfpathmoveto{\pgfqpoint{2.594958in}{0.852810in}}%
\pgfpathlineto{\pgfqpoint{2.594958in}{1.781780in}}%
\pgfusepath{stroke}%
\end{pgfscope}%
\begin{pgfscope}%
\pgfpathrectangle{\pgfqpoint{2.243603in}{0.862810in}}{\pgfqpoint{1.405419in}{0.918970in}}%
\pgfusepath{clip}%
\pgfsetbuttcap%
\pgfsetroundjoin%
\pgfsetlinewidth{0.501875pt}%
\definecolor{currentstroke}{rgb}{0.392157,0.396078,0.403922}%
\pgfsetstrokecolor{currentstroke}%
\pgfsetdash{}{0pt}%
\pgfpathmoveto{\pgfqpoint{2.946312in}{0.852810in}}%
\pgfpathlineto{\pgfqpoint{2.946312in}{1.781780in}}%
\pgfusepath{stroke}%
\end{pgfscope}%
\begin{pgfscope}%
\pgfpathrectangle{\pgfqpoint{2.243603in}{0.862810in}}{\pgfqpoint{1.405419in}{0.918970in}}%
\pgfusepath{clip}%
\pgfsetbuttcap%
\pgfsetroundjoin%
\pgfsetlinewidth{0.501875pt}%
\definecolor{currentstroke}{rgb}{0.392157,0.396078,0.403922}%
\pgfsetstrokecolor{currentstroke}%
\pgfsetdash{}{0pt}%
\pgfpathmoveto{\pgfqpoint{3.297667in}{0.852810in}}%
\pgfpathlineto{\pgfqpoint{3.297667in}{1.781780in}}%
\pgfusepath{stroke}%
\end{pgfscope}%
\begin{pgfscope}%
\pgfpathrectangle{\pgfqpoint{2.243603in}{0.862810in}}{\pgfqpoint{1.405419in}{0.918970in}}%
\pgfusepath{clip}%
\pgfsetrectcap%
\pgfsetroundjoin%
\pgfsetlinewidth{0.752812pt}%
\definecolor{currentstroke}{rgb}{0.000000,0.000000,0.000000}%
\pgfsetstrokecolor{currentstroke}%
\pgfsetdash{}{0pt}%
\pgfpathmoveto{\pgfqpoint{2.349010in}{1.172089in}}%
\pgfpathlineto{\pgfqpoint{2.349010in}{1.110583in}}%
\pgfusepath{stroke}%
\end{pgfscope}%
\begin{pgfscope}%
\pgfpathrectangle{\pgfqpoint{2.243603in}{0.862810in}}{\pgfqpoint{1.405419in}{0.918970in}}%
\pgfusepath{clip}%
\pgfsetrectcap%
\pgfsetroundjoin%
\pgfsetlinewidth{0.752812pt}%
\definecolor{currentstroke}{rgb}{0.000000,0.000000,0.000000}%
\pgfsetstrokecolor{currentstroke}%
\pgfsetdash{}{0pt}%
\pgfpathmoveto{\pgfqpoint{2.349010in}{1.318840in}}%
\pgfpathlineto{\pgfqpoint{2.349010in}{1.318840in}}%
\pgfusepath{stroke}%
\end{pgfscope}%
\begin{pgfscope}%
\pgfpathrectangle{\pgfqpoint{2.243603in}{0.862810in}}{\pgfqpoint{1.405419in}{0.918970in}}%
\pgfusepath{clip}%
\pgfsetrectcap%
\pgfsetroundjoin%
\pgfsetlinewidth{0.752812pt}%
\definecolor{currentstroke}{rgb}{0.000000,0.000000,0.000000}%
\pgfsetstrokecolor{currentstroke}%
\pgfsetdash{}{0pt}%
\pgfpathmoveto{\pgfqpoint{2.314577in}{1.110583in}}%
\pgfpathlineto{\pgfqpoint{2.383442in}{1.110583in}}%
\pgfusepath{stroke}%
\end{pgfscope}%
\begin{pgfscope}%
\pgfpathrectangle{\pgfqpoint{2.243603in}{0.862810in}}{\pgfqpoint{1.405419in}{0.918970in}}%
\pgfusepath{clip}%
\pgfsetrectcap%
\pgfsetroundjoin%
\pgfsetlinewidth{0.752812pt}%
\definecolor{currentstroke}{rgb}{0.000000,0.000000,0.000000}%
\pgfsetstrokecolor{currentstroke}%
\pgfsetdash{}{0pt}%
\pgfpathmoveto{\pgfqpoint{2.314577in}{1.318840in}}%
\pgfpathlineto{\pgfqpoint{2.383442in}{1.318840in}}%
\pgfusepath{stroke}%
\end{pgfscope}%
\begin{pgfscope}%
\pgfpathrectangle{\pgfqpoint{2.243603in}{0.862810in}}{\pgfqpoint{1.405419in}{0.918970in}}%
\pgfusepath{clip}%
\pgfsetbuttcap%
\pgfsetmiterjoin%
\definecolor{currentfill}{rgb}{0.000000,0.000000,0.000000}%
\pgfsetfillcolor{currentfill}%
\pgfsetlinewidth{1.003750pt}%
\definecolor{currentstroke}{rgb}{0.000000,0.000000,0.000000}%
\pgfsetstrokecolor{currentstroke}%
\pgfsetdash{}{0pt}%
\pgfsys@defobject{currentmarker}{\pgfqpoint{-0.011785in}{-0.019642in}}{\pgfqpoint{0.011785in}{0.019642in}}{%
\pgfpathmoveto{\pgfqpoint{-0.000000in}{-0.019642in}}%
\pgfpathlineto{\pgfqpoint{0.011785in}{0.000000in}}%
\pgfpathlineto{\pgfqpoint{0.000000in}{0.019642in}}%
\pgfpathlineto{\pgfqpoint{-0.011785in}{0.000000in}}%
\pgfpathclose%
\pgfusepath{stroke,fill}%
}%
\begin{pgfscope}%
\pgfsys@transformshift{2.349010in}{1.631810in}%
\pgfsys@useobject{currentmarker}{}%
\end{pgfscope}%
\end{pgfscope}%
\begin{pgfscope}%
\pgfpathrectangle{\pgfqpoint{2.243603in}{0.862810in}}{\pgfqpoint{1.405419in}{0.918970in}}%
\pgfusepath{clip}%
\pgfsetrectcap%
\pgfsetroundjoin%
\pgfsetlinewidth{0.752812pt}%
\definecolor{currentstroke}{rgb}{0.000000,0.000000,0.000000}%
\pgfsetstrokecolor{currentstroke}%
\pgfsetdash{}{0pt}%
\pgfpathmoveto{\pgfqpoint{2.489551in}{1.271447in}}%
\pgfpathlineto{\pgfqpoint{2.489551in}{1.125082in}}%
\pgfusepath{stroke}%
\end{pgfscope}%
\begin{pgfscope}%
\pgfpathrectangle{\pgfqpoint{2.243603in}{0.862810in}}{\pgfqpoint{1.405419in}{0.918970in}}%
\pgfusepath{clip}%
\pgfsetrectcap%
\pgfsetroundjoin%
\pgfsetlinewidth{0.752812pt}%
\definecolor{currentstroke}{rgb}{0.000000,0.000000,0.000000}%
\pgfsetstrokecolor{currentstroke}%
\pgfsetdash{}{0pt}%
\pgfpathmoveto{\pgfqpoint{2.489551in}{1.460991in}}%
\pgfpathlineto{\pgfqpoint{2.489551in}{1.464073in}}%
\pgfusepath{stroke}%
\end{pgfscope}%
\begin{pgfscope}%
\pgfpathrectangle{\pgfqpoint{2.243603in}{0.862810in}}{\pgfqpoint{1.405419in}{0.918970in}}%
\pgfusepath{clip}%
\pgfsetrectcap%
\pgfsetroundjoin%
\pgfsetlinewidth{0.752812pt}%
\definecolor{currentstroke}{rgb}{0.000000,0.000000,0.000000}%
\pgfsetstrokecolor{currentstroke}%
\pgfsetdash{}{0pt}%
\pgfpathmoveto{\pgfqpoint{2.455119in}{1.125082in}}%
\pgfpathlineto{\pgfqpoint{2.523984in}{1.125082in}}%
\pgfusepath{stroke}%
\end{pgfscope}%
\begin{pgfscope}%
\pgfpathrectangle{\pgfqpoint{2.243603in}{0.862810in}}{\pgfqpoint{1.405419in}{0.918970in}}%
\pgfusepath{clip}%
\pgfsetrectcap%
\pgfsetroundjoin%
\pgfsetlinewidth{0.752812pt}%
\definecolor{currentstroke}{rgb}{0.000000,0.000000,0.000000}%
\pgfsetstrokecolor{currentstroke}%
\pgfsetdash{}{0pt}%
\pgfpathmoveto{\pgfqpoint{2.455119in}{1.464073in}}%
\pgfpathlineto{\pgfqpoint{2.523984in}{1.464073in}}%
\pgfusepath{stroke}%
\end{pgfscope}%
\begin{pgfscope}%
\pgfpathrectangle{\pgfqpoint{2.243603in}{0.862810in}}{\pgfqpoint{1.405419in}{0.918970in}}%
\pgfusepath{clip}%
\pgfsetrectcap%
\pgfsetroundjoin%
\pgfsetlinewidth{0.752812pt}%
\definecolor{currentstroke}{rgb}{0.000000,0.000000,0.000000}%
\pgfsetstrokecolor{currentstroke}%
\pgfsetdash{}{0pt}%
\pgfpathmoveto{\pgfqpoint{2.700364in}{1.526244in}}%
\pgfpathlineto{\pgfqpoint{2.700364in}{1.526244in}}%
\pgfusepath{stroke}%
\end{pgfscope}%
\begin{pgfscope}%
\pgfpathrectangle{\pgfqpoint{2.243603in}{0.862810in}}{\pgfqpoint{1.405419in}{0.918970in}}%
\pgfusepath{clip}%
\pgfsetrectcap%
\pgfsetroundjoin%
\pgfsetlinewidth{0.752812pt}%
\definecolor{currentstroke}{rgb}{0.000000,0.000000,0.000000}%
\pgfsetstrokecolor{currentstroke}%
\pgfsetdash{}{0pt}%
\pgfpathmoveto{\pgfqpoint{2.700364in}{1.598749in}}%
\pgfpathlineto{\pgfqpoint{2.700364in}{1.636511in}}%
\pgfusepath{stroke}%
\end{pgfscope}%
\begin{pgfscope}%
\pgfpathrectangle{\pgfqpoint{2.243603in}{0.862810in}}{\pgfqpoint{1.405419in}{0.918970in}}%
\pgfusepath{clip}%
\pgfsetrectcap%
\pgfsetroundjoin%
\pgfsetlinewidth{0.752812pt}%
\definecolor{currentstroke}{rgb}{0.000000,0.000000,0.000000}%
\pgfsetstrokecolor{currentstroke}%
\pgfsetdash{}{0pt}%
\pgfpathmoveto{\pgfqpoint{2.665931in}{1.526244in}}%
\pgfpathlineto{\pgfqpoint{2.734797in}{1.526244in}}%
\pgfusepath{stroke}%
\end{pgfscope}%
\begin{pgfscope}%
\pgfpathrectangle{\pgfqpoint{2.243603in}{0.862810in}}{\pgfqpoint{1.405419in}{0.918970in}}%
\pgfusepath{clip}%
\pgfsetrectcap%
\pgfsetroundjoin%
\pgfsetlinewidth{0.752812pt}%
\definecolor{currentstroke}{rgb}{0.000000,0.000000,0.000000}%
\pgfsetstrokecolor{currentstroke}%
\pgfsetdash{}{0pt}%
\pgfpathmoveto{\pgfqpoint{2.665931in}{1.636511in}}%
\pgfpathlineto{\pgfqpoint{2.734797in}{1.636511in}}%
\pgfusepath{stroke}%
\end{pgfscope}%
\begin{pgfscope}%
\pgfpathrectangle{\pgfqpoint{2.243603in}{0.862810in}}{\pgfqpoint{1.405419in}{0.918970in}}%
\pgfusepath{clip}%
\pgfsetbuttcap%
\pgfsetmiterjoin%
\definecolor{currentfill}{rgb}{0.000000,0.000000,0.000000}%
\pgfsetfillcolor{currentfill}%
\pgfsetlinewidth{1.003750pt}%
\definecolor{currentstroke}{rgb}{0.000000,0.000000,0.000000}%
\pgfsetstrokecolor{currentstroke}%
\pgfsetdash{}{0pt}%
\pgfsys@defobject{currentmarker}{\pgfqpoint{-0.011785in}{-0.019642in}}{\pgfqpoint{0.011785in}{0.019642in}}{%
\pgfpathmoveto{\pgfqpoint{-0.000000in}{-0.019642in}}%
\pgfpathlineto{\pgfqpoint{0.011785in}{0.000000in}}%
\pgfpathlineto{\pgfqpoint{0.000000in}{0.019642in}}%
\pgfpathlineto{\pgfqpoint{-0.011785in}{0.000000in}}%
\pgfpathclose%
\pgfusepath{stroke,fill}%
}%
\begin{pgfscope}%
\pgfsys@transformshift{2.700364in}{1.248384in}%
\pgfsys@useobject{currentmarker}{}%
\end{pgfscope}%
\end{pgfscope}%
\begin{pgfscope}%
\pgfpathrectangle{\pgfqpoint{2.243603in}{0.862810in}}{\pgfqpoint{1.405419in}{0.918970in}}%
\pgfusepath{clip}%
\pgfsetrectcap%
\pgfsetroundjoin%
\pgfsetlinewidth{0.752812pt}%
\definecolor{currentstroke}{rgb}{0.000000,0.000000,0.000000}%
\pgfsetstrokecolor{currentstroke}%
\pgfsetdash{}{0pt}%
\pgfpathmoveto{\pgfqpoint{2.840906in}{1.473760in}}%
\pgfpathlineto{\pgfqpoint{2.840906in}{1.455710in}}%
\pgfusepath{stroke}%
\end{pgfscope}%
\begin{pgfscope}%
\pgfpathrectangle{\pgfqpoint{2.243603in}{0.862810in}}{\pgfqpoint{1.405419in}{0.918970in}}%
\pgfusepath{clip}%
\pgfsetrectcap%
\pgfsetroundjoin%
\pgfsetlinewidth{0.752812pt}%
\definecolor{currentstroke}{rgb}{0.000000,0.000000,0.000000}%
\pgfsetstrokecolor{currentstroke}%
\pgfsetdash{}{0pt}%
\pgfpathmoveto{\pgfqpoint{2.840906in}{1.764028in}}%
\pgfpathlineto{\pgfqpoint{2.840906in}{1.767573in}}%
\pgfusepath{stroke}%
\end{pgfscope}%
\begin{pgfscope}%
\pgfpathrectangle{\pgfqpoint{2.243603in}{0.862810in}}{\pgfqpoint{1.405419in}{0.918970in}}%
\pgfusepath{clip}%
\pgfsetrectcap%
\pgfsetroundjoin%
\pgfsetlinewidth{0.752812pt}%
\definecolor{currentstroke}{rgb}{0.000000,0.000000,0.000000}%
\pgfsetstrokecolor{currentstroke}%
\pgfsetdash{}{0pt}%
\pgfpathmoveto{\pgfqpoint{2.806473in}{1.455710in}}%
\pgfpathlineto{\pgfqpoint{2.875339in}{1.455710in}}%
\pgfusepath{stroke}%
\end{pgfscope}%
\begin{pgfscope}%
\pgfpathrectangle{\pgfqpoint{2.243603in}{0.862810in}}{\pgfqpoint{1.405419in}{0.918970in}}%
\pgfusepath{clip}%
\pgfsetrectcap%
\pgfsetroundjoin%
\pgfsetlinewidth{0.752812pt}%
\definecolor{currentstroke}{rgb}{0.000000,0.000000,0.000000}%
\pgfsetstrokecolor{currentstroke}%
\pgfsetdash{}{0pt}%
\pgfpathmoveto{\pgfqpoint{2.806473in}{1.767573in}}%
\pgfpathlineto{\pgfqpoint{2.875339in}{1.767573in}}%
\pgfusepath{stroke}%
\end{pgfscope}%
\begin{pgfscope}%
\pgfpathrectangle{\pgfqpoint{2.243603in}{0.862810in}}{\pgfqpoint{1.405419in}{0.918970in}}%
\pgfusepath{clip}%
\pgfsetrectcap%
\pgfsetroundjoin%
\pgfsetlinewidth{0.752812pt}%
\definecolor{currentstroke}{rgb}{0.000000,0.000000,0.000000}%
\pgfsetstrokecolor{currentstroke}%
\pgfsetdash{}{0pt}%
\pgfpathmoveto{\pgfqpoint{3.051719in}{0.967363in}}%
\pgfpathlineto{\pgfqpoint{3.051719in}{0.922314in}}%
\pgfusepath{stroke}%
\end{pgfscope}%
\begin{pgfscope}%
\pgfpathrectangle{\pgfqpoint{2.243603in}{0.862810in}}{\pgfqpoint{1.405419in}{0.918970in}}%
\pgfusepath{clip}%
\pgfsetrectcap%
\pgfsetroundjoin%
\pgfsetlinewidth{0.752812pt}%
\definecolor{currentstroke}{rgb}{0.000000,0.000000,0.000000}%
\pgfsetstrokecolor{currentstroke}%
\pgfsetdash{}{0pt}%
\pgfpathmoveto{\pgfqpoint{3.051719in}{1.355326in}}%
\pgfpathlineto{\pgfqpoint{3.051719in}{1.461633in}}%
\pgfusepath{stroke}%
\end{pgfscope}%
\begin{pgfscope}%
\pgfpathrectangle{\pgfqpoint{2.243603in}{0.862810in}}{\pgfqpoint{1.405419in}{0.918970in}}%
\pgfusepath{clip}%
\pgfsetrectcap%
\pgfsetroundjoin%
\pgfsetlinewidth{0.752812pt}%
\definecolor{currentstroke}{rgb}{0.000000,0.000000,0.000000}%
\pgfsetstrokecolor{currentstroke}%
\pgfsetdash{}{0pt}%
\pgfpathmoveto{\pgfqpoint{3.017286in}{0.922314in}}%
\pgfpathlineto{\pgfqpoint{3.086152in}{0.922314in}}%
\pgfusepath{stroke}%
\end{pgfscope}%
\begin{pgfscope}%
\pgfpathrectangle{\pgfqpoint{2.243603in}{0.862810in}}{\pgfqpoint{1.405419in}{0.918970in}}%
\pgfusepath{clip}%
\pgfsetrectcap%
\pgfsetroundjoin%
\pgfsetlinewidth{0.752812pt}%
\definecolor{currentstroke}{rgb}{0.000000,0.000000,0.000000}%
\pgfsetstrokecolor{currentstroke}%
\pgfsetdash{}{0pt}%
\pgfpathmoveto{\pgfqpoint{3.017286in}{1.461633in}}%
\pgfpathlineto{\pgfqpoint{3.086152in}{1.461633in}}%
\pgfusepath{stroke}%
\end{pgfscope}%
\begin{pgfscope}%
\pgfpathrectangle{\pgfqpoint{2.243603in}{0.862810in}}{\pgfqpoint{1.405419in}{0.918970in}}%
\pgfusepath{clip}%
\pgfsetrectcap%
\pgfsetroundjoin%
\pgfsetlinewidth{0.752812pt}%
\definecolor{currentstroke}{rgb}{0.000000,0.000000,0.000000}%
\pgfsetstrokecolor{currentstroke}%
\pgfsetdash{}{0pt}%
\pgfpathmoveto{\pgfqpoint{3.192261in}{1.000192in}}%
\pgfpathlineto{\pgfqpoint{3.192261in}{0.884618in}}%
\pgfusepath{stroke}%
\end{pgfscope}%
\begin{pgfscope}%
\pgfpathrectangle{\pgfqpoint{2.243603in}{0.862810in}}{\pgfqpoint{1.405419in}{0.918970in}}%
\pgfusepath{clip}%
\pgfsetrectcap%
\pgfsetroundjoin%
\pgfsetlinewidth{0.752812pt}%
\definecolor{currentstroke}{rgb}{0.000000,0.000000,0.000000}%
\pgfsetstrokecolor{currentstroke}%
\pgfsetdash{}{0pt}%
\pgfpathmoveto{\pgfqpoint{3.192261in}{1.202038in}}%
\pgfpathlineto{\pgfqpoint{3.192261in}{1.202038in}}%
\pgfusepath{stroke}%
\end{pgfscope}%
\begin{pgfscope}%
\pgfpathrectangle{\pgfqpoint{2.243603in}{0.862810in}}{\pgfqpoint{1.405419in}{0.918970in}}%
\pgfusepath{clip}%
\pgfsetrectcap%
\pgfsetroundjoin%
\pgfsetlinewidth{0.752812pt}%
\definecolor{currentstroke}{rgb}{0.000000,0.000000,0.000000}%
\pgfsetstrokecolor{currentstroke}%
\pgfsetdash{}{0pt}%
\pgfpathmoveto{\pgfqpoint{3.157828in}{0.884618in}}%
\pgfpathlineto{\pgfqpoint{3.226693in}{0.884618in}}%
\pgfusepath{stroke}%
\end{pgfscope}%
\begin{pgfscope}%
\pgfpathrectangle{\pgfqpoint{2.243603in}{0.862810in}}{\pgfqpoint{1.405419in}{0.918970in}}%
\pgfusepath{clip}%
\pgfsetrectcap%
\pgfsetroundjoin%
\pgfsetlinewidth{0.752812pt}%
\definecolor{currentstroke}{rgb}{0.000000,0.000000,0.000000}%
\pgfsetstrokecolor{currentstroke}%
\pgfsetdash{}{0pt}%
\pgfpathmoveto{\pgfqpoint{3.157828in}{1.202038in}}%
\pgfpathlineto{\pgfqpoint{3.226693in}{1.202038in}}%
\pgfusepath{stroke}%
\end{pgfscope}%
\begin{pgfscope}%
\pgfpathrectangle{\pgfqpoint{2.243603in}{0.862810in}}{\pgfqpoint{1.405419in}{0.918970in}}%
\pgfusepath{clip}%
\pgfsetbuttcap%
\pgfsetmiterjoin%
\definecolor{currentfill}{rgb}{0.000000,0.000000,0.000000}%
\pgfsetfillcolor{currentfill}%
\pgfsetlinewidth{1.003750pt}%
\definecolor{currentstroke}{rgb}{0.000000,0.000000,0.000000}%
\pgfsetstrokecolor{currentstroke}%
\pgfsetdash{}{0pt}%
\pgfsys@defobject{currentmarker}{\pgfqpoint{-0.011785in}{-0.019642in}}{\pgfqpoint{0.011785in}{0.019642in}}{%
\pgfpathmoveto{\pgfqpoint{-0.000000in}{-0.019642in}}%
\pgfpathlineto{\pgfqpoint{0.011785in}{0.000000in}}%
\pgfpathlineto{\pgfqpoint{0.000000in}{0.019642in}}%
\pgfpathlineto{\pgfqpoint{-0.011785in}{0.000000in}}%
\pgfpathclose%
\pgfusepath{stroke,fill}%
}%
\begin{pgfscope}%
\pgfsys@transformshift{3.192261in}{1.683757in}%
\pgfsys@useobject{currentmarker}{}%
\end{pgfscope}%
\end{pgfscope}%
\begin{pgfscope}%
\pgfpathrectangle{\pgfqpoint{2.243603in}{0.862810in}}{\pgfqpoint{1.405419in}{0.918970in}}%
\pgfusepath{clip}%
\pgfsetrectcap%
\pgfsetroundjoin%
\pgfsetlinewidth{0.752812pt}%
\definecolor{currentstroke}{rgb}{0.000000,0.000000,0.000000}%
\pgfsetstrokecolor{currentstroke}%
\pgfsetdash{}{0pt}%
\pgfpathmoveto{\pgfqpoint{3.403073in}{1.516878in}}%
\pgfpathlineto{\pgfqpoint{3.403073in}{1.516878in}}%
\pgfusepath{stroke}%
\end{pgfscope}%
\begin{pgfscope}%
\pgfpathrectangle{\pgfqpoint{2.243603in}{0.862810in}}{\pgfqpoint{1.405419in}{0.918970in}}%
\pgfusepath{clip}%
\pgfsetrectcap%
\pgfsetroundjoin%
\pgfsetlinewidth{0.752812pt}%
\definecolor{currentstroke}{rgb}{0.000000,0.000000,0.000000}%
\pgfsetstrokecolor{currentstroke}%
\pgfsetdash{}{0pt}%
\pgfpathmoveto{\pgfqpoint{3.403073in}{1.596485in}}%
\pgfpathlineto{\pgfqpoint{3.403073in}{1.596485in}}%
\pgfusepath{stroke}%
\end{pgfscope}%
\begin{pgfscope}%
\pgfpathrectangle{\pgfqpoint{2.243603in}{0.862810in}}{\pgfqpoint{1.405419in}{0.918970in}}%
\pgfusepath{clip}%
\pgfsetrectcap%
\pgfsetroundjoin%
\pgfsetlinewidth{0.752812pt}%
\definecolor{currentstroke}{rgb}{0.000000,0.000000,0.000000}%
\pgfsetstrokecolor{currentstroke}%
\pgfsetdash{}{0pt}%
\pgfpathmoveto{\pgfqpoint{3.368641in}{1.516878in}}%
\pgfpathlineto{\pgfqpoint{3.437506in}{1.516878in}}%
\pgfusepath{stroke}%
\end{pgfscope}%
\begin{pgfscope}%
\pgfpathrectangle{\pgfqpoint{2.243603in}{0.862810in}}{\pgfqpoint{1.405419in}{0.918970in}}%
\pgfusepath{clip}%
\pgfsetrectcap%
\pgfsetroundjoin%
\pgfsetlinewidth{0.752812pt}%
\definecolor{currentstroke}{rgb}{0.000000,0.000000,0.000000}%
\pgfsetstrokecolor{currentstroke}%
\pgfsetdash{}{0pt}%
\pgfpathmoveto{\pgfqpoint{3.368641in}{1.596485in}}%
\pgfpathlineto{\pgfqpoint{3.437506in}{1.596485in}}%
\pgfusepath{stroke}%
\end{pgfscope}%
\begin{pgfscope}%
\pgfpathrectangle{\pgfqpoint{2.243603in}{0.862810in}}{\pgfqpoint{1.405419in}{0.918970in}}%
\pgfusepath{clip}%
\pgfsetbuttcap%
\pgfsetmiterjoin%
\definecolor{currentfill}{rgb}{0.000000,0.000000,0.000000}%
\pgfsetfillcolor{currentfill}%
\pgfsetlinewidth{1.003750pt}%
\definecolor{currentstroke}{rgb}{0.000000,0.000000,0.000000}%
\pgfsetstrokecolor{currentstroke}%
\pgfsetdash{}{0pt}%
\pgfsys@defobject{currentmarker}{\pgfqpoint{-0.011785in}{-0.019642in}}{\pgfqpoint{0.011785in}{0.019642in}}{%
\pgfpathmoveto{\pgfqpoint{-0.000000in}{-0.019642in}}%
\pgfpathlineto{\pgfqpoint{0.011785in}{0.000000in}}%
\pgfpathlineto{\pgfqpoint{0.000000in}{0.019642in}}%
\pgfpathlineto{\pgfqpoint{-0.011785in}{0.000000in}}%
\pgfpathclose%
\pgfusepath{stroke,fill}%
}%
\begin{pgfscope}%
\pgfsys@transformshift{3.403073in}{1.069359in}%
\pgfsys@useobject{currentmarker}{}%
\end{pgfscope}%
\begin{pgfscope}%
\pgfsys@transformshift{3.403073in}{1.719380in}%
\pgfsys@useobject{currentmarker}{}%
\end{pgfscope}%
\end{pgfscope}%
\begin{pgfscope}%
\pgfpathrectangle{\pgfqpoint{2.243603in}{0.862810in}}{\pgfqpoint{1.405419in}{0.918970in}}%
\pgfusepath{clip}%
\pgfsetrectcap%
\pgfsetroundjoin%
\pgfsetlinewidth{0.752812pt}%
\definecolor{currentstroke}{rgb}{0.000000,0.000000,0.000000}%
\pgfsetstrokecolor{currentstroke}%
\pgfsetdash{}{0pt}%
\pgfpathmoveto{\pgfqpoint{3.543615in}{1.430330in}}%
\pgfpathlineto{\pgfqpoint{3.543615in}{1.375119in}}%
\pgfusepath{stroke}%
\end{pgfscope}%
\begin{pgfscope}%
\pgfpathrectangle{\pgfqpoint{2.243603in}{0.862810in}}{\pgfqpoint{1.405419in}{0.918970in}}%
\pgfusepath{clip}%
\pgfsetrectcap%
\pgfsetroundjoin%
\pgfsetlinewidth{0.752812pt}%
\definecolor{currentstroke}{rgb}{0.000000,0.000000,0.000000}%
\pgfsetstrokecolor{currentstroke}%
\pgfsetdash{}{0pt}%
\pgfpathmoveto{\pgfqpoint{3.543615in}{1.669468in}}%
\pgfpathlineto{\pgfqpoint{3.543615in}{1.768697in}}%
\pgfusepath{stroke}%
\end{pgfscope}%
\begin{pgfscope}%
\pgfpathrectangle{\pgfqpoint{2.243603in}{0.862810in}}{\pgfqpoint{1.405419in}{0.918970in}}%
\pgfusepath{clip}%
\pgfsetrectcap%
\pgfsetroundjoin%
\pgfsetlinewidth{0.752812pt}%
\definecolor{currentstroke}{rgb}{0.000000,0.000000,0.000000}%
\pgfsetstrokecolor{currentstroke}%
\pgfsetdash{}{0pt}%
\pgfpathmoveto{\pgfqpoint{3.509183in}{1.375119in}}%
\pgfpathlineto{\pgfqpoint{3.578048in}{1.375119in}}%
\pgfusepath{stroke}%
\end{pgfscope}%
\begin{pgfscope}%
\pgfpathrectangle{\pgfqpoint{2.243603in}{0.862810in}}{\pgfqpoint{1.405419in}{0.918970in}}%
\pgfusepath{clip}%
\pgfsetrectcap%
\pgfsetroundjoin%
\pgfsetlinewidth{0.752812pt}%
\definecolor{currentstroke}{rgb}{0.000000,0.000000,0.000000}%
\pgfsetstrokecolor{currentstroke}%
\pgfsetdash{}{0pt}%
\pgfpathmoveto{\pgfqpoint{3.509183in}{1.768697in}}%
\pgfpathlineto{\pgfqpoint{3.578048in}{1.768697in}}%
\pgfusepath{stroke}%
\end{pgfscope}%
\begin{pgfscope}%
\pgfpathrectangle{\pgfqpoint{2.243603in}{0.862810in}}{\pgfqpoint{1.405419in}{0.918970in}}%
\pgfusepath{clip}%
\pgfsetrectcap%
\pgfsetroundjoin%
\pgfsetlinewidth{0.752812pt}%
\definecolor{currentstroke}{rgb}{0.000000,0.000000,0.000000}%
\pgfsetstrokecolor{currentstroke}%
\pgfsetdash{}{0pt}%
\pgfpathmoveto{\pgfqpoint{2.280144in}{1.311618in}}%
\pgfpathlineto{\pgfqpoint{2.417875in}{1.311618in}}%
\pgfusepath{stroke}%
\end{pgfscope}%
\begin{pgfscope}%
\pgfpathrectangle{\pgfqpoint{2.243603in}{0.862810in}}{\pgfqpoint{1.405419in}{0.918970in}}%
\pgfusepath{clip}%
\pgfsetbuttcap%
\pgfsetroundjoin%
\definecolor{currentfill}{rgb}{1.000000,1.000000,1.000000}%
\pgfsetfillcolor{currentfill}%
\pgfsetlinewidth{1.003750pt}%
\definecolor{currentstroke}{rgb}{0.000000,0.000000,0.000000}%
\pgfsetstrokecolor{currentstroke}%
\pgfsetdash{}{0pt}%
\pgfsys@defobject{currentmarker}{\pgfqpoint{-0.027778in}{-0.027778in}}{\pgfqpoint{0.027778in}{0.027778in}}{%
\pgfpathmoveto{\pgfqpoint{0.000000in}{-0.027778in}}%
\pgfpathcurveto{\pgfqpoint{0.007367in}{-0.027778in}}{\pgfqpoint{0.014433in}{-0.024851in}}{\pgfqpoint{0.019642in}{-0.019642in}}%
\pgfpathcurveto{\pgfqpoint{0.024851in}{-0.014433in}}{\pgfqpoint{0.027778in}{-0.007367in}}{\pgfqpoint{0.027778in}{0.000000in}}%
\pgfpathcurveto{\pgfqpoint{0.027778in}{0.007367in}}{\pgfqpoint{0.024851in}{0.014433in}}{\pgfqpoint{0.019642in}{0.019642in}}%
\pgfpathcurveto{\pgfqpoint{0.014433in}{0.024851in}}{\pgfqpoint{0.007367in}{0.027778in}}{\pgfqpoint{0.000000in}{0.027778in}}%
\pgfpathcurveto{\pgfqpoint{-0.007367in}{0.027778in}}{\pgfqpoint{-0.014433in}{0.024851in}}{\pgfqpoint{-0.019642in}{0.019642in}}%
\pgfpathcurveto{\pgfqpoint{-0.024851in}{0.014433in}}{\pgfqpoint{-0.027778in}{0.007367in}}{\pgfqpoint{-0.027778in}{0.000000in}}%
\pgfpathcurveto{\pgfqpoint{-0.027778in}{-0.007367in}}{\pgfqpoint{-0.024851in}{-0.014433in}}{\pgfqpoint{-0.019642in}{-0.019642in}}%
\pgfpathcurveto{\pgfqpoint{-0.014433in}{-0.024851in}}{\pgfqpoint{-0.007367in}{-0.027778in}}{\pgfqpoint{0.000000in}{-0.027778in}}%
\pgfpathclose%
\pgfusepath{stroke,fill}%
}%
\begin{pgfscope}%
\pgfsys@transformshift{2.349010in}{1.325921in}%
\pgfsys@useobject{currentmarker}{}%
\end{pgfscope}%
\end{pgfscope}%
\begin{pgfscope}%
\pgfpathrectangle{\pgfqpoint{2.243603in}{0.862810in}}{\pgfqpoint{1.405419in}{0.918970in}}%
\pgfusepath{clip}%
\pgfsetrectcap%
\pgfsetroundjoin%
\pgfsetlinewidth{0.752812pt}%
\definecolor{currentstroke}{rgb}{0.000000,0.000000,0.000000}%
\pgfsetstrokecolor{currentstroke}%
\pgfsetdash{}{0pt}%
\pgfpathmoveto{\pgfqpoint{2.420686in}{1.283534in}}%
\pgfpathlineto{\pgfqpoint{2.558417in}{1.283534in}}%
\pgfusepath{stroke}%
\end{pgfscope}%
\begin{pgfscope}%
\pgfpathrectangle{\pgfqpoint{2.243603in}{0.862810in}}{\pgfqpoint{1.405419in}{0.918970in}}%
\pgfusepath{clip}%
\pgfsetbuttcap%
\pgfsetroundjoin%
\definecolor{currentfill}{rgb}{1.000000,1.000000,1.000000}%
\pgfsetfillcolor{currentfill}%
\pgfsetlinewidth{1.003750pt}%
\definecolor{currentstroke}{rgb}{0.000000,0.000000,0.000000}%
\pgfsetstrokecolor{currentstroke}%
\pgfsetdash{}{0pt}%
\pgfsys@defobject{currentmarker}{\pgfqpoint{-0.027778in}{-0.027778in}}{\pgfqpoint{0.027778in}{0.027778in}}{%
\pgfpathmoveto{\pgfqpoint{0.000000in}{-0.027778in}}%
\pgfpathcurveto{\pgfqpoint{0.007367in}{-0.027778in}}{\pgfqpoint{0.014433in}{-0.024851in}}{\pgfqpoint{0.019642in}{-0.019642in}}%
\pgfpathcurveto{\pgfqpoint{0.024851in}{-0.014433in}}{\pgfqpoint{0.027778in}{-0.007367in}}{\pgfqpoint{0.027778in}{0.000000in}}%
\pgfpathcurveto{\pgfqpoint{0.027778in}{0.007367in}}{\pgfqpoint{0.024851in}{0.014433in}}{\pgfqpoint{0.019642in}{0.019642in}}%
\pgfpathcurveto{\pgfqpoint{0.014433in}{0.024851in}}{\pgfqpoint{0.007367in}{0.027778in}}{\pgfqpoint{0.000000in}{0.027778in}}%
\pgfpathcurveto{\pgfqpoint{-0.007367in}{0.027778in}}{\pgfqpoint{-0.014433in}{0.024851in}}{\pgfqpoint{-0.019642in}{0.019642in}}%
\pgfpathcurveto{\pgfqpoint{-0.024851in}{0.014433in}}{\pgfqpoint{-0.027778in}{0.007367in}}{\pgfqpoint{-0.027778in}{0.000000in}}%
\pgfpathcurveto{\pgfqpoint{-0.027778in}{-0.007367in}}{\pgfqpoint{-0.024851in}{-0.014433in}}{\pgfqpoint{-0.019642in}{-0.019642in}}%
\pgfpathcurveto{\pgfqpoint{-0.014433in}{-0.024851in}}{\pgfqpoint{-0.007367in}{-0.027778in}}{\pgfqpoint{0.000000in}{-0.027778in}}%
\pgfpathclose%
\pgfusepath{stroke,fill}%
}%
\begin{pgfscope}%
\pgfsys@transformshift{2.489551in}{1.329151in}%
\pgfsys@useobject{currentmarker}{}%
\end{pgfscope}%
\end{pgfscope}%
\begin{pgfscope}%
\pgfpathrectangle{\pgfqpoint{2.243603in}{0.862810in}}{\pgfqpoint{1.405419in}{0.918970in}}%
\pgfusepath{clip}%
\pgfsetrectcap%
\pgfsetroundjoin%
\pgfsetlinewidth{0.752812pt}%
\definecolor{currentstroke}{rgb}{0.000000,0.000000,0.000000}%
\pgfsetstrokecolor{currentstroke}%
\pgfsetdash{}{0pt}%
\pgfpathmoveto{\pgfqpoint{2.631499in}{1.573988in}}%
\pgfpathlineto{\pgfqpoint{2.769230in}{1.573988in}}%
\pgfusepath{stroke}%
\end{pgfscope}%
\begin{pgfscope}%
\pgfpathrectangle{\pgfqpoint{2.243603in}{0.862810in}}{\pgfqpoint{1.405419in}{0.918970in}}%
\pgfusepath{clip}%
\pgfsetbuttcap%
\pgfsetroundjoin%
\definecolor{currentfill}{rgb}{1.000000,1.000000,1.000000}%
\pgfsetfillcolor{currentfill}%
\pgfsetlinewidth{1.003750pt}%
\definecolor{currentstroke}{rgb}{0.000000,0.000000,0.000000}%
\pgfsetstrokecolor{currentstroke}%
\pgfsetdash{}{0pt}%
\pgfsys@defobject{currentmarker}{\pgfqpoint{-0.027778in}{-0.027778in}}{\pgfqpoint{0.027778in}{0.027778in}}{%
\pgfpathmoveto{\pgfqpoint{0.000000in}{-0.027778in}}%
\pgfpathcurveto{\pgfqpoint{0.007367in}{-0.027778in}}{\pgfqpoint{0.014433in}{-0.024851in}}{\pgfqpoint{0.019642in}{-0.019642in}}%
\pgfpathcurveto{\pgfqpoint{0.024851in}{-0.014433in}}{\pgfqpoint{0.027778in}{-0.007367in}}{\pgfqpoint{0.027778in}{0.000000in}}%
\pgfpathcurveto{\pgfqpoint{0.027778in}{0.007367in}}{\pgfqpoint{0.024851in}{0.014433in}}{\pgfqpoint{0.019642in}{0.019642in}}%
\pgfpathcurveto{\pgfqpoint{0.014433in}{0.024851in}}{\pgfqpoint{0.007367in}{0.027778in}}{\pgfqpoint{0.000000in}{0.027778in}}%
\pgfpathcurveto{\pgfqpoint{-0.007367in}{0.027778in}}{\pgfqpoint{-0.014433in}{0.024851in}}{\pgfqpoint{-0.019642in}{0.019642in}}%
\pgfpathcurveto{\pgfqpoint{-0.024851in}{0.014433in}}{\pgfqpoint{-0.027778in}{0.007367in}}{\pgfqpoint{-0.027778in}{0.000000in}}%
\pgfpathcurveto{\pgfqpoint{-0.027778in}{-0.007367in}}{\pgfqpoint{-0.024851in}{-0.014433in}}{\pgfqpoint{-0.019642in}{-0.019642in}}%
\pgfpathcurveto{\pgfqpoint{-0.014433in}{-0.024851in}}{\pgfqpoint{-0.007367in}{-0.027778in}}{\pgfqpoint{0.000000in}{-0.027778in}}%
\pgfpathclose%
\pgfusepath{stroke,fill}%
}%
\begin{pgfscope}%
\pgfsys@transformshift{2.700364in}{1.525835in}%
\pgfsys@useobject{currentmarker}{}%
\end{pgfscope}%
\end{pgfscope}%
\begin{pgfscope}%
\pgfpathrectangle{\pgfqpoint{2.243603in}{0.862810in}}{\pgfqpoint{1.405419in}{0.918970in}}%
\pgfusepath{clip}%
\pgfsetrectcap%
\pgfsetroundjoin%
\pgfsetlinewidth{0.752812pt}%
\definecolor{currentstroke}{rgb}{0.000000,0.000000,0.000000}%
\pgfsetstrokecolor{currentstroke}%
\pgfsetdash{}{0pt}%
\pgfpathmoveto{\pgfqpoint{2.772041in}{1.629652in}}%
\pgfpathlineto{\pgfqpoint{2.909772in}{1.629652in}}%
\pgfusepath{stroke}%
\end{pgfscope}%
\begin{pgfscope}%
\pgfpathrectangle{\pgfqpoint{2.243603in}{0.862810in}}{\pgfqpoint{1.405419in}{0.918970in}}%
\pgfusepath{clip}%
\pgfsetbuttcap%
\pgfsetroundjoin%
\definecolor{currentfill}{rgb}{1.000000,1.000000,1.000000}%
\pgfsetfillcolor{currentfill}%
\pgfsetlinewidth{1.003750pt}%
\definecolor{currentstroke}{rgb}{0.000000,0.000000,0.000000}%
\pgfsetstrokecolor{currentstroke}%
\pgfsetdash{}{0pt}%
\pgfsys@defobject{currentmarker}{\pgfqpoint{-0.027778in}{-0.027778in}}{\pgfqpoint{0.027778in}{0.027778in}}{%
\pgfpathmoveto{\pgfqpoint{0.000000in}{-0.027778in}}%
\pgfpathcurveto{\pgfqpoint{0.007367in}{-0.027778in}}{\pgfqpoint{0.014433in}{-0.024851in}}{\pgfqpoint{0.019642in}{-0.019642in}}%
\pgfpathcurveto{\pgfqpoint{0.024851in}{-0.014433in}}{\pgfqpoint{0.027778in}{-0.007367in}}{\pgfqpoint{0.027778in}{0.000000in}}%
\pgfpathcurveto{\pgfqpoint{0.027778in}{0.007367in}}{\pgfqpoint{0.024851in}{0.014433in}}{\pgfqpoint{0.019642in}{0.019642in}}%
\pgfpathcurveto{\pgfqpoint{0.014433in}{0.024851in}}{\pgfqpoint{0.007367in}{0.027778in}}{\pgfqpoint{0.000000in}{0.027778in}}%
\pgfpathcurveto{\pgfqpoint{-0.007367in}{0.027778in}}{\pgfqpoint{-0.014433in}{0.024851in}}{\pgfqpoint{-0.019642in}{0.019642in}}%
\pgfpathcurveto{\pgfqpoint{-0.024851in}{0.014433in}}{\pgfqpoint{-0.027778in}{0.007367in}}{\pgfqpoint{-0.027778in}{0.000000in}}%
\pgfpathcurveto{\pgfqpoint{-0.027778in}{-0.007367in}}{\pgfqpoint{-0.024851in}{-0.014433in}}{\pgfqpoint{-0.019642in}{-0.019642in}}%
\pgfpathcurveto{\pgfqpoint{-0.014433in}{-0.024851in}}{\pgfqpoint{-0.007367in}{-0.027778in}}{\pgfqpoint{0.000000in}{-0.027778in}}%
\pgfpathclose%
\pgfusepath{stroke,fill}%
}%
\begin{pgfscope}%
\pgfsys@transformshift{2.840906in}{1.627164in}%
\pgfsys@useobject{currentmarker}{}%
\end{pgfscope}%
\end{pgfscope}%
\begin{pgfscope}%
\pgfpathrectangle{\pgfqpoint{2.243603in}{0.862810in}}{\pgfqpoint{1.405419in}{0.918970in}}%
\pgfusepath{clip}%
\pgfsetrectcap%
\pgfsetroundjoin%
\pgfsetlinewidth{0.752812pt}%
\definecolor{currentstroke}{rgb}{0.000000,0.000000,0.000000}%
\pgfsetstrokecolor{currentstroke}%
\pgfsetdash{}{0pt}%
\pgfpathmoveto{\pgfqpoint{2.982853in}{0.984904in}}%
\pgfpathlineto{\pgfqpoint{3.120584in}{0.984904in}}%
\pgfusepath{stroke}%
\end{pgfscope}%
\begin{pgfscope}%
\pgfpathrectangle{\pgfqpoint{2.243603in}{0.862810in}}{\pgfqpoint{1.405419in}{0.918970in}}%
\pgfusepath{clip}%
\pgfsetbuttcap%
\pgfsetroundjoin%
\definecolor{currentfill}{rgb}{1.000000,1.000000,1.000000}%
\pgfsetfillcolor{currentfill}%
\pgfsetlinewidth{1.003750pt}%
\definecolor{currentstroke}{rgb}{0.000000,0.000000,0.000000}%
\pgfsetstrokecolor{currentstroke}%
\pgfsetdash{}{0pt}%
\pgfsys@defobject{currentmarker}{\pgfqpoint{-0.027778in}{-0.027778in}}{\pgfqpoint{0.027778in}{0.027778in}}{%
\pgfpathmoveto{\pgfqpoint{0.000000in}{-0.027778in}}%
\pgfpathcurveto{\pgfqpoint{0.007367in}{-0.027778in}}{\pgfqpoint{0.014433in}{-0.024851in}}{\pgfqpoint{0.019642in}{-0.019642in}}%
\pgfpathcurveto{\pgfqpoint{0.024851in}{-0.014433in}}{\pgfqpoint{0.027778in}{-0.007367in}}{\pgfqpoint{0.027778in}{0.000000in}}%
\pgfpathcurveto{\pgfqpoint{0.027778in}{0.007367in}}{\pgfqpoint{0.024851in}{0.014433in}}{\pgfqpoint{0.019642in}{0.019642in}}%
\pgfpathcurveto{\pgfqpoint{0.014433in}{0.024851in}}{\pgfqpoint{0.007367in}{0.027778in}}{\pgfqpoint{0.000000in}{0.027778in}}%
\pgfpathcurveto{\pgfqpoint{-0.007367in}{0.027778in}}{\pgfqpoint{-0.014433in}{0.024851in}}{\pgfqpoint{-0.019642in}{0.019642in}}%
\pgfpathcurveto{\pgfqpoint{-0.024851in}{0.014433in}}{\pgfqpoint{-0.027778in}{0.007367in}}{\pgfqpoint{-0.027778in}{0.000000in}}%
\pgfpathcurveto{\pgfqpoint{-0.027778in}{-0.007367in}}{\pgfqpoint{-0.024851in}{-0.014433in}}{\pgfqpoint{-0.019642in}{-0.019642in}}%
\pgfpathcurveto{\pgfqpoint{-0.014433in}{-0.024851in}}{\pgfqpoint{-0.007367in}{-0.027778in}}{\pgfqpoint{0.000000in}{-0.027778in}}%
\pgfpathclose%
\pgfusepath{stroke,fill}%
}%
\begin{pgfscope}%
\pgfsys@transformshift{3.051719in}{1.163992in}%
\pgfsys@useobject{currentmarker}{}%
\end{pgfscope}%
\end{pgfscope}%
\begin{pgfscope}%
\pgfpathrectangle{\pgfqpoint{2.243603in}{0.862810in}}{\pgfqpoint{1.405419in}{0.918970in}}%
\pgfusepath{clip}%
\pgfsetrectcap%
\pgfsetroundjoin%
\pgfsetlinewidth{0.752812pt}%
\definecolor{currentstroke}{rgb}{0.000000,0.000000,0.000000}%
\pgfsetstrokecolor{currentstroke}%
\pgfsetdash{}{0pt}%
\pgfpathmoveto{\pgfqpoint{3.123395in}{1.116982in}}%
\pgfpathlineto{\pgfqpoint{3.261126in}{1.116982in}}%
\pgfusepath{stroke}%
\end{pgfscope}%
\begin{pgfscope}%
\pgfpathrectangle{\pgfqpoint{2.243603in}{0.862810in}}{\pgfqpoint{1.405419in}{0.918970in}}%
\pgfusepath{clip}%
\pgfsetbuttcap%
\pgfsetroundjoin%
\definecolor{currentfill}{rgb}{1.000000,1.000000,1.000000}%
\pgfsetfillcolor{currentfill}%
\pgfsetlinewidth{1.003750pt}%
\definecolor{currentstroke}{rgb}{0.000000,0.000000,0.000000}%
\pgfsetstrokecolor{currentstroke}%
\pgfsetdash{}{0pt}%
\pgfsys@defobject{currentmarker}{\pgfqpoint{-0.027778in}{-0.027778in}}{\pgfqpoint{0.027778in}{0.027778in}}{%
\pgfpathmoveto{\pgfqpoint{0.000000in}{-0.027778in}}%
\pgfpathcurveto{\pgfqpoint{0.007367in}{-0.027778in}}{\pgfqpoint{0.014433in}{-0.024851in}}{\pgfqpoint{0.019642in}{-0.019642in}}%
\pgfpathcurveto{\pgfqpoint{0.024851in}{-0.014433in}}{\pgfqpoint{0.027778in}{-0.007367in}}{\pgfqpoint{0.027778in}{0.000000in}}%
\pgfpathcurveto{\pgfqpoint{0.027778in}{0.007367in}}{\pgfqpoint{0.024851in}{0.014433in}}{\pgfqpoint{0.019642in}{0.019642in}}%
\pgfpathcurveto{\pgfqpoint{0.014433in}{0.024851in}}{\pgfqpoint{0.007367in}{0.027778in}}{\pgfqpoint{0.000000in}{0.027778in}}%
\pgfpathcurveto{\pgfqpoint{-0.007367in}{0.027778in}}{\pgfqpoint{-0.014433in}{0.024851in}}{\pgfqpoint{-0.019642in}{0.019642in}}%
\pgfpathcurveto{\pgfqpoint{-0.024851in}{0.014433in}}{\pgfqpoint{-0.027778in}{0.007367in}}{\pgfqpoint{-0.027778in}{0.000000in}}%
\pgfpathcurveto{\pgfqpoint{-0.027778in}{-0.007367in}}{\pgfqpoint{-0.024851in}{-0.014433in}}{\pgfqpoint{-0.019642in}{-0.019642in}}%
\pgfpathcurveto{\pgfqpoint{-0.014433in}{-0.024851in}}{\pgfqpoint{-0.007367in}{-0.027778in}}{\pgfqpoint{0.000000in}{-0.027778in}}%
\pgfpathclose%
\pgfusepath{stroke,fill}%
}%
\begin{pgfscope}%
\pgfsys@transformshift{3.192261in}{1.218288in}%
\pgfsys@useobject{currentmarker}{}%
\end{pgfscope}%
\end{pgfscope}%
\begin{pgfscope}%
\pgfpathrectangle{\pgfqpoint{2.243603in}{0.862810in}}{\pgfqpoint{1.405419in}{0.918970in}}%
\pgfusepath{clip}%
\pgfsetrectcap%
\pgfsetroundjoin%
\pgfsetlinewidth{0.752812pt}%
\definecolor{currentstroke}{rgb}{0.000000,0.000000,0.000000}%
\pgfsetstrokecolor{currentstroke}%
\pgfsetdash{}{0pt}%
\pgfpathmoveto{\pgfqpoint{3.334208in}{1.584078in}}%
\pgfpathlineto{\pgfqpoint{3.471939in}{1.584078in}}%
\pgfusepath{stroke}%
\end{pgfscope}%
\begin{pgfscope}%
\pgfpathrectangle{\pgfqpoint{2.243603in}{0.862810in}}{\pgfqpoint{1.405419in}{0.918970in}}%
\pgfusepath{clip}%
\pgfsetbuttcap%
\pgfsetroundjoin%
\definecolor{currentfill}{rgb}{1.000000,1.000000,1.000000}%
\pgfsetfillcolor{currentfill}%
\pgfsetlinewidth{1.003750pt}%
\definecolor{currentstroke}{rgb}{0.000000,0.000000,0.000000}%
\pgfsetstrokecolor{currentstroke}%
\pgfsetdash{}{0pt}%
\pgfsys@defobject{currentmarker}{\pgfqpoint{-0.027778in}{-0.027778in}}{\pgfqpoint{0.027778in}{0.027778in}}{%
\pgfpathmoveto{\pgfqpoint{0.000000in}{-0.027778in}}%
\pgfpathcurveto{\pgfqpoint{0.007367in}{-0.027778in}}{\pgfqpoint{0.014433in}{-0.024851in}}{\pgfqpoint{0.019642in}{-0.019642in}}%
\pgfpathcurveto{\pgfqpoint{0.024851in}{-0.014433in}}{\pgfqpoint{0.027778in}{-0.007367in}}{\pgfqpoint{0.027778in}{0.000000in}}%
\pgfpathcurveto{\pgfqpoint{0.027778in}{0.007367in}}{\pgfqpoint{0.024851in}{0.014433in}}{\pgfqpoint{0.019642in}{0.019642in}}%
\pgfpathcurveto{\pgfqpoint{0.014433in}{0.024851in}}{\pgfqpoint{0.007367in}{0.027778in}}{\pgfqpoint{0.000000in}{0.027778in}}%
\pgfpathcurveto{\pgfqpoint{-0.007367in}{0.027778in}}{\pgfqpoint{-0.014433in}{0.024851in}}{\pgfqpoint{-0.019642in}{0.019642in}}%
\pgfpathcurveto{\pgfqpoint{-0.024851in}{0.014433in}}{\pgfqpoint{-0.027778in}{0.007367in}}{\pgfqpoint{-0.027778in}{0.000000in}}%
\pgfpathcurveto{\pgfqpoint{-0.027778in}{-0.007367in}}{\pgfqpoint{-0.024851in}{-0.014433in}}{\pgfqpoint{-0.019642in}{-0.019642in}}%
\pgfpathcurveto{\pgfqpoint{-0.014433in}{-0.024851in}}{\pgfqpoint{-0.007367in}{-0.027778in}}{\pgfqpoint{0.000000in}{-0.027778in}}%
\pgfpathclose%
\pgfusepath{stroke,fill}%
}%
\begin{pgfscope}%
\pgfsys@transformshift{3.403073in}{1.520022in}%
\pgfsys@useobject{currentmarker}{}%
\end{pgfscope}%
\end{pgfscope}%
\begin{pgfscope}%
\pgfpathrectangle{\pgfqpoint{2.243603in}{0.862810in}}{\pgfqpoint{1.405419in}{0.918970in}}%
\pgfusepath{clip}%
\pgfsetrectcap%
\pgfsetroundjoin%
\pgfsetlinewidth{0.752812pt}%
\definecolor{currentstroke}{rgb}{0.000000,0.000000,0.000000}%
\pgfsetstrokecolor{currentstroke}%
\pgfsetdash{}{0pt}%
\pgfpathmoveto{\pgfqpoint{3.474750in}{1.453404in}}%
\pgfpathlineto{\pgfqpoint{3.612481in}{1.453404in}}%
\pgfusepath{stroke}%
\end{pgfscope}%
\begin{pgfscope}%
\pgfpathrectangle{\pgfqpoint{2.243603in}{0.862810in}}{\pgfqpoint{1.405419in}{0.918970in}}%
\pgfusepath{clip}%
\pgfsetbuttcap%
\pgfsetroundjoin%
\definecolor{currentfill}{rgb}{1.000000,1.000000,1.000000}%
\pgfsetfillcolor{currentfill}%
\pgfsetlinewidth{1.003750pt}%
\definecolor{currentstroke}{rgb}{0.000000,0.000000,0.000000}%
\pgfsetstrokecolor{currentstroke}%
\pgfsetdash{}{0pt}%
\pgfsys@defobject{currentmarker}{\pgfqpoint{-0.027778in}{-0.027778in}}{\pgfqpoint{0.027778in}{0.027778in}}{%
\pgfpathmoveto{\pgfqpoint{0.000000in}{-0.027778in}}%
\pgfpathcurveto{\pgfqpoint{0.007367in}{-0.027778in}}{\pgfqpoint{0.014433in}{-0.024851in}}{\pgfqpoint{0.019642in}{-0.019642in}}%
\pgfpathcurveto{\pgfqpoint{0.024851in}{-0.014433in}}{\pgfqpoint{0.027778in}{-0.007367in}}{\pgfqpoint{0.027778in}{0.000000in}}%
\pgfpathcurveto{\pgfqpoint{0.027778in}{0.007367in}}{\pgfqpoint{0.024851in}{0.014433in}}{\pgfqpoint{0.019642in}{0.019642in}}%
\pgfpathcurveto{\pgfqpoint{0.014433in}{0.024851in}}{\pgfqpoint{0.007367in}{0.027778in}}{\pgfqpoint{0.000000in}{0.027778in}}%
\pgfpathcurveto{\pgfqpoint{-0.007367in}{0.027778in}}{\pgfqpoint{-0.014433in}{0.024851in}}{\pgfqpoint{-0.019642in}{0.019642in}}%
\pgfpathcurveto{\pgfqpoint{-0.024851in}{0.014433in}}{\pgfqpoint{-0.027778in}{0.007367in}}{\pgfqpoint{-0.027778in}{0.000000in}}%
\pgfpathcurveto{\pgfqpoint{-0.027778in}{-0.007367in}}{\pgfqpoint{-0.024851in}{-0.014433in}}{\pgfqpoint{-0.019642in}{-0.019642in}}%
\pgfpathcurveto{\pgfqpoint{-0.014433in}{-0.024851in}}{\pgfqpoint{-0.007367in}{-0.027778in}}{\pgfqpoint{0.000000in}{-0.027778in}}%
\pgfpathclose%
\pgfusepath{stroke,fill}%
}%
\begin{pgfscope}%
\pgfsys@transformshift{3.543615in}{1.551173in}%
\pgfsys@useobject{currentmarker}{}%
\end{pgfscope}%
\end{pgfscope}%
\begin{pgfscope}%
\pgfsetrectcap%
\pgfsetmiterjoin%
\pgfsetlinewidth{0.803000pt}%
\definecolor{currentstroke}{rgb}{0.000000,0.000000,0.000000}%
\pgfsetstrokecolor{currentstroke}%
\pgfsetdash{}{0pt}%
\pgfpathmoveto{\pgfqpoint{2.243603in}{0.862810in}}%
\pgfpathlineto{\pgfqpoint{2.243603in}{1.781780in}}%
\pgfusepath{stroke}%
\end{pgfscope}%
\begin{pgfscope}%
\pgfsetrectcap%
\pgfsetmiterjoin%
\pgfsetlinewidth{0.803000pt}%
\definecolor{currentstroke}{rgb}{0.000000,0.000000,0.000000}%
\pgfsetstrokecolor{currentstroke}%
\pgfsetdash{}{0pt}%
\pgfpathmoveto{\pgfqpoint{3.649022in}{0.862810in}}%
\pgfpathlineto{\pgfqpoint{3.649022in}{1.781780in}}%
\pgfusepath{stroke}%
\end{pgfscope}%
\begin{pgfscope}%
\pgfsetrectcap%
\pgfsetmiterjoin%
\pgfsetlinewidth{0.803000pt}%
\definecolor{currentstroke}{rgb}{0.000000,0.000000,0.000000}%
\pgfsetstrokecolor{currentstroke}%
\pgfsetdash{}{0pt}%
\pgfpathmoveto{\pgfqpoint{2.243603in}{0.862810in}}%
\pgfpathlineto{\pgfqpoint{3.649022in}{0.862810in}}%
\pgfusepath{stroke}%
\end{pgfscope}%
\begin{pgfscope}%
\pgfsetrectcap%
\pgfsetmiterjoin%
\pgfsetlinewidth{0.803000pt}%
\definecolor{currentstroke}{rgb}{0.000000,0.000000,0.000000}%
\pgfsetstrokecolor{currentstroke}%
\pgfsetdash{}{0pt}%
\pgfpathmoveto{\pgfqpoint{2.243603in}{1.781780in}}%
\pgfpathlineto{\pgfqpoint{3.649022in}{1.781780in}}%
\pgfusepath{stroke}%
\end{pgfscope}%
\begin{pgfscope}%
\definecolor{textcolor}{rgb}{0.000000,0.000000,0.000000}%
\pgfsetstrokecolor{textcolor}%
\pgfsetfillcolor{textcolor}%
\pgftext[x=2.946312in,y=1.815946in,,base]{\color{textcolor}\rmfamily\fontsize{12.000000}{14.400000}\selectfont Zoom with log-scale}%
\end{pgfscope}%
\end{pgfpicture}%
\makeatother%
\endgroup%

    \caption[Results of the one-dimensional moving parabola.]{Results of the one-dimensional moving parabola. \gls{ctvbo} results in lower regret and \gls{ui} forgetting shows lower sensitivity to a shifted mean. The formatting is as in Figure~\ref{fig:WMC_cumulative_regret_1D}.}
    \label{fig:Parabola1D_cumulative_regret}
\end{figure}
Again, it is apparent that taking into account the temporal change in the objective function is advisable in terms of regret as all variations outperform exploiting the posterior mean after the initialization.

As in previous experiments, \gls{b2p} forgetting shows a higher sensitivity to the optimistic mean with an increased mean regret supporting Hypothesis~\ref{hyp:ui_structural_information}. The variation with the lowest mean regret independent of the prior mean is the variation combining both proposed methods, \gls{uitvbo} and \gls{ctvbo}.

\newpage
\subsection{2-D Moving Parabola}
\label{sec:2D}

Based on similar concepts, the one-dimensional moving parabola is extended to two dimensions. The objective function for the two-dimensional moving parabola is
\begin{equation}
    f_t(\mathbf{x}=\{x_1,x_2\}) = \begin{cases}
        g_{\text{2D}}(\mathbf{x},t) \, ,& t < 140\\
        a_1 \left( (a_2 \cdot x_1 - 2a_2)^2 + (a_2 \cdot x_2 - 2a_2)^2 \right) +b \, ,&t \ge 140 \\
        \end{cases}
\end{equation}
with
\begin{align}
    g_{\text{2D}}(\mathbf{x}=\{x_1,x_2\},t) =& a_1 \left( (a_2 \cdot x_1)^2 + (a_2 \cdot x_2 - 0.5\sin(a_5\cdot t))^2 \right) \nonumber \\
    &+ 2(a_2 \cdot x_1 \sin(a_5\cdot t)) - \cos(a_5\cdot t)^2 +b.
\end{align}
Again, it consists of a part with gradual change for $t<140$ and a subsequent sudden change at $t=140$. The coefficients $a_1$ to $a_5$ and $b$ are the same as for the one-dimensional moving parabola in \Cref{sec:1D} and are displayed in Table~\ref{tab:coefficients_1D}. Since the factors of the objective function are identical, the temporal change is very similar. Therefore, the forgetting factors in Table~\ref{tab:1d} are used to evaluate the variations. The feasible set for the two-dimensional moving parabola was $\mathcal{X}=[-7, 7]^2$, Gamma hyperpriors were used for the length scales as $\boldsymbol\Lambda_{11}, \boldsymbol\Lambda_{22} \sim \mathcal{G}(15, \nicefrac{10}{3})$ \eqref{eq:gamma} with bounds $\boldsymbol\Lambda_{11},\boldsymbol\Lambda_{22} \in [2, 7]$, and the bounding functions were defined as $a(\mathbf{X}_v)=0$ and $b(\mathbf{X}_v)=4$. As in the one-dimensional moving parabola benchmark, scaling based on the initial set was applied. The results with firstly a well-defined prior mean and secondly an optimistic prior mean are shown in Figure~\ref{fig:Parabola2D_cumulative_regret}.
\begin{figure}[h]
    \centering
    %% Creator: Matplotlib, PGF backend
%%
%% To include the figure in your LaTeX document, write
%%   \input{<filename>.pgf}
%%
%% Make sure the required packages are loaded in your preamble
%%   \usepackage{pgf}
%%
%% Figures using additional raster images can only be included by \input if
%% they are in the same directory as the main LaTeX file. For loading figures
%% from other directories you can use the `import` package
%%   \usepackage{import}
%%
%% and then include the figures with
%%   \import{<path to file>}{<filename>.pgf}
%%
%% Matplotlib used the following preamble
%%   \usepackage{fontspec}
%%
\begingroup%
\makeatletter%
\begin{pgfpicture}%
\pgfpathrectangle{\pgfpointorigin}{\pgfqpoint{5.507126in}{2.552693in}}%
\pgfusepath{use as bounding box, clip}%
\begin{pgfscope}%
\pgfsetbuttcap%
\pgfsetmiterjoin%
\definecolor{currentfill}{rgb}{1.000000,1.000000,1.000000}%
\pgfsetfillcolor{currentfill}%
\pgfsetlinewidth{0.000000pt}%
\definecolor{currentstroke}{rgb}{1.000000,1.000000,1.000000}%
\pgfsetstrokecolor{currentstroke}%
\pgfsetdash{}{0pt}%
\pgfpathmoveto{\pgfqpoint{0.000000in}{0.000000in}}%
\pgfpathlineto{\pgfqpoint{5.507126in}{0.000000in}}%
\pgfpathlineto{\pgfqpoint{5.507126in}{2.552693in}}%
\pgfpathlineto{\pgfqpoint{0.000000in}{2.552693in}}%
\pgfpathclose%
\pgfusepath{fill}%
\end{pgfscope}%
\begin{pgfscope}%
\pgfsetbuttcap%
\pgfsetmiterjoin%
\definecolor{currentfill}{rgb}{1.000000,1.000000,1.000000}%
\pgfsetfillcolor{currentfill}%
\pgfsetlinewidth{0.000000pt}%
\definecolor{currentstroke}{rgb}{0.000000,0.000000,0.000000}%
\pgfsetstrokecolor{currentstroke}%
\pgfsetstrokeopacity{0.000000}%
\pgfsetdash{}{0pt}%
\pgfpathmoveto{\pgfqpoint{0.550713in}{1.728870in}}%
\pgfpathlineto{\pgfqpoint{3.744846in}{1.728870in}}%
\pgfpathlineto{\pgfqpoint{3.744846in}{2.425059in}}%
\pgfpathlineto{\pgfqpoint{0.550713in}{2.425059in}}%
\pgfpathclose%
\pgfusepath{fill}%
\end{pgfscope}%
\begin{pgfscope}%
\pgfpathrectangle{\pgfqpoint{0.550713in}{1.728870in}}{\pgfqpoint{3.194133in}{0.696189in}}%
\pgfusepath{clip}%
\pgfsetbuttcap%
\pgfsetmiterjoin%
\definecolor{currentfill}{rgb}{0.631373,0.062745,0.207843}%
\pgfsetfillcolor{currentfill}%
\pgfsetlinewidth{0.752812pt}%
\definecolor{currentstroke}{rgb}{0.000000,0.000000,0.000000}%
\pgfsetstrokecolor{currentstroke}%
\pgfsetdash{}{0pt}%
\pgfpathmoveto{\pgfqpoint{0.592236in}{0.866303in}}%
\pgfpathlineto{\pgfqpoint{0.748749in}{0.866303in}}%
\pgfpathlineto{\pgfqpoint{0.748749in}{1.019137in}}%
\pgfpathlineto{\pgfqpoint{0.592236in}{1.019137in}}%
\pgfpathlineto{\pgfqpoint{0.592236in}{0.866303in}}%
\pgfpathclose%
\pgfusepath{stroke,fill}%
\end{pgfscope}%
\begin{pgfscope}%
\pgfpathrectangle{\pgfqpoint{0.550713in}{1.728870in}}{\pgfqpoint{3.194133in}{0.696189in}}%
\pgfusepath{clip}%
\pgfsetbuttcap%
\pgfsetmiterjoin%
\definecolor{currentfill}{rgb}{0.898039,0.772549,0.752941}%
\pgfsetfillcolor{currentfill}%
\pgfsetlinewidth{0.752812pt}%
\definecolor{currentstroke}{rgb}{0.000000,0.000000,0.000000}%
\pgfsetstrokecolor{currentstroke}%
\pgfsetdash{}{0pt}%
\pgfpathmoveto{\pgfqpoint{0.751943in}{1.315571in}}%
\pgfpathlineto{\pgfqpoint{0.908456in}{1.315571in}}%
\pgfpathlineto{\pgfqpoint{0.908456in}{1.774677in}}%
\pgfpathlineto{\pgfqpoint{0.751943in}{1.774677in}}%
\pgfpathlineto{\pgfqpoint{0.751943in}{1.315571in}}%
\pgfpathclose%
\pgfusepath{stroke,fill}%
\end{pgfscope}%
\begin{pgfscope}%
\pgfpathrectangle{\pgfqpoint{0.550713in}{1.728870in}}{\pgfqpoint{3.194133in}{0.696189in}}%
\pgfusepath{clip}%
\pgfsetbuttcap%
\pgfsetmiterjoin%
\definecolor{currentfill}{rgb}{0.890196,0.000000,0.400000}%
\pgfsetfillcolor{currentfill}%
\pgfsetlinewidth{0.752812pt}%
\definecolor{currentstroke}{rgb}{0.000000,0.000000,0.000000}%
\pgfsetstrokecolor{currentstroke}%
\pgfsetdash{}{0pt}%
\pgfpathmoveto{\pgfqpoint{0.991503in}{0.883874in}}%
\pgfpathlineto{\pgfqpoint{1.148015in}{0.883874in}}%
\pgfpathlineto{\pgfqpoint{1.148015in}{1.299092in}}%
\pgfpathlineto{\pgfqpoint{0.991503in}{1.299092in}}%
\pgfpathlineto{\pgfqpoint{0.991503in}{0.883874in}}%
\pgfpathclose%
\pgfusepath{stroke,fill}%
\end{pgfscope}%
\begin{pgfscope}%
\pgfpathrectangle{\pgfqpoint{0.550713in}{1.728870in}}{\pgfqpoint{3.194133in}{0.696189in}}%
\pgfusepath{clip}%
\pgfsetbuttcap%
\pgfsetmiterjoin%
\definecolor{currentfill}{rgb}{0.976471,0.823529,0.854902}%
\pgfsetfillcolor{currentfill}%
\pgfsetlinewidth{0.752812pt}%
\definecolor{currentstroke}{rgb}{0.000000,0.000000,0.000000}%
\pgfsetstrokecolor{currentstroke}%
\pgfsetdash{}{0pt}%
\pgfpathmoveto{\pgfqpoint{1.151210in}{1.879743in}}%
\pgfpathlineto{\pgfqpoint{1.307722in}{1.879743in}}%
\pgfpathlineto{\pgfqpoint{1.307722in}{2.276778in}}%
\pgfpathlineto{\pgfqpoint{1.151210in}{2.276778in}}%
\pgfpathlineto{\pgfqpoint{1.151210in}{1.879743in}}%
\pgfpathclose%
\pgfusepath{stroke,fill}%
\end{pgfscope}%
\begin{pgfscope}%
\pgfpathrectangle{\pgfqpoint{0.550713in}{1.728870in}}{\pgfqpoint{3.194133in}{0.696189in}}%
\pgfusepath{clip}%
\pgfsetbuttcap%
\pgfsetmiterjoin%
\definecolor{currentfill}{rgb}{0.000000,0.329412,0.623529}%
\pgfsetfillcolor{currentfill}%
\pgfsetlinewidth{0.752812pt}%
\definecolor{currentstroke}{rgb}{0.000000,0.000000,0.000000}%
\pgfsetstrokecolor{currentstroke}%
\pgfsetdash{}{0pt}%
\pgfpathmoveto{\pgfqpoint{1.390770in}{0.868161in}}%
\pgfpathlineto{\pgfqpoint{1.547282in}{0.868161in}}%
\pgfpathlineto{\pgfqpoint{1.547282in}{1.011080in}}%
\pgfpathlineto{\pgfqpoint{1.390770in}{1.011080in}}%
\pgfpathlineto{\pgfqpoint{1.390770in}{0.868161in}}%
\pgfpathclose%
\pgfusepath{stroke,fill}%
\end{pgfscope}%
\begin{pgfscope}%
\pgfpathrectangle{\pgfqpoint{0.550713in}{1.728870in}}{\pgfqpoint{3.194133in}{0.696189in}}%
\pgfusepath{clip}%
\pgfsetbuttcap%
\pgfsetmiterjoin%
\definecolor{currentfill}{rgb}{0.780392,0.866667,0.949020}%
\pgfsetfillcolor{currentfill}%
\pgfsetlinewidth{0.752812pt}%
\definecolor{currentstroke}{rgb}{0.000000,0.000000,0.000000}%
\pgfsetstrokecolor{currentstroke}%
\pgfsetdash{}{0pt}%
\pgfpathmoveto{\pgfqpoint{1.550476in}{0.987155in}}%
\pgfpathlineto{\pgfqpoint{1.706989in}{0.987155in}}%
\pgfpathlineto{\pgfqpoint{1.706989in}{1.093815in}}%
\pgfpathlineto{\pgfqpoint{1.550476in}{1.093815in}}%
\pgfpathlineto{\pgfqpoint{1.550476in}{0.987155in}}%
\pgfpathclose%
\pgfusepath{stroke,fill}%
\end{pgfscope}%
\begin{pgfscope}%
\pgfpathrectangle{\pgfqpoint{0.550713in}{1.728870in}}{\pgfqpoint{3.194133in}{0.696189in}}%
\pgfusepath{clip}%
\pgfsetbuttcap%
\pgfsetmiterjoin%
\definecolor{currentfill}{rgb}{0.000000,0.380392,0.396078}%
\pgfsetfillcolor{currentfill}%
\pgfsetlinewidth{0.752812pt}%
\definecolor{currentstroke}{rgb}{0.000000,0.000000,0.000000}%
\pgfsetstrokecolor{currentstroke}%
\pgfsetdash{}{0pt}%
\pgfpathmoveto{\pgfqpoint{1.790036in}{1.021976in}}%
\pgfpathlineto{\pgfqpoint{1.946549in}{1.021976in}}%
\pgfpathlineto{\pgfqpoint{1.946549in}{1.171123in}}%
\pgfpathlineto{\pgfqpoint{1.790036in}{1.171123in}}%
\pgfpathlineto{\pgfqpoint{1.790036in}{1.021976in}}%
\pgfpathclose%
\pgfusepath{stroke,fill}%
\end{pgfscope}%
\begin{pgfscope}%
\pgfpathrectangle{\pgfqpoint{0.550713in}{1.728870in}}{\pgfqpoint{3.194133in}{0.696189in}}%
\pgfusepath{clip}%
\pgfsetbuttcap%
\pgfsetmiterjoin%
\definecolor{currentfill}{rgb}{0.749020,0.815686,0.819608}%
\pgfsetfillcolor{currentfill}%
\pgfsetlinewidth{0.752812pt}%
\definecolor{currentstroke}{rgb}{0.000000,0.000000,0.000000}%
\pgfsetstrokecolor{currentstroke}%
\pgfsetdash{}{0pt}%
\pgfpathmoveto{\pgfqpoint{1.949743in}{1.125204in}}%
\pgfpathlineto{\pgfqpoint{2.106255in}{1.125204in}}%
\pgfpathlineto{\pgfqpoint{2.106255in}{1.176903in}}%
\pgfpathlineto{\pgfqpoint{1.949743in}{1.176903in}}%
\pgfpathlineto{\pgfqpoint{1.949743in}{1.125204in}}%
\pgfpathclose%
\pgfusepath{stroke,fill}%
\end{pgfscope}%
\begin{pgfscope}%
\pgfpathrectangle{\pgfqpoint{0.550713in}{1.728870in}}{\pgfqpoint{3.194133in}{0.696189in}}%
\pgfusepath{clip}%
\pgfsetbuttcap%
\pgfsetmiterjoin%
\definecolor{currentfill}{rgb}{0.380392,0.129412,0.345098}%
\pgfsetfillcolor{currentfill}%
\pgfsetlinewidth{0.752812pt}%
\definecolor{currentstroke}{rgb}{0.000000,0.000000,0.000000}%
\pgfsetstrokecolor{currentstroke}%
\pgfsetdash{}{0pt}%
\pgfpathmoveto{\pgfqpoint{2.189303in}{0.809565in}}%
\pgfpathlineto{\pgfqpoint{2.345815in}{0.809565in}}%
\pgfpathlineto{\pgfqpoint{2.345815in}{0.932934in}}%
\pgfpathlineto{\pgfqpoint{2.189303in}{0.932934in}}%
\pgfpathlineto{\pgfqpoint{2.189303in}{0.809565in}}%
\pgfpathclose%
\pgfusepath{stroke,fill}%
\end{pgfscope}%
\begin{pgfscope}%
\pgfpathrectangle{\pgfqpoint{0.550713in}{1.728870in}}{\pgfqpoint{3.194133in}{0.696189in}}%
\pgfusepath{clip}%
\pgfsetbuttcap%
\pgfsetmiterjoin%
\definecolor{currentfill}{rgb}{0.823529,0.752941,0.803922}%
\pgfsetfillcolor{currentfill}%
\pgfsetlinewidth{0.752812pt}%
\definecolor{currentstroke}{rgb}{0.000000,0.000000,0.000000}%
\pgfsetstrokecolor{currentstroke}%
\pgfsetdash{}{0pt}%
\pgfpathmoveto{\pgfqpoint{2.349010in}{0.893819in}}%
\pgfpathlineto{\pgfqpoint{2.505522in}{0.893819in}}%
\pgfpathlineto{\pgfqpoint{2.505522in}{1.063238in}}%
\pgfpathlineto{\pgfqpoint{2.349010in}{1.063238in}}%
\pgfpathlineto{\pgfqpoint{2.349010in}{0.893819in}}%
\pgfpathclose%
\pgfusepath{stroke,fill}%
\end{pgfscope}%
\begin{pgfscope}%
\pgfpathrectangle{\pgfqpoint{0.550713in}{1.728870in}}{\pgfqpoint{3.194133in}{0.696189in}}%
\pgfusepath{clip}%
\pgfsetbuttcap%
\pgfsetmiterjoin%
\definecolor{currentfill}{rgb}{0.964706,0.658824,0.000000}%
\pgfsetfillcolor{currentfill}%
\pgfsetlinewidth{0.752812pt}%
\definecolor{currentstroke}{rgb}{0.000000,0.000000,0.000000}%
\pgfsetstrokecolor{currentstroke}%
\pgfsetdash{}{0pt}%
\pgfpathmoveto{\pgfqpoint{2.588570in}{0.873787in}}%
\pgfpathlineto{\pgfqpoint{2.745082in}{0.873787in}}%
\pgfpathlineto{\pgfqpoint{2.745082in}{0.993703in}}%
\pgfpathlineto{\pgfqpoint{2.588570in}{0.993703in}}%
\pgfpathlineto{\pgfqpoint{2.588570in}{0.873787in}}%
\pgfpathclose%
\pgfusepath{stroke,fill}%
\end{pgfscope}%
\begin{pgfscope}%
\pgfpathrectangle{\pgfqpoint{0.550713in}{1.728870in}}{\pgfqpoint{3.194133in}{0.696189in}}%
\pgfusepath{clip}%
\pgfsetbuttcap%
\pgfsetmiterjoin%
\definecolor{currentfill}{rgb}{0.996078,0.917647,0.788235}%
\pgfsetfillcolor{currentfill}%
\pgfsetlinewidth{0.752812pt}%
\definecolor{currentstroke}{rgb}{0.000000,0.000000,0.000000}%
\pgfsetstrokecolor{currentstroke}%
\pgfsetdash{}{0pt}%
\pgfpathmoveto{\pgfqpoint{2.748276in}{1.049740in}}%
\pgfpathlineto{\pgfqpoint{2.904789in}{1.049740in}}%
\pgfpathlineto{\pgfqpoint{2.904789in}{1.116699in}}%
\pgfpathlineto{\pgfqpoint{2.748276in}{1.116699in}}%
\pgfpathlineto{\pgfqpoint{2.748276in}{1.049740in}}%
\pgfpathclose%
\pgfusepath{stroke,fill}%
\end{pgfscope}%
\begin{pgfscope}%
\pgfpathrectangle{\pgfqpoint{0.550713in}{1.728870in}}{\pgfqpoint{3.194133in}{0.696189in}}%
\pgfusepath{clip}%
\pgfsetbuttcap%
\pgfsetmiterjoin%
\definecolor{currentfill}{rgb}{0.341176,0.670588,0.152941}%
\pgfsetfillcolor{currentfill}%
\pgfsetlinewidth{0.752812pt}%
\definecolor{currentstroke}{rgb}{0.000000,0.000000,0.000000}%
\pgfsetstrokecolor{currentstroke}%
\pgfsetdash{}{0pt}%
\pgfpathmoveto{\pgfqpoint{2.987836in}{0.817397in}}%
\pgfpathlineto{\pgfqpoint{3.144349in}{0.817397in}}%
\pgfpathlineto{\pgfqpoint{3.144349in}{0.887881in}}%
\pgfpathlineto{\pgfqpoint{2.987836in}{0.887881in}}%
\pgfpathlineto{\pgfqpoint{2.987836in}{0.817397in}}%
\pgfpathclose%
\pgfusepath{stroke,fill}%
\end{pgfscope}%
\begin{pgfscope}%
\pgfpathrectangle{\pgfqpoint{0.550713in}{1.728870in}}{\pgfqpoint{3.194133in}{0.696189in}}%
\pgfusepath{clip}%
\pgfsetbuttcap%
\pgfsetmiterjoin%
\definecolor{currentfill}{rgb}{0.866667,0.921569,0.807843}%
\pgfsetfillcolor{currentfill}%
\pgfsetlinewidth{0.752812pt}%
\definecolor{currentstroke}{rgb}{0.000000,0.000000,0.000000}%
\pgfsetstrokecolor{currentstroke}%
\pgfsetdash{}{0pt}%
\pgfpathmoveto{\pgfqpoint{3.147543in}{0.826225in}}%
\pgfpathlineto{\pgfqpoint{3.304055in}{0.826225in}}%
\pgfpathlineto{\pgfqpoint{3.304055in}{0.941735in}}%
\pgfpathlineto{\pgfqpoint{3.147543in}{0.941735in}}%
\pgfpathlineto{\pgfqpoint{3.147543in}{0.826225in}}%
\pgfpathclose%
\pgfusepath{stroke,fill}%
\end{pgfscope}%
\begin{pgfscope}%
\pgfpathrectangle{\pgfqpoint{0.550713in}{1.728870in}}{\pgfqpoint{3.194133in}{0.696189in}}%
\pgfusepath{clip}%
\pgfsetbuttcap%
\pgfsetmiterjoin%
\definecolor{currentfill}{rgb}{0.478431,0.435294,0.674510}%
\pgfsetfillcolor{currentfill}%
\pgfsetlinewidth{0.752812pt}%
\definecolor{currentstroke}{rgb}{0.000000,0.000000,0.000000}%
\pgfsetstrokecolor{currentstroke}%
\pgfsetdash{}{0pt}%
\pgfpathmoveto{\pgfqpoint{3.387103in}{0.921841in}}%
\pgfpathlineto{\pgfqpoint{3.543615in}{0.921841in}}%
\pgfpathlineto{\pgfqpoint{3.543615in}{0.987409in}}%
\pgfpathlineto{\pgfqpoint{3.387103in}{0.987409in}}%
\pgfpathlineto{\pgfqpoint{3.387103in}{0.921841in}}%
\pgfpathclose%
\pgfusepath{stroke,fill}%
\end{pgfscope}%
\begin{pgfscope}%
\pgfpathrectangle{\pgfqpoint{0.550713in}{1.728870in}}{\pgfqpoint{3.194133in}{0.696189in}}%
\pgfusepath{clip}%
\pgfsetbuttcap%
\pgfsetmiterjoin%
\definecolor{currentfill}{rgb}{0.870588,0.854902,0.921569}%
\pgfsetfillcolor{currentfill}%
\pgfsetlinewidth{0.752812pt}%
\definecolor{currentstroke}{rgb}{0.000000,0.000000,0.000000}%
\pgfsetstrokecolor{currentstroke}%
\pgfsetdash{}{0pt}%
\pgfpathmoveto{\pgfqpoint{3.546809in}{0.872796in}}%
\pgfpathlineto{\pgfqpoint{3.703322in}{0.872796in}}%
\pgfpathlineto{\pgfqpoint{3.703322in}{0.947884in}}%
\pgfpathlineto{\pgfqpoint{3.546809in}{0.947884in}}%
\pgfpathlineto{\pgfqpoint{3.546809in}{0.872796in}}%
\pgfpathclose%
\pgfusepath{stroke,fill}%
\end{pgfscope}%
\begin{pgfscope}%
\pgfpathrectangle{\pgfqpoint{0.550713in}{1.728870in}}{\pgfqpoint{3.194133in}{0.696189in}}%
\pgfusepath{clip}%
\pgfsetbuttcap%
\pgfsetmiterjoin%
\definecolor{currentfill}{rgb}{0.000000,0.000000,0.000000}%
\pgfsetfillcolor{currentfill}%
\pgfsetlinewidth{0.376406pt}%
\definecolor{currentstroke}{rgb}{0.000000,0.000000,0.000000}%
\pgfsetstrokecolor{currentstroke}%
\pgfsetdash{}{0pt}%
\pgfpathmoveto{\pgfqpoint{0.750346in}{0.510539in}}%
\pgfpathlineto{\pgfqpoint{0.750346in}{0.510539in}}%
\pgfpathlineto{\pgfqpoint{0.750346in}{0.510539in}}%
\pgfpathlineto{\pgfqpoint{0.750346in}{0.510539in}}%
\pgfpathclose%
\pgfusepath{stroke,fill}%
\end{pgfscope}%
\begin{pgfscope}%
\pgfpathrectangle{\pgfqpoint{0.550713in}{1.728870in}}{\pgfqpoint{3.194133in}{0.696189in}}%
\pgfusepath{clip}%
\pgfsetbuttcap%
\pgfsetmiterjoin%
\definecolor{currentfill}{rgb}{0.813235,0.819118,0.822059}%
\pgfsetfillcolor{currentfill}%
\pgfsetlinewidth{0.376406pt}%
\definecolor{currentstroke}{rgb}{0.000000,0.000000,0.000000}%
\pgfsetstrokecolor{currentstroke}%
\pgfsetdash{}{0pt}%
\pgfpathmoveto{\pgfqpoint{0.750346in}{0.510539in}}%
\pgfpathlineto{\pgfqpoint{0.750346in}{0.510539in}}%
\pgfpathlineto{\pgfqpoint{0.750346in}{0.510539in}}%
\pgfpathlineto{\pgfqpoint{0.750346in}{0.510539in}}%
\pgfpathclose%
\pgfusepath{stroke,fill}%
\end{pgfscope}%
\begin{pgfscope}%
\pgfsetbuttcap%
\pgfsetroundjoin%
\definecolor{currentfill}{rgb}{0.000000,0.000000,0.000000}%
\pgfsetfillcolor{currentfill}%
\pgfsetlinewidth{0.803000pt}%
\definecolor{currentstroke}{rgb}{0.000000,0.000000,0.000000}%
\pgfsetstrokecolor{currentstroke}%
\pgfsetdash{}{0pt}%
\pgfsys@defobject{currentmarker}{\pgfqpoint{-0.048611in}{0.000000in}}{\pgfqpoint{-0.000000in}{0.000000in}}{%
\pgfpathmoveto{\pgfqpoint{-0.000000in}{0.000000in}}%
\pgfpathlineto{\pgfqpoint{-0.048611in}{0.000000in}}%
\pgfusepath{stroke,fill}%
}%
\begin{pgfscope}%
\pgfsys@transformshift{0.550713in}{1.902917in}%
\pgfsys@useobject{currentmarker}{}%
\end{pgfscope}%
\end{pgfscope}%
\begin{pgfscope}%
\definecolor{textcolor}{rgb}{0.000000,0.000000,0.000000}%
\pgfsetstrokecolor{textcolor}%
\pgfsetfillcolor{textcolor}%
\pgftext[x=0.245156in, y=1.854722in, left, base]{\color{textcolor}\rmfamily\fontsize{10.000000}{12.000000}\selectfont \(\displaystyle {400}\)}%
\end{pgfscope}%
\begin{pgfscope}%
\pgfsetbuttcap%
\pgfsetroundjoin%
\definecolor{currentfill}{rgb}{0.000000,0.000000,0.000000}%
\pgfsetfillcolor{currentfill}%
\pgfsetlinewidth{0.803000pt}%
\definecolor{currentstroke}{rgb}{0.000000,0.000000,0.000000}%
\pgfsetstrokecolor{currentstroke}%
\pgfsetdash{}{0pt}%
\pgfsys@defobject{currentmarker}{\pgfqpoint{-0.048611in}{0.000000in}}{\pgfqpoint{-0.000000in}{0.000000in}}{%
\pgfpathmoveto{\pgfqpoint{-0.000000in}{0.000000in}}%
\pgfpathlineto{\pgfqpoint{-0.048611in}{0.000000in}}%
\pgfusepath{stroke,fill}%
}%
\begin{pgfscope}%
\pgfsys@transformshift{0.550713in}{2.251011in}%
\pgfsys@useobject{currentmarker}{}%
\end{pgfscope}%
\end{pgfscope}%
\begin{pgfscope}%
\definecolor{textcolor}{rgb}{0.000000,0.000000,0.000000}%
\pgfsetstrokecolor{textcolor}%
\pgfsetfillcolor{textcolor}%
\pgftext[x=0.245156in, y=2.202817in, left, base]{\color{textcolor}\rmfamily\fontsize{10.000000}{12.000000}\selectfont \(\displaystyle {500}\)}%
\end{pgfscope}%
\begin{pgfscope}%
\pgfpathrectangle{\pgfqpoint{0.550713in}{1.728870in}}{\pgfqpoint{3.194133in}{0.696189in}}%
\pgfusepath{clip}%
\pgfsetbuttcap%
\pgfsetroundjoin%
\pgfsetlinewidth{0.501875pt}%
\definecolor{currentstroke}{rgb}{0.392157,0.396078,0.403922}%
\pgfsetstrokecolor{currentstroke}%
\pgfsetdash{}{0pt}%
\pgfpathmoveto{\pgfqpoint{0.949979in}{1.728870in}}%
\pgfpathlineto{\pgfqpoint{0.949979in}{2.425059in}}%
\pgfusepath{stroke}%
\end{pgfscope}%
\begin{pgfscope}%
\pgfpathrectangle{\pgfqpoint{0.550713in}{1.728870in}}{\pgfqpoint{3.194133in}{0.696189in}}%
\pgfusepath{clip}%
\pgfsetbuttcap%
\pgfsetroundjoin%
\pgfsetlinewidth{0.501875pt}%
\definecolor{currentstroke}{rgb}{0.392157,0.396078,0.403922}%
\pgfsetstrokecolor{currentstroke}%
\pgfsetdash{}{0pt}%
\pgfpathmoveto{\pgfqpoint{1.349246in}{1.728870in}}%
\pgfpathlineto{\pgfqpoint{1.349246in}{2.425059in}}%
\pgfusepath{stroke}%
\end{pgfscope}%
\begin{pgfscope}%
\pgfpathrectangle{\pgfqpoint{0.550713in}{1.728870in}}{\pgfqpoint{3.194133in}{0.696189in}}%
\pgfusepath{clip}%
\pgfsetbuttcap%
\pgfsetroundjoin%
\pgfsetlinewidth{0.501875pt}%
\definecolor{currentstroke}{rgb}{0.392157,0.396078,0.403922}%
\pgfsetstrokecolor{currentstroke}%
\pgfsetdash{}{0pt}%
\pgfpathmoveto{\pgfqpoint{1.748513in}{1.728870in}}%
\pgfpathlineto{\pgfqpoint{1.748513in}{2.425059in}}%
\pgfusepath{stroke}%
\end{pgfscope}%
\begin{pgfscope}%
\pgfpathrectangle{\pgfqpoint{0.550713in}{1.728870in}}{\pgfqpoint{3.194133in}{0.696189in}}%
\pgfusepath{clip}%
\pgfsetbuttcap%
\pgfsetroundjoin%
\pgfsetlinewidth{0.501875pt}%
\definecolor{currentstroke}{rgb}{0.392157,0.396078,0.403922}%
\pgfsetstrokecolor{currentstroke}%
\pgfsetdash{}{0pt}%
\pgfpathmoveto{\pgfqpoint{2.147779in}{1.728870in}}%
\pgfpathlineto{\pgfqpoint{2.147779in}{2.425059in}}%
\pgfusepath{stroke}%
\end{pgfscope}%
\begin{pgfscope}%
\pgfpathrectangle{\pgfqpoint{0.550713in}{1.728870in}}{\pgfqpoint{3.194133in}{0.696189in}}%
\pgfusepath{clip}%
\pgfsetbuttcap%
\pgfsetroundjoin%
\pgfsetlinewidth{0.501875pt}%
\definecolor{currentstroke}{rgb}{0.392157,0.396078,0.403922}%
\pgfsetstrokecolor{currentstroke}%
\pgfsetdash{}{0pt}%
\pgfpathmoveto{\pgfqpoint{2.547046in}{1.728870in}}%
\pgfpathlineto{\pgfqpoint{2.547046in}{2.425059in}}%
\pgfusepath{stroke}%
\end{pgfscope}%
\begin{pgfscope}%
\pgfpathrectangle{\pgfqpoint{0.550713in}{1.728870in}}{\pgfqpoint{3.194133in}{0.696189in}}%
\pgfusepath{clip}%
\pgfsetbuttcap%
\pgfsetroundjoin%
\pgfsetlinewidth{0.501875pt}%
\definecolor{currentstroke}{rgb}{0.392157,0.396078,0.403922}%
\pgfsetstrokecolor{currentstroke}%
\pgfsetdash{}{0pt}%
\pgfpathmoveto{\pgfqpoint{2.946312in}{1.728870in}}%
\pgfpathlineto{\pgfqpoint{2.946312in}{2.425059in}}%
\pgfusepath{stroke}%
\end{pgfscope}%
\begin{pgfscope}%
\pgfpathrectangle{\pgfqpoint{0.550713in}{1.728870in}}{\pgfqpoint{3.194133in}{0.696189in}}%
\pgfusepath{clip}%
\pgfsetbuttcap%
\pgfsetroundjoin%
\pgfsetlinewidth{0.501875pt}%
\definecolor{currentstroke}{rgb}{0.392157,0.396078,0.403922}%
\pgfsetstrokecolor{currentstroke}%
\pgfsetdash{}{0pt}%
\pgfpathmoveto{\pgfqpoint{3.345579in}{1.728870in}}%
\pgfpathlineto{\pgfqpoint{3.345579in}{2.425059in}}%
\pgfusepath{stroke}%
\end{pgfscope}%
\begin{pgfscope}%
\pgfpathrectangle{\pgfqpoint{0.550713in}{1.728870in}}{\pgfqpoint{3.194133in}{0.696189in}}%
\pgfusepath{clip}%
\pgfsetbuttcap%
\pgfsetroundjoin%
\pgfsetlinewidth{0.853187pt}%
\definecolor{currentstroke}{rgb}{0.392157,0.396078,0.403922}%
\pgfsetstrokecolor{currentstroke}%
\pgfsetdash{{3.145000pt}{1.360000pt}}{0.000000pt}%
\pgfpathmoveto{\pgfqpoint{0.540713in}{2.207125in}}%
\pgfpathlineto{\pgfqpoint{3.754846in}{2.207125in}}%
\pgfusepath{stroke}%
\end{pgfscope}%
\begin{pgfscope}%
\pgfpathrectangle{\pgfqpoint{0.550713in}{1.728870in}}{\pgfqpoint{3.194133in}{0.696189in}}%
\pgfusepath{clip}%
\pgfsetrectcap%
\pgfsetroundjoin%
\pgfsetlinewidth{0.752812pt}%
\definecolor{currentstroke}{rgb}{0.000000,0.000000,0.000000}%
\pgfsetstrokecolor{currentstroke}%
\pgfsetdash{}{0pt}%
\pgfusepath{stroke}%
\end{pgfscope}%
\begin{pgfscope}%
\pgfpathrectangle{\pgfqpoint{0.550713in}{1.728870in}}{\pgfqpoint{3.194133in}{0.696189in}}%
\pgfusepath{clip}%
\pgfsetrectcap%
\pgfsetroundjoin%
\pgfsetlinewidth{0.752812pt}%
\definecolor{currentstroke}{rgb}{0.000000,0.000000,0.000000}%
\pgfsetstrokecolor{currentstroke}%
\pgfsetdash{}{0pt}%
\pgfusepath{stroke}%
\end{pgfscope}%
\begin{pgfscope}%
\pgfpathrectangle{\pgfqpoint{0.550713in}{1.728870in}}{\pgfqpoint{3.194133in}{0.696189in}}%
\pgfusepath{clip}%
\pgfsetrectcap%
\pgfsetroundjoin%
\pgfsetlinewidth{0.752812pt}%
\definecolor{currentstroke}{rgb}{0.000000,0.000000,0.000000}%
\pgfsetstrokecolor{currentstroke}%
\pgfsetdash{}{0pt}%
\pgfusepath{stroke}%
\end{pgfscope}%
\begin{pgfscope}%
\pgfpathrectangle{\pgfqpoint{0.550713in}{1.728870in}}{\pgfqpoint{3.194133in}{0.696189in}}%
\pgfusepath{clip}%
\pgfsetrectcap%
\pgfsetroundjoin%
\pgfsetlinewidth{0.752812pt}%
\definecolor{currentstroke}{rgb}{0.000000,0.000000,0.000000}%
\pgfsetstrokecolor{currentstroke}%
\pgfsetdash{}{0pt}%
\pgfusepath{stroke}%
\end{pgfscope}%
\begin{pgfscope}%
\pgfpathrectangle{\pgfqpoint{0.550713in}{1.728870in}}{\pgfqpoint{3.194133in}{0.696189in}}%
\pgfusepath{clip}%
\pgfsetrectcap%
\pgfsetroundjoin%
\pgfsetlinewidth{0.752812pt}%
\definecolor{currentstroke}{rgb}{0.000000,0.000000,0.000000}%
\pgfsetstrokecolor{currentstroke}%
\pgfsetdash{}{0pt}%
\pgfusepath{stroke}%
\end{pgfscope}%
\begin{pgfscope}%
\pgfpathrectangle{\pgfqpoint{0.550713in}{1.728870in}}{\pgfqpoint{3.194133in}{0.696189in}}%
\pgfusepath{clip}%
\pgfsetrectcap%
\pgfsetroundjoin%
\pgfsetlinewidth{0.752812pt}%
\definecolor{currentstroke}{rgb}{0.000000,0.000000,0.000000}%
\pgfsetstrokecolor{currentstroke}%
\pgfsetdash{}{0pt}%
\pgfpathmoveto{\pgfqpoint{0.830199in}{1.774677in}}%
\pgfpathlineto{\pgfqpoint{0.830199in}{1.829248in}}%
\pgfusepath{stroke}%
\end{pgfscope}%
\begin{pgfscope}%
\pgfpathrectangle{\pgfqpoint{0.550713in}{1.728870in}}{\pgfqpoint{3.194133in}{0.696189in}}%
\pgfusepath{clip}%
\pgfsetrectcap%
\pgfsetroundjoin%
\pgfsetlinewidth{0.752812pt}%
\definecolor{currentstroke}{rgb}{0.000000,0.000000,0.000000}%
\pgfsetstrokecolor{currentstroke}%
\pgfsetdash{}{0pt}%
\pgfusepath{stroke}%
\end{pgfscope}%
\begin{pgfscope}%
\pgfpathrectangle{\pgfqpoint{0.550713in}{1.728870in}}{\pgfqpoint{3.194133in}{0.696189in}}%
\pgfusepath{clip}%
\pgfsetrectcap%
\pgfsetroundjoin%
\pgfsetlinewidth{0.752812pt}%
\definecolor{currentstroke}{rgb}{0.000000,0.000000,0.000000}%
\pgfsetstrokecolor{currentstroke}%
\pgfsetdash{}{0pt}%
\pgfpathmoveto{\pgfqpoint{0.791071in}{1.829248in}}%
\pgfpathlineto{\pgfqpoint{0.869327in}{1.829248in}}%
\pgfusepath{stroke}%
\end{pgfscope}%
\begin{pgfscope}%
\pgfpathrectangle{\pgfqpoint{0.550713in}{1.728870in}}{\pgfqpoint{3.194133in}{0.696189in}}%
\pgfusepath{clip}%
\pgfsetrectcap%
\pgfsetroundjoin%
\pgfsetlinewidth{0.752812pt}%
\definecolor{currentstroke}{rgb}{0.000000,0.000000,0.000000}%
\pgfsetstrokecolor{currentstroke}%
\pgfsetdash{}{0pt}%
\pgfusepath{stroke}%
\end{pgfscope}%
\begin{pgfscope}%
\pgfpathrectangle{\pgfqpoint{0.550713in}{1.728870in}}{\pgfqpoint{3.194133in}{0.696189in}}%
\pgfusepath{clip}%
\pgfsetrectcap%
\pgfsetroundjoin%
\pgfsetlinewidth{0.752812pt}%
\definecolor{currentstroke}{rgb}{0.000000,0.000000,0.000000}%
\pgfsetstrokecolor{currentstroke}%
\pgfsetdash{}{0pt}%
\pgfusepath{stroke}%
\end{pgfscope}%
\begin{pgfscope}%
\pgfpathrectangle{\pgfqpoint{0.550713in}{1.728870in}}{\pgfqpoint{3.194133in}{0.696189in}}%
\pgfusepath{clip}%
\pgfsetrectcap%
\pgfsetroundjoin%
\pgfsetlinewidth{0.752812pt}%
\definecolor{currentstroke}{rgb}{0.000000,0.000000,0.000000}%
\pgfsetstrokecolor{currentstroke}%
\pgfsetdash{}{0pt}%
\pgfusepath{stroke}%
\end{pgfscope}%
\begin{pgfscope}%
\pgfpathrectangle{\pgfqpoint{0.550713in}{1.728870in}}{\pgfqpoint{3.194133in}{0.696189in}}%
\pgfusepath{clip}%
\pgfsetrectcap%
\pgfsetroundjoin%
\pgfsetlinewidth{0.752812pt}%
\definecolor{currentstroke}{rgb}{0.000000,0.000000,0.000000}%
\pgfsetstrokecolor{currentstroke}%
\pgfsetdash{}{0pt}%
\pgfusepath{stroke}%
\end{pgfscope}%
\begin{pgfscope}%
\pgfpathrectangle{\pgfqpoint{0.550713in}{1.728870in}}{\pgfqpoint{3.194133in}{0.696189in}}%
\pgfusepath{clip}%
\pgfsetrectcap%
\pgfsetroundjoin%
\pgfsetlinewidth{0.752812pt}%
\definecolor{currentstroke}{rgb}{0.000000,0.000000,0.000000}%
\pgfsetstrokecolor{currentstroke}%
\pgfsetdash{}{0pt}%
\pgfpathmoveto{\pgfqpoint{1.229466in}{1.879743in}}%
\pgfpathlineto{\pgfqpoint{1.229466in}{1.718870in}}%
\pgfusepath{stroke}%
\end{pgfscope}%
\begin{pgfscope}%
\pgfpathrectangle{\pgfqpoint{0.550713in}{1.728870in}}{\pgfqpoint{3.194133in}{0.696189in}}%
\pgfusepath{clip}%
\pgfsetrectcap%
\pgfsetroundjoin%
\pgfsetlinewidth{0.752812pt}%
\definecolor{currentstroke}{rgb}{0.000000,0.000000,0.000000}%
\pgfsetstrokecolor{currentstroke}%
\pgfsetdash{}{0pt}%
\pgfpathmoveto{\pgfqpoint{1.229466in}{2.276778in}}%
\pgfpathlineto{\pgfqpoint{1.229466in}{2.297115in}}%
\pgfusepath{stroke}%
\end{pgfscope}%
\begin{pgfscope}%
\pgfpathrectangle{\pgfqpoint{0.550713in}{1.728870in}}{\pgfqpoint{3.194133in}{0.696189in}}%
\pgfusepath{clip}%
\pgfsetrectcap%
\pgfsetroundjoin%
\pgfsetlinewidth{0.752812pt}%
\definecolor{currentstroke}{rgb}{0.000000,0.000000,0.000000}%
\pgfsetstrokecolor{currentstroke}%
\pgfsetdash{}{0pt}%
\pgfusepath{stroke}%
\end{pgfscope}%
\begin{pgfscope}%
\pgfpathrectangle{\pgfqpoint{0.550713in}{1.728870in}}{\pgfqpoint{3.194133in}{0.696189in}}%
\pgfusepath{clip}%
\pgfsetrectcap%
\pgfsetroundjoin%
\pgfsetlinewidth{0.752812pt}%
\definecolor{currentstroke}{rgb}{0.000000,0.000000,0.000000}%
\pgfsetstrokecolor{currentstroke}%
\pgfsetdash{}{0pt}%
\pgfpathmoveto{\pgfqpoint{1.190338in}{2.297115in}}%
\pgfpathlineto{\pgfqpoint{1.268594in}{2.297115in}}%
\pgfusepath{stroke}%
\end{pgfscope}%
\begin{pgfscope}%
\pgfpathrectangle{\pgfqpoint{0.550713in}{1.728870in}}{\pgfqpoint{3.194133in}{0.696189in}}%
\pgfusepath{clip}%
\pgfsetrectcap%
\pgfsetroundjoin%
\pgfsetlinewidth{0.752812pt}%
\definecolor{currentstroke}{rgb}{0.000000,0.000000,0.000000}%
\pgfsetstrokecolor{currentstroke}%
\pgfsetdash{}{0pt}%
\pgfusepath{stroke}%
\end{pgfscope}%
\begin{pgfscope}%
\pgfpathrectangle{\pgfqpoint{0.550713in}{1.728870in}}{\pgfqpoint{3.194133in}{0.696189in}}%
\pgfusepath{clip}%
\pgfsetrectcap%
\pgfsetroundjoin%
\pgfsetlinewidth{0.752812pt}%
\definecolor{currentstroke}{rgb}{0.000000,0.000000,0.000000}%
\pgfsetstrokecolor{currentstroke}%
\pgfsetdash{}{0pt}%
\pgfusepath{stroke}%
\end{pgfscope}%
\begin{pgfscope}%
\pgfpathrectangle{\pgfqpoint{0.550713in}{1.728870in}}{\pgfqpoint{3.194133in}{0.696189in}}%
\pgfusepath{clip}%
\pgfsetrectcap%
\pgfsetroundjoin%
\pgfsetlinewidth{0.752812pt}%
\definecolor{currentstroke}{rgb}{0.000000,0.000000,0.000000}%
\pgfsetstrokecolor{currentstroke}%
\pgfsetdash{}{0pt}%
\pgfusepath{stroke}%
\end{pgfscope}%
\begin{pgfscope}%
\pgfpathrectangle{\pgfqpoint{0.550713in}{1.728870in}}{\pgfqpoint{3.194133in}{0.696189in}}%
\pgfusepath{clip}%
\pgfsetrectcap%
\pgfsetroundjoin%
\pgfsetlinewidth{0.752812pt}%
\definecolor{currentstroke}{rgb}{0.000000,0.000000,0.000000}%
\pgfsetstrokecolor{currentstroke}%
\pgfsetdash{}{0pt}%
\pgfusepath{stroke}%
\end{pgfscope}%
\begin{pgfscope}%
\pgfpathrectangle{\pgfqpoint{0.550713in}{1.728870in}}{\pgfqpoint{3.194133in}{0.696189in}}%
\pgfusepath{clip}%
\pgfsetrectcap%
\pgfsetroundjoin%
\pgfsetlinewidth{0.752812pt}%
\definecolor{currentstroke}{rgb}{0.000000,0.000000,0.000000}%
\pgfsetstrokecolor{currentstroke}%
\pgfsetdash{}{0pt}%
\pgfusepath{stroke}%
\end{pgfscope}%
\begin{pgfscope}%
\pgfpathrectangle{\pgfqpoint{0.550713in}{1.728870in}}{\pgfqpoint{3.194133in}{0.696189in}}%
\pgfusepath{clip}%
\pgfsetrectcap%
\pgfsetroundjoin%
\pgfsetlinewidth{0.752812pt}%
\definecolor{currentstroke}{rgb}{0.000000,0.000000,0.000000}%
\pgfsetstrokecolor{currentstroke}%
\pgfsetdash{}{0pt}%
\pgfusepath{stroke}%
\end{pgfscope}%
\begin{pgfscope}%
\pgfpathrectangle{\pgfqpoint{0.550713in}{1.728870in}}{\pgfqpoint{3.194133in}{0.696189in}}%
\pgfusepath{clip}%
\pgfsetrectcap%
\pgfsetroundjoin%
\pgfsetlinewidth{0.752812pt}%
\definecolor{currentstroke}{rgb}{0.000000,0.000000,0.000000}%
\pgfsetstrokecolor{currentstroke}%
\pgfsetdash{}{0pt}%
\pgfusepath{stroke}%
\end{pgfscope}%
\begin{pgfscope}%
\pgfpathrectangle{\pgfqpoint{0.550713in}{1.728870in}}{\pgfqpoint{3.194133in}{0.696189in}}%
\pgfusepath{clip}%
\pgfsetrectcap%
\pgfsetroundjoin%
\pgfsetlinewidth{0.752812pt}%
\definecolor{currentstroke}{rgb}{0.000000,0.000000,0.000000}%
\pgfsetstrokecolor{currentstroke}%
\pgfsetdash{}{0pt}%
\pgfusepath{stroke}%
\end{pgfscope}%
\begin{pgfscope}%
\pgfpathrectangle{\pgfqpoint{0.550713in}{1.728870in}}{\pgfqpoint{3.194133in}{0.696189in}}%
\pgfusepath{clip}%
\pgfsetrectcap%
\pgfsetroundjoin%
\pgfsetlinewidth{0.752812pt}%
\definecolor{currentstroke}{rgb}{0.000000,0.000000,0.000000}%
\pgfsetstrokecolor{currentstroke}%
\pgfsetdash{}{0pt}%
\pgfusepath{stroke}%
\end{pgfscope}%
\begin{pgfscope}%
\pgfpathrectangle{\pgfqpoint{0.550713in}{1.728870in}}{\pgfqpoint{3.194133in}{0.696189in}}%
\pgfusepath{clip}%
\pgfsetrectcap%
\pgfsetroundjoin%
\pgfsetlinewidth{0.752812pt}%
\definecolor{currentstroke}{rgb}{0.000000,0.000000,0.000000}%
\pgfsetstrokecolor{currentstroke}%
\pgfsetdash{}{0pt}%
\pgfusepath{stroke}%
\end{pgfscope}%
\begin{pgfscope}%
\pgfpathrectangle{\pgfqpoint{0.550713in}{1.728870in}}{\pgfqpoint{3.194133in}{0.696189in}}%
\pgfusepath{clip}%
\pgfsetrectcap%
\pgfsetroundjoin%
\pgfsetlinewidth{0.752812pt}%
\definecolor{currentstroke}{rgb}{0.000000,0.000000,0.000000}%
\pgfsetstrokecolor{currentstroke}%
\pgfsetdash{}{0pt}%
\pgfusepath{stroke}%
\end{pgfscope}%
\begin{pgfscope}%
\pgfpathrectangle{\pgfqpoint{0.550713in}{1.728870in}}{\pgfqpoint{3.194133in}{0.696189in}}%
\pgfusepath{clip}%
\pgfsetrectcap%
\pgfsetroundjoin%
\pgfsetlinewidth{0.752812pt}%
\definecolor{currentstroke}{rgb}{0.000000,0.000000,0.000000}%
\pgfsetstrokecolor{currentstroke}%
\pgfsetdash{}{0pt}%
\pgfusepath{stroke}%
\end{pgfscope}%
\begin{pgfscope}%
\pgfpathrectangle{\pgfqpoint{0.550713in}{1.728870in}}{\pgfqpoint{3.194133in}{0.696189in}}%
\pgfusepath{clip}%
\pgfsetrectcap%
\pgfsetroundjoin%
\pgfsetlinewidth{0.752812pt}%
\definecolor{currentstroke}{rgb}{0.000000,0.000000,0.000000}%
\pgfsetstrokecolor{currentstroke}%
\pgfsetdash{}{0pt}%
\pgfusepath{stroke}%
\end{pgfscope}%
\begin{pgfscope}%
\pgfpathrectangle{\pgfqpoint{0.550713in}{1.728870in}}{\pgfqpoint{3.194133in}{0.696189in}}%
\pgfusepath{clip}%
\pgfsetrectcap%
\pgfsetroundjoin%
\pgfsetlinewidth{0.752812pt}%
\definecolor{currentstroke}{rgb}{0.000000,0.000000,0.000000}%
\pgfsetstrokecolor{currentstroke}%
\pgfsetdash{}{0pt}%
\pgfusepath{stroke}%
\end{pgfscope}%
\begin{pgfscope}%
\pgfpathrectangle{\pgfqpoint{0.550713in}{1.728870in}}{\pgfqpoint{3.194133in}{0.696189in}}%
\pgfusepath{clip}%
\pgfsetrectcap%
\pgfsetroundjoin%
\pgfsetlinewidth{0.752812pt}%
\definecolor{currentstroke}{rgb}{0.000000,0.000000,0.000000}%
\pgfsetstrokecolor{currentstroke}%
\pgfsetdash{}{0pt}%
\pgfusepath{stroke}%
\end{pgfscope}%
\begin{pgfscope}%
\pgfpathrectangle{\pgfqpoint{0.550713in}{1.728870in}}{\pgfqpoint{3.194133in}{0.696189in}}%
\pgfusepath{clip}%
\pgfsetrectcap%
\pgfsetroundjoin%
\pgfsetlinewidth{0.752812pt}%
\definecolor{currentstroke}{rgb}{0.000000,0.000000,0.000000}%
\pgfsetstrokecolor{currentstroke}%
\pgfsetdash{}{0pt}%
\pgfusepath{stroke}%
\end{pgfscope}%
\begin{pgfscope}%
\pgfpathrectangle{\pgfqpoint{0.550713in}{1.728870in}}{\pgfqpoint{3.194133in}{0.696189in}}%
\pgfusepath{clip}%
\pgfsetbuttcap%
\pgfsetmiterjoin%
\definecolor{currentfill}{rgb}{0.000000,0.000000,0.000000}%
\pgfsetfillcolor{currentfill}%
\pgfsetlinewidth{1.003750pt}%
\definecolor{currentstroke}{rgb}{0.000000,0.000000,0.000000}%
\pgfsetstrokecolor{currentstroke}%
\pgfsetdash{}{0pt}%
\pgfsys@defobject{currentmarker}{\pgfqpoint{-0.011785in}{-0.019642in}}{\pgfqpoint{0.011785in}{0.019642in}}{%
\pgfpathmoveto{\pgfqpoint{-0.000000in}{-0.019642in}}%
\pgfpathlineto{\pgfqpoint{0.011785in}{0.000000in}}%
\pgfpathlineto{\pgfqpoint{0.000000in}{0.019642in}}%
\pgfpathlineto{\pgfqpoint{-0.011785in}{0.000000in}}%
\pgfpathclose%
\pgfusepath{stroke,fill}%
}%
\begin{pgfscope}%
\pgfsys@transformshift{2.027999in}{0.968950in}%
\pgfsys@useobject{currentmarker}{}%
\end{pgfscope}%
\begin{pgfscope}%
\pgfsys@transformshift{2.027999in}{1.318093in}%
\pgfsys@useobject{currentmarker}{}%
\end{pgfscope}%
\end{pgfscope}%
\begin{pgfscope}%
\pgfpathrectangle{\pgfqpoint{0.550713in}{1.728870in}}{\pgfqpoint{3.194133in}{0.696189in}}%
\pgfusepath{clip}%
\pgfsetrectcap%
\pgfsetroundjoin%
\pgfsetlinewidth{0.752812pt}%
\definecolor{currentstroke}{rgb}{0.000000,0.000000,0.000000}%
\pgfsetstrokecolor{currentstroke}%
\pgfsetdash{}{0pt}%
\pgfusepath{stroke}%
\end{pgfscope}%
\begin{pgfscope}%
\pgfpathrectangle{\pgfqpoint{0.550713in}{1.728870in}}{\pgfqpoint{3.194133in}{0.696189in}}%
\pgfusepath{clip}%
\pgfsetrectcap%
\pgfsetroundjoin%
\pgfsetlinewidth{0.752812pt}%
\definecolor{currentstroke}{rgb}{0.000000,0.000000,0.000000}%
\pgfsetstrokecolor{currentstroke}%
\pgfsetdash{}{0pt}%
\pgfusepath{stroke}%
\end{pgfscope}%
\begin{pgfscope}%
\pgfpathrectangle{\pgfqpoint{0.550713in}{1.728870in}}{\pgfqpoint{3.194133in}{0.696189in}}%
\pgfusepath{clip}%
\pgfsetrectcap%
\pgfsetroundjoin%
\pgfsetlinewidth{0.752812pt}%
\definecolor{currentstroke}{rgb}{0.000000,0.000000,0.000000}%
\pgfsetstrokecolor{currentstroke}%
\pgfsetdash{}{0pt}%
\pgfusepath{stroke}%
\end{pgfscope}%
\begin{pgfscope}%
\pgfpathrectangle{\pgfqpoint{0.550713in}{1.728870in}}{\pgfqpoint{3.194133in}{0.696189in}}%
\pgfusepath{clip}%
\pgfsetrectcap%
\pgfsetroundjoin%
\pgfsetlinewidth{0.752812pt}%
\definecolor{currentstroke}{rgb}{0.000000,0.000000,0.000000}%
\pgfsetstrokecolor{currentstroke}%
\pgfsetdash{}{0pt}%
\pgfusepath{stroke}%
\end{pgfscope}%
\begin{pgfscope}%
\pgfpathrectangle{\pgfqpoint{0.550713in}{1.728870in}}{\pgfqpoint{3.194133in}{0.696189in}}%
\pgfusepath{clip}%
\pgfsetrectcap%
\pgfsetroundjoin%
\pgfsetlinewidth{0.752812pt}%
\definecolor{currentstroke}{rgb}{0.000000,0.000000,0.000000}%
\pgfsetstrokecolor{currentstroke}%
\pgfsetdash{}{0pt}%
\pgfusepath{stroke}%
\end{pgfscope}%
\begin{pgfscope}%
\pgfpathrectangle{\pgfqpoint{0.550713in}{1.728870in}}{\pgfqpoint{3.194133in}{0.696189in}}%
\pgfusepath{clip}%
\pgfsetrectcap%
\pgfsetroundjoin%
\pgfsetlinewidth{0.752812pt}%
\definecolor{currentstroke}{rgb}{0.000000,0.000000,0.000000}%
\pgfsetstrokecolor{currentstroke}%
\pgfsetdash{}{0pt}%
\pgfusepath{stroke}%
\end{pgfscope}%
\begin{pgfscope}%
\pgfpathrectangle{\pgfqpoint{0.550713in}{1.728870in}}{\pgfqpoint{3.194133in}{0.696189in}}%
\pgfusepath{clip}%
\pgfsetrectcap%
\pgfsetroundjoin%
\pgfsetlinewidth{0.752812pt}%
\definecolor{currentstroke}{rgb}{0.000000,0.000000,0.000000}%
\pgfsetstrokecolor{currentstroke}%
\pgfsetdash{}{0pt}%
\pgfusepath{stroke}%
\end{pgfscope}%
\begin{pgfscope}%
\pgfpathrectangle{\pgfqpoint{0.550713in}{1.728870in}}{\pgfqpoint{3.194133in}{0.696189in}}%
\pgfusepath{clip}%
\pgfsetrectcap%
\pgfsetroundjoin%
\pgfsetlinewidth{0.752812pt}%
\definecolor{currentstroke}{rgb}{0.000000,0.000000,0.000000}%
\pgfsetstrokecolor{currentstroke}%
\pgfsetdash{}{0pt}%
\pgfusepath{stroke}%
\end{pgfscope}%
\begin{pgfscope}%
\pgfpathrectangle{\pgfqpoint{0.550713in}{1.728870in}}{\pgfqpoint{3.194133in}{0.696189in}}%
\pgfusepath{clip}%
\pgfsetrectcap%
\pgfsetroundjoin%
\pgfsetlinewidth{0.752812pt}%
\definecolor{currentstroke}{rgb}{0.000000,0.000000,0.000000}%
\pgfsetstrokecolor{currentstroke}%
\pgfsetdash{}{0pt}%
\pgfusepath{stroke}%
\end{pgfscope}%
\begin{pgfscope}%
\pgfpathrectangle{\pgfqpoint{0.550713in}{1.728870in}}{\pgfqpoint{3.194133in}{0.696189in}}%
\pgfusepath{clip}%
\pgfsetrectcap%
\pgfsetroundjoin%
\pgfsetlinewidth{0.752812pt}%
\definecolor{currentstroke}{rgb}{0.000000,0.000000,0.000000}%
\pgfsetstrokecolor{currentstroke}%
\pgfsetdash{}{0pt}%
\pgfusepath{stroke}%
\end{pgfscope}%
\begin{pgfscope}%
\pgfpathrectangle{\pgfqpoint{0.550713in}{1.728870in}}{\pgfqpoint{3.194133in}{0.696189in}}%
\pgfusepath{clip}%
\pgfsetrectcap%
\pgfsetroundjoin%
\pgfsetlinewidth{0.752812pt}%
\definecolor{currentstroke}{rgb}{0.000000,0.000000,0.000000}%
\pgfsetstrokecolor{currentstroke}%
\pgfsetdash{}{0pt}%
\pgfusepath{stroke}%
\end{pgfscope}%
\begin{pgfscope}%
\pgfpathrectangle{\pgfqpoint{0.550713in}{1.728870in}}{\pgfqpoint{3.194133in}{0.696189in}}%
\pgfusepath{clip}%
\pgfsetrectcap%
\pgfsetroundjoin%
\pgfsetlinewidth{0.752812pt}%
\definecolor{currentstroke}{rgb}{0.000000,0.000000,0.000000}%
\pgfsetstrokecolor{currentstroke}%
\pgfsetdash{}{0pt}%
\pgfusepath{stroke}%
\end{pgfscope}%
\begin{pgfscope}%
\pgfpathrectangle{\pgfqpoint{0.550713in}{1.728870in}}{\pgfqpoint{3.194133in}{0.696189in}}%
\pgfusepath{clip}%
\pgfsetrectcap%
\pgfsetroundjoin%
\pgfsetlinewidth{0.752812pt}%
\definecolor{currentstroke}{rgb}{0.000000,0.000000,0.000000}%
\pgfsetstrokecolor{currentstroke}%
\pgfsetdash{}{0pt}%
\pgfusepath{stroke}%
\end{pgfscope}%
\begin{pgfscope}%
\pgfpathrectangle{\pgfqpoint{0.550713in}{1.728870in}}{\pgfqpoint{3.194133in}{0.696189in}}%
\pgfusepath{clip}%
\pgfsetrectcap%
\pgfsetroundjoin%
\pgfsetlinewidth{0.752812pt}%
\definecolor{currentstroke}{rgb}{0.000000,0.000000,0.000000}%
\pgfsetstrokecolor{currentstroke}%
\pgfsetdash{}{0pt}%
\pgfusepath{stroke}%
\end{pgfscope}%
\begin{pgfscope}%
\pgfpathrectangle{\pgfqpoint{0.550713in}{1.728870in}}{\pgfqpoint{3.194133in}{0.696189in}}%
\pgfusepath{clip}%
\pgfsetrectcap%
\pgfsetroundjoin%
\pgfsetlinewidth{0.752812pt}%
\definecolor{currentstroke}{rgb}{0.000000,0.000000,0.000000}%
\pgfsetstrokecolor{currentstroke}%
\pgfsetdash{}{0pt}%
\pgfusepath{stroke}%
\end{pgfscope}%
\begin{pgfscope}%
\pgfpathrectangle{\pgfqpoint{0.550713in}{1.728870in}}{\pgfqpoint{3.194133in}{0.696189in}}%
\pgfusepath{clip}%
\pgfsetrectcap%
\pgfsetroundjoin%
\pgfsetlinewidth{0.752812pt}%
\definecolor{currentstroke}{rgb}{0.000000,0.000000,0.000000}%
\pgfsetstrokecolor{currentstroke}%
\pgfsetdash{}{0pt}%
\pgfusepath{stroke}%
\end{pgfscope}%
\begin{pgfscope}%
\pgfpathrectangle{\pgfqpoint{0.550713in}{1.728870in}}{\pgfqpoint{3.194133in}{0.696189in}}%
\pgfusepath{clip}%
\pgfsetrectcap%
\pgfsetroundjoin%
\pgfsetlinewidth{0.752812pt}%
\definecolor{currentstroke}{rgb}{0.000000,0.000000,0.000000}%
\pgfsetstrokecolor{currentstroke}%
\pgfsetdash{}{0pt}%
\pgfusepath{stroke}%
\end{pgfscope}%
\begin{pgfscope}%
\pgfpathrectangle{\pgfqpoint{0.550713in}{1.728870in}}{\pgfqpoint{3.194133in}{0.696189in}}%
\pgfusepath{clip}%
\pgfsetrectcap%
\pgfsetroundjoin%
\pgfsetlinewidth{0.752812pt}%
\definecolor{currentstroke}{rgb}{0.000000,0.000000,0.000000}%
\pgfsetstrokecolor{currentstroke}%
\pgfsetdash{}{0pt}%
\pgfusepath{stroke}%
\end{pgfscope}%
\begin{pgfscope}%
\pgfpathrectangle{\pgfqpoint{0.550713in}{1.728870in}}{\pgfqpoint{3.194133in}{0.696189in}}%
\pgfusepath{clip}%
\pgfsetrectcap%
\pgfsetroundjoin%
\pgfsetlinewidth{0.752812pt}%
\definecolor{currentstroke}{rgb}{0.000000,0.000000,0.000000}%
\pgfsetstrokecolor{currentstroke}%
\pgfsetdash{}{0pt}%
\pgfusepath{stroke}%
\end{pgfscope}%
\begin{pgfscope}%
\pgfpathrectangle{\pgfqpoint{0.550713in}{1.728870in}}{\pgfqpoint{3.194133in}{0.696189in}}%
\pgfusepath{clip}%
\pgfsetrectcap%
\pgfsetroundjoin%
\pgfsetlinewidth{0.752812pt}%
\definecolor{currentstroke}{rgb}{0.000000,0.000000,0.000000}%
\pgfsetstrokecolor{currentstroke}%
\pgfsetdash{}{0pt}%
\pgfusepath{stroke}%
\end{pgfscope}%
\begin{pgfscope}%
\pgfpathrectangle{\pgfqpoint{0.550713in}{1.728870in}}{\pgfqpoint{3.194133in}{0.696189in}}%
\pgfusepath{clip}%
\pgfsetrectcap%
\pgfsetroundjoin%
\pgfsetlinewidth{0.752812pt}%
\definecolor{currentstroke}{rgb}{0.000000,0.000000,0.000000}%
\pgfsetstrokecolor{currentstroke}%
\pgfsetdash{}{0pt}%
\pgfusepath{stroke}%
\end{pgfscope}%
\begin{pgfscope}%
\pgfpathrectangle{\pgfqpoint{0.550713in}{1.728870in}}{\pgfqpoint{3.194133in}{0.696189in}}%
\pgfusepath{clip}%
\pgfsetrectcap%
\pgfsetroundjoin%
\pgfsetlinewidth{0.752812pt}%
\definecolor{currentstroke}{rgb}{0.000000,0.000000,0.000000}%
\pgfsetstrokecolor{currentstroke}%
\pgfsetdash{}{0pt}%
\pgfusepath{stroke}%
\end{pgfscope}%
\begin{pgfscope}%
\pgfpathrectangle{\pgfqpoint{0.550713in}{1.728870in}}{\pgfqpoint{3.194133in}{0.696189in}}%
\pgfusepath{clip}%
\pgfsetrectcap%
\pgfsetroundjoin%
\pgfsetlinewidth{0.752812pt}%
\definecolor{currentstroke}{rgb}{0.000000,0.000000,0.000000}%
\pgfsetstrokecolor{currentstroke}%
\pgfsetdash{}{0pt}%
\pgfusepath{stroke}%
\end{pgfscope}%
\begin{pgfscope}%
\pgfpathrectangle{\pgfqpoint{0.550713in}{1.728870in}}{\pgfqpoint{3.194133in}{0.696189in}}%
\pgfusepath{clip}%
\pgfsetrectcap%
\pgfsetroundjoin%
\pgfsetlinewidth{0.752812pt}%
\definecolor{currentstroke}{rgb}{0.000000,0.000000,0.000000}%
\pgfsetstrokecolor{currentstroke}%
\pgfsetdash{}{0pt}%
\pgfusepath{stroke}%
\end{pgfscope}%
\begin{pgfscope}%
\pgfpathrectangle{\pgfqpoint{0.550713in}{1.728870in}}{\pgfqpoint{3.194133in}{0.696189in}}%
\pgfusepath{clip}%
\pgfsetrectcap%
\pgfsetroundjoin%
\pgfsetlinewidth{0.752812pt}%
\definecolor{currentstroke}{rgb}{0.000000,0.000000,0.000000}%
\pgfsetstrokecolor{currentstroke}%
\pgfsetdash{}{0pt}%
\pgfusepath{stroke}%
\end{pgfscope}%
\begin{pgfscope}%
\pgfpathrectangle{\pgfqpoint{0.550713in}{1.728870in}}{\pgfqpoint{3.194133in}{0.696189in}}%
\pgfusepath{clip}%
\pgfsetrectcap%
\pgfsetroundjoin%
\pgfsetlinewidth{0.752812pt}%
\definecolor{currentstroke}{rgb}{0.000000,0.000000,0.000000}%
\pgfsetstrokecolor{currentstroke}%
\pgfsetdash{}{0pt}%
\pgfusepath{stroke}%
\end{pgfscope}%
\begin{pgfscope}%
\pgfpathrectangle{\pgfqpoint{0.550713in}{1.728870in}}{\pgfqpoint{3.194133in}{0.696189in}}%
\pgfusepath{clip}%
\pgfsetrectcap%
\pgfsetroundjoin%
\pgfsetlinewidth{0.752812pt}%
\definecolor{currentstroke}{rgb}{0.000000,0.000000,0.000000}%
\pgfsetstrokecolor{currentstroke}%
\pgfsetdash{}{0pt}%
\pgfusepath{stroke}%
\end{pgfscope}%
\begin{pgfscope}%
\pgfpathrectangle{\pgfqpoint{0.550713in}{1.728870in}}{\pgfqpoint{3.194133in}{0.696189in}}%
\pgfusepath{clip}%
\pgfsetrectcap%
\pgfsetroundjoin%
\pgfsetlinewidth{0.752812pt}%
\definecolor{currentstroke}{rgb}{0.000000,0.000000,0.000000}%
\pgfsetstrokecolor{currentstroke}%
\pgfsetdash{}{0pt}%
\pgfusepath{stroke}%
\end{pgfscope}%
\begin{pgfscope}%
\pgfpathrectangle{\pgfqpoint{0.550713in}{1.728870in}}{\pgfqpoint{3.194133in}{0.696189in}}%
\pgfusepath{clip}%
\pgfsetrectcap%
\pgfsetroundjoin%
\pgfsetlinewidth{0.752812pt}%
\definecolor{currentstroke}{rgb}{0.000000,0.000000,0.000000}%
\pgfsetstrokecolor{currentstroke}%
\pgfsetdash{}{0pt}%
\pgfusepath{stroke}%
\end{pgfscope}%
\begin{pgfscope}%
\pgfpathrectangle{\pgfqpoint{0.550713in}{1.728870in}}{\pgfqpoint{3.194133in}{0.696189in}}%
\pgfusepath{clip}%
\pgfsetrectcap%
\pgfsetroundjoin%
\pgfsetlinewidth{0.752812pt}%
\definecolor{currentstroke}{rgb}{0.000000,0.000000,0.000000}%
\pgfsetstrokecolor{currentstroke}%
\pgfsetdash{}{0pt}%
\pgfusepath{stroke}%
\end{pgfscope}%
\begin{pgfscope}%
\pgfpathrectangle{\pgfqpoint{0.550713in}{1.728870in}}{\pgfqpoint{3.194133in}{0.696189in}}%
\pgfusepath{clip}%
\pgfsetrectcap%
\pgfsetroundjoin%
\pgfsetlinewidth{0.752812pt}%
\definecolor{currentstroke}{rgb}{0.000000,0.000000,0.000000}%
\pgfsetstrokecolor{currentstroke}%
\pgfsetdash{}{0pt}%
\pgfusepath{stroke}%
\end{pgfscope}%
\begin{pgfscope}%
\pgfpathrectangle{\pgfqpoint{0.550713in}{1.728870in}}{\pgfqpoint{3.194133in}{0.696189in}}%
\pgfusepath{clip}%
\pgfsetrectcap%
\pgfsetroundjoin%
\pgfsetlinewidth{0.752812pt}%
\definecolor{currentstroke}{rgb}{0.000000,0.000000,0.000000}%
\pgfsetstrokecolor{currentstroke}%
\pgfsetdash{}{0pt}%
\pgfusepath{stroke}%
\end{pgfscope}%
\begin{pgfscope}%
\pgfsetbuttcap%
\pgfsetroundjoin%
\definecolor{currentfill}{rgb}{0.000000,0.000000,0.000000}%
\pgfsetfillcolor{currentfill}%
\pgfsetlinewidth{0.752812pt}%
\definecolor{currentstroke}{rgb}{0.000000,0.000000,0.000000}%
\pgfsetstrokecolor{currentstroke}%
\pgfsetdash{}{0pt}%
\pgfsys@defobject{currentmarker}{\pgfqpoint{-0.055556in}{-0.027778in}}{\pgfqpoint{0.055556in}{0.027778in}}{%
\pgfpathmoveto{\pgfqpoint{-0.055556in}{-0.027778in}}%
\pgfpathlineto{\pgfqpoint{0.055556in}{0.027778in}}%
\pgfusepath{stroke,fill}%
}%
\begin{pgfscope}%
\pgfsys@transformshift{0.550713in}{1.728870in}%
\pgfsys@useobject{currentmarker}{}%
\end{pgfscope}%
\begin{pgfscope}%
\pgfsys@transformshift{3.744846in}{1.728870in}%
\pgfsys@useobject{currentmarker}{}%
\end{pgfscope}%
\end{pgfscope}%
\begin{pgfscope}%
\pgfpathrectangle{\pgfqpoint{0.550713in}{1.728870in}}{\pgfqpoint{3.194133in}{0.696189in}}%
\pgfusepath{clip}%
\pgfsetrectcap%
\pgfsetroundjoin%
\pgfsetlinewidth{0.752812pt}%
\definecolor{currentstroke}{rgb}{0.000000,0.000000,0.000000}%
\pgfsetstrokecolor{currentstroke}%
\pgfsetdash{}{0pt}%
\pgfusepath{stroke}%
\end{pgfscope}%
\begin{pgfscope}%
\pgfpathrectangle{\pgfqpoint{0.550713in}{1.728870in}}{\pgfqpoint{3.194133in}{0.696189in}}%
\pgfusepath{clip}%
\pgfsetbuttcap%
\pgfsetroundjoin%
\definecolor{currentfill}{rgb}{1.000000,1.000000,1.000000}%
\pgfsetfillcolor{currentfill}%
\pgfsetlinewidth{1.003750pt}%
\definecolor{currentstroke}{rgb}{0.000000,0.000000,0.000000}%
\pgfsetstrokecolor{currentstroke}%
\pgfsetdash{}{0pt}%
\pgfsys@defobject{currentmarker}{\pgfqpoint{-0.027778in}{-0.027778in}}{\pgfqpoint{0.027778in}{0.027778in}}{%
\pgfpathmoveto{\pgfqpoint{0.000000in}{-0.027778in}}%
\pgfpathcurveto{\pgfqpoint{0.007367in}{-0.027778in}}{\pgfqpoint{0.014433in}{-0.024851in}}{\pgfqpoint{0.019642in}{-0.019642in}}%
\pgfpathcurveto{\pgfqpoint{0.024851in}{-0.014433in}}{\pgfqpoint{0.027778in}{-0.007367in}}{\pgfqpoint{0.027778in}{0.000000in}}%
\pgfpathcurveto{\pgfqpoint{0.027778in}{0.007367in}}{\pgfqpoint{0.024851in}{0.014433in}}{\pgfqpoint{0.019642in}{0.019642in}}%
\pgfpathcurveto{\pgfqpoint{0.014433in}{0.024851in}}{\pgfqpoint{0.007367in}{0.027778in}}{\pgfqpoint{0.000000in}{0.027778in}}%
\pgfpathcurveto{\pgfqpoint{-0.007367in}{0.027778in}}{\pgfqpoint{-0.014433in}{0.024851in}}{\pgfqpoint{-0.019642in}{0.019642in}}%
\pgfpathcurveto{\pgfqpoint{-0.024851in}{0.014433in}}{\pgfqpoint{-0.027778in}{0.007367in}}{\pgfqpoint{-0.027778in}{0.000000in}}%
\pgfpathcurveto{\pgfqpoint{-0.027778in}{-0.007367in}}{\pgfqpoint{-0.024851in}{-0.014433in}}{\pgfqpoint{-0.019642in}{-0.019642in}}%
\pgfpathcurveto{\pgfqpoint{-0.014433in}{-0.024851in}}{\pgfqpoint{-0.007367in}{-0.027778in}}{\pgfqpoint{0.000000in}{-0.027778in}}%
\pgfpathclose%
\pgfusepath{stroke,fill}%
}%
\begin{pgfscope}%
\pgfsys@transformshift{0.670493in}{0.950882in}%
\pgfsys@useobject{currentmarker}{}%
\end{pgfscope}%
\end{pgfscope}%
\begin{pgfscope}%
\pgfpathrectangle{\pgfqpoint{0.550713in}{1.728870in}}{\pgfqpoint{3.194133in}{0.696189in}}%
\pgfusepath{clip}%
\pgfsetrectcap%
\pgfsetroundjoin%
\pgfsetlinewidth{0.752812pt}%
\definecolor{currentstroke}{rgb}{0.000000,0.000000,0.000000}%
\pgfsetstrokecolor{currentstroke}%
\pgfsetdash{}{0pt}%
\pgfusepath{stroke}%
\end{pgfscope}%
\begin{pgfscope}%
\pgfpathrectangle{\pgfqpoint{0.550713in}{1.728870in}}{\pgfqpoint{3.194133in}{0.696189in}}%
\pgfusepath{clip}%
\pgfsetbuttcap%
\pgfsetroundjoin%
\definecolor{currentfill}{rgb}{1.000000,1.000000,1.000000}%
\pgfsetfillcolor{currentfill}%
\pgfsetlinewidth{1.003750pt}%
\definecolor{currentstroke}{rgb}{0.000000,0.000000,0.000000}%
\pgfsetstrokecolor{currentstroke}%
\pgfsetdash{}{0pt}%
\pgfsys@defobject{currentmarker}{\pgfqpoint{-0.027778in}{-0.027778in}}{\pgfqpoint{0.027778in}{0.027778in}}{%
\pgfpathmoveto{\pgfqpoint{0.000000in}{-0.027778in}}%
\pgfpathcurveto{\pgfqpoint{0.007367in}{-0.027778in}}{\pgfqpoint{0.014433in}{-0.024851in}}{\pgfqpoint{0.019642in}{-0.019642in}}%
\pgfpathcurveto{\pgfqpoint{0.024851in}{-0.014433in}}{\pgfqpoint{0.027778in}{-0.007367in}}{\pgfqpoint{0.027778in}{0.000000in}}%
\pgfpathcurveto{\pgfqpoint{0.027778in}{0.007367in}}{\pgfqpoint{0.024851in}{0.014433in}}{\pgfqpoint{0.019642in}{0.019642in}}%
\pgfpathcurveto{\pgfqpoint{0.014433in}{0.024851in}}{\pgfqpoint{0.007367in}{0.027778in}}{\pgfqpoint{0.000000in}{0.027778in}}%
\pgfpathcurveto{\pgfqpoint{-0.007367in}{0.027778in}}{\pgfqpoint{-0.014433in}{0.024851in}}{\pgfqpoint{-0.019642in}{0.019642in}}%
\pgfpathcurveto{\pgfqpoint{-0.024851in}{0.014433in}}{\pgfqpoint{-0.027778in}{0.007367in}}{\pgfqpoint{-0.027778in}{0.000000in}}%
\pgfpathcurveto{\pgfqpoint{-0.027778in}{-0.007367in}}{\pgfqpoint{-0.024851in}{-0.014433in}}{\pgfqpoint{-0.019642in}{-0.019642in}}%
\pgfpathcurveto{\pgfqpoint{-0.014433in}{-0.024851in}}{\pgfqpoint{-0.007367in}{-0.027778in}}{\pgfqpoint{0.000000in}{-0.027778in}}%
\pgfpathclose%
\pgfusepath{stroke,fill}%
}%
\begin{pgfscope}%
\pgfsys@transformshift{0.830199in}{1.484486in}%
\pgfsys@useobject{currentmarker}{}%
\end{pgfscope}%
\end{pgfscope}%
\begin{pgfscope}%
\pgfpathrectangle{\pgfqpoint{0.550713in}{1.728870in}}{\pgfqpoint{3.194133in}{0.696189in}}%
\pgfusepath{clip}%
\pgfsetrectcap%
\pgfsetroundjoin%
\pgfsetlinewidth{0.752812pt}%
\definecolor{currentstroke}{rgb}{0.000000,0.000000,0.000000}%
\pgfsetstrokecolor{currentstroke}%
\pgfsetdash{}{0pt}%
\pgfusepath{stroke}%
\end{pgfscope}%
\begin{pgfscope}%
\pgfpathrectangle{\pgfqpoint{0.550713in}{1.728870in}}{\pgfqpoint{3.194133in}{0.696189in}}%
\pgfusepath{clip}%
\pgfsetbuttcap%
\pgfsetroundjoin%
\definecolor{currentfill}{rgb}{1.000000,1.000000,1.000000}%
\pgfsetfillcolor{currentfill}%
\pgfsetlinewidth{1.003750pt}%
\definecolor{currentstroke}{rgb}{0.000000,0.000000,0.000000}%
\pgfsetstrokecolor{currentstroke}%
\pgfsetdash{}{0pt}%
\pgfsys@defobject{currentmarker}{\pgfqpoint{-0.027778in}{-0.027778in}}{\pgfqpoint{0.027778in}{0.027778in}}{%
\pgfpathmoveto{\pgfqpoint{0.000000in}{-0.027778in}}%
\pgfpathcurveto{\pgfqpoint{0.007367in}{-0.027778in}}{\pgfqpoint{0.014433in}{-0.024851in}}{\pgfqpoint{0.019642in}{-0.019642in}}%
\pgfpathcurveto{\pgfqpoint{0.024851in}{-0.014433in}}{\pgfqpoint{0.027778in}{-0.007367in}}{\pgfqpoint{0.027778in}{0.000000in}}%
\pgfpathcurveto{\pgfqpoint{0.027778in}{0.007367in}}{\pgfqpoint{0.024851in}{0.014433in}}{\pgfqpoint{0.019642in}{0.019642in}}%
\pgfpathcurveto{\pgfqpoint{0.014433in}{0.024851in}}{\pgfqpoint{0.007367in}{0.027778in}}{\pgfqpoint{0.000000in}{0.027778in}}%
\pgfpathcurveto{\pgfqpoint{-0.007367in}{0.027778in}}{\pgfqpoint{-0.014433in}{0.024851in}}{\pgfqpoint{-0.019642in}{0.019642in}}%
\pgfpathcurveto{\pgfqpoint{-0.024851in}{0.014433in}}{\pgfqpoint{-0.027778in}{0.007367in}}{\pgfqpoint{-0.027778in}{0.000000in}}%
\pgfpathcurveto{\pgfqpoint{-0.027778in}{-0.007367in}}{\pgfqpoint{-0.024851in}{-0.014433in}}{\pgfqpoint{-0.019642in}{-0.019642in}}%
\pgfpathcurveto{\pgfqpoint{-0.014433in}{-0.024851in}}{\pgfqpoint{-0.007367in}{-0.027778in}}{\pgfqpoint{0.000000in}{-0.027778in}}%
\pgfpathclose%
\pgfusepath{stroke,fill}%
}%
\begin{pgfscope}%
\pgfsys@transformshift{1.069759in}{1.071892in}%
\pgfsys@useobject{currentmarker}{}%
\end{pgfscope}%
\end{pgfscope}%
\begin{pgfscope}%
\pgfpathrectangle{\pgfqpoint{0.550713in}{1.728870in}}{\pgfqpoint{3.194133in}{0.696189in}}%
\pgfusepath{clip}%
\pgfsetrectcap%
\pgfsetroundjoin%
\pgfsetlinewidth{0.752812pt}%
\definecolor{currentstroke}{rgb}{0.000000,0.000000,0.000000}%
\pgfsetstrokecolor{currentstroke}%
\pgfsetdash{}{0pt}%
\pgfpathmoveto{\pgfqpoint{1.151210in}{2.150545in}}%
\pgfpathlineto{\pgfqpoint{1.307722in}{2.150545in}}%
\pgfusepath{stroke}%
\end{pgfscope}%
\begin{pgfscope}%
\pgfpathrectangle{\pgfqpoint{0.550713in}{1.728870in}}{\pgfqpoint{3.194133in}{0.696189in}}%
\pgfusepath{clip}%
\pgfsetbuttcap%
\pgfsetroundjoin%
\definecolor{currentfill}{rgb}{1.000000,1.000000,1.000000}%
\pgfsetfillcolor{currentfill}%
\pgfsetlinewidth{1.003750pt}%
\definecolor{currentstroke}{rgb}{0.000000,0.000000,0.000000}%
\pgfsetstrokecolor{currentstroke}%
\pgfsetdash{}{0pt}%
\pgfsys@defobject{currentmarker}{\pgfqpoint{-0.027778in}{-0.027778in}}{\pgfqpoint{0.027778in}{0.027778in}}{%
\pgfpathmoveto{\pgfqpoint{0.000000in}{-0.027778in}}%
\pgfpathcurveto{\pgfqpoint{0.007367in}{-0.027778in}}{\pgfqpoint{0.014433in}{-0.024851in}}{\pgfqpoint{0.019642in}{-0.019642in}}%
\pgfpathcurveto{\pgfqpoint{0.024851in}{-0.014433in}}{\pgfqpoint{0.027778in}{-0.007367in}}{\pgfqpoint{0.027778in}{0.000000in}}%
\pgfpathcurveto{\pgfqpoint{0.027778in}{0.007367in}}{\pgfqpoint{0.024851in}{0.014433in}}{\pgfqpoint{0.019642in}{0.019642in}}%
\pgfpathcurveto{\pgfqpoint{0.014433in}{0.024851in}}{\pgfqpoint{0.007367in}{0.027778in}}{\pgfqpoint{0.000000in}{0.027778in}}%
\pgfpathcurveto{\pgfqpoint{-0.007367in}{0.027778in}}{\pgfqpoint{-0.014433in}{0.024851in}}{\pgfqpoint{-0.019642in}{0.019642in}}%
\pgfpathcurveto{\pgfqpoint{-0.024851in}{0.014433in}}{\pgfqpoint{-0.027778in}{0.007367in}}{\pgfqpoint{-0.027778in}{0.000000in}}%
\pgfpathcurveto{\pgfqpoint{-0.027778in}{-0.007367in}}{\pgfqpoint{-0.024851in}{-0.014433in}}{\pgfqpoint{-0.019642in}{-0.019642in}}%
\pgfpathcurveto{\pgfqpoint{-0.014433in}{-0.024851in}}{\pgfqpoint{-0.007367in}{-0.027778in}}{\pgfqpoint{0.000000in}{-0.027778in}}%
\pgfpathclose%
\pgfusepath{stroke,fill}%
}%
\begin{pgfscope}%
\pgfsys@transformshift{1.229466in}{2.013613in}%
\pgfsys@useobject{currentmarker}{}%
\end{pgfscope}%
\end{pgfscope}%
\begin{pgfscope}%
\pgfpathrectangle{\pgfqpoint{0.550713in}{1.728870in}}{\pgfqpoint{3.194133in}{0.696189in}}%
\pgfusepath{clip}%
\pgfsetrectcap%
\pgfsetroundjoin%
\pgfsetlinewidth{0.752812pt}%
\definecolor{currentstroke}{rgb}{0.000000,0.000000,0.000000}%
\pgfsetstrokecolor{currentstroke}%
\pgfsetdash{}{0pt}%
\pgfusepath{stroke}%
\end{pgfscope}%
\begin{pgfscope}%
\pgfpathrectangle{\pgfqpoint{0.550713in}{1.728870in}}{\pgfqpoint{3.194133in}{0.696189in}}%
\pgfusepath{clip}%
\pgfsetbuttcap%
\pgfsetroundjoin%
\definecolor{currentfill}{rgb}{1.000000,1.000000,1.000000}%
\pgfsetfillcolor{currentfill}%
\pgfsetlinewidth{1.003750pt}%
\definecolor{currentstroke}{rgb}{0.000000,0.000000,0.000000}%
\pgfsetstrokecolor{currentstroke}%
\pgfsetdash{}{0pt}%
\pgfsys@defobject{currentmarker}{\pgfqpoint{-0.027778in}{-0.027778in}}{\pgfqpoint{0.027778in}{0.027778in}}{%
\pgfpathmoveto{\pgfqpoint{0.000000in}{-0.027778in}}%
\pgfpathcurveto{\pgfqpoint{0.007367in}{-0.027778in}}{\pgfqpoint{0.014433in}{-0.024851in}}{\pgfqpoint{0.019642in}{-0.019642in}}%
\pgfpathcurveto{\pgfqpoint{0.024851in}{-0.014433in}}{\pgfqpoint{0.027778in}{-0.007367in}}{\pgfqpoint{0.027778in}{0.000000in}}%
\pgfpathcurveto{\pgfqpoint{0.027778in}{0.007367in}}{\pgfqpoint{0.024851in}{0.014433in}}{\pgfqpoint{0.019642in}{0.019642in}}%
\pgfpathcurveto{\pgfqpoint{0.014433in}{0.024851in}}{\pgfqpoint{0.007367in}{0.027778in}}{\pgfqpoint{0.000000in}{0.027778in}}%
\pgfpathcurveto{\pgfqpoint{-0.007367in}{0.027778in}}{\pgfqpoint{-0.014433in}{0.024851in}}{\pgfqpoint{-0.019642in}{0.019642in}}%
\pgfpathcurveto{\pgfqpoint{-0.024851in}{0.014433in}}{\pgfqpoint{-0.027778in}{0.007367in}}{\pgfqpoint{-0.027778in}{0.000000in}}%
\pgfpathcurveto{\pgfqpoint{-0.027778in}{-0.007367in}}{\pgfqpoint{-0.024851in}{-0.014433in}}{\pgfqpoint{-0.019642in}{-0.019642in}}%
\pgfpathcurveto{\pgfqpoint{-0.014433in}{-0.024851in}}{\pgfqpoint{-0.007367in}{-0.027778in}}{\pgfqpoint{0.000000in}{-0.027778in}}%
\pgfpathclose%
\pgfusepath{stroke,fill}%
}%
\begin{pgfscope}%
\pgfsys@transformshift{1.469026in}{0.944380in}%
\pgfsys@useobject{currentmarker}{}%
\end{pgfscope}%
\end{pgfscope}%
\begin{pgfscope}%
\pgfpathrectangle{\pgfqpoint{0.550713in}{1.728870in}}{\pgfqpoint{3.194133in}{0.696189in}}%
\pgfusepath{clip}%
\pgfsetrectcap%
\pgfsetroundjoin%
\pgfsetlinewidth{0.752812pt}%
\definecolor{currentstroke}{rgb}{0.000000,0.000000,0.000000}%
\pgfsetstrokecolor{currentstroke}%
\pgfsetdash{}{0pt}%
\pgfusepath{stroke}%
\end{pgfscope}%
\begin{pgfscope}%
\pgfpathrectangle{\pgfqpoint{0.550713in}{1.728870in}}{\pgfqpoint{3.194133in}{0.696189in}}%
\pgfusepath{clip}%
\pgfsetbuttcap%
\pgfsetroundjoin%
\definecolor{currentfill}{rgb}{1.000000,1.000000,1.000000}%
\pgfsetfillcolor{currentfill}%
\pgfsetlinewidth{1.003750pt}%
\definecolor{currentstroke}{rgb}{0.000000,0.000000,0.000000}%
\pgfsetstrokecolor{currentstroke}%
\pgfsetdash{}{0pt}%
\pgfsys@defobject{currentmarker}{\pgfqpoint{-0.027778in}{-0.027778in}}{\pgfqpoint{0.027778in}{0.027778in}}{%
\pgfpathmoveto{\pgfqpoint{0.000000in}{-0.027778in}}%
\pgfpathcurveto{\pgfqpoint{0.007367in}{-0.027778in}}{\pgfqpoint{0.014433in}{-0.024851in}}{\pgfqpoint{0.019642in}{-0.019642in}}%
\pgfpathcurveto{\pgfqpoint{0.024851in}{-0.014433in}}{\pgfqpoint{0.027778in}{-0.007367in}}{\pgfqpoint{0.027778in}{0.000000in}}%
\pgfpathcurveto{\pgfqpoint{0.027778in}{0.007367in}}{\pgfqpoint{0.024851in}{0.014433in}}{\pgfqpoint{0.019642in}{0.019642in}}%
\pgfpathcurveto{\pgfqpoint{0.014433in}{0.024851in}}{\pgfqpoint{0.007367in}{0.027778in}}{\pgfqpoint{0.000000in}{0.027778in}}%
\pgfpathcurveto{\pgfqpoint{-0.007367in}{0.027778in}}{\pgfqpoint{-0.014433in}{0.024851in}}{\pgfqpoint{-0.019642in}{0.019642in}}%
\pgfpathcurveto{\pgfqpoint{-0.024851in}{0.014433in}}{\pgfqpoint{-0.027778in}{0.007367in}}{\pgfqpoint{-0.027778in}{0.000000in}}%
\pgfpathcurveto{\pgfqpoint{-0.027778in}{-0.007367in}}{\pgfqpoint{-0.024851in}{-0.014433in}}{\pgfqpoint{-0.019642in}{-0.019642in}}%
\pgfpathcurveto{\pgfqpoint{-0.014433in}{-0.024851in}}{\pgfqpoint{-0.007367in}{-0.027778in}}{\pgfqpoint{0.000000in}{-0.027778in}}%
\pgfpathclose%
\pgfusepath{stroke,fill}%
}%
\begin{pgfscope}%
\pgfsys@transformshift{1.628733in}{1.022847in}%
\pgfsys@useobject{currentmarker}{}%
\end{pgfscope}%
\end{pgfscope}%
\begin{pgfscope}%
\pgfpathrectangle{\pgfqpoint{0.550713in}{1.728870in}}{\pgfqpoint{3.194133in}{0.696189in}}%
\pgfusepath{clip}%
\pgfsetrectcap%
\pgfsetroundjoin%
\pgfsetlinewidth{0.752812pt}%
\definecolor{currentstroke}{rgb}{0.000000,0.000000,0.000000}%
\pgfsetstrokecolor{currentstroke}%
\pgfsetdash{}{0pt}%
\pgfusepath{stroke}%
\end{pgfscope}%
\begin{pgfscope}%
\pgfpathrectangle{\pgfqpoint{0.550713in}{1.728870in}}{\pgfqpoint{3.194133in}{0.696189in}}%
\pgfusepath{clip}%
\pgfsetbuttcap%
\pgfsetroundjoin%
\definecolor{currentfill}{rgb}{1.000000,1.000000,1.000000}%
\pgfsetfillcolor{currentfill}%
\pgfsetlinewidth{1.003750pt}%
\definecolor{currentstroke}{rgb}{0.000000,0.000000,0.000000}%
\pgfsetstrokecolor{currentstroke}%
\pgfsetdash{}{0pt}%
\pgfsys@defobject{currentmarker}{\pgfqpoint{-0.027778in}{-0.027778in}}{\pgfqpoint{0.027778in}{0.027778in}}{%
\pgfpathmoveto{\pgfqpoint{0.000000in}{-0.027778in}}%
\pgfpathcurveto{\pgfqpoint{0.007367in}{-0.027778in}}{\pgfqpoint{0.014433in}{-0.024851in}}{\pgfqpoint{0.019642in}{-0.019642in}}%
\pgfpathcurveto{\pgfqpoint{0.024851in}{-0.014433in}}{\pgfqpoint{0.027778in}{-0.007367in}}{\pgfqpoint{0.027778in}{0.000000in}}%
\pgfpathcurveto{\pgfqpoint{0.027778in}{0.007367in}}{\pgfqpoint{0.024851in}{0.014433in}}{\pgfqpoint{0.019642in}{0.019642in}}%
\pgfpathcurveto{\pgfqpoint{0.014433in}{0.024851in}}{\pgfqpoint{0.007367in}{0.027778in}}{\pgfqpoint{0.000000in}{0.027778in}}%
\pgfpathcurveto{\pgfqpoint{-0.007367in}{0.027778in}}{\pgfqpoint{-0.014433in}{0.024851in}}{\pgfqpoint{-0.019642in}{0.019642in}}%
\pgfpathcurveto{\pgfqpoint{-0.024851in}{0.014433in}}{\pgfqpoint{-0.027778in}{0.007367in}}{\pgfqpoint{-0.027778in}{0.000000in}}%
\pgfpathcurveto{\pgfqpoint{-0.027778in}{-0.007367in}}{\pgfqpoint{-0.024851in}{-0.014433in}}{\pgfqpoint{-0.019642in}{-0.019642in}}%
\pgfpathcurveto{\pgfqpoint{-0.014433in}{-0.024851in}}{\pgfqpoint{-0.007367in}{-0.027778in}}{\pgfqpoint{0.000000in}{-0.027778in}}%
\pgfpathclose%
\pgfusepath{stroke,fill}%
}%
\begin{pgfscope}%
\pgfsys@transformshift{1.868293in}{1.118028in}%
\pgfsys@useobject{currentmarker}{}%
\end{pgfscope}%
\end{pgfscope}%
\begin{pgfscope}%
\pgfpathrectangle{\pgfqpoint{0.550713in}{1.728870in}}{\pgfqpoint{3.194133in}{0.696189in}}%
\pgfusepath{clip}%
\pgfsetrectcap%
\pgfsetroundjoin%
\pgfsetlinewidth{0.752812pt}%
\definecolor{currentstroke}{rgb}{0.000000,0.000000,0.000000}%
\pgfsetstrokecolor{currentstroke}%
\pgfsetdash{}{0pt}%
\pgfusepath{stroke}%
\end{pgfscope}%
\begin{pgfscope}%
\pgfpathrectangle{\pgfqpoint{0.550713in}{1.728870in}}{\pgfqpoint{3.194133in}{0.696189in}}%
\pgfusepath{clip}%
\pgfsetbuttcap%
\pgfsetroundjoin%
\definecolor{currentfill}{rgb}{1.000000,1.000000,1.000000}%
\pgfsetfillcolor{currentfill}%
\pgfsetlinewidth{1.003750pt}%
\definecolor{currentstroke}{rgb}{0.000000,0.000000,0.000000}%
\pgfsetstrokecolor{currentstroke}%
\pgfsetdash{}{0pt}%
\pgfsys@defobject{currentmarker}{\pgfqpoint{-0.027778in}{-0.027778in}}{\pgfqpoint{0.027778in}{0.027778in}}{%
\pgfpathmoveto{\pgfqpoint{0.000000in}{-0.027778in}}%
\pgfpathcurveto{\pgfqpoint{0.007367in}{-0.027778in}}{\pgfqpoint{0.014433in}{-0.024851in}}{\pgfqpoint{0.019642in}{-0.019642in}}%
\pgfpathcurveto{\pgfqpoint{0.024851in}{-0.014433in}}{\pgfqpoint{0.027778in}{-0.007367in}}{\pgfqpoint{0.027778in}{0.000000in}}%
\pgfpathcurveto{\pgfqpoint{0.027778in}{0.007367in}}{\pgfqpoint{0.024851in}{0.014433in}}{\pgfqpoint{0.019642in}{0.019642in}}%
\pgfpathcurveto{\pgfqpoint{0.014433in}{0.024851in}}{\pgfqpoint{0.007367in}{0.027778in}}{\pgfqpoint{0.000000in}{0.027778in}}%
\pgfpathcurveto{\pgfqpoint{-0.007367in}{0.027778in}}{\pgfqpoint{-0.014433in}{0.024851in}}{\pgfqpoint{-0.019642in}{0.019642in}}%
\pgfpathcurveto{\pgfqpoint{-0.024851in}{0.014433in}}{\pgfqpoint{-0.027778in}{0.007367in}}{\pgfqpoint{-0.027778in}{0.000000in}}%
\pgfpathcurveto{\pgfqpoint{-0.027778in}{-0.007367in}}{\pgfqpoint{-0.024851in}{-0.014433in}}{\pgfqpoint{-0.019642in}{-0.019642in}}%
\pgfpathcurveto{\pgfqpoint{-0.014433in}{-0.024851in}}{\pgfqpoint{-0.007367in}{-0.027778in}}{\pgfqpoint{0.000000in}{-0.027778in}}%
\pgfpathclose%
\pgfusepath{stroke,fill}%
}%
\begin{pgfscope}%
\pgfsys@transformshift{2.027999in}{1.146977in}%
\pgfsys@useobject{currentmarker}{}%
\end{pgfscope}%
\end{pgfscope}%
\begin{pgfscope}%
\pgfpathrectangle{\pgfqpoint{0.550713in}{1.728870in}}{\pgfqpoint{3.194133in}{0.696189in}}%
\pgfusepath{clip}%
\pgfsetrectcap%
\pgfsetroundjoin%
\pgfsetlinewidth{0.752812pt}%
\definecolor{currentstroke}{rgb}{0.000000,0.000000,0.000000}%
\pgfsetstrokecolor{currentstroke}%
\pgfsetdash{}{0pt}%
\pgfusepath{stroke}%
\end{pgfscope}%
\begin{pgfscope}%
\pgfpathrectangle{\pgfqpoint{0.550713in}{1.728870in}}{\pgfqpoint{3.194133in}{0.696189in}}%
\pgfusepath{clip}%
\pgfsetbuttcap%
\pgfsetroundjoin%
\definecolor{currentfill}{rgb}{1.000000,1.000000,1.000000}%
\pgfsetfillcolor{currentfill}%
\pgfsetlinewidth{1.003750pt}%
\definecolor{currentstroke}{rgb}{0.000000,0.000000,0.000000}%
\pgfsetstrokecolor{currentstroke}%
\pgfsetdash{}{0pt}%
\pgfsys@defobject{currentmarker}{\pgfqpoint{-0.027778in}{-0.027778in}}{\pgfqpoint{0.027778in}{0.027778in}}{%
\pgfpathmoveto{\pgfqpoint{0.000000in}{-0.027778in}}%
\pgfpathcurveto{\pgfqpoint{0.007367in}{-0.027778in}}{\pgfqpoint{0.014433in}{-0.024851in}}{\pgfqpoint{0.019642in}{-0.019642in}}%
\pgfpathcurveto{\pgfqpoint{0.024851in}{-0.014433in}}{\pgfqpoint{0.027778in}{-0.007367in}}{\pgfqpoint{0.027778in}{0.000000in}}%
\pgfpathcurveto{\pgfqpoint{0.027778in}{0.007367in}}{\pgfqpoint{0.024851in}{0.014433in}}{\pgfqpoint{0.019642in}{0.019642in}}%
\pgfpathcurveto{\pgfqpoint{0.014433in}{0.024851in}}{\pgfqpoint{0.007367in}{0.027778in}}{\pgfqpoint{0.000000in}{0.027778in}}%
\pgfpathcurveto{\pgfqpoint{-0.007367in}{0.027778in}}{\pgfqpoint{-0.014433in}{0.024851in}}{\pgfqpoint{-0.019642in}{0.019642in}}%
\pgfpathcurveto{\pgfqpoint{-0.024851in}{0.014433in}}{\pgfqpoint{-0.027778in}{0.007367in}}{\pgfqpoint{-0.027778in}{0.000000in}}%
\pgfpathcurveto{\pgfqpoint{-0.027778in}{-0.007367in}}{\pgfqpoint{-0.024851in}{-0.014433in}}{\pgfqpoint{-0.019642in}{-0.019642in}}%
\pgfpathcurveto{\pgfqpoint{-0.014433in}{-0.024851in}}{\pgfqpoint{-0.007367in}{-0.027778in}}{\pgfqpoint{0.000000in}{-0.027778in}}%
\pgfpathclose%
\pgfusepath{stroke,fill}%
}%
\begin{pgfscope}%
\pgfsys@transformshift{2.267559in}{0.871364in}%
\pgfsys@useobject{currentmarker}{}%
\end{pgfscope}%
\end{pgfscope}%
\begin{pgfscope}%
\pgfpathrectangle{\pgfqpoint{0.550713in}{1.728870in}}{\pgfqpoint{3.194133in}{0.696189in}}%
\pgfusepath{clip}%
\pgfsetrectcap%
\pgfsetroundjoin%
\pgfsetlinewidth{0.752812pt}%
\definecolor{currentstroke}{rgb}{0.000000,0.000000,0.000000}%
\pgfsetstrokecolor{currentstroke}%
\pgfsetdash{}{0pt}%
\pgfusepath{stroke}%
\end{pgfscope}%
\begin{pgfscope}%
\pgfpathrectangle{\pgfqpoint{0.550713in}{1.728870in}}{\pgfqpoint{3.194133in}{0.696189in}}%
\pgfusepath{clip}%
\pgfsetbuttcap%
\pgfsetroundjoin%
\definecolor{currentfill}{rgb}{1.000000,1.000000,1.000000}%
\pgfsetfillcolor{currentfill}%
\pgfsetlinewidth{1.003750pt}%
\definecolor{currentstroke}{rgb}{0.000000,0.000000,0.000000}%
\pgfsetstrokecolor{currentstroke}%
\pgfsetdash{}{0pt}%
\pgfsys@defobject{currentmarker}{\pgfqpoint{-0.027778in}{-0.027778in}}{\pgfqpoint{0.027778in}{0.027778in}}{%
\pgfpathmoveto{\pgfqpoint{0.000000in}{-0.027778in}}%
\pgfpathcurveto{\pgfqpoint{0.007367in}{-0.027778in}}{\pgfqpoint{0.014433in}{-0.024851in}}{\pgfqpoint{0.019642in}{-0.019642in}}%
\pgfpathcurveto{\pgfqpoint{0.024851in}{-0.014433in}}{\pgfqpoint{0.027778in}{-0.007367in}}{\pgfqpoint{0.027778in}{0.000000in}}%
\pgfpathcurveto{\pgfqpoint{0.027778in}{0.007367in}}{\pgfqpoint{0.024851in}{0.014433in}}{\pgfqpoint{0.019642in}{0.019642in}}%
\pgfpathcurveto{\pgfqpoint{0.014433in}{0.024851in}}{\pgfqpoint{0.007367in}{0.027778in}}{\pgfqpoint{0.000000in}{0.027778in}}%
\pgfpathcurveto{\pgfqpoint{-0.007367in}{0.027778in}}{\pgfqpoint{-0.014433in}{0.024851in}}{\pgfqpoint{-0.019642in}{0.019642in}}%
\pgfpathcurveto{\pgfqpoint{-0.024851in}{0.014433in}}{\pgfqpoint{-0.027778in}{0.007367in}}{\pgfqpoint{-0.027778in}{0.000000in}}%
\pgfpathcurveto{\pgfqpoint{-0.027778in}{-0.007367in}}{\pgfqpoint{-0.024851in}{-0.014433in}}{\pgfqpoint{-0.019642in}{-0.019642in}}%
\pgfpathcurveto{\pgfqpoint{-0.014433in}{-0.024851in}}{\pgfqpoint{-0.007367in}{-0.027778in}}{\pgfqpoint{0.000000in}{-0.027778in}}%
\pgfpathclose%
\pgfusepath{stroke,fill}%
}%
\begin{pgfscope}%
\pgfsys@transformshift{2.427266in}{0.967606in}%
\pgfsys@useobject{currentmarker}{}%
\end{pgfscope}%
\end{pgfscope}%
\begin{pgfscope}%
\pgfpathrectangle{\pgfqpoint{0.550713in}{1.728870in}}{\pgfqpoint{3.194133in}{0.696189in}}%
\pgfusepath{clip}%
\pgfsetrectcap%
\pgfsetroundjoin%
\pgfsetlinewidth{0.752812pt}%
\definecolor{currentstroke}{rgb}{0.000000,0.000000,0.000000}%
\pgfsetstrokecolor{currentstroke}%
\pgfsetdash{}{0pt}%
\pgfusepath{stroke}%
\end{pgfscope}%
\begin{pgfscope}%
\pgfpathrectangle{\pgfqpoint{0.550713in}{1.728870in}}{\pgfqpoint{3.194133in}{0.696189in}}%
\pgfusepath{clip}%
\pgfsetbuttcap%
\pgfsetroundjoin%
\definecolor{currentfill}{rgb}{1.000000,1.000000,1.000000}%
\pgfsetfillcolor{currentfill}%
\pgfsetlinewidth{1.003750pt}%
\definecolor{currentstroke}{rgb}{0.000000,0.000000,0.000000}%
\pgfsetstrokecolor{currentstroke}%
\pgfsetdash{}{0pt}%
\pgfsys@defobject{currentmarker}{\pgfqpoint{-0.027778in}{-0.027778in}}{\pgfqpoint{0.027778in}{0.027778in}}{%
\pgfpathmoveto{\pgfqpoint{0.000000in}{-0.027778in}}%
\pgfpathcurveto{\pgfqpoint{0.007367in}{-0.027778in}}{\pgfqpoint{0.014433in}{-0.024851in}}{\pgfqpoint{0.019642in}{-0.019642in}}%
\pgfpathcurveto{\pgfqpoint{0.024851in}{-0.014433in}}{\pgfqpoint{0.027778in}{-0.007367in}}{\pgfqpoint{0.027778in}{0.000000in}}%
\pgfpathcurveto{\pgfqpoint{0.027778in}{0.007367in}}{\pgfqpoint{0.024851in}{0.014433in}}{\pgfqpoint{0.019642in}{0.019642in}}%
\pgfpathcurveto{\pgfqpoint{0.014433in}{0.024851in}}{\pgfqpoint{0.007367in}{0.027778in}}{\pgfqpoint{0.000000in}{0.027778in}}%
\pgfpathcurveto{\pgfqpoint{-0.007367in}{0.027778in}}{\pgfqpoint{-0.014433in}{0.024851in}}{\pgfqpoint{-0.019642in}{0.019642in}}%
\pgfpathcurveto{\pgfqpoint{-0.024851in}{0.014433in}}{\pgfqpoint{-0.027778in}{0.007367in}}{\pgfqpoint{-0.027778in}{0.000000in}}%
\pgfpathcurveto{\pgfqpoint{-0.027778in}{-0.007367in}}{\pgfqpoint{-0.024851in}{-0.014433in}}{\pgfqpoint{-0.019642in}{-0.019642in}}%
\pgfpathcurveto{\pgfqpoint{-0.014433in}{-0.024851in}}{\pgfqpoint{-0.007367in}{-0.027778in}}{\pgfqpoint{0.000000in}{-0.027778in}}%
\pgfpathclose%
\pgfusepath{stroke,fill}%
}%
\begin{pgfscope}%
\pgfsys@transformshift{2.666826in}{0.935712in}%
\pgfsys@useobject{currentmarker}{}%
\end{pgfscope}%
\end{pgfscope}%
\begin{pgfscope}%
\pgfpathrectangle{\pgfqpoint{0.550713in}{1.728870in}}{\pgfqpoint{3.194133in}{0.696189in}}%
\pgfusepath{clip}%
\pgfsetrectcap%
\pgfsetroundjoin%
\pgfsetlinewidth{0.752812pt}%
\definecolor{currentstroke}{rgb}{0.000000,0.000000,0.000000}%
\pgfsetstrokecolor{currentstroke}%
\pgfsetdash{}{0pt}%
\pgfusepath{stroke}%
\end{pgfscope}%
\begin{pgfscope}%
\pgfpathrectangle{\pgfqpoint{0.550713in}{1.728870in}}{\pgfqpoint{3.194133in}{0.696189in}}%
\pgfusepath{clip}%
\pgfsetbuttcap%
\pgfsetroundjoin%
\definecolor{currentfill}{rgb}{1.000000,1.000000,1.000000}%
\pgfsetfillcolor{currentfill}%
\pgfsetlinewidth{1.003750pt}%
\definecolor{currentstroke}{rgb}{0.000000,0.000000,0.000000}%
\pgfsetstrokecolor{currentstroke}%
\pgfsetdash{}{0pt}%
\pgfsys@defobject{currentmarker}{\pgfqpoint{-0.027778in}{-0.027778in}}{\pgfqpoint{0.027778in}{0.027778in}}{%
\pgfpathmoveto{\pgfqpoint{0.000000in}{-0.027778in}}%
\pgfpathcurveto{\pgfqpoint{0.007367in}{-0.027778in}}{\pgfqpoint{0.014433in}{-0.024851in}}{\pgfqpoint{0.019642in}{-0.019642in}}%
\pgfpathcurveto{\pgfqpoint{0.024851in}{-0.014433in}}{\pgfqpoint{0.027778in}{-0.007367in}}{\pgfqpoint{0.027778in}{0.000000in}}%
\pgfpathcurveto{\pgfqpoint{0.027778in}{0.007367in}}{\pgfqpoint{0.024851in}{0.014433in}}{\pgfqpoint{0.019642in}{0.019642in}}%
\pgfpathcurveto{\pgfqpoint{0.014433in}{0.024851in}}{\pgfqpoint{0.007367in}{0.027778in}}{\pgfqpoint{0.000000in}{0.027778in}}%
\pgfpathcurveto{\pgfqpoint{-0.007367in}{0.027778in}}{\pgfqpoint{-0.014433in}{0.024851in}}{\pgfqpoint{-0.019642in}{0.019642in}}%
\pgfpathcurveto{\pgfqpoint{-0.024851in}{0.014433in}}{\pgfqpoint{-0.027778in}{0.007367in}}{\pgfqpoint{-0.027778in}{0.000000in}}%
\pgfpathcurveto{\pgfqpoint{-0.027778in}{-0.007367in}}{\pgfqpoint{-0.024851in}{-0.014433in}}{\pgfqpoint{-0.019642in}{-0.019642in}}%
\pgfpathcurveto{\pgfqpoint{-0.014433in}{-0.024851in}}{\pgfqpoint{-0.007367in}{-0.027778in}}{\pgfqpoint{0.000000in}{-0.027778in}}%
\pgfpathclose%
\pgfusepath{stroke,fill}%
}%
\begin{pgfscope}%
\pgfsys@transformshift{2.826532in}{1.070012in}%
\pgfsys@useobject{currentmarker}{}%
\end{pgfscope}%
\end{pgfscope}%
\begin{pgfscope}%
\pgfpathrectangle{\pgfqpoint{0.550713in}{1.728870in}}{\pgfqpoint{3.194133in}{0.696189in}}%
\pgfusepath{clip}%
\pgfsetrectcap%
\pgfsetroundjoin%
\pgfsetlinewidth{0.752812pt}%
\definecolor{currentstroke}{rgb}{0.000000,0.000000,0.000000}%
\pgfsetstrokecolor{currentstroke}%
\pgfsetdash{}{0pt}%
\pgfusepath{stroke}%
\end{pgfscope}%
\begin{pgfscope}%
\pgfpathrectangle{\pgfqpoint{0.550713in}{1.728870in}}{\pgfqpoint{3.194133in}{0.696189in}}%
\pgfusepath{clip}%
\pgfsetbuttcap%
\pgfsetroundjoin%
\definecolor{currentfill}{rgb}{1.000000,1.000000,1.000000}%
\pgfsetfillcolor{currentfill}%
\pgfsetlinewidth{1.003750pt}%
\definecolor{currentstroke}{rgb}{0.000000,0.000000,0.000000}%
\pgfsetstrokecolor{currentstroke}%
\pgfsetdash{}{0pt}%
\pgfsys@defobject{currentmarker}{\pgfqpoint{-0.027778in}{-0.027778in}}{\pgfqpoint{0.027778in}{0.027778in}}{%
\pgfpathmoveto{\pgfqpoint{0.000000in}{-0.027778in}}%
\pgfpathcurveto{\pgfqpoint{0.007367in}{-0.027778in}}{\pgfqpoint{0.014433in}{-0.024851in}}{\pgfqpoint{0.019642in}{-0.019642in}}%
\pgfpathcurveto{\pgfqpoint{0.024851in}{-0.014433in}}{\pgfqpoint{0.027778in}{-0.007367in}}{\pgfqpoint{0.027778in}{0.000000in}}%
\pgfpathcurveto{\pgfqpoint{0.027778in}{0.007367in}}{\pgfqpoint{0.024851in}{0.014433in}}{\pgfqpoint{0.019642in}{0.019642in}}%
\pgfpathcurveto{\pgfqpoint{0.014433in}{0.024851in}}{\pgfqpoint{0.007367in}{0.027778in}}{\pgfqpoint{0.000000in}{0.027778in}}%
\pgfpathcurveto{\pgfqpoint{-0.007367in}{0.027778in}}{\pgfqpoint{-0.014433in}{0.024851in}}{\pgfqpoint{-0.019642in}{0.019642in}}%
\pgfpathcurveto{\pgfqpoint{-0.024851in}{0.014433in}}{\pgfqpoint{-0.027778in}{0.007367in}}{\pgfqpoint{-0.027778in}{0.000000in}}%
\pgfpathcurveto{\pgfqpoint{-0.027778in}{-0.007367in}}{\pgfqpoint{-0.024851in}{-0.014433in}}{\pgfqpoint{-0.019642in}{-0.019642in}}%
\pgfpathcurveto{\pgfqpoint{-0.014433in}{-0.024851in}}{\pgfqpoint{-0.007367in}{-0.027778in}}{\pgfqpoint{0.000000in}{-0.027778in}}%
\pgfpathclose%
\pgfusepath{stroke,fill}%
}%
\begin{pgfscope}%
\pgfsys@transformshift{3.066092in}{0.845548in}%
\pgfsys@useobject{currentmarker}{}%
\end{pgfscope}%
\end{pgfscope}%
\begin{pgfscope}%
\pgfpathrectangle{\pgfqpoint{0.550713in}{1.728870in}}{\pgfqpoint{3.194133in}{0.696189in}}%
\pgfusepath{clip}%
\pgfsetrectcap%
\pgfsetroundjoin%
\pgfsetlinewidth{0.752812pt}%
\definecolor{currentstroke}{rgb}{0.000000,0.000000,0.000000}%
\pgfsetstrokecolor{currentstroke}%
\pgfsetdash{}{0pt}%
\pgfusepath{stroke}%
\end{pgfscope}%
\begin{pgfscope}%
\pgfpathrectangle{\pgfqpoint{0.550713in}{1.728870in}}{\pgfqpoint{3.194133in}{0.696189in}}%
\pgfusepath{clip}%
\pgfsetbuttcap%
\pgfsetroundjoin%
\definecolor{currentfill}{rgb}{1.000000,1.000000,1.000000}%
\pgfsetfillcolor{currentfill}%
\pgfsetlinewidth{1.003750pt}%
\definecolor{currentstroke}{rgb}{0.000000,0.000000,0.000000}%
\pgfsetstrokecolor{currentstroke}%
\pgfsetdash{}{0pt}%
\pgfsys@defobject{currentmarker}{\pgfqpoint{-0.027778in}{-0.027778in}}{\pgfqpoint{0.027778in}{0.027778in}}{%
\pgfpathmoveto{\pgfqpoint{0.000000in}{-0.027778in}}%
\pgfpathcurveto{\pgfqpoint{0.007367in}{-0.027778in}}{\pgfqpoint{0.014433in}{-0.024851in}}{\pgfqpoint{0.019642in}{-0.019642in}}%
\pgfpathcurveto{\pgfqpoint{0.024851in}{-0.014433in}}{\pgfqpoint{0.027778in}{-0.007367in}}{\pgfqpoint{0.027778in}{0.000000in}}%
\pgfpathcurveto{\pgfqpoint{0.027778in}{0.007367in}}{\pgfqpoint{0.024851in}{0.014433in}}{\pgfqpoint{0.019642in}{0.019642in}}%
\pgfpathcurveto{\pgfqpoint{0.014433in}{0.024851in}}{\pgfqpoint{0.007367in}{0.027778in}}{\pgfqpoint{0.000000in}{0.027778in}}%
\pgfpathcurveto{\pgfqpoint{-0.007367in}{0.027778in}}{\pgfqpoint{-0.014433in}{0.024851in}}{\pgfqpoint{-0.019642in}{0.019642in}}%
\pgfpathcurveto{\pgfqpoint{-0.024851in}{0.014433in}}{\pgfqpoint{-0.027778in}{0.007367in}}{\pgfqpoint{-0.027778in}{0.000000in}}%
\pgfpathcurveto{\pgfqpoint{-0.027778in}{-0.007367in}}{\pgfqpoint{-0.024851in}{-0.014433in}}{\pgfqpoint{-0.019642in}{-0.019642in}}%
\pgfpathcurveto{\pgfqpoint{-0.014433in}{-0.024851in}}{\pgfqpoint{-0.007367in}{-0.027778in}}{\pgfqpoint{0.000000in}{-0.027778in}}%
\pgfpathclose%
\pgfusepath{stroke,fill}%
}%
\begin{pgfscope}%
\pgfsys@transformshift{3.225799in}{0.871759in}%
\pgfsys@useobject{currentmarker}{}%
\end{pgfscope}%
\end{pgfscope}%
\begin{pgfscope}%
\pgfpathrectangle{\pgfqpoint{0.550713in}{1.728870in}}{\pgfqpoint{3.194133in}{0.696189in}}%
\pgfusepath{clip}%
\pgfsetrectcap%
\pgfsetroundjoin%
\pgfsetlinewidth{0.752812pt}%
\definecolor{currentstroke}{rgb}{0.000000,0.000000,0.000000}%
\pgfsetstrokecolor{currentstroke}%
\pgfsetdash{}{0pt}%
\pgfusepath{stroke}%
\end{pgfscope}%
\begin{pgfscope}%
\pgfpathrectangle{\pgfqpoint{0.550713in}{1.728870in}}{\pgfqpoint{3.194133in}{0.696189in}}%
\pgfusepath{clip}%
\pgfsetbuttcap%
\pgfsetroundjoin%
\definecolor{currentfill}{rgb}{1.000000,1.000000,1.000000}%
\pgfsetfillcolor{currentfill}%
\pgfsetlinewidth{1.003750pt}%
\definecolor{currentstroke}{rgb}{0.000000,0.000000,0.000000}%
\pgfsetstrokecolor{currentstroke}%
\pgfsetdash{}{0pt}%
\pgfsys@defobject{currentmarker}{\pgfqpoint{-0.027778in}{-0.027778in}}{\pgfqpoint{0.027778in}{0.027778in}}{%
\pgfpathmoveto{\pgfqpoint{0.000000in}{-0.027778in}}%
\pgfpathcurveto{\pgfqpoint{0.007367in}{-0.027778in}}{\pgfqpoint{0.014433in}{-0.024851in}}{\pgfqpoint{0.019642in}{-0.019642in}}%
\pgfpathcurveto{\pgfqpoint{0.024851in}{-0.014433in}}{\pgfqpoint{0.027778in}{-0.007367in}}{\pgfqpoint{0.027778in}{0.000000in}}%
\pgfpathcurveto{\pgfqpoint{0.027778in}{0.007367in}}{\pgfqpoint{0.024851in}{0.014433in}}{\pgfqpoint{0.019642in}{0.019642in}}%
\pgfpathcurveto{\pgfqpoint{0.014433in}{0.024851in}}{\pgfqpoint{0.007367in}{0.027778in}}{\pgfqpoint{0.000000in}{0.027778in}}%
\pgfpathcurveto{\pgfqpoint{-0.007367in}{0.027778in}}{\pgfqpoint{-0.014433in}{0.024851in}}{\pgfqpoint{-0.019642in}{0.019642in}}%
\pgfpathcurveto{\pgfqpoint{-0.024851in}{0.014433in}}{\pgfqpoint{-0.027778in}{0.007367in}}{\pgfqpoint{-0.027778in}{0.000000in}}%
\pgfpathcurveto{\pgfqpoint{-0.027778in}{-0.007367in}}{\pgfqpoint{-0.024851in}{-0.014433in}}{\pgfqpoint{-0.019642in}{-0.019642in}}%
\pgfpathcurveto{\pgfqpoint{-0.014433in}{-0.024851in}}{\pgfqpoint{-0.007367in}{-0.027778in}}{\pgfqpoint{0.000000in}{-0.027778in}}%
\pgfpathclose%
\pgfusepath{stroke,fill}%
}%
\begin{pgfscope}%
\pgfsys@transformshift{3.465359in}{0.943826in}%
\pgfsys@useobject{currentmarker}{}%
\end{pgfscope}%
\end{pgfscope}%
\begin{pgfscope}%
\pgfpathrectangle{\pgfqpoint{0.550713in}{1.728870in}}{\pgfqpoint{3.194133in}{0.696189in}}%
\pgfusepath{clip}%
\pgfsetrectcap%
\pgfsetroundjoin%
\pgfsetlinewidth{0.752812pt}%
\definecolor{currentstroke}{rgb}{0.000000,0.000000,0.000000}%
\pgfsetstrokecolor{currentstroke}%
\pgfsetdash{}{0pt}%
\pgfusepath{stroke}%
\end{pgfscope}%
\begin{pgfscope}%
\pgfpathrectangle{\pgfqpoint{0.550713in}{1.728870in}}{\pgfqpoint{3.194133in}{0.696189in}}%
\pgfusepath{clip}%
\pgfsetbuttcap%
\pgfsetroundjoin%
\definecolor{currentfill}{rgb}{1.000000,1.000000,1.000000}%
\pgfsetfillcolor{currentfill}%
\pgfsetlinewidth{1.003750pt}%
\definecolor{currentstroke}{rgb}{0.000000,0.000000,0.000000}%
\pgfsetstrokecolor{currentstroke}%
\pgfsetdash{}{0pt}%
\pgfsys@defobject{currentmarker}{\pgfqpoint{-0.027778in}{-0.027778in}}{\pgfqpoint{0.027778in}{0.027778in}}{%
\pgfpathmoveto{\pgfqpoint{0.000000in}{-0.027778in}}%
\pgfpathcurveto{\pgfqpoint{0.007367in}{-0.027778in}}{\pgfqpoint{0.014433in}{-0.024851in}}{\pgfqpoint{0.019642in}{-0.019642in}}%
\pgfpathcurveto{\pgfqpoint{0.024851in}{-0.014433in}}{\pgfqpoint{0.027778in}{-0.007367in}}{\pgfqpoint{0.027778in}{0.000000in}}%
\pgfpathcurveto{\pgfqpoint{0.027778in}{0.007367in}}{\pgfqpoint{0.024851in}{0.014433in}}{\pgfqpoint{0.019642in}{0.019642in}}%
\pgfpathcurveto{\pgfqpoint{0.014433in}{0.024851in}}{\pgfqpoint{0.007367in}{0.027778in}}{\pgfqpoint{0.000000in}{0.027778in}}%
\pgfpathcurveto{\pgfqpoint{-0.007367in}{0.027778in}}{\pgfqpoint{-0.014433in}{0.024851in}}{\pgfqpoint{-0.019642in}{0.019642in}}%
\pgfpathcurveto{\pgfqpoint{-0.024851in}{0.014433in}}{\pgfqpoint{-0.027778in}{0.007367in}}{\pgfqpoint{-0.027778in}{0.000000in}}%
\pgfpathcurveto{\pgfqpoint{-0.027778in}{-0.007367in}}{\pgfqpoint{-0.024851in}{-0.014433in}}{\pgfqpoint{-0.019642in}{-0.019642in}}%
\pgfpathcurveto{\pgfqpoint{-0.014433in}{-0.024851in}}{\pgfqpoint{-0.007367in}{-0.027778in}}{\pgfqpoint{0.000000in}{-0.027778in}}%
\pgfpathclose%
\pgfusepath{stroke,fill}%
}%
\begin{pgfscope}%
\pgfsys@transformshift{3.625066in}{0.929816in}%
\pgfsys@useobject{currentmarker}{}%
\end{pgfscope}%
\end{pgfscope}%
\begin{pgfscope}%
\pgfsetrectcap%
\pgfsetmiterjoin%
\pgfsetlinewidth{0.803000pt}%
\definecolor{currentstroke}{rgb}{0.000000,0.000000,0.000000}%
\pgfsetstrokecolor{currentstroke}%
\pgfsetdash{}{0pt}%
\pgfpathmoveto{\pgfqpoint{0.550713in}{1.728870in}}%
\pgfpathlineto{\pgfqpoint{0.550713in}{2.425059in}}%
\pgfusepath{stroke}%
\end{pgfscope}%
\begin{pgfscope}%
\pgfsetrectcap%
\pgfsetmiterjoin%
\pgfsetlinewidth{0.803000pt}%
\definecolor{currentstroke}{rgb}{0.000000,0.000000,0.000000}%
\pgfsetstrokecolor{currentstroke}%
\pgfsetdash{}{0pt}%
\pgfpathmoveto{\pgfqpoint{3.744846in}{1.728870in}}%
\pgfpathlineto{\pgfqpoint{3.744846in}{2.425059in}}%
\pgfusepath{stroke}%
\end{pgfscope}%
\begin{pgfscope}%
\pgfsetrectcap%
\pgfsetmiterjoin%
\pgfsetlinewidth{0.803000pt}%
\definecolor{currentstroke}{rgb}{0.000000,0.000000,0.000000}%
\pgfsetstrokecolor{currentstroke}%
\pgfsetdash{}{0pt}%
\pgfpathmoveto{\pgfqpoint{0.550713in}{2.425059in}}%
\pgfpathlineto{\pgfqpoint{3.744846in}{2.425059in}}%
\pgfusepath{stroke}%
\end{pgfscope}%
\begin{pgfscope}%
\pgfsetbuttcap%
\pgfsetmiterjoin%
\definecolor{currentfill}{rgb}{1.000000,1.000000,1.000000}%
\pgfsetfillcolor{currentfill}%
\pgfsetlinewidth{1.003750pt}%
\definecolor{currentstroke}{rgb}{1.000000,1.000000,1.000000}%
\pgfsetstrokecolor{currentstroke}%
\pgfsetdash{}{0pt}%
\pgfpathmoveto{\pgfqpoint{2.736198in}{1.968269in}}%
\pgfpathlineto{\pgfqpoint{3.689290in}{1.968269in}}%
\pgfpathquadraticcurveto{\pgfqpoint{3.717068in}{1.968269in}}{\pgfqpoint{3.717068in}{1.996046in}}%
\pgfpathlineto{\pgfqpoint{3.717068in}{2.369503in}}%
\pgfpathquadraticcurveto{\pgfqpoint{3.717068in}{2.397281in}}{\pgfqpoint{3.689290in}{2.397281in}}%
\pgfpathlineto{\pgfqpoint{2.736198in}{2.397281in}}%
\pgfpathquadraticcurveto{\pgfqpoint{2.708420in}{2.397281in}}{\pgfqpoint{2.708420in}{2.369503in}}%
\pgfpathlineto{\pgfqpoint{2.708420in}{1.996046in}}%
\pgfpathquadraticcurveto{\pgfqpoint{2.708420in}{1.968269in}}{\pgfqpoint{2.736198in}{1.968269in}}%
\pgfpathclose%
\pgfusepath{stroke,fill}%
\end{pgfscope}%
\begin{pgfscope}%
\pgfsetbuttcap%
\pgfsetmiterjoin%
\definecolor{currentfill}{rgb}{0.000000,0.000000,0.000000}%
\pgfsetfillcolor{currentfill}%
\pgfsetlinewidth{0.000000pt}%
\definecolor{currentstroke}{rgb}{0.000000,0.000000,0.000000}%
\pgfsetstrokecolor{currentstroke}%
\pgfsetstrokeopacity{0.000000}%
\pgfsetdash{}{0pt}%
\pgfpathmoveto{\pgfqpoint{2.763976in}{2.244503in}}%
\pgfpathlineto{\pgfqpoint{3.041753in}{2.244503in}}%
\pgfpathlineto{\pgfqpoint{3.041753in}{2.341725in}}%
\pgfpathlineto{\pgfqpoint{2.763976in}{2.341725in}}%
\pgfpathclose%
\pgfusepath{fill}%
\end{pgfscope}%
\begin{pgfscope}%
\definecolor{textcolor}{rgb}{0.000000,0.000000,0.000000}%
\pgfsetstrokecolor{textcolor}%
\pgfsetfillcolor{textcolor}%
\pgftext[x=3.152864in,y=2.244503in,left,base]{\color{textcolor}\rmfamily\fontsize{10.000000}{12.000000}\selectfont \(\displaystyle \mu_0=0\)}%
\end{pgfscope}%
\begin{pgfscope}%
\pgfsetbuttcap%
\pgfsetmiterjoin%
\definecolor{currentfill}{rgb}{0.811765,0.819608,0.823529}%
\pgfsetfillcolor{currentfill}%
\pgfsetlinewidth{0.000000pt}%
\definecolor{currentstroke}{rgb}{0.000000,0.000000,0.000000}%
\pgfsetstrokecolor{currentstroke}%
\pgfsetstrokeopacity{0.000000}%
\pgfsetdash{}{0pt}%
\pgfpathmoveto{\pgfqpoint{2.763976in}{2.050830in}}%
\pgfpathlineto{\pgfqpoint{3.041753in}{2.050830in}}%
\pgfpathlineto{\pgfqpoint{3.041753in}{2.148053in}}%
\pgfpathlineto{\pgfqpoint{2.763976in}{2.148053in}}%
\pgfpathclose%
\pgfusepath{fill}%
\end{pgfscope}%
\begin{pgfscope}%
\definecolor{textcolor}{rgb}{0.000000,0.000000,0.000000}%
\pgfsetstrokecolor{textcolor}%
\pgfsetfillcolor{textcolor}%
\pgftext[x=3.152864in,y=2.050830in,left,base]{\color{textcolor}\rmfamily\fontsize{10.000000}{12.000000}\selectfont \(\displaystyle \mu_0=-1\)}%
\end{pgfscope}%
\begin{pgfscope}%
\pgfsetbuttcap%
\pgfsetmiterjoin%
\definecolor{currentfill}{rgb}{1.000000,1.000000,1.000000}%
\pgfsetfillcolor{currentfill}%
\pgfsetlinewidth{0.000000pt}%
\definecolor{currentstroke}{rgb}{0.000000,0.000000,0.000000}%
\pgfsetstrokecolor{currentstroke}%
\pgfsetstrokeopacity{0.000000}%
\pgfsetdash{}{0pt}%
\pgfpathmoveto{\pgfqpoint{0.550713in}{0.127635in}}%
\pgfpathlineto{\pgfqpoint{3.744846in}{0.127635in}}%
\pgfpathlineto{\pgfqpoint{3.744846in}{1.624441in}}%
\pgfpathlineto{\pgfqpoint{0.550713in}{1.624441in}}%
\pgfpathclose%
\pgfusepath{fill}%
\end{pgfscope}%
\begin{pgfscope}%
\pgfpathrectangle{\pgfqpoint{0.550713in}{0.127635in}}{\pgfqpoint{3.194133in}{1.496807in}}%
\pgfusepath{clip}%
\pgfsetbuttcap%
\pgfsetmiterjoin%
\definecolor{currentfill}{rgb}{0.631373,0.062745,0.207843}%
\pgfsetfillcolor{currentfill}%
\pgfsetlinewidth{0.752812pt}%
\definecolor{currentstroke}{rgb}{0.000000,0.000000,0.000000}%
\pgfsetstrokecolor{currentstroke}%
\pgfsetdash{}{0pt}%
\pgfpathmoveto{\pgfqpoint{0.592236in}{0.637564in}}%
\pgfpathlineto{\pgfqpoint{0.748749in}{0.637564in}}%
\pgfpathlineto{\pgfqpoint{0.748749in}{0.856626in}}%
\pgfpathlineto{\pgfqpoint{0.592236in}{0.856626in}}%
\pgfpathlineto{\pgfqpoint{0.592236in}{0.637564in}}%
\pgfpathclose%
\pgfusepath{stroke,fill}%
\end{pgfscope}%
\begin{pgfscope}%
\pgfpathrectangle{\pgfqpoint{0.550713in}{0.127635in}}{\pgfqpoint{3.194133in}{1.496807in}}%
\pgfusepath{clip}%
\pgfsetbuttcap%
\pgfsetmiterjoin%
\definecolor{currentfill}{rgb}{0.898039,0.772549,0.752941}%
\pgfsetfillcolor{currentfill}%
\pgfsetlinewidth{0.752812pt}%
\definecolor{currentstroke}{rgb}{0.000000,0.000000,0.000000}%
\pgfsetstrokecolor{currentstroke}%
\pgfsetdash{}{0pt}%
\pgfpathmoveto{\pgfqpoint{0.751943in}{1.281515in}}%
\pgfpathlineto{\pgfqpoint{0.908456in}{1.281515in}}%
\pgfpathlineto{\pgfqpoint{0.908456in}{1.939567in}}%
\pgfpathlineto{\pgfqpoint{0.751943in}{1.939567in}}%
\pgfpathlineto{\pgfqpoint{0.751943in}{1.281515in}}%
\pgfpathclose%
\pgfusepath{stroke,fill}%
\end{pgfscope}%
\begin{pgfscope}%
\pgfpathrectangle{\pgfqpoint{0.550713in}{0.127635in}}{\pgfqpoint{3.194133in}{1.496807in}}%
\pgfusepath{clip}%
\pgfsetbuttcap%
\pgfsetmiterjoin%
\definecolor{currentfill}{rgb}{0.890196,0.000000,0.400000}%
\pgfsetfillcolor{currentfill}%
\pgfsetlinewidth{0.752812pt}%
\definecolor{currentstroke}{rgb}{0.000000,0.000000,0.000000}%
\pgfsetstrokecolor{currentstroke}%
\pgfsetdash{}{0pt}%
\pgfpathmoveto{\pgfqpoint{0.991503in}{0.662749in}}%
\pgfpathlineto{\pgfqpoint{1.148015in}{0.662749in}}%
\pgfpathlineto{\pgfqpoint{1.148015in}{1.257895in}}%
\pgfpathlineto{\pgfqpoint{0.991503in}{1.257895in}}%
\pgfpathlineto{\pgfqpoint{0.991503in}{0.662749in}}%
\pgfpathclose%
\pgfusepath{stroke,fill}%
\end{pgfscope}%
\begin{pgfscope}%
\pgfpathrectangle{\pgfqpoint{0.550713in}{0.127635in}}{\pgfqpoint{3.194133in}{1.496807in}}%
\pgfusepath{clip}%
\pgfsetbuttcap%
\pgfsetmiterjoin%
\definecolor{currentfill}{rgb}{0.976471,0.823529,0.854902}%
\pgfsetfillcolor{currentfill}%
\pgfsetlinewidth{0.752812pt}%
\definecolor{currentstroke}{rgb}{0.000000,0.000000,0.000000}%
\pgfsetstrokecolor{currentstroke}%
\pgfsetdash{}{0pt}%
\pgfpathmoveto{\pgfqpoint{1.151210in}{2.090162in}}%
\pgfpathlineto{\pgfqpoint{1.307722in}{2.090162in}}%
\pgfpathlineto{\pgfqpoint{1.307722in}{2.659244in}}%
\pgfpathlineto{\pgfqpoint{1.151210in}{2.659244in}}%
\pgfpathlineto{\pgfqpoint{1.151210in}{2.090162in}}%
\pgfpathclose%
\pgfusepath{stroke,fill}%
\end{pgfscope}%
\begin{pgfscope}%
\pgfpathrectangle{\pgfqpoint{0.550713in}{0.127635in}}{\pgfqpoint{3.194133in}{1.496807in}}%
\pgfusepath{clip}%
\pgfsetbuttcap%
\pgfsetmiterjoin%
\definecolor{currentfill}{rgb}{0.000000,0.329412,0.623529}%
\pgfsetfillcolor{currentfill}%
\pgfsetlinewidth{0.752812pt}%
\definecolor{currentstroke}{rgb}{0.000000,0.000000,0.000000}%
\pgfsetstrokecolor{currentstroke}%
\pgfsetdash{}{0pt}%
\pgfpathmoveto{\pgfqpoint{1.390770in}{0.640227in}}%
\pgfpathlineto{\pgfqpoint{1.547282in}{0.640227in}}%
\pgfpathlineto{\pgfqpoint{1.547282in}{0.845077in}}%
\pgfpathlineto{\pgfqpoint{1.390770in}{0.845077in}}%
\pgfpathlineto{\pgfqpoint{1.390770in}{0.640227in}}%
\pgfpathclose%
\pgfusepath{stroke,fill}%
\end{pgfscope}%
\begin{pgfscope}%
\pgfpathrectangle{\pgfqpoint{0.550713in}{0.127635in}}{\pgfqpoint{3.194133in}{1.496807in}}%
\pgfusepath{clip}%
\pgfsetbuttcap%
\pgfsetmiterjoin%
\definecolor{currentfill}{rgb}{0.780392,0.866667,0.949020}%
\pgfsetfillcolor{currentfill}%
\pgfsetlinewidth{0.752812pt}%
\definecolor{currentstroke}{rgb}{0.000000,0.000000,0.000000}%
\pgfsetstrokecolor{currentstroke}%
\pgfsetdash{}{0pt}%
\pgfpathmoveto{\pgfqpoint{1.550476in}{0.810784in}}%
\pgfpathlineto{\pgfqpoint{1.706989in}{0.810784in}}%
\pgfpathlineto{\pgfqpoint{1.706989in}{0.963664in}}%
\pgfpathlineto{\pgfqpoint{1.550476in}{0.963664in}}%
\pgfpathlineto{\pgfqpoint{1.550476in}{0.810784in}}%
\pgfpathclose%
\pgfusepath{stroke,fill}%
\end{pgfscope}%
\begin{pgfscope}%
\pgfpathrectangle{\pgfqpoint{0.550713in}{0.127635in}}{\pgfqpoint{3.194133in}{1.496807in}}%
\pgfusepath{clip}%
\pgfsetbuttcap%
\pgfsetmiterjoin%
\definecolor{currentfill}{rgb}{0.000000,0.380392,0.396078}%
\pgfsetfillcolor{currentfill}%
\pgfsetlinewidth{0.752812pt}%
\definecolor{currentstroke}{rgb}{0.000000,0.000000,0.000000}%
\pgfsetstrokecolor{currentstroke}%
\pgfsetdash{}{0pt}%
\pgfpathmoveto{\pgfqpoint{1.790036in}{0.860695in}}%
\pgfpathlineto{\pgfqpoint{1.946549in}{0.860695in}}%
\pgfpathlineto{\pgfqpoint{1.946549in}{1.074473in}}%
\pgfpathlineto{\pgfqpoint{1.790036in}{1.074473in}}%
\pgfpathlineto{\pgfqpoint{1.790036in}{0.860695in}}%
\pgfpathclose%
\pgfusepath{stroke,fill}%
\end{pgfscope}%
\begin{pgfscope}%
\pgfpathrectangle{\pgfqpoint{0.550713in}{0.127635in}}{\pgfqpoint{3.194133in}{1.496807in}}%
\pgfusepath{clip}%
\pgfsetbuttcap%
\pgfsetmiterjoin%
\definecolor{currentfill}{rgb}{0.749020,0.815686,0.819608}%
\pgfsetfillcolor{currentfill}%
\pgfsetlinewidth{0.752812pt}%
\definecolor{currentstroke}{rgb}{0.000000,0.000000,0.000000}%
\pgfsetstrokecolor{currentstroke}%
\pgfsetdash{}{0pt}%
\pgfpathmoveto{\pgfqpoint{1.949743in}{1.008655in}}%
\pgfpathlineto{\pgfqpoint{2.106255in}{1.008655in}}%
\pgfpathlineto{\pgfqpoint{2.106255in}{1.082756in}}%
\pgfpathlineto{\pgfqpoint{1.949743in}{1.082756in}}%
\pgfpathlineto{\pgfqpoint{1.949743in}{1.008655in}}%
\pgfpathclose%
\pgfusepath{stroke,fill}%
\end{pgfscope}%
\begin{pgfscope}%
\pgfpathrectangle{\pgfqpoint{0.550713in}{0.127635in}}{\pgfqpoint{3.194133in}{1.496807in}}%
\pgfusepath{clip}%
\pgfsetbuttcap%
\pgfsetmiterjoin%
\definecolor{currentfill}{rgb}{0.380392,0.129412,0.345098}%
\pgfsetfillcolor{currentfill}%
\pgfsetlinewidth{0.752812pt}%
\definecolor{currentstroke}{rgb}{0.000000,0.000000,0.000000}%
\pgfsetstrokecolor{currentstroke}%
\pgfsetdash{}{0pt}%
\pgfpathmoveto{\pgfqpoint{2.189303in}{0.556240in}}%
\pgfpathlineto{\pgfqpoint{2.345815in}{0.556240in}}%
\pgfpathlineto{\pgfqpoint{2.345815in}{0.733068in}}%
\pgfpathlineto{\pgfqpoint{2.189303in}{0.733068in}}%
\pgfpathlineto{\pgfqpoint{2.189303in}{0.556240in}}%
\pgfpathclose%
\pgfusepath{stroke,fill}%
\end{pgfscope}%
\begin{pgfscope}%
\pgfpathrectangle{\pgfqpoint{0.550713in}{0.127635in}}{\pgfqpoint{3.194133in}{1.496807in}}%
\pgfusepath{clip}%
\pgfsetbuttcap%
\pgfsetmiterjoin%
\definecolor{currentfill}{rgb}{0.823529,0.752941,0.803922}%
\pgfsetfillcolor{currentfill}%
\pgfsetlinewidth{0.752812pt}%
\definecolor{currentstroke}{rgb}{0.000000,0.000000,0.000000}%
\pgfsetstrokecolor{currentstroke}%
\pgfsetdash{}{0pt}%
\pgfpathmoveto{\pgfqpoint{2.349010in}{0.677003in}}%
\pgfpathlineto{\pgfqpoint{2.505522in}{0.677003in}}%
\pgfpathlineto{\pgfqpoint{2.505522in}{0.919837in}}%
\pgfpathlineto{\pgfqpoint{2.349010in}{0.919837in}}%
\pgfpathlineto{\pgfqpoint{2.349010in}{0.677003in}}%
\pgfpathclose%
\pgfusepath{stroke,fill}%
\end{pgfscope}%
\begin{pgfscope}%
\pgfpathrectangle{\pgfqpoint{0.550713in}{0.127635in}}{\pgfqpoint{3.194133in}{1.496807in}}%
\pgfusepath{clip}%
\pgfsetbuttcap%
\pgfsetmiterjoin%
\definecolor{currentfill}{rgb}{0.964706,0.658824,0.000000}%
\pgfsetfillcolor{currentfill}%
\pgfsetlinewidth{0.752812pt}%
\definecolor{currentstroke}{rgb}{0.000000,0.000000,0.000000}%
\pgfsetstrokecolor{currentstroke}%
\pgfsetdash{}{0pt}%
\pgfpathmoveto{\pgfqpoint{2.588570in}{0.648291in}}%
\pgfpathlineto{\pgfqpoint{2.745082in}{0.648291in}}%
\pgfpathlineto{\pgfqpoint{2.745082in}{0.820171in}}%
\pgfpathlineto{\pgfqpoint{2.588570in}{0.820171in}}%
\pgfpathlineto{\pgfqpoint{2.588570in}{0.648291in}}%
\pgfpathclose%
\pgfusepath{stroke,fill}%
\end{pgfscope}%
\begin{pgfscope}%
\pgfpathrectangle{\pgfqpoint{0.550713in}{0.127635in}}{\pgfqpoint{3.194133in}{1.496807in}}%
\pgfusepath{clip}%
\pgfsetbuttcap%
\pgfsetmiterjoin%
\definecolor{currentfill}{rgb}{0.996078,0.917647,0.788235}%
\pgfsetfillcolor{currentfill}%
\pgfsetlinewidth{0.752812pt}%
\definecolor{currentstroke}{rgb}{0.000000,0.000000,0.000000}%
\pgfsetstrokecolor{currentstroke}%
\pgfsetdash{}{0pt}%
\pgfpathmoveto{\pgfqpoint{2.748276in}{0.900490in}}%
\pgfpathlineto{\pgfqpoint{2.904789in}{0.900490in}}%
\pgfpathlineto{\pgfqpoint{2.904789in}{0.996464in}}%
\pgfpathlineto{\pgfqpoint{2.748276in}{0.996464in}}%
\pgfpathlineto{\pgfqpoint{2.748276in}{0.900490in}}%
\pgfpathclose%
\pgfusepath{stroke,fill}%
\end{pgfscope}%
\begin{pgfscope}%
\pgfpathrectangle{\pgfqpoint{0.550713in}{0.127635in}}{\pgfqpoint{3.194133in}{1.496807in}}%
\pgfusepath{clip}%
\pgfsetbuttcap%
\pgfsetmiterjoin%
\definecolor{currentfill}{rgb}{0.341176,0.670588,0.152941}%
\pgfsetfillcolor{currentfill}%
\pgfsetlinewidth{0.752812pt}%
\definecolor{currentstroke}{rgb}{0.000000,0.000000,0.000000}%
\pgfsetstrokecolor{currentstroke}%
\pgfsetdash{}{0pt}%
\pgfpathmoveto{\pgfqpoint{2.987836in}{0.567466in}}%
\pgfpathlineto{\pgfqpoint{3.144349in}{0.567466in}}%
\pgfpathlineto{\pgfqpoint{3.144349in}{0.668492in}}%
\pgfpathlineto{\pgfqpoint{2.987836in}{0.668492in}}%
\pgfpathlineto{\pgfqpoint{2.987836in}{0.567466in}}%
\pgfpathclose%
\pgfusepath{stroke,fill}%
\end{pgfscope}%
\begin{pgfscope}%
\pgfpathrectangle{\pgfqpoint{0.550713in}{0.127635in}}{\pgfqpoint{3.194133in}{1.496807in}}%
\pgfusepath{clip}%
\pgfsetbuttcap%
\pgfsetmiterjoin%
\definecolor{currentfill}{rgb}{0.866667,0.921569,0.807843}%
\pgfsetfillcolor{currentfill}%
\pgfsetlinewidth{0.752812pt}%
\definecolor{currentstroke}{rgb}{0.000000,0.000000,0.000000}%
\pgfsetstrokecolor{currentstroke}%
\pgfsetdash{}{0pt}%
\pgfpathmoveto{\pgfqpoint{3.147543in}{0.580119in}}%
\pgfpathlineto{\pgfqpoint{3.304055in}{0.580119in}}%
\pgfpathlineto{\pgfqpoint{3.304055in}{0.745682in}}%
\pgfpathlineto{\pgfqpoint{3.147543in}{0.745682in}}%
\pgfpathlineto{\pgfqpoint{3.147543in}{0.580119in}}%
\pgfpathclose%
\pgfusepath{stroke,fill}%
\end{pgfscope}%
\begin{pgfscope}%
\pgfpathrectangle{\pgfqpoint{0.550713in}{0.127635in}}{\pgfqpoint{3.194133in}{1.496807in}}%
\pgfusepath{clip}%
\pgfsetbuttcap%
\pgfsetmiterjoin%
\definecolor{currentfill}{rgb}{0.478431,0.435294,0.674510}%
\pgfsetfillcolor{currentfill}%
\pgfsetlinewidth{0.752812pt}%
\definecolor{currentstroke}{rgb}{0.000000,0.000000,0.000000}%
\pgfsetstrokecolor{currentstroke}%
\pgfsetdash{}{0pt}%
\pgfpathmoveto{\pgfqpoint{3.387103in}{0.717168in}}%
\pgfpathlineto{\pgfqpoint{3.543615in}{0.717168in}}%
\pgfpathlineto{\pgfqpoint{3.543615in}{0.811149in}}%
\pgfpathlineto{\pgfqpoint{3.387103in}{0.811149in}}%
\pgfpathlineto{\pgfqpoint{3.387103in}{0.717168in}}%
\pgfpathclose%
\pgfusepath{stroke,fill}%
\end{pgfscope}%
\begin{pgfscope}%
\pgfpathrectangle{\pgfqpoint{0.550713in}{0.127635in}}{\pgfqpoint{3.194133in}{1.496807in}}%
\pgfusepath{clip}%
\pgfsetbuttcap%
\pgfsetmiterjoin%
\definecolor{currentfill}{rgb}{0.870588,0.854902,0.921569}%
\pgfsetfillcolor{currentfill}%
\pgfsetlinewidth{0.752812pt}%
\definecolor{currentstroke}{rgb}{0.000000,0.000000,0.000000}%
\pgfsetstrokecolor{currentstroke}%
\pgfsetdash{}{0pt}%
\pgfpathmoveto{\pgfqpoint{3.546809in}{0.646870in}}%
\pgfpathlineto{\pgfqpoint{3.703322in}{0.646870in}}%
\pgfpathlineto{\pgfqpoint{3.703322in}{0.754496in}}%
\pgfpathlineto{\pgfqpoint{3.546809in}{0.754496in}}%
\pgfpathlineto{\pgfqpoint{3.546809in}{0.646870in}}%
\pgfpathclose%
\pgfusepath{stroke,fill}%
\end{pgfscope}%
\begin{pgfscope}%
\pgfpathrectangle{\pgfqpoint{0.550713in}{0.127635in}}{\pgfqpoint{3.194133in}{1.496807in}}%
\pgfusepath{clip}%
\pgfsetbuttcap%
\pgfsetmiterjoin%
\definecolor{currentfill}{rgb}{0.000000,0.000000,0.000000}%
\pgfsetfillcolor{currentfill}%
\pgfsetlinewidth{0.376406pt}%
\definecolor{currentstroke}{rgb}{0.000000,0.000000,0.000000}%
\pgfsetstrokecolor{currentstroke}%
\pgfsetdash{}{0pt}%
\pgfpathmoveto{\pgfqpoint{0.750346in}{0.127635in}}%
\pgfpathlineto{\pgfqpoint{0.750346in}{0.127635in}}%
\pgfpathlineto{\pgfqpoint{0.750346in}{0.127635in}}%
\pgfpathlineto{\pgfqpoint{0.750346in}{0.127635in}}%
\pgfpathclose%
\pgfusepath{stroke,fill}%
\end{pgfscope}%
\begin{pgfscope}%
\pgfpathrectangle{\pgfqpoint{0.550713in}{0.127635in}}{\pgfqpoint{3.194133in}{1.496807in}}%
\pgfusepath{clip}%
\pgfsetbuttcap%
\pgfsetmiterjoin%
\definecolor{currentfill}{rgb}{0.813235,0.819118,0.822059}%
\pgfsetfillcolor{currentfill}%
\pgfsetlinewidth{0.376406pt}%
\definecolor{currentstroke}{rgb}{0.000000,0.000000,0.000000}%
\pgfsetstrokecolor{currentstroke}%
\pgfsetdash{}{0pt}%
\pgfpathmoveto{\pgfqpoint{0.750346in}{0.127635in}}%
\pgfpathlineto{\pgfqpoint{0.750346in}{0.127635in}}%
\pgfpathlineto{\pgfqpoint{0.750346in}{0.127635in}}%
\pgfpathlineto{\pgfqpoint{0.750346in}{0.127635in}}%
\pgfpathclose%
\pgfusepath{stroke,fill}%
\end{pgfscope}%
\begin{pgfscope}%
\pgfsetbuttcap%
\pgfsetroundjoin%
\definecolor{currentfill}{rgb}{0.000000,0.000000,0.000000}%
\pgfsetfillcolor{currentfill}%
\pgfsetlinewidth{0.803000pt}%
\definecolor{currentstroke}{rgb}{0.000000,0.000000,0.000000}%
\pgfsetstrokecolor{currentstroke}%
\pgfsetdash{}{0pt}%
\pgfsys@defobject{currentmarker}{\pgfqpoint{-0.048611in}{0.000000in}}{\pgfqpoint{-0.000000in}{0.000000in}}{%
\pgfpathmoveto{\pgfqpoint{-0.000000in}{0.000000in}}%
\pgfpathlineto{\pgfqpoint{-0.048611in}{0.000000in}}%
\pgfusepath{stroke,fill}%
}%
\begin{pgfscope}%
\pgfsys@transformshift{0.550713in}{0.127635in}%
\pgfsys@useobject{currentmarker}{}%
\end{pgfscope}%
\end{pgfscope}%
\begin{pgfscope}%
\definecolor{textcolor}{rgb}{0.000000,0.000000,0.000000}%
\pgfsetstrokecolor{textcolor}%
\pgfsetfillcolor{textcolor}%
\pgftext[x=0.384046in, y=0.079440in, left, base]{\color{textcolor}\rmfamily\fontsize{10.000000}{12.000000}\selectfont \(\displaystyle {0}\)}%
\end{pgfscope}%
\begin{pgfscope}%
\pgfsetbuttcap%
\pgfsetroundjoin%
\definecolor{currentfill}{rgb}{0.000000,0.000000,0.000000}%
\pgfsetfillcolor{currentfill}%
\pgfsetlinewidth{0.803000pt}%
\definecolor{currentstroke}{rgb}{0.000000,0.000000,0.000000}%
\pgfsetstrokecolor{currentstroke}%
\pgfsetdash{}{0pt}%
\pgfsys@defobject{currentmarker}{\pgfqpoint{-0.048611in}{0.000000in}}{\pgfqpoint{-0.000000in}{0.000000in}}{%
\pgfpathmoveto{\pgfqpoint{-0.000000in}{0.000000in}}%
\pgfpathlineto{\pgfqpoint{-0.048611in}{0.000000in}}%
\pgfusepath{stroke,fill}%
}%
\begin{pgfscope}%
\pgfsys@transformshift{0.550713in}{0.626570in}%
\pgfsys@useobject{currentmarker}{}%
\end{pgfscope}%
\end{pgfscope}%
\begin{pgfscope}%
\definecolor{textcolor}{rgb}{0.000000,0.000000,0.000000}%
\pgfsetstrokecolor{textcolor}%
\pgfsetfillcolor{textcolor}%
\pgftext[x=0.245156in, y=0.578376in, left, base]{\color{textcolor}\rmfamily\fontsize{10.000000}{12.000000}\selectfont \(\displaystyle {100}\)}%
\end{pgfscope}%
\begin{pgfscope}%
\pgfsetbuttcap%
\pgfsetroundjoin%
\definecolor{currentfill}{rgb}{0.000000,0.000000,0.000000}%
\pgfsetfillcolor{currentfill}%
\pgfsetlinewidth{0.803000pt}%
\definecolor{currentstroke}{rgb}{0.000000,0.000000,0.000000}%
\pgfsetstrokecolor{currentstroke}%
\pgfsetdash{}{0pt}%
\pgfsys@defobject{currentmarker}{\pgfqpoint{-0.048611in}{0.000000in}}{\pgfqpoint{-0.000000in}{0.000000in}}{%
\pgfpathmoveto{\pgfqpoint{-0.000000in}{0.000000in}}%
\pgfpathlineto{\pgfqpoint{-0.048611in}{0.000000in}}%
\pgfusepath{stroke,fill}%
}%
\begin{pgfscope}%
\pgfsys@transformshift{0.550713in}{1.125506in}%
\pgfsys@useobject{currentmarker}{}%
\end{pgfscope}%
\end{pgfscope}%
\begin{pgfscope}%
\definecolor{textcolor}{rgb}{0.000000,0.000000,0.000000}%
\pgfsetstrokecolor{textcolor}%
\pgfsetfillcolor{textcolor}%
\pgftext[x=0.245156in, y=1.077311in, left, base]{\color{textcolor}\rmfamily\fontsize{10.000000}{12.000000}\selectfont \(\displaystyle {200}\)}%
\end{pgfscope}%
\begin{pgfscope}%
\pgfsetbuttcap%
\pgfsetroundjoin%
\definecolor{currentfill}{rgb}{0.000000,0.000000,0.000000}%
\pgfsetfillcolor{currentfill}%
\pgfsetlinewidth{0.803000pt}%
\definecolor{currentstroke}{rgb}{0.000000,0.000000,0.000000}%
\pgfsetstrokecolor{currentstroke}%
\pgfsetdash{}{0pt}%
\pgfsys@defobject{currentmarker}{\pgfqpoint{-0.048611in}{0.000000in}}{\pgfqpoint{-0.000000in}{0.000000in}}{%
\pgfpathmoveto{\pgfqpoint{-0.000000in}{0.000000in}}%
\pgfpathlineto{\pgfqpoint{-0.048611in}{0.000000in}}%
\pgfusepath{stroke,fill}%
}%
\begin{pgfscope}%
\pgfsys@transformshift{0.550713in}{1.624441in}%
\pgfsys@useobject{currentmarker}{}%
\end{pgfscope}%
\end{pgfscope}%
\begin{pgfscope}%
\definecolor{textcolor}{rgb}{0.000000,0.000000,0.000000}%
\pgfsetstrokecolor{textcolor}%
\pgfsetfillcolor{textcolor}%
\pgftext[x=0.245156in, y=1.576247in, left, base]{\color{textcolor}\rmfamily\fontsize{10.000000}{12.000000}\selectfont \(\displaystyle {300}\)}%
\end{pgfscope}%
\begin{pgfscope}%
\pgfpathrectangle{\pgfqpoint{0.550713in}{0.127635in}}{\pgfqpoint{3.194133in}{1.496807in}}%
\pgfusepath{clip}%
\pgfsetbuttcap%
\pgfsetroundjoin%
\pgfsetlinewidth{0.501875pt}%
\definecolor{currentstroke}{rgb}{0.392157,0.396078,0.403922}%
\pgfsetstrokecolor{currentstroke}%
\pgfsetdash{}{0pt}%
\pgfpathmoveto{\pgfqpoint{0.949979in}{0.127635in}}%
\pgfpathlineto{\pgfqpoint{0.949979in}{1.634441in}}%
\pgfusepath{stroke}%
\end{pgfscope}%
\begin{pgfscope}%
\pgfpathrectangle{\pgfqpoint{0.550713in}{0.127635in}}{\pgfqpoint{3.194133in}{1.496807in}}%
\pgfusepath{clip}%
\pgfsetbuttcap%
\pgfsetroundjoin%
\pgfsetlinewidth{0.501875pt}%
\definecolor{currentstroke}{rgb}{0.392157,0.396078,0.403922}%
\pgfsetstrokecolor{currentstroke}%
\pgfsetdash{}{0pt}%
\pgfpathmoveto{\pgfqpoint{1.349246in}{0.127635in}}%
\pgfpathlineto{\pgfqpoint{1.349246in}{1.634441in}}%
\pgfusepath{stroke}%
\end{pgfscope}%
\begin{pgfscope}%
\pgfpathrectangle{\pgfqpoint{0.550713in}{0.127635in}}{\pgfqpoint{3.194133in}{1.496807in}}%
\pgfusepath{clip}%
\pgfsetbuttcap%
\pgfsetroundjoin%
\pgfsetlinewidth{0.501875pt}%
\definecolor{currentstroke}{rgb}{0.392157,0.396078,0.403922}%
\pgfsetstrokecolor{currentstroke}%
\pgfsetdash{}{0pt}%
\pgfpathmoveto{\pgfqpoint{1.748513in}{0.127635in}}%
\pgfpathlineto{\pgfqpoint{1.748513in}{1.634441in}}%
\pgfusepath{stroke}%
\end{pgfscope}%
\begin{pgfscope}%
\pgfpathrectangle{\pgfqpoint{0.550713in}{0.127635in}}{\pgfqpoint{3.194133in}{1.496807in}}%
\pgfusepath{clip}%
\pgfsetbuttcap%
\pgfsetroundjoin%
\pgfsetlinewidth{0.501875pt}%
\definecolor{currentstroke}{rgb}{0.392157,0.396078,0.403922}%
\pgfsetstrokecolor{currentstroke}%
\pgfsetdash{}{0pt}%
\pgfpathmoveto{\pgfqpoint{2.147779in}{0.127635in}}%
\pgfpathlineto{\pgfqpoint{2.147779in}{1.634441in}}%
\pgfusepath{stroke}%
\end{pgfscope}%
\begin{pgfscope}%
\pgfpathrectangle{\pgfqpoint{0.550713in}{0.127635in}}{\pgfqpoint{3.194133in}{1.496807in}}%
\pgfusepath{clip}%
\pgfsetbuttcap%
\pgfsetroundjoin%
\pgfsetlinewidth{0.501875pt}%
\definecolor{currentstroke}{rgb}{0.392157,0.396078,0.403922}%
\pgfsetstrokecolor{currentstroke}%
\pgfsetdash{}{0pt}%
\pgfpathmoveto{\pgfqpoint{2.547046in}{0.127635in}}%
\pgfpathlineto{\pgfqpoint{2.547046in}{1.634441in}}%
\pgfusepath{stroke}%
\end{pgfscope}%
\begin{pgfscope}%
\pgfpathrectangle{\pgfqpoint{0.550713in}{0.127635in}}{\pgfqpoint{3.194133in}{1.496807in}}%
\pgfusepath{clip}%
\pgfsetbuttcap%
\pgfsetroundjoin%
\pgfsetlinewidth{0.501875pt}%
\definecolor{currentstroke}{rgb}{0.392157,0.396078,0.403922}%
\pgfsetstrokecolor{currentstroke}%
\pgfsetdash{}{0pt}%
\pgfpathmoveto{\pgfqpoint{2.946312in}{0.127635in}}%
\pgfpathlineto{\pgfqpoint{2.946312in}{1.634441in}}%
\pgfusepath{stroke}%
\end{pgfscope}%
\begin{pgfscope}%
\pgfpathrectangle{\pgfqpoint{0.550713in}{0.127635in}}{\pgfqpoint{3.194133in}{1.496807in}}%
\pgfusepath{clip}%
\pgfsetbuttcap%
\pgfsetroundjoin%
\pgfsetlinewidth{0.501875pt}%
\definecolor{currentstroke}{rgb}{0.392157,0.396078,0.403922}%
\pgfsetstrokecolor{currentstroke}%
\pgfsetdash{}{0pt}%
\pgfpathmoveto{\pgfqpoint{3.345579in}{0.127635in}}%
\pgfpathlineto{\pgfqpoint{3.345579in}{1.634441in}}%
\pgfusepath{stroke}%
\end{pgfscope}%
\begin{pgfscope}%
\pgfpathrectangle{\pgfqpoint{0.550713in}{0.127635in}}{\pgfqpoint{3.194133in}{1.496807in}}%
\pgfusepath{clip}%
\pgfsetrectcap%
\pgfsetroundjoin%
\pgfsetlinewidth{0.752812pt}%
\definecolor{currentstroke}{rgb}{0.000000,0.000000,0.000000}%
\pgfsetstrokecolor{currentstroke}%
\pgfsetdash{}{0pt}%
\pgfpathmoveto{\pgfqpoint{0.670493in}{0.637564in}}%
\pgfpathlineto{\pgfqpoint{0.670493in}{0.466732in}}%
\pgfusepath{stroke}%
\end{pgfscope}%
\begin{pgfscope}%
\pgfpathrectangle{\pgfqpoint{0.550713in}{0.127635in}}{\pgfqpoint{3.194133in}{1.496807in}}%
\pgfusepath{clip}%
\pgfsetrectcap%
\pgfsetroundjoin%
\pgfsetlinewidth{0.752812pt}%
\definecolor{currentstroke}{rgb}{0.000000,0.000000,0.000000}%
\pgfsetstrokecolor{currentstroke}%
\pgfsetdash{}{0pt}%
\pgfpathmoveto{\pgfqpoint{0.670493in}{0.856626in}}%
\pgfpathlineto{\pgfqpoint{0.670493in}{1.147933in}}%
\pgfusepath{stroke}%
\end{pgfscope}%
\begin{pgfscope}%
\pgfpathrectangle{\pgfqpoint{0.550713in}{0.127635in}}{\pgfqpoint{3.194133in}{1.496807in}}%
\pgfusepath{clip}%
\pgfsetrectcap%
\pgfsetroundjoin%
\pgfsetlinewidth{0.752812pt}%
\definecolor{currentstroke}{rgb}{0.000000,0.000000,0.000000}%
\pgfsetstrokecolor{currentstroke}%
\pgfsetdash{}{0pt}%
\pgfpathmoveto{\pgfqpoint{0.631364in}{0.466732in}}%
\pgfpathlineto{\pgfqpoint{0.709621in}{0.466732in}}%
\pgfusepath{stroke}%
\end{pgfscope}%
\begin{pgfscope}%
\pgfpathrectangle{\pgfqpoint{0.550713in}{0.127635in}}{\pgfqpoint{3.194133in}{1.496807in}}%
\pgfusepath{clip}%
\pgfsetrectcap%
\pgfsetroundjoin%
\pgfsetlinewidth{0.752812pt}%
\definecolor{currentstroke}{rgb}{0.000000,0.000000,0.000000}%
\pgfsetstrokecolor{currentstroke}%
\pgfsetdash{}{0pt}%
\pgfpathmoveto{\pgfqpoint{0.631364in}{1.147933in}}%
\pgfpathlineto{\pgfqpoint{0.709621in}{1.147933in}}%
\pgfusepath{stroke}%
\end{pgfscope}%
\begin{pgfscope}%
\pgfpathrectangle{\pgfqpoint{0.550713in}{0.127635in}}{\pgfqpoint{3.194133in}{1.496807in}}%
\pgfusepath{clip}%
\pgfsetrectcap%
\pgfsetroundjoin%
\pgfsetlinewidth{0.752812pt}%
\definecolor{currentstroke}{rgb}{0.000000,0.000000,0.000000}%
\pgfsetstrokecolor{currentstroke}%
\pgfsetdash{}{0pt}%
\pgfpathmoveto{\pgfqpoint{0.830199in}{1.281515in}}%
\pgfpathlineto{\pgfqpoint{0.830199in}{0.683118in}}%
\pgfusepath{stroke}%
\end{pgfscope}%
\begin{pgfscope}%
\pgfpathrectangle{\pgfqpoint{0.550713in}{0.127635in}}{\pgfqpoint{3.194133in}{1.496807in}}%
\pgfusepath{clip}%
\pgfsetrectcap%
\pgfsetroundjoin%
\pgfsetlinewidth{0.752812pt}%
\definecolor{currentstroke}{rgb}{0.000000,0.000000,0.000000}%
\pgfsetstrokecolor{currentstroke}%
\pgfsetdash{}{0pt}%
\pgfusepath{stroke}%
\end{pgfscope}%
\begin{pgfscope}%
\pgfpathrectangle{\pgfqpoint{0.550713in}{0.127635in}}{\pgfqpoint{3.194133in}{1.496807in}}%
\pgfusepath{clip}%
\pgfsetrectcap%
\pgfsetroundjoin%
\pgfsetlinewidth{0.752812pt}%
\definecolor{currentstroke}{rgb}{0.000000,0.000000,0.000000}%
\pgfsetstrokecolor{currentstroke}%
\pgfsetdash{}{0pt}%
\pgfpathmoveto{\pgfqpoint{0.791071in}{0.683118in}}%
\pgfpathlineto{\pgfqpoint{0.869327in}{0.683118in}}%
\pgfusepath{stroke}%
\end{pgfscope}%
\begin{pgfscope}%
\pgfpathrectangle{\pgfqpoint{0.550713in}{0.127635in}}{\pgfqpoint{3.194133in}{1.496807in}}%
\pgfusepath{clip}%
\pgfsetrectcap%
\pgfsetroundjoin%
\pgfsetlinewidth{0.752812pt}%
\definecolor{currentstroke}{rgb}{0.000000,0.000000,0.000000}%
\pgfsetstrokecolor{currentstroke}%
\pgfsetdash{}{0pt}%
\pgfusepath{stroke}%
\end{pgfscope}%
\begin{pgfscope}%
\pgfpathrectangle{\pgfqpoint{0.550713in}{0.127635in}}{\pgfqpoint{3.194133in}{1.496807in}}%
\pgfusepath{clip}%
\pgfsetrectcap%
\pgfsetroundjoin%
\pgfsetlinewidth{0.752812pt}%
\definecolor{currentstroke}{rgb}{0.000000,0.000000,0.000000}%
\pgfsetstrokecolor{currentstroke}%
\pgfsetdash{}{0pt}%
\pgfpathmoveto{\pgfqpoint{1.069759in}{0.662749in}}%
\pgfpathlineto{\pgfqpoint{1.069759in}{0.537266in}}%
\pgfusepath{stroke}%
\end{pgfscope}%
\begin{pgfscope}%
\pgfpathrectangle{\pgfqpoint{0.550713in}{0.127635in}}{\pgfqpoint{3.194133in}{1.496807in}}%
\pgfusepath{clip}%
\pgfsetrectcap%
\pgfsetroundjoin%
\pgfsetlinewidth{0.752812pt}%
\definecolor{currentstroke}{rgb}{0.000000,0.000000,0.000000}%
\pgfsetstrokecolor{currentstroke}%
\pgfsetdash{}{0pt}%
\pgfpathmoveto{\pgfqpoint{1.069759in}{1.257895in}}%
\pgfpathlineto{\pgfqpoint{1.069759in}{1.496239in}}%
\pgfusepath{stroke}%
\end{pgfscope}%
\begin{pgfscope}%
\pgfpathrectangle{\pgfqpoint{0.550713in}{0.127635in}}{\pgfqpoint{3.194133in}{1.496807in}}%
\pgfusepath{clip}%
\pgfsetrectcap%
\pgfsetroundjoin%
\pgfsetlinewidth{0.752812pt}%
\definecolor{currentstroke}{rgb}{0.000000,0.000000,0.000000}%
\pgfsetstrokecolor{currentstroke}%
\pgfsetdash{}{0pt}%
\pgfpathmoveto{\pgfqpoint{1.030631in}{0.537266in}}%
\pgfpathlineto{\pgfqpoint{1.108887in}{0.537266in}}%
\pgfusepath{stroke}%
\end{pgfscope}%
\begin{pgfscope}%
\pgfpathrectangle{\pgfqpoint{0.550713in}{0.127635in}}{\pgfqpoint{3.194133in}{1.496807in}}%
\pgfusepath{clip}%
\pgfsetrectcap%
\pgfsetroundjoin%
\pgfsetlinewidth{0.752812pt}%
\definecolor{currentstroke}{rgb}{0.000000,0.000000,0.000000}%
\pgfsetstrokecolor{currentstroke}%
\pgfsetdash{}{0pt}%
\pgfpathmoveto{\pgfqpoint{1.030631in}{1.496239in}}%
\pgfpathlineto{\pgfqpoint{1.108887in}{1.496239in}}%
\pgfusepath{stroke}%
\end{pgfscope}%
\begin{pgfscope}%
\pgfpathrectangle{\pgfqpoint{0.550713in}{0.127635in}}{\pgfqpoint{3.194133in}{1.496807in}}%
\pgfusepath{clip}%
\pgfsetrectcap%
\pgfsetroundjoin%
\pgfsetlinewidth{0.752812pt}%
\definecolor{currentstroke}{rgb}{0.000000,0.000000,0.000000}%
\pgfsetstrokecolor{currentstroke}%
\pgfsetdash{}{0pt}%
\pgfpathmoveto{\pgfqpoint{1.229466in}{1.634441in}}%
\pgfpathlineto{\pgfqpoint{1.229466in}{1.494095in}}%
\pgfusepath{stroke}%
\end{pgfscope}%
\begin{pgfscope}%
\pgfpathrectangle{\pgfqpoint{0.550713in}{0.127635in}}{\pgfqpoint{3.194133in}{1.496807in}}%
\pgfusepath{clip}%
\pgfsetrectcap%
\pgfsetroundjoin%
\pgfsetlinewidth{0.752812pt}%
\definecolor{currentstroke}{rgb}{0.000000,0.000000,0.000000}%
\pgfsetstrokecolor{currentstroke}%
\pgfsetdash{}{0pt}%
\pgfusepath{stroke}%
\end{pgfscope}%
\begin{pgfscope}%
\pgfpathrectangle{\pgfqpoint{0.550713in}{0.127635in}}{\pgfqpoint{3.194133in}{1.496807in}}%
\pgfusepath{clip}%
\pgfsetrectcap%
\pgfsetroundjoin%
\pgfsetlinewidth{0.752812pt}%
\definecolor{currentstroke}{rgb}{0.000000,0.000000,0.000000}%
\pgfsetstrokecolor{currentstroke}%
\pgfsetdash{}{0pt}%
\pgfpathmoveto{\pgfqpoint{1.190338in}{1.494095in}}%
\pgfpathlineto{\pgfqpoint{1.268594in}{1.494095in}}%
\pgfusepath{stroke}%
\end{pgfscope}%
\begin{pgfscope}%
\pgfpathrectangle{\pgfqpoint{0.550713in}{0.127635in}}{\pgfqpoint{3.194133in}{1.496807in}}%
\pgfusepath{clip}%
\pgfsetrectcap%
\pgfsetroundjoin%
\pgfsetlinewidth{0.752812pt}%
\definecolor{currentstroke}{rgb}{0.000000,0.000000,0.000000}%
\pgfsetstrokecolor{currentstroke}%
\pgfsetdash{}{0pt}%
\pgfusepath{stroke}%
\end{pgfscope}%
\begin{pgfscope}%
\pgfpathrectangle{\pgfqpoint{0.550713in}{0.127635in}}{\pgfqpoint{3.194133in}{1.496807in}}%
\pgfusepath{clip}%
\pgfsetrectcap%
\pgfsetroundjoin%
\pgfsetlinewidth{0.752812pt}%
\definecolor{currentstroke}{rgb}{0.000000,0.000000,0.000000}%
\pgfsetstrokecolor{currentstroke}%
\pgfsetdash{}{0pt}%
\pgfpathmoveto{\pgfqpoint{1.469026in}{0.640227in}}%
\pgfpathlineto{\pgfqpoint{1.469026in}{0.592523in}}%
\pgfusepath{stroke}%
\end{pgfscope}%
\begin{pgfscope}%
\pgfpathrectangle{\pgfqpoint{0.550713in}{0.127635in}}{\pgfqpoint{3.194133in}{1.496807in}}%
\pgfusepath{clip}%
\pgfsetrectcap%
\pgfsetroundjoin%
\pgfsetlinewidth{0.752812pt}%
\definecolor{currentstroke}{rgb}{0.000000,0.000000,0.000000}%
\pgfsetstrokecolor{currentstroke}%
\pgfsetdash{}{0pt}%
\pgfpathmoveto{\pgfqpoint{1.469026in}{0.845077in}}%
\pgfpathlineto{\pgfqpoint{1.469026in}{0.858745in}}%
\pgfusepath{stroke}%
\end{pgfscope}%
\begin{pgfscope}%
\pgfpathrectangle{\pgfqpoint{0.550713in}{0.127635in}}{\pgfqpoint{3.194133in}{1.496807in}}%
\pgfusepath{clip}%
\pgfsetrectcap%
\pgfsetroundjoin%
\pgfsetlinewidth{0.752812pt}%
\definecolor{currentstroke}{rgb}{0.000000,0.000000,0.000000}%
\pgfsetstrokecolor{currentstroke}%
\pgfsetdash{}{0pt}%
\pgfpathmoveto{\pgfqpoint{1.429898in}{0.592523in}}%
\pgfpathlineto{\pgfqpoint{1.508154in}{0.592523in}}%
\pgfusepath{stroke}%
\end{pgfscope}%
\begin{pgfscope}%
\pgfpathrectangle{\pgfqpoint{0.550713in}{0.127635in}}{\pgfqpoint{3.194133in}{1.496807in}}%
\pgfusepath{clip}%
\pgfsetrectcap%
\pgfsetroundjoin%
\pgfsetlinewidth{0.752812pt}%
\definecolor{currentstroke}{rgb}{0.000000,0.000000,0.000000}%
\pgfsetstrokecolor{currentstroke}%
\pgfsetdash{}{0pt}%
\pgfpathmoveto{\pgfqpoint{1.429898in}{0.858745in}}%
\pgfpathlineto{\pgfqpoint{1.508154in}{0.858745in}}%
\pgfusepath{stroke}%
\end{pgfscope}%
\begin{pgfscope}%
\pgfpathrectangle{\pgfqpoint{0.550713in}{0.127635in}}{\pgfqpoint{3.194133in}{1.496807in}}%
\pgfusepath{clip}%
\pgfsetrectcap%
\pgfsetroundjoin%
\pgfsetlinewidth{0.752812pt}%
\definecolor{currentstroke}{rgb}{0.000000,0.000000,0.000000}%
\pgfsetstrokecolor{currentstroke}%
\pgfsetdash{}{0pt}%
\pgfpathmoveto{\pgfqpoint{1.628733in}{0.810784in}}%
\pgfpathlineto{\pgfqpoint{1.628733in}{0.700735in}}%
\pgfusepath{stroke}%
\end{pgfscope}%
\begin{pgfscope}%
\pgfpathrectangle{\pgfqpoint{0.550713in}{0.127635in}}{\pgfqpoint{3.194133in}{1.496807in}}%
\pgfusepath{clip}%
\pgfsetrectcap%
\pgfsetroundjoin%
\pgfsetlinewidth{0.752812pt}%
\definecolor{currentstroke}{rgb}{0.000000,0.000000,0.000000}%
\pgfsetstrokecolor{currentstroke}%
\pgfsetdash{}{0pt}%
\pgfpathmoveto{\pgfqpoint{1.628733in}{0.963664in}}%
\pgfpathlineto{\pgfqpoint{1.628733in}{0.970558in}}%
\pgfusepath{stroke}%
\end{pgfscope}%
\begin{pgfscope}%
\pgfpathrectangle{\pgfqpoint{0.550713in}{0.127635in}}{\pgfqpoint{3.194133in}{1.496807in}}%
\pgfusepath{clip}%
\pgfsetrectcap%
\pgfsetroundjoin%
\pgfsetlinewidth{0.752812pt}%
\definecolor{currentstroke}{rgb}{0.000000,0.000000,0.000000}%
\pgfsetstrokecolor{currentstroke}%
\pgfsetdash{}{0pt}%
\pgfpathmoveto{\pgfqpoint{1.589604in}{0.700735in}}%
\pgfpathlineto{\pgfqpoint{1.667861in}{0.700735in}}%
\pgfusepath{stroke}%
\end{pgfscope}%
\begin{pgfscope}%
\pgfpathrectangle{\pgfqpoint{0.550713in}{0.127635in}}{\pgfqpoint{3.194133in}{1.496807in}}%
\pgfusepath{clip}%
\pgfsetrectcap%
\pgfsetroundjoin%
\pgfsetlinewidth{0.752812pt}%
\definecolor{currentstroke}{rgb}{0.000000,0.000000,0.000000}%
\pgfsetstrokecolor{currentstroke}%
\pgfsetdash{}{0pt}%
\pgfpathmoveto{\pgfqpoint{1.589604in}{0.970558in}}%
\pgfpathlineto{\pgfqpoint{1.667861in}{0.970558in}}%
\pgfusepath{stroke}%
\end{pgfscope}%
\begin{pgfscope}%
\pgfpathrectangle{\pgfqpoint{0.550713in}{0.127635in}}{\pgfqpoint{3.194133in}{1.496807in}}%
\pgfusepath{clip}%
\pgfsetrectcap%
\pgfsetroundjoin%
\pgfsetlinewidth{0.752812pt}%
\definecolor{currentstroke}{rgb}{0.000000,0.000000,0.000000}%
\pgfsetstrokecolor{currentstroke}%
\pgfsetdash{}{0pt}%
\pgfpathmoveto{\pgfqpoint{1.868293in}{0.860695in}}%
\pgfpathlineto{\pgfqpoint{1.868293in}{0.748393in}}%
\pgfusepath{stroke}%
\end{pgfscope}%
\begin{pgfscope}%
\pgfpathrectangle{\pgfqpoint{0.550713in}{0.127635in}}{\pgfqpoint{3.194133in}{1.496807in}}%
\pgfusepath{clip}%
\pgfsetrectcap%
\pgfsetroundjoin%
\pgfsetlinewidth{0.752812pt}%
\definecolor{currentstroke}{rgb}{0.000000,0.000000,0.000000}%
\pgfsetstrokecolor{currentstroke}%
\pgfsetdash{}{0pt}%
\pgfpathmoveto{\pgfqpoint{1.868293in}{1.074473in}}%
\pgfpathlineto{\pgfqpoint{1.868293in}{1.287619in}}%
\pgfusepath{stroke}%
\end{pgfscope}%
\begin{pgfscope}%
\pgfpathrectangle{\pgfqpoint{0.550713in}{0.127635in}}{\pgfqpoint{3.194133in}{1.496807in}}%
\pgfusepath{clip}%
\pgfsetrectcap%
\pgfsetroundjoin%
\pgfsetlinewidth{0.752812pt}%
\definecolor{currentstroke}{rgb}{0.000000,0.000000,0.000000}%
\pgfsetstrokecolor{currentstroke}%
\pgfsetdash{}{0pt}%
\pgfpathmoveto{\pgfqpoint{1.829164in}{0.748393in}}%
\pgfpathlineto{\pgfqpoint{1.907421in}{0.748393in}}%
\pgfusepath{stroke}%
\end{pgfscope}%
\begin{pgfscope}%
\pgfpathrectangle{\pgfqpoint{0.550713in}{0.127635in}}{\pgfqpoint{3.194133in}{1.496807in}}%
\pgfusepath{clip}%
\pgfsetrectcap%
\pgfsetroundjoin%
\pgfsetlinewidth{0.752812pt}%
\definecolor{currentstroke}{rgb}{0.000000,0.000000,0.000000}%
\pgfsetstrokecolor{currentstroke}%
\pgfsetdash{}{0pt}%
\pgfpathmoveto{\pgfqpoint{1.829164in}{1.287619in}}%
\pgfpathlineto{\pgfqpoint{1.907421in}{1.287619in}}%
\pgfusepath{stroke}%
\end{pgfscope}%
\begin{pgfscope}%
\pgfpathrectangle{\pgfqpoint{0.550713in}{0.127635in}}{\pgfqpoint{3.194133in}{1.496807in}}%
\pgfusepath{clip}%
\pgfsetrectcap%
\pgfsetroundjoin%
\pgfsetlinewidth{0.752812pt}%
\definecolor{currentstroke}{rgb}{0.000000,0.000000,0.000000}%
\pgfsetstrokecolor{currentstroke}%
\pgfsetdash{}{0pt}%
\pgfpathmoveto{\pgfqpoint{2.027999in}{1.008655in}}%
\pgfpathlineto{\pgfqpoint{2.027999in}{1.008655in}}%
\pgfusepath{stroke}%
\end{pgfscope}%
\begin{pgfscope}%
\pgfpathrectangle{\pgfqpoint{0.550713in}{0.127635in}}{\pgfqpoint{3.194133in}{1.496807in}}%
\pgfusepath{clip}%
\pgfsetrectcap%
\pgfsetroundjoin%
\pgfsetlinewidth{0.752812pt}%
\definecolor{currentstroke}{rgb}{0.000000,0.000000,0.000000}%
\pgfsetstrokecolor{currentstroke}%
\pgfsetdash{}{0pt}%
\pgfpathmoveto{\pgfqpoint{2.027999in}{1.082756in}}%
\pgfpathlineto{\pgfqpoint{2.027999in}{1.082756in}}%
\pgfusepath{stroke}%
\end{pgfscope}%
\begin{pgfscope}%
\pgfpathrectangle{\pgfqpoint{0.550713in}{0.127635in}}{\pgfqpoint{3.194133in}{1.496807in}}%
\pgfusepath{clip}%
\pgfsetrectcap%
\pgfsetroundjoin%
\pgfsetlinewidth{0.752812pt}%
\definecolor{currentstroke}{rgb}{0.000000,0.000000,0.000000}%
\pgfsetstrokecolor{currentstroke}%
\pgfsetdash{}{0pt}%
\pgfpathmoveto{\pgfqpoint{1.988871in}{1.008655in}}%
\pgfpathlineto{\pgfqpoint{2.067127in}{1.008655in}}%
\pgfusepath{stroke}%
\end{pgfscope}%
\begin{pgfscope}%
\pgfpathrectangle{\pgfqpoint{0.550713in}{0.127635in}}{\pgfqpoint{3.194133in}{1.496807in}}%
\pgfusepath{clip}%
\pgfsetrectcap%
\pgfsetroundjoin%
\pgfsetlinewidth{0.752812pt}%
\definecolor{currentstroke}{rgb}{0.000000,0.000000,0.000000}%
\pgfsetstrokecolor{currentstroke}%
\pgfsetdash{}{0pt}%
\pgfpathmoveto{\pgfqpoint{1.988871in}{1.082756in}}%
\pgfpathlineto{\pgfqpoint{2.067127in}{1.082756in}}%
\pgfusepath{stroke}%
\end{pgfscope}%
\begin{pgfscope}%
\pgfpathrectangle{\pgfqpoint{0.550713in}{0.127635in}}{\pgfqpoint{3.194133in}{1.496807in}}%
\pgfusepath{clip}%
\pgfsetbuttcap%
\pgfsetmiterjoin%
\definecolor{currentfill}{rgb}{0.000000,0.000000,0.000000}%
\pgfsetfillcolor{currentfill}%
\pgfsetlinewidth{1.003750pt}%
\definecolor{currentstroke}{rgb}{0.000000,0.000000,0.000000}%
\pgfsetstrokecolor{currentstroke}%
\pgfsetdash{}{0pt}%
\pgfsys@defobject{currentmarker}{\pgfqpoint{-0.011785in}{-0.019642in}}{\pgfqpoint{0.011785in}{0.019642in}}{%
\pgfpathmoveto{\pgfqpoint{-0.000000in}{-0.019642in}}%
\pgfpathlineto{\pgfqpoint{0.011785in}{0.000000in}}%
\pgfpathlineto{\pgfqpoint{0.000000in}{0.019642in}}%
\pgfpathlineto{\pgfqpoint{-0.011785in}{0.000000in}}%
\pgfpathclose%
\pgfusepath{stroke,fill}%
}%
\begin{pgfscope}%
\pgfsys@transformshift{2.027999in}{0.784691in}%
\pgfsys@useobject{currentmarker}{}%
\end{pgfscope}%
\begin{pgfscope}%
\pgfsys@transformshift{2.027999in}{1.285129in}%
\pgfsys@useobject{currentmarker}{}%
\end{pgfscope}%
\end{pgfscope}%
\begin{pgfscope}%
\pgfpathrectangle{\pgfqpoint{0.550713in}{0.127635in}}{\pgfqpoint{3.194133in}{1.496807in}}%
\pgfusepath{clip}%
\pgfsetrectcap%
\pgfsetroundjoin%
\pgfsetlinewidth{0.752812pt}%
\definecolor{currentstroke}{rgb}{0.000000,0.000000,0.000000}%
\pgfsetstrokecolor{currentstroke}%
\pgfsetdash{}{0pt}%
\pgfpathmoveto{\pgfqpoint{2.267559in}{0.556240in}}%
\pgfpathlineto{\pgfqpoint{2.267559in}{0.553415in}}%
\pgfusepath{stroke}%
\end{pgfscope}%
\begin{pgfscope}%
\pgfpathrectangle{\pgfqpoint{0.550713in}{0.127635in}}{\pgfqpoint{3.194133in}{1.496807in}}%
\pgfusepath{clip}%
\pgfsetrectcap%
\pgfsetroundjoin%
\pgfsetlinewidth{0.752812pt}%
\definecolor{currentstroke}{rgb}{0.000000,0.000000,0.000000}%
\pgfsetstrokecolor{currentstroke}%
\pgfsetdash{}{0pt}%
\pgfpathmoveto{\pgfqpoint{2.267559in}{0.733068in}}%
\pgfpathlineto{\pgfqpoint{2.267559in}{0.807693in}}%
\pgfusepath{stroke}%
\end{pgfscope}%
\begin{pgfscope}%
\pgfpathrectangle{\pgfqpoint{0.550713in}{0.127635in}}{\pgfqpoint{3.194133in}{1.496807in}}%
\pgfusepath{clip}%
\pgfsetrectcap%
\pgfsetroundjoin%
\pgfsetlinewidth{0.752812pt}%
\definecolor{currentstroke}{rgb}{0.000000,0.000000,0.000000}%
\pgfsetstrokecolor{currentstroke}%
\pgfsetdash{}{0pt}%
\pgfpathmoveto{\pgfqpoint{2.228431in}{0.553415in}}%
\pgfpathlineto{\pgfqpoint{2.306687in}{0.553415in}}%
\pgfusepath{stroke}%
\end{pgfscope}%
\begin{pgfscope}%
\pgfpathrectangle{\pgfqpoint{0.550713in}{0.127635in}}{\pgfqpoint{3.194133in}{1.496807in}}%
\pgfusepath{clip}%
\pgfsetrectcap%
\pgfsetroundjoin%
\pgfsetlinewidth{0.752812pt}%
\definecolor{currentstroke}{rgb}{0.000000,0.000000,0.000000}%
\pgfsetstrokecolor{currentstroke}%
\pgfsetdash{}{0pt}%
\pgfpathmoveto{\pgfqpoint{2.228431in}{0.807693in}}%
\pgfpathlineto{\pgfqpoint{2.306687in}{0.807693in}}%
\pgfusepath{stroke}%
\end{pgfscope}%
\begin{pgfscope}%
\pgfpathrectangle{\pgfqpoint{0.550713in}{0.127635in}}{\pgfqpoint{3.194133in}{1.496807in}}%
\pgfusepath{clip}%
\pgfsetrectcap%
\pgfsetroundjoin%
\pgfsetlinewidth{0.752812pt}%
\definecolor{currentstroke}{rgb}{0.000000,0.000000,0.000000}%
\pgfsetstrokecolor{currentstroke}%
\pgfsetdash{}{0pt}%
\pgfpathmoveto{\pgfqpoint{2.427266in}{0.677003in}}%
\pgfpathlineto{\pgfqpoint{2.427266in}{0.663795in}}%
\pgfusepath{stroke}%
\end{pgfscope}%
\begin{pgfscope}%
\pgfpathrectangle{\pgfqpoint{0.550713in}{0.127635in}}{\pgfqpoint{3.194133in}{1.496807in}}%
\pgfusepath{clip}%
\pgfsetrectcap%
\pgfsetroundjoin%
\pgfsetlinewidth{0.752812pt}%
\definecolor{currentstroke}{rgb}{0.000000,0.000000,0.000000}%
\pgfsetstrokecolor{currentstroke}%
\pgfsetdash{}{0pt}%
\pgfpathmoveto{\pgfqpoint{2.427266in}{0.919837in}}%
\pgfpathlineto{\pgfqpoint{2.427266in}{0.948977in}}%
\pgfusepath{stroke}%
\end{pgfscope}%
\begin{pgfscope}%
\pgfpathrectangle{\pgfqpoint{0.550713in}{0.127635in}}{\pgfqpoint{3.194133in}{1.496807in}}%
\pgfusepath{clip}%
\pgfsetrectcap%
\pgfsetroundjoin%
\pgfsetlinewidth{0.752812pt}%
\definecolor{currentstroke}{rgb}{0.000000,0.000000,0.000000}%
\pgfsetstrokecolor{currentstroke}%
\pgfsetdash{}{0pt}%
\pgfpathmoveto{\pgfqpoint{2.388138in}{0.663795in}}%
\pgfpathlineto{\pgfqpoint{2.466394in}{0.663795in}}%
\pgfusepath{stroke}%
\end{pgfscope}%
\begin{pgfscope}%
\pgfpathrectangle{\pgfqpoint{0.550713in}{0.127635in}}{\pgfqpoint{3.194133in}{1.496807in}}%
\pgfusepath{clip}%
\pgfsetrectcap%
\pgfsetroundjoin%
\pgfsetlinewidth{0.752812pt}%
\definecolor{currentstroke}{rgb}{0.000000,0.000000,0.000000}%
\pgfsetstrokecolor{currentstroke}%
\pgfsetdash{}{0pt}%
\pgfpathmoveto{\pgfqpoint{2.388138in}{0.948977in}}%
\pgfpathlineto{\pgfqpoint{2.466394in}{0.948977in}}%
\pgfusepath{stroke}%
\end{pgfscope}%
\begin{pgfscope}%
\pgfpathrectangle{\pgfqpoint{0.550713in}{0.127635in}}{\pgfqpoint{3.194133in}{1.496807in}}%
\pgfusepath{clip}%
\pgfsetrectcap%
\pgfsetroundjoin%
\pgfsetlinewidth{0.752812pt}%
\definecolor{currentstroke}{rgb}{0.000000,0.000000,0.000000}%
\pgfsetstrokecolor{currentstroke}%
\pgfsetdash{}{0pt}%
\pgfpathmoveto{\pgfqpoint{2.666826in}{0.648291in}}%
\pgfpathlineto{\pgfqpoint{2.666826in}{0.641968in}}%
\pgfusepath{stroke}%
\end{pgfscope}%
\begin{pgfscope}%
\pgfpathrectangle{\pgfqpoint{0.550713in}{0.127635in}}{\pgfqpoint{3.194133in}{1.496807in}}%
\pgfusepath{clip}%
\pgfsetrectcap%
\pgfsetroundjoin%
\pgfsetlinewidth{0.752812pt}%
\definecolor{currentstroke}{rgb}{0.000000,0.000000,0.000000}%
\pgfsetstrokecolor{currentstroke}%
\pgfsetdash{}{0pt}%
\pgfpathmoveto{\pgfqpoint{2.666826in}{0.820171in}}%
\pgfpathlineto{\pgfqpoint{2.666826in}{0.876371in}}%
\pgfusepath{stroke}%
\end{pgfscope}%
\begin{pgfscope}%
\pgfpathrectangle{\pgfqpoint{0.550713in}{0.127635in}}{\pgfqpoint{3.194133in}{1.496807in}}%
\pgfusepath{clip}%
\pgfsetrectcap%
\pgfsetroundjoin%
\pgfsetlinewidth{0.752812pt}%
\definecolor{currentstroke}{rgb}{0.000000,0.000000,0.000000}%
\pgfsetstrokecolor{currentstroke}%
\pgfsetdash{}{0pt}%
\pgfpathmoveto{\pgfqpoint{2.627698in}{0.641968in}}%
\pgfpathlineto{\pgfqpoint{2.705954in}{0.641968in}}%
\pgfusepath{stroke}%
\end{pgfscope}%
\begin{pgfscope}%
\pgfpathrectangle{\pgfqpoint{0.550713in}{0.127635in}}{\pgfqpoint{3.194133in}{1.496807in}}%
\pgfusepath{clip}%
\pgfsetrectcap%
\pgfsetroundjoin%
\pgfsetlinewidth{0.752812pt}%
\definecolor{currentstroke}{rgb}{0.000000,0.000000,0.000000}%
\pgfsetstrokecolor{currentstroke}%
\pgfsetdash{}{0pt}%
\pgfpathmoveto{\pgfqpoint{2.627698in}{0.876371in}}%
\pgfpathlineto{\pgfqpoint{2.705954in}{0.876371in}}%
\pgfusepath{stroke}%
\end{pgfscope}%
\begin{pgfscope}%
\pgfpathrectangle{\pgfqpoint{0.550713in}{0.127635in}}{\pgfqpoint{3.194133in}{1.496807in}}%
\pgfusepath{clip}%
\pgfsetrectcap%
\pgfsetroundjoin%
\pgfsetlinewidth{0.752812pt}%
\definecolor{currentstroke}{rgb}{0.000000,0.000000,0.000000}%
\pgfsetstrokecolor{currentstroke}%
\pgfsetdash{}{0pt}%
\pgfpathmoveto{\pgfqpoint{2.826532in}{0.900490in}}%
\pgfpathlineto{\pgfqpoint{2.826532in}{0.794782in}}%
\pgfusepath{stroke}%
\end{pgfscope}%
\begin{pgfscope}%
\pgfpathrectangle{\pgfqpoint{0.550713in}{0.127635in}}{\pgfqpoint{3.194133in}{1.496807in}}%
\pgfusepath{clip}%
\pgfsetrectcap%
\pgfsetroundjoin%
\pgfsetlinewidth{0.752812pt}%
\definecolor{currentstroke}{rgb}{0.000000,0.000000,0.000000}%
\pgfsetstrokecolor{currentstroke}%
\pgfsetdash{}{0pt}%
\pgfpathmoveto{\pgfqpoint{2.826532in}{0.996464in}}%
\pgfpathlineto{\pgfqpoint{2.826532in}{1.032968in}}%
\pgfusepath{stroke}%
\end{pgfscope}%
\begin{pgfscope}%
\pgfpathrectangle{\pgfqpoint{0.550713in}{0.127635in}}{\pgfqpoint{3.194133in}{1.496807in}}%
\pgfusepath{clip}%
\pgfsetrectcap%
\pgfsetroundjoin%
\pgfsetlinewidth{0.752812pt}%
\definecolor{currentstroke}{rgb}{0.000000,0.000000,0.000000}%
\pgfsetstrokecolor{currentstroke}%
\pgfsetdash{}{0pt}%
\pgfpathmoveto{\pgfqpoint{2.787404in}{0.794782in}}%
\pgfpathlineto{\pgfqpoint{2.865661in}{0.794782in}}%
\pgfusepath{stroke}%
\end{pgfscope}%
\begin{pgfscope}%
\pgfpathrectangle{\pgfqpoint{0.550713in}{0.127635in}}{\pgfqpoint{3.194133in}{1.496807in}}%
\pgfusepath{clip}%
\pgfsetrectcap%
\pgfsetroundjoin%
\pgfsetlinewidth{0.752812pt}%
\definecolor{currentstroke}{rgb}{0.000000,0.000000,0.000000}%
\pgfsetstrokecolor{currentstroke}%
\pgfsetdash{}{0pt}%
\pgfpathmoveto{\pgfqpoint{2.787404in}{1.032968in}}%
\pgfpathlineto{\pgfqpoint{2.865661in}{1.032968in}}%
\pgfusepath{stroke}%
\end{pgfscope}%
\begin{pgfscope}%
\pgfpathrectangle{\pgfqpoint{0.550713in}{0.127635in}}{\pgfqpoint{3.194133in}{1.496807in}}%
\pgfusepath{clip}%
\pgfsetrectcap%
\pgfsetroundjoin%
\pgfsetlinewidth{0.752812pt}%
\definecolor{currentstroke}{rgb}{0.000000,0.000000,0.000000}%
\pgfsetstrokecolor{currentstroke}%
\pgfsetdash{}{0pt}%
\pgfpathmoveto{\pgfqpoint{3.066092in}{0.567466in}}%
\pgfpathlineto{\pgfqpoint{3.066092in}{0.543145in}}%
\pgfusepath{stroke}%
\end{pgfscope}%
\begin{pgfscope}%
\pgfpathrectangle{\pgfqpoint{0.550713in}{0.127635in}}{\pgfqpoint{3.194133in}{1.496807in}}%
\pgfusepath{clip}%
\pgfsetrectcap%
\pgfsetroundjoin%
\pgfsetlinewidth{0.752812pt}%
\definecolor{currentstroke}{rgb}{0.000000,0.000000,0.000000}%
\pgfsetstrokecolor{currentstroke}%
\pgfsetdash{}{0pt}%
\pgfpathmoveto{\pgfqpoint{3.066092in}{0.668492in}}%
\pgfpathlineto{\pgfqpoint{3.066092in}{0.675225in}}%
\pgfusepath{stroke}%
\end{pgfscope}%
\begin{pgfscope}%
\pgfpathrectangle{\pgfqpoint{0.550713in}{0.127635in}}{\pgfqpoint{3.194133in}{1.496807in}}%
\pgfusepath{clip}%
\pgfsetrectcap%
\pgfsetroundjoin%
\pgfsetlinewidth{0.752812pt}%
\definecolor{currentstroke}{rgb}{0.000000,0.000000,0.000000}%
\pgfsetstrokecolor{currentstroke}%
\pgfsetdash{}{0pt}%
\pgfpathmoveto{\pgfqpoint{3.026964in}{0.543145in}}%
\pgfpathlineto{\pgfqpoint{3.105221in}{0.543145in}}%
\pgfusepath{stroke}%
\end{pgfscope}%
\begin{pgfscope}%
\pgfpathrectangle{\pgfqpoint{0.550713in}{0.127635in}}{\pgfqpoint{3.194133in}{1.496807in}}%
\pgfusepath{clip}%
\pgfsetrectcap%
\pgfsetroundjoin%
\pgfsetlinewidth{0.752812pt}%
\definecolor{currentstroke}{rgb}{0.000000,0.000000,0.000000}%
\pgfsetstrokecolor{currentstroke}%
\pgfsetdash{}{0pt}%
\pgfpathmoveto{\pgfqpoint{3.026964in}{0.675225in}}%
\pgfpathlineto{\pgfqpoint{3.105221in}{0.675225in}}%
\pgfusepath{stroke}%
\end{pgfscope}%
\begin{pgfscope}%
\pgfpathrectangle{\pgfqpoint{0.550713in}{0.127635in}}{\pgfqpoint{3.194133in}{1.496807in}}%
\pgfusepath{clip}%
\pgfsetrectcap%
\pgfsetroundjoin%
\pgfsetlinewidth{0.752812pt}%
\definecolor{currentstroke}{rgb}{0.000000,0.000000,0.000000}%
\pgfsetstrokecolor{currentstroke}%
\pgfsetdash{}{0pt}%
\pgfpathmoveto{\pgfqpoint{3.225799in}{0.580119in}}%
\pgfpathlineto{\pgfqpoint{3.225799in}{0.529276in}}%
\pgfusepath{stroke}%
\end{pgfscope}%
\begin{pgfscope}%
\pgfpathrectangle{\pgfqpoint{0.550713in}{0.127635in}}{\pgfqpoint{3.194133in}{1.496807in}}%
\pgfusepath{clip}%
\pgfsetrectcap%
\pgfsetroundjoin%
\pgfsetlinewidth{0.752812pt}%
\definecolor{currentstroke}{rgb}{0.000000,0.000000,0.000000}%
\pgfsetstrokecolor{currentstroke}%
\pgfsetdash{}{0pt}%
\pgfpathmoveto{\pgfqpoint{3.225799in}{0.745682in}}%
\pgfpathlineto{\pgfqpoint{3.225799in}{0.746796in}}%
\pgfusepath{stroke}%
\end{pgfscope}%
\begin{pgfscope}%
\pgfpathrectangle{\pgfqpoint{0.550713in}{0.127635in}}{\pgfqpoint{3.194133in}{1.496807in}}%
\pgfusepath{clip}%
\pgfsetrectcap%
\pgfsetroundjoin%
\pgfsetlinewidth{0.752812pt}%
\definecolor{currentstroke}{rgb}{0.000000,0.000000,0.000000}%
\pgfsetstrokecolor{currentstroke}%
\pgfsetdash{}{0pt}%
\pgfpathmoveto{\pgfqpoint{3.186671in}{0.529276in}}%
\pgfpathlineto{\pgfqpoint{3.264927in}{0.529276in}}%
\pgfusepath{stroke}%
\end{pgfscope}%
\begin{pgfscope}%
\pgfpathrectangle{\pgfqpoint{0.550713in}{0.127635in}}{\pgfqpoint{3.194133in}{1.496807in}}%
\pgfusepath{clip}%
\pgfsetrectcap%
\pgfsetroundjoin%
\pgfsetlinewidth{0.752812pt}%
\definecolor{currentstroke}{rgb}{0.000000,0.000000,0.000000}%
\pgfsetstrokecolor{currentstroke}%
\pgfsetdash{}{0pt}%
\pgfpathmoveto{\pgfqpoint{3.186671in}{0.746796in}}%
\pgfpathlineto{\pgfqpoint{3.264927in}{0.746796in}}%
\pgfusepath{stroke}%
\end{pgfscope}%
\begin{pgfscope}%
\pgfpathrectangle{\pgfqpoint{0.550713in}{0.127635in}}{\pgfqpoint{3.194133in}{1.496807in}}%
\pgfusepath{clip}%
\pgfsetrectcap%
\pgfsetroundjoin%
\pgfsetlinewidth{0.752812pt}%
\definecolor{currentstroke}{rgb}{0.000000,0.000000,0.000000}%
\pgfsetstrokecolor{currentstroke}%
\pgfsetdash{}{0pt}%
\pgfpathmoveto{\pgfqpoint{3.465359in}{0.717168in}}%
\pgfpathlineto{\pgfqpoint{3.465359in}{0.591147in}}%
\pgfusepath{stroke}%
\end{pgfscope}%
\begin{pgfscope}%
\pgfpathrectangle{\pgfqpoint{0.550713in}{0.127635in}}{\pgfqpoint{3.194133in}{1.496807in}}%
\pgfusepath{clip}%
\pgfsetrectcap%
\pgfsetroundjoin%
\pgfsetlinewidth{0.752812pt}%
\definecolor{currentstroke}{rgb}{0.000000,0.000000,0.000000}%
\pgfsetstrokecolor{currentstroke}%
\pgfsetdash{}{0pt}%
\pgfpathmoveto{\pgfqpoint{3.465359in}{0.811149in}}%
\pgfpathlineto{\pgfqpoint{3.465359in}{0.838055in}}%
\pgfusepath{stroke}%
\end{pgfscope}%
\begin{pgfscope}%
\pgfpathrectangle{\pgfqpoint{0.550713in}{0.127635in}}{\pgfqpoint{3.194133in}{1.496807in}}%
\pgfusepath{clip}%
\pgfsetrectcap%
\pgfsetroundjoin%
\pgfsetlinewidth{0.752812pt}%
\definecolor{currentstroke}{rgb}{0.000000,0.000000,0.000000}%
\pgfsetstrokecolor{currentstroke}%
\pgfsetdash{}{0pt}%
\pgfpathmoveto{\pgfqpoint{3.426231in}{0.591147in}}%
\pgfpathlineto{\pgfqpoint{3.504487in}{0.591147in}}%
\pgfusepath{stroke}%
\end{pgfscope}%
\begin{pgfscope}%
\pgfpathrectangle{\pgfqpoint{0.550713in}{0.127635in}}{\pgfqpoint{3.194133in}{1.496807in}}%
\pgfusepath{clip}%
\pgfsetrectcap%
\pgfsetroundjoin%
\pgfsetlinewidth{0.752812pt}%
\definecolor{currentstroke}{rgb}{0.000000,0.000000,0.000000}%
\pgfsetstrokecolor{currentstroke}%
\pgfsetdash{}{0pt}%
\pgfpathmoveto{\pgfqpoint{3.426231in}{0.838055in}}%
\pgfpathlineto{\pgfqpoint{3.504487in}{0.838055in}}%
\pgfusepath{stroke}%
\end{pgfscope}%
\begin{pgfscope}%
\pgfpathrectangle{\pgfqpoint{0.550713in}{0.127635in}}{\pgfqpoint{3.194133in}{1.496807in}}%
\pgfusepath{clip}%
\pgfsetrectcap%
\pgfsetroundjoin%
\pgfsetlinewidth{0.752812pt}%
\definecolor{currentstroke}{rgb}{0.000000,0.000000,0.000000}%
\pgfsetstrokecolor{currentstroke}%
\pgfsetdash{}{0pt}%
\pgfpathmoveto{\pgfqpoint{3.625066in}{0.646870in}}%
\pgfpathlineto{\pgfqpoint{3.625066in}{0.638256in}}%
\pgfusepath{stroke}%
\end{pgfscope}%
\begin{pgfscope}%
\pgfpathrectangle{\pgfqpoint{0.550713in}{0.127635in}}{\pgfqpoint{3.194133in}{1.496807in}}%
\pgfusepath{clip}%
\pgfsetrectcap%
\pgfsetroundjoin%
\pgfsetlinewidth{0.752812pt}%
\definecolor{currentstroke}{rgb}{0.000000,0.000000,0.000000}%
\pgfsetstrokecolor{currentstroke}%
\pgfsetdash{}{0pt}%
\pgfpathmoveto{\pgfqpoint{3.625066in}{0.754496in}}%
\pgfpathlineto{\pgfqpoint{3.625066in}{0.855182in}}%
\pgfusepath{stroke}%
\end{pgfscope}%
\begin{pgfscope}%
\pgfpathrectangle{\pgfqpoint{0.550713in}{0.127635in}}{\pgfqpoint{3.194133in}{1.496807in}}%
\pgfusepath{clip}%
\pgfsetrectcap%
\pgfsetroundjoin%
\pgfsetlinewidth{0.752812pt}%
\definecolor{currentstroke}{rgb}{0.000000,0.000000,0.000000}%
\pgfsetstrokecolor{currentstroke}%
\pgfsetdash{}{0pt}%
\pgfpathmoveto{\pgfqpoint{3.585938in}{0.638256in}}%
\pgfpathlineto{\pgfqpoint{3.664194in}{0.638256in}}%
\pgfusepath{stroke}%
\end{pgfscope}%
\begin{pgfscope}%
\pgfpathrectangle{\pgfqpoint{0.550713in}{0.127635in}}{\pgfqpoint{3.194133in}{1.496807in}}%
\pgfusepath{clip}%
\pgfsetrectcap%
\pgfsetroundjoin%
\pgfsetlinewidth{0.752812pt}%
\definecolor{currentstroke}{rgb}{0.000000,0.000000,0.000000}%
\pgfsetstrokecolor{currentstroke}%
\pgfsetdash{}{0pt}%
\pgfpathmoveto{\pgfqpoint{3.585938in}{0.855182in}}%
\pgfpathlineto{\pgfqpoint{3.664194in}{0.855182in}}%
\pgfusepath{stroke}%
\end{pgfscope}%
\begin{pgfscope}%
\pgfsetbuttcap%
\pgfsetroundjoin%
\definecolor{currentfill}{rgb}{0.000000,0.000000,0.000000}%
\pgfsetfillcolor{currentfill}%
\pgfsetlinewidth{0.752812pt}%
\definecolor{currentstroke}{rgb}{0.000000,0.000000,0.000000}%
\pgfsetstrokecolor{currentstroke}%
\pgfsetdash{}{0pt}%
\pgfsys@defobject{currentmarker}{\pgfqpoint{-0.055556in}{-0.027778in}}{\pgfqpoint{0.055556in}{0.027778in}}{%
\pgfpathmoveto{\pgfqpoint{-0.055556in}{-0.027778in}}%
\pgfpathlineto{\pgfqpoint{0.055556in}{0.027778in}}%
\pgfusepath{stroke,fill}%
}%
\begin{pgfscope}%
\pgfsys@transformshift{0.550713in}{1.624441in}%
\pgfsys@useobject{currentmarker}{}%
\end{pgfscope}%
\begin{pgfscope}%
\pgfsys@transformshift{3.744846in}{1.624441in}%
\pgfsys@useobject{currentmarker}{}%
\end{pgfscope}%
\end{pgfscope}%
\begin{pgfscope}%
\pgfpathrectangle{\pgfqpoint{0.550713in}{0.127635in}}{\pgfqpoint{3.194133in}{1.496807in}}%
\pgfusepath{clip}%
\pgfsetrectcap%
\pgfsetroundjoin%
\pgfsetlinewidth{0.752812pt}%
\definecolor{currentstroke}{rgb}{0.000000,0.000000,0.000000}%
\pgfsetstrokecolor{currentstroke}%
\pgfsetdash{}{0pt}%
\pgfpathmoveto{\pgfqpoint{0.592236in}{0.685110in}}%
\pgfpathlineto{\pgfqpoint{0.748749in}{0.685110in}}%
\pgfusepath{stroke}%
\end{pgfscope}%
\begin{pgfscope}%
\pgfpathrectangle{\pgfqpoint{0.550713in}{0.127635in}}{\pgfqpoint{3.194133in}{1.496807in}}%
\pgfusepath{clip}%
\pgfsetbuttcap%
\pgfsetroundjoin%
\definecolor{currentfill}{rgb}{1.000000,1.000000,1.000000}%
\pgfsetfillcolor{currentfill}%
\pgfsetlinewidth{1.003750pt}%
\definecolor{currentstroke}{rgb}{0.000000,0.000000,0.000000}%
\pgfsetstrokecolor{currentstroke}%
\pgfsetdash{}{0pt}%
\pgfsys@defobject{currentmarker}{\pgfqpoint{-0.027778in}{-0.027778in}}{\pgfqpoint{0.027778in}{0.027778in}}{%
\pgfpathmoveto{\pgfqpoint{0.000000in}{-0.027778in}}%
\pgfpathcurveto{\pgfqpoint{0.007367in}{-0.027778in}}{\pgfqpoint{0.014433in}{-0.024851in}}{\pgfqpoint{0.019642in}{-0.019642in}}%
\pgfpathcurveto{\pgfqpoint{0.024851in}{-0.014433in}}{\pgfqpoint{0.027778in}{-0.007367in}}{\pgfqpoint{0.027778in}{0.000000in}}%
\pgfpathcurveto{\pgfqpoint{0.027778in}{0.007367in}}{\pgfqpoint{0.024851in}{0.014433in}}{\pgfqpoint{0.019642in}{0.019642in}}%
\pgfpathcurveto{\pgfqpoint{0.014433in}{0.024851in}}{\pgfqpoint{0.007367in}{0.027778in}}{\pgfqpoint{0.000000in}{0.027778in}}%
\pgfpathcurveto{\pgfqpoint{-0.007367in}{0.027778in}}{\pgfqpoint{-0.014433in}{0.024851in}}{\pgfqpoint{-0.019642in}{0.019642in}}%
\pgfpathcurveto{\pgfqpoint{-0.024851in}{0.014433in}}{\pgfqpoint{-0.027778in}{0.007367in}}{\pgfqpoint{-0.027778in}{0.000000in}}%
\pgfpathcurveto{\pgfqpoint{-0.027778in}{-0.007367in}}{\pgfqpoint{-0.024851in}{-0.014433in}}{\pgfqpoint{-0.019642in}{-0.019642in}}%
\pgfpathcurveto{\pgfqpoint{-0.014433in}{-0.024851in}}{\pgfqpoint{-0.007367in}{-0.027778in}}{\pgfqpoint{0.000000in}{-0.027778in}}%
\pgfpathclose%
\pgfusepath{stroke,fill}%
}%
\begin{pgfscope}%
\pgfsys@transformshift{0.670493in}{0.758793in}%
\pgfsys@useobject{currentmarker}{}%
\end{pgfscope}%
\end{pgfscope}%
\begin{pgfscope}%
\pgfpathrectangle{\pgfqpoint{0.550713in}{0.127635in}}{\pgfqpoint{3.194133in}{1.496807in}}%
\pgfusepath{clip}%
\pgfsetrectcap%
\pgfsetroundjoin%
\pgfsetlinewidth{0.752812pt}%
\definecolor{currentstroke}{rgb}{0.000000,0.000000,0.000000}%
\pgfsetstrokecolor{currentstroke}%
\pgfsetdash{}{0pt}%
\pgfusepath{stroke}%
\end{pgfscope}%
\begin{pgfscope}%
\pgfpathrectangle{\pgfqpoint{0.550713in}{0.127635in}}{\pgfqpoint{3.194133in}{1.496807in}}%
\pgfusepath{clip}%
\pgfsetbuttcap%
\pgfsetroundjoin%
\definecolor{currentfill}{rgb}{1.000000,1.000000,1.000000}%
\pgfsetfillcolor{currentfill}%
\pgfsetlinewidth{1.003750pt}%
\definecolor{currentstroke}{rgb}{0.000000,0.000000,0.000000}%
\pgfsetstrokecolor{currentstroke}%
\pgfsetdash{}{0pt}%
\pgfsys@defobject{currentmarker}{\pgfqpoint{-0.027778in}{-0.027778in}}{\pgfqpoint{0.027778in}{0.027778in}}{%
\pgfpathmoveto{\pgfqpoint{0.000000in}{-0.027778in}}%
\pgfpathcurveto{\pgfqpoint{0.007367in}{-0.027778in}}{\pgfqpoint{0.014433in}{-0.024851in}}{\pgfqpoint{0.019642in}{-0.019642in}}%
\pgfpathcurveto{\pgfqpoint{0.024851in}{-0.014433in}}{\pgfqpoint{0.027778in}{-0.007367in}}{\pgfqpoint{0.027778in}{0.000000in}}%
\pgfpathcurveto{\pgfqpoint{0.027778in}{0.007367in}}{\pgfqpoint{0.024851in}{0.014433in}}{\pgfqpoint{0.019642in}{0.019642in}}%
\pgfpathcurveto{\pgfqpoint{0.014433in}{0.024851in}}{\pgfqpoint{0.007367in}{0.027778in}}{\pgfqpoint{0.000000in}{0.027778in}}%
\pgfpathcurveto{\pgfqpoint{-0.007367in}{0.027778in}}{\pgfqpoint{-0.014433in}{0.024851in}}{\pgfqpoint{-0.019642in}{0.019642in}}%
\pgfpathcurveto{\pgfqpoint{-0.024851in}{0.014433in}}{\pgfqpoint{-0.027778in}{0.007367in}}{\pgfqpoint{-0.027778in}{0.000000in}}%
\pgfpathcurveto{\pgfqpoint{-0.027778in}{-0.007367in}}{\pgfqpoint{-0.024851in}{-0.014433in}}{\pgfqpoint{-0.019642in}{-0.019642in}}%
\pgfpathcurveto{\pgfqpoint{-0.014433in}{-0.024851in}}{\pgfqpoint{-0.007367in}{-0.027778in}}{\pgfqpoint{0.000000in}{-0.027778in}}%
\pgfpathclose%
\pgfusepath{stroke,fill}%
}%
\begin{pgfscope}%
\pgfsys@transformshift{0.830199in}{1.523625in}%
\pgfsys@useobject{currentmarker}{}%
\end{pgfscope}%
\end{pgfscope}%
\begin{pgfscope}%
\pgfpathrectangle{\pgfqpoint{0.550713in}{0.127635in}}{\pgfqpoint{3.194133in}{1.496807in}}%
\pgfusepath{clip}%
\pgfsetrectcap%
\pgfsetroundjoin%
\pgfsetlinewidth{0.752812pt}%
\definecolor{currentstroke}{rgb}{0.000000,0.000000,0.000000}%
\pgfsetstrokecolor{currentstroke}%
\pgfsetdash{}{0pt}%
\pgfpathmoveto{\pgfqpoint{0.991503in}{0.707058in}}%
\pgfpathlineto{\pgfqpoint{1.148015in}{0.707058in}}%
\pgfusepath{stroke}%
\end{pgfscope}%
\begin{pgfscope}%
\pgfpathrectangle{\pgfqpoint{0.550713in}{0.127635in}}{\pgfqpoint{3.194133in}{1.496807in}}%
\pgfusepath{clip}%
\pgfsetbuttcap%
\pgfsetroundjoin%
\definecolor{currentfill}{rgb}{1.000000,1.000000,1.000000}%
\pgfsetfillcolor{currentfill}%
\pgfsetlinewidth{1.003750pt}%
\definecolor{currentstroke}{rgb}{0.000000,0.000000,0.000000}%
\pgfsetstrokecolor{currentstroke}%
\pgfsetdash{}{0pt}%
\pgfsys@defobject{currentmarker}{\pgfqpoint{-0.027778in}{-0.027778in}}{\pgfqpoint{0.027778in}{0.027778in}}{%
\pgfpathmoveto{\pgfqpoint{0.000000in}{-0.027778in}}%
\pgfpathcurveto{\pgfqpoint{0.007367in}{-0.027778in}}{\pgfqpoint{0.014433in}{-0.024851in}}{\pgfqpoint{0.019642in}{-0.019642in}}%
\pgfpathcurveto{\pgfqpoint{0.024851in}{-0.014433in}}{\pgfqpoint{0.027778in}{-0.007367in}}{\pgfqpoint{0.027778in}{0.000000in}}%
\pgfpathcurveto{\pgfqpoint{0.027778in}{0.007367in}}{\pgfqpoint{0.024851in}{0.014433in}}{\pgfqpoint{0.019642in}{0.019642in}}%
\pgfpathcurveto{\pgfqpoint{0.014433in}{0.024851in}}{\pgfqpoint{0.007367in}{0.027778in}}{\pgfqpoint{0.000000in}{0.027778in}}%
\pgfpathcurveto{\pgfqpoint{-0.007367in}{0.027778in}}{\pgfqpoint{-0.014433in}{0.024851in}}{\pgfqpoint{-0.019642in}{0.019642in}}%
\pgfpathcurveto{\pgfqpoint{-0.024851in}{0.014433in}}{\pgfqpoint{-0.027778in}{0.007367in}}{\pgfqpoint{-0.027778in}{0.000000in}}%
\pgfpathcurveto{\pgfqpoint{-0.027778in}{-0.007367in}}{\pgfqpoint{-0.024851in}{-0.014433in}}{\pgfqpoint{-0.019642in}{-0.019642in}}%
\pgfpathcurveto{\pgfqpoint{-0.014433in}{-0.024851in}}{\pgfqpoint{-0.007367in}{-0.027778in}}{\pgfqpoint{0.000000in}{-0.027778in}}%
\pgfpathclose%
\pgfusepath{stroke,fill}%
}%
\begin{pgfscope}%
\pgfsys@transformshift{1.069759in}{0.932241in}%
\pgfsys@useobject{currentmarker}{}%
\end{pgfscope}%
\end{pgfscope}%
\begin{pgfscope}%
\pgfpathrectangle{\pgfqpoint{0.550713in}{0.127635in}}{\pgfqpoint{3.194133in}{1.496807in}}%
\pgfusepath{clip}%
\pgfsetrectcap%
\pgfsetroundjoin%
\pgfsetlinewidth{0.752812pt}%
\definecolor{currentstroke}{rgb}{0.000000,0.000000,0.000000}%
\pgfsetstrokecolor{currentstroke}%
\pgfsetdash{}{0pt}%
\pgfusepath{stroke}%
\end{pgfscope}%
\begin{pgfscope}%
\pgfpathrectangle{\pgfqpoint{0.550713in}{0.127635in}}{\pgfqpoint{3.194133in}{1.496807in}}%
\pgfusepath{clip}%
\pgfsetbuttcap%
\pgfsetroundjoin%
\definecolor{currentfill}{rgb}{1.000000,1.000000,1.000000}%
\pgfsetfillcolor{currentfill}%
\pgfsetlinewidth{1.003750pt}%
\definecolor{currentstroke}{rgb}{0.000000,0.000000,0.000000}%
\pgfsetstrokecolor{currentstroke}%
\pgfsetdash{}{0pt}%
\pgfsys@defobject{currentmarker}{\pgfqpoint{-0.027778in}{-0.027778in}}{\pgfqpoint{0.027778in}{0.027778in}}{%
\pgfpathmoveto{\pgfqpoint{0.000000in}{-0.027778in}}%
\pgfpathcurveto{\pgfqpoint{0.007367in}{-0.027778in}}{\pgfqpoint{0.014433in}{-0.024851in}}{\pgfqpoint{0.019642in}{-0.019642in}}%
\pgfpathcurveto{\pgfqpoint{0.024851in}{-0.014433in}}{\pgfqpoint{0.027778in}{-0.007367in}}{\pgfqpoint{0.027778in}{0.000000in}}%
\pgfpathcurveto{\pgfqpoint{0.027778in}{0.007367in}}{\pgfqpoint{0.024851in}{0.014433in}}{\pgfqpoint{0.019642in}{0.019642in}}%
\pgfpathcurveto{\pgfqpoint{0.014433in}{0.024851in}}{\pgfqpoint{0.007367in}{0.027778in}}{\pgfqpoint{0.000000in}{0.027778in}}%
\pgfpathcurveto{\pgfqpoint{-0.007367in}{0.027778in}}{\pgfqpoint{-0.014433in}{0.024851in}}{\pgfqpoint{-0.019642in}{0.019642in}}%
\pgfpathcurveto{\pgfqpoint{-0.024851in}{0.014433in}}{\pgfqpoint{-0.027778in}{0.007367in}}{\pgfqpoint{-0.027778in}{0.000000in}}%
\pgfpathcurveto{\pgfqpoint{-0.027778in}{-0.007367in}}{\pgfqpoint{-0.024851in}{-0.014433in}}{\pgfqpoint{-0.019642in}{-0.019642in}}%
\pgfpathcurveto{\pgfqpoint{-0.014433in}{-0.024851in}}{\pgfqpoint{-0.007367in}{-0.027778in}}{\pgfqpoint{0.000000in}{-0.027778in}}%
\pgfpathclose%
\pgfusepath{stroke,fill}%
}%
\begin{pgfscope}%
\pgfsys@transformshift{1.229466in}{2.282041in}%
\pgfsys@useobject{currentmarker}{}%
\end{pgfscope}%
\end{pgfscope}%
\begin{pgfscope}%
\pgfpathrectangle{\pgfqpoint{0.550713in}{0.127635in}}{\pgfqpoint{3.194133in}{1.496807in}}%
\pgfusepath{clip}%
\pgfsetrectcap%
\pgfsetroundjoin%
\pgfsetlinewidth{0.752812pt}%
\definecolor{currentstroke}{rgb}{0.000000,0.000000,0.000000}%
\pgfsetstrokecolor{currentstroke}%
\pgfsetdash{}{0pt}%
\pgfpathmoveto{\pgfqpoint{1.390770in}{0.810800in}}%
\pgfpathlineto{\pgfqpoint{1.547282in}{0.810800in}}%
\pgfusepath{stroke}%
\end{pgfscope}%
\begin{pgfscope}%
\pgfpathrectangle{\pgfqpoint{0.550713in}{0.127635in}}{\pgfqpoint{3.194133in}{1.496807in}}%
\pgfusepath{clip}%
\pgfsetbuttcap%
\pgfsetroundjoin%
\definecolor{currentfill}{rgb}{1.000000,1.000000,1.000000}%
\pgfsetfillcolor{currentfill}%
\pgfsetlinewidth{1.003750pt}%
\definecolor{currentstroke}{rgb}{0.000000,0.000000,0.000000}%
\pgfsetstrokecolor{currentstroke}%
\pgfsetdash{}{0pt}%
\pgfsys@defobject{currentmarker}{\pgfqpoint{-0.027778in}{-0.027778in}}{\pgfqpoint{0.027778in}{0.027778in}}{%
\pgfpathmoveto{\pgfqpoint{0.000000in}{-0.027778in}}%
\pgfpathcurveto{\pgfqpoint{0.007367in}{-0.027778in}}{\pgfqpoint{0.014433in}{-0.024851in}}{\pgfqpoint{0.019642in}{-0.019642in}}%
\pgfpathcurveto{\pgfqpoint{0.024851in}{-0.014433in}}{\pgfqpoint{0.027778in}{-0.007367in}}{\pgfqpoint{0.027778in}{0.000000in}}%
\pgfpathcurveto{\pgfqpoint{0.027778in}{0.007367in}}{\pgfqpoint{0.024851in}{0.014433in}}{\pgfqpoint{0.019642in}{0.019642in}}%
\pgfpathcurveto{\pgfqpoint{0.014433in}{0.024851in}}{\pgfqpoint{0.007367in}{0.027778in}}{\pgfqpoint{0.000000in}{0.027778in}}%
\pgfpathcurveto{\pgfqpoint{-0.007367in}{0.027778in}}{\pgfqpoint{-0.014433in}{0.024851in}}{\pgfqpoint{-0.019642in}{0.019642in}}%
\pgfpathcurveto{\pgfqpoint{-0.024851in}{0.014433in}}{\pgfqpoint{-0.027778in}{0.007367in}}{\pgfqpoint{-0.027778in}{0.000000in}}%
\pgfpathcurveto{\pgfqpoint{-0.027778in}{-0.007367in}}{\pgfqpoint{-0.024851in}{-0.014433in}}{\pgfqpoint{-0.019642in}{-0.019642in}}%
\pgfpathcurveto{\pgfqpoint{-0.014433in}{-0.024851in}}{\pgfqpoint{-0.007367in}{-0.027778in}}{\pgfqpoint{0.000000in}{-0.027778in}}%
\pgfpathclose%
\pgfusepath{stroke,fill}%
}%
\begin{pgfscope}%
\pgfsys@transformshift{1.469026in}{0.749474in}%
\pgfsys@useobject{currentmarker}{}%
\end{pgfscope}%
\end{pgfscope}%
\begin{pgfscope}%
\pgfpathrectangle{\pgfqpoint{0.550713in}{0.127635in}}{\pgfqpoint{3.194133in}{1.496807in}}%
\pgfusepath{clip}%
\pgfsetrectcap%
\pgfsetroundjoin%
\pgfsetlinewidth{0.752812pt}%
\definecolor{currentstroke}{rgb}{0.000000,0.000000,0.000000}%
\pgfsetstrokecolor{currentstroke}%
\pgfsetdash{}{0pt}%
\pgfpathmoveto{\pgfqpoint{1.550476in}{0.863973in}}%
\pgfpathlineto{\pgfqpoint{1.706989in}{0.863973in}}%
\pgfusepath{stroke}%
\end{pgfscope}%
\begin{pgfscope}%
\pgfpathrectangle{\pgfqpoint{0.550713in}{0.127635in}}{\pgfqpoint{3.194133in}{1.496807in}}%
\pgfusepath{clip}%
\pgfsetbuttcap%
\pgfsetroundjoin%
\definecolor{currentfill}{rgb}{1.000000,1.000000,1.000000}%
\pgfsetfillcolor{currentfill}%
\pgfsetlinewidth{1.003750pt}%
\definecolor{currentstroke}{rgb}{0.000000,0.000000,0.000000}%
\pgfsetstrokecolor{currentstroke}%
\pgfsetdash{}{0pt}%
\pgfsys@defobject{currentmarker}{\pgfqpoint{-0.027778in}{-0.027778in}}{\pgfqpoint{0.027778in}{0.027778in}}{%
\pgfpathmoveto{\pgfqpoint{0.000000in}{-0.027778in}}%
\pgfpathcurveto{\pgfqpoint{0.007367in}{-0.027778in}}{\pgfqpoint{0.014433in}{-0.024851in}}{\pgfqpoint{0.019642in}{-0.019642in}}%
\pgfpathcurveto{\pgfqpoint{0.024851in}{-0.014433in}}{\pgfqpoint{0.027778in}{-0.007367in}}{\pgfqpoint{0.027778in}{0.000000in}}%
\pgfpathcurveto{\pgfqpoint{0.027778in}{0.007367in}}{\pgfqpoint{0.024851in}{0.014433in}}{\pgfqpoint{0.019642in}{0.019642in}}%
\pgfpathcurveto{\pgfqpoint{0.014433in}{0.024851in}}{\pgfqpoint{0.007367in}{0.027778in}}{\pgfqpoint{0.000000in}{0.027778in}}%
\pgfpathcurveto{\pgfqpoint{-0.007367in}{0.027778in}}{\pgfqpoint{-0.014433in}{0.024851in}}{\pgfqpoint{-0.019642in}{0.019642in}}%
\pgfpathcurveto{\pgfqpoint{-0.024851in}{0.014433in}}{\pgfqpoint{-0.027778in}{0.007367in}}{\pgfqpoint{-0.027778in}{0.000000in}}%
\pgfpathcurveto{\pgfqpoint{-0.027778in}{-0.007367in}}{\pgfqpoint{-0.024851in}{-0.014433in}}{\pgfqpoint{-0.019642in}{-0.019642in}}%
\pgfpathcurveto{\pgfqpoint{-0.014433in}{-0.024851in}}{\pgfqpoint{-0.007367in}{-0.027778in}}{\pgfqpoint{0.000000in}{-0.027778in}}%
\pgfpathclose%
\pgfusepath{stroke,fill}%
}%
\begin{pgfscope}%
\pgfsys@transformshift{1.628733in}{0.861943in}%
\pgfsys@useobject{currentmarker}{}%
\end{pgfscope}%
\end{pgfscope}%
\begin{pgfscope}%
\pgfpathrectangle{\pgfqpoint{0.550713in}{0.127635in}}{\pgfqpoint{3.194133in}{1.496807in}}%
\pgfusepath{clip}%
\pgfsetrectcap%
\pgfsetroundjoin%
\pgfsetlinewidth{0.752812pt}%
\definecolor{currentstroke}{rgb}{0.000000,0.000000,0.000000}%
\pgfsetstrokecolor{currentstroke}%
\pgfsetdash{}{0pt}%
\pgfpathmoveto{\pgfqpoint{1.790036in}{1.020664in}}%
\pgfpathlineto{\pgfqpoint{1.946549in}{1.020664in}}%
\pgfusepath{stroke}%
\end{pgfscope}%
\begin{pgfscope}%
\pgfpathrectangle{\pgfqpoint{0.550713in}{0.127635in}}{\pgfqpoint{3.194133in}{1.496807in}}%
\pgfusepath{clip}%
\pgfsetbuttcap%
\pgfsetroundjoin%
\definecolor{currentfill}{rgb}{1.000000,1.000000,1.000000}%
\pgfsetfillcolor{currentfill}%
\pgfsetlinewidth{1.003750pt}%
\definecolor{currentstroke}{rgb}{0.000000,0.000000,0.000000}%
\pgfsetstrokecolor{currentstroke}%
\pgfsetdash{}{0pt}%
\pgfsys@defobject{currentmarker}{\pgfqpoint{-0.027778in}{-0.027778in}}{\pgfqpoint{0.027778in}{0.027778in}}{%
\pgfpathmoveto{\pgfqpoint{0.000000in}{-0.027778in}}%
\pgfpathcurveto{\pgfqpoint{0.007367in}{-0.027778in}}{\pgfqpoint{0.014433in}{-0.024851in}}{\pgfqpoint{0.019642in}{-0.019642in}}%
\pgfpathcurveto{\pgfqpoint{0.024851in}{-0.014433in}}{\pgfqpoint{0.027778in}{-0.007367in}}{\pgfqpoint{0.027778in}{0.000000in}}%
\pgfpathcurveto{\pgfqpoint{0.027778in}{0.007367in}}{\pgfqpoint{0.024851in}{0.014433in}}{\pgfqpoint{0.019642in}{0.019642in}}%
\pgfpathcurveto{\pgfqpoint{0.014433in}{0.024851in}}{\pgfqpoint{0.007367in}{0.027778in}}{\pgfqpoint{0.000000in}{0.027778in}}%
\pgfpathcurveto{\pgfqpoint{-0.007367in}{0.027778in}}{\pgfqpoint{-0.014433in}{0.024851in}}{\pgfqpoint{-0.019642in}{0.019642in}}%
\pgfpathcurveto{\pgfqpoint{-0.024851in}{0.014433in}}{\pgfqpoint{-0.027778in}{0.007367in}}{\pgfqpoint{-0.027778in}{0.000000in}}%
\pgfpathcurveto{\pgfqpoint{-0.027778in}{-0.007367in}}{\pgfqpoint{-0.024851in}{-0.014433in}}{\pgfqpoint{-0.019642in}{-0.019642in}}%
\pgfpathcurveto{\pgfqpoint{-0.014433in}{-0.024851in}}{\pgfqpoint{-0.007367in}{-0.027778in}}{\pgfqpoint{0.000000in}{-0.027778in}}%
\pgfpathclose%
\pgfusepath{stroke,fill}%
}%
\begin{pgfscope}%
\pgfsys@transformshift{1.868293in}{0.998369in}%
\pgfsys@useobject{currentmarker}{}%
\end{pgfscope}%
\end{pgfscope}%
\begin{pgfscope}%
\pgfpathrectangle{\pgfqpoint{0.550713in}{0.127635in}}{\pgfqpoint{3.194133in}{1.496807in}}%
\pgfusepath{clip}%
\pgfsetrectcap%
\pgfsetroundjoin%
\pgfsetlinewidth{0.752812pt}%
\definecolor{currentstroke}{rgb}{0.000000,0.000000,0.000000}%
\pgfsetstrokecolor{currentstroke}%
\pgfsetdash{}{0pt}%
\pgfpathmoveto{\pgfqpoint{1.949743in}{1.038087in}}%
\pgfpathlineto{\pgfqpoint{2.106255in}{1.038087in}}%
\pgfusepath{stroke}%
\end{pgfscope}%
\begin{pgfscope}%
\pgfpathrectangle{\pgfqpoint{0.550713in}{0.127635in}}{\pgfqpoint{3.194133in}{1.496807in}}%
\pgfusepath{clip}%
\pgfsetbuttcap%
\pgfsetroundjoin%
\definecolor{currentfill}{rgb}{1.000000,1.000000,1.000000}%
\pgfsetfillcolor{currentfill}%
\pgfsetlinewidth{1.003750pt}%
\definecolor{currentstroke}{rgb}{0.000000,0.000000,0.000000}%
\pgfsetstrokecolor{currentstroke}%
\pgfsetdash{}{0pt}%
\pgfsys@defobject{currentmarker}{\pgfqpoint{-0.027778in}{-0.027778in}}{\pgfqpoint{0.027778in}{0.027778in}}{%
\pgfpathmoveto{\pgfqpoint{0.000000in}{-0.027778in}}%
\pgfpathcurveto{\pgfqpoint{0.007367in}{-0.027778in}}{\pgfqpoint{0.014433in}{-0.024851in}}{\pgfqpoint{0.019642in}{-0.019642in}}%
\pgfpathcurveto{\pgfqpoint{0.024851in}{-0.014433in}}{\pgfqpoint{0.027778in}{-0.007367in}}{\pgfqpoint{0.027778in}{0.000000in}}%
\pgfpathcurveto{\pgfqpoint{0.027778in}{0.007367in}}{\pgfqpoint{0.024851in}{0.014433in}}{\pgfqpoint{0.019642in}{0.019642in}}%
\pgfpathcurveto{\pgfqpoint{0.014433in}{0.024851in}}{\pgfqpoint{0.007367in}{0.027778in}}{\pgfqpoint{0.000000in}{0.027778in}}%
\pgfpathcurveto{\pgfqpoint{-0.007367in}{0.027778in}}{\pgfqpoint{-0.014433in}{0.024851in}}{\pgfqpoint{-0.019642in}{0.019642in}}%
\pgfpathcurveto{\pgfqpoint{-0.024851in}{0.014433in}}{\pgfqpoint{-0.027778in}{0.007367in}}{\pgfqpoint{-0.027778in}{0.000000in}}%
\pgfpathcurveto{\pgfqpoint{-0.027778in}{-0.007367in}}{\pgfqpoint{-0.024851in}{-0.014433in}}{\pgfqpoint{-0.019642in}{-0.019642in}}%
\pgfpathcurveto{\pgfqpoint{-0.014433in}{-0.024851in}}{\pgfqpoint{-0.007367in}{-0.027778in}}{\pgfqpoint{0.000000in}{-0.027778in}}%
\pgfpathclose%
\pgfusepath{stroke,fill}%
}%
\begin{pgfscope}%
\pgfsys@transformshift{2.027999in}{1.039864in}%
\pgfsys@useobject{currentmarker}{}%
\end{pgfscope}%
\end{pgfscope}%
\begin{pgfscope}%
\pgfpathrectangle{\pgfqpoint{0.550713in}{0.127635in}}{\pgfqpoint{3.194133in}{1.496807in}}%
\pgfusepath{clip}%
\pgfsetrectcap%
\pgfsetroundjoin%
\pgfsetlinewidth{0.752812pt}%
\definecolor{currentstroke}{rgb}{0.000000,0.000000,0.000000}%
\pgfsetstrokecolor{currentstroke}%
\pgfsetdash{}{0pt}%
\pgfpathmoveto{\pgfqpoint{2.189303in}{0.573672in}}%
\pgfpathlineto{\pgfqpoint{2.345815in}{0.573672in}}%
\pgfusepath{stroke}%
\end{pgfscope}%
\begin{pgfscope}%
\pgfpathrectangle{\pgfqpoint{0.550713in}{0.127635in}}{\pgfqpoint{3.194133in}{1.496807in}}%
\pgfusepath{clip}%
\pgfsetbuttcap%
\pgfsetroundjoin%
\definecolor{currentfill}{rgb}{1.000000,1.000000,1.000000}%
\pgfsetfillcolor{currentfill}%
\pgfsetlinewidth{1.003750pt}%
\definecolor{currentstroke}{rgb}{0.000000,0.000000,0.000000}%
\pgfsetstrokecolor{currentstroke}%
\pgfsetdash{}{0pt}%
\pgfsys@defobject{currentmarker}{\pgfqpoint{-0.027778in}{-0.027778in}}{\pgfqpoint{0.027778in}{0.027778in}}{%
\pgfpathmoveto{\pgfqpoint{0.000000in}{-0.027778in}}%
\pgfpathcurveto{\pgfqpoint{0.007367in}{-0.027778in}}{\pgfqpoint{0.014433in}{-0.024851in}}{\pgfqpoint{0.019642in}{-0.019642in}}%
\pgfpathcurveto{\pgfqpoint{0.024851in}{-0.014433in}}{\pgfqpoint{0.027778in}{-0.007367in}}{\pgfqpoint{0.027778in}{0.000000in}}%
\pgfpathcurveto{\pgfqpoint{0.027778in}{0.007367in}}{\pgfqpoint{0.024851in}{0.014433in}}{\pgfqpoint{0.019642in}{0.019642in}}%
\pgfpathcurveto{\pgfqpoint{0.014433in}{0.024851in}}{\pgfqpoint{0.007367in}{0.027778in}}{\pgfqpoint{0.000000in}{0.027778in}}%
\pgfpathcurveto{\pgfqpoint{-0.007367in}{0.027778in}}{\pgfqpoint{-0.014433in}{0.024851in}}{\pgfqpoint{-0.019642in}{0.019642in}}%
\pgfpathcurveto{\pgfqpoint{-0.024851in}{0.014433in}}{\pgfqpoint{-0.027778in}{0.007367in}}{\pgfqpoint{-0.027778in}{0.000000in}}%
\pgfpathcurveto{\pgfqpoint{-0.027778in}{-0.007367in}}{\pgfqpoint{-0.024851in}{-0.014433in}}{\pgfqpoint{-0.019642in}{-0.019642in}}%
\pgfpathcurveto{\pgfqpoint{-0.014433in}{-0.024851in}}{\pgfqpoint{-0.007367in}{-0.027778in}}{\pgfqpoint{0.000000in}{-0.027778in}}%
\pgfpathclose%
\pgfusepath{stroke,fill}%
}%
\begin{pgfscope}%
\pgfsys@transformshift{2.267559in}{0.644817in}%
\pgfsys@useobject{currentmarker}{}%
\end{pgfscope}%
\end{pgfscope}%
\begin{pgfscope}%
\pgfpathrectangle{\pgfqpoint{0.550713in}{0.127635in}}{\pgfqpoint{3.194133in}{1.496807in}}%
\pgfusepath{clip}%
\pgfsetrectcap%
\pgfsetroundjoin%
\pgfsetlinewidth{0.752812pt}%
\definecolor{currentstroke}{rgb}{0.000000,0.000000,0.000000}%
\pgfsetstrokecolor{currentstroke}%
\pgfsetdash{}{0pt}%
\pgfpathmoveto{\pgfqpoint{2.349010in}{0.704211in}}%
\pgfpathlineto{\pgfqpoint{2.505522in}{0.704211in}}%
\pgfusepath{stroke}%
\end{pgfscope}%
\begin{pgfscope}%
\pgfpathrectangle{\pgfqpoint{0.550713in}{0.127635in}}{\pgfqpoint{3.194133in}{1.496807in}}%
\pgfusepath{clip}%
\pgfsetbuttcap%
\pgfsetroundjoin%
\definecolor{currentfill}{rgb}{1.000000,1.000000,1.000000}%
\pgfsetfillcolor{currentfill}%
\pgfsetlinewidth{1.003750pt}%
\definecolor{currentstroke}{rgb}{0.000000,0.000000,0.000000}%
\pgfsetstrokecolor{currentstroke}%
\pgfsetdash{}{0pt}%
\pgfsys@defobject{currentmarker}{\pgfqpoint{-0.027778in}{-0.027778in}}{\pgfqpoint{0.027778in}{0.027778in}}{%
\pgfpathmoveto{\pgfqpoint{0.000000in}{-0.027778in}}%
\pgfpathcurveto{\pgfqpoint{0.007367in}{-0.027778in}}{\pgfqpoint{0.014433in}{-0.024851in}}{\pgfqpoint{0.019642in}{-0.019642in}}%
\pgfpathcurveto{\pgfqpoint{0.024851in}{-0.014433in}}{\pgfqpoint{0.027778in}{-0.007367in}}{\pgfqpoint{0.027778in}{0.000000in}}%
\pgfpathcurveto{\pgfqpoint{0.027778in}{0.007367in}}{\pgfqpoint{0.024851in}{0.014433in}}{\pgfqpoint{0.019642in}{0.019642in}}%
\pgfpathcurveto{\pgfqpoint{0.014433in}{0.024851in}}{\pgfqpoint{0.007367in}{0.027778in}}{\pgfqpoint{0.000000in}{0.027778in}}%
\pgfpathcurveto{\pgfqpoint{-0.007367in}{0.027778in}}{\pgfqpoint{-0.014433in}{0.024851in}}{\pgfqpoint{-0.019642in}{0.019642in}}%
\pgfpathcurveto{\pgfqpoint{-0.024851in}{0.014433in}}{\pgfqpoint{-0.027778in}{0.007367in}}{\pgfqpoint{-0.027778in}{0.000000in}}%
\pgfpathcurveto{\pgfqpoint{-0.027778in}{-0.007367in}}{\pgfqpoint{-0.024851in}{-0.014433in}}{\pgfqpoint{-0.019642in}{-0.019642in}}%
\pgfpathcurveto{\pgfqpoint{-0.014433in}{-0.024851in}}{\pgfqpoint{-0.007367in}{-0.027778in}}{\pgfqpoint{0.000000in}{-0.027778in}}%
\pgfpathclose%
\pgfusepath{stroke,fill}%
}%
\begin{pgfscope}%
\pgfsys@transformshift{2.427266in}{0.782764in}%
\pgfsys@useobject{currentmarker}{}%
\end{pgfscope}%
\end{pgfscope}%
\begin{pgfscope}%
\pgfpathrectangle{\pgfqpoint{0.550713in}{0.127635in}}{\pgfqpoint{3.194133in}{1.496807in}}%
\pgfusepath{clip}%
\pgfsetrectcap%
\pgfsetroundjoin%
\pgfsetlinewidth{0.752812pt}%
\definecolor{currentstroke}{rgb}{0.000000,0.000000,0.000000}%
\pgfsetstrokecolor{currentstroke}%
\pgfsetdash{}{0pt}%
\pgfpathmoveto{\pgfqpoint{2.588570in}{0.698445in}}%
\pgfpathlineto{\pgfqpoint{2.745082in}{0.698445in}}%
\pgfusepath{stroke}%
\end{pgfscope}%
\begin{pgfscope}%
\pgfpathrectangle{\pgfqpoint{0.550713in}{0.127635in}}{\pgfqpoint{3.194133in}{1.496807in}}%
\pgfusepath{clip}%
\pgfsetbuttcap%
\pgfsetroundjoin%
\definecolor{currentfill}{rgb}{1.000000,1.000000,1.000000}%
\pgfsetfillcolor{currentfill}%
\pgfsetlinewidth{1.003750pt}%
\definecolor{currentstroke}{rgb}{0.000000,0.000000,0.000000}%
\pgfsetstrokecolor{currentstroke}%
\pgfsetdash{}{0pt}%
\pgfsys@defobject{currentmarker}{\pgfqpoint{-0.027778in}{-0.027778in}}{\pgfqpoint{0.027778in}{0.027778in}}{%
\pgfpathmoveto{\pgfqpoint{0.000000in}{-0.027778in}}%
\pgfpathcurveto{\pgfqpoint{0.007367in}{-0.027778in}}{\pgfqpoint{0.014433in}{-0.024851in}}{\pgfqpoint{0.019642in}{-0.019642in}}%
\pgfpathcurveto{\pgfqpoint{0.024851in}{-0.014433in}}{\pgfqpoint{0.027778in}{-0.007367in}}{\pgfqpoint{0.027778in}{0.000000in}}%
\pgfpathcurveto{\pgfqpoint{0.027778in}{0.007367in}}{\pgfqpoint{0.024851in}{0.014433in}}{\pgfqpoint{0.019642in}{0.019642in}}%
\pgfpathcurveto{\pgfqpoint{0.014433in}{0.024851in}}{\pgfqpoint{0.007367in}{0.027778in}}{\pgfqpoint{0.000000in}{0.027778in}}%
\pgfpathcurveto{\pgfqpoint{-0.007367in}{0.027778in}}{\pgfqpoint{-0.014433in}{0.024851in}}{\pgfqpoint{-0.019642in}{0.019642in}}%
\pgfpathcurveto{\pgfqpoint{-0.024851in}{0.014433in}}{\pgfqpoint{-0.027778in}{0.007367in}}{\pgfqpoint{-0.027778in}{0.000000in}}%
\pgfpathcurveto{\pgfqpoint{-0.027778in}{-0.007367in}}{\pgfqpoint{-0.024851in}{-0.014433in}}{\pgfqpoint{-0.019642in}{-0.019642in}}%
\pgfpathcurveto{\pgfqpoint{-0.014433in}{-0.024851in}}{\pgfqpoint{-0.007367in}{-0.027778in}}{\pgfqpoint{0.000000in}{-0.027778in}}%
\pgfpathclose%
\pgfusepath{stroke,fill}%
}%
\begin{pgfscope}%
\pgfsys@transformshift{2.666826in}{0.737049in}%
\pgfsys@useobject{currentmarker}{}%
\end{pgfscope}%
\end{pgfscope}%
\begin{pgfscope}%
\pgfpathrectangle{\pgfqpoint{0.550713in}{0.127635in}}{\pgfqpoint{3.194133in}{1.496807in}}%
\pgfusepath{clip}%
\pgfsetrectcap%
\pgfsetroundjoin%
\pgfsetlinewidth{0.752812pt}%
\definecolor{currentstroke}{rgb}{0.000000,0.000000,0.000000}%
\pgfsetstrokecolor{currentstroke}%
\pgfsetdash{}{0pt}%
\pgfpathmoveto{\pgfqpoint{2.748276in}{0.923024in}}%
\pgfpathlineto{\pgfqpoint{2.904789in}{0.923024in}}%
\pgfusepath{stroke}%
\end{pgfscope}%
\begin{pgfscope}%
\pgfpathrectangle{\pgfqpoint{0.550713in}{0.127635in}}{\pgfqpoint{3.194133in}{1.496807in}}%
\pgfusepath{clip}%
\pgfsetbuttcap%
\pgfsetroundjoin%
\definecolor{currentfill}{rgb}{1.000000,1.000000,1.000000}%
\pgfsetfillcolor{currentfill}%
\pgfsetlinewidth{1.003750pt}%
\definecolor{currentstroke}{rgb}{0.000000,0.000000,0.000000}%
\pgfsetstrokecolor{currentstroke}%
\pgfsetdash{}{0pt}%
\pgfsys@defobject{currentmarker}{\pgfqpoint{-0.027778in}{-0.027778in}}{\pgfqpoint{0.027778in}{0.027778in}}{%
\pgfpathmoveto{\pgfqpoint{0.000000in}{-0.027778in}}%
\pgfpathcurveto{\pgfqpoint{0.007367in}{-0.027778in}}{\pgfqpoint{0.014433in}{-0.024851in}}{\pgfqpoint{0.019642in}{-0.019642in}}%
\pgfpathcurveto{\pgfqpoint{0.024851in}{-0.014433in}}{\pgfqpoint{0.027778in}{-0.007367in}}{\pgfqpoint{0.027778in}{0.000000in}}%
\pgfpathcurveto{\pgfqpoint{0.027778in}{0.007367in}}{\pgfqpoint{0.024851in}{0.014433in}}{\pgfqpoint{0.019642in}{0.019642in}}%
\pgfpathcurveto{\pgfqpoint{0.014433in}{0.024851in}}{\pgfqpoint{0.007367in}{0.027778in}}{\pgfqpoint{0.000000in}{0.027778in}}%
\pgfpathcurveto{\pgfqpoint{-0.007367in}{0.027778in}}{\pgfqpoint{-0.014433in}{0.024851in}}{\pgfqpoint{-0.019642in}{0.019642in}}%
\pgfpathcurveto{\pgfqpoint{-0.024851in}{0.014433in}}{\pgfqpoint{-0.027778in}{0.007367in}}{\pgfqpoint{-0.027778in}{0.000000in}}%
\pgfpathcurveto{\pgfqpoint{-0.027778in}{-0.007367in}}{\pgfqpoint{-0.024851in}{-0.014433in}}{\pgfqpoint{-0.019642in}{-0.019642in}}%
\pgfpathcurveto{\pgfqpoint{-0.014433in}{-0.024851in}}{\pgfqpoint{-0.007367in}{-0.027778in}}{\pgfqpoint{0.000000in}{-0.027778in}}%
\pgfpathclose%
\pgfusepath{stroke,fill}%
}%
\begin{pgfscope}%
\pgfsys@transformshift{2.826532in}{0.929546in}%
\pgfsys@useobject{currentmarker}{}%
\end{pgfscope}%
\end{pgfscope}%
\begin{pgfscope}%
\pgfpathrectangle{\pgfqpoint{0.550713in}{0.127635in}}{\pgfqpoint{3.194133in}{1.496807in}}%
\pgfusepath{clip}%
\pgfsetrectcap%
\pgfsetroundjoin%
\pgfsetlinewidth{0.752812pt}%
\definecolor{currentstroke}{rgb}{0.000000,0.000000,0.000000}%
\pgfsetstrokecolor{currentstroke}%
\pgfsetdash{}{0pt}%
\pgfpathmoveto{\pgfqpoint{2.987836in}{0.584749in}}%
\pgfpathlineto{\pgfqpoint{3.144349in}{0.584749in}}%
\pgfusepath{stroke}%
\end{pgfscope}%
\begin{pgfscope}%
\pgfpathrectangle{\pgfqpoint{0.550713in}{0.127635in}}{\pgfqpoint{3.194133in}{1.496807in}}%
\pgfusepath{clip}%
\pgfsetbuttcap%
\pgfsetroundjoin%
\definecolor{currentfill}{rgb}{1.000000,1.000000,1.000000}%
\pgfsetfillcolor{currentfill}%
\pgfsetlinewidth{1.003750pt}%
\definecolor{currentstroke}{rgb}{0.000000,0.000000,0.000000}%
\pgfsetstrokecolor{currentstroke}%
\pgfsetdash{}{0pt}%
\pgfsys@defobject{currentmarker}{\pgfqpoint{-0.027778in}{-0.027778in}}{\pgfqpoint{0.027778in}{0.027778in}}{%
\pgfpathmoveto{\pgfqpoint{0.000000in}{-0.027778in}}%
\pgfpathcurveto{\pgfqpoint{0.007367in}{-0.027778in}}{\pgfqpoint{0.014433in}{-0.024851in}}{\pgfqpoint{0.019642in}{-0.019642in}}%
\pgfpathcurveto{\pgfqpoint{0.024851in}{-0.014433in}}{\pgfqpoint{0.027778in}{-0.007367in}}{\pgfqpoint{0.027778in}{0.000000in}}%
\pgfpathcurveto{\pgfqpoint{0.027778in}{0.007367in}}{\pgfqpoint{0.024851in}{0.014433in}}{\pgfqpoint{0.019642in}{0.019642in}}%
\pgfpathcurveto{\pgfqpoint{0.014433in}{0.024851in}}{\pgfqpoint{0.007367in}{0.027778in}}{\pgfqpoint{0.000000in}{0.027778in}}%
\pgfpathcurveto{\pgfqpoint{-0.007367in}{0.027778in}}{\pgfqpoint{-0.014433in}{0.024851in}}{\pgfqpoint{-0.019642in}{0.019642in}}%
\pgfpathcurveto{\pgfqpoint{-0.024851in}{0.014433in}}{\pgfqpoint{-0.027778in}{0.007367in}}{\pgfqpoint{-0.027778in}{0.000000in}}%
\pgfpathcurveto{\pgfqpoint{-0.027778in}{-0.007367in}}{\pgfqpoint{-0.024851in}{-0.014433in}}{\pgfqpoint{-0.019642in}{-0.019642in}}%
\pgfpathcurveto{\pgfqpoint{-0.014433in}{-0.024851in}}{\pgfqpoint{-0.007367in}{-0.027778in}}{\pgfqpoint{0.000000in}{-0.027778in}}%
\pgfpathclose%
\pgfusepath{stroke,fill}%
}%
\begin{pgfscope}%
\pgfsys@transformshift{3.066092in}{0.607815in}%
\pgfsys@useobject{currentmarker}{}%
\end{pgfscope}%
\end{pgfscope}%
\begin{pgfscope}%
\pgfpathrectangle{\pgfqpoint{0.550713in}{0.127635in}}{\pgfqpoint{3.194133in}{1.496807in}}%
\pgfusepath{clip}%
\pgfsetrectcap%
\pgfsetroundjoin%
\pgfsetlinewidth{0.752812pt}%
\definecolor{currentstroke}{rgb}{0.000000,0.000000,0.000000}%
\pgfsetstrokecolor{currentstroke}%
\pgfsetdash{}{0pt}%
\pgfpathmoveto{\pgfqpoint{3.147543in}{0.625047in}}%
\pgfpathlineto{\pgfqpoint{3.304055in}{0.625047in}}%
\pgfusepath{stroke}%
\end{pgfscope}%
\begin{pgfscope}%
\pgfpathrectangle{\pgfqpoint{0.550713in}{0.127635in}}{\pgfqpoint{3.194133in}{1.496807in}}%
\pgfusepath{clip}%
\pgfsetbuttcap%
\pgfsetroundjoin%
\definecolor{currentfill}{rgb}{1.000000,1.000000,1.000000}%
\pgfsetfillcolor{currentfill}%
\pgfsetlinewidth{1.003750pt}%
\definecolor{currentstroke}{rgb}{0.000000,0.000000,0.000000}%
\pgfsetstrokecolor{currentstroke}%
\pgfsetdash{}{0pt}%
\pgfsys@defobject{currentmarker}{\pgfqpoint{-0.027778in}{-0.027778in}}{\pgfqpoint{0.027778in}{0.027778in}}{%
\pgfpathmoveto{\pgfqpoint{0.000000in}{-0.027778in}}%
\pgfpathcurveto{\pgfqpoint{0.007367in}{-0.027778in}}{\pgfqpoint{0.014433in}{-0.024851in}}{\pgfqpoint{0.019642in}{-0.019642in}}%
\pgfpathcurveto{\pgfqpoint{0.024851in}{-0.014433in}}{\pgfqpoint{0.027778in}{-0.007367in}}{\pgfqpoint{0.027778in}{0.000000in}}%
\pgfpathcurveto{\pgfqpoint{0.027778in}{0.007367in}}{\pgfqpoint{0.024851in}{0.014433in}}{\pgfqpoint{0.019642in}{0.019642in}}%
\pgfpathcurveto{\pgfqpoint{0.014433in}{0.024851in}}{\pgfqpoint{0.007367in}{0.027778in}}{\pgfqpoint{0.000000in}{0.027778in}}%
\pgfpathcurveto{\pgfqpoint{-0.007367in}{0.027778in}}{\pgfqpoint{-0.014433in}{0.024851in}}{\pgfqpoint{-0.019642in}{0.019642in}}%
\pgfpathcurveto{\pgfqpoint{-0.024851in}{0.014433in}}{\pgfqpoint{-0.027778in}{0.007367in}}{\pgfqpoint{-0.027778in}{0.000000in}}%
\pgfpathcurveto{\pgfqpoint{-0.027778in}{-0.007367in}}{\pgfqpoint{-0.024851in}{-0.014433in}}{\pgfqpoint{-0.019642in}{-0.019642in}}%
\pgfpathcurveto{\pgfqpoint{-0.014433in}{-0.024851in}}{\pgfqpoint{-0.007367in}{-0.027778in}}{\pgfqpoint{0.000000in}{-0.027778in}}%
\pgfpathclose%
\pgfusepath{stroke,fill}%
}%
\begin{pgfscope}%
\pgfsys@transformshift{3.225799in}{0.645384in}%
\pgfsys@useobject{currentmarker}{}%
\end{pgfscope}%
\end{pgfscope}%
\begin{pgfscope}%
\pgfpathrectangle{\pgfqpoint{0.550713in}{0.127635in}}{\pgfqpoint{3.194133in}{1.496807in}}%
\pgfusepath{clip}%
\pgfsetrectcap%
\pgfsetroundjoin%
\pgfsetlinewidth{0.752812pt}%
\definecolor{currentstroke}{rgb}{0.000000,0.000000,0.000000}%
\pgfsetstrokecolor{currentstroke}%
\pgfsetdash{}{0pt}%
\pgfpathmoveto{\pgfqpoint{3.387103in}{0.785877in}}%
\pgfpathlineto{\pgfqpoint{3.543615in}{0.785877in}}%
\pgfusepath{stroke}%
\end{pgfscope}%
\begin{pgfscope}%
\pgfpathrectangle{\pgfqpoint{0.550713in}{0.127635in}}{\pgfqpoint{3.194133in}{1.496807in}}%
\pgfusepath{clip}%
\pgfsetbuttcap%
\pgfsetroundjoin%
\definecolor{currentfill}{rgb}{1.000000,1.000000,1.000000}%
\pgfsetfillcolor{currentfill}%
\pgfsetlinewidth{1.003750pt}%
\definecolor{currentstroke}{rgb}{0.000000,0.000000,0.000000}%
\pgfsetstrokecolor{currentstroke}%
\pgfsetdash{}{0pt}%
\pgfsys@defobject{currentmarker}{\pgfqpoint{-0.027778in}{-0.027778in}}{\pgfqpoint{0.027778in}{0.027778in}}{%
\pgfpathmoveto{\pgfqpoint{0.000000in}{-0.027778in}}%
\pgfpathcurveto{\pgfqpoint{0.007367in}{-0.027778in}}{\pgfqpoint{0.014433in}{-0.024851in}}{\pgfqpoint{0.019642in}{-0.019642in}}%
\pgfpathcurveto{\pgfqpoint{0.024851in}{-0.014433in}}{\pgfqpoint{0.027778in}{-0.007367in}}{\pgfqpoint{0.027778in}{0.000000in}}%
\pgfpathcurveto{\pgfqpoint{0.027778in}{0.007367in}}{\pgfqpoint{0.024851in}{0.014433in}}{\pgfqpoint{0.019642in}{0.019642in}}%
\pgfpathcurveto{\pgfqpoint{0.014433in}{0.024851in}}{\pgfqpoint{0.007367in}{0.027778in}}{\pgfqpoint{0.000000in}{0.027778in}}%
\pgfpathcurveto{\pgfqpoint{-0.007367in}{0.027778in}}{\pgfqpoint{-0.014433in}{0.024851in}}{\pgfqpoint{-0.019642in}{0.019642in}}%
\pgfpathcurveto{\pgfqpoint{-0.024851in}{0.014433in}}{\pgfqpoint{-0.027778in}{0.007367in}}{\pgfqpoint{-0.027778in}{0.000000in}}%
\pgfpathcurveto{\pgfqpoint{-0.027778in}{-0.007367in}}{\pgfqpoint{-0.024851in}{-0.014433in}}{\pgfqpoint{-0.019642in}{-0.019642in}}%
\pgfpathcurveto{\pgfqpoint{-0.014433in}{-0.024851in}}{\pgfqpoint{-0.007367in}{-0.027778in}}{\pgfqpoint{0.000000in}{-0.027778in}}%
\pgfpathclose%
\pgfusepath{stroke,fill}%
}%
\begin{pgfscope}%
\pgfsys@transformshift{3.465359in}{0.748679in}%
\pgfsys@useobject{currentmarker}{}%
\end{pgfscope}%
\end{pgfscope}%
\begin{pgfscope}%
\pgfpathrectangle{\pgfqpoint{0.550713in}{0.127635in}}{\pgfqpoint{3.194133in}{1.496807in}}%
\pgfusepath{clip}%
\pgfsetrectcap%
\pgfsetroundjoin%
\pgfsetlinewidth{0.752812pt}%
\definecolor{currentstroke}{rgb}{0.000000,0.000000,0.000000}%
\pgfsetstrokecolor{currentstroke}%
\pgfsetdash{}{0pt}%
\pgfpathmoveto{\pgfqpoint{3.546809in}{0.748189in}}%
\pgfpathlineto{\pgfqpoint{3.703322in}{0.748189in}}%
\pgfusepath{stroke}%
\end{pgfscope}%
\begin{pgfscope}%
\pgfpathrectangle{\pgfqpoint{0.550713in}{0.127635in}}{\pgfqpoint{3.194133in}{1.496807in}}%
\pgfusepath{clip}%
\pgfsetbuttcap%
\pgfsetroundjoin%
\definecolor{currentfill}{rgb}{1.000000,1.000000,1.000000}%
\pgfsetfillcolor{currentfill}%
\pgfsetlinewidth{1.003750pt}%
\definecolor{currentstroke}{rgb}{0.000000,0.000000,0.000000}%
\pgfsetstrokecolor{currentstroke}%
\pgfsetdash{}{0pt}%
\pgfsys@defobject{currentmarker}{\pgfqpoint{-0.027778in}{-0.027778in}}{\pgfqpoint{0.027778in}{0.027778in}}{%
\pgfpathmoveto{\pgfqpoint{0.000000in}{-0.027778in}}%
\pgfpathcurveto{\pgfqpoint{0.007367in}{-0.027778in}}{\pgfqpoint{0.014433in}{-0.024851in}}{\pgfqpoint{0.019642in}{-0.019642in}}%
\pgfpathcurveto{\pgfqpoint{0.024851in}{-0.014433in}}{\pgfqpoint{0.027778in}{-0.007367in}}{\pgfqpoint{0.027778in}{0.000000in}}%
\pgfpathcurveto{\pgfqpoint{0.027778in}{0.007367in}}{\pgfqpoint{0.024851in}{0.014433in}}{\pgfqpoint{0.019642in}{0.019642in}}%
\pgfpathcurveto{\pgfqpoint{0.014433in}{0.024851in}}{\pgfqpoint{0.007367in}{0.027778in}}{\pgfqpoint{0.000000in}{0.027778in}}%
\pgfpathcurveto{\pgfqpoint{-0.007367in}{0.027778in}}{\pgfqpoint{-0.014433in}{0.024851in}}{\pgfqpoint{-0.019642in}{0.019642in}}%
\pgfpathcurveto{\pgfqpoint{-0.024851in}{0.014433in}}{\pgfqpoint{-0.027778in}{0.007367in}}{\pgfqpoint{-0.027778in}{0.000000in}}%
\pgfpathcurveto{\pgfqpoint{-0.027778in}{-0.007367in}}{\pgfqpoint{-0.024851in}{-0.014433in}}{\pgfqpoint{-0.019642in}{-0.019642in}}%
\pgfpathcurveto{\pgfqpoint{-0.014433in}{-0.024851in}}{\pgfqpoint{-0.007367in}{-0.027778in}}{\pgfqpoint{0.000000in}{-0.027778in}}%
\pgfpathclose%
\pgfusepath{stroke,fill}%
}%
\begin{pgfscope}%
\pgfsys@transformshift{3.625066in}{0.728598in}%
\pgfsys@useobject{currentmarker}{}%
\end{pgfscope}%
\end{pgfscope}%
\begin{pgfscope}%
\pgfsetrectcap%
\pgfsetmiterjoin%
\pgfsetlinewidth{0.803000pt}%
\definecolor{currentstroke}{rgb}{0.000000,0.000000,0.000000}%
\pgfsetstrokecolor{currentstroke}%
\pgfsetdash{}{0pt}%
\pgfpathmoveto{\pgfqpoint{0.550713in}{0.127635in}}%
\pgfpathlineto{\pgfqpoint{0.550713in}{1.624441in}}%
\pgfusepath{stroke}%
\end{pgfscope}%
\begin{pgfscope}%
\pgfsetrectcap%
\pgfsetmiterjoin%
\pgfsetlinewidth{0.803000pt}%
\definecolor{currentstroke}{rgb}{0.000000,0.000000,0.000000}%
\pgfsetstrokecolor{currentstroke}%
\pgfsetdash{}{0pt}%
\pgfpathmoveto{\pgfqpoint{3.744846in}{0.127635in}}%
\pgfpathlineto{\pgfqpoint{3.744846in}{1.624441in}}%
\pgfusepath{stroke}%
\end{pgfscope}%
\begin{pgfscope}%
\pgfsetrectcap%
\pgfsetmiterjoin%
\pgfsetlinewidth{0.803000pt}%
\definecolor{currentstroke}{rgb}{0.000000,0.000000,0.000000}%
\pgfsetstrokecolor{currentstroke}%
\pgfsetdash{}{0pt}%
\pgfpathmoveto{\pgfqpoint{0.550713in}{0.127635in}}%
\pgfpathlineto{\pgfqpoint{3.744846in}{0.127635in}}%
\pgfusepath{stroke}%
\end{pgfscope}%
\begin{pgfscope}%
\pgfsetbuttcap%
\pgfsetroundjoin%
\pgfsetlinewidth{1.003750pt}%
\definecolor{currentstroke}{rgb}{0.392157,0.396078,0.403922}%
\pgfsetstrokecolor{currentstroke}%
\pgfsetdash{{3.700000pt}{1.600000pt}}{0.000000pt}%
\pgfpathmoveto{\pgfqpoint{3.869846in}{2.038572in}}%
\pgfpathlineto{\pgfqpoint{4.147623in}{2.038572in}}%
\pgfusepath{stroke}%
\end{pgfscope}%
\begin{pgfscope}%
\definecolor{textcolor}{rgb}{0.000000,0.000000,0.000000}%
\pgfsetstrokecolor{textcolor}%
\pgfsetfillcolor{textcolor}%
\pgftext[x=4.258735in,y=1.989961in,left,base]{\color{textcolor}\rmfamily\fontsize{10.000000}{12.000000}\selectfont Only Exploitation}%
\end{pgfscope}%
\begin{pgfscope}%
\pgfsetbuttcap%
\pgfsetmiterjoin%
\definecolor{currentfill}{rgb}{0.631373,0.062745,0.207843}%
\pgfsetfillcolor{currentfill}%
\pgfsetlinewidth{0.000000pt}%
\definecolor{currentstroke}{rgb}{0.000000,0.000000,0.000000}%
\pgfsetstrokecolor{currentstroke}%
\pgfsetstrokeopacity{0.000000}%
\pgfsetdash{}{0pt}%
\pgfpathmoveto{\pgfqpoint{3.869846in}{1.794822in}}%
\pgfpathlineto{\pgfqpoint{4.147623in}{1.794822in}}%
\pgfpathlineto{\pgfqpoint{4.147623in}{1.892045in}}%
\pgfpathlineto{\pgfqpoint{3.869846in}{1.892045in}}%
\pgfpathclose%
\pgfusepath{fill}%
\end{pgfscope}%
\begin{pgfscope}%
\definecolor{textcolor}{rgb}{0.000000,0.000000,0.000000}%
\pgfsetstrokecolor{textcolor}%
\pgfsetfillcolor{textcolor}%
\pgftext[x=4.258735in,y=1.794822in,left,base]{\color{textcolor}\rmfamily\fontsize{10.000000}{12.000000}\selectfont TV-GP-UCB}%
\end{pgfscope}%
\begin{pgfscope}%
\pgfsetbuttcap%
\pgfsetmiterjoin%
\definecolor{currentfill}{rgb}{0.890196,0.000000,0.400000}%
\pgfsetfillcolor{currentfill}%
\pgfsetlinewidth{0.000000pt}%
\definecolor{currentstroke}{rgb}{0.000000,0.000000,0.000000}%
\pgfsetstrokecolor{currentstroke}%
\pgfsetstrokeopacity{0.000000}%
\pgfsetdash{}{0pt}%
\pgfpathmoveto{\pgfqpoint{3.869846in}{1.601211in}}%
\pgfpathlineto{\pgfqpoint{4.147623in}{1.601211in}}%
\pgfpathlineto{\pgfqpoint{4.147623in}{1.698434in}}%
\pgfpathlineto{\pgfqpoint{3.869846in}{1.698434in}}%
\pgfpathclose%
\pgfusepath{fill}%
\end{pgfscope}%
\begin{pgfscope}%
\definecolor{textcolor}{rgb}{0.000000,0.000000,0.000000}%
\pgfsetstrokecolor{textcolor}%
\pgfsetfillcolor{textcolor}%
\pgftext[x=4.258735in,y=1.601211in,left,base]{\color{textcolor}\rmfamily\fontsize{10.000000}{12.000000}\selectfont SW TV-GP-UCB}%
\end{pgfscope}%
\begin{pgfscope}%
\pgfsetbuttcap%
\pgfsetmiterjoin%
\definecolor{currentfill}{rgb}{0.000000,0.329412,0.623529}%
\pgfsetfillcolor{currentfill}%
\pgfsetlinewidth{0.000000pt}%
\definecolor{currentstroke}{rgb}{0.000000,0.000000,0.000000}%
\pgfsetstrokecolor{currentstroke}%
\pgfsetstrokeopacity{0.000000}%
\pgfsetdash{}{0pt}%
\pgfpathmoveto{\pgfqpoint{3.869846in}{1.407600in}}%
\pgfpathlineto{\pgfqpoint{4.147623in}{1.407600in}}%
\pgfpathlineto{\pgfqpoint{4.147623in}{1.504823in}}%
\pgfpathlineto{\pgfqpoint{3.869846in}{1.504823in}}%
\pgfpathclose%
\pgfusepath{fill}%
\end{pgfscope}%
\begin{pgfscope}%
\definecolor{textcolor}{rgb}{0.000000,0.000000,0.000000}%
\pgfsetstrokecolor{textcolor}%
\pgfsetfillcolor{textcolor}%
\pgftext[x=4.258735in,y=1.407600in,left,base]{\color{textcolor}\rmfamily\fontsize{10.000000}{12.000000}\selectfont UI-TVBO}%
\end{pgfscope}%
\begin{pgfscope}%
\pgfsetbuttcap%
\pgfsetmiterjoin%
\definecolor{currentfill}{rgb}{0.000000,0.380392,0.396078}%
\pgfsetfillcolor{currentfill}%
\pgfsetlinewidth{0.000000pt}%
\definecolor{currentstroke}{rgb}{0.000000,0.000000,0.000000}%
\pgfsetstrokecolor{currentstroke}%
\pgfsetstrokeopacity{0.000000}%
\pgfsetdash{}{0pt}%
\pgfpathmoveto{\pgfqpoint{3.869846in}{1.213989in}}%
\pgfpathlineto{\pgfqpoint{4.147623in}{1.213989in}}%
\pgfpathlineto{\pgfqpoint{4.147623in}{1.311212in}}%
\pgfpathlineto{\pgfqpoint{3.869846in}{1.311212in}}%
\pgfpathclose%
\pgfusepath{fill}%
\end{pgfscope}%
\begin{pgfscope}%
\definecolor{textcolor}{rgb}{0.000000,0.000000,0.000000}%
\pgfsetstrokecolor{textcolor}%
\pgfsetfillcolor{textcolor}%
\pgftext[x=4.258735in,y=1.213989in,left,base]{\color{textcolor}\rmfamily\fontsize{10.000000}{12.000000}\selectfont B UI-TVBO}%
\end{pgfscope}%
\begin{pgfscope}%
\pgfsetbuttcap%
\pgfsetmiterjoin%
\definecolor{currentfill}{rgb}{0.380392,0.129412,0.345098}%
\pgfsetfillcolor{currentfill}%
\pgfsetlinewidth{0.000000pt}%
\definecolor{currentstroke}{rgb}{0.000000,0.000000,0.000000}%
\pgfsetstrokecolor{currentstroke}%
\pgfsetstrokeopacity{0.000000}%
\pgfsetdash{}{0pt}%
\pgfpathmoveto{\pgfqpoint{3.869846in}{1.020379in}}%
\pgfpathlineto{\pgfqpoint{4.147623in}{1.020379in}}%
\pgfpathlineto{\pgfqpoint{4.147623in}{1.117601in}}%
\pgfpathlineto{\pgfqpoint{3.869846in}{1.117601in}}%
\pgfpathclose%
\pgfusepath{fill}%
\end{pgfscope}%
\begin{pgfscope}%
\definecolor{textcolor}{rgb}{0.000000,0.000000,0.000000}%
\pgfsetstrokecolor{textcolor}%
\pgfsetfillcolor{textcolor}%
\pgftext[x=4.258735in,y=1.020379in,left,base]{\color{textcolor}\rmfamily\fontsize{10.000000}{12.000000}\selectfont C-TV-GP-UCB}%
\end{pgfscope}%
\begin{pgfscope}%
\pgfsetbuttcap%
\pgfsetmiterjoin%
\definecolor{currentfill}{rgb}{0.964706,0.658824,0.000000}%
\pgfsetfillcolor{currentfill}%
\pgfsetlinewidth{0.000000pt}%
\definecolor{currentstroke}{rgb}{0.000000,0.000000,0.000000}%
\pgfsetstrokecolor{currentstroke}%
\pgfsetstrokeopacity{0.000000}%
\pgfsetdash{}{0pt}%
\pgfpathmoveto{\pgfqpoint{3.869846in}{0.826768in}}%
\pgfpathlineto{\pgfqpoint{4.147623in}{0.826768in}}%
\pgfpathlineto{\pgfqpoint{4.147623in}{0.923990in}}%
\pgfpathlineto{\pgfqpoint{3.869846in}{0.923990in}}%
\pgfpathclose%
\pgfusepath{fill}%
\end{pgfscope}%
\begin{pgfscope}%
\definecolor{textcolor}{rgb}{0.000000,0.000000,0.000000}%
\pgfsetstrokecolor{textcolor}%
\pgfsetfillcolor{textcolor}%
\pgftext[x=4.258735in,y=0.826768in,left,base]{\color{textcolor}\rmfamily\fontsize{10.000000}{12.000000}\selectfont SW C-TV-GP-UCB}%
\end{pgfscope}%
\begin{pgfscope}%
\pgfsetbuttcap%
\pgfsetmiterjoin%
\definecolor{currentfill}{rgb}{0.341176,0.670588,0.152941}%
\pgfsetfillcolor{currentfill}%
\pgfsetlinewidth{0.000000pt}%
\definecolor{currentstroke}{rgb}{0.000000,0.000000,0.000000}%
\pgfsetstrokecolor{currentstroke}%
\pgfsetstrokeopacity{0.000000}%
\pgfsetdash{}{0pt}%
\pgfpathmoveto{\pgfqpoint{3.869846in}{0.633157in}}%
\pgfpathlineto{\pgfqpoint{4.147623in}{0.633157in}}%
\pgfpathlineto{\pgfqpoint{4.147623in}{0.730379in}}%
\pgfpathlineto{\pgfqpoint{3.869846in}{0.730379in}}%
\pgfpathclose%
\pgfusepath{fill}%
\end{pgfscope}%
\begin{pgfscope}%
\definecolor{textcolor}{rgb}{0.000000,0.000000,0.000000}%
\pgfsetstrokecolor{textcolor}%
\pgfsetfillcolor{textcolor}%
\pgftext[x=4.258735in,y=0.633157in,left,base]{\color{textcolor}\rmfamily\fontsize{10.000000}{12.000000}\selectfont C-UI-TVBO}%
\end{pgfscope}%
\begin{pgfscope}%
\pgfsetbuttcap%
\pgfsetmiterjoin%
\definecolor{currentfill}{rgb}{0.478431,0.435294,0.674510}%
\pgfsetfillcolor{currentfill}%
\pgfsetlinewidth{0.000000pt}%
\definecolor{currentstroke}{rgb}{0.000000,0.000000,0.000000}%
\pgfsetstrokecolor{currentstroke}%
\pgfsetstrokeopacity{0.000000}%
\pgfsetdash{}{0pt}%
\pgfpathmoveto{\pgfqpoint{3.869846in}{0.439546in}}%
\pgfpathlineto{\pgfqpoint{4.147623in}{0.439546in}}%
\pgfpathlineto{\pgfqpoint{4.147623in}{0.536768in}}%
\pgfpathlineto{\pgfqpoint{3.869846in}{0.536768in}}%
\pgfpathclose%
\pgfusepath{fill}%
\end{pgfscope}%
\begin{pgfscope}%
\definecolor{textcolor}{rgb}{0.000000,0.000000,0.000000}%
\pgfsetstrokecolor{textcolor}%
\pgfsetfillcolor{textcolor}%
\pgftext[x=4.258735in,y=0.439546in,left,base]{\color{textcolor}\rmfamily\fontsize{10.000000}{12.000000}\selectfont B C-UI-TVBO}%
\end{pgfscope}%
\begin{pgfscope}%
\pgfsetbuttcap%
\pgfsetroundjoin%
\pgfsetlinewidth{1.003750pt}%
\definecolor{currentstroke}{rgb}{0.392157,0.396078,0.403922}%
\pgfsetstrokecolor{currentstroke}%
\pgfsetdash{{3.700000pt}{1.600000pt}}{0.000000pt}%
\pgfpathmoveto{\pgfqpoint{3.869846in}{2.038572in}}%
\pgfpathlineto{\pgfqpoint{4.147623in}{2.038572in}}%
\pgfusepath{stroke}%
\end{pgfscope}%
\begin{pgfscope}%
\definecolor{textcolor}{rgb}{0.000000,0.000000,0.000000}%
\pgfsetstrokecolor{textcolor}%
\pgfsetfillcolor{textcolor}%
\pgftext[x=4.258735in,y=1.989961in,left,base]{\color{textcolor}\rmfamily\fontsize{10.000000}{12.000000}\selectfont Only Exploitation}%
\end{pgfscope}%
\begin{pgfscope}%
\pgfsetbuttcap%
\pgfsetmiterjoin%
\definecolor{currentfill}{rgb}{0.631373,0.062745,0.207843}%
\pgfsetfillcolor{currentfill}%
\pgfsetlinewidth{0.000000pt}%
\definecolor{currentstroke}{rgb}{0.000000,0.000000,0.000000}%
\pgfsetstrokecolor{currentstroke}%
\pgfsetstrokeopacity{0.000000}%
\pgfsetdash{}{0pt}%
\pgfpathmoveto{\pgfqpoint{3.869846in}{1.794822in}}%
\pgfpathlineto{\pgfqpoint{4.147623in}{1.794822in}}%
\pgfpathlineto{\pgfqpoint{4.147623in}{1.892045in}}%
\pgfpathlineto{\pgfqpoint{3.869846in}{1.892045in}}%
\pgfpathclose%
\pgfusepath{fill}%
\end{pgfscope}%
\begin{pgfscope}%
\definecolor{textcolor}{rgb}{0.000000,0.000000,0.000000}%
\pgfsetstrokecolor{textcolor}%
\pgfsetfillcolor{textcolor}%
\pgftext[x=4.258735in,y=1.794822in,left,base]{\color{textcolor}\rmfamily\fontsize{10.000000}{12.000000}\selectfont TV-GP-UCB}%
\end{pgfscope}%
\begin{pgfscope}%
\pgfsetbuttcap%
\pgfsetmiterjoin%
\definecolor{currentfill}{rgb}{0.890196,0.000000,0.400000}%
\pgfsetfillcolor{currentfill}%
\pgfsetlinewidth{0.000000pt}%
\definecolor{currentstroke}{rgb}{0.000000,0.000000,0.000000}%
\pgfsetstrokecolor{currentstroke}%
\pgfsetstrokeopacity{0.000000}%
\pgfsetdash{}{0pt}%
\pgfpathmoveto{\pgfqpoint{3.869846in}{1.601211in}}%
\pgfpathlineto{\pgfqpoint{4.147623in}{1.601211in}}%
\pgfpathlineto{\pgfqpoint{4.147623in}{1.698434in}}%
\pgfpathlineto{\pgfqpoint{3.869846in}{1.698434in}}%
\pgfpathclose%
\pgfusepath{fill}%
\end{pgfscope}%
\begin{pgfscope}%
\definecolor{textcolor}{rgb}{0.000000,0.000000,0.000000}%
\pgfsetstrokecolor{textcolor}%
\pgfsetfillcolor{textcolor}%
\pgftext[x=4.258735in,y=1.601211in,left,base]{\color{textcolor}\rmfamily\fontsize{10.000000}{12.000000}\selectfont SW TV-GP-UCB}%
\end{pgfscope}%
\begin{pgfscope}%
\pgfsetbuttcap%
\pgfsetmiterjoin%
\definecolor{currentfill}{rgb}{0.000000,0.329412,0.623529}%
\pgfsetfillcolor{currentfill}%
\pgfsetlinewidth{0.000000pt}%
\definecolor{currentstroke}{rgb}{0.000000,0.000000,0.000000}%
\pgfsetstrokecolor{currentstroke}%
\pgfsetstrokeopacity{0.000000}%
\pgfsetdash{}{0pt}%
\pgfpathmoveto{\pgfqpoint{3.869846in}{1.407600in}}%
\pgfpathlineto{\pgfqpoint{4.147623in}{1.407600in}}%
\pgfpathlineto{\pgfqpoint{4.147623in}{1.504823in}}%
\pgfpathlineto{\pgfqpoint{3.869846in}{1.504823in}}%
\pgfpathclose%
\pgfusepath{fill}%
\end{pgfscope}%
\begin{pgfscope}%
\definecolor{textcolor}{rgb}{0.000000,0.000000,0.000000}%
\pgfsetstrokecolor{textcolor}%
\pgfsetfillcolor{textcolor}%
\pgftext[x=4.258735in,y=1.407600in,left,base]{\color{textcolor}\rmfamily\fontsize{10.000000}{12.000000}\selectfont UI-TVBO}%
\end{pgfscope}%
\begin{pgfscope}%
\pgfsetbuttcap%
\pgfsetmiterjoin%
\definecolor{currentfill}{rgb}{0.000000,0.380392,0.396078}%
\pgfsetfillcolor{currentfill}%
\pgfsetlinewidth{0.000000pt}%
\definecolor{currentstroke}{rgb}{0.000000,0.000000,0.000000}%
\pgfsetstrokecolor{currentstroke}%
\pgfsetstrokeopacity{0.000000}%
\pgfsetdash{}{0pt}%
\pgfpathmoveto{\pgfqpoint{3.869846in}{1.213989in}}%
\pgfpathlineto{\pgfqpoint{4.147623in}{1.213989in}}%
\pgfpathlineto{\pgfqpoint{4.147623in}{1.311212in}}%
\pgfpathlineto{\pgfqpoint{3.869846in}{1.311212in}}%
\pgfpathclose%
\pgfusepath{fill}%
\end{pgfscope}%
\begin{pgfscope}%
\definecolor{textcolor}{rgb}{0.000000,0.000000,0.000000}%
\pgfsetstrokecolor{textcolor}%
\pgfsetfillcolor{textcolor}%
\pgftext[x=4.258735in,y=1.213989in,left,base]{\color{textcolor}\rmfamily\fontsize{10.000000}{12.000000}\selectfont B UI-TVBO}%
\end{pgfscope}%
\begin{pgfscope}%
\pgfsetbuttcap%
\pgfsetmiterjoin%
\definecolor{currentfill}{rgb}{0.380392,0.129412,0.345098}%
\pgfsetfillcolor{currentfill}%
\pgfsetlinewidth{0.000000pt}%
\definecolor{currentstroke}{rgb}{0.000000,0.000000,0.000000}%
\pgfsetstrokecolor{currentstroke}%
\pgfsetstrokeopacity{0.000000}%
\pgfsetdash{}{0pt}%
\pgfpathmoveto{\pgfqpoint{3.869846in}{1.020379in}}%
\pgfpathlineto{\pgfqpoint{4.147623in}{1.020379in}}%
\pgfpathlineto{\pgfqpoint{4.147623in}{1.117601in}}%
\pgfpathlineto{\pgfqpoint{3.869846in}{1.117601in}}%
\pgfpathclose%
\pgfusepath{fill}%
\end{pgfscope}%
\begin{pgfscope}%
\definecolor{textcolor}{rgb}{0.000000,0.000000,0.000000}%
\pgfsetstrokecolor{textcolor}%
\pgfsetfillcolor{textcolor}%
\pgftext[x=4.258735in,y=1.020379in,left,base]{\color{textcolor}\rmfamily\fontsize{10.000000}{12.000000}\selectfont C-TV-GP-UCB}%
\end{pgfscope}%
\begin{pgfscope}%
\pgfsetbuttcap%
\pgfsetmiterjoin%
\definecolor{currentfill}{rgb}{0.964706,0.658824,0.000000}%
\pgfsetfillcolor{currentfill}%
\pgfsetlinewidth{0.000000pt}%
\definecolor{currentstroke}{rgb}{0.000000,0.000000,0.000000}%
\pgfsetstrokecolor{currentstroke}%
\pgfsetstrokeopacity{0.000000}%
\pgfsetdash{}{0pt}%
\pgfpathmoveto{\pgfqpoint{3.869846in}{0.826768in}}%
\pgfpathlineto{\pgfqpoint{4.147623in}{0.826768in}}%
\pgfpathlineto{\pgfqpoint{4.147623in}{0.923990in}}%
\pgfpathlineto{\pgfqpoint{3.869846in}{0.923990in}}%
\pgfpathclose%
\pgfusepath{fill}%
\end{pgfscope}%
\begin{pgfscope}%
\definecolor{textcolor}{rgb}{0.000000,0.000000,0.000000}%
\pgfsetstrokecolor{textcolor}%
\pgfsetfillcolor{textcolor}%
\pgftext[x=4.258735in,y=0.826768in,left,base]{\color{textcolor}\rmfamily\fontsize{10.000000}{12.000000}\selectfont SW C-TV-GP-UCB}%
\end{pgfscope}%
\begin{pgfscope}%
\pgfsetbuttcap%
\pgfsetmiterjoin%
\definecolor{currentfill}{rgb}{0.341176,0.670588,0.152941}%
\pgfsetfillcolor{currentfill}%
\pgfsetlinewidth{0.000000pt}%
\definecolor{currentstroke}{rgb}{0.000000,0.000000,0.000000}%
\pgfsetstrokecolor{currentstroke}%
\pgfsetstrokeopacity{0.000000}%
\pgfsetdash{}{0pt}%
\pgfpathmoveto{\pgfqpoint{3.869846in}{0.633157in}}%
\pgfpathlineto{\pgfqpoint{4.147623in}{0.633157in}}%
\pgfpathlineto{\pgfqpoint{4.147623in}{0.730379in}}%
\pgfpathlineto{\pgfqpoint{3.869846in}{0.730379in}}%
\pgfpathclose%
\pgfusepath{fill}%
\end{pgfscope}%
\begin{pgfscope}%
\definecolor{textcolor}{rgb}{0.000000,0.000000,0.000000}%
\pgfsetstrokecolor{textcolor}%
\pgfsetfillcolor{textcolor}%
\pgftext[x=4.258735in,y=0.633157in,left,base]{\color{textcolor}\rmfamily\fontsize{10.000000}{12.000000}\selectfont C-UI-TVBO}%
\end{pgfscope}%
\begin{pgfscope}%
\pgfsetbuttcap%
\pgfsetmiterjoin%
\definecolor{currentfill}{rgb}{0.478431,0.435294,0.674510}%
\pgfsetfillcolor{currentfill}%
\pgfsetlinewidth{0.000000pt}%
\definecolor{currentstroke}{rgb}{0.000000,0.000000,0.000000}%
\pgfsetstrokecolor{currentstroke}%
\pgfsetstrokeopacity{0.000000}%
\pgfsetdash{}{0pt}%
\pgfpathmoveto{\pgfqpoint{3.869846in}{0.439546in}}%
\pgfpathlineto{\pgfqpoint{4.147623in}{0.439546in}}%
\pgfpathlineto{\pgfqpoint{4.147623in}{0.536768in}}%
\pgfpathlineto{\pgfqpoint{3.869846in}{0.536768in}}%
\pgfpathclose%
\pgfusepath{fill}%
\end{pgfscope}%
\begin{pgfscope}%
\definecolor{textcolor}{rgb}{0.000000,0.000000,0.000000}%
\pgfsetstrokecolor{textcolor}%
\pgfsetfillcolor{textcolor}%
\pgftext[x=4.258735in,y=0.439546in,left,base]{\color{textcolor}\rmfamily\fontsize{10.000000}{12.000000}\selectfont B C-UI-TVBO}%
\end{pgfscope}%
\begin{pgfscope}%
\definecolor{textcolor}{rgb}{0.000000,0.000000,0.000000}%
\pgfsetstrokecolor{textcolor}%
\pgfsetfillcolor{textcolor}%
\pgftext[x=0.178996in, y=1.179871in, left, base,rotate=90.000000]{\color{textcolor}\rmfamily\fontsize{10.000000}{12.000000}\selectfont \(\displaystyle R_T\)}%
\end{pgfscope}%
\end{pgfpicture}%
\makeatother%
\endgroup%

    \caption[Results of the two-dimensional moving parabola.]{Results of the two-dimensional moving parabola. The white circles represent the mean of each variation. The darker shades show the performance with a well-defined mean, while the lighter shades show the performance with an optimistic mean.}
    \label{fig:Parabola2D_cumulative_regret}
\end{figure}

It shows the same trends as in the one-dimensional moving parabola: \gls{b2p} shows a higher sensitivity to the shifted mean (Hypothesis~\ref{hyp:ui_structural_information}), \gls{b2p} and \gls{ui} forgetting have similar performance in terms of regret for with a well-defined prior mean (Hypothesis~\ref{hyp:ui_good_mean}), and \gls{ctvbo} reduces the regret as well as its variance compared to standard \gls{tvbo} (Hypothesis~\ref{hyp:ctvbo}). The best performing variation in terms of cumulative regret is again the combination of the proposed methods \gls{uitvbo} and \gls{ctvbo}.

However, it is interesting to note that similar to the two-dimensional simulations of within and out-of-model comparison, the differences in regret between standard \gls{tvbo} and \gls{ctvbo} are smaller than in the one-dimensional examples. One reason for this could be the smaller number of \glspl{vop} per dimension $N_{v/D}$ chosen for computational feasibility.

\section{LQR Problem of an Inverted Pendulum}
\label{sec:LQR}

As the last experiment, the variations in Table \ref{tab:models} are applied to a real-world problem -- the \gls{lqr} problem of an inverted pendulum. The \gls{lqr} problem is a classic problem in fundamental control theory controlling a linear dynamical system by minimizing a quadratic cost function $J$. It therefore satisfies Assumption \ref{ass:prior_knowledge_convex} with the objective function as $f_t \coloneqq J$ making it a suitable application to benchmark the proposed methods. In the following, $t$ denotes the time step of the \gls{tvbo} algorithm, whereas $\hat{t}$ denotes the time steps of the linear system. For an infinite horizon and discrete time steps the \gls{lqr} problem is formalized as the optimization problem
\begin{align}
    \min_{\mathbf{u}_{\hat{t}}(\cdot)}\, &J=\lim_{\hat{T} \to \infty} \EX \left[ \sum_{{\hat{t}}=0}^{\hat{T}-1} \mathbf{x}_{\hat{t}}^T \mathbf{Q}\mathbf{x}_{\hat{t}} +  \mathbf{u}_{\hat{t}}^T \mathbf{R}\mathbf{u}_{\hat{t}} \right] \label{eq:lqr_cost}\\
    &\text{s.t. }\mathbf{x}_{\hat{t}+1} = \mathbf{A}_k\mathbf{x}_{\hat{t}} + \mathbf{B}_k\mathbf{u}_{\hat{t}} + \mathbf{w}_{\hat{t}} \label{eq:lqr_system}
\end{align}
with $\mathbf{x}_{\hat{t}} \in \R^D$ as the state, $\mathbf{u}_{\hat{t}} \in \R^P$ as the input, and $\mathbf{w}_{\hat{t}} \sim \mathcal{N}(\mathbf{0}, \bar{\sigma}_n^2 \mathbf{I})$ as iid Gaussian noise at each time step $\hat{t}$. $\mathbf{Q}$ and $\mathbf{R}$ in \eqref{eq:lqr_cost} are positive-definite weighting matrices. Furthermore, \eqref{eq:lqr_system} describes a time-discrete linear model with $\mathbf{A}_k$ as the state matrix and $\mathbf{B}_k$ as the input matrix. If the linear model is known, the optimal feedback controller as
\begin{equation}
    \mathbf{u}_{\hat{t}} = \mathbf{K}^* \, \mathbf{x}_{\hat{t}}
\end{equation}
with the optimal controller gain $\mathbf{K}^* \in \R^{P \times D}$ can be calculated by solving the discrete algebraic Ricatti equation
\begin{equation}
    \mathbf{P} = \mathbf{A}_k^T \mathbf{P} \mathbf{A}- \mathbf{A}_k^T \mathbf{P} \mathbf{B}_k \left(\mathbf{R}+\mathbf{B}_k^T \mathbf{P} \mathbf{B}_k\right)^{-1} \mathbf{B}_k^T \mathbf{P} \mathbf{A}_k+\mathbf{Q}
\end{equation}
and setting
\begin{equation}
    \mathbf{K}^* = -\left(\mathbf{R}+\mathbf{B}_k^T\mathbf{P}\mathbf{B}_k\right)^{-1} \mathbf{B}_k^T\mathbf{P}\mathbf{A}_k.
    \label{eq:optimal_controller}
\end{equation}
As mentioned, the system at consideration is the inverted pendulum. A pendulum is attached to a horizontally moving cart as shown in Figure~\ref{fig:inverted_pendulum}.
\begin{figure}[h]
   \centering
   \hspace{1cm}
   \import{thesis/figures/pdf_figures/}{Zeichnung.pdf_tex}
 \caption[Inverted pendulum.]{Inverted pendulum with parameters and state variables.}
 \label{fig:inverted_pendulum}
\end{figure}

\newpage
The goal of the \gls{lqr} problem is to find the optimal controller stabilizing the unstable upper equilibrium point
\begin{equation}
    x = 0,\, \dot{x} = 0,\, \varphi = 0,\, \dot{\varphi} = 0
    \label{eq:equi}
\end{equation}
and minimizing the costs in \eqref{eq:lqr_cost}.

Since the pendulum introduces trigonometric functions into the force balance equations, the inverted pendulum is non-linear. The non-linear system equation for the angular acceleration $\ddot{\varphi}$ is
\begin{equation}
    \ddot{\varphi}=\frac{1}{2}\frac{m_p g l}{J_d} \cdot \sin(\varphi)-\frac{1}{2}\frac{m_p l}{J_d}\cdot \cos(\varphi)\cdot\ddot{x}-\frac{\mu_p}{J_d}\cdot\dot{\varphi}
    \label{eq:nicht_eingesetzt}
\end{equation}
with $m_p$ as the mass, $J_d$ as the moment of inertia, and $l$ as the length of the pendulum. Furthermore, $\mu_p$ denotes the fiction in the bearing as shown in Figure~\ref{fig:inverted_pendulum}.
The control variable is the velocity of the cart as $u \coloneqq \dot{x}_{sp}$ modeled as a first-order lag transfer function resulting in the differential equation for the acceleration as
\begin{equation}
    \ddot{x}= \frac{1}{T_1}(K_u\cdot \dot{x}_{sp}-\dot{x})=\frac{1}{T_1}(K_u \cdot u-\dot{x}).
    \label{eq:pt1}
\end{equation}
Substituting \eqref{eq:pt1} into \eqref{eq:nicht_eingesetzt} yields
\begin{equation}
    \ddot{\varphi}=\frac{1}{2}\frac{m_p g l}{J_d} \cdot \sin(\varphi)-\frac{1}{2}\frac{m_p l}{J_d} \cdot \cos(\varphi) \cdot \frac{1}{T_1}(K_u\cdot u-\dot{x})-\frac{\mu_p}{J_d}\cdot\dot{\varphi}.
    \label{eq:phiacc}
\end{equation}
The equations \eqref{eq:pt1} and \eqref{eq:phiacc} build the non-linear state space of the system. As the \gls{lqr} problem requires a linear model, the non-linear system is linearized around the upper equilibrium points in \eqref{eq:equi} yielding in a linear continuous state space model as
\begin{equation}
    \dot{\mathbf{x}} = \mathbf{A} \mathbf{x} + \mathbf{B} u
    \label{eq:cont_state_space}
\end{equation}
with $\mathbf{x} = [x, \dot{x}, \varphi, \dot{\varphi}]^T$ as the state vector. The state matrix and input matrix are 
\begin{equation}
    \mathbf{A} = \left[\begin{array}{cccc}
         0 & 1 & 0 & 0 \\
         0 & -\frac{1}{T_1} & 0 & 0\\
         0 & 0 & 0 & 1\\
         0 & \frac{1}{2}\frac{m_p l}{J_d T_1} &
     \frac{1}{2} \frac{m_p l g}{J_d} & -\frac{\mu_p}{J_d}
    \end{array} \right], \quad \mathbf{B} = \left[\begin{array}{c}
    0 \\
    \frac{K_u}{T_1} \\
    0 \\
    -\frac{1}{2}\frac{m_p l}{J_d}\frac{K_u}{T_1}
    \end{array} \right].
\end{equation}
The continuous state space model in \eqref{eq:cont_state_space} is discretized with zero-order hold and a sampling interval of $T_s=0.02\si{\second}$ yielding in a time-invariant discrete state space model as in \eqref{eq:lqr_system}. 
The parameter of the system are shown in Table~\ref{tab:params_lqr}.
\bgroup
\def\arraystretch{1.2}
\begin{table}[h]
    \centering
    \begin{tabular}{c||c c c c c c }
        \textbf{Param.} & $m_p$ &$J_d$&$l$&$\mu_p \,|\, \mu_{p,0}$&$K_u$ & $T_1$ \\\hline\hline
        \textbf{Value} & $0.0804\si{\kilogram}$ &$0,5813\cdot 10^{-3}\si{\kilogram\meter}^2$&$0.147\si{\meter}$&$2.2\cdot10^{-3}\si{\newton\meter\second}$&$1$ & $1\si{\second}$
    \end{tabular}
    \caption{Parameters of the inverted pendulum.}
    \label{tab:params_lqr}
\end{table}
\egroup

The algorithms considered in this thesis are designed for time-varying objective functions. Therefore, the friction in the bearing $\mu_p$ is assumed to be time-varying as
\begin{equation}
    \mu_p(t) = \begin{cases}
        \mu_{p,0} \, ,& t < 50 \\
        \mu_{p,0}+\mu_{p,0} \cdot \left(-1.5 \cdot\cos\left(\frac{\pi}{50}(t-50)\right) + 1.5 \right)  \, ,& 50 \leq t \leq 100 \\
        3 \mu_{p,0} + \frac{1}{2}\mu_{p,0} \cdot \sin\left(-\frac{\pi}{100}t\right) \, ,& t > 100 \\
        \end{cases}
\end{equation}
yielding a time-varying \gls{lqr} cost function $J_t$ with $\mathbf{K}_t^*$ as the optimal controller and $J_t^*$ as the optimal cost at time step $t$. Figure~\ref{fig:LQR_cost} shows the optimal cost over time normalized by the initial optimal cost $J_0^*$ with a standard deviation of the noise in the system as $\bar{\sigma}_n = 6\cdot10^{-4}$, a time horizon for the \gls{lqr} cost function of $\hat{T} = 20$, and the initial conditions as
\begin{equation}
    x_0 = 4\si{\meter},\, \dot{x}_0= 0\frac{\si{\meter}}{\si{\second}} ,\, \varphi_0 = 0.1 \si{\radian},\, \dot{\varphi}_0 = 0.1\frac{\si{\radian}}{\si{\second}}.
    \label{eq:inital_conditions}
\end{equation}
Furthermore, the dashed line in Figure~\ref{fig:LQR_cost} shows the cost over time if the controller gain is kept constant over time as $\mathbf{K}_0^*$, not adjusting to the changing system dynamics. Although the system remains stable, the costs are significantly higher compared to the optimal cost trajectory.
\begin{figure}[h]
    \centering
    %% Creator: Matplotlib, PGF backend
%%
%% To include the figure in your LaTeX document, write
%%   \input{<filename>.pgf}
%%
%% Make sure the required packages are loaded in your preamble
%%   \usepackage{pgf}
%%
%% Figures using additional raster images can only be included by \input if
%% they are in the same directory as the main LaTeX file. For loading figures
%% from other directories you can use the `import` package
%%   \usepackage{import}
%%
%% and then include the figures with
%%   \import{<path to file>}{<filename>.pgf}
%%
%% Matplotlib used the following preamble
%%   \usepackage[utf8x]{inputenc}\usepackage[light]{kpfonts}\usepackage{amsfonts}
%%   \usepackage{fontspec}
%%
\begingroup%
\makeatletter%
\begin{pgfpicture}%
\pgfpathrectangle{\pgfpointorigin}{\pgfqpoint{4.130345in}{2.042155in}}%
\pgfusepath{use as bounding box, clip}%
\begin{pgfscope}%
\pgfsetbuttcap%
\pgfsetmiterjoin%
\definecolor{currentfill}{rgb}{1.000000,1.000000,1.000000}%
\pgfsetfillcolor{currentfill}%
\pgfsetlinewidth{0.000000pt}%
\definecolor{currentstroke}{rgb}{1.000000,1.000000,1.000000}%
\pgfsetstrokecolor{currentstroke}%
\pgfsetdash{}{0pt}%
\pgfpathmoveto{\pgfqpoint{0.000000in}{0.000000in}}%
\pgfpathlineto{\pgfqpoint{4.130345in}{0.000000in}}%
\pgfpathlineto{\pgfqpoint{4.130345in}{2.042155in}}%
\pgfpathlineto{\pgfqpoint{0.000000in}{2.042155in}}%
\pgfpathclose%
\pgfusepath{fill}%
\end{pgfscope}%
\begin{pgfscope}%
\pgfsetbuttcap%
\pgfsetmiterjoin%
\definecolor{currentfill}{rgb}{1.000000,1.000000,1.000000}%
\pgfsetfillcolor{currentfill}%
\pgfsetlinewidth{0.000000pt}%
\definecolor{currentstroke}{rgb}{0.000000,0.000000,0.000000}%
\pgfsetstrokecolor{currentstroke}%
\pgfsetstrokeopacity{0.000000}%
\pgfsetdash{}{0pt}%
\pgfpathmoveto{\pgfqpoint{0.743462in}{0.408431in}}%
\pgfpathlineto{\pgfqpoint{4.006434in}{0.408431in}}%
\pgfpathlineto{\pgfqpoint{4.006434in}{1.940047in}}%
\pgfpathlineto{\pgfqpoint{0.743462in}{1.940047in}}%
\pgfpathclose%
\pgfusepath{fill}%
\end{pgfscope}%
\begin{pgfscope}%
\pgfsetbuttcap%
\pgfsetroundjoin%
\definecolor{currentfill}{rgb}{0.000000,0.000000,0.000000}%
\pgfsetfillcolor{currentfill}%
\pgfsetlinewidth{0.803000pt}%
\definecolor{currentstroke}{rgb}{0.000000,0.000000,0.000000}%
\pgfsetstrokecolor{currentstroke}%
\pgfsetdash{}{0pt}%
\pgfsys@defobject{currentmarker}{\pgfqpoint{0.000000in}{-0.048611in}}{\pgfqpoint{0.000000in}{0.000000in}}{%
\pgfpathmoveto{\pgfqpoint{0.000000in}{0.000000in}}%
\pgfpathlineto{\pgfqpoint{0.000000in}{-0.048611in}}%
\pgfusepath{stroke,fill}%
}%
\begin{pgfscope}%
\pgfsys@transformshift{0.985164in}{0.408431in}%
\pgfsys@useobject{currentmarker}{}%
\end{pgfscope}%
\end{pgfscope}%
\begin{pgfscope}%
\definecolor{textcolor}{rgb}{0.000000,0.000000,0.000000}%
\pgfsetstrokecolor{textcolor}%
\pgfsetfillcolor{textcolor}%
\pgftext[x=0.985164in,y=0.311209in,,top]{\color{textcolor}\rmfamily\fontsize{10.000000}{12.000000}\selectfont \(\displaystyle {50}\)}%
\end{pgfscope}%
\begin{pgfscope}%
\pgfsetbuttcap%
\pgfsetroundjoin%
\definecolor{currentfill}{rgb}{0.000000,0.000000,0.000000}%
\pgfsetfillcolor{currentfill}%
\pgfsetlinewidth{0.803000pt}%
\definecolor{currentstroke}{rgb}{0.000000,0.000000,0.000000}%
\pgfsetstrokecolor{currentstroke}%
\pgfsetdash{}{0pt}%
\pgfsys@defobject{currentmarker}{\pgfqpoint{0.000000in}{-0.048611in}}{\pgfqpoint{0.000000in}{0.000000in}}{%
\pgfpathmoveto{\pgfqpoint{0.000000in}{0.000000in}}%
\pgfpathlineto{\pgfqpoint{0.000000in}{-0.048611in}}%
\pgfusepath{stroke,fill}%
}%
\begin{pgfscope}%
\pgfsys@transformshift{1.589418in}{0.408431in}%
\pgfsys@useobject{currentmarker}{}%
\end{pgfscope}%
\end{pgfscope}%
\begin{pgfscope}%
\definecolor{textcolor}{rgb}{0.000000,0.000000,0.000000}%
\pgfsetstrokecolor{textcolor}%
\pgfsetfillcolor{textcolor}%
\pgftext[x=1.589418in,y=0.311209in,,top]{\color{textcolor}\rmfamily\fontsize{10.000000}{12.000000}\selectfont \(\displaystyle {100}\)}%
\end{pgfscope}%
\begin{pgfscope}%
\pgfsetbuttcap%
\pgfsetroundjoin%
\definecolor{currentfill}{rgb}{0.000000,0.000000,0.000000}%
\pgfsetfillcolor{currentfill}%
\pgfsetlinewidth{0.803000pt}%
\definecolor{currentstroke}{rgb}{0.000000,0.000000,0.000000}%
\pgfsetstrokecolor{currentstroke}%
\pgfsetdash{}{0pt}%
\pgfsys@defobject{currentmarker}{\pgfqpoint{0.000000in}{-0.048611in}}{\pgfqpoint{0.000000in}{0.000000in}}{%
\pgfpathmoveto{\pgfqpoint{0.000000in}{0.000000in}}%
\pgfpathlineto{\pgfqpoint{0.000000in}{-0.048611in}}%
\pgfusepath{stroke,fill}%
}%
\begin{pgfscope}%
\pgfsys@transformshift{2.193672in}{0.408431in}%
\pgfsys@useobject{currentmarker}{}%
\end{pgfscope}%
\end{pgfscope}%
\begin{pgfscope}%
\definecolor{textcolor}{rgb}{0.000000,0.000000,0.000000}%
\pgfsetstrokecolor{textcolor}%
\pgfsetfillcolor{textcolor}%
\pgftext[x=2.193672in,y=0.311209in,,top]{\color{textcolor}\rmfamily\fontsize{10.000000}{12.000000}\selectfont \(\displaystyle {150}\)}%
\end{pgfscope}%
\begin{pgfscope}%
\pgfsetbuttcap%
\pgfsetroundjoin%
\definecolor{currentfill}{rgb}{0.000000,0.000000,0.000000}%
\pgfsetfillcolor{currentfill}%
\pgfsetlinewidth{0.803000pt}%
\definecolor{currentstroke}{rgb}{0.000000,0.000000,0.000000}%
\pgfsetstrokecolor{currentstroke}%
\pgfsetdash{}{0pt}%
\pgfsys@defobject{currentmarker}{\pgfqpoint{0.000000in}{-0.048611in}}{\pgfqpoint{0.000000in}{0.000000in}}{%
\pgfpathmoveto{\pgfqpoint{0.000000in}{0.000000in}}%
\pgfpathlineto{\pgfqpoint{0.000000in}{-0.048611in}}%
\pgfusepath{stroke,fill}%
}%
\begin{pgfscope}%
\pgfsys@transformshift{2.797926in}{0.408431in}%
\pgfsys@useobject{currentmarker}{}%
\end{pgfscope}%
\end{pgfscope}%
\begin{pgfscope}%
\definecolor{textcolor}{rgb}{0.000000,0.000000,0.000000}%
\pgfsetstrokecolor{textcolor}%
\pgfsetfillcolor{textcolor}%
\pgftext[x=2.797926in,y=0.311209in,,top]{\color{textcolor}\rmfamily\fontsize{10.000000}{12.000000}\selectfont \(\displaystyle {200}\)}%
\end{pgfscope}%
\begin{pgfscope}%
\pgfsetbuttcap%
\pgfsetroundjoin%
\definecolor{currentfill}{rgb}{0.000000,0.000000,0.000000}%
\pgfsetfillcolor{currentfill}%
\pgfsetlinewidth{0.803000pt}%
\definecolor{currentstroke}{rgb}{0.000000,0.000000,0.000000}%
\pgfsetstrokecolor{currentstroke}%
\pgfsetdash{}{0pt}%
\pgfsys@defobject{currentmarker}{\pgfqpoint{0.000000in}{-0.048611in}}{\pgfqpoint{0.000000in}{0.000000in}}{%
\pgfpathmoveto{\pgfqpoint{0.000000in}{0.000000in}}%
\pgfpathlineto{\pgfqpoint{0.000000in}{-0.048611in}}%
\pgfusepath{stroke,fill}%
}%
\begin{pgfscope}%
\pgfsys@transformshift{3.402180in}{0.408431in}%
\pgfsys@useobject{currentmarker}{}%
\end{pgfscope}%
\end{pgfscope}%
\begin{pgfscope}%
\definecolor{textcolor}{rgb}{0.000000,0.000000,0.000000}%
\pgfsetstrokecolor{textcolor}%
\pgfsetfillcolor{textcolor}%
\pgftext[x=3.402180in,y=0.311209in,,top]{\color{textcolor}\rmfamily\fontsize{10.000000}{12.000000}\selectfont \(\displaystyle {250}\)}%
\end{pgfscope}%
\begin{pgfscope}%
\pgfsetbuttcap%
\pgfsetroundjoin%
\definecolor{currentfill}{rgb}{0.000000,0.000000,0.000000}%
\pgfsetfillcolor{currentfill}%
\pgfsetlinewidth{0.803000pt}%
\definecolor{currentstroke}{rgb}{0.000000,0.000000,0.000000}%
\pgfsetstrokecolor{currentstroke}%
\pgfsetdash{}{0pt}%
\pgfsys@defobject{currentmarker}{\pgfqpoint{0.000000in}{-0.048611in}}{\pgfqpoint{0.000000in}{0.000000in}}{%
\pgfpathmoveto{\pgfqpoint{0.000000in}{0.000000in}}%
\pgfpathlineto{\pgfqpoint{0.000000in}{-0.048611in}}%
\pgfusepath{stroke,fill}%
}%
\begin{pgfscope}%
\pgfsys@transformshift{4.006434in}{0.408431in}%
\pgfsys@useobject{currentmarker}{}%
\end{pgfscope}%
\end{pgfscope}%
\begin{pgfscope}%
\definecolor{textcolor}{rgb}{0.000000,0.000000,0.000000}%
\pgfsetstrokecolor{textcolor}%
\pgfsetfillcolor{textcolor}%
\pgftext[x=4.006434in,y=0.311209in,,top]{\color{textcolor}\rmfamily\fontsize{10.000000}{12.000000}\selectfont \(\displaystyle {300}\)}%
\end{pgfscope}%
\begin{pgfscope}%
\definecolor{textcolor}{rgb}{0.000000,0.000000,0.000000}%
\pgfsetstrokecolor{textcolor}%
\pgfsetfillcolor{textcolor}%
\pgftext[x=2.374948in,y=0.132320in,,top]{\color{textcolor}\rmfamily\fontsize{10.000000}{12.000000}\selectfont \(\displaystyle t\)}%
\end{pgfscope}%
\begin{pgfscope}%
\pgfsetbuttcap%
\pgfsetroundjoin%
\definecolor{currentfill}{rgb}{0.000000,0.000000,0.000000}%
\pgfsetfillcolor{currentfill}%
\pgfsetlinewidth{0.803000pt}%
\definecolor{currentstroke}{rgb}{0.000000,0.000000,0.000000}%
\pgfsetstrokecolor{currentstroke}%
\pgfsetdash{}{0pt}%
\pgfsys@defobject{currentmarker}{\pgfqpoint{-0.048611in}{0.000000in}}{\pgfqpoint{-0.000000in}{0.000000in}}{%
\pgfpathmoveto{\pgfqpoint{-0.000000in}{0.000000in}}%
\pgfpathlineto{\pgfqpoint{-0.048611in}{0.000000in}}%
\pgfusepath{stroke,fill}%
}%
\begin{pgfscope}%
\pgfsys@transformshift{0.743462in}{0.492760in}%
\pgfsys@useobject{currentmarker}{}%
\end{pgfscope}%
\end{pgfscope}%
\begin{pgfscope}%
\definecolor{textcolor}{rgb}{0.000000,0.000000,0.000000}%
\pgfsetstrokecolor{textcolor}%
\pgfsetfillcolor{textcolor}%
\pgftext[x=0.460823in, y=0.444565in, left, base]{\color{textcolor}\rmfamily\fontsize{10.000000}{12.000000}\selectfont \(\displaystyle {1.0}\)}%
\end{pgfscope}%
\begin{pgfscope}%
\pgfsetbuttcap%
\pgfsetroundjoin%
\definecolor{currentfill}{rgb}{0.000000,0.000000,0.000000}%
\pgfsetfillcolor{currentfill}%
\pgfsetlinewidth{0.803000pt}%
\definecolor{currentstroke}{rgb}{0.000000,0.000000,0.000000}%
\pgfsetstrokecolor{currentstroke}%
\pgfsetdash{}{0pt}%
\pgfsys@defobject{currentmarker}{\pgfqpoint{-0.048611in}{0.000000in}}{\pgfqpoint{-0.000000in}{0.000000in}}{%
\pgfpathmoveto{\pgfqpoint{-0.000000in}{0.000000in}}%
\pgfpathlineto{\pgfqpoint{-0.048611in}{0.000000in}}%
\pgfusepath{stroke,fill}%
}%
\begin{pgfscope}%
\pgfsys@transformshift{0.743462in}{1.068134in}%
\pgfsys@useobject{currentmarker}{}%
\end{pgfscope}%
\end{pgfscope}%
\begin{pgfscope}%
\definecolor{textcolor}{rgb}{0.000000,0.000000,0.000000}%
\pgfsetstrokecolor{textcolor}%
\pgfsetfillcolor{textcolor}%
\pgftext[x=0.460823in, y=1.019939in, left, base]{\color{textcolor}\rmfamily\fontsize{10.000000}{12.000000}\selectfont \(\displaystyle {1.1}\)}%
\end{pgfscope}%
\begin{pgfscope}%
\pgfsetbuttcap%
\pgfsetroundjoin%
\definecolor{currentfill}{rgb}{0.000000,0.000000,0.000000}%
\pgfsetfillcolor{currentfill}%
\pgfsetlinewidth{0.803000pt}%
\definecolor{currentstroke}{rgb}{0.000000,0.000000,0.000000}%
\pgfsetstrokecolor{currentstroke}%
\pgfsetdash{}{0pt}%
\pgfsys@defobject{currentmarker}{\pgfqpoint{-0.048611in}{0.000000in}}{\pgfqpoint{-0.000000in}{0.000000in}}{%
\pgfpathmoveto{\pgfqpoint{-0.000000in}{0.000000in}}%
\pgfpathlineto{\pgfqpoint{-0.048611in}{0.000000in}}%
\pgfusepath{stroke,fill}%
}%
\begin{pgfscope}%
\pgfsys@transformshift{0.743462in}{1.643507in}%
\pgfsys@useobject{currentmarker}{}%
\end{pgfscope}%
\end{pgfscope}%
\begin{pgfscope}%
\definecolor{textcolor}{rgb}{0.000000,0.000000,0.000000}%
\pgfsetstrokecolor{textcolor}%
\pgfsetfillcolor{textcolor}%
\pgftext[x=0.460823in, y=1.595313in, left, base]{\color{textcolor}\rmfamily\fontsize{10.000000}{12.000000}\selectfont \(\displaystyle {1.2}\)}%
\end{pgfscope}%
\begin{pgfscope}%
\definecolor{textcolor}{rgb}{0.000000,0.000000,0.000000}%
\pgfsetstrokecolor{textcolor}%
\pgfsetfillcolor{textcolor}%
\pgftext[x=0.405268in,y=1.174239in,,bottom,rotate=90.000000]{\color{textcolor}\rmfamily\fontsize{10.000000}{12.000000}\selectfont \(\displaystyle \frac{J_t}{J_0^*}\)}%
\end{pgfscope}%
\begin{pgfscope}%
\pgfpathrectangle{\pgfqpoint{0.743462in}{0.408431in}}{\pgfqpoint{3.262972in}{1.531616in}}%
\pgfusepath{clip}%
\pgfsetrectcap%
\pgfsetroundjoin%
\pgfsetlinewidth{0.752812pt}%
\definecolor{currentstroke}{rgb}{0.000000,0.000000,0.000000}%
\pgfsetstrokecolor{currentstroke}%
\pgfsetdash{}{0pt}%
\pgfpathmoveto{\pgfqpoint{0.755547in}{0.492760in}}%
\pgfpathlineto{\pgfqpoint{0.767632in}{0.521746in}}%
\pgfpathlineto{\pgfqpoint{0.779717in}{0.499438in}}%
\pgfpathlineto{\pgfqpoint{0.791802in}{0.501805in}}%
\pgfpathlineto{\pgfqpoint{0.803887in}{0.505991in}}%
\pgfpathlineto{\pgfqpoint{0.815973in}{0.478050in}}%
\pgfpathlineto{\pgfqpoint{0.828058in}{0.512478in}}%
\pgfpathlineto{\pgfqpoint{0.840143in}{0.524523in}}%
\pgfpathlineto{\pgfqpoint{0.852228in}{0.507465in}}%
\pgfpathlineto{\pgfqpoint{0.876398in}{0.490397in}}%
\pgfpathlineto{\pgfqpoint{0.888483in}{0.494595in}}%
\pgfpathlineto{\pgfqpoint{0.900568in}{0.512448in}}%
\pgfpathlineto{\pgfqpoint{0.912653in}{0.514950in}}%
\pgfpathlineto{\pgfqpoint{0.924738in}{0.492078in}}%
\pgfpathlineto{\pgfqpoint{0.936823in}{0.498768in}}%
\pgfpathlineto{\pgfqpoint{0.948908in}{0.491276in}}%
\pgfpathlineto{\pgfqpoint{0.960993in}{0.501223in}}%
\pgfpathlineto{\pgfqpoint{0.973079in}{0.519111in}}%
\pgfpathlineto{\pgfqpoint{0.985164in}{0.502411in}}%
\pgfpathlineto{\pgfqpoint{0.997249in}{0.496208in}}%
\pgfpathlineto{\pgfqpoint{1.009334in}{0.520352in}}%
\pgfpathlineto{\pgfqpoint{1.021419in}{0.516019in}}%
\pgfpathlineto{\pgfqpoint{1.033504in}{0.505962in}}%
\pgfpathlineto{\pgfqpoint{1.045589in}{0.506246in}}%
\pgfpathlineto{\pgfqpoint{1.057674in}{0.529827in}}%
\pgfpathlineto{\pgfqpoint{1.069759in}{0.529201in}}%
\pgfpathlineto{\pgfqpoint{1.081844in}{0.519570in}}%
\pgfpathlineto{\pgfqpoint{1.093929in}{0.558152in}}%
\pgfpathlineto{\pgfqpoint{1.118100in}{0.557981in}}%
\pgfpathlineto{\pgfqpoint{1.130185in}{0.568765in}}%
\pgfpathlineto{\pgfqpoint{1.142270in}{0.617512in}}%
\pgfpathlineto{\pgfqpoint{1.154355in}{0.590761in}}%
\pgfpathlineto{\pgfqpoint{1.178525in}{0.651647in}}%
\pgfpathlineto{\pgfqpoint{1.190610in}{0.663889in}}%
\pgfpathlineto{\pgfqpoint{1.202695in}{0.704729in}}%
\pgfpathlineto{\pgfqpoint{1.214780in}{0.704719in}}%
\pgfpathlineto{\pgfqpoint{1.226865in}{0.734965in}}%
\pgfpathlineto{\pgfqpoint{1.238950in}{0.757283in}}%
\pgfpathlineto{\pgfqpoint{1.251035in}{0.749056in}}%
\pgfpathlineto{\pgfqpoint{1.263121in}{0.805162in}}%
\pgfpathlineto{\pgfqpoint{1.275206in}{0.832081in}}%
\pgfpathlineto{\pgfqpoint{1.299376in}{0.851460in}}%
\pgfpathlineto{\pgfqpoint{1.311461in}{0.869786in}}%
\pgfpathlineto{\pgfqpoint{1.323546in}{0.906856in}}%
\pgfpathlineto{\pgfqpoint{1.335631in}{0.907232in}}%
\pgfpathlineto{\pgfqpoint{1.347716in}{0.965332in}}%
\pgfpathlineto{\pgfqpoint{1.359801in}{0.958106in}}%
\pgfpathlineto{\pgfqpoint{1.371886in}{0.994345in}}%
\pgfpathlineto{\pgfqpoint{1.383971in}{1.023679in}}%
\pgfpathlineto{\pgfqpoint{1.396056in}{1.004496in}}%
\pgfpathlineto{\pgfqpoint{1.408142in}{1.042254in}}%
\pgfpathlineto{\pgfqpoint{1.420227in}{1.086777in}}%
\pgfpathlineto{\pgfqpoint{1.432312in}{1.092126in}}%
\pgfpathlineto{\pgfqpoint{1.444397in}{1.115445in}}%
\pgfpathlineto{\pgfqpoint{1.456482in}{1.126250in}}%
\pgfpathlineto{\pgfqpoint{1.468567in}{1.153175in}}%
\pgfpathlineto{\pgfqpoint{1.480652in}{1.170651in}}%
\pgfpathlineto{\pgfqpoint{1.492737in}{1.176491in}}%
\pgfpathlineto{\pgfqpoint{1.504822in}{1.205361in}}%
\pgfpathlineto{\pgfqpoint{1.516907in}{1.194290in}}%
\pgfpathlineto{\pgfqpoint{1.528992in}{1.227929in}}%
\pgfpathlineto{\pgfqpoint{1.541077in}{1.208783in}}%
\pgfpathlineto{\pgfqpoint{1.553163in}{1.206353in}}%
\pgfpathlineto{\pgfqpoint{1.565248in}{1.217172in}}%
\pgfpathlineto{\pgfqpoint{1.577333in}{1.231602in}}%
\pgfpathlineto{\pgfqpoint{1.589418in}{1.233851in}}%
\pgfpathlineto{\pgfqpoint{1.601503in}{1.246201in}}%
\pgfpathlineto{\pgfqpoint{1.613588in}{1.236936in}}%
\pgfpathlineto{\pgfqpoint{1.625673in}{1.245820in}}%
\pgfpathlineto{\pgfqpoint{1.637758in}{1.234143in}}%
\pgfpathlineto{\pgfqpoint{1.649843in}{1.252666in}}%
\pgfpathlineto{\pgfqpoint{1.661928in}{1.249012in}}%
\pgfpathlineto{\pgfqpoint{1.674013in}{1.242061in}}%
\pgfpathlineto{\pgfqpoint{1.686098in}{1.293510in}}%
\pgfpathlineto{\pgfqpoint{1.698184in}{1.293686in}}%
\pgfpathlineto{\pgfqpoint{1.710269in}{1.274253in}}%
\pgfpathlineto{\pgfqpoint{1.722354in}{1.264547in}}%
\pgfpathlineto{\pgfqpoint{1.734439in}{1.273158in}}%
\pgfpathlineto{\pgfqpoint{1.746524in}{1.295867in}}%
\pgfpathlineto{\pgfqpoint{1.758609in}{1.268612in}}%
\pgfpathlineto{\pgfqpoint{1.770694in}{1.292842in}}%
\pgfpathlineto{\pgfqpoint{1.782779in}{1.288217in}}%
\pgfpathlineto{\pgfqpoint{1.794864in}{1.321740in}}%
\pgfpathlineto{\pgfqpoint{1.806949in}{1.321365in}}%
\pgfpathlineto{\pgfqpoint{1.819034in}{1.331679in}}%
\pgfpathlineto{\pgfqpoint{1.831119in}{1.335149in}}%
\pgfpathlineto{\pgfqpoint{1.843204in}{1.333787in}}%
\pgfpathlineto{\pgfqpoint{1.855290in}{1.287569in}}%
\pgfpathlineto{\pgfqpoint{1.867375in}{1.329759in}}%
\pgfpathlineto{\pgfqpoint{1.891545in}{1.311625in}}%
\pgfpathlineto{\pgfqpoint{1.903630in}{1.323593in}}%
\pgfpathlineto{\pgfqpoint{1.915715in}{1.344555in}}%
\pgfpathlineto{\pgfqpoint{1.927800in}{1.344215in}}%
\pgfpathlineto{\pgfqpoint{1.939885in}{1.305988in}}%
\pgfpathlineto{\pgfqpoint{1.951970in}{1.333210in}}%
\pgfpathlineto{\pgfqpoint{1.964055in}{1.332928in}}%
\pgfpathlineto{\pgfqpoint{1.976140in}{1.333925in}}%
\pgfpathlineto{\pgfqpoint{1.988225in}{1.365439in}}%
\pgfpathlineto{\pgfqpoint{2.000311in}{1.375142in}}%
\pgfpathlineto{\pgfqpoint{2.012396in}{1.362561in}}%
\pgfpathlineto{\pgfqpoint{2.024481in}{1.359403in}}%
\pgfpathlineto{\pgfqpoint{2.036566in}{1.392097in}}%
\pgfpathlineto{\pgfqpoint{2.048651in}{1.370747in}}%
\pgfpathlineto{\pgfqpoint{2.060736in}{1.361753in}}%
\pgfpathlineto{\pgfqpoint{2.072821in}{1.399412in}}%
\pgfpathlineto{\pgfqpoint{2.084906in}{1.356500in}}%
\pgfpathlineto{\pgfqpoint{2.096991in}{1.383428in}}%
\pgfpathlineto{\pgfqpoint{2.109076in}{1.365317in}}%
\pgfpathlineto{\pgfqpoint{2.121161in}{1.367301in}}%
\pgfpathlineto{\pgfqpoint{2.133246in}{1.352629in}}%
\pgfpathlineto{\pgfqpoint{2.145332in}{1.363893in}}%
\pgfpathlineto{\pgfqpoint{2.157417in}{1.372114in}}%
\pgfpathlineto{\pgfqpoint{2.169502in}{1.382350in}}%
\pgfpathlineto{\pgfqpoint{2.181587in}{1.358226in}}%
\pgfpathlineto{\pgfqpoint{2.193672in}{1.366866in}}%
\pgfpathlineto{\pgfqpoint{2.205757in}{1.401285in}}%
\pgfpathlineto{\pgfqpoint{2.217842in}{1.343891in}}%
\pgfpathlineto{\pgfqpoint{2.229927in}{1.372697in}}%
\pgfpathlineto{\pgfqpoint{2.242012in}{1.380518in}}%
\pgfpathlineto{\pgfqpoint{2.254097in}{1.365684in}}%
\pgfpathlineto{\pgfqpoint{2.266182in}{1.371900in}}%
\pgfpathlineto{\pgfqpoint{2.278267in}{1.366258in}}%
\pgfpathlineto{\pgfqpoint{2.290353in}{1.368793in}}%
\pgfpathlineto{\pgfqpoint{2.302438in}{1.356837in}}%
\pgfpathlineto{\pgfqpoint{2.314523in}{1.375810in}}%
\pgfpathlineto{\pgfqpoint{2.326608in}{1.364059in}}%
\pgfpathlineto{\pgfqpoint{2.338693in}{1.369164in}}%
\pgfpathlineto{\pgfqpoint{2.350778in}{1.359044in}}%
\pgfpathlineto{\pgfqpoint{2.362863in}{1.363829in}}%
\pgfpathlineto{\pgfqpoint{2.374948in}{1.346256in}}%
\pgfpathlineto{\pgfqpoint{2.387033in}{1.350635in}}%
\pgfpathlineto{\pgfqpoint{2.399118in}{1.364811in}}%
\pgfpathlineto{\pgfqpoint{2.411203in}{1.338484in}}%
\pgfpathlineto{\pgfqpoint{2.423288in}{1.329930in}}%
\pgfpathlineto{\pgfqpoint{2.435374in}{1.330092in}}%
\pgfpathlineto{\pgfqpoint{2.447459in}{1.359381in}}%
\pgfpathlineto{\pgfqpoint{2.459544in}{1.359367in}}%
\pgfpathlineto{\pgfqpoint{2.471629in}{1.375409in}}%
\pgfpathlineto{\pgfqpoint{2.483714in}{1.314967in}}%
\pgfpathlineto{\pgfqpoint{2.495799in}{1.348989in}}%
\pgfpathlineto{\pgfqpoint{2.507884in}{1.350339in}}%
\pgfpathlineto{\pgfqpoint{2.519969in}{1.322445in}}%
\pgfpathlineto{\pgfqpoint{2.532054in}{1.311782in}}%
\pgfpathlineto{\pgfqpoint{2.544139in}{1.325856in}}%
\pgfpathlineto{\pgfqpoint{2.556224in}{1.300217in}}%
\pgfpathlineto{\pgfqpoint{2.568309in}{1.283701in}}%
\pgfpathlineto{\pgfqpoint{2.580395in}{1.322576in}}%
\pgfpathlineto{\pgfqpoint{2.592480in}{1.305590in}}%
\pgfpathlineto{\pgfqpoint{2.604565in}{1.306331in}}%
\pgfpathlineto{\pgfqpoint{2.616650in}{1.295630in}}%
\pgfpathlineto{\pgfqpoint{2.628735in}{1.295912in}}%
\pgfpathlineto{\pgfqpoint{2.640820in}{1.273634in}}%
\pgfpathlineto{\pgfqpoint{2.652905in}{1.273366in}}%
\pgfpathlineto{\pgfqpoint{2.664990in}{1.256323in}}%
\pgfpathlineto{\pgfqpoint{2.677075in}{1.285976in}}%
\pgfpathlineto{\pgfqpoint{2.689160in}{1.251754in}}%
\pgfpathlineto{\pgfqpoint{2.701245in}{1.249996in}}%
\pgfpathlineto{\pgfqpoint{2.713330in}{1.266350in}}%
\pgfpathlineto{\pgfqpoint{2.725415in}{1.262738in}}%
\pgfpathlineto{\pgfqpoint{2.737501in}{1.246591in}}%
\pgfpathlineto{\pgfqpoint{2.749586in}{1.239799in}}%
\pgfpathlineto{\pgfqpoint{2.761671in}{1.266804in}}%
\pgfpathlineto{\pgfqpoint{2.773756in}{1.221942in}}%
\pgfpathlineto{\pgfqpoint{2.785841in}{1.257103in}}%
\pgfpathlineto{\pgfqpoint{2.797926in}{1.243552in}}%
\pgfpathlineto{\pgfqpoint{2.810011in}{1.234100in}}%
\pgfpathlineto{\pgfqpoint{2.822096in}{1.210566in}}%
\pgfpathlineto{\pgfqpoint{2.834181in}{1.235832in}}%
\pgfpathlineto{\pgfqpoint{2.846266in}{1.235638in}}%
\pgfpathlineto{\pgfqpoint{2.858351in}{1.201048in}}%
\pgfpathlineto{\pgfqpoint{2.870436in}{1.188839in}}%
\pgfpathlineto{\pgfqpoint{2.882522in}{1.195631in}}%
\pgfpathlineto{\pgfqpoint{2.894607in}{1.179309in}}%
\pgfpathlineto{\pgfqpoint{2.906692in}{1.205279in}}%
\pgfpathlineto{\pgfqpoint{2.918777in}{1.177203in}}%
\pgfpathlineto{\pgfqpoint{2.930862in}{1.172939in}}%
\pgfpathlineto{\pgfqpoint{2.942947in}{1.178555in}}%
\pgfpathlineto{\pgfqpoint{2.955032in}{1.161988in}}%
\pgfpathlineto{\pgfqpoint{2.967117in}{1.153300in}}%
\pgfpathlineto{\pgfqpoint{2.979202in}{1.146731in}}%
\pgfpathlineto{\pgfqpoint{2.991287in}{1.173588in}}%
\pgfpathlineto{\pgfqpoint{3.003372in}{1.176074in}}%
\pgfpathlineto{\pgfqpoint{3.015457in}{1.129540in}}%
\pgfpathlineto{\pgfqpoint{3.027543in}{1.143653in}}%
\pgfpathlineto{\pgfqpoint{3.039628in}{1.149123in}}%
\pgfpathlineto{\pgfqpoint{3.051713in}{1.133208in}}%
\pgfpathlineto{\pgfqpoint{3.063798in}{1.152643in}}%
\pgfpathlineto{\pgfqpoint{3.075883in}{1.113709in}}%
\pgfpathlineto{\pgfqpoint{3.087968in}{1.113660in}}%
\pgfpathlineto{\pgfqpoint{3.100053in}{1.127126in}}%
\pgfpathlineto{\pgfqpoint{3.112138in}{1.135632in}}%
\pgfpathlineto{\pgfqpoint{3.124223in}{1.113206in}}%
\pgfpathlineto{\pgfqpoint{3.136308in}{1.128080in}}%
\pgfpathlineto{\pgfqpoint{3.148393in}{1.139237in}}%
\pgfpathlineto{\pgfqpoint{3.160478in}{1.091544in}}%
\pgfpathlineto{\pgfqpoint{3.172564in}{1.077867in}}%
\pgfpathlineto{\pgfqpoint{3.184649in}{1.099937in}}%
\pgfpathlineto{\pgfqpoint{3.196734in}{1.111545in}}%
\pgfpathlineto{\pgfqpoint{3.208819in}{1.112501in}}%
\pgfpathlineto{\pgfqpoint{3.220904in}{1.105206in}}%
\pgfpathlineto{\pgfqpoint{3.232989in}{1.108007in}}%
\pgfpathlineto{\pgfqpoint{3.245074in}{1.089493in}}%
\pgfpathlineto{\pgfqpoint{3.257159in}{1.096385in}}%
\pgfpathlineto{\pgfqpoint{3.269244in}{1.092818in}}%
\pgfpathlineto{\pgfqpoint{3.281329in}{1.114165in}}%
\pgfpathlineto{\pgfqpoint{3.293414in}{1.097459in}}%
\pgfpathlineto{\pgfqpoint{3.305499in}{1.112826in}}%
\pgfpathlineto{\pgfqpoint{3.317585in}{1.085171in}}%
\pgfpathlineto{\pgfqpoint{3.329670in}{1.079576in}}%
\pgfpathlineto{\pgfqpoint{3.341755in}{1.058112in}}%
\pgfpathlineto{\pgfqpoint{3.353840in}{1.106106in}}%
\pgfpathlineto{\pgfqpoint{3.365925in}{1.082618in}}%
\pgfpathlineto{\pgfqpoint{3.378010in}{1.113418in}}%
\pgfpathlineto{\pgfqpoint{3.390095in}{1.091709in}}%
\pgfpathlineto{\pgfqpoint{3.402180in}{1.085588in}}%
\pgfpathlineto{\pgfqpoint{3.414265in}{1.071721in}}%
\pgfpathlineto{\pgfqpoint{3.426350in}{1.107787in}}%
\pgfpathlineto{\pgfqpoint{3.438435in}{1.093490in}}%
\pgfpathlineto{\pgfqpoint{3.450520in}{1.113195in}}%
\pgfpathlineto{\pgfqpoint{3.462606in}{1.080351in}}%
\pgfpathlineto{\pgfqpoint{3.474691in}{1.092547in}}%
\pgfpathlineto{\pgfqpoint{3.486776in}{1.079187in}}%
\pgfpathlineto{\pgfqpoint{3.498861in}{1.097655in}}%
\pgfpathlineto{\pgfqpoint{3.510946in}{1.095356in}}%
\pgfpathlineto{\pgfqpoint{3.535116in}{1.099069in}}%
\pgfpathlineto{\pgfqpoint{3.547201in}{1.094867in}}%
\pgfpathlineto{\pgfqpoint{3.559286in}{1.107994in}}%
\pgfpathlineto{\pgfqpoint{3.571371in}{1.112902in}}%
\pgfpathlineto{\pgfqpoint{3.583456in}{1.103544in}}%
\pgfpathlineto{\pgfqpoint{3.595541in}{1.105790in}}%
\pgfpathlineto{\pgfqpoint{3.619712in}{1.113134in}}%
\pgfpathlineto{\pgfqpoint{3.631797in}{1.107765in}}%
\pgfpathlineto{\pgfqpoint{3.643882in}{1.104597in}}%
\pgfpathlineto{\pgfqpoint{3.655967in}{1.127312in}}%
\pgfpathlineto{\pgfqpoint{3.668052in}{1.117209in}}%
\pgfpathlineto{\pgfqpoint{3.680137in}{1.126336in}}%
\pgfpathlineto{\pgfqpoint{3.692222in}{1.107931in}}%
\pgfpathlineto{\pgfqpoint{3.704307in}{1.133946in}}%
\pgfpathlineto{\pgfqpoint{3.716392in}{1.134577in}}%
\pgfpathlineto{\pgfqpoint{3.728477in}{1.136914in}}%
\pgfpathlineto{\pgfqpoint{3.740562in}{1.150734in}}%
\pgfpathlineto{\pgfqpoint{3.752647in}{1.169480in}}%
\pgfpathlineto{\pgfqpoint{3.764733in}{1.140374in}}%
\pgfpathlineto{\pgfqpoint{3.776818in}{1.118630in}}%
\pgfpathlineto{\pgfqpoint{3.788903in}{1.166398in}}%
\pgfpathlineto{\pgfqpoint{3.800988in}{1.150431in}}%
\pgfpathlineto{\pgfqpoint{3.813073in}{1.180091in}}%
\pgfpathlineto{\pgfqpoint{3.825158in}{1.126081in}}%
\pgfpathlineto{\pgfqpoint{3.849328in}{1.185335in}}%
\pgfpathlineto{\pgfqpoint{3.861413in}{1.182043in}}%
\pgfpathlineto{\pgfqpoint{3.873498in}{1.177218in}}%
\pgfpathlineto{\pgfqpoint{3.885583in}{1.182364in}}%
\pgfpathlineto{\pgfqpoint{3.897668in}{1.172105in}}%
\pgfpathlineto{\pgfqpoint{3.909754in}{1.174891in}}%
\pgfpathlineto{\pgfqpoint{3.921839in}{1.179746in}}%
\pgfpathlineto{\pgfqpoint{3.933924in}{1.195992in}}%
\pgfpathlineto{\pgfqpoint{3.946009in}{1.188804in}}%
\pgfpathlineto{\pgfqpoint{3.958094in}{1.183114in}}%
\pgfpathlineto{\pgfqpoint{3.970179in}{1.210458in}}%
\pgfpathlineto{\pgfqpoint{3.982264in}{1.212524in}}%
\pgfpathlineto{\pgfqpoint{3.994349in}{1.204908in}}%
\pgfpathlineto{\pgfqpoint{4.006434in}{1.242323in}}%
\pgfpathlineto{\pgfqpoint{4.006434in}{1.242323in}}%
\pgfusepath{stroke}%
\end{pgfscope}%
\begin{pgfscope}%
\pgfpathrectangle{\pgfqpoint{0.743462in}{0.408431in}}{\pgfqpoint{3.262972in}{1.531616in}}%
\pgfusepath{clip}%
\pgfsetbuttcap%
\pgfsetroundjoin%
\pgfsetlinewidth{0.752812pt}%
\definecolor{currentstroke}{rgb}{0.392157,0.396078,0.403922}%
\pgfsetstrokecolor{currentstroke}%
\pgfsetdash{{2.775000pt}{1.200000pt}}{0.000000pt}%
\pgfpathmoveto{\pgfqpoint{0.755547in}{0.492760in}}%
\pgfpathlineto{\pgfqpoint{0.767632in}{0.521746in}}%
\pgfpathlineto{\pgfqpoint{0.779717in}{0.499438in}}%
\pgfpathlineto{\pgfqpoint{0.791802in}{0.501805in}}%
\pgfpathlineto{\pgfqpoint{0.803887in}{0.505991in}}%
\pgfpathlineto{\pgfqpoint{0.815973in}{0.478050in}}%
\pgfpathlineto{\pgfqpoint{0.828058in}{0.512478in}}%
\pgfpathlineto{\pgfqpoint{0.840143in}{0.524523in}}%
\pgfpathlineto{\pgfqpoint{0.852228in}{0.507465in}}%
\pgfpathlineto{\pgfqpoint{0.876398in}{0.490397in}}%
\pgfpathlineto{\pgfqpoint{0.888483in}{0.494595in}}%
\pgfpathlineto{\pgfqpoint{0.900568in}{0.512448in}}%
\pgfpathlineto{\pgfqpoint{0.912653in}{0.514950in}}%
\pgfpathlineto{\pgfqpoint{0.924738in}{0.492078in}}%
\pgfpathlineto{\pgfqpoint{0.936823in}{0.498768in}}%
\pgfpathlineto{\pgfqpoint{0.948908in}{0.491276in}}%
\pgfpathlineto{\pgfqpoint{0.960993in}{0.501223in}}%
\pgfpathlineto{\pgfqpoint{0.973079in}{0.519111in}}%
\pgfpathlineto{\pgfqpoint{0.985164in}{0.502411in}}%
\pgfpathlineto{\pgfqpoint{0.997249in}{0.496211in}}%
\pgfpathlineto{\pgfqpoint{1.009334in}{0.520368in}}%
\pgfpathlineto{\pgfqpoint{1.021419in}{0.516033in}}%
\pgfpathlineto{\pgfqpoint{1.033504in}{0.506095in}}%
\pgfpathlineto{\pgfqpoint{1.045589in}{0.506322in}}%
\pgfpathlineto{\pgfqpoint{1.057674in}{0.530005in}}%
\pgfpathlineto{\pgfqpoint{1.069759in}{0.530057in}}%
\pgfpathlineto{\pgfqpoint{1.081844in}{0.520299in}}%
\pgfpathlineto{\pgfqpoint{1.093929in}{0.559348in}}%
\pgfpathlineto{\pgfqpoint{1.118100in}{0.561091in}}%
\pgfpathlineto{\pgfqpoint{1.130185in}{0.571247in}}%
\pgfpathlineto{\pgfqpoint{1.142270in}{0.621747in}}%
\pgfpathlineto{\pgfqpoint{1.154355in}{0.598233in}}%
\pgfpathlineto{\pgfqpoint{1.166440in}{0.631806in}}%
\pgfpathlineto{\pgfqpoint{1.178525in}{0.661375in}}%
\pgfpathlineto{\pgfqpoint{1.190610in}{0.678552in}}%
\pgfpathlineto{\pgfqpoint{1.202695in}{0.722033in}}%
\pgfpathlineto{\pgfqpoint{1.214780in}{0.726479in}}%
\pgfpathlineto{\pgfqpoint{1.226865in}{0.760868in}}%
\pgfpathlineto{\pgfqpoint{1.238950in}{0.786010in}}%
\pgfpathlineto{\pgfqpoint{1.251035in}{0.785041in}}%
\pgfpathlineto{\pgfqpoint{1.263121in}{0.848876in}}%
\pgfpathlineto{\pgfqpoint{1.275206in}{0.879733in}}%
\pgfpathlineto{\pgfqpoint{1.287291in}{0.898181in}}%
\pgfpathlineto{\pgfqpoint{1.299376in}{0.914425in}}%
\pgfpathlineto{\pgfqpoint{1.311461in}{0.944500in}}%
\pgfpathlineto{\pgfqpoint{1.323546in}{0.985645in}}%
\pgfpathlineto{\pgfqpoint{1.335631in}{1.005337in}}%
\pgfpathlineto{\pgfqpoint{1.347716in}{1.066971in}}%
\pgfpathlineto{\pgfqpoint{1.359801in}{1.073187in}}%
\pgfpathlineto{\pgfqpoint{1.371886in}{1.126257in}}%
\pgfpathlineto{\pgfqpoint{1.383971in}{1.157250in}}%
\pgfpathlineto{\pgfqpoint{1.396056in}{1.158647in}}%
\pgfpathlineto{\pgfqpoint{1.420227in}{1.260319in}}%
\pgfpathlineto{\pgfqpoint{1.432312in}{1.275093in}}%
\pgfpathlineto{\pgfqpoint{1.444397in}{1.324045in}}%
\pgfpathlineto{\pgfqpoint{1.456482in}{1.342016in}}%
\pgfpathlineto{\pgfqpoint{1.468567in}{1.387359in}}%
\pgfpathlineto{\pgfqpoint{1.492737in}{1.418489in}}%
\pgfpathlineto{\pgfqpoint{1.504822in}{1.467613in}}%
\pgfpathlineto{\pgfqpoint{1.516907in}{1.467070in}}%
\pgfpathlineto{\pgfqpoint{1.528992in}{1.514996in}}%
\pgfpathlineto{\pgfqpoint{1.541077in}{1.485049in}}%
\pgfpathlineto{\pgfqpoint{1.553163in}{1.504069in}}%
\pgfpathlineto{\pgfqpoint{1.565248in}{1.498303in}}%
\pgfpathlineto{\pgfqpoint{1.577333in}{1.529626in}}%
\pgfpathlineto{\pgfqpoint{1.589418in}{1.530054in}}%
\pgfpathlineto{\pgfqpoint{1.601503in}{1.556509in}}%
\pgfpathlineto{\pgfqpoint{1.613588in}{1.545074in}}%
\pgfpathlineto{\pgfqpoint{1.625673in}{1.582344in}}%
\pgfpathlineto{\pgfqpoint{1.637758in}{1.551074in}}%
\pgfpathlineto{\pgfqpoint{1.649843in}{1.564865in}}%
\pgfpathlineto{\pgfqpoint{1.661928in}{1.564705in}}%
\pgfpathlineto{\pgfqpoint{1.674013in}{1.560924in}}%
\pgfpathlineto{\pgfqpoint{1.686098in}{1.616516in}}%
\pgfpathlineto{\pgfqpoint{1.698184in}{1.627239in}}%
\pgfpathlineto{\pgfqpoint{1.710269in}{1.608172in}}%
\pgfpathlineto{\pgfqpoint{1.722354in}{1.617050in}}%
\pgfpathlineto{\pgfqpoint{1.734439in}{1.621891in}}%
\pgfpathlineto{\pgfqpoint{1.746524in}{1.644268in}}%
\pgfpathlineto{\pgfqpoint{1.758609in}{1.622464in}}%
\pgfpathlineto{\pgfqpoint{1.770694in}{1.635641in}}%
\pgfpathlineto{\pgfqpoint{1.782779in}{1.656616in}}%
\pgfpathlineto{\pgfqpoint{1.806949in}{1.716789in}}%
\pgfpathlineto{\pgfqpoint{1.819034in}{1.732827in}}%
\pgfpathlineto{\pgfqpoint{1.831119in}{1.720058in}}%
\pgfpathlineto{\pgfqpoint{1.843204in}{1.733391in}}%
\pgfpathlineto{\pgfqpoint{1.855290in}{1.676487in}}%
\pgfpathlineto{\pgfqpoint{1.867375in}{1.714412in}}%
\pgfpathlineto{\pgfqpoint{1.879460in}{1.713598in}}%
\pgfpathlineto{\pgfqpoint{1.891545in}{1.698456in}}%
\pgfpathlineto{\pgfqpoint{1.903630in}{1.720115in}}%
\pgfpathlineto{\pgfqpoint{1.915715in}{1.770643in}}%
\pgfpathlineto{\pgfqpoint{1.927800in}{1.759401in}}%
\pgfpathlineto{\pgfqpoint{1.939885in}{1.721299in}}%
\pgfpathlineto{\pgfqpoint{1.951970in}{1.768400in}}%
\pgfpathlineto{\pgfqpoint{1.964055in}{1.747547in}}%
\pgfpathlineto{\pgfqpoint{1.976140in}{1.761142in}}%
\pgfpathlineto{\pgfqpoint{1.988225in}{1.790580in}}%
\pgfpathlineto{\pgfqpoint{2.000311in}{1.805957in}}%
\pgfpathlineto{\pgfqpoint{2.012396in}{1.792891in}}%
\pgfpathlineto{\pgfqpoint{2.024481in}{1.792253in}}%
\pgfpathlineto{\pgfqpoint{2.036566in}{1.845878in}}%
\pgfpathlineto{\pgfqpoint{2.048651in}{1.811512in}}%
\pgfpathlineto{\pgfqpoint{2.060736in}{1.793788in}}%
\pgfpathlineto{\pgfqpoint{2.072821in}{1.850014in}}%
\pgfpathlineto{\pgfqpoint{2.084906in}{1.787732in}}%
\pgfpathlineto{\pgfqpoint{2.096991in}{1.831179in}}%
\pgfpathlineto{\pgfqpoint{2.109076in}{1.796721in}}%
\pgfpathlineto{\pgfqpoint{2.121161in}{1.816094in}}%
\pgfpathlineto{\pgfqpoint{2.133246in}{1.819221in}}%
\pgfpathlineto{\pgfqpoint{2.145332in}{1.824576in}}%
\pgfpathlineto{\pgfqpoint{2.157417in}{1.818499in}}%
\pgfpathlineto{\pgfqpoint{2.169502in}{1.831674in}}%
\pgfpathlineto{\pgfqpoint{2.181587in}{1.802862in}}%
\pgfpathlineto{\pgfqpoint{2.193672in}{1.826441in}}%
\pgfpathlineto{\pgfqpoint{2.205757in}{1.870428in}}%
\pgfpathlineto{\pgfqpoint{2.217842in}{1.803889in}}%
\pgfpathlineto{\pgfqpoint{2.229927in}{1.849146in}}%
\pgfpathlineto{\pgfqpoint{2.242012in}{1.858116in}}%
\pgfpathlineto{\pgfqpoint{2.254097in}{1.820910in}}%
\pgfpathlineto{\pgfqpoint{2.266182in}{1.825198in}}%
\pgfpathlineto{\pgfqpoint{2.278267in}{1.805603in}}%
\pgfpathlineto{\pgfqpoint{2.290353in}{1.818597in}}%
\pgfpathlineto{\pgfqpoint{2.302438in}{1.789320in}}%
\pgfpathlineto{\pgfqpoint{2.314523in}{1.820918in}}%
\pgfpathlineto{\pgfqpoint{2.326608in}{1.825194in}}%
\pgfpathlineto{\pgfqpoint{2.338693in}{1.804331in}}%
\pgfpathlineto{\pgfqpoint{2.350778in}{1.818895in}}%
\pgfpathlineto{\pgfqpoint{2.374948in}{1.768923in}}%
\pgfpathlineto{\pgfqpoint{2.387033in}{1.778395in}}%
\pgfpathlineto{\pgfqpoint{2.399118in}{1.792260in}}%
\pgfpathlineto{\pgfqpoint{2.411203in}{1.755833in}}%
\pgfpathlineto{\pgfqpoint{2.423288in}{1.758358in}}%
\pgfpathlineto{\pgfqpoint{2.435374in}{1.754063in}}%
\pgfpathlineto{\pgfqpoint{2.447459in}{1.779404in}}%
\pgfpathlineto{\pgfqpoint{2.459544in}{1.767922in}}%
\pgfpathlineto{\pgfqpoint{2.471629in}{1.802501in}}%
\pgfpathlineto{\pgfqpoint{2.483714in}{1.717002in}}%
\pgfpathlineto{\pgfqpoint{2.495799in}{1.763980in}}%
\pgfpathlineto{\pgfqpoint{2.507884in}{1.754942in}}%
\pgfpathlineto{\pgfqpoint{2.519969in}{1.728226in}}%
\pgfpathlineto{\pgfqpoint{2.532054in}{1.691071in}}%
\pgfpathlineto{\pgfqpoint{2.544139in}{1.720397in}}%
\pgfpathlineto{\pgfqpoint{2.556224in}{1.662790in}}%
\pgfpathlineto{\pgfqpoint{2.568309in}{1.683162in}}%
\pgfpathlineto{\pgfqpoint{2.580395in}{1.688373in}}%
\pgfpathlineto{\pgfqpoint{2.592480in}{1.684455in}}%
\pgfpathlineto{\pgfqpoint{2.604565in}{1.660777in}}%
\pgfpathlineto{\pgfqpoint{2.616650in}{1.646725in}}%
\pgfpathlineto{\pgfqpoint{2.628735in}{1.659316in}}%
\pgfpathlineto{\pgfqpoint{2.640820in}{1.636449in}}%
\pgfpathlineto{\pgfqpoint{2.652905in}{1.623659in}}%
\pgfpathlineto{\pgfqpoint{2.664990in}{1.605433in}}%
\pgfpathlineto{\pgfqpoint{2.677075in}{1.620724in}}%
\pgfpathlineto{\pgfqpoint{2.689160in}{1.583869in}}%
\pgfpathlineto{\pgfqpoint{2.701245in}{1.601455in}}%
\pgfpathlineto{\pgfqpoint{2.713330in}{1.587427in}}%
\pgfpathlineto{\pgfqpoint{2.725415in}{1.569003in}}%
\pgfpathlineto{\pgfqpoint{2.737501in}{1.556225in}}%
\pgfpathlineto{\pgfqpoint{2.749586in}{1.559147in}}%
\pgfpathlineto{\pgfqpoint{2.761671in}{1.588472in}}%
\pgfpathlineto{\pgfqpoint{2.773756in}{1.522071in}}%
\pgfpathlineto{\pgfqpoint{2.785841in}{1.576364in}}%
\pgfpathlineto{\pgfqpoint{2.797926in}{1.534257in}}%
\pgfpathlineto{\pgfqpoint{2.810011in}{1.543672in}}%
\pgfpathlineto{\pgfqpoint{2.822096in}{1.509313in}}%
\pgfpathlineto{\pgfqpoint{2.834181in}{1.530885in}}%
\pgfpathlineto{\pgfqpoint{2.846266in}{1.508631in}}%
\pgfpathlineto{\pgfqpoint{2.858351in}{1.475425in}}%
\pgfpathlineto{\pgfqpoint{2.870436in}{1.458553in}}%
\pgfpathlineto{\pgfqpoint{2.882522in}{1.473959in}}%
\pgfpathlineto{\pgfqpoint{2.894607in}{1.441032in}}%
\pgfpathlineto{\pgfqpoint{2.906692in}{1.472631in}}%
\pgfpathlineto{\pgfqpoint{2.918777in}{1.432437in}}%
\pgfpathlineto{\pgfqpoint{2.930862in}{1.427340in}}%
\pgfpathlineto{\pgfqpoint{2.942947in}{1.423713in}}%
\pgfpathlineto{\pgfqpoint{2.955032in}{1.408845in}}%
\pgfpathlineto{\pgfqpoint{2.967117in}{1.408776in}}%
\pgfpathlineto{\pgfqpoint{2.979202in}{1.404194in}}%
\pgfpathlineto{\pgfqpoint{2.991287in}{1.420939in}}%
\pgfpathlineto{\pgfqpoint{3.003372in}{1.408675in}}%
\pgfpathlineto{\pgfqpoint{3.015457in}{1.358630in}}%
\pgfpathlineto{\pgfqpoint{3.027543in}{1.377940in}}%
\pgfpathlineto{\pgfqpoint{3.039628in}{1.364628in}}%
\pgfpathlineto{\pgfqpoint{3.051713in}{1.354366in}}%
\pgfpathlineto{\pgfqpoint{3.063798in}{1.373352in}}%
\pgfpathlineto{\pgfqpoint{3.075883in}{1.337183in}}%
\pgfpathlineto{\pgfqpoint{3.087968in}{1.325556in}}%
\pgfpathlineto{\pgfqpoint{3.100053in}{1.347772in}}%
\pgfpathlineto{\pgfqpoint{3.112138in}{1.352451in}}%
\pgfpathlineto{\pgfqpoint{3.124223in}{1.326670in}}%
\pgfpathlineto{\pgfqpoint{3.136308in}{1.339618in}}%
\pgfpathlineto{\pgfqpoint{3.148393in}{1.340554in}}%
\pgfpathlineto{\pgfqpoint{3.160478in}{1.298938in}}%
\pgfpathlineto{\pgfqpoint{3.172564in}{1.291106in}}%
\pgfpathlineto{\pgfqpoint{3.184649in}{1.297482in}}%
\pgfpathlineto{\pgfqpoint{3.196734in}{1.316274in}}%
\pgfpathlineto{\pgfqpoint{3.208819in}{1.309801in}}%
\pgfpathlineto{\pgfqpoint{3.220904in}{1.306714in}}%
\pgfpathlineto{\pgfqpoint{3.232989in}{1.312577in}}%
\pgfpathlineto{\pgfqpoint{3.245074in}{1.281339in}}%
\pgfpathlineto{\pgfqpoint{3.257159in}{1.285503in}}%
\pgfpathlineto{\pgfqpoint{3.269244in}{1.278374in}}%
\pgfpathlineto{\pgfqpoint{3.281329in}{1.301620in}}%
\pgfpathlineto{\pgfqpoint{3.293414in}{1.283487in}}%
\pgfpathlineto{\pgfqpoint{3.305499in}{1.312140in}}%
\pgfpathlineto{\pgfqpoint{3.317585in}{1.285894in}}%
\pgfpathlineto{\pgfqpoint{3.329670in}{1.274131in}}%
\pgfpathlineto{\pgfqpoint{3.341755in}{1.245971in}}%
\pgfpathlineto{\pgfqpoint{3.353840in}{1.288815in}}%
\pgfpathlineto{\pgfqpoint{3.365925in}{1.268882in}}%
\pgfpathlineto{\pgfqpoint{3.378010in}{1.305886in}}%
\pgfpathlineto{\pgfqpoint{3.390095in}{1.281256in}}%
\pgfpathlineto{\pgfqpoint{3.402180in}{1.278250in}}%
\pgfpathlineto{\pgfqpoint{3.414265in}{1.247975in}}%
\pgfpathlineto{\pgfqpoint{3.426350in}{1.284180in}}%
\pgfpathlineto{\pgfqpoint{3.438435in}{1.279282in}}%
\pgfpathlineto{\pgfqpoint{3.450520in}{1.297398in}}%
\pgfpathlineto{\pgfqpoint{3.462606in}{1.272110in}}%
\pgfpathlineto{\pgfqpoint{3.474691in}{1.291027in}}%
\pgfpathlineto{\pgfqpoint{3.486776in}{1.262679in}}%
\pgfpathlineto{\pgfqpoint{3.498861in}{1.298547in}}%
\pgfpathlineto{\pgfqpoint{3.510946in}{1.283100in}}%
\pgfpathlineto{\pgfqpoint{3.523031in}{1.282961in}}%
\pgfpathlineto{\pgfqpoint{3.535116in}{1.290528in}}%
\pgfpathlineto{\pgfqpoint{3.547201in}{1.295090in}}%
\pgfpathlineto{\pgfqpoint{3.559286in}{1.294100in}}%
\pgfpathlineto{\pgfqpoint{3.571371in}{1.298173in}}%
\pgfpathlineto{\pgfqpoint{3.583456in}{1.295988in}}%
\pgfpathlineto{\pgfqpoint{3.595541in}{1.299353in}}%
\pgfpathlineto{\pgfqpoint{3.607626in}{1.319167in}}%
\pgfpathlineto{\pgfqpoint{3.619712in}{1.318183in}}%
\pgfpathlineto{\pgfqpoint{3.631797in}{1.313344in}}%
\pgfpathlineto{\pgfqpoint{3.643882in}{1.300571in}}%
\pgfpathlineto{\pgfqpoint{3.655967in}{1.332191in}}%
\pgfpathlineto{\pgfqpoint{3.668052in}{1.320088in}}%
\pgfpathlineto{\pgfqpoint{3.680137in}{1.337230in}}%
\pgfpathlineto{\pgfqpoint{3.692222in}{1.324813in}}%
\pgfpathlineto{\pgfqpoint{3.704307in}{1.346354in}}%
\pgfpathlineto{\pgfqpoint{3.716392in}{1.352305in}}%
\pgfpathlineto{\pgfqpoint{3.728477in}{1.350335in}}%
\pgfpathlineto{\pgfqpoint{3.740562in}{1.366145in}}%
\pgfpathlineto{\pgfqpoint{3.752647in}{1.407533in}}%
\pgfpathlineto{\pgfqpoint{3.764733in}{1.357556in}}%
\pgfpathlineto{\pgfqpoint{3.776818in}{1.364857in}}%
\pgfpathlineto{\pgfqpoint{3.788903in}{1.404607in}}%
\pgfpathlineto{\pgfqpoint{3.800988in}{1.386321in}}%
\pgfpathlineto{\pgfqpoint{3.813073in}{1.417698in}}%
\pgfpathlineto{\pgfqpoint{3.825158in}{1.383138in}}%
\pgfpathlineto{\pgfqpoint{3.837243in}{1.406077in}}%
\pgfpathlineto{\pgfqpoint{3.849328in}{1.444473in}}%
\pgfpathlineto{\pgfqpoint{3.861413in}{1.434593in}}%
\pgfpathlineto{\pgfqpoint{3.873498in}{1.437260in}}%
\pgfpathlineto{\pgfqpoint{3.885583in}{1.436190in}}%
\pgfpathlineto{\pgfqpoint{3.909754in}{1.442498in}}%
\pgfpathlineto{\pgfqpoint{3.921839in}{1.451096in}}%
\pgfpathlineto{\pgfqpoint{3.933924in}{1.483523in}}%
\pgfpathlineto{\pgfqpoint{3.946009in}{1.471060in}}%
\pgfpathlineto{\pgfqpoint{3.958094in}{1.463338in}}%
\pgfpathlineto{\pgfqpoint{3.970179in}{1.509558in}}%
\pgfpathlineto{\pgfqpoint{3.982264in}{1.510020in}}%
\pgfpathlineto{\pgfqpoint{3.994349in}{1.503032in}}%
\pgfpathlineto{\pgfqpoint{4.006434in}{1.539699in}}%
\pgfpathlineto{\pgfqpoint{4.006434in}{1.539699in}}%
\pgfusepath{stroke}%
\end{pgfscope}%
\begin{pgfscope}%
\pgfsetrectcap%
\pgfsetmiterjoin%
\pgfsetlinewidth{0.803000pt}%
\definecolor{currentstroke}{rgb}{0.000000,0.000000,0.000000}%
\pgfsetstrokecolor{currentstroke}%
\pgfsetdash{}{0pt}%
\pgfpathmoveto{\pgfqpoint{0.743462in}{0.408431in}}%
\pgfpathlineto{\pgfqpoint{0.743462in}{1.940047in}}%
\pgfusepath{stroke}%
\end{pgfscope}%
\begin{pgfscope}%
\pgfsetrectcap%
\pgfsetmiterjoin%
\pgfsetlinewidth{0.803000pt}%
\definecolor{currentstroke}{rgb}{0.000000,0.000000,0.000000}%
\pgfsetstrokecolor{currentstroke}%
\pgfsetdash{}{0pt}%
\pgfpathmoveto{\pgfqpoint{4.006434in}{0.408431in}}%
\pgfpathlineto{\pgfqpoint{4.006434in}{1.940047in}}%
\pgfusepath{stroke}%
\end{pgfscope}%
\begin{pgfscope}%
\pgfsetrectcap%
\pgfsetmiterjoin%
\pgfsetlinewidth{0.803000pt}%
\definecolor{currentstroke}{rgb}{0.000000,0.000000,0.000000}%
\pgfsetstrokecolor{currentstroke}%
\pgfsetdash{}{0pt}%
\pgfpathmoveto{\pgfqpoint{0.743462in}{0.408431in}}%
\pgfpathlineto{\pgfqpoint{4.006434in}{0.408431in}}%
\pgfusepath{stroke}%
\end{pgfscope}%
\begin{pgfscope}%
\pgfsetrectcap%
\pgfsetmiterjoin%
\pgfsetlinewidth{0.803000pt}%
\definecolor{currentstroke}{rgb}{0.000000,0.000000,0.000000}%
\pgfsetstrokecolor{currentstroke}%
\pgfsetdash{}{0pt}%
\pgfpathmoveto{\pgfqpoint{0.743462in}{1.940047in}}%
\pgfpathlineto{\pgfqpoint{4.006434in}{1.940047in}}%
\pgfusepath{stroke}%
\end{pgfscope}%
\begin{pgfscope}%
\pgfsetrectcap%
\pgfsetroundjoin%
\pgfsetlinewidth{0.752812pt}%
\definecolor{currentstroke}{rgb}{0.000000,0.000000,0.000000}%
\pgfsetstrokecolor{currentstroke}%
\pgfsetdash{}{0pt}%
\pgfpathmoveto{\pgfqpoint{3.322379in}{0.829746in}}%
\pgfpathlineto{\pgfqpoint{3.600157in}{0.829746in}}%
\pgfusepath{stroke}%
\end{pgfscope}%
\begin{pgfscope}%
\definecolor{textcolor}{rgb}{0.000000,0.000000,0.000000}%
\pgfsetstrokecolor{textcolor}%
\pgfsetfillcolor{textcolor}%
\pgftext[x=3.711268in,y=0.781135in,left,base]{\color{textcolor}\rmfamily\fontsize{10.000000}{12.000000}\selectfont \(\displaystyle \mathbf{K}_t^*\)}%
\end{pgfscope}%
\begin{pgfscope}%
\pgfsetbuttcap%
\pgfsetroundjoin%
\pgfsetlinewidth{0.752812pt}%
\definecolor{currentstroke}{rgb}{0.392157,0.396078,0.403922}%
\pgfsetstrokecolor{currentstroke}%
\pgfsetdash{{2.775000pt}{1.200000pt}}{0.000000pt}%
\pgfpathmoveto{\pgfqpoint{3.322379in}{0.623234in}}%
\pgfpathlineto{\pgfqpoint{3.600157in}{0.623234in}}%
\pgfusepath{stroke}%
\end{pgfscope}%
\begin{pgfscope}%
\definecolor{textcolor}{rgb}{0.000000,0.000000,0.000000}%
\pgfsetstrokecolor{textcolor}%
\pgfsetfillcolor{textcolor}%
\pgftext[x=3.711268in,y=0.574622in,left,base]{\color{textcolor}\rmfamily\fontsize{10.000000}{12.000000}\selectfont \(\displaystyle \mathbf{K}_0^*\)}%
\end{pgfscope}%
\end{pgfpicture}%
\makeatother%
\endgroup%

    \caption[Increase in \gls{lqr} cost due to time-varying friction.]{Normalized optimal costs of the \gls{lqr} problem of an inverted pendulum with the optimal controller at each time step (solid black line) and the optimal controller from the time step $t=0$ (dashed gray line).}
    \label{fig:LQR_cost}
\end{figure}

In the following, the variations in Table~\ref{tab:models} are benchmarked on this time-varying cost function without prior knowledge of the system dynamics and the controller gain $\mathbf{K}$ as the decision variable for the optimization. The controller gain $\mathbf{K}$ of the feedback controller has the dimensions $1\times4$ as 
\begin{equation}
    \mathbf{K} = [\theta_1,\theta_2,\theta_3,\theta_4].
\end{equation}
Since \gls{ctvbo} does not scale well in the dimensions (see Figure~\ref{fig:dims_vops}), the first two entries are always set to the optimal values $\theta_1^*,\theta_2^*$ calculated according to \eqref{eq:optimal_controller} and the black-box optimization is performed using $\theta_3$ and $\theta_4$ as degrees of freedom. The weighting matrices are set to $\mathbf{Q}=10\cdot\text{eye}(4)$ and $\mathbf{R} = 1$. To have accurate feedback about the cost, the simulations are performed using the linearized system, not the non-linear system. 

The feasible set is $\mathcal{X}=[-50, -25]\times[-4, -2]$ considering only stable controllers and avoiding numerical issues. Furthermore, the feasible set is scaled using $[3, \nicefrac{1}{4}]$ to have similar spatial intervals in each dimension. Gamma hyperpriors are used for the length scales as $\boldsymbol\Lambda_{11}, \boldsymbol\Lambda_{22} \sim \mathcal{G}(6, \nicefrac{10}{3})$ \eqref{eq:gamma} with bounds $\boldsymbol\Lambda_{11},\boldsymbol\Lambda_{22} \in [0.5, 6]$. As the \gls{lqr} cost function is flat around the optimum, the  bounding functions for \gls{ctvbo} are defined as $a(\mathbf{X}_v)=0$ and $b(\mathbf{X}_v)=2$. Furthermore, the number of \glspl{vop} per dimension is set as $N_{v/D} = 4$ and $\delta =1.2$ \eqref{eq:delta}. As in the moving parabola experiments, the initial training data of $N=30$ data points is normalized to zero mean and a standard deviation of one. The forgetting factors were chosen as $\hat{\sigma}_w^2 = \epsilon = 0.03$. As the optimal cost $J_t^*$ increases after $t=50$ (Figure~\ref{fig:LQR_cost}), the initial prior mean will result in an optimistic prior mean over time.
Therefore, it is expected that the variations using the proposed modeling approach with \gls{ui} forgetting will outperform the variations using \gls{b2p} forgetting as stated in Hypothesis~\ref{hyp:ui_structural_information}.

Figure~\ref{fig:LQR_cumulative_regret} shows the results of five different but for each variant consistent initializations.
\begin{figure}[h]
    \centering
    \input{thesis/figures/pgf_figures/LQR_regret.pgf}
    \caption[Results of the \gls{lqr} problem of an inverted pendulum.]{Results of the \gls{lqr} problem of an inverted pendulum. \gls{ui} forgetting is less sensitive to the increase in cost over time. \gls{ctvbo} further reduces the regret as well as its variance.}
    \label{fig:LQR_cumulative_regret}
\end{figure}
All variations outperform the initial optimal controller $\mathbf{K}_0^*$. 

Furthermore, using the proposed method \gls{ctvbo} reduces the mean regret compared to standard \gls{tvbo} (Hypothesis~\ref{hyp:ctvbo}). As expected, the variations using \gls{b2p} forgetting are more sensitive to the increase in cost over time compared to \gls{ui} forgetting resulting in a higher regret. To further investigate the exploration behavior of the variations, the regret is split into an exploration regret and an exploitation regret as
% \begin{align}
%     R_T &= \sum_{t=1}^T \left(f_t(\mathbf{x}_t) - f_t(\mathbf{x}_t^*)\right) \\
%     &= \underbrace{\sum_{t=1}^T \left(f_t(\mathbf{x}_t) - f_t(\mathbf{\hat{x}}_t)\right)}_{\coloneqq \hat{R}_T \text{ (Exploration)}} + \underbrace{\sum_{t=1}^T \left(f_t(\mathbf{\hat{x}}_t) - f_t(\mathbf{x}_t^*)\right)}_{\coloneqq R_T^* \text{ (Exploitation)}}.
% \end{align}
\begin{equation}
    R_T = \sum_{t=1}^T \left(f_t(\mathbf{x}_t) - f_t(\mathbf{x}_t^*)\right) = \underbrace{\sum_{t=1}^T \left(f_t(\mathbf{x}_t) - f_t(\mathbf{\hat{x}}_t)\right)}_{\coloneqq \hat{R}_T \text{ (Exploration)}} + \underbrace{\sum_{t=1}^T \left(f_t(\mathbf{\hat{x}}_t) - f_t(\mathbf{x}_t^*)\right)}_{\coloneqq R_T^* \text{ (Exploitation)}}.
    \label{eq:split_regret}
\end{equation}
The exploitation regret $R_T^*$ represents the cost of the predicted optimum $\mathbf{\hat{x}}_t$ deviating from the true optimum $\mathbf{x}_t^*$. In contrast, the exploration regret $\hat{R}_T$ captures the cost of choosing a query deviating from the predicted optimum. The split regret for the \gls{lqr} problem is displayed in Figure~\ref{fig:LQR_split_regret}. It shows that the constrained models deviate more from the optimum initially than is the case for the unconstrained models. A reason for this is that the objective function is very flat in the beginning. However, after the increase in cost, it becomes apparent that regardless of the algorithm, the optimum can be tracked equally well in comparison since the lines of $R_T^*$ of unconstrained and constrained variation run roughly parallel for $t>100$.
\begin{figure}[h]
    \centering
    %% Creator: Matplotlib, PGF backend
%%
%% To include the figure in your LaTeX document, write
%%   \input{<filename>.pgf}
%%
%% Make sure the required packages are loaded in your preamble
%%   \usepackage{pgf}
%%
%% Figures using additional raster images can only be included by \input if
%% they are in the same directory as the main LaTeX file. For loading figures
%% from other directories you can use the `import` package
%%   \usepackage{import}
%%
%% and then include the figures with
%%   \import{<path to file>}{<filename>.pgf}
%%
%% Matplotlib used the following preamble
%%   \usepackage{fontspec}
%%
\begingroup%
\makeatletter%
\begin{pgfpicture}%
\pgfpathrectangle{\pgfpointorigin}{\pgfqpoint{5.507126in}{2.552693in}}%
\pgfusepath{use as bounding box, clip}%
\begin{pgfscope}%
\pgfsetbuttcap%
\pgfsetmiterjoin%
\definecolor{currentfill}{rgb}{1.000000,1.000000,1.000000}%
\pgfsetfillcolor{currentfill}%
\pgfsetlinewidth{0.000000pt}%
\definecolor{currentstroke}{rgb}{1.000000,1.000000,1.000000}%
\pgfsetstrokecolor{currentstroke}%
\pgfsetdash{}{0pt}%
\pgfpathmoveto{\pgfqpoint{0.000000in}{0.000000in}}%
\pgfpathlineto{\pgfqpoint{5.507126in}{0.000000in}}%
\pgfpathlineto{\pgfqpoint{5.507126in}{2.552693in}}%
\pgfpathlineto{\pgfqpoint{0.000000in}{2.552693in}}%
\pgfpathclose%
\pgfusepath{fill}%
\end{pgfscope}%
\begin{pgfscope}%
\pgfsetbuttcap%
\pgfsetmiterjoin%
\definecolor{currentfill}{rgb}{1.000000,1.000000,1.000000}%
\pgfsetfillcolor{currentfill}%
\pgfsetlinewidth{0.000000pt}%
\definecolor{currentstroke}{rgb}{0.000000,0.000000,0.000000}%
\pgfsetstrokecolor{currentstroke}%
\pgfsetstrokeopacity{0.000000}%
\pgfsetdash{}{0pt}%
\pgfpathmoveto{\pgfqpoint{0.550713in}{1.428886in}}%
\pgfpathlineto{\pgfqpoint{3.744846in}{1.428886in}}%
\pgfpathlineto{\pgfqpoint{3.744846in}{2.425059in}}%
\pgfpathlineto{\pgfqpoint{0.550713in}{2.425059in}}%
\pgfpathclose%
\pgfusepath{fill}%
\end{pgfscope}%
\begin{pgfscope}%
\pgfsetbuttcap%
\pgfsetroundjoin%
\definecolor{currentfill}{rgb}{0.000000,0.000000,0.000000}%
\pgfsetfillcolor{currentfill}%
\pgfsetlinewidth{0.803000pt}%
\definecolor{currentstroke}{rgb}{0.000000,0.000000,0.000000}%
\pgfsetstrokecolor{currentstroke}%
\pgfsetdash{}{0pt}%
\pgfsys@defobject{currentmarker}{\pgfqpoint{0.000000in}{-0.048611in}}{\pgfqpoint{0.000000in}{0.000000in}}{%
\pgfpathmoveto{\pgfqpoint{0.000000in}{0.000000in}}%
\pgfpathlineto{\pgfqpoint{0.000000in}{-0.048611in}}%
\pgfusepath{stroke,fill}%
}%
\begin{pgfscope}%
\pgfsys@transformshift{0.787315in}{1.428886in}%
\pgfsys@useobject{currentmarker}{}%
\end{pgfscope}%
\end{pgfscope}%
\begin{pgfscope}%
\pgfsetbuttcap%
\pgfsetroundjoin%
\definecolor{currentfill}{rgb}{0.000000,0.000000,0.000000}%
\pgfsetfillcolor{currentfill}%
\pgfsetlinewidth{0.803000pt}%
\definecolor{currentstroke}{rgb}{0.000000,0.000000,0.000000}%
\pgfsetstrokecolor{currentstroke}%
\pgfsetdash{}{0pt}%
\pgfsys@defobject{currentmarker}{\pgfqpoint{0.000000in}{-0.048611in}}{\pgfqpoint{0.000000in}{0.000000in}}{%
\pgfpathmoveto{\pgfqpoint{0.000000in}{0.000000in}}%
\pgfpathlineto{\pgfqpoint{0.000000in}{-0.048611in}}%
\pgfusepath{stroke,fill}%
}%
\begin{pgfscope}%
\pgfsys@transformshift{1.378821in}{1.428886in}%
\pgfsys@useobject{currentmarker}{}%
\end{pgfscope}%
\end{pgfscope}%
\begin{pgfscope}%
\pgfsetbuttcap%
\pgfsetroundjoin%
\definecolor{currentfill}{rgb}{0.000000,0.000000,0.000000}%
\pgfsetfillcolor{currentfill}%
\pgfsetlinewidth{0.803000pt}%
\definecolor{currentstroke}{rgb}{0.000000,0.000000,0.000000}%
\pgfsetstrokecolor{currentstroke}%
\pgfsetdash{}{0pt}%
\pgfsys@defobject{currentmarker}{\pgfqpoint{0.000000in}{-0.048611in}}{\pgfqpoint{0.000000in}{0.000000in}}{%
\pgfpathmoveto{\pgfqpoint{0.000000in}{0.000000in}}%
\pgfpathlineto{\pgfqpoint{0.000000in}{-0.048611in}}%
\pgfusepath{stroke,fill}%
}%
\begin{pgfscope}%
\pgfsys@transformshift{1.970327in}{1.428886in}%
\pgfsys@useobject{currentmarker}{}%
\end{pgfscope}%
\end{pgfscope}%
\begin{pgfscope}%
\pgfsetbuttcap%
\pgfsetroundjoin%
\definecolor{currentfill}{rgb}{0.000000,0.000000,0.000000}%
\pgfsetfillcolor{currentfill}%
\pgfsetlinewidth{0.803000pt}%
\definecolor{currentstroke}{rgb}{0.000000,0.000000,0.000000}%
\pgfsetstrokecolor{currentstroke}%
\pgfsetdash{}{0pt}%
\pgfsys@defobject{currentmarker}{\pgfqpoint{0.000000in}{-0.048611in}}{\pgfqpoint{0.000000in}{0.000000in}}{%
\pgfpathmoveto{\pgfqpoint{0.000000in}{0.000000in}}%
\pgfpathlineto{\pgfqpoint{0.000000in}{-0.048611in}}%
\pgfusepath{stroke,fill}%
}%
\begin{pgfscope}%
\pgfsys@transformshift{2.561833in}{1.428886in}%
\pgfsys@useobject{currentmarker}{}%
\end{pgfscope}%
\end{pgfscope}%
\begin{pgfscope}%
\pgfsetbuttcap%
\pgfsetroundjoin%
\definecolor{currentfill}{rgb}{0.000000,0.000000,0.000000}%
\pgfsetfillcolor{currentfill}%
\pgfsetlinewidth{0.803000pt}%
\definecolor{currentstroke}{rgb}{0.000000,0.000000,0.000000}%
\pgfsetstrokecolor{currentstroke}%
\pgfsetdash{}{0pt}%
\pgfsys@defobject{currentmarker}{\pgfqpoint{0.000000in}{-0.048611in}}{\pgfqpoint{0.000000in}{0.000000in}}{%
\pgfpathmoveto{\pgfqpoint{0.000000in}{0.000000in}}%
\pgfpathlineto{\pgfqpoint{0.000000in}{-0.048611in}}%
\pgfusepath{stroke,fill}%
}%
\begin{pgfscope}%
\pgfsys@transformshift{3.153340in}{1.428886in}%
\pgfsys@useobject{currentmarker}{}%
\end{pgfscope}%
\end{pgfscope}%
\begin{pgfscope}%
\pgfsetbuttcap%
\pgfsetroundjoin%
\definecolor{currentfill}{rgb}{0.000000,0.000000,0.000000}%
\pgfsetfillcolor{currentfill}%
\pgfsetlinewidth{0.803000pt}%
\definecolor{currentstroke}{rgb}{0.000000,0.000000,0.000000}%
\pgfsetstrokecolor{currentstroke}%
\pgfsetdash{}{0pt}%
\pgfsys@defobject{currentmarker}{\pgfqpoint{0.000000in}{-0.048611in}}{\pgfqpoint{0.000000in}{0.000000in}}{%
\pgfpathmoveto{\pgfqpoint{0.000000in}{0.000000in}}%
\pgfpathlineto{\pgfqpoint{0.000000in}{-0.048611in}}%
\pgfusepath{stroke,fill}%
}%
\begin{pgfscope}%
\pgfsys@transformshift{3.744846in}{1.428886in}%
\pgfsys@useobject{currentmarker}{}%
\end{pgfscope}%
\end{pgfscope}%
\begin{pgfscope}%
\pgfsetbuttcap%
\pgfsetroundjoin%
\definecolor{currentfill}{rgb}{0.000000,0.000000,0.000000}%
\pgfsetfillcolor{currentfill}%
\pgfsetlinewidth{0.803000pt}%
\definecolor{currentstroke}{rgb}{0.000000,0.000000,0.000000}%
\pgfsetstrokecolor{currentstroke}%
\pgfsetdash{}{0pt}%
\pgfsys@defobject{currentmarker}{\pgfqpoint{-0.048611in}{0.000000in}}{\pgfqpoint{-0.000000in}{0.000000in}}{%
\pgfpathmoveto{\pgfqpoint{-0.000000in}{0.000000in}}%
\pgfpathlineto{\pgfqpoint{-0.048611in}{0.000000in}}%
\pgfusepath{stroke,fill}%
}%
\begin{pgfscope}%
\pgfsys@transformshift{0.550713in}{1.473974in}%
\pgfsys@useobject{currentmarker}{}%
\end{pgfscope}%
\end{pgfscope}%
\begin{pgfscope}%
\definecolor{textcolor}{rgb}{0.000000,0.000000,0.000000}%
\pgfsetstrokecolor{textcolor}%
\pgfsetfillcolor{textcolor}%
\pgftext[x=0.384046in, y=1.425779in, left, base]{\color{textcolor}\rmfamily\fontsize{10.000000}{12.000000}\selectfont \(\displaystyle {0}\)}%
\end{pgfscope}%
\begin{pgfscope}%
\pgfsetbuttcap%
\pgfsetroundjoin%
\definecolor{currentfill}{rgb}{0.000000,0.000000,0.000000}%
\pgfsetfillcolor{currentfill}%
\pgfsetlinewidth{0.803000pt}%
\definecolor{currentstroke}{rgb}{0.000000,0.000000,0.000000}%
\pgfsetstrokecolor{currentstroke}%
\pgfsetdash{}{0pt}%
\pgfsys@defobject{currentmarker}{\pgfqpoint{-0.048611in}{0.000000in}}{\pgfqpoint{-0.000000in}{0.000000in}}{%
\pgfpathmoveto{\pgfqpoint{-0.000000in}{0.000000in}}%
\pgfpathlineto{\pgfqpoint{-0.048611in}{0.000000in}}%
\pgfusepath{stroke,fill}%
}%
\begin{pgfscope}%
\pgfsys@transformshift{0.550713in}{1.871626in}%
\pgfsys@useobject{currentmarker}{}%
\end{pgfscope}%
\end{pgfscope}%
\begin{pgfscope}%
\definecolor{textcolor}{rgb}{0.000000,0.000000,0.000000}%
\pgfsetstrokecolor{textcolor}%
\pgfsetfillcolor{textcolor}%
\pgftext[x=0.314601in, y=1.823432in, left, base]{\color{textcolor}\rmfamily\fontsize{10.000000}{12.000000}\selectfont \(\displaystyle {20}\)}%
\end{pgfscope}%
\begin{pgfscope}%
\pgfsetbuttcap%
\pgfsetroundjoin%
\definecolor{currentfill}{rgb}{0.000000,0.000000,0.000000}%
\pgfsetfillcolor{currentfill}%
\pgfsetlinewidth{0.803000pt}%
\definecolor{currentstroke}{rgb}{0.000000,0.000000,0.000000}%
\pgfsetstrokecolor{currentstroke}%
\pgfsetdash{}{0pt}%
\pgfsys@defobject{currentmarker}{\pgfqpoint{-0.048611in}{0.000000in}}{\pgfqpoint{-0.000000in}{0.000000in}}{%
\pgfpathmoveto{\pgfqpoint{-0.000000in}{0.000000in}}%
\pgfpathlineto{\pgfqpoint{-0.048611in}{0.000000in}}%
\pgfusepath{stroke,fill}%
}%
\begin{pgfscope}%
\pgfsys@transformshift{0.550713in}{2.269279in}%
\pgfsys@useobject{currentmarker}{}%
\end{pgfscope}%
\end{pgfscope}%
\begin{pgfscope}%
\definecolor{textcolor}{rgb}{0.000000,0.000000,0.000000}%
\pgfsetstrokecolor{textcolor}%
\pgfsetfillcolor{textcolor}%
\pgftext[x=0.314601in, y=2.221085in, left, base]{\color{textcolor}\rmfamily\fontsize{10.000000}{12.000000}\selectfont \(\displaystyle {40}\)}%
\end{pgfscope}%
\begin{pgfscope}%
\definecolor{textcolor}{rgb}{0.000000,0.000000,0.000000}%
\pgfsetstrokecolor{textcolor}%
\pgfsetfillcolor{textcolor}%
\pgftext[x=0.259046in,y=1.926972in,,bottom,rotate=90.000000]{\color{textcolor}\rmfamily\fontsize{10.000000}{12.000000}\selectfont Exploration \(\displaystyle \hat{R}_T\)}%
\end{pgfscope}%
\begin{pgfscope}%
\pgfpathrectangle{\pgfqpoint{0.550713in}{1.428886in}}{\pgfqpoint{3.194133in}{0.996173in}}%
\pgfusepath{clip}%
\pgfsetrectcap%
\pgfsetroundjoin%
\pgfsetlinewidth{0.853187pt}%
\definecolor{currentstroke}{rgb}{0.631373,0.062745,0.207843}%
\pgfsetstrokecolor{currentstroke}%
\pgfsetdash{}{0pt}%
\pgfpathmoveto{\pgfqpoint{0.562543in}{1.476236in}}%
\pgfpathlineto{\pgfqpoint{0.574373in}{1.478607in}}%
\pgfpathlineto{\pgfqpoint{0.657184in}{1.480517in}}%
\pgfpathlineto{\pgfqpoint{0.739995in}{1.483791in}}%
\pgfpathlineto{\pgfqpoint{0.787315in}{1.486775in}}%
\pgfpathlineto{\pgfqpoint{0.822805in}{1.488221in}}%
\pgfpathlineto{\pgfqpoint{0.858296in}{1.490083in}}%
\pgfpathlineto{\pgfqpoint{0.917446in}{1.493544in}}%
\pgfpathlineto{\pgfqpoint{0.929277in}{1.496592in}}%
\pgfpathlineto{\pgfqpoint{0.952937in}{1.497740in}}%
\pgfpathlineto{\pgfqpoint{1.000257in}{1.508352in}}%
\pgfpathlineto{\pgfqpoint{1.012087in}{1.512310in}}%
\pgfpathlineto{\pgfqpoint{1.023918in}{1.514179in}}%
\pgfpathlineto{\pgfqpoint{1.059408in}{1.526869in}}%
\pgfpathlineto{\pgfqpoint{1.083068in}{1.532202in}}%
\pgfpathlineto{\pgfqpoint{1.094898in}{1.538246in}}%
\pgfpathlineto{\pgfqpoint{1.118558in}{1.543262in}}%
\pgfpathlineto{\pgfqpoint{1.130389in}{1.548743in}}%
\pgfpathlineto{\pgfqpoint{1.154049in}{1.567062in}}%
\pgfpathlineto{\pgfqpoint{1.165879in}{1.568041in}}%
\pgfpathlineto{\pgfqpoint{1.177709in}{1.571590in}}%
\pgfpathlineto{\pgfqpoint{1.189539in}{1.577230in}}%
\pgfpathlineto{\pgfqpoint{1.201369in}{1.579951in}}%
\pgfpathlineto{\pgfqpoint{1.225030in}{1.590848in}}%
\pgfpathlineto{\pgfqpoint{1.236860in}{1.592174in}}%
\pgfpathlineto{\pgfqpoint{1.260520in}{1.600635in}}%
\pgfpathlineto{\pgfqpoint{1.272350in}{1.601567in}}%
\pgfpathlineto{\pgfqpoint{1.284180in}{1.624620in}}%
\pgfpathlineto{\pgfqpoint{1.296010in}{1.637440in}}%
\pgfpathlineto{\pgfqpoint{1.307840in}{1.640454in}}%
\pgfpathlineto{\pgfqpoint{1.319671in}{1.647004in}}%
\pgfpathlineto{\pgfqpoint{1.331501in}{1.647423in}}%
\pgfpathlineto{\pgfqpoint{1.355161in}{1.650118in}}%
\pgfpathlineto{\pgfqpoint{1.366991in}{1.654655in}}%
\pgfpathlineto{\pgfqpoint{1.390651in}{1.655735in}}%
\pgfpathlineto{\pgfqpoint{1.402481in}{1.656996in}}%
\pgfpathlineto{\pgfqpoint{1.414312in}{1.661211in}}%
\pgfpathlineto{\pgfqpoint{1.437972in}{1.663156in}}%
\pgfpathlineto{\pgfqpoint{1.449802in}{1.670064in}}%
\pgfpathlineto{\pgfqpoint{1.461632in}{1.670771in}}%
\pgfpathlineto{\pgfqpoint{1.473462in}{1.674971in}}%
\pgfpathlineto{\pgfqpoint{1.485292in}{1.687587in}}%
\pgfpathlineto{\pgfqpoint{1.497122in}{1.716699in}}%
\pgfpathlineto{\pgfqpoint{1.508953in}{1.718713in}}%
\pgfpathlineto{\pgfqpoint{1.520783in}{1.722892in}}%
\pgfpathlineto{\pgfqpoint{1.532613in}{1.724707in}}%
\pgfpathlineto{\pgfqpoint{1.544443in}{1.729136in}}%
\pgfpathlineto{\pgfqpoint{1.579933in}{1.732224in}}%
\pgfpathlineto{\pgfqpoint{1.591763in}{1.734521in}}%
\pgfpathlineto{\pgfqpoint{1.603594in}{1.735345in}}%
\pgfpathlineto{\pgfqpoint{1.615424in}{1.740522in}}%
\pgfpathlineto{\pgfqpoint{1.627254in}{1.740683in}}%
\pgfpathlineto{\pgfqpoint{1.639084in}{1.742253in}}%
\pgfpathlineto{\pgfqpoint{1.650914in}{1.742452in}}%
\pgfpathlineto{\pgfqpoint{1.662744in}{1.760535in}}%
\pgfpathlineto{\pgfqpoint{1.674574in}{1.760881in}}%
\pgfpathlineto{\pgfqpoint{1.686404in}{1.771809in}}%
\pgfpathlineto{\pgfqpoint{1.698235in}{1.771920in}}%
\pgfpathlineto{\pgfqpoint{1.710065in}{1.775017in}}%
\pgfpathlineto{\pgfqpoint{1.721895in}{1.791604in}}%
\pgfpathlineto{\pgfqpoint{1.733725in}{1.792276in}}%
\pgfpathlineto{\pgfqpoint{1.745555in}{1.799818in}}%
\pgfpathlineto{\pgfqpoint{1.757385in}{1.803565in}}%
\pgfpathlineto{\pgfqpoint{1.769215in}{1.804706in}}%
\pgfpathlineto{\pgfqpoint{1.781045in}{1.807893in}}%
\pgfpathlineto{\pgfqpoint{1.804706in}{1.810242in}}%
\pgfpathlineto{\pgfqpoint{1.816536in}{1.815326in}}%
\pgfpathlineto{\pgfqpoint{1.828366in}{1.815564in}}%
\pgfpathlineto{\pgfqpoint{1.840196in}{1.837578in}}%
\pgfpathlineto{\pgfqpoint{1.852026in}{1.842083in}}%
\pgfpathlineto{\pgfqpoint{1.899347in}{1.846434in}}%
\pgfpathlineto{\pgfqpoint{1.911177in}{1.853154in}}%
\pgfpathlineto{\pgfqpoint{1.934837in}{1.855792in}}%
\pgfpathlineto{\pgfqpoint{1.946667in}{1.857041in}}%
\pgfpathlineto{\pgfqpoint{1.970327in}{1.864996in}}%
\pgfpathlineto{\pgfqpoint{1.982157in}{1.867931in}}%
\pgfpathlineto{\pgfqpoint{2.041308in}{1.875020in}}%
\pgfpathlineto{\pgfqpoint{2.053138in}{1.879862in}}%
\pgfpathlineto{\pgfqpoint{2.064968in}{1.883102in}}%
\pgfpathlineto{\pgfqpoint{2.088629in}{1.885982in}}%
\pgfpathlineto{\pgfqpoint{2.100459in}{1.891865in}}%
\pgfpathlineto{\pgfqpoint{2.112289in}{1.895473in}}%
\pgfpathlineto{\pgfqpoint{2.135949in}{1.898257in}}%
\pgfpathlineto{\pgfqpoint{2.147779in}{1.919491in}}%
\pgfpathlineto{\pgfqpoint{2.159609in}{1.923278in}}%
\pgfpathlineto{\pgfqpoint{2.183270in}{1.925825in}}%
\pgfpathlineto{\pgfqpoint{2.195100in}{1.926238in}}%
\pgfpathlineto{\pgfqpoint{2.206930in}{1.928813in}}%
\pgfpathlineto{\pgfqpoint{2.218760in}{1.928781in}}%
\pgfpathlineto{\pgfqpoint{2.230590in}{1.931379in}}%
\pgfpathlineto{\pgfqpoint{2.242420in}{1.931792in}}%
\pgfpathlineto{\pgfqpoint{2.254250in}{1.935431in}}%
\pgfpathlineto{\pgfqpoint{2.266080in}{1.956350in}}%
\pgfpathlineto{\pgfqpoint{2.277911in}{1.964551in}}%
\pgfpathlineto{\pgfqpoint{2.289741in}{1.983856in}}%
\pgfpathlineto{\pgfqpoint{2.313401in}{1.985925in}}%
\pgfpathlineto{\pgfqpoint{2.325231in}{1.990643in}}%
\pgfpathlineto{\pgfqpoint{2.337061in}{1.990708in}}%
\pgfpathlineto{\pgfqpoint{2.348891in}{1.993996in}}%
\pgfpathlineto{\pgfqpoint{2.419872in}{1.999551in}}%
\pgfpathlineto{\pgfqpoint{2.443532in}{2.001205in}}%
\pgfpathlineto{\pgfqpoint{2.455362in}{2.001903in}}%
\pgfpathlineto{\pgfqpoint{2.479023in}{2.010211in}}%
\pgfpathlineto{\pgfqpoint{2.490853in}{2.017512in}}%
\pgfpathlineto{\pgfqpoint{2.502683in}{2.021405in}}%
\pgfpathlineto{\pgfqpoint{2.526343in}{2.023269in}}%
\pgfpathlineto{\pgfqpoint{2.538173in}{2.037739in}}%
\pgfpathlineto{\pgfqpoint{2.573664in}{2.037852in}}%
\pgfpathlineto{\pgfqpoint{2.620984in}{2.042384in}}%
\pgfpathlineto{\pgfqpoint{2.644644in}{2.043377in}}%
\pgfpathlineto{\pgfqpoint{2.668305in}{2.044075in}}%
\pgfpathlineto{\pgfqpoint{2.680135in}{2.046281in}}%
\pgfpathlineto{\pgfqpoint{2.691965in}{2.046291in}}%
\pgfpathlineto{\pgfqpoint{2.703795in}{2.055030in}}%
\pgfpathlineto{\pgfqpoint{2.727455in}{2.055686in}}%
\pgfpathlineto{\pgfqpoint{2.739285in}{2.057153in}}%
\pgfpathlineto{\pgfqpoint{2.751115in}{2.065103in}}%
\pgfpathlineto{\pgfqpoint{2.774776in}{2.067230in}}%
\pgfpathlineto{\pgfqpoint{2.786606in}{2.071376in}}%
\pgfpathlineto{\pgfqpoint{2.822096in}{2.073878in}}%
\pgfpathlineto{\pgfqpoint{2.845756in}{2.075405in}}%
\pgfpathlineto{\pgfqpoint{2.857587in}{2.077441in}}%
\pgfpathlineto{\pgfqpoint{2.869417in}{2.085135in}}%
\pgfpathlineto{\pgfqpoint{2.893077in}{2.087898in}}%
\pgfpathlineto{\pgfqpoint{2.928567in}{2.090825in}}%
\pgfpathlineto{\pgfqpoint{2.940397in}{2.093714in}}%
\pgfpathlineto{\pgfqpoint{2.952228in}{2.093926in}}%
\pgfpathlineto{\pgfqpoint{2.964058in}{2.098919in}}%
\pgfpathlineto{\pgfqpoint{2.975888in}{2.099236in}}%
\pgfpathlineto{\pgfqpoint{2.999548in}{2.103087in}}%
\pgfpathlineto{\pgfqpoint{3.011378in}{2.111201in}}%
\pgfpathlineto{\pgfqpoint{3.035038in}{2.114026in}}%
\pgfpathlineto{\pgfqpoint{3.046868in}{2.119123in}}%
\pgfpathlineto{\pgfqpoint{3.082359in}{2.122025in}}%
\pgfpathlineto{\pgfqpoint{3.094189in}{2.130622in}}%
\pgfpathlineto{\pgfqpoint{3.106019in}{2.130912in}}%
\pgfpathlineto{\pgfqpoint{3.117849in}{2.132887in}}%
\pgfpathlineto{\pgfqpoint{3.141509in}{2.133514in}}%
\pgfpathlineto{\pgfqpoint{3.153340in}{2.134887in}}%
\pgfpathlineto{\pgfqpoint{3.165170in}{2.140028in}}%
\pgfpathlineto{\pgfqpoint{3.188830in}{2.143005in}}%
\pgfpathlineto{\pgfqpoint{3.200660in}{2.147428in}}%
\pgfpathlineto{\pgfqpoint{3.212490in}{2.150138in}}%
\pgfpathlineto{\pgfqpoint{3.236150in}{2.151832in}}%
\pgfpathlineto{\pgfqpoint{3.247981in}{2.154397in}}%
\pgfpathlineto{\pgfqpoint{3.259811in}{2.159643in}}%
\pgfpathlineto{\pgfqpoint{3.271641in}{2.171194in}}%
\pgfpathlineto{\pgfqpoint{3.283471in}{2.176151in}}%
\pgfpathlineto{\pgfqpoint{3.307131in}{2.179035in}}%
\pgfpathlineto{\pgfqpoint{3.318961in}{2.182143in}}%
\pgfpathlineto{\pgfqpoint{3.378112in}{2.191138in}}%
\pgfpathlineto{\pgfqpoint{3.389942in}{2.191922in}}%
\pgfpathlineto{\pgfqpoint{3.401772in}{2.197635in}}%
\pgfpathlineto{\pgfqpoint{3.413602in}{2.197768in}}%
\pgfpathlineto{\pgfqpoint{3.425432in}{2.200906in}}%
\pgfpathlineto{\pgfqpoint{3.437263in}{2.209773in}}%
\pgfpathlineto{\pgfqpoint{3.449093in}{2.211836in}}%
\pgfpathlineto{\pgfqpoint{3.460923in}{2.220854in}}%
\pgfpathlineto{\pgfqpoint{3.472753in}{2.226716in}}%
\pgfpathlineto{\pgfqpoint{3.484583in}{2.236028in}}%
\pgfpathlineto{\pgfqpoint{3.496413in}{2.236203in}}%
\pgfpathlineto{\pgfqpoint{3.508243in}{2.248836in}}%
\pgfpathlineto{\pgfqpoint{3.520073in}{2.249901in}}%
\pgfpathlineto{\pgfqpoint{3.531904in}{2.252432in}}%
\pgfpathlineto{\pgfqpoint{3.543734in}{2.252839in}}%
\pgfpathlineto{\pgfqpoint{3.555564in}{2.254432in}}%
\pgfpathlineto{\pgfqpoint{3.567394in}{2.259614in}}%
\pgfpathlineto{\pgfqpoint{3.579224in}{2.261483in}}%
\pgfpathlineto{\pgfqpoint{3.591054in}{2.264943in}}%
\pgfpathlineto{\pgfqpoint{3.602884in}{2.265864in}}%
\pgfpathlineto{\pgfqpoint{3.614714in}{2.271126in}}%
\pgfpathlineto{\pgfqpoint{3.626544in}{2.272126in}}%
\pgfpathlineto{\pgfqpoint{3.638375in}{2.277834in}}%
\pgfpathlineto{\pgfqpoint{3.650205in}{2.279477in}}%
\pgfpathlineto{\pgfqpoint{3.662035in}{2.284106in}}%
\pgfpathlineto{\pgfqpoint{3.673865in}{2.291533in}}%
\pgfpathlineto{\pgfqpoint{3.685695in}{2.293484in}}%
\pgfpathlineto{\pgfqpoint{3.697525in}{2.296762in}}%
\pgfpathlineto{\pgfqpoint{3.709355in}{2.297658in}}%
\pgfpathlineto{\pgfqpoint{3.721185in}{2.311572in}}%
\pgfpathlineto{\pgfqpoint{3.744846in}{2.317902in}}%
\pgfpathlineto{\pgfqpoint{3.744846in}{2.317902in}}%
\pgfusepath{stroke}%
\end{pgfscope}%
\begin{pgfscope}%
\pgfpathrectangle{\pgfqpoint{0.550713in}{1.428886in}}{\pgfqpoint{3.194133in}{0.996173in}}%
\pgfusepath{clip}%
\pgfsetrectcap%
\pgfsetroundjoin%
\pgfsetlinewidth{0.853187pt}%
\definecolor{currentstroke}{rgb}{0.890196,0.000000,0.400000}%
\pgfsetstrokecolor{currentstroke}%
\pgfsetdash{}{0pt}%
\pgfpathmoveto{\pgfqpoint{0.562543in}{1.476236in}}%
\pgfpathlineto{\pgfqpoint{0.574373in}{1.478604in}}%
\pgfpathlineto{\pgfqpoint{0.657184in}{1.480516in}}%
\pgfpathlineto{\pgfqpoint{0.763655in}{1.484931in}}%
\pgfpathlineto{\pgfqpoint{0.787315in}{1.486785in}}%
\pgfpathlineto{\pgfqpoint{0.822805in}{1.488244in}}%
\pgfpathlineto{\pgfqpoint{0.858296in}{1.490106in}}%
\pgfpathlineto{\pgfqpoint{0.917446in}{1.493551in}}%
\pgfpathlineto{\pgfqpoint{0.929277in}{1.496593in}}%
\pgfpathlineto{\pgfqpoint{0.952937in}{1.498050in}}%
\pgfpathlineto{\pgfqpoint{0.964767in}{1.499806in}}%
\pgfpathlineto{\pgfqpoint{1.000257in}{1.509579in}}%
\pgfpathlineto{\pgfqpoint{1.012087in}{1.510899in}}%
\pgfpathlineto{\pgfqpoint{1.035748in}{1.516848in}}%
\pgfpathlineto{\pgfqpoint{1.047578in}{1.523177in}}%
\pgfpathlineto{\pgfqpoint{1.059408in}{1.525764in}}%
\pgfpathlineto{\pgfqpoint{1.094898in}{1.540880in}}%
\pgfpathlineto{\pgfqpoint{1.106728in}{1.541974in}}%
\pgfpathlineto{\pgfqpoint{1.118558in}{1.545594in}}%
\pgfpathlineto{\pgfqpoint{1.130389in}{1.551366in}}%
\pgfpathlineto{\pgfqpoint{1.142219in}{1.553782in}}%
\pgfpathlineto{\pgfqpoint{1.154049in}{1.559214in}}%
\pgfpathlineto{\pgfqpoint{1.165879in}{1.561938in}}%
\pgfpathlineto{\pgfqpoint{1.177709in}{1.573852in}}%
\pgfpathlineto{\pgfqpoint{1.189539in}{1.574983in}}%
\pgfpathlineto{\pgfqpoint{1.201369in}{1.578480in}}%
\pgfpathlineto{\pgfqpoint{1.213199in}{1.593890in}}%
\pgfpathlineto{\pgfqpoint{1.225030in}{1.614430in}}%
\pgfpathlineto{\pgfqpoint{1.260520in}{1.617536in}}%
\pgfpathlineto{\pgfqpoint{1.272350in}{1.630635in}}%
\pgfpathlineto{\pgfqpoint{1.284180in}{1.630663in}}%
\pgfpathlineto{\pgfqpoint{1.296010in}{1.640572in}}%
\pgfpathlineto{\pgfqpoint{1.307840in}{1.644095in}}%
\pgfpathlineto{\pgfqpoint{1.319671in}{1.649553in}}%
\pgfpathlineto{\pgfqpoint{1.331501in}{1.650226in}}%
\pgfpathlineto{\pgfqpoint{1.343331in}{1.652305in}}%
\pgfpathlineto{\pgfqpoint{1.355161in}{1.656228in}}%
\pgfpathlineto{\pgfqpoint{1.366991in}{1.656706in}}%
\pgfpathlineto{\pgfqpoint{1.378821in}{1.668178in}}%
\pgfpathlineto{\pgfqpoint{1.402481in}{1.674605in}}%
\pgfpathlineto{\pgfqpoint{1.414312in}{1.675201in}}%
\pgfpathlineto{\pgfqpoint{1.426142in}{1.690674in}}%
\pgfpathlineto{\pgfqpoint{1.461632in}{1.694482in}}%
\pgfpathlineto{\pgfqpoint{1.473462in}{1.697494in}}%
\pgfpathlineto{\pgfqpoint{1.485292in}{1.703954in}}%
\pgfpathlineto{\pgfqpoint{1.497122in}{1.720428in}}%
\pgfpathlineto{\pgfqpoint{1.508953in}{1.727492in}}%
\pgfpathlineto{\pgfqpoint{1.532613in}{1.729804in}}%
\pgfpathlineto{\pgfqpoint{1.544443in}{1.734037in}}%
\pgfpathlineto{\pgfqpoint{1.556273in}{1.735539in}}%
\pgfpathlineto{\pgfqpoint{1.579933in}{1.744502in}}%
\pgfpathlineto{\pgfqpoint{1.591763in}{1.744822in}}%
\pgfpathlineto{\pgfqpoint{1.603594in}{1.750080in}}%
\pgfpathlineto{\pgfqpoint{1.627254in}{1.751339in}}%
\pgfpathlineto{\pgfqpoint{1.639084in}{1.756253in}}%
\pgfpathlineto{\pgfqpoint{1.650914in}{1.757832in}}%
\pgfpathlineto{\pgfqpoint{1.662744in}{1.762594in}}%
\pgfpathlineto{\pgfqpoint{1.698235in}{1.768010in}}%
\pgfpathlineto{\pgfqpoint{1.710065in}{1.773475in}}%
\pgfpathlineto{\pgfqpoint{1.745555in}{1.776281in}}%
\pgfpathlineto{\pgfqpoint{1.769215in}{1.778916in}}%
\pgfpathlineto{\pgfqpoint{1.781045in}{1.780947in}}%
\pgfpathlineto{\pgfqpoint{1.792875in}{1.788606in}}%
\pgfpathlineto{\pgfqpoint{1.804706in}{1.788985in}}%
\pgfpathlineto{\pgfqpoint{1.840196in}{1.801866in}}%
\pgfpathlineto{\pgfqpoint{1.863856in}{1.808414in}}%
\pgfpathlineto{\pgfqpoint{1.875686in}{1.810199in}}%
\pgfpathlineto{\pgfqpoint{1.887516in}{1.814504in}}%
\pgfpathlineto{\pgfqpoint{1.911177in}{1.816963in}}%
\pgfpathlineto{\pgfqpoint{1.923007in}{1.822328in}}%
\pgfpathlineto{\pgfqpoint{1.934837in}{1.829327in}}%
\pgfpathlineto{\pgfqpoint{1.946667in}{1.852145in}}%
\pgfpathlineto{\pgfqpoint{1.958497in}{1.853975in}}%
\pgfpathlineto{\pgfqpoint{1.970327in}{1.858124in}}%
\pgfpathlineto{\pgfqpoint{1.993988in}{1.860778in}}%
\pgfpathlineto{\pgfqpoint{2.017648in}{1.864923in}}%
\pgfpathlineto{\pgfqpoint{2.029478in}{1.865079in}}%
\pgfpathlineto{\pgfqpoint{2.041308in}{1.869845in}}%
\pgfpathlineto{\pgfqpoint{2.053138in}{1.870858in}}%
\pgfpathlineto{\pgfqpoint{2.076798in}{1.913172in}}%
\pgfpathlineto{\pgfqpoint{2.088629in}{1.935665in}}%
\pgfpathlineto{\pgfqpoint{2.100459in}{1.939252in}}%
\pgfpathlineto{\pgfqpoint{2.112289in}{1.965026in}}%
\pgfpathlineto{\pgfqpoint{2.124119in}{1.966385in}}%
\pgfpathlineto{\pgfqpoint{2.135949in}{1.970543in}}%
\pgfpathlineto{\pgfqpoint{2.159609in}{1.973966in}}%
\pgfpathlineto{\pgfqpoint{2.183270in}{1.977248in}}%
\pgfpathlineto{\pgfqpoint{2.218760in}{1.980158in}}%
\pgfpathlineto{\pgfqpoint{2.230590in}{1.981109in}}%
\pgfpathlineto{\pgfqpoint{2.242420in}{1.988886in}}%
\pgfpathlineto{\pgfqpoint{2.266080in}{1.998063in}}%
\pgfpathlineto{\pgfqpoint{2.289741in}{2.000946in}}%
\pgfpathlineto{\pgfqpoint{2.396212in}{2.012738in}}%
\pgfpathlineto{\pgfqpoint{2.408042in}{2.018572in}}%
\pgfpathlineto{\pgfqpoint{2.443532in}{2.022928in}}%
\pgfpathlineto{\pgfqpoint{2.455362in}{2.038692in}}%
\pgfpathlineto{\pgfqpoint{2.467192in}{2.048965in}}%
\pgfpathlineto{\pgfqpoint{2.490853in}{2.050809in}}%
\pgfpathlineto{\pgfqpoint{2.502683in}{2.055560in}}%
\pgfpathlineto{\pgfqpoint{2.526343in}{2.056488in}}%
\pgfpathlineto{\pgfqpoint{2.538173in}{2.061258in}}%
\pgfpathlineto{\pgfqpoint{2.585494in}{2.065502in}}%
\pgfpathlineto{\pgfqpoint{2.597324in}{2.067184in}}%
\pgfpathlineto{\pgfqpoint{2.609154in}{2.077339in}}%
\pgfpathlineto{\pgfqpoint{2.632814in}{2.079363in}}%
\pgfpathlineto{\pgfqpoint{2.644644in}{2.089058in}}%
\pgfpathlineto{\pgfqpoint{2.656474in}{2.102432in}}%
\pgfpathlineto{\pgfqpoint{2.668305in}{2.105326in}}%
\pgfpathlineto{\pgfqpoint{2.680135in}{2.110246in}}%
\pgfpathlineto{\pgfqpoint{2.703795in}{2.111484in}}%
\pgfpathlineto{\pgfqpoint{2.715625in}{2.114898in}}%
\pgfpathlineto{\pgfqpoint{2.739285in}{2.118324in}}%
\pgfpathlineto{\pgfqpoint{2.751115in}{2.119163in}}%
\pgfpathlineto{\pgfqpoint{2.762946in}{2.128284in}}%
\pgfpathlineto{\pgfqpoint{2.774776in}{2.128867in}}%
\pgfpathlineto{\pgfqpoint{2.786606in}{2.132803in}}%
\pgfpathlineto{\pgfqpoint{2.798436in}{2.132878in}}%
\pgfpathlineto{\pgfqpoint{2.810266in}{2.135039in}}%
\pgfpathlineto{\pgfqpoint{2.822096in}{2.135152in}}%
\pgfpathlineto{\pgfqpoint{2.845756in}{2.143421in}}%
\pgfpathlineto{\pgfqpoint{2.857587in}{2.143730in}}%
\pgfpathlineto{\pgfqpoint{2.869417in}{2.146455in}}%
\pgfpathlineto{\pgfqpoint{2.881247in}{2.147030in}}%
\pgfpathlineto{\pgfqpoint{2.916737in}{2.153688in}}%
\pgfpathlineto{\pgfqpoint{2.928567in}{2.154254in}}%
\pgfpathlineto{\pgfqpoint{2.940397in}{2.159599in}}%
\pgfpathlineto{\pgfqpoint{2.975888in}{2.160640in}}%
\pgfpathlineto{\pgfqpoint{2.987718in}{2.164801in}}%
\pgfpathlineto{\pgfqpoint{3.011378in}{2.167435in}}%
\pgfpathlineto{\pgfqpoint{3.023208in}{2.174505in}}%
\pgfpathlineto{\pgfqpoint{3.035038in}{2.183430in}}%
\pgfpathlineto{\pgfqpoint{3.046868in}{2.196150in}}%
\pgfpathlineto{\pgfqpoint{3.058699in}{2.196665in}}%
\pgfpathlineto{\pgfqpoint{3.070529in}{2.213729in}}%
\pgfpathlineto{\pgfqpoint{3.082359in}{2.213798in}}%
\pgfpathlineto{\pgfqpoint{3.094189in}{2.216575in}}%
\pgfpathlineto{\pgfqpoint{3.106019in}{2.217050in}}%
\pgfpathlineto{\pgfqpoint{3.117849in}{2.218863in}}%
\pgfpathlineto{\pgfqpoint{3.129679in}{2.223149in}}%
\pgfpathlineto{\pgfqpoint{3.153340in}{2.224161in}}%
\pgfpathlineto{\pgfqpoint{3.165170in}{2.224974in}}%
\pgfpathlineto{\pgfqpoint{3.177000in}{2.234695in}}%
\pgfpathlineto{\pgfqpoint{3.188830in}{2.234984in}}%
\pgfpathlineto{\pgfqpoint{3.200660in}{2.239536in}}%
\pgfpathlineto{\pgfqpoint{3.283471in}{2.245588in}}%
\pgfpathlineto{\pgfqpoint{3.295301in}{2.254068in}}%
\pgfpathlineto{\pgfqpoint{3.318961in}{2.255859in}}%
\pgfpathlineto{\pgfqpoint{3.330791in}{2.261095in}}%
\pgfpathlineto{\pgfqpoint{3.366282in}{2.266954in}}%
\pgfpathlineto{\pgfqpoint{3.378112in}{2.271570in}}%
\pgfpathlineto{\pgfqpoint{3.401772in}{2.274660in}}%
\pgfpathlineto{\pgfqpoint{3.413602in}{2.284643in}}%
\pgfpathlineto{\pgfqpoint{3.425432in}{2.285066in}}%
\pgfpathlineto{\pgfqpoint{3.449093in}{2.287944in}}%
\pgfpathlineto{\pgfqpoint{3.460923in}{2.292744in}}%
\pgfpathlineto{\pgfqpoint{3.472753in}{2.299652in}}%
\pgfpathlineto{\pgfqpoint{3.484583in}{2.301977in}}%
\pgfpathlineto{\pgfqpoint{3.508243in}{2.310537in}}%
\pgfpathlineto{\pgfqpoint{3.531904in}{2.311711in}}%
\pgfpathlineto{\pgfqpoint{3.543734in}{2.313157in}}%
\pgfpathlineto{\pgfqpoint{3.555564in}{2.313161in}}%
\pgfpathlineto{\pgfqpoint{3.567394in}{2.324570in}}%
\pgfpathlineto{\pgfqpoint{3.579224in}{2.325831in}}%
\pgfpathlineto{\pgfqpoint{3.591054in}{2.329727in}}%
\pgfpathlineto{\pgfqpoint{3.602884in}{2.339908in}}%
\pgfpathlineto{\pgfqpoint{3.614714in}{2.345979in}}%
\pgfpathlineto{\pgfqpoint{3.626544in}{2.346094in}}%
\pgfpathlineto{\pgfqpoint{3.638375in}{2.348146in}}%
\pgfpathlineto{\pgfqpoint{3.650205in}{2.361157in}}%
\pgfpathlineto{\pgfqpoint{3.673865in}{2.364606in}}%
\pgfpathlineto{\pgfqpoint{3.685695in}{2.366033in}}%
\pgfpathlineto{\pgfqpoint{3.697525in}{2.370510in}}%
\pgfpathlineto{\pgfqpoint{3.709355in}{2.372442in}}%
\pgfpathlineto{\pgfqpoint{3.733016in}{2.373758in}}%
\pgfpathlineto{\pgfqpoint{3.744846in}{2.379778in}}%
\pgfpathlineto{\pgfqpoint{3.744846in}{2.379778in}}%
\pgfusepath{stroke}%
\end{pgfscope}%
\begin{pgfscope}%
\pgfpathrectangle{\pgfqpoint{0.550713in}{1.428886in}}{\pgfqpoint{3.194133in}{0.996173in}}%
\pgfusepath{clip}%
\pgfsetrectcap%
\pgfsetroundjoin%
\pgfsetlinewidth{0.853187pt}%
\definecolor{currentstroke}{rgb}{0.000000,0.329412,0.623529}%
\pgfsetstrokecolor{currentstroke}%
\pgfsetdash{}{0pt}%
\pgfpathmoveto{\pgfqpoint{0.562543in}{1.476869in}}%
\pgfpathlineto{\pgfqpoint{0.574373in}{1.481078in}}%
\pgfpathlineto{\pgfqpoint{0.586203in}{1.483889in}}%
\pgfpathlineto{\pgfqpoint{0.669014in}{1.489145in}}%
\pgfpathlineto{\pgfqpoint{0.763655in}{1.492113in}}%
\pgfpathlineto{\pgfqpoint{0.775485in}{1.492623in}}%
\pgfpathlineto{\pgfqpoint{0.787315in}{1.494980in}}%
\pgfpathlineto{\pgfqpoint{0.834636in}{1.496281in}}%
\pgfpathlineto{\pgfqpoint{0.929277in}{1.501575in}}%
\pgfpathlineto{\pgfqpoint{0.952937in}{1.505132in}}%
\pgfpathlineto{\pgfqpoint{0.976597in}{1.507923in}}%
\pgfpathlineto{\pgfqpoint{1.000257in}{1.513075in}}%
\pgfpathlineto{\pgfqpoint{1.012087in}{1.514833in}}%
\pgfpathlineto{\pgfqpoint{1.035748in}{1.516075in}}%
\pgfpathlineto{\pgfqpoint{1.047578in}{1.518995in}}%
\pgfpathlineto{\pgfqpoint{1.071238in}{1.531309in}}%
\pgfpathlineto{\pgfqpoint{1.083068in}{1.532357in}}%
\pgfpathlineto{\pgfqpoint{1.106728in}{1.547160in}}%
\pgfpathlineto{\pgfqpoint{1.165879in}{1.552843in}}%
\pgfpathlineto{\pgfqpoint{1.177709in}{1.558223in}}%
\pgfpathlineto{\pgfqpoint{1.189539in}{1.560521in}}%
\pgfpathlineto{\pgfqpoint{1.201369in}{1.564890in}}%
\pgfpathlineto{\pgfqpoint{1.213199in}{1.571472in}}%
\pgfpathlineto{\pgfqpoint{1.225030in}{1.573212in}}%
\pgfpathlineto{\pgfqpoint{1.236860in}{1.573334in}}%
\pgfpathlineto{\pgfqpoint{1.248690in}{1.574783in}}%
\pgfpathlineto{\pgfqpoint{1.260520in}{1.578543in}}%
\pgfpathlineto{\pgfqpoint{1.272350in}{1.579440in}}%
\pgfpathlineto{\pgfqpoint{1.284180in}{1.582351in}}%
\pgfpathlineto{\pgfqpoint{1.296010in}{1.583934in}}%
\pgfpathlineto{\pgfqpoint{1.319671in}{1.584928in}}%
\pgfpathlineto{\pgfqpoint{1.343331in}{1.586446in}}%
\pgfpathlineto{\pgfqpoint{1.366991in}{1.588262in}}%
\pgfpathlineto{\pgfqpoint{1.378821in}{1.589848in}}%
\pgfpathlineto{\pgfqpoint{1.390651in}{1.596802in}}%
\pgfpathlineto{\pgfqpoint{1.402481in}{1.597196in}}%
\pgfpathlineto{\pgfqpoint{1.414312in}{1.602358in}}%
\pgfpathlineto{\pgfqpoint{1.520783in}{1.609523in}}%
\pgfpathlineto{\pgfqpoint{1.532613in}{1.611492in}}%
\pgfpathlineto{\pgfqpoint{1.544443in}{1.616225in}}%
\pgfpathlineto{\pgfqpoint{1.556273in}{1.617940in}}%
\pgfpathlineto{\pgfqpoint{1.603594in}{1.619218in}}%
\pgfpathlineto{\pgfqpoint{1.615424in}{1.621216in}}%
\pgfpathlineto{\pgfqpoint{1.639084in}{1.621664in}}%
\pgfpathlineto{\pgfqpoint{1.650914in}{1.640092in}}%
\pgfpathlineto{\pgfqpoint{1.662744in}{1.641945in}}%
\pgfpathlineto{\pgfqpoint{1.674574in}{1.642383in}}%
\pgfpathlineto{\pgfqpoint{1.686404in}{1.645544in}}%
\pgfpathlineto{\pgfqpoint{1.698235in}{1.645790in}}%
\pgfpathlineto{\pgfqpoint{1.710065in}{1.648188in}}%
\pgfpathlineto{\pgfqpoint{1.816536in}{1.656619in}}%
\pgfpathlineto{\pgfqpoint{1.828366in}{1.659781in}}%
\pgfpathlineto{\pgfqpoint{1.840196in}{1.659909in}}%
\pgfpathlineto{\pgfqpoint{1.852026in}{1.679653in}}%
\pgfpathlineto{\pgfqpoint{1.863856in}{1.683214in}}%
\pgfpathlineto{\pgfqpoint{1.875686in}{1.683265in}}%
\pgfpathlineto{\pgfqpoint{1.887516in}{1.685908in}}%
\pgfpathlineto{\pgfqpoint{1.958497in}{1.688166in}}%
\pgfpathlineto{\pgfqpoint{1.970327in}{1.689827in}}%
\pgfpathlineto{\pgfqpoint{1.993988in}{1.689933in}}%
\pgfpathlineto{\pgfqpoint{2.005818in}{1.691385in}}%
\pgfpathlineto{\pgfqpoint{2.041308in}{1.692092in}}%
\pgfpathlineto{\pgfqpoint{2.064968in}{1.693382in}}%
\pgfpathlineto{\pgfqpoint{2.088629in}{1.694595in}}%
\pgfpathlineto{\pgfqpoint{2.124119in}{1.695544in}}%
\pgfpathlineto{\pgfqpoint{2.171439in}{1.698046in}}%
\pgfpathlineto{\pgfqpoint{2.183270in}{1.697998in}}%
\pgfpathlineto{\pgfqpoint{2.195100in}{1.701462in}}%
\pgfpathlineto{\pgfqpoint{2.206930in}{1.702839in}}%
\pgfpathlineto{\pgfqpoint{2.254250in}{1.703170in}}%
\pgfpathlineto{\pgfqpoint{2.266080in}{1.704526in}}%
\pgfpathlineto{\pgfqpoint{2.301571in}{1.705487in}}%
\pgfpathlineto{\pgfqpoint{2.360721in}{1.707418in}}%
\pgfpathlineto{\pgfqpoint{2.431702in}{1.708723in}}%
\pgfpathlineto{\pgfqpoint{2.526343in}{1.711984in}}%
\pgfpathlineto{\pgfqpoint{2.550003in}{1.712880in}}%
\pgfpathlineto{\pgfqpoint{2.609154in}{1.713525in}}%
\pgfpathlineto{\pgfqpoint{2.620984in}{1.724374in}}%
\pgfpathlineto{\pgfqpoint{2.644644in}{1.725479in}}%
\pgfpathlineto{\pgfqpoint{2.691965in}{1.726484in}}%
\pgfpathlineto{\pgfqpoint{2.703795in}{1.728772in}}%
\pgfpathlineto{\pgfqpoint{2.751115in}{1.729591in}}%
\pgfpathlineto{\pgfqpoint{2.798436in}{1.731672in}}%
\pgfpathlineto{\pgfqpoint{2.833926in}{1.732572in}}%
\pgfpathlineto{\pgfqpoint{2.857587in}{1.735137in}}%
\pgfpathlineto{\pgfqpoint{2.940397in}{1.735547in}}%
\pgfpathlineto{\pgfqpoint{2.975888in}{1.737352in}}%
\pgfpathlineto{\pgfqpoint{3.035038in}{1.738559in}}%
\pgfpathlineto{\pgfqpoint{3.070529in}{1.740365in}}%
\pgfpathlineto{\pgfqpoint{3.129679in}{1.740979in}}%
\pgfpathlineto{\pgfqpoint{3.153340in}{1.741834in}}%
\pgfpathlineto{\pgfqpoint{3.177000in}{1.742444in}}%
\pgfpathlineto{\pgfqpoint{3.188830in}{1.744491in}}%
\pgfpathlineto{\pgfqpoint{3.354452in}{1.752317in}}%
\pgfpathlineto{\pgfqpoint{3.366282in}{1.754739in}}%
\pgfpathlineto{\pgfqpoint{3.389942in}{1.755445in}}%
\pgfpathlineto{\pgfqpoint{3.401772in}{1.755967in}}%
\pgfpathlineto{\pgfqpoint{3.413602in}{1.763475in}}%
\pgfpathlineto{\pgfqpoint{3.425432in}{1.765561in}}%
\pgfpathlineto{\pgfqpoint{3.449093in}{1.766411in}}%
\pgfpathlineto{\pgfqpoint{3.520073in}{1.769647in}}%
\pgfpathlineto{\pgfqpoint{3.531904in}{1.770975in}}%
\pgfpathlineto{\pgfqpoint{3.579224in}{1.771591in}}%
\pgfpathlineto{\pgfqpoint{3.602884in}{1.774168in}}%
\pgfpathlineto{\pgfqpoint{3.626544in}{1.782621in}}%
\pgfpathlineto{\pgfqpoint{3.650205in}{1.783767in}}%
\pgfpathlineto{\pgfqpoint{3.662035in}{1.787498in}}%
\pgfpathlineto{\pgfqpoint{3.744846in}{1.792081in}}%
\pgfpathlineto{\pgfqpoint{3.744846in}{1.792081in}}%
\pgfusepath{stroke}%
\end{pgfscope}%
\begin{pgfscope}%
\pgfpathrectangle{\pgfqpoint{0.550713in}{1.428886in}}{\pgfqpoint{3.194133in}{0.996173in}}%
\pgfusepath{clip}%
\pgfsetrectcap%
\pgfsetroundjoin%
\pgfsetlinewidth{0.853187pt}%
\definecolor{currentstroke}{rgb}{0.000000,0.380392,0.396078}%
\pgfsetstrokecolor{currentstroke}%
\pgfsetdash{}{0pt}%
\pgfpathmoveto{\pgfqpoint{0.562543in}{1.476869in}}%
\pgfpathlineto{\pgfqpoint{0.574373in}{1.481097in}}%
\pgfpathlineto{\pgfqpoint{0.598033in}{1.486568in}}%
\pgfpathlineto{\pgfqpoint{0.775485in}{1.493699in}}%
\pgfpathlineto{\pgfqpoint{0.787315in}{1.496171in}}%
\pgfpathlineto{\pgfqpoint{0.822805in}{1.497844in}}%
\pgfpathlineto{\pgfqpoint{0.858296in}{1.499995in}}%
\pgfpathlineto{\pgfqpoint{0.881956in}{1.500236in}}%
\pgfpathlineto{\pgfqpoint{0.893786in}{1.502082in}}%
\pgfpathlineto{\pgfqpoint{0.905616in}{1.502282in}}%
\pgfpathlineto{\pgfqpoint{0.917446in}{1.505023in}}%
\pgfpathlineto{\pgfqpoint{0.929277in}{1.505415in}}%
\pgfpathlineto{\pgfqpoint{0.941107in}{1.509442in}}%
\pgfpathlineto{\pgfqpoint{0.976597in}{1.513393in}}%
\pgfpathlineto{\pgfqpoint{0.988427in}{1.515975in}}%
\pgfpathlineto{\pgfqpoint{1.012087in}{1.518786in}}%
\pgfpathlineto{\pgfqpoint{1.023918in}{1.522842in}}%
\pgfpathlineto{\pgfqpoint{1.035748in}{1.523649in}}%
\pgfpathlineto{\pgfqpoint{1.047578in}{1.528910in}}%
\pgfpathlineto{\pgfqpoint{1.059408in}{1.530634in}}%
\pgfpathlineto{\pgfqpoint{1.071238in}{1.534940in}}%
\pgfpathlineto{\pgfqpoint{1.083068in}{1.542144in}}%
\pgfpathlineto{\pgfqpoint{1.094898in}{1.542272in}}%
\pgfpathlineto{\pgfqpoint{1.106728in}{1.544658in}}%
\pgfpathlineto{\pgfqpoint{1.118558in}{1.544568in}}%
\pgfpathlineto{\pgfqpoint{1.130389in}{1.549933in}}%
\pgfpathlineto{\pgfqpoint{1.142219in}{1.557186in}}%
\pgfpathlineto{\pgfqpoint{1.154049in}{1.562769in}}%
\pgfpathlineto{\pgfqpoint{1.165879in}{1.571152in}}%
\pgfpathlineto{\pgfqpoint{1.177709in}{1.577125in}}%
\pgfpathlineto{\pgfqpoint{1.189539in}{1.578939in}}%
\pgfpathlineto{\pgfqpoint{1.225030in}{1.594441in}}%
\pgfpathlineto{\pgfqpoint{1.236860in}{1.596251in}}%
\pgfpathlineto{\pgfqpoint{1.248690in}{1.604618in}}%
\pgfpathlineto{\pgfqpoint{1.260520in}{1.605836in}}%
\pgfpathlineto{\pgfqpoint{1.272350in}{1.610903in}}%
\pgfpathlineto{\pgfqpoint{1.296010in}{1.612774in}}%
\pgfpathlineto{\pgfqpoint{1.307840in}{1.625169in}}%
\pgfpathlineto{\pgfqpoint{1.331501in}{1.628161in}}%
\pgfpathlineto{\pgfqpoint{1.343331in}{1.629745in}}%
\pgfpathlineto{\pgfqpoint{1.355161in}{1.634165in}}%
\pgfpathlineto{\pgfqpoint{1.366991in}{1.634952in}}%
\pgfpathlineto{\pgfqpoint{1.378821in}{1.647641in}}%
\pgfpathlineto{\pgfqpoint{1.414312in}{1.648391in}}%
\pgfpathlineto{\pgfqpoint{1.437972in}{1.650880in}}%
\pgfpathlineto{\pgfqpoint{1.449802in}{1.655089in}}%
\pgfpathlineto{\pgfqpoint{1.497122in}{1.657756in}}%
\pgfpathlineto{\pgfqpoint{1.520783in}{1.657978in}}%
\pgfpathlineto{\pgfqpoint{1.532613in}{1.660805in}}%
\pgfpathlineto{\pgfqpoint{1.544443in}{1.661302in}}%
\pgfpathlineto{\pgfqpoint{1.556273in}{1.674060in}}%
\pgfpathlineto{\pgfqpoint{1.568103in}{1.678681in}}%
\pgfpathlineto{\pgfqpoint{1.579933in}{1.679017in}}%
\pgfpathlineto{\pgfqpoint{1.591763in}{1.683598in}}%
\pgfpathlineto{\pgfqpoint{1.603594in}{1.698095in}}%
\pgfpathlineto{\pgfqpoint{1.615424in}{1.701840in}}%
\pgfpathlineto{\pgfqpoint{1.639084in}{1.703914in}}%
\pgfpathlineto{\pgfqpoint{1.650914in}{1.707478in}}%
\pgfpathlineto{\pgfqpoint{1.662744in}{1.707617in}}%
\pgfpathlineto{\pgfqpoint{1.674574in}{1.710096in}}%
\pgfpathlineto{\pgfqpoint{1.698235in}{1.711508in}}%
\pgfpathlineto{\pgfqpoint{1.710065in}{1.711951in}}%
\pgfpathlineto{\pgfqpoint{1.721895in}{1.731765in}}%
\pgfpathlineto{\pgfqpoint{1.733725in}{1.732877in}}%
\pgfpathlineto{\pgfqpoint{1.745555in}{1.738141in}}%
\pgfpathlineto{\pgfqpoint{1.769215in}{1.739310in}}%
\pgfpathlineto{\pgfqpoint{1.863856in}{1.748237in}}%
\pgfpathlineto{\pgfqpoint{1.911177in}{1.750677in}}%
\pgfpathlineto{\pgfqpoint{1.923007in}{1.754670in}}%
\pgfpathlineto{\pgfqpoint{1.934837in}{1.754883in}}%
\pgfpathlineto{\pgfqpoint{1.958497in}{1.756802in}}%
\pgfpathlineto{\pgfqpoint{2.005818in}{1.760029in}}%
\pgfpathlineto{\pgfqpoint{2.017648in}{1.761789in}}%
\pgfpathlineto{\pgfqpoint{2.064968in}{1.763067in}}%
\pgfpathlineto{\pgfqpoint{2.088629in}{1.765919in}}%
\pgfpathlineto{\pgfqpoint{2.112289in}{1.772012in}}%
\pgfpathlineto{\pgfqpoint{2.159609in}{1.775163in}}%
\pgfpathlineto{\pgfqpoint{2.171439in}{1.778840in}}%
\pgfpathlineto{\pgfqpoint{2.242420in}{1.781361in}}%
\pgfpathlineto{\pgfqpoint{2.289741in}{1.784296in}}%
\pgfpathlineto{\pgfqpoint{2.301571in}{1.784406in}}%
\pgfpathlineto{\pgfqpoint{2.313401in}{1.787784in}}%
\pgfpathlineto{\pgfqpoint{2.360721in}{1.788398in}}%
\pgfpathlineto{\pgfqpoint{2.372551in}{1.790351in}}%
\pgfpathlineto{\pgfqpoint{2.455362in}{1.792205in}}%
\pgfpathlineto{\pgfqpoint{2.479023in}{1.793092in}}%
\pgfpathlineto{\pgfqpoint{2.490853in}{1.797079in}}%
\pgfpathlineto{\pgfqpoint{2.514513in}{1.797883in}}%
\pgfpathlineto{\pgfqpoint{2.526343in}{1.799135in}}%
\pgfpathlineto{\pgfqpoint{2.573664in}{1.799506in}}%
\pgfpathlineto{\pgfqpoint{2.632814in}{1.803045in}}%
\pgfpathlineto{\pgfqpoint{2.668305in}{1.804256in}}%
\pgfpathlineto{\pgfqpoint{2.739285in}{1.807754in}}%
\pgfpathlineto{\pgfqpoint{2.762946in}{1.809470in}}%
\pgfpathlineto{\pgfqpoint{2.833926in}{1.811711in}}%
\pgfpathlineto{\pgfqpoint{2.881247in}{1.812770in}}%
\pgfpathlineto{\pgfqpoint{2.904907in}{1.814101in}}%
\pgfpathlineto{\pgfqpoint{2.916737in}{1.816113in}}%
\pgfpathlineto{\pgfqpoint{2.975888in}{1.818747in}}%
\pgfpathlineto{\pgfqpoint{2.987718in}{1.819894in}}%
\pgfpathlineto{\pgfqpoint{2.999548in}{1.819802in}}%
\pgfpathlineto{\pgfqpoint{3.046868in}{1.824152in}}%
\pgfpathlineto{\pgfqpoint{3.094189in}{1.824824in}}%
\pgfpathlineto{\pgfqpoint{3.117849in}{1.826436in}}%
\pgfpathlineto{\pgfqpoint{3.129679in}{1.826587in}}%
\pgfpathlineto{\pgfqpoint{3.188830in}{1.832497in}}%
\pgfpathlineto{\pgfqpoint{3.200660in}{1.832549in}}%
\pgfpathlineto{\pgfqpoint{3.212490in}{1.833739in}}%
\pgfpathlineto{\pgfqpoint{3.247981in}{1.834373in}}%
\pgfpathlineto{\pgfqpoint{3.259811in}{1.835041in}}%
\pgfpathlineto{\pgfqpoint{3.271641in}{1.837384in}}%
\pgfpathlineto{\pgfqpoint{3.354452in}{1.841232in}}%
\pgfpathlineto{\pgfqpoint{3.389942in}{1.844048in}}%
\pgfpathlineto{\pgfqpoint{3.449093in}{1.848741in}}%
\pgfpathlineto{\pgfqpoint{3.460923in}{1.853929in}}%
\pgfpathlineto{\pgfqpoint{3.520073in}{1.856635in}}%
\pgfpathlineto{\pgfqpoint{3.531904in}{1.859236in}}%
\pgfpathlineto{\pgfqpoint{3.543734in}{1.859589in}}%
\pgfpathlineto{\pgfqpoint{3.555564in}{1.864851in}}%
\pgfpathlineto{\pgfqpoint{3.567394in}{1.865039in}}%
\pgfpathlineto{\pgfqpoint{3.579224in}{1.868831in}}%
\pgfpathlineto{\pgfqpoint{3.591054in}{1.871263in}}%
\pgfpathlineto{\pgfqpoint{3.614714in}{1.874031in}}%
\pgfpathlineto{\pgfqpoint{3.626544in}{1.878978in}}%
\pgfpathlineto{\pgfqpoint{3.638375in}{1.879245in}}%
\pgfpathlineto{\pgfqpoint{3.650205in}{1.884204in}}%
\pgfpathlineto{\pgfqpoint{3.662035in}{1.885335in}}%
\pgfpathlineto{\pgfqpoint{3.673865in}{1.889612in}}%
\pgfpathlineto{\pgfqpoint{3.685695in}{1.891712in}}%
\pgfpathlineto{\pgfqpoint{3.697525in}{1.896400in}}%
\pgfpathlineto{\pgfqpoint{3.709355in}{1.897764in}}%
\pgfpathlineto{\pgfqpoint{3.721185in}{1.900594in}}%
\pgfpathlineto{\pgfqpoint{3.733016in}{1.901515in}}%
\pgfpathlineto{\pgfqpoint{3.744846in}{1.906038in}}%
\pgfpathlineto{\pgfqpoint{3.744846in}{1.906038in}}%
\pgfusepath{stroke}%
\end{pgfscope}%
\begin{pgfscope}%
\pgfpathrectangle{\pgfqpoint{0.550713in}{1.428886in}}{\pgfqpoint{3.194133in}{0.996173in}}%
\pgfusepath{clip}%
\pgfsetrectcap%
\pgfsetroundjoin%
\pgfsetlinewidth{0.853187pt}%
\definecolor{currentstroke}{rgb}{0.380392,0.129412,0.345098}%
\pgfsetstrokecolor{currentstroke}%
\pgfsetdash{}{0pt}%
\pgfpathmoveto{\pgfqpoint{0.562543in}{1.474166in}}%
\pgfpathlineto{\pgfqpoint{0.621693in}{1.475141in}}%
\pgfpathlineto{\pgfqpoint{0.905616in}{1.479488in}}%
\pgfpathlineto{\pgfqpoint{1.047578in}{1.482827in}}%
\pgfpathlineto{\pgfqpoint{1.059408in}{1.480787in}}%
\pgfpathlineto{\pgfqpoint{1.071238in}{1.481048in}}%
\pgfpathlineto{\pgfqpoint{1.083068in}{1.479475in}}%
\pgfpathlineto{\pgfqpoint{1.094898in}{1.480589in}}%
\pgfpathlineto{\pgfqpoint{1.106728in}{1.474873in}}%
\pgfpathlineto{\pgfqpoint{1.130389in}{1.477718in}}%
\pgfpathlineto{\pgfqpoint{1.154049in}{1.480137in}}%
\pgfpathlineto{\pgfqpoint{1.165879in}{1.482823in}}%
\pgfpathlineto{\pgfqpoint{1.236860in}{1.490288in}}%
\pgfpathlineto{\pgfqpoint{1.248690in}{1.493470in}}%
\pgfpathlineto{\pgfqpoint{1.272350in}{1.496975in}}%
\pgfpathlineto{\pgfqpoint{1.284180in}{1.500355in}}%
\pgfpathlineto{\pgfqpoint{1.296010in}{1.502055in}}%
\pgfpathlineto{\pgfqpoint{1.307840in}{1.502524in}}%
\pgfpathlineto{\pgfqpoint{1.319671in}{1.504968in}}%
\pgfpathlineto{\pgfqpoint{1.331501in}{1.505722in}}%
\pgfpathlineto{\pgfqpoint{1.343331in}{1.507884in}}%
\pgfpathlineto{\pgfqpoint{1.366991in}{1.509738in}}%
\pgfpathlineto{\pgfqpoint{1.414312in}{1.514371in}}%
\pgfpathlineto{\pgfqpoint{1.426142in}{1.521774in}}%
\pgfpathlineto{\pgfqpoint{1.449802in}{1.525407in}}%
\pgfpathlineto{\pgfqpoint{1.461632in}{1.526103in}}%
\pgfpathlineto{\pgfqpoint{1.473462in}{1.528654in}}%
\pgfpathlineto{\pgfqpoint{1.532613in}{1.531236in}}%
\pgfpathlineto{\pgfqpoint{1.556273in}{1.535673in}}%
\pgfpathlineto{\pgfqpoint{1.591763in}{1.538597in}}%
\pgfpathlineto{\pgfqpoint{1.603594in}{1.540142in}}%
\pgfpathlineto{\pgfqpoint{1.615424in}{1.540284in}}%
\pgfpathlineto{\pgfqpoint{1.627254in}{1.542293in}}%
\pgfpathlineto{\pgfqpoint{1.639084in}{1.542673in}}%
\pgfpathlineto{\pgfqpoint{1.650914in}{1.544579in}}%
\pgfpathlineto{\pgfqpoint{1.686404in}{1.546054in}}%
\pgfpathlineto{\pgfqpoint{1.698235in}{1.548118in}}%
\pgfpathlineto{\pgfqpoint{1.710065in}{1.548371in}}%
\pgfpathlineto{\pgfqpoint{1.733725in}{1.550844in}}%
\pgfpathlineto{\pgfqpoint{1.769215in}{1.552993in}}%
\pgfpathlineto{\pgfqpoint{1.781045in}{1.554768in}}%
\pgfpathlineto{\pgfqpoint{1.816536in}{1.557369in}}%
\pgfpathlineto{\pgfqpoint{1.852026in}{1.560096in}}%
\pgfpathlineto{\pgfqpoint{1.863856in}{1.561906in}}%
\pgfpathlineto{\pgfqpoint{1.899347in}{1.564594in}}%
\pgfpathlineto{\pgfqpoint{1.934837in}{1.569645in}}%
\pgfpathlineto{\pgfqpoint{1.970327in}{1.570689in}}%
\pgfpathlineto{\pgfqpoint{1.993988in}{1.572913in}}%
\pgfpathlineto{\pgfqpoint{2.005818in}{1.573296in}}%
\pgfpathlineto{\pgfqpoint{2.017648in}{1.579942in}}%
\pgfpathlineto{\pgfqpoint{2.029478in}{1.580028in}}%
\pgfpathlineto{\pgfqpoint{2.041308in}{1.583448in}}%
\pgfpathlineto{\pgfqpoint{2.064968in}{1.584196in}}%
\pgfpathlineto{\pgfqpoint{2.112289in}{1.586761in}}%
\pgfpathlineto{\pgfqpoint{2.159609in}{1.587713in}}%
\pgfpathlineto{\pgfqpoint{2.171439in}{1.589661in}}%
\pgfpathlineto{\pgfqpoint{2.206930in}{1.591147in}}%
\pgfpathlineto{\pgfqpoint{2.230590in}{1.592631in}}%
\pgfpathlineto{\pgfqpoint{2.266080in}{1.594397in}}%
\pgfpathlineto{\pgfqpoint{2.325231in}{1.597087in}}%
\pgfpathlineto{\pgfqpoint{2.348891in}{1.598920in}}%
\pgfpathlineto{\pgfqpoint{2.372551in}{1.603932in}}%
\pgfpathlineto{\pgfqpoint{2.396212in}{1.607121in}}%
\pgfpathlineto{\pgfqpoint{2.408042in}{1.606934in}}%
\pgfpathlineto{\pgfqpoint{2.419872in}{1.610115in}}%
\pgfpathlineto{\pgfqpoint{2.431702in}{1.610099in}}%
\pgfpathlineto{\pgfqpoint{2.443532in}{1.611361in}}%
\pgfpathlineto{\pgfqpoint{2.455362in}{1.611411in}}%
\pgfpathlineto{\pgfqpoint{2.467192in}{1.613352in}}%
\pgfpathlineto{\pgfqpoint{2.502683in}{1.616441in}}%
\pgfpathlineto{\pgfqpoint{2.526343in}{1.617843in}}%
\pgfpathlineto{\pgfqpoint{2.632814in}{1.623337in}}%
\pgfpathlineto{\pgfqpoint{2.644644in}{1.625748in}}%
\pgfpathlineto{\pgfqpoint{2.691965in}{1.629635in}}%
\pgfpathlineto{\pgfqpoint{2.715625in}{1.631532in}}%
\pgfpathlineto{\pgfqpoint{2.751115in}{1.633123in}}%
\pgfpathlineto{\pgfqpoint{2.762946in}{1.637220in}}%
\pgfpathlineto{\pgfqpoint{2.774776in}{1.637220in}}%
\pgfpathlineto{\pgfqpoint{2.786606in}{1.641654in}}%
\pgfpathlineto{\pgfqpoint{2.798436in}{1.641968in}}%
\pgfpathlineto{\pgfqpoint{2.810266in}{1.644415in}}%
\pgfpathlineto{\pgfqpoint{2.845756in}{1.644874in}}%
\pgfpathlineto{\pgfqpoint{2.857587in}{1.646534in}}%
\pgfpathlineto{\pgfqpoint{2.893077in}{1.647519in}}%
\pgfpathlineto{\pgfqpoint{3.046868in}{1.653365in}}%
\pgfpathlineto{\pgfqpoint{3.070529in}{1.655700in}}%
\pgfpathlineto{\pgfqpoint{3.106019in}{1.657950in}}%
\pgfpathlineto{\pgfqpoint{3.117849in}{1.661294in}}%
\pgfpathlineto{\pgfqpoint{3.129679in}{1.661435in}}%
\pgfpathlineto{\pgfqpoint{3.141509in}{1.663270in}}%
\pgfpathlineto{\pgfqpoint{3.200660in}{1.666006in}}%
\pgfpathlineto{\pgfqpoint{3.212490in}{1.667560in}}%
\pgfpathlineto{\pgfqpoint{3.259811in}{1.669989in}}%
\pgfpathlineto{\pgfqpoint{3.283471in}{1.673061in}}%
\pgfpathlineto{\pgfqpoint{3.307131in}{1.675433in}}%
\pgfpathlineto{\pgfqpoint{3.366282in}{1.677525in}}%
\pgfpathlineto{\pgfqpoint{3.378112in}{1.679497in}}%
\pgfpathlineto{\pgfqpoint{3.389942in}{1.679759in}}%
\pgfpathlineto{\pgfqpoint{3.460923in}{1.688873in}}%
\pgfpathlineto{\pgfqpoint{3.508243in}{1.692342in}}%
\pgfpathlineto{\pgfqpoint{3.531904in}{1.693936in}}%
\pgfpathlineto{\pgfqpoint{3.614714in}{1.699522in}}%
\pgfpathlineto{\pgfqpoint{3.638375in}{1.701773in}}%
\pgfpathlineto{\pgfqpoint{3.650205in}{1.701752in}}%
\pgfpathlineto{\pgfqpoint{3.662035in}{1.704847in}}%
\pgfpathlineto{\pgfqpoint{3.673865in}{1.705198in}}%
\pgfpathlineto{\pgfqpoint{3.685695in}{1.708263in}}%
\pgfpathlineto{\pgfqpoint{3.697525in}{1.709024in}}%
\pgfpathlineto{\pgfqpoint{3.709355in}{1.711276in}}%
\pgfpathlineto{\pgfqpoint{3.733016in}{1.713136in}}%
\pgfpathlineto{\pgfqpoint{3.744846in}{1.714391in}}%
\pgfpathlineto{\pgfqpoint{3.744846in}{1.714391in}}%
\pgfusepath{stroke}%
\end{pgfscope}%
\begin{pgfscope}%
\pgfpathrectangle{\pgfqpoint{0.550713in}{1.428886in}}{\pgfqpoint{3.194133in}{0.996173in}}%
\pgfusepath{clip}%
\pgfsetrectcap%
\pgfsetroundjoin%
\pgfsetlinewidth{0.853187pt}%
\definecolor{currentstroke}{rgb}{0.964706,0.658824,0.000000}%
\pgfsetstrokecolor{currentstroke}%
\pgfsetdash{}{0pt}%
\pgfpathmoveto{\pgfqpoint{0.562543in}{1.474168in}}%
\pgfpathlineto{\pgfqpoint{0.633523in}{1.475275in}}%
\pgfpathlineto{\pgfqpoint{0.976597in}{1.483410in}}%
\pgfpathlineto{\pgfqpoint{1.012087in}{1.484484in}}%
\pgfpathlineto{\pgfqpoint{1.047578in}{1.485018in}}%
\pgfpathlineto{\pgfqpoint{1.059408in}{1.487799in}}%
\pgfpathlineto{\pgfqpoint{1.071238in}{1.484212in}}%
\pgfpathlineto{\pgfqpoint{1.094898in}{1.486188in}}%
\pgfpathlineto{\pgfqpoint{1.106728in}{1.484068in}}%
\pgfpathlineto{\pgfqpoint{1.130389in}{1.485555in}}%
\pgfpathlineto{\pgfqpoint{1.154049in}{1.488726in}}%
\pgfpathlineto{\pgfqpoint{1.201369in}{1.494125in}}%
\pgfpathlineto{\pgfqpoint{1.225030in}{1.497763in}}%
\pgfpathlineto{\pgfqpoint{1.236860in}{1.505355in}}%
\pgfpathlineto{\pgfqpoint{1.248690in}{1.506364in}}%
\pgfpathlineto{\pgfqpoint{1.260520in}{1.509265in}}%
\pgfpathlineto{\pgfqpoint{1.272350in}{1.510007in}}%
\pgfpathlineto{\pgfqpoint{1.284180in}{1.519701in}}%
\pgfpathlineto{\pgfqpoint{1.307840in}{1.522882in}}%
\pgfpathlineto{\pgfqpoint{1.319671in}{1.523641in}}%
\pgfpathlineto{\pgfqpoint{1.331501in}{1.528988in}}%
\pgfpathlineto{\pgfqpoint{1.343331in}{1.529455in}}%
\pgfpathlineto{\pgfqpoint{1.355161in}{1.531279in}}%
\pgfpathlineto{\pgfqpoint{1.414312in}{1.534884in}}%
\pgfpathlineto{\pgfqpoint{1.485292in}{1.542704in}}%
\pgfpathlineto{\pgfqpoint{1.520783in}{1.544990in}}%
\pgfpathlineto{\pgfqpoint{1.532613in}{1.547870in}}%
\pgfpathlineto{\pgfqpoint{1.556273in}{1.548591in}}%
\pgfpathlineto{\pgfqpoint{1.591763in}{1.553047in}}%
\pgfpathlineto{\pgfqpoint{1.627254in}{1.554740in}}%
\pgfpathlineto{\pgfqpoint{1.710065in}{1.560601in}}%
\pgfpathlineto{\pgfqpoint{1.816536in}{1.567095in}}%
\pgfpathlineto{\pgfqpoint{1.828366in}{1.568945in}}%
\pgfpathlineto{\pgfqpoint{1.852026in}{1.570495in}}%
\pgfpathlineto{\pgfqpoint{1.863856in}{1.572818in}}%
\pgfpathlineto{\pgfqpoint{1.875686in}{1.572976in}}%
\pgfpathlineto{\pgfqpoint{1.899347in}{1.574712in}}%
\pgfpathlineto{\pgfqpoint{1.911177in}{1.574896in}}%
\pgfpathlineto{\pgfqpoint{1.923007in}{1.576649in}}%
\pgfpathlineto{\pgfqpoint{1.946667in}{1.577219in}}%
\pgfpathlineto{\pgfqpoint{1.958497in}{1.577638in}}%
\pgfpathlineto{\pgfqpoint{1.970327in}{1.579269in}}%
\pgfpathlineto{\pgfqpoint{1.993988in}{1.580260in}}%
\pgfpathlineto{\pgfqpoint{2.041308in}{1.584346in}}%
\pgfpathlineto{\pgfqpoint{2.053138in}{1.584531in}}%
\pgfpathlineto{\pgfqpoint{2.064968in}{1.586488in}}%
\pgfpathlineto{\pgfqpoint{2.147779in}{1.591726in}}%
\pgfpathlineto{\pgfqpoint{2.159609in}{1.593567in}}%
\pgfpathlineto{\pgfqpoint{2.183270in}{1.595363in}}%
\pgfpathlineto{\pgfqpoint{2.206930in}{1.599720in}}%
\pgfpathlineto{\pgfqpoint{2.218760in}{1.601651in}}%
\pgfpathlineto{\pgfqpoint{2.254250in}{1.602572in}}%
\pgfpathlineto{\pgfqpoint{2.325231in}{1.607077in}}%
\pgfpathlineto{\pgfqpoint{2.337061in}{1.610738in}}%
\pgfpathlineto{\pgfqpoint{2.431702in}{1.615248in}}%
\pgfpathlineto{\pgfqpoint{2.443532in}{1.617175in}}%
\pgfpathlineto{\pgfqpoint{2.467192in}{1.618574in}}%
\pgfpathlineto{\pgfqpoint{2.502683in}{1.620552in}}%
\pgfpathlineto{\pgfqpoint{2.514513in}{1.624048in}}%
\pgfpathlineto{\pgfqpoint{2.550003in}{1.627756in}}%
\pgfpathlineto{\pgfqpoint{2.573664in}{1.629256in}}%
\pgfpathlineto{\pgfqpoint{2.597324in}{1.630943in}}%
\pgfpathlineto{\pgfqpoint{2.609154in}{1.633133in}}%
\pgfpathlineto{\pgfqpoint{2.680135in}{1.636400in}}%
\pgfpathlineto{\pgfqpoint{2.703795in}{1.637415in}}%
\pgfpathlineto{\pgfqpoint{2.715625in}{1.641961in}}%
\pgfpathlineto{\pgfqpoint{2.869417in}{1.650081in}}%
\pgfpathlineto{\pgfqpoint{2.916737in}{1.652714in}}%
\pgfpathlineto{\pgfqpoint{2.952228in}{1.654428in}}%
\pgfpathlineto{\pgfqpoint{2.975888in}{1.655138in}}%
\pgfpathlineto{\pgfqpoint{3.011378in}{1.660458in}}%
\pgfpathlineto{\pgfqpoint{3.366282in}{1.681409in}}%
\pgfpathlineto{\pgfqpoint{3.579224in}{1.696056in}}%
\pgfpathlineto{\pgfqpoint{3.591054in}{1.698010in}}%
\pgfpathlineto{\pgfqpoint{3.614714in}{1.700442in}}%
\pgfpathlineto{\pgfqpoint{3.626544in}{1.703430in}}%
\pgfpathlineto{\pgfqpoint{3.673865in}{1.706673in}}%
\pgfpathlineto{\pgfqpoint{3.685695in}{1.709454in}}%
\pgfpathlineto{\pgfqpoint{3.697525in}{1.709766in}}%
\pgfpathlineto{\pgfqpoint{3.709355in}{1.712352in}}%
\pgfpathlineto{\pgfqpoint{3.744846in}{1.715760in}}%
\pgfpathlineto{\pgfqpoint{3.744846in}{1.715760in}}%
\pgfusepath{stroke}%
\end{pgfscope}%
\begin{pgfscope}%
\pgfpathrectangle{\pgfqpoint{0.550713in}{1.428886in}}{\pgfqpoint{3.194133in}{0.996173in}}%
\pgfusepath{clip}%
\pgfsetrectcap%
\pgfsetroundjoin%
\pgfsetlinewidth{0.853187pt}%
\definecolor{currentstroke}{rgb}{0.341176,0.670588,0.152941}%
\pgfsetstrokecolor{currentstroke}%
\pgfsetdash{}{0pt}%
\pgfpathmoveto{\pgfqpoint{0.562543in}{1.474175in}}%
\pgfpathlineto{\pgfqpoint{0.598033in}{1.475222in}}%
\pgfpathlineto{\pgfqpoint{0.692674in}{1.477202in}}%
\pgfpathlineto{\pgfqpoint{0.929277in}{1.482700in}}%
\pgfpathlineto{\pgfqpoint{0.952937in}{1.482882in}}%
\pgfpathlineto{\pgfqpoint{0.964767in}{1.481123in}}%
\pgfpathlineto{\pgfqpoint{0.988427in}{1.484841in}}%
\pgfpathlineto{\pgfqpoint{1.035748in}{1.487752in}}%
\pgfpathlineto{\pgfqpoint{1.047578in}{1.486095in}}%
\pgfpathlineto{\pgfqpoint{1.059408in}{1.486445in}}%
\pgfpathlineto{\pgfqpoint{1.071238in}{1.488538in}}%
\pgfpathlineto{\pgfqpoint{1.142219in}{1.494206in}}%
\pgfpathlineto{\pgfqpoint{1.154049in}{1.497309in}}%
\pgfpathlineto{\pgfqpoint{1.165879in}{1.498401in}}%
\pgfpathlineto{\pgfqpoint{1.189539in}{1.502810in}}%
\pgfpathlineto{\pgfqpoint{1.201369in}{1.503955in}}%
\pgfpathlineto{\pgfqpoint{1.213199in}{1.510282in}}%
\pgfpathlineto{\pgfqpoint{1.225030in}{1.511299in}}%
\pgfpathlineto{\pgfqpoint{1.236860in}{1.522274in}}%
\pgfpathlineto{\pgfqpoint{1.248690in}{1.522729in}}%
\pgfpathlineto{\pgfqpoint{1.260520in}{1.526013in}}%
\pgfpathlineto{\pgfqpoint{1.284180in}{1.528163in}}%
\pgfpathlineto{\pgfqpoint{1.296010in}{1.531406in}}%
\pgfpathlineto{\pgfqpoint{1.307840in}{1.531649in}}%
\pgfpathlineto{\pgfqpoint{1.319671in}{1.536231in}}%
\pgfpathlineto{\pgfqpoint{1.366991in}{1.538623in}}%
\pgfpathlineto{\pgfqpoint{1.378821in}{1.540632in}}%
\pgfpathlineto{\pgfqpoint{1.437972in}{1.543915in}}%
\pgfpathlineto{\pgfqpoint{1.449802in}{1.547815in}}%
\pgfpathlineto{\pgfqpoint{1.520783in}{1.551038in}}%
\pgfpathlineto{\pgfqpoint{1.556273in}{1.552421in}}%
\pgfpathlineto{\pgfqpoint{1.650914in}{1.557012in}}%
\pgfpathlineto{\pgfqpoint{1.686404in}{1.558946in}}%
\pgfpathlineto{\pgfqpoint{1.721895in}{1.563524in}}%
\pgfpathlineto{\pgfqpoint{1.769215in}{1.564266in}}%
\pgfpathlineto{\pgfqpoint{1.816536in}{1.566685in}}%
\pgfpathlineto{\pgfqpoint{1.863856in}{1.568063in}}%
\pgfpathlineto{\pgfqpoint{2.064968in}{1.578699in}}%
\pgfpathlineto{\pgfqpoint{2.620984in}{1.594223in}}%
\pgfpathlineto{\pgfqpoint{2.975888in}{1.601449in}}%
\pgfpathlineto{\pgfqpoint{3.058699in}{1.603807in}}%
\pgfpathlineto{\pgfqpoint{3.070529in}{1.605722in}}%
\pgfpathlineto{\pgfqpoint{3.567394in}{1.620613in}}%
\pgfpathlineto{\pgfqpoint{3.579224in}{1.622835in}}%
\pgfpathlineto{\pgfqpoint{3.650205in}{1.624557in}}%
\pgfpathlineto{\pgfqpoint{3.709355in}{1.626288in}}%
\pgfpathlineto{\pgfqpoint{3.744846in}{1.627184in}}%
\pgfpathlineto{\pgfqpoint{3.744846in}{1.627184in}}%
\pgfusepath{stroke}%
\end{pgfscope}%
\begin{pgfscope}%
\pgfpathrectangle{\pgfqpoint{0.550713in}{1.428886in}}{\pgfqpoint{3.194133in}{0.996173in}}%
\pgfusepath{clip}%
\pgfsetrectcap%
\pgfsetroundjoin%
\pgfsetlinewidth{0.853187pt}%
\definecolor{currentstroke}{rgb}{0.478431,0.435294,0.674510}%
\pgfsetstrokecolor{currentstroke}%
\pgfsetdash{}{0pt}%
\pgfpathmoveto{\pgfqpoint{0.562543in}{1.474182in}}%
\pgfpathlineto{\pgfqpoint{0.598033in}{1.474994in}}%
\pgfpathlineto{\pgfqpoint{0.633523in}{1.475637in}}%
\pgfpathlineto{\pgfqpoint{0.787315in}{1.478592in}}%
\pgfpathlineto{\pgfqpoint{0.810975in}{1.479541in}}%
\pgfpathlineto{\pgfqpoint{0.905616in}{1.481441in}}%
\pgfpathlineto{\pgfqpoint{0.964767in}{1.483845in}}%
\pgfpathlineto{\pgfqpoint{1.047578in}{1.489265in}}%
\pgfpathlineto{\pgfqpoint{1.059408in}{1.491172in}}%
\pgfpathlineto{\pgfqpoint{1.071238in}{1.489526in}}%
\pgfpathlineto{\pgfqpoint{1.083068in}{1.494243in}}%
\pgfpathlineto{\pgfqpoint{1.106728in}{1.495596in}}%
\pgfpathlineto{\pgfqpoint{1.130389in}{1.501470in}}%
\pgfpathlineto{\pgfqpoint{1.142219in}{1.502063in}}%
\pgfpathlineto{\pgfqpoint{1.154049in}{1.504592in}}%
\pgfpathlineto{\pgfqpoint{1.213199in}{1.508148in}}%
\pgfpathlineto{\pgfqpoint{1.236860in}{1.510212in}}%
\pgfpathlineto{\pgfqpoint{1.260520in}{1.512971in}}%
\pgfpathlineto{\pgfqpoint{1.272350in}{1.515745in}}%
\pgfpathlineto{\pgfqpoint{1.296010in}{1.517043in}}%
\pgfpathlineto{\pgfqpoint{1.366991in}{1.524648in}}%
\pgfpathlineto{\pgfqpoint{1.378821in}{1.527091in}}%
\pgfpathlineto{\pgfqpoint{1.402481in}{1.529353in}}%
\pgfpathlineto{\pgfqpoint{1.473462in}{1.533865in}}%
\pgfpathlineto{\pgfqpoint{1.508953in}{1.535248in}}%
\pgfpathlineto{\pgfqpoint{1.532613in}{1.537334in}}%
\pgfpathlineto{\pgfqpoint{1.591763in}{1.542031in}}%
\pgfpathlineto{\pgfqpoint{1.615424in}{1.544317in}}%
\pgfpathlineto{\pgfqpoint{1.627254in}{1.544851in}}%
\pgfpathlineto{\pgfqpoint{1.639084in}{1.550192in}}%
\pgfpathlineto{\pgfqpoint{1.662744in}{1.551070in}}%
\pgfpathlineto{\pgfqpoint{1.674574in}{1.553749in}}%
\pgfpathlineto{\pgfqpoint{1.757385in}{1.557746in}}%
\pgfpathlineto{\pgfqpoint{1.781045in}{1.560659in}}%
\pgfpathlineto{\pgfqpoint{1.828366in}{1.562803in}}%
\pgfpathlineto{\pgfqpoint{1.911177in}{1.567468in}}%
\pgfpathlineto{\pgfqpoint{1.934837in}{1.570157in}}%
\pgfpathlineto{\pgfqpoint{1.982157in}{1.572040in}}%
\pgfpathlineto{\pgfqpoint{2.005818in}{1.573667in}}%
\pgfpathlineto{\pgfqpoint{2.041308in}{1.575635in}}%
\pgfpathlineto{\pgfqpoint{2.064968in}{1.576717in}}%
\pgfpathlineto{\pgfqpoint{2.088629in}{1.577775in}}%
\pgfpathlineto{\pgfqpoint{2.112289in}{1.579066in}}%
\pgfpathlineto{\pgfqpoint{2.124119in}{1.582154in}}%
\pgfpathlineto{\pgfqpoint{2.206930in}{1.585594in}}%
\pgfpathlineto{\pgfqpoint{2.254250in}{1.586708in}}%
\pgfpathlineto{\pgfqpoint{2.325231in}{1.590169in}}%
\pgfpathlineto{\pgfqpoint{2.348891in}{1.590607in}}%
\pgfpathlineto{\pgfqpoint{2.360721in}{1.592851in}}%
\pgfpathlineto{\pgfqpoint{2.372551in}{1.592929in}}%
\pgfpathlineto{\pgfqpoint{2.396212in}{1.594930in}}%
\pgfpathlineto{\pgfqpoint{2.443532in}{1.596849in}}%
\pgfpathlineto{\pgfqpoint{2.490853in}{1.597797in}}%
\pgfpathlineto{\pgfqpoint{2.573664in}{1.599858in}}%
\pgfpathlineto{\pgfqpoint{2.597324in}{1.600879in}}%
\pgfpathlineto{\pgfqpoint{2.632814in}{1.602155in}}%
\pgfpathlineto{\pgfqpoint{2.656474in}{1.603118in}}%
\pgfpathlineto{\pgfqpoint{2.680135in}{1.604246in}}%
\pgfpathlineto{\pgfqpoint{2.739285in}{1.605260in}}%
\pgfpathlineto{\pgfqpoint{2.881247in}{1.609297in}}%
\pgfpathlineto{\pgfqpoint{2.987718in}{1.612594in}}%
\pgfpathlineto{\pgfqpoint{3.070529in}{1.616149in}}%
\pgfpathlineto{\pgfqpoint{3.188830in}{1.620410in}}%
\pgfpathlineto{\pgfqpoint{3.200660in}{1.620590in}}%
\pgfpathlineto{\pgfqpoint{3.212490in}{1.622660in}}%
\pgfpathlineto{\pgfqpoint{3.318961in}{1.626219in}}%
\pgfpathlineto{\pgfqpoint{3.354452in}{1.627634in}}%
\pgfpathlineto{\pgfqpoint{3.484583in}{1.631890in}}%
\pgfpathlineto{\pgfqpoint{3.508243in}{1.633963in}}%
\pgfpathlineto{\pgfqpoint{3.555564in}{1.635607in}}%
\pgfpathlineto{\pgfqpoint{3.591054in}{1.636311in}}%
\pgfpathlineto{\pgfqpoint{3.744846in}{1.647823in}}%
\pgfpathlineto{\pgfqpoint{3.744846in}{1.647823in}}%
\pgfusepath{stroke}%
\end{pgfscope}%
\begin{pgfscope}%
\pgfsetrectcap%
\pgfsetmiterjoin%
\pgfsetlinewidth{0.803000pt}%
\definecolor{currentstroke}{rgb}{0.000000,0.000000,0.000000}%
\pgfsetstrokecolor{currentstroke}%
\pgfsetdash{}{0pt}%
\pgfpathmoveto{\pgfqpoint{0.550713in}{1.428886in}}%
\pgfpathlineto{\pgfqpoint{0.550713in}{2.425059in}}%
\pgfusepath{stroke}%
\end{pgfscope}%
\begin{pgfscope}%
\pgfsetrectcap%
\pgfsetmiterjoin%
\pgfsetlinewidth{0.803000pt}%
\definecolor{currentstroke}{rgb}{0.000000,0.000000,0.000000}%
\pgfsetstrokecolor{currentstroke}%
\pgfsetdash{}{0pt}%
\pgfpathmoveto{\pgfqpoint{3.744846in}{1.428886in}}%
\pgfpathlineto{\pgfqpoint{3.744846in}{2.425059in}}%
\pgfusepath{stroke}%
\end{pgfscope}%
\begin{pgfscope}%
\pgfsetrectcap%
\pgfsetmiterjoin%
\pgfsetlinewidth{0.803000pt}%
\definecolor{currentstroke}{rgb}{0.000000,0.000000,0.000000}%
\pgfsetstrokecolor{currentstroke}%
\pgfsetdash{}{0pt}%
\pgfpathmoveto{\pgfqpoint{0.550713in}{1.428886in}}%
\pgfpathlineto{\pgfqpoint{3.744846in}{1.428886in}}%
\pgfusepath{stroke}%
\end{pgfscope}%
\begin{pgfscope}%
\pgfsetrectcap%
\pgfsetmiterjoin%
\pgfsetlinewidth{0.803000pt}%
\definecolor{currentstroke}{rgb}{0.000000,0.000000,0.000000}%
\pgfsetstrokecolor{currentstroke}%
\pgfsetdash{}{0pt}%
\pgfpathmoveto{\pgfqpoint{0.550713in}{2.425059in}}%
\pgfpathlineto{\pgfqpoint{3.744846in}{2.425059in}}%
\pgfusepath{stroke}%
\end{pgfscope}%
\begin{pgfscope}%
\pgfsetbuttcap%
\pgfsetmiterjoin%
\definecolor{currentfill}{rgb}{1.000000,1.000000,1.000000}%
\pgfsetfillcolor{currentfill}%
\pgfsetlinewidth{0.000000pt}%
\definecolor{currentstroke}{rgb}{0.000000,0.000000,0.000000}%
\pgfsetstrokecolor{currentstroke}%
\pgfsetstrokeopacity{0.000000}%
\pgfsetdash{}{0pt}%
\pgfpathmoveto{\pgfqpoint{0.550713in}{0.382904in}}%
\pgfpathlineto{\pgfqpoint{3.744846in}{0.382904in}}%
\pgfpathlineto{\pgfqpoint{3.744846in}{1.379077in}}%
\pgfpathlineto{\pgfqpoint{0.550713in}{1.379077in}}%
\pgfpathclose%
\pgfusepath{fill}%
\end{pgfscope}%
\begin{pgfscope}%
\pgfsetbuttcap%
\pgfsetroundjoin%
\definecolor{currentfill}{rgb}{0.000000,0.000000,0.000000}%
\pgfsetfillcolor{currentfill}%
\pgfsetlinewidth{0.803000pt}%
\definecolor{currentstroke}{rgb}{0.000000,0.000000,0.000000}%
\pgfsetstrokecolor{currentstroke}%
\pgfsetdash{}{0pt}%
\pgfsys@defobject{currentmarker}{\pgfqpoint{0.000000in}{-0.048611in}}{\pgfqpoint{0.000000in}{0.000000in}}{%
\pgfpathmoveto{\pgfqpoint{0.000000in}{0.000000in}}%
\pgfpathlineto{\pgfqpoint{0.000000in}{-0.048611in}}%
\pgfusepath{stroke,fill}%
}%
\begin{pgfscope}%
\pgfsys@transformshift{0.787315in}{0.382904in}%
\pgfsys@useobject{currentmarker}{}%
\end{pgfscope}%
\end{pgfscope}%
\begin{pgfscope}%
\definecolor{textcolor}{rgb}{0.000000,0.000000,0.000000}%
\pgfsetstrokecolor{textcolor}%
\pgfsetfillcolor{textcolor}%
\pgftext[x=0.787315in,y=0.285682in,,top]{\color{textcolor}\rmfamily\fontsize{10.000000}{12.000000}\selectfont \(\displaystyle {50}\)}%
\end{pgfscope}%
\begin{pgfscope}%
\pgfsetbuttcap%
\pgfsetroundjoin%
\definecolor{currentfill}{rgb}{0.000000,0.000000,0.000000}%
\pgfsetfillcolor{currentfill}%
\pgfsetlinewidth{0.803000pt}%
\definecolor{currentstroke}{rgb}{0.000000,0.000000,0.000000}%
\pgfsetstrokecolor{currentstroke}%
\pgfsetdash{}{0pt}%
\pgfsys@defobject{currentmarker}{\pgfqpoint{0.000000in}{-0.048611in}}{\pgfqpoint{0.000000in}{0.000000in}}{%
\pgfpathmoveto{\pgfqpoint{0.000000in}{0.000000in}}%
\pgfpathlineto{\pgfqpoint{0.000000in}{-0.048611in}}%
\pgfusepath{stroke,fill}%
}%
\begin{pgfscope}%
\pgfsys@transformshift{1.378821in}{0.382904in}%
\pgfsys@useobject{currentmarker}{}%
\end{pgfscope}%
\end{pgfscope}%
\begin{pgfscope}%
\definecolor{textcolor}{rgb}{0.000000,0.000000,0.000000}%
\pgfsetstrokecolor{textcolor}%
\pgfsetfillcolor{textcolor}%
\pgftext[x=1.378821in,y=0.285682in,,top]{\color{textcolor}\rmfamily\fontsize{10.000000}{12.000000}\selectfont \(\displaystyle {100}\)}%
\end{pgfscope}%
\begin{pgfscope}%
\pgfsetbuttcap%
\pgfsetroundjoin%
\definecolor{currentfill}{rgb}{0.000000,0.000000,0.000000}%
\pgfsetfillcolor{currentfill}%
\pgfsetlinewidth{0.803000pt}%
\definecolor{currentstroke}{rgb}{0.000000,0.000000,0.000000}%
\pgfsetstrokecolor{currentstroke}%
\pgfsetdash{}{0pt}%
\pgfsys@defobject{currentmarker}{\pgfqpoint{0.000000in}{-0.048611in}}{\pgfqpoint{0.000000in}{0.000000in}}{%
\pgfpathmoveto{\pgfqpoint{0.000000in}{0.000000in}}%
\pgfpathlineto{\pgfqpoint{0.000000in}{-0.048611in}}%
\pgfusepath{stroke,fill}%
}%
\begin{pgfscope}%
\pgfsys@transformshift{1.970327in}{0.382904in}%
\pgfsys@useobject{currentmarker}{}%
\end{pgfscope}%
\end{pgfscope}%
\begin{pgfscope}%
\definecolor{textcolor}{rgb}{0.000000,0.000000,0.000000}%
\pgfsetstrokecolor{textcolor}%
\pgfsetfillcolor{textcolor}%
\pgftext[x=1.970327in,y=0.285682in,,top]{\color{textcolor}\rmfamily\fontsize{10.000000}{12.000000}\selectfont \(\displaystyle {150}\)}%
\end{pgfscope}%
\begin{pgfscope}%
\pgfsetbuttcap%
\pgfsetroundjoin%
\definecolor{currentfill}{rgb}{0.000000,0.000000,0.000000}%
\pgfsetfillcolor{currentfill}%
\pgfsetlinewidth{0.803000pt}%
\definecolor{currentstroke}{rgb}{0.000000,0.000000,0.000000}%
\pgfsetstrokecolor{currentstroke}%
\pgfsetdash{}{0pt}%
\pgfsys@defobject{currentmarker}{\pgfqpoint{0.000000in}{-0.048611in}}{\pgfqpoint{0.000000in}{0.000000in}}{%
\pgfpathmoveto{\pgfqpoint{0.000000in}{0.000000in}}%
\pgfpathlineto{\pgfqpoint{0.000000in}{-0.048611in}}%
\pgfusepath{stroke,fill}%
}%
\begin{pgfscope}%
\pgfsys@transformshift{2.561833in}{0.382904in}%
\pgfsys@useobject{currentmarker}{}%
\end{pgfscope}%
\end{pgfscope}%
\begin{pgfscope}%
\definecolor{textcolor}{rgb}{0.000000,0.000000,0.000000}%
\pgfsetstrokecolor{textcolor}%
\pgfsetfillcolor{textcolor}%
\pgftext[x=2.561833in,y=0.285682in,,top]{\color{textcolor}\rmfamily\fontsize{10.000000}{12.000000}\selectfont \(\displaystyle {200}\)}%
\end{pgfscope}%
\begin{pgfscope}%
\pgfsetbuttcap%
\pgfsetroundjoin%
\definecolor{currentfill}{rgb}{0.000000,0.000000,0.000000}%
\pgfsetfillcolor{currentfill}%
\pgfsetlinewidth{0.803000pt}%
\definecolor{currentstroke}{rgb}{0.000000,0.000000,0.000000}%
\pgfsetstrokecolor{currentstroke}%
\pgfsetdash{}{0pt}%
\pgfsys@defobject{currentmarker}{\pgfqpoint{0.000000in}{-0.048611in}}{\pgfqpoint{0.000000in}{0.000000in}}{%
\pgfpathmoveto{\pgfqpoint{0.000000in}{0.000000in}}%
\pgfpathlineto{\pgfqpoint{0.000000in}{-0.048611in}}%
\pgfusepath{stroke,fill}%
}%
\begin{pgfscope}%
\pgfsys@transformshift{3.153340in}{0.382904in}%
\pgfsys@useobject{currentmarker}{}%
\end{pgfscope}%
\end{pgfscope}%
\begin{pgfscope}%
\definecolor{textcolor}{rgb}{0.000000,0.000000,0.000000}%
\pgfsetstrokecolor{textcolor}%
\pgfsetfillcolor{textcolor}%
\pgftext[x=3.153340in,y=0.285682in,,top]{\color{textcolor}\rmfamily\fontsize{10.000000}{12.000000}\selectfont \(\displaystyle {250}\)}%
\end{pgfscope}%
\begin{pgfscope}%
\pgfsetbuttcap%
\pgfsetroundjoin%
\definecolor{currentfill}{rgb}{0.000000,0.000000,0.000000}%
\pgfsetfillcolor{currentfill}%
\pgfsetlinewidth{0.803000pt}%
\definecolor{currentstroke}{rgb}{0.000000,0.000000,0.000000}%
\pgfsetstrokecolor{currentstroke}%
\pgfsetdash{}{0pt}%
\pgfsys@defobject{currentmarker}{\pgfqpoint{0.000000in}{-0.048611in}}{\pgfqpoint{0.000000in}{0.000000in}}{%
\pgfpathmoveto{\pgfqpoint{0.000000in}{0.000000in}}%
\pgfpathlineto{\pgfqpoint{0.000000in}{-0.048611in}}%
\pgfusepath{stroke,fill}%
}%
\begin{pgfscope}%
\pgfsys@transformshift{3.744846in}{0.382904in}%
\pgfsys@useobject{currentmarker}{}%
\end{pgfscope}%
\end{pgfscope}%
\begin{pgfscope}%
\definecolor{textcolor}{rgb}{0.000000,0.000000,0.000000}%
\pgfsetstrokecolor{textcolor}%
\pgfsetfillcolor{textcolor}%
\pgftext[x=3.744846in,y=0.285682in,,top]{\color{textcolor}\rmfamily\fontsize{10.000000}{12.000000}\selectfont \(\displaystyle {300}\)}%
\end{pgfscope}%
\begin{pgfscope}%
\definecolor{textcolor}{rgb}{0.000000,0.000000,0.000000}%
\pgfsetstrokecolor{textcolor}%
\pgfsetfillcolor{textcolor}%
\pgftext[x=2.147779in,y=0.106793in,,top]{\color{textcolor}\rmfamily\fontsize{10.000000}{12.000000}\selectfont \(\displaystyle t\)}%
\end{pgfscope}%
\begin{pgfscope}%
\pgfsetbuttcap%
\pgfsetroundjoin%
\definecolor{currentfill}{rgb}{0.000000,0.000000,0.000000}%
\pgfsetfillcolor{currentfill}%
\pgfsetlinewidth{0.803000pt}%
\definecolor{currentstroke}{rgb}{0.000000,0.000000,0.000000}%
\pgfsetstrokecolor{currentstroke}%
\pgfsetdash{}{0pt}%
\pgfsys@defobject{currentmarker}{\pgfqpoint{-0.048611in}{0.000000in}}{\pgfqpoint{-0.000000in}{0.000000in}}{%
\pgfpathmoveto{\pgfqpoint{-0.000000in}{0.000000in}}%
\pgfpathlineto{\pgfqpoint{-0.048611in}{0.000000in}}%
\pgfusepath{stroke,fill}%
}%
\begin{pgfscope}%
\pgfsys@transformshift{0.550713in}{0.427227in}%
\pgfsys@useobject{currentmarker}{}%
\end{pgfscope}%
\end{pgfscope}%
\begin{pgfscope}%
\definecolor{textcolor}{rgb}{0.000000,0.000000,0.000000}%
\pgfsetstrokecolor{textcolor}%
\pgfsetfillcolor{textcolor}%
\pgftext[x=0.384046in, y=0.379033in, left, base]{\color{textcolor}\rmfamily\fontsize{10.000000}{12.000000}\selectfont \(\displaystyle {0}\)}%
\end{pgfscope}%
\begin{pgfscope}%
\pgfsetbuttcap%
\pgfsetroundjoin%
\definecolor{currentfill}{rgb}{0.000000,0.000000,0.000000}%
\pgfsetfillcolor{currentfill}%
\pgfsetlinewidth{0.803000pt}%
\definecolor{currentstroke}{rgb}{0.000000,0.000000,0.000000}%
\pgfsetstrokecolor{currentstroke}%
\pgfsetdash{}{0pt}%
\pgfsys@defobject{currentmarker}{\pgfqpoint{-0.048611in}{0.000000in}}{\pgfqpoint{-0.000000in}{0.000000in}}{%
\pgfpathmoveto{\pgfqpoint{-0.000000in}{0.000000in}}%
\pgfpathlineto{\pgfqpoint{-0.048611in}{0.000000in}}%
\pgfusepath{stroke,fill}%
}%
\begin{pgfscope}%
\pgfsys@transformshift{0.550713in}{0.863445in}%
\pgfsys@useobject{currentmarker}{}%
\end{pgfscope}%
\end{pgfscope}%
\begin{pgfscope}%
\definecolor{textcolor}{rgb}{0.000000,0.000000,0.000000}%
\pgfsetstrokecolor{textcolor}%
\pgfsetfillcolor{textcolor}%
\pgftext[x=0.314601in, y=0.815251in, left, base]{\color{textcolor}\rmfamily\fontsize{10.000000}{12.000000}\selectfont \(\displaystyle {10}\)}%
\end{pgfscope}%
\begin{pgfscope}%
\pgfsetbuttcap%
\pgfsetroundjoin%
\definecolor{currentfill}{rgb}{0.000000,0.000000,0.000000}%
\pgfsetfillcolor{currentfill}%
\pgfsetlinewidth{0.803000pt}%
\definecolor{currentstroke}{rgb}{0.000000,0.000000,0.000000}%
\pgfsetstrokecolor{currentstroke}%
\pgfsetdash{}{0pt}%
\pgfsys@defobject{currentmarker}{\pgfqpoint{-0.048611in}{0.000000in}}{\pgfqpoint{-0.000000in}{0.000000in}}{%
\pgfpathmoveto{\pgfqpoint{-0.000000in}{0.000000in}}%
\pgfpathlineto{\pgfqpoint{-0.048611in}{0.000000in}}%
\pgfusepath{stroke,fill}%
}%
\begin{pgfscope}%
\pgfsys@transformshift{0.550713in}{1.299663in}%
\pgfsys@useobject{currentmarker}{}%
\end{pgfscope}%
\end{pgfscope}%
\begin{pgfscope}%
\definecolor{textcolor}{rgb}{0.000000,0.000000,0.000000}%
\pgfsetstrokecolor{textcolor}%
\pgfsetfillcolor{textcolor}%
\pgftext[x=0.314601in, y=1.251468in, left, base]{\color{textcolor}\rmfamily\fontsize{10.000000}{12.000000}\selectfont \(\displaystyle {20}\)}%
\end{pgfscope}%
\begin{pgfscope}%
\definecolor{textcolor}{rgb}{0.000000,0.000000,0.000000}%
\pgfsetstrokecolor{textcolor}%
\pgfsetfillcolor{textcolor}%
\pgftext[x=0.259046in,y=0.880990in,,bottom,rotate=90.000000]{\color{textcolor}\rmfamily\fontsize{10.000000}{12.000000}\selectfont Exploitation \(\displaystyle R_T^*\)}%
\end{pgfscope}%
\begin{pgfscope}%
\pgfpathrectangle{\pgfqpoint{0.550713in}{0.382904in}}{\pgfqpoint{3.194133in}{0.996173in}}%
\pgfusepath{clip}%
\pgfsetrectcap%
\pgfsetroundjoin%
\pgfsetlinewidth{0.853187pt}%
\definecolor{currentstroke}{rgb}{0.631373,0.062745,0.207843}%
\pgfsetstrokecolor{currentstroke}%
\pgfsetdash{}{0pt}%
\pgfpathmoveto{\pgfqpoint{0.562543in}{0.428185in}}%
\pgfpathlineto{\pgfqpoint{0.657184in}{0.429932in}}%
\pgfpathlineto{\pgfqpoint{0.751825in}{0.432338in}}%
\pgfpathlineto{\pgfqpoint{0.964767in}{0.437558in}}%
\pgfpathlineto{\pgfqpoint{0.988427in}{0.442319in}}%
\pgfpathlineto{\pgfqpoint{1.059408in}{0.446940in}}%
\pgfpathlineto{\pgfqpoint{1.130389in}{0.455302in}}%
\pgfpathlineto{\pgfqpoint{1.154049in}{0.458597in}}%
\pgfpathlineto{\pgfqpoint{1.165879in}{0.461715in}}%
\pgfpathlineto{\pgfqpoint{1.189539in}{0.464716in}}%
\pgfpathlineto{\pgfqpoint{1.201369in}{0.466949in}}%
\pgfpathlineto{\pgfqpoint{1.248690in}{0.470749in}}%
\pgfpathlineto{\pgfqpoint{1.284180in}{0.474722in}}%
\pgfpathlineto{\pgfqpoint{1.307840in}{0.482904in}}%
\pgfpathlineto{\pgfqpoint{1.355161in}{0.489816in}}%
\pgfpathlineto{\pgfqpoint{1.437972in}{0.499803in}}%
\pgfpathlineto{\pgfqpoint{1.508953in}{0.508540in}}%
\pgfpathlineto{\pgfqpoint{1.591763in}{0.518311in}}%
\pgfpathlineto{\pgfqpoint{1.662744in}{0.525512in}}%
\pgfpathlineto{\pgfqpoint{1.745555in}{0.534939in}}%
\pgfpathlineto{\pgfqpoint{1.804706in}{0.541503in}}%
\pgfpathlineto{\pgfqpoint{1.840196in}{0.545111in}}%
\pgfpathlineto{\pgfqpoint{1.946667in}{0.556937in}}%
\pgfpathlineto{\pgfqpoint{2.041308in}{0.566896in}}%
\pgfpathlineto{\pgfqpoint{2.088629in}{0.572239in}}%
\pgfpathlineto{\pgfqpoint{2.171439in}{0.579376in}}%
\pgfpathlineto{\pgfqpoint{2.396212in}{0.601291in}}%
\pgfpathlineto{\pgfqpoint{2.526343in}{0.617823in}}%
\pgfpathlineto{\pgfqpoint{2.597324in}{0.627634in}}%
\pgfpathlineto{\pgfqpoint{2.656474in}{0.635011in}}%
\pgfpathlineto{\pgfqpoint{2.703795in}{0.641433in}}%
\pgfpathlineto{\pgfqpoint{2.845756in}{0.660693in}}%
\pgfpathlineto{\pgfqpoint{3.070529in}{0.688975in}}%
\pgfpathlineto{\pgfqpoint{3.413602in}{0.711157in}}%
\pgfpathlineto{\pgfqpoint{3.508243in}{0.719710in}}%
\pgfpathlineto{\pgfqpoint{3.520073in}{0.722466in}}%
\pgfpathlineto{\pgfqpoint{3.555564in}{0.725865in}}%
\pgfpathlineto{\pgfqpoint{3.685695in}{0.734609in}}%
\pgfpathlineto{\pgfqpoint{3.721185in}{0.738173in}}%
\pgfpathlineto{\pgfqpoint{3.744846in}{0.740051in}}%
\pgfpathlineto{\pgfqpoint{3.744846in}{0.740051in}}%
\pgfusepath{stroke}%
\end{pgfscope}%
\begin{pgfscope}%
\pgfpathrectangle{\pgfqpoint{0.550713in}{0.382904in}}{\pgfqpoint{3.194133in}{0.996173in}}%
\pgfusepath{clip}%
\pgfsetrectcap%
\pgfsetroundjoin%
\pgfsetlinewidth{0.853187pt}%
\definecolor{currentstroke}{rgb}{0.890196,0.000000,0.400000}%
\pgfsetstrokecolor{currentstroke}%
\pgfsetdash{}{0pt}%
\pgfpathmoveto{\pgfqpoint{0.562543in}{0.428185in}}%
\pgfpathlineto{\pgfqpoint{0.657184in}{0.429934in}}%
\pgfpathlineto{\pgfqpoint{0.751825in}{0.432338in}}%
\pgfpathlineto{\pgfqpoint{0.964767in}{0.437386in}}%
\pgfpathlineto{\pgfqpoint{0.988427in}{0.442064in}}%
\pgfpathlineto{\pgfqpoint{1.035748in}{0.445943in}}%
\pgfpathlineto{\pgfqpoint{1.071238in}{0.450075in}}%
\pgfpathlineto{\pgfqpoint{1.106728in}{0.453461in}}%
\pgfpathlineto{\pgfqpoint{1.177709in}{0.459418in}}%
\pgfpathlineto{\pgfqpoint{1.236860in}{0.472125in}}%
\pgfpathlineto{\pgfqpoint{1.260520in}{0.476576in}}%
\pgfpathlineto{\pgfqpoint{1.272350in}{0.478430in}}%
\pgfpathlineto{\pgfqpoint{1.284180in}{0.488976in}}%
\pgfpathlineto{\pgfqpoint{1.296010in}{0.491168in}}%
\pgfpathlineto{\pgfqpoint{1.307840in}{0.496781in}}%
\pgfpathlineto{\pgfqpoint{1.508953in}{0.525930in}}%
\pgfpathlineto{\pgfqpoint{1.520783in}{0.529066in}}%
\pgfpathlineto{\pgfqpoint{1.615424in}{0.540159in}}%
\pgfpathlineto{\pgfqpoint{1.721895in}{0.551416in}}%
\pgfpathlineto{\pgfqpoint{1.828366in}{0.564256in}}%
\pgfpathlineto{\pgfqpoint{1.923007in}{0.573407in}}%
\pgfpathlineto{\pgfqpoint{1.982157in}{0.578015in}}%
\pgfpathlineto{\pgfqpoint{2.242420in}{0.600852in}}%
\pgfpathlineto{\pgfqpoint{2.313401in}{0.607684in}}%
\pgfpathlineto{\pgfqpoint{2.360721in}{0.611650in}}%
\pgfpathlineto{\pgfqpoint{2.408042in}{0.615153in}}%
\pgfpathlineto{\pgfqpoint{2.490853in}{0.622413in}}%
\pgfpathlineto{\pgfqpoint{2.573664in}{0.631510in}}%
\pgfpathlineto{\pgfqpoint{2.644644in}{0.636442in}}%
\pgfpathlineto{\pgfqpoint{2.727455in}{0.644112in}}%
\pgfpathlineto{\pgfqpoint{2.798436in}{0.652632in}}%
\pgfpathlineto{\pgfqpoint{2.869417in}{0.660269in}}%
\pgfpathlineto{\pgfqpoint{2.999548in}{0.670480in}}%
\pgfpathlineto{\pgfqpoint{3.058699in}{0.677075in}}%
\pgfpathlineto{\pgfqpoint{3.106019in}{0.682617in}}%
\pgfpathlineto{\pgfqpoint{3.117849in}{0.685069in}}%
\pgfpathlineto{\pgfqpoint{3.200660in}{0.694493in}}%
\pgfpathlineto{\pgfqpoint{3.259811in}{0.701292in}}%
\pgfpathlineto{\pgfqpoint{3.401772in}{0.712742in}}%
\pgfpathlineto{\pgfqpoint{3.437263in}{0.716093in}}%
\pgfpathlineto{\pgfqpoint{3.555564in}{0.728467in}}%
\pgfpathlineto{\pgfqpoint{3.602884in}{0.731608in}}%
\pgfpathlineto{\pgfqpoint{3.614714in}{0.757860in}}%
\pgfpathlineto{\pgfqpoint{3.744846in}{0.768698in}}%
\pgfpathlineto{\pgfqpoint{3.744846in}{0.768698in}}%
\pgfusepath{stroke}%
\end{pgfscope}%
\begin{pgfscope}%
\pgfpathrectangle{\pgfqpoint{0.550713in}{0.382904in}}{\pgfqpoint{3.194133in}{0.996173in}}%
\pgfusepath{clip}%
\pgfsetrectcap%
\pgfsetroundjoin%
\pgfsetlinewidth{0.853187pt}%
\definecolor{currentstroke}{rgb}{0.000000,0.329412,0.623529}%
\pgfsetstrokecolor{currentstroke}%
\pgfsetdash{}{0pt}%
\pgfpathmoveto{\pgfqpoint{0.562543in}{0.428947in}}%
\pgfpathlineto{\pgfqpoint{0.657184in}{0.430976in}}%
\pgfpathlineto{\pgfqpoint{0.763655in}{0.433454in}}%
\pgfpathlineto{\pgfqpoint{0.846466in}{0.435081in}}%
\pgfpathlineto{\pgfqpoint{1.012087in}{0.438953in}}%
\pgfpathlineto{\pgfqpoint{1.094898in}{0.443769in}}%
\pgfpathlineto{\pgfqpoint{1.106728in}{0.446268in}}%
\pgfpathlineto{\pgfqpoint{1.177709in}{0.450068in}}%
\pgfpathlineto{\pgfqpoint{1.225030in}{0.452906in}}%
\pgfpathlineto{\pgfqpoint{1.260520in}{0.457607in}}%
\pgfpathlineto{\pgfqpoint{1.319671in}{0.466528in}}%
\pgfpathlineto{\pgfqpoint{1.485292in}{0.482685in}}%
\pgfpathlineto{\pgfqpoint{1.698235in}{0.496508in}}%
\pgfpathlineto{\pgfqpoint{1.816536in}{0.500210in}}%
\pgfpathlineto{\pgfqpoint{1.863856in}{0.502112in}}%
\pgfpathlineto{\pgfqpoint{2.041308in}{0.511024in}}%
\pgfpathlineto{\pgfqpoint{2.112289in}{0.513241in}}%
\pgfpathlineto{\pgfqpoint{2.159609in}{0.515283in}}%
\pgfpathlineto{\pgfqpoint{2.289741in}{0.522277in}}%
\pgfpathlineto{\pgfqpoint{2.479023in}{0.528601in}}%
\pgfpathlineto{\pgfqpoint{2.538173in}{0.533192in}}%
\pgfpathlineto{\pgfqpoint{2.680135in}{0.547474in}}%
\pgfpathlineto{\pgfqpoint{2.739285in}{0.555702in}}%
\pgfpathlineto{\pgfqpoint{2.940397in}{0.582454in}}%
\pgfpathlineto{\pgfqpoint{3.330791in}{0.617639in}}%
\pgfpathlineto{\pgfqpoint{3.437263in}{0.623017in}}%
\pgfpathlineto{\pgfqpoint{3.567394in}{0.628241in}}%
\pgfpathlineto{\pgfqpoint{3.744846in}{0.634942in}}%
\pgfpathlineto{\pgfqpoint{3.744846in}{0.634942in}}%
\pgfusepath{stroke}%
\end{pgfscope}%
\begin{pgfscope}%
\pgfpathrectangle{\pgfqpoint{0.550713in}{0.382904in}}{\pgfqpoint{3.194133in}{0.996173in}}%
\pgfusepath{clip}%
\pgfsetrectcap%
\pgfsetroundjoin%
\pgfsetlinewidth{0.853187pt}%
\definecolor{currentstroke}{rgb}{0.000000,0.380392,0.396078}%
\pgfsetstrokecolor{currentstroke}%
\pgfsetdash{}{0pt}%
\pgfpathmoveto{\pgfqpoint{0.562543in}{0.428947in}}%
\pgfpathlineto{\pgfqpoint{0.917446in}{0.438867in}}%
\pgfpathlineto{\pgfqpoint{1.000257in}{0.441866in}}%
\pgfpathlineto{\pgfqpoint{1.059408in}{0.445607in}}%
\pgfpathlineto{\pgfqpoint{1.177709in}{0.453213in}}%
\pgfpathlineto{\pgfqpoint{1.201369in}{0.454620in}}%
\pgfpathlineto{\pgfqpoint{1.225030in}{0.457072in}}%
\pgfpathlineto{\pgfqpoint{1.248690in}{0.459577in}}%
\pgfpathlineto{\pgfqpoint{1.284180in}{0.463667in}}%
\pgfpathlineto{\pgfqpoint{1.307840in}{0.466659in}}%
\pgfpathlineto{\pgfqpoint{1.319671in}{0.469384in}}%
\pgfpathlineto{\pgfqpoint{1.426142in}{0.480954in}}%
\pgfpathlineto{\pgfqpoint{1.544443in}{0.490604in}}%
\pgfpathlineto{\pgfqpoint{1.579933in}{0.492609in}}%
\pgfpathlineto{\pgfqpoint{1.710065in}{0.497191in}}%
\pgfpathlineto{\pgfqpoint{1.804706in}{0.500455in}}%
\pgfpathlineto{\pgfqpoint{1.911177in}{0.504372in}}%
\pgfpathlineto{\pgfqpoint{2.076798in}{0.513051in}}%
\pgfpathlineto{\pgfqpoint{2.360721in}{0.520569in}}%
\pgfpathlineto{\pgfqpoint{2.443532in}{0.522543in}}%
\pgfpathlineto{\pgfqpoint{2.514513in}{0.525949in}}%
\pgfpathlineto{\pgfqpoint{2.644644in}{0.533002in}}%
\pgfpathlineto{\pgfqpoint{2.727455in}{0.537862in}}%
\pgfpathlineto{\pgfqpoint{2.893077in}{0.550036in}}%
\pgfpathlineto{\pgfqpoint{2.975888in}{0.556201in}}%
\pgfpathlineto{\pgfqpoint{3.046868in}{0.562050in}}%
\pgfpathlineto{\pgfqpoint{3.082359in}{0.565161in}}%
\pgfpathlineto{\pgfqpoint{3.141509in}{0.569091in}}%
\pgfpathlineto{\pgfqpoint{3.330791in}{0.579570in}}%
\pgfpathlineto{\pgfqpoint{3.425432in}{0.585698in}}%
\pgfpathlineto{\pgfqpoint{3.555564in}{0.593020in}}%
\pgfpathlineto{\pgfqpoint{3.662035in}{0.595199in}}%
\pgfpathlineto{\pgfqpoint{3.744846in}{0.597174in}}%
\pgfpathlineto{\pgfqpoint{3.744846in}{0.597174in}}%
\pgfusepath{stroke}%
\end{pgfscope}%
\begin{pgfscope}%
\pgfpathrectangle{\pgfqpoint{0.550713in}{0.382904in}}{\pgfqpoint{3.194133in}{0.996173in}}%
\pgfusepath{clip}%
\pgfsetrectcap%
\pgfsetroundjoin%
\pgfsetlinewidth{0.853187pt}%
\definecolor{currentstroke}{rgb}{0.380392,0.129412,0.345098}%
\pgfsetstrokecolor{currentstroke}%
\pgfsetdash{}{0pt}%
\pgfpathmoveto{\pgfqpoint{0.562543in}{0.431838in}}%
\pgfpathlineto{\pgfqpoint{0.598033in}{0.445004in}}%
\pgfpathlineto{\pgfqpoint{0.763655in}{0.506964in}}%
\pgfpathlineto{\pgfqpoint{0.858296in}{0.545756in}}%
\pgfpathlineto{\pgfqpoint{0.905616in}{0.571210in}}%
\pgfpathlineto{\pgfqpoint{0.929277in}{0.586717in}}%
\pgfpathlineto{\pgfqpoint{0.952937in}{0.606230in}}%
\pgfpathlineto{\pgfqpoint{0.976597in}{0.628666in}}%
\pgfpathlineto{\pgfqpoint{1.000257in}{0.656747in}}%
\pgfpathlineto{\pgfqpoint{1.023918in}{0.689991in}}%
\pgfpathlineto{\pgfqpoint{1.059408in}{0.746362in}}%
\pgfpathlineto{\pgfqpoint{1.071238in}{0.761861in}}%
\pgfpathlineto{\pgfqpoint{1.083068in}{0.780085in}}%
\pgfpathlineto{\pgfqpoint{1.106728in}{0.807880in}}%
\pgfpathlineto{\pgfqpoint{1.213199in}{0.829483in}}%
\pgfpathlineto{\pgfqpoint{1.248690in}{0.834706in}}%
\pgfpathlineto{\pgfqpoint{1.284180in}{0.840640in}}%
\pgfpathlineto{\pgfqpoint{1.319671in}{0.849028in}}%
\pgfpathlineto{\pgfqpoint{1.331501in}{0.850325in}}%
\pgfpathlineto{\pgfqpoint{1.343331in}{0.853248in}}%
\pgfpathlineto{\pgfqpoint{1.402481in}{0.861405in}}%
\pgfpathlineto{\pgfqpoint{1.414312in}{0.864634in}}%
\pgfpathlineto{\pgfqpoint{1.426142in}{0.869299in}}%
\pgfpathlineto{\pgfqpoint{1.449802in}{0.876196in}}%
\pgfpathlineto{\pgfqpoint{1.473462in}{0.883542in}}%
\pgfpathlineto{\pgfqpoint{1.544443in}{0.895571in}}%
\pgfpathlineto{\pgfqpoint{1.579933in}{0.902012in}}%
\pgfpathlineto{\pgfqpoint{1.591763in}{0.903414in}}%
\pgfpathlineto{\pgfqpoint{1.603594in}{0.906105in}}%
\pgfpathlineto{\pgfqpoint{1.615424in}{0.907591in}}%
\pgfpathlineto{\pgfqpoint{1.627254in}{0.910464in}}%
\pgfpathlineto{\pgfqpoint{1.733725in}{0.926505in}}%
\pgfpathlineto{\pgfqpoint{1.852026in}{0.946915in}}%
\pgfpathlineto{\pgfqpoint{1.863856in}{0.950767in}}%
\pgfpathlineto{\pgfqpoint{1.899347in}{0.956916in}}%
\pgfpathlineto{\pgfqpoint{1.923007in}{0.961841in}}%
\pgfpathlineto{\pgfqpoint{1.934837in}{0.966729in}}%
\pgfpathlineto{\pgfqpoint{1.982157in}{0.973933in}}%
\pgfpathlineto{\pgfqpoint{2.005818in}{0.977914in}}%
\pgfpathlineto{\pgfqpoint{2.017648in}{0.981931in}}%
\pgfpathlineto{\pgfqpoint{2.029478in}{0.984236in}}%
\pgfpathlineto{\pgfqpoint{2.041308in}{0.989677in}}%
\pgfpathlineto{\pgfqpoint{2.100459in}{1.000577in}}%
\pgfpathlineto{\pgfqpoint{2.135949in}{1.007755in}}%
\pgfpathlineto{\pgfqpoint{2.277911in}{1.031960in}}%
\pgfpathlineto{\pgfqpoint{2.325231in}{1.038948in}}%
\pgfpathlineto{\pgfqpoint{2.348891in}{1.042669in}}%
\pgfpathlineto{\pgfqpoint{2.372551in}{1.048422in}}%
\pgfpathlineto{\pgfqpoint{2.396212in}{1.056956in}}%
\pgfpathlineto{\pgfqpoint{2.408042in}{1.059783in}}%
\pgfpathlineto{\pgfqpoint{2.419872in}{1.064259in}}%
\pgfpathlineto{\pgfqpoint{2.502683in}{1.078989in}}%
\pgfpathlineto{\pgfqpoint{2.538173in}{1.085268in}}%
\pgfpathlineto{\pgfqpoint{2.573664in}{1.090549in}}%
\pgfpathlineto{\pgfqpoint{2.632814in}{1.101141in}}%
\pgfpathlineto{\pgfqpoint{2.668305in}{1.107404in}}%
\pgfpathlineto{\pgfqpoint{2.739285in}{1.124929in}}%
\pgfpathlineto{\pgfqpoint{2.774776in}{1.134507in}}%
\pgfpathlineto{\pgfqpoint{2.786606in}{1.139065in}}%
\pgfpathlineto{\pgfqpoint{2.798436in}{1.142313in}}%
\pgfpathlineto{\pgfqpoint{2.810266in}{1.146826in}}%
\pgfpathlineto{\pgfqpoint{2.893077in}{1.165563in}}%
\pgfpathlineto{\pgfqpoint{2.964058in}{1.181788in}}%
\pgfpathlineto{\pgfqpoint{3.046868in}{1.197107in}}%
\pgfpathlineto{\pgfqpoint{3.094189in}{1.206205in}}%
\pgfpathlineto{\pgfqpoint{3.129679in}{1.213333in}}%
\pgfpathlineto{\pgfqpoint{3.141509in}{1.217423in}}%
\pgfpathlineto{\pgfqpoint{3.236150in}{1.235863in}}%
\pgfpathlineto{\pgfqpoint{3.271641in}{1.241789in}}%
\pgfpathlineto{\pgfqpoint{3.318961in}{1.252126in}}%
\pgfpathlineto{\pgfqpoint{3.413602in}{1.266369in}}%
\pgfpathlineto{\pgfqpoint{3.472753in}{1.280146in}}%
\pgfpathlineto{\pgfqpoint{3.508243in}{1.288683in}}%
\pgfpathlineto{\pgfqpoint{3.531904in}{1.292426in}}%
\pgfpathlineto{\pgfqpoint{3.591054in}{1.302122in}}%
\pgfpathlineto{\pgfqpoint{3.662035in}{1.315759in}}%
\pgfpathlineto{\pgfqpoint{3.673865in}{1.317013in}}%
\pgfpathlineto{\pgfqpoint{3.733016in}{1.329291in}}%
\pgfpathlineto{\pgfqpoint{3.744846in}{1.333796in}}%
\pgfpathlineto{\pgfqpoint{3.744846in}{1.333796in}}%
\pgfusepath{stroke}%
\end{pgfscope}%
\begin{pgfscope}%
\pgfpathrectangle{\pgfqpoint{0.550713in}{0.382904in}}{\pgfqpoint{3.194133in}{0.996173in}}%
\pgfusepath{clip}%
\pgfsetrectcap%
\pgfsetroundjoin%
\pgfsetlinewidth{0.853187pt}%
\definecolor{currentstroke}{rgb}{0.964706,0.658824,0.000000}%
\pgfsetstrokecolor{currentstroke}%
\pgfsetdash{}{0pt}%
\pgfpathmoveto{\pgfqpoint{0.562543in}{0.431821in}}%
\pgfpathlineto{\pgfqpoint{0.645354in}{0.460363in}}%
\pgfpathlineto{\pgfqpoint{0.716334in}{0.482870in}}%
\pgfpathlineto{\pgfqpoint{0.775485in}{0.502746in}}%
\pgfpathlineto{\pgfqpoint{0.834636in}{0.523173in}}%
\pgfpathlineto{\pgfqpoint{0.858296in}{0.531636in}}%
\pgfpathlineto{\pgfqpoint{0.905616in}{0.553000in}}%
\pgfpathlineto{\pgfqpoint{0.941107in}{0.573400in}}%
\pgfpathlineto{\pgfqpoint{0.964767in}{0.590594in}}%
\pgfpathlineto{\pgfqpoint{0.976597in}{0.599930in}}%
\pgfpathlineto{\pgfqpoint{1.000257in}{0.622719in}}%
\pgfpathlineto{\pgfqpoint{1.012087in}{0.635314in}}%
\pgfpathlineto{\pgfqpoint{1.047578in}{0.679938in}}%
\pgfpathlineto{\pgfqpoint{1.071238in}{0.713355in}}%
\pgfpathlineto{\pgfqpoint{1.106728in}{0.737731in}}%
\pgfpathlineto{\pgfqpoint{1.142219in}{0.742723in}}%
\pgfpathlineto{\pgfqpoint{1.165879in}{0.747378in}}%
\pgfpathlineto{\pgfqpoint{1.177709in}{0.748860in}}%
\pgfpathlineto{\pgfqpoint{1.189539in}{0.752033in}}%
\pgfpathlineto{\pgfqpoint{1.201369in}{0.752625in}}%
\pgfpathlineto{\pgfqpoint{1.225030in}{0.756811in}}%
\pgfpathlineto{\pgfqpoint{1.236860in}{0.760212in}}%
\pgfpathlineto{\pgfqpoint{1.248690in}{0.761985in}}%
\pgfpathlineto{\pgfqpoint{1.331501in}{0.786522in}}%
\pgfpathlineto{\pgfqpoint{1.343331in}{0.787571in}}%
\pgfpathlineto{\pgfqpoint{1.355161in}{0.791180in}}%
\pgfpathlineto{\pgfqpoint{1.414312in}{0.799996in}}%
\pgfpathlineto{\pgfqpoint{1.508953in}{0.819505in}}%
\pgfpathlineto{\pgfqpoint{1.579933in}{0.833405in}}%
\pgfpathlineto{\pgfqpoint{1.591763in}{0.836644in}}%
\pgfpathlineto{\pgfqpoint{1.627254in}{0.842949in}}%
\pgfpathlineto{\pgfqpoint{1.650914in}{0.848149in}}%
\pgfpathlineto{\pgfqpoint{1.698235in}{0.856120in}}%
\pgfpathlineto{\pgfqpoint{1.733725in}{0.864531in}}%
\pgfpathlineto{\pgfqpoint{1.781045in}{0.871805in}}%
\pgfpathlineto{\pgfqpoint{1.816536in}{0.877059in}}%
\pgfpathlineto{\pgfqpoint{1.828366in}{0.880972in}}%
\pgfpathlineto{\pgfqpoint{1.852026in}{0.886407in}}%
\pgfpathlineto{\pgfqpoint{1.863856in}{0.889295in}}%
\pgfpathlineto{\pgfqpoint{1.875686in}{0.890652in}}%
\pgfpathlineto{\pgfqpoint{1.887516in}{0.893768in}}%
\pgfpathlineto{\pgfqpoint{2.005818in}{0.909792in}}%
\pgfpathlineto{\pgfqpoint{2.041308in}{0.914348in}}%
\pgfpathlineto{\pgfqpoint{2.053138in}{0.915962in}}%
\pgfpathlineto{\pgfqpoint{2.088629in}{0.923262in}}%
\pgfpathlineto{\pgfqpoint{2.100459in}{0.924581in}}%
\pgfpathlineto{\pgfqpoint{2.112289in}{0.927346in}}%
\pgfpathlineto{\pgfqpoint{2.135949in}{0.931596in}}%
\pgfpathlineto{\pgfqpoint{2.206930in}{0.945729in}}%
\pgfpathlineto{\pgfqpoint{2.218760in}{0.949043in}}%
\pgfpathlineto{\pgfqpoint{2.266080in}{0.955847in}}%
\pgfpathlineto{\pgfqpoint{2.325231in}{0.966417in}}%
\pgfpathlineto{\pgfqpoint{2.337061in}{0.969778in}}%
\pgfpathlineto{\pgfqpoint{2.372551in}{0.975505in}}%
\pgfpathlineto{\pgfqpoint{2.384382in}{0.978436in}}%
\pgfpathlineto{\pgfqpoint{2.431702in}{0.986107in}}%
\pgfpathlineto{\pgfqpoint{2.443532in}{0.990362in}}%
\pgfpathlineto{\pgfqpoint{2.467192in}{0.994756in}}%
\pgfpathlineto{\pgfqpoint{2.490853in}{0.999827in}}%
\pgfpathlineto{\pgfqpoint{2.550003in}{1.012076in}}%
\pgfpathlineto{\pgfqpoint{2.573664in}{1.016772in}}%
\pgfpathlineto{\pgfqpoint{2.703795in}{1.038191in}}%
\pgfpathlineto{\pgfqpoint{2.715625in}{1.041482in}}%
\pgfpathlineto{\pgfqpoint{2.774776in}{1.050551in}}%
\pgfpathlineto{\pgfqpoint{2.810266in}{1.055200in}}%
\pgfpathlineto{\pgfqpoint{2.869417in}{1.063943in}}%
\pgfpathlineto{\pgfqpoint{2.916737in}{1.072344in}}%
\pgfpathlineto{\pgfqpoint{3.117849in}{1.111600in}}%
\pgfpathlineto{\pgfqpoint{3.188830in}{1.123536in}}%
\pgfpathlineto{\pgfqpoint{3.236150in}{1.131322in}}%
\pgfpathlineto{\pgfqpoint{3.295301in}{1.142576in}}%
\pgfpathlineto{\pgfqpoint{3.342622in}{1.153577in}}%
\pgfpathlineto{\pgfqpoint{3.378112in}{1.160772in}}%
\pgfpathlineto{\pgfqpoint{3.425432in}{1.171626in}}%
\pgfpathlineto{\pgfqpoint{3.496413in}{1.183694in}}%
\pgfpathlineto{\pgfqpoint{3.520073in}{1.187141in}}%
\pgfpathlineto{\pgfqpoint{3.591054in}{1.195694in}}%
\pgfpathlineto{\pgfqpoint{3.614714in}{1.197691in}}%
\pgfpathlineto{\pgfqpoint{3.662035in}{1.203588in}}%
\pgfpathlineto{\pgfqpoint{3.673865in}{1.204558in}}%
\pgfpathlineto{\pgfqpoint{3.685695in}{1.207029in}}%
\pgfpathlineto{\pgfqpoint{3.697525in}{1.207710in}}%
\pgfpathlineto{\pgfqpoint{3.709355in}{1.209964in}}%
\pgfpathlineto{\pgfqpoint{3.744846in}{1.213443in}}%
\pgfpathlineto{\pgfqpoint{3.744846in}{1.213443in}}%
\pgfusepath{stroke}%
\end{pgfscope}%
\begin{pgfscope}%
\pgfpathrectangle{\pgfqpoint{0.550713in}{0.382904in}}{\pgfqpoint{3.194133in}{0.996173in}}%
\pgfusepath{clip}%
\pgfsetrectcap%
\pgfsetroundjoin%
\pgfsetlinewidth{0.853187pt}%
\definecolor{currentstroke}{rgb}{0.341176,0.670588,0.152941}%
\pgfsetstrokecolor{currentstroke}%
\pgfsetdash{}{0pt}%
\pgfpathmoveto{\pgfqpoint{0.562543in}{0.432016in}}%
\pgfpathlineto{\pgfqpoint{0.609863in}{0.448258in}}%
\pgfpathlineto{\pgfqpoint{0.657184in}{0.462699in}}%
\pgfpathlineto{\pgfqpoint{0.739995in}{0.488622in}}%
\pgfpathlineto{\pgfqpoint{0.787315in}{0.503194in}}%
\pgfpathlineto{\pgfqpoint{0.858296in}{0.526445in}}%
\pgfpathlineto{\pgfqpoint{0.905616in}{0.546639in}}%
\pgfpathlineto{\pgfqpoint{0.929277in}{0.559118in}}%
\pgfpathlineto{\pgfqpoint{0.964767in}{0.580992in}}%
\pgfpathlineto{\pgfqpoint{1.023918in}{0.598907in}}%
\pgfpathlineto{\pgfqpoint{1.047578in}{0.607570in}}%
\pgfpathlineto{\pgfqpoint{1.142219in}{0.613608in}}%
\pgfpathlineto{\pgfqpoint{1.165879in}{0.617317in}}%
\pgfpathlineto{\pgfqpoint{1.201369in}{0.623860in}}%
\pgfpathlineto{\pgfqpoint{1.272350in}{0.635232in}}%
\pgfpathlineto{\pgfqpoint{1.307840in}{0.637936in}}%
\pgfpathlineto{\pgfqpoint{1.319671in}{0.639725in}}%
\pgfpathlineto{\pgfqpoint{1.437972in}{0.645463in}}%
\pgfpathlineto{\pgfqpoint{1.532613in}{0.651948in}}%
\pgfpathlineto{\pgfqpoint{1.674574in}{0.656720in}}%
\pgfpathlineto{\pgfqpoint{1.710065in}{0.659813in}}%
\pgfpathlineto{\pgfqpoint{1.721895in}{0.661933in}}%
\pgfpathlineto{\pgfqpoint{1.781045in}{0.664861in}}%
\pgfpathlineto{\pgfqpoint{1.863856in}{0.668999in}}%
\pgfpathlineto{\pgfqpoint{1.946667in}{0.672218in}}%
\pgfpathlineto{\pgfqpoint{2.017648in}{0.674962in}}%
\pgfpathlineto{\pgfqpoint{2.159609in}{0.679191in}}%
\pgfpathlineto{\pgfqpoint{2.266080in}{0.681772in}}%
\pgfpathlineto{\pgfqpoint{2.360721in}{0.684199in}}%
\pgfpathlineto{\pgfqpoint{2.455362in}{0.687250in}}%
\pgfpathlineto{\pgfqpoint{2.550003in}{0.690806in}}%
\pgfpathlineto{\pgfqpoint{2.620984in}{0.694159in}}%
\pgfpathlineto{\pgfqpoint{2.703795in}{0.696600in}}%
\pgfpathlineto{\pgfqpoint{2.822096in}{0.701388in}}%
\pgfpathlineto{\pgfqpoint{3.035038in}{0.707664in}}%
\pgfpathlineto{\pgfqpoint{3.307131in}{0.718306in}}%
\pgfpathlineto{\pgfqpoint{3.437263in}{0.721065in}}%
\pgfpathlineto{\pgfqpoint{3.614714in}{0.726697in}}%
\pgfpathlineto{\pgfqpoint{3.744846in}{0.730991in}}%
\pgfpathlineto{\pgfqpoint{3.744846in}{0.730991in}}%
\pgfusepath{stroke}%
\end{pgfscope}%
\begin{pgfscope}%
\pgfpathrectangle{\pgfqpoint{0.550713in}{0.382904in}}{\pgfqpoint{3.194133in}{0.996173in}}%
\pgfusepath{clip}%
\pgfsetrectcap%
\pgfsetroundjoin%
\pgfsetlinewidth{0.853187pt}%
\definecolor{currentstroke}{rgb}{0.478431,0.435294,0.674510}%
\pgfsetstrokecolor{currentstroke}%
\pgfsetdash{}{0pt}%
\pgfpathmoveto{\pgfqpoint{0.562543in}{0.432023in}}%
\pgfpathlineto{\pgfqpoint{0.633523in}{0.456725in}}%
\pgfpathlineto{\pgfqpoint{0.763655in}{0.496936in}}%
\pgfpathlineto{\pgfqpoint{0.846466in}{0.524337in}}%
\pgfpathlineto{\pgfqpoint{0.881956in}{0.538434in}}%
\pgfpathlineto{\pgfqpoint{0.905616in}{0.549220in}}%
\pgfpathlineto{\pgfqpoint{0.976597in}{0.585092in}}%
\pgfpathlineto{\pgfqpoint{0.988427in}{0.590745in}}%
\pgfpathlineto{\pgfqpoint{1.012087in}{0.598281in}}%
\pgfpathlineto{\pgfqpoint{1.035748in}{0.606169in}}%
\pgfpathlineto{\pgfqpoint{1.083068in}{0.626202in}}%
\pgfpathlineto{\pgfqpoint{1.106728in}{0.628355in}}%
\pgfpathlineto{\pgfqpoint{1.130389in}{0.631527in}}%
\pgfpathlineto{\pgfqpoint{1.307840in}{0.642334in}}%
\pgfpathlineto{\pgfqpoint{1.366991in}{0.646341in}}%
\pgfpathlineto{\pgfqpoint{1.378821in}{0.648513in}}%
\pgfpathlineto{\pgfqpoint{1.437972in}{0.652276in}}%
\pgfpathlineto{\pgfqpoint{1.603594in}{0.660259in}}%
\pgfpathlineto{\pgfqpoint{1.627254in}{0.662406in}}%
\pgfpathlineto{\pgfqpoint{1.674574in}{0.668538in}}%
\pgfpathlineto{\pgfqpoint{1.769215in}{0.673766in}}%
\pgfpathlineto{\pgfqpoint{1.852026in}{0.677104in}}%
\pgfpathlineto{\pgfqpoint{1.934837in}{0.681255in}}%
\pgfpathlineto{\pgfqpoint{2.029478in}{0.686750in}}%
\pgfpathlineto{\pgfqpoint{2.124119in}{0.690425in}}%
\pgfpathlineto{\pgfqpoint{2.277911in}{0.699952in}}%
\pgfpathlineto{\pgfqpoint{2.360721in}{0.702742in}}%
\pgfpathlineto{\pgfqpoint{2.479023in}{0.711166in}}%
\pgfpathlineto{\pgfqpoint{2.526343in}{0.714735in}}%
\pgfpathlineto{\pgfqpoint{2.585494in}{0.716723in}}%
\pgfpathlineto{\pgfqpoint{2.715625in}{0.721496in}}%
\pgfpathlineto{\pgfqpoint{2.975888in}{0.731239in}}%
\pgfpathlineto{\pgfqpoint{3.106019in}{0.734330in}}%
\pgfpathlineto{\pgfqpoint{3.413602in}{0.743175in}}%
\pgfpathlineto{\pgfqpoint{3.638375in}{0.750073in}}%
\pgfpathlineto{\pgfqpoint{3.697525in}{0.753514in}}%
\pgfpathlineto{\pgfqpoint{3.733016in}{0.755110in}}%
\pgfpathlineto{\pgfqpoint{3.744846in}{0.755379in}}%
\pgfpathlineto{\pgfqpoint{3.744846in}{0.755379in}}%
\pgfusepath{stroke}%
\end{pgfscope}%
\begin{pgfscope}%
\pgfsetrectcap%
\pgfsetmiterjoin%
\pgfsetlinewidth{0.803000pt}%
\definecolor{currentstroke}{rgb}{0.000000,0.000000,0.000000}%
\pgfsetstrokecolor{currentstroke}%
\pgfsetdash{}{0pt}%
\pgfpathmoveto{\pgfqpoint{0.550713in}{0.382904in}}%
\pgfpathlineto{\pgfqpoint{0.550713in}{1.379077in}}%
\pgfusepath{stroke}%
\end{pgfscope}%
\begin{pgfscope}%
\pgfsetrectcap%
\pgfsetmiterjoin%
\pgfsetlinewidth{0.803000pt}%
\definecolor{currentstroke}{rgb}{0.000000,0.000000,0.000000}%
\pgfsetstrokecolor{currentstroke}%
\pgfsetdash{}{0pt}%
\pgfpathmoveto{\pgfqpoint{3.744846in}{0.382904in}}%
\pgfpathlineto{\pgfqpoint{3.744846in}{1.379077in}}%
\pgfusepath{stroke}%
\end{pgfscope}%
\begin{pgfscope}%
\pgfsetrectcap%
\pgfsetmiterjoin%
\pgfsetlinewidth{0.803000pt}%
\definecolor{currentstroke}{rgb}{0.000000,0.000000,0.000000}%
\pgfsetstrokecolor{currentstroke}%
\pgfsetdash{}{0pt}%
\pgfpathmoveto{\pgfqpoint{0.550713in}{0.382904in}}%
\pgfpathlineto{\pgfqpoint{3.744846in}{0.382904in}}%
\pgfusepath{stroke}%
\end{pgfscope}%
\begin{pgfscope}%
\pgfsetrectcap%
\pgfsetmiterjoin%
\pgfsetlinewidth{0.803000pt}%
\definecolor{currentstroke}{rgb}{0.000000,0.000000,0.000000}%
\pgfsetstrokecolor{currentstroke}%
\pgfsetdash{}{0pt}%
\pgfpathmoveto{\pgfqpoint{0.550713in}{1.379077in}}%
\pgfpathlineto{\pgfqpoint{3.744846in}{1.379077in}}%
\pgfusepath{stroke}%
\end{pgfscope}%
\begin{pgfscope}%
\pgfsetrectcap%
\pgfsetroundjoin%
\pgfsetlinewidth{0.853187pt}%
\definecolor{currentstroke}{rgb}{0.631373,0.062745,0.207843}%
\pgfsetstrokecolor{currentstroke}%
\pgfsetdash{}{0pt}%
\pgfpathmoveto{\pgfqpoint{3.869846in}{2.070188in}}%
\pgfpathlineto{\pgfqpoint{4.147623in}{2.070188in}}%
\pgfusepath{stroke}%
\end{pgfscope}%
\begin{pgfscope}%
\definecolor{textcolor}{rgb}{0.000000,0.000000,0.000000}%
\pgfsetstrokecolor{textcolor}%
\pgfsetfillcolor{textcolor}%
\pgftext[x=4.258735in,y=2.021576in,left,base]{\color{textcolor}\rmfamily\fontsize{10.000000}{12.000000}\selectfont TV-GP-UCB}%
\end{pgfscope}%
\begin{pgfscope}%
\pgfsetrectcap%
\pgfsetroundjoin%
\pgfsetlinewidth{0.853187pt}%
\definecolor{currentstroke}{rgb}{0.890196,0.000000,0.400000}%
\pgfsetstrokecolor{currentstroke}%
\pgfsetdash{}{0pt}%
\pgfpathmoveto{\pgfqpoint{3.869846in}{1.876577in}}%
\pgfpathlineto{\pgfqpoint{4.147623in}{1.876577in}}%
\pgfusepath{stroke}%
\end{pgfscope}%
\begin{pgfscope}%
\definecolor{textcolor}{rgb}{0.000000,0.000000,0.000000}%
\pgfsetstrokecolor{textcolor}%
\pgfsetfillcolor{textcolor}%
\pgftext[x=4.258735in,y=1.827965in,left,base]{\color{textcolor}\rmfamily\fontsize{10.000000}{12.000000}\selectfont SW TV-GP-UCB}%
\end{pgfscope}%
\begin{pgfscope}%
\pgfsetrectcap%
\pgfsetroundjoin%
\pgfsetlinewidth{0.853187pt}%
\definecolor{currentstroke}{rgb}{0.000000,0.329412,0.623529}%
\pgfsetstrokecolor{currentstroke}%
\pgfsetdash{}{0pt}%
\pgfpathmoveto{\pgfqpoint{3.869846in}{1.682966in}}%
\pgfpathlineto{\pgfqpoint{4.147623in}{1.682966in}}%
\pgfusepath{stroke}%
\end{pgfscope}%
\begin{pgfscope}%
\definecolor{textcolor}{rgb}{0.000000,0.000000,0.000000}%
\pgfsetstrokecolor{textcolor}%
\pgfsetfillcolor{textcolor}%
\pgftext[x=4.258735in,y=1.634354in,left,base]{\color{textcolor}\rmfamily\fontsize{10.000000}{12.000000}\selectfont UI-TVBO}%
\end{pgfscope}%
\begin{pgfscope}%
\pgfsetrectcap%
\pgfsetroundjoin%
\pgfsetlinewidth{0.853187pt}%
\definecolor{currentstroke}{rgb}{0.000000,0.380392,0.396078}%
\pgfsetstrokecolor{currentstroke}%
\pgfsetdash{}{0pt}%
\pgfpathmoveto{\pgfqpoint{3.869846in}{1.489355in}}%
\pgfpathlineto{\pgfqpoint{4.147623in}{1.489355in}}%
\pgfusepath{stroke}%
\end{pgfscope}%
\begin{pgfscope}%
\definecolor{textcolor}{rgb}{0.000000,0.000000,0.000000}%
\pgfsetstrokecolor{textcolor}%
\pgfsetfillcolor{textcolor}%
\pgftext[x=4.258735in,y=1.440744in,left,base]{\color{textcolor}\rmfamily\fontsize{10.000000}{12.000000}\selectfont B UI-TVBO}%
\end{pgfscope}%
\begin{pgfscope}%
\pgfsetrectcap%
\pgfsetroundjoin%
\pgfsetlinewidth{0.853187pt}%
\definecolor{currentstroke}{rgb}{0.380392,0.129412,0.345098}%
\pgfsetstrokecolor{currentstroke}%
\pgfsetdash{}{0pt}%
\pgfpathmoveto{\pgfqpoint{3.869846in}{1.295744in}}%
\pgfpathlineto{\pgfqpoint{4.147623in}{1.295744in}}%
\pgfusepath{stroke}%
\end{pgfscope}%
\begin{pgfscope}%
\definecolor{textcolor}{rgb}{0.000000,0.000000,0.000000}%
\pgfsetstrokecolor{textcolor}%
\pgfsetfillcolor{textcolor}%
\pgftext[x=4.258735in,y=1.247133in,left,base]{\color{textcolor}\rmfamily\fontsize{10.000000}{12.000000}\selectfont C-TV-GP-UCB}%
\end{pgfscope}%
\begin{pgfscope}%
\pgfsetrectcap%
\pgfsetroundjoin%
\pgfsetlinewidth{0.853187pt}%
\definecolor{currentstroke}{rgb}{0.964706,0.658824,0.000000}%
\pgfsetstrokecolor{currentstroke}%
\pgfsetdash{}{0pt}%
\pgfpathmoveto{\pgfqpoint{3.869846in}{1.102133in}}%
\pgfpathlineto{\pgfqpoint{4.147623in}{1.102133in}}%
\pgfusepath{stroke}%
\end{pgfscope}%
\begin{pgfscope}%
\definecolor{textcolor}{rgb}{0.000000,0.000000,0.000000}%
\pgfsetstrokecolor{textcolor}%
\pgfsetfillcolor{textcolor}%
\pgftext[x=4.258735in,y=1.053522in,left,base]{\color{textcolor}\rmfamily\fontsize{10.000000}{12.000000}\selectfont SW C-TV-GP-UCB}%
\end{pgfscope}%
\begin{pgfscope}%
\pgfsetrectcap%
\pgfsetroundjoin%
\pgfsetlinewidth{0.853187pt}%
\definecolor{currentstroke}{rgb}{0.341176,0.670588,0.152941}%
\pgfsetstrokecolor{currentstroke}%
\pgfsetdash{}{0pt}%
\pgfpathmoveto{\pgfqpoint{3.869846in}{0.908522in}}%
\pgfpathlineto{\pgfqpoint{4.147623in}{0.908522in}}%
\pgfusepath{stroke}%
\end{pgfscope}%
\begin{pgfscope}%
\definecolor{textcolor}{rgb}{0.000000,0.000000,0.000000}%
\pgfsetstrokecolor{textcolor}%
\pgfsetfillcolor{textcolor}%
\pgftext[x=4.258735in,y=0.859911in,left,base]{\color{textcolor}\rmfamily\fontsize{10.000000}{12.000000}\selectfont C-UI-TVBO}%
\end{pgfscope}%
\begin{pgfscope}%
\pgfsetrectcap%
\pgfsetroundjoin%
\pgfsetlinewidth{0.853187pt}%
\definecolor{currentstroke}{rgb}{0.478431,0.435294,0.674510}%
\pgfsetstrokecolor{currentstroke}%
\pgfsetdash{}{0pt}%
\pgfpathmoveto{\pgfqpoint{3.869846in}{0.714911in}}%
\pgfpathlineto{\pgfqpoint{4.147623in}{0.714911in}}%
\pgfusepath{stroke}%
\end{pgfscope}%
\begin{pgfscope}%
\definecolor{textcolor}{rgb}{0.000000,0.000000,0.000000}%
\pgfsetstrokecolor{textcolor}%
\pgfsetfillcolor{textcolor}%
\pgftext[x=4.258735in,y=0.666300in,left,base]{\color{textcolor}\rmfamily\fontsize{10.000000}{12.000000}\selectfont B C-UI-TVBO}%
\end{pgfscope}%
\end{pgfpicture}%
\makeatother%
\endgroup%

    \caption[Exploration and exploitation regret of the \gls{lqr} problem of an inverted pendulum.]{Exploration and exploitation regret of the \gls{lqr} problem of an inverted pendulum. The regret is split according to \eqref{eq:split_regret}.}
    \label{fig:LQR_split_regret}
\end{figure}

The upper subplot in Figure~\ref{fig:LQR_split_regret}, however, shows that the cost caused by exploration is significantly higher for the unconstrained variations as the constraints in \gls{ctvbo} limit the exploration as discussed in \Cref{sec:model_convex_functions}. Especially for the variations that use \gls{b2p} forgetting, the cost caused by excessive exploration is high compared to \gls{ui} forgetting. Furthermore, the tracking of the optimum is better with \gls{ui} forgetting, despite the lower exploration cost.

\section{Discussion}

Different synthetic experiments (\Cref{sec:synthetic_experiments}) and an application example (\Cref{sec:LQR}) were conducted to compare the variations, evaluate the proposed methods \gls{uitvbo} and \gls{ctvbo} presented in \Cref{chap:concept}, and test Hypotheses~\ref{hyp:ui_structural_information} though \ref{hyp:ctvbo}.
The hypotheses are individually revisited and discussed, and afterwards, further particularities that emerged in the results are discussed.

\subsubsection{Regarding Hypothesis~\ref{hyp:ui_structural_information}}

Hypothesis~\ref{hyp:ui_structural_information} stated that \gls{b2p} forgetting would be more sensitive to an optimistic prior mean compared to \gls{ui} forgetting, which would be reflected in higher regret. All synthetic experiments confirmed the hypothesis. Only the two-dimensional within- and out-of-model comparisons could not directly confirm the hypothesis. However, further investigations of the within-model comparison also showed higher sensitivity. Furthermore, it could be concluded that a pessimistic prior mean can also lead to an increased regret since it can restrict the exploratory behavior of \gls{b2p} forgetting too much and thus aggravating the learning of the temporal change. The \gls{lqr} application example showed that \gls{b2p} forgetting was very sensitive to the increase in cost over time reflected in increased exploration behavior as shown in Figure~\ref{fig:LQR_split_regret}. 

\subsubsection{Regarding Hypothesis~\ref{hyp:ui_good_mean}}

Hypothesis~\ref{hyp:ui_good_mean} stated, that given a well-defined prior mean, the regret of \gls{b2p} forgetting and \gls{ui} forgetting is comparable. This was also confirmed by the synthetic experiments of the moving parabola (\Cref{sec:1D,sec:2D}), which were designed for a quantitative comparison. Here, the optimal forgetting factors for \gls{b2p} and \gls{ui} forgetting were selected on the basis of a sensitivity analysis and both, \gls{b2p} and \gls{ui} forgetting, showed very similar performance in terms of regret for optimal forgetting parameters.

\subsubsection{Regarding Hypothesis~\ref{hyp:ctvbo}}

The final hypothesis, Hypothesis~\ref{hyp:ctvbo}, stated that \gls{ctvbo} would result in lower regret compared to standard \gls{tvbo}, regardless of the forgetting strategy, because it incorporates prior knowledge. This could not be confirmed, since for very flat objective functions the combination of \gls{b2p} forgetting and \gls{ctvbo} resulted in higher regret compared to standard \gls{tvbo}. The constraints and the flat posterior due to \gls{b2p} forgetting increased the sampling radius around the optimum such that the regret was higher than the one caused by occasional sampling at the bounds by unconstrained \gls{b2p} forgetting. However, for \gls{ui} forgetting it was demonstrated that the use of \gls{ctvbo} always results in lower regret compared to standard \gls{tvbo}. In contrast to \gls{b2p} forgetting, the posterior around the expected optimum does not propagate to the prior mean as described in \Cref{sec:1D}. Thus, the sampling radius is not significantly increased. 

\subsubsection{Further Observations}

It could be observed that the differences between standard \gls{tvbo} and \gls{ctvbo} were smaller in the two-dimensional examples than in the one-dimensional examples. One reason for this may be the smaller number of \glspl{vop} per dimension $N_{v/D}$, which enforce convexity. Further investigations would be necessary, e.g. reducing $N_{v/D}$ in the one-dimensional examples, in order to be able to confirm this assumption. Nevertheless, this also directly points out a deficiency of \gls{ctvbo}. If the sensitivity to $N_{v/D}$ is very high, methods other than proposed by \textcite{Agrell_2019} are needed to enforce the convexity of the posterior. Also, to scale \gls{ctvbo} to higher dimensions, other approaches would be needed.

While the main focus of the comparisons was on the forgetting strategies, the data selection strategies were also used for the individual variations. Here, it was shown that the size of a sliding window of $W=30$ is too small for \gls{b2p} forgetting in the one-dimensional experiments. This was expected since it was not chosen according to \eqref{eq:sliding_window} but to have a comparable number of training points as the binning approach. In the two-dimensional experiments, a sliding window size of $W=80$ already showed to be competitive. The chosen number of bins per dimension of $20$ for \gls{ui} forgetting turned out to be sufficient since, especially for \gls{ctvbo}, the regret was very similar to that of \gls{ui} forgetting without data selection strategy.

Lastly, it should be noted that learning hyperparameters in \gls{tvbo} is difficult because the samples are strongly correlated through the acquisition function and not iid, which is required for maximum likelihood approaches. Thus, bounds for the length scale had to be used to avoid length scales that were either too small or too large. Furthermore, the noise was fixed to capture the temporal change and not disregard them as noise.
The out-of-model comparisons in \Cref{sec:out_of_model} also showed that learning hyperparameters is even more challenging if \gls{b2p} forgetting is used due to the propagation of the expected value to the prior mean. The discussed data selection strategies improved learning the hyperparameters since they reduced the data set and neglected irrelevant data.

%%%%% Emacs-related stuff
%%% Local Variables: 
%%% mode: latex
%%% TeX-master: "../../main"
%%% End: 
