\chapter{Appendix}

\section{Numerically Stable Calculation of the Constrained Gaussian Process Posterior Distribution}
\label{apx:numerically_stable_factors}

A more stable calculation of the factors $A_i$, $B_i$ and $\Sigma$ is provided in \cite[Lemma~2]{Agrell_2019} using Cholesky factorization instead of calculating inverses. In the following, $\mathrm{chol}(P)$ is defined as the lower triangular Cholesky factor of $P$ and $X = (A \setminus B)$ as the solution to the linear system $AX = B$, which can be computed very efficiently if $A$ is rectangular.

Following now \cite[Lemma~2]{Agrell_2019}, let $L = \mathrm{chol}(K_{\mathbf{X},\mathbf{X}} + \sigma_v^2\mathbf{I})$, $v_1 = L \setminus \mathcal{L}K_{\mathbf{X}_v,\mathbf{X}}$ and $v_2 = L \setminus K_{\mathbf{X},\mathbf{X}_*}$. Then the factors \eqref{eq:A1} to \eqref{eq:B3} can be computed as 
\begin{align}
    A_1 &= \left(L^T \setminus v_1\right)^T \\
    A_2 &= \left(L^T \setminus v_2\right)^T \\
    B_1 &= \mathcal{L}K_{\mathbf{X}_v,\mathbf{X}_v}\mathcal{L}^T + \sigma_v^2 \mathbf{I} - v_1^T v_1 \\
    B_2 &= K_{\mathbf{X}_*,\mathbf{X}_*} - v_2^T v_2 \\
    B_3 &= K_{\mathbf{X}_*,\mathbf{X}_v}\mathcal{L}^T - v_2^T v_1.
\end{align}
Let $L_1 = \mathrm{chol}(B_1)$ and let $v_3 = L_1^T \setminus B_3^T$, than the final factors for the posterior distribution \eqref{eq:constrained_posterior_distribution} can be computed as
\begin{equation}
    A = \left(L_1^T \setminus v_3\right)^T, \quad B = A_2 - A A_1, \quad \Sigma = B_2 - v_3^T v_3.
\end{equation}
For the derivation and proof it is referred to \cite[Appendix~B]{Agrell_2019}.

\section{Sampling from the Constrained Gaussian Process Prior Distribution}
\label{apx:sampling_from_prior}

Sampling from the \gls{gp} prior distribution is similar to sampling from the posterior distribution as described in \cref{sec:sampling_posterior}. To sample from the constrained prior, the joint distribution of $\mathbf{y}$, $\mathbf{f_*}$ and $\tilde{C}$ in \eqref{eq:unconstrained_joint_distribution} has to be first conditioned on $\tilde{C}$. Afterwards, the prior distribution $\mathbf{f_*} \sim \mathcal{N}(\mu_{\mathbf{X}_*} + \hat{A} (\MatBold{\hat{C}} - \mathcal{L}\mu_{\mathbf{X}_v}), \hat{\Sigma})$ can be obtained through marginalizing out $\mathbf{f_*}$ from the conditioned joint distribution. The resulting multivariate normal distribution is compound Gaussian distribution with a truncated mean with the following factors

\begin{equation}
    \begin{aligned}[t]
    \hat{B}_1 &= \mathcal{L}K_{\mathbf{X}_v,\mathbf{X}_v}\mathcal{L}^T + \sigma_v^2\mathbf{I}\\
    \hat{B}_2 &= K_{ \mathbf{X}_*, \mathbf{X}_*} \\
    \hat{B}_3 &= K_{ \mathbf{X}_*,\mathbf{X}_v}\mathcal{L}^T \\
    \hat{L}_1 &= \mathrm{chol}(\hat{B}_1)
\end{aligned}
\qquad 
\begin{aligned}[t]
    \hat{v}_3 &= \hat{L}_1^T \setminus \hat{B}_3^T \\
    \hat{A} &= \left(\hat{L}_1^T \setminus \hat{v}_3\right)^T \\
    \hat{\Sigma} &= \hat{B}_2 - \hat{v}_3^T \hat{v}_3,
\end{aligned}
\end{equation}
and the truncated multivariate normal distribution
\begin{equation}
    \MatBold{\hat{C}}=\hat{C} | C \sim \mathcal{TN}\left(\mathbf{0}, \hat{B}_1, a( \mathbf{X}_v), b( \mathbf{X}_v) \right).
    \label{eq:constrained_prior}
\end{equation}
The algorithm for sampling from the prior is displayed in Algorithm~\ref{algo:constrained_prior} below.

\begin{algorithm}[h]
\centering
\caption{Sampling form the constrained prior distribution}
\begin{algorithmic}[1]
\Require Calculate factors $\hat{A}$, $\hat{\Sigma}$, $\hat{B}_1$
\State Find a matrix $\MatBold{Q}$ s.t. $\MatBold{Q}^T \MatBold{Q} = \Sigma \in \R^{M \times M}$ using Cholesky decomposition.
\State Generate $\hat{\MatBold{C}}_k$, a $N_v \times k$ matrix where each column is a sample of $\hat{C} | C$ from the truncated multivariate normal distribution \eqref{eq:constrained_prior}.
\State Generate $\MatBold{U}_k$, a $M \times k$ matrix with k samples of the multivariate standard normal distribution $\mathcal{N}(\mathbf{0}, \mathbf{I}_M)$ with $\mathbf{I}_M \in \R^{M \times M}$.
\State Calculate the $M \times k$ matrix where each column is a sample from the distribution $\mathbf{f_*} | C$ as
\begin{equation}
    \mu_{ \mathbf{X}_*} \oplus \left[A(- \mathcal{L}\mu_{ \mathbf{X}_v} \oplus \tilde{\MatBold{C}}_k) +  \MatBold{Q}\MatBold{U}_k \right]
\end{equation}
with $\oplus$ representing the operation of adding the $M \times 1$ vector on the left hand side to each column of the $M\times k$ matrix on the right hand side.
\end{algorithmic}
\label{algo:constrained_prior}
\end{algorithm}

\section{Derivatives of the Squared-Exponential Kernel}
\label{apx:derivatives}

To constrain the \gls{gp} posterior, the partial derivatives of the spatial kernel are needed. Following are the partial derivatives of the \gls{se} kernel

\begin{equation}
    k(\mathbf{x}, \mathbf{x}') = \sigma_k^2 \exp\left(-\frac{1}{2} (\mathbf{x} - \mathbf{x}')^T \MatBold{\Lambda}^{-1} (\mathbf{x} - \mathbf{x}')\right), \quad \MatBold{\Lambda} = \begin{bmatrix*}[c]
                            \MatBold{\Lambda}_{11} &  \cdots & 0\\
                            \vdots & \ddots & \vdots\\
                            0 &  \cdots & \MatBold{\Lambda}_{DD}
                        \end{bmatrix*}.
\end{equation}


\subsubsection{Second derivative w.r.t. $x'_{j}$:}
\begin{equation}
    \frac{\partial^2 k(\mathbf{x}, \mathbf{x}')}{\partial {x'}_j^2}  = \Lambda_{jj}^{-1} \Big(\hat{\mathrm{d}}_j^2(\mathbf{x}, \mathbf{x}')  - 1\Big) k(\mathbf{x}, \mathbf{x}')
\end{equation}

\subsubsection{Second derivative w.r.t. $x_{j}$ and $x'_{j}$ (diagonal elements of $\mathcal{L} K_{*,*} \mathcal{L}^T$):}
\begin{equation}
    \frac{\partial^4 k(\mathbf{x}, \mathbf{x}')}{\partial x_{j}^2 \partial {x'}_{j}^2}  = \Lambda_{jj}^{-2} \Big(\hat{\mathrm{d}}_j^2(\mathbf{x}, \mathbf{x}') \,\hat{\mathrm{d}}_j^2(\mathbf{x}, \mathbf{x}')- 6\, \hat{\mathrm{d}}_j^2(\mathbf{x}, \mathbf{x}') +3 \Big) k(\mathbf{x}, \mathbf{x}')
\end{equation}
% \begin{equation}
% \begin{split}
%     \frac{\partial^4 k(\mathbf{x}_1, \mathbf{x}_2)}{\partial x_{1,i}^2 \partial x_{2,j}^2}  = \Lambda_{ii}^{-1}\Lambda_{jj}^{-1} \Big(&\hat{\mathrm{d}}_i^2(\mathbf{x}_1, \mathbf{x}_2)\, \hat{\mathrm{d}}_j^2(\mathbf{x}_1, \MatBold{x}_2) - \hat{\mathrm{d}}_i^2(\MatBold{x}_1, \MatBold{x}_2) \\
%     &- \hat{\mathrm{d}}_j^2(\MatBold{x}_1, \MatBold{x}_2) +1 \Big) k(\MatBold{x}_1, \MatBold{x}_2)
% \end{split}
% \end{equation}
\subsubsection{Second derivative w.r.t. $x_{i}$ and $x'_{j}$ (off-diagonal elements of $\mathcal{L} K_{*,*} \mathcal{L}^T$):}
\begin{equation}
    \frac{\partial^4 k(\mathbf{x}, \mathbf{x}')}{\partial x_{i}^2 \partial {x'}_{j}^2}  = \Lambda_{ii}^{-1}\Lambda_{jj}^{-1} \Big(\hat{\mathrm{d}}_i^2(\mathbf{x}, \mathbf{x}')\, \hat{\mathrm{d}}_j^2(\mathbf{x}, \MatBold{x}') - \hat{\mathrm{d}}_i^2(\MatBold{x}, \MatBold{x}') - \hat{\mathrm{d}}_j^2(\MatBold{x}, \MatBold{x}') +1 \Big) k(\MatBold{x}, \MatBold{x}')
\end{equation}

with the squared distance in dimension $k$ normalized by the corresponding lengthscale $\hat{\mathrm{d}}_k^2(\MatBold{x}, \MatBold{x}') = \Lambda_{kk}^{-1} (x_{k} - x'_{k})^2$.


\section{Correlation between Forgetting Factors}
\label{apx:forgetting_factors}

The forgetting factors of \gls{b2p} forgetting as in $k_{T,tv}$ and \gls{ui} forgetting as in $k_{T,wp}$ both imply the variance for $\tau=0$ after one time step after observing a measurement. This is shown below for one training point $x$ at time step $t_1$ and a test points $x_*$ at time step $t_2$ with $\tau = x-x_* = 0$.

\subsubsection{Back-2-Prior Forgetting}

Posterior covariance using the temporal kernel $k_{T,tv}$, $\tau = 0$, $t_2>t_1$, and $\Delta t = 1$:
\begin{align}
    \sigma_k^2 \cdot &(1-\epsilon)^{\frac{|t_2-t_2|}{2}} - \sigma_k^2 \cdot (1-\epsilon)^{\frac{|t_2-t_1|}{2}} \left[ \sigma_k^2 \cdot (1-\epsilon)^{\frac{|t_1-t_1|}{2}} \right]^{-1} \sigma_k^2 \cdot (1-\epsilon)^{\frac{|t_1-t_2|}{2}} \\
    &=\sigma_k^2 - \sigma_k^2 \cdot (1-\epsilon)^{\frac{|t_2-t_1|}{2}} \,(1-\epsilon)^{\frac{|t_1-t_2|}{2}} \\
    &=\sigma_k^2 - \sigma_k^2 \cdot (1-\epsilon)^{|t_2-t_1|} \\
    &= \sigma_k^2 \cdot \epsilon  \quad \text{with } \Delta t = 1
\end{align}
For $\sigma_k^2 = 1$ it can bee seen, that the variance after one time step is $\epsilon$.

\subsubsection{Uncertainty-Injection Forgetting}

Posterior covariance using the temporal kernel $k_{T,wp}$, $\tau = 0$, $t_2>t_1$, and $\Delta t = 1$::
\begin{align}
    \sigma_k^2 \cdot &\sigma_w^2 (\min(t_2,t_2)-c_0)  - \frac{\sigma_k^2 \cdot \sigma_w^2 (\min(t_2,t_1)-c_0) \cdot\sigma_k^2 \cdot \sigma_w^2 (\min(t_2,t_1)-c_0)}{\sigma_k^2 \cdot \sigma_w^2 (\min(t_1,t_1)-c_0)}  \\
    &=\sigma_k^2 \cdot \sigma_w^2 (\min(t_2,t_2)-c_0) - \sigma_k^2 \cdot \sigma_w^2 (\min(t_2,t_1)-c_0) \\
    &=\sigma_k^2 \cdot \sigma_w^2 (t_2-c_0) - \sigma_k^2 \cdot \sigma_w^2 (t_1-c_0) \\
    &= \sigma_k^2 \cdot \sigma_w^2 (t_2-t_1) \\
    &= \sigma_k^2 \cdot \sigma_w^2 = \hat{\sigma}_w^2 \quad \text{with } \Delta t = 1
\end{align}
For $\sigma_k^2 = 1$ it can bee seen, that the variance after one time step is $\sigma_w^2 = \hat{\sigma}_w^2$.

\newpage
\section{Trajectories of the 1-D Moving Parabola}
\label{apx:trajectories_1D_parabola}

\begin{figure}[h]
    \centering
    \input{thesis/figures/pgf_figures/Parabola1D_B2P_unconstrained.pgf}
    \caption[Trajectory of unconstrained \gls{b2p} forgetting for the one-dimensional moving parabola.]{Trajectory of unconstrained \gls{b2p} forgetting ($\epsilon=0.028$) for the one-dimensional moving parabola. The white circles denote the initial training data.}
    \label{fig:Parabola1D_B2P_unconstrained}
\end{figure}
\begin{figure}[b]
    \centering
    \vspace{-5cm}
    \input{thesis/figures/pgf_figures/Parabola1D_B2P_constrained.pgf}
    \caption[Trajectory of constrained \gls{b2p} forgetting for the one-dimensional moving parabola.]{Trajectory of constrained \gls{b2p} forgetting ($\epsilon=0.009$) for the one-dimensional moving parabola. The white circles denote the initial training data.}
    \label{fig:Parabola1D_B2P_constrained}
\end{figure}

\begin{figure}[h]
    \centering
    %% Creator: Matplotlib, PGF backend
%%
%% To include the figure in your LaTeX document, write
%%   \input{<filename>.pgf}
%%
%% Make sure the required packages are loaded in your preamble
%%   \usepackage{pgf}
%%
%% Figures using additional raster images can only be included by \input if
%% they are in the same directory as the main LaTeX file. For loading figures
%% from other directories you can use the `import` package
%%   \usepackage{import}
%%
%% and then include the figures with
%%   \import{<path to file>}{<filename>.pgf}
%%
%% Matplotlib used the following preamble
%%   \usepackage{fontspec}
%%
\begingroup%
\makeatletter%
\begin{pgfpicture}%
\pgfpathrectangle{\pgfpointorigin}{\pgfqpoint{5.507126in}{2.552693in}}%
\pgfusepath{use as bounding box, clip}%
\begin{pgfscope}%
\pgfsetbuttcap%
\pgfsetmiterjoin%
\definecolor{currentfill}{rgb}{1.000000,1.000000,1.000000}%
\pgfsetfillcolor{currentfill}%
\pgfsetlinewidth{0.000000pt}%
\definecolor{currentstroke}{rgb}{1.000000,1.000000,1.000000}%
\pgfsetstrokecolor{currentstroke}%
\pgfsetdash{}{0pt}%
\pgfpathmoveto{\pgfqpoint{0.000000in}{0.000000in}}%
\pgfpathlineto{\pgfqpoint{5.507126in}{0.000000in}}%
\pgfpathlineto{\pgfqpoint{5.507126in}{2.552693in}}%
\pgfpathlineto{\pgfqpoint{0.000000in}{2.552693in}}%
\pgfpathclose%
\pgfusepath{fill}%
\end{pgfscope}%
\begin{pgfscope}%
\pgfsetbuttcap%
\pgfsetmiterjoin%
\definecolor{currentfill}{rgb}{1.000000,1.000000,1.000000}%
\pgfsetfillcolor{currentfill}%
\pgfsetlinewidth{0.000000pt}%
\definecolor{currentstroke}{rgb}{0.000000,0.000000,0.000000}%
\pgfsetstrokecolor{currentstroke}%
\pgfsetstrokeopacity{0.000000}%
\pgfsetdash{}{0pt}%
\pgfpathmoveto{\pgfqpoint{0.605784in}{0.382904in}}%
\pgfpathlineto{\pgfqpoint{4.669272in}{0.382904in}}%
\pgfpathlineto{\pgfqpoint{4.669272in}{2.425059in}}%
\pgfpathlineto{\pgfqpoint{0.605784in}{2.425059in}}%
\pgfpathclose%
\pgfusepath{fill}%
\end{pgfscope}%
\begin{pgfscope}%
\pgfpathrectangle{\pgfqpoint{0.605784in}{0.382904in}}{\pgfqpoint{4.063488in}{2.042155in}}%
\pgfusepath{clip}%
\pgfsetbuttcap%
\pgfsetroundjoin%
\definecolor{currentfill}{rgb}{0.267004,0.004874,0.329415}%
\pgfsetfillcolor{currentfill}%
\pgfsetlinewidth{1.003750pt}%
\definecolor{currentstroke}{rgb}{0.267004,0.004874,0.329415}%
\pgfsetstrokecolor{currentstroke}%
\pgfsetdash{}{0pt}%
\pgfsys@defobject{currentmarker}{\pgfqpoint{1.981347in}{1.748620in}}{\pgfqpoint{2.236483in}{2.066156in}}{%
\pgfpathmoveto{\pgfqpoint{1.997736in}{1.764968in}}%
\pgfpathlineto{\pgfqpoint{1.987498in}{1.785596in}}%
\pgfpathlineto{\pgfqpoint{1.982648in}{1.806224in}}%
\pgfpathlineto{\pgfqpoint{1.981347in}{1.826852in}}%
\pgfpathlineto{\pgfqpoint{1.982509in}{1.847480in}}%
\pgfpathlineto{\pgfqpoint{1.985449in}{1.868107in}}%
\pgfpathlineto{\pgfqpoint{1.989715in}{1.888735in}}%
\pgfpathlineto{\pgfqpoint{1.994996in}{1.909363in}}%
\pgfpathlineto{\pgfqpoint{2.001071in}{1.929991in}}%
\pgfpathlineto{\pgfqpoint{2.001325in}{1.930723in}}%
\pgfpathlineto{\pgfqpoint{2.011779in}{1.950619in}}%
\pgfpathlineto{\pgfqpoint{2.023073in}{1.971247in}}%
\pgfpathlineto{\pgfqpoint{2.034694in}{1.991874in}}%
\pgfpathlineto{\pgfqpoint{2.042371in}{2.004864in}}%
\pgfpathlineto{\pgfqpoint{2.049487in}{2.012502in}}%
\pgfpathlineto{\pgfqpoint{2.069643in}{2.033130in}}%
\pgfpathlineto{\pgfqpoint{2.083416in}{2.046973in}}%
\pgfpathlineto{\pgfqpoint{2.097367in}{2.053758in}}%
\pgfpathlineto{\pgfqpoint{2.124461in}{2.066156in}}%
\pgfpathlineto{\pgfqpoint{2.165507in}{2.060743in}}%
\pgfpathlineto{\pgfqpoint{2.173373in}{2.053758in}}%
\pgfpathlineto{\pgfqpoint{2.194269in}{2.033130in}}%
\pgfpathlineto{\pgfqpoint{2.206552in}{2.019521in}}%
\pgfpathlineto{\pgfqpoint{2.209704in}{2.012502in}}%
\pgfpathlineto{\pgfqpoint{2.217669in}{1.991874in}}%
\pgfpathlineto{\pgfqpoint{2.224397in}{1.971247in}}%
\pgfpathlineto{\pgfqpoint{2.229777in}{1.950619in}}%
\pgfpathlineto{\pgfqpoint{2.233683in}{1.929991in}}%
\pgfpathlineto{\pgfqpoint{2.235972in}{1.909363in}}%
\pgfpathlineto{\pgfqpoint{2.236483in}{1.888735in}}%
\pgfpathlineto{\pgfqpoint{2.235033in}{1.868107in}}%
\pgfpathlineto{\pgfqpoint{2.231412in}{1.847480in}}%
\pgfpathlineto{\pgfqpoint{2.225381in}{1.826852in}}%
\pgfpathlineto{\pgfqpoint{2.216661in}{1.806224in}}%
\pgfpathlineto{\pgfqpoint{2.206552in}{1.788350in}}%
\pgfpathlineto{\pgfqpoint{2.203094in}{1.785596in}}%
\pgfpathlineto{\pgfqpoint{2.170667in}{1.764968in}}%
\pgfpathlineto{\pgfqpoint{2.165507in}{1.762261in}}%
\pgfpathlineto{\pgfqpoint{2.124461in}{1.754160in}}%
\pgfpathlineto{\pgfqpoint{2.083416in}{1.750254in}}%
\pgfpathlineto{\pgfqpoint{2.042371in}{1.748620in}}%
\pgfpathlineto{\pgfqpoint{2.001325in}{1.760437in}}%
\pgfpathclose%
\pgfusepath{stroke,fill}%
}%
\begin{pgfscope}%
\pgfsys@transformshift{0.000000in}{0.000000in}%
\pgfsys@useobject{currentmarker}{}%
\end{pgfscope}%
\end{pgfscope}%
\begin{pgfscope}%
\pgfpathrectangle{\pgfqpoint{0.605784in}{0.382904in}}{\pgfqpoint{4.063488in}{2.042155in}}%
\pgfusepath{clip}%
\pgfsetbuttcap%
\pgfsetroundjoin%
\definecolor{currentfill}{rgb}{0.277941,0.056324,0.381191}%
\pgfsetfillcolor{currentfill}%
\pgfsetlinewidth{1.003750pt}%
\definecolor{currentstroke}{rgb}{0.277941,0.056324,0.381191}%
\pgfsetstrokecolor{currentstroke}%
\pgfsetdash{}{0pt}%
\pgfpathmoveto{\pgfqpoint{0.625055in}{1.042994in}}%
\pgfpathlineto{\pgfqpoint{0.605784in}{1.048119in}}%
\pgfpathlineto{\pgfqpoint{0.605784in}{1.063622in}}%
\pgfpathlineto{\pgfqpoint{0.605784in}{1.084250in}}%
\pgfpathlineto{\pgfqpoint{0.605784in}{1.104878in}}%
\pgfpathlineto{\pgfqpoint{0.605784in}{1.125506in}}%
\pgfpathlineto{\pgfqpoint{0.605784in}{1.146134in}}%
\pgfpathlineto{\pgfqpoint{0.605784in}{1.166761in}}%
\pgfpathlineto{\pgfqpoint{0.605784in}{1.187389in}}%
\pgfpathlineto{\pgfqpoint{0.605784in}{1.208017in}}%
\pgfpathlineto{\pgfqpoint{0.605784in}{1.228645in}}%
\pgfpathlineto{\pgfqpoint{0.605784in}{1.249273in}}%
\pgfpathlineto{\pgfqpoint{0.605784in}{1.269900in}}%
\pgfpathlineto{\pgfqpoint{0.605784in}{1.290528in}}%
\pgfpathlineto{\pgfqpoint{0.605784in}{1.311156in}}%
\pgfpathlineto{\pgfqpoint{0.605784in}{1.331784in}}%
\pgfpathlineto{\pgfqpoint{0.605784in}{1.352412in}}%
\pgfpathlineto{\pgfqpoint{0.605784in}{1.373040in}}%
\pgfpathlineto{\pgfqpoint{0.605784in}{1.393667in}}%
\pgfpathlineto{\pgfqpoint{0.605784in}{1.414295in}}%
\pgfpathlineto{\pgfqpoint{0.605784in}{1.434923in}}%
\pgfpathlineto{\pgfqpoint{0.605784in}{1.455551in}}%
\pgfpathlineto{\pgfqpoint{0.605784in}{1.476179in}}%
\pgfpathlineto{\pgfqpoint{0.605784in}{1.496807in}}%
\pgfpathlineto{\pgfqpoint{0.605784in}{1.517434in}}%
\pgfpathlineto{\pgfqpoint{0.605784in}{1.532937in}}%
\pgfpathlineto{\pgfqpoint{0.627507in}{1.517434in}}%
\pgfpathlineto{\pgfqpoint{0.646829in}{1.503614in}}%
\pgfpathlineto{\pgfqpoint{0.653540in}{1.496807in}}%
\pgfpathlineto{\pgfqpoint{0.673037in}{1.476179in}}%
\pgfpathlineto{\pgfqpoint{0.687875in}{1.460008in}}%
\pgfpathlineto{\pgfqpoint{0.691769in}{1.455551in}}%
\pgfpathlineto{\pgfqpoint{0.708519in}{1.434923in}}%
\pgfpathlineto{\pgfqpoint{0.724243in}{1.414295in}}%
\pgfpathlineto{\pgfqpoint{0.728920in}{1.407586in}}%
\pgfpathlineto{\pgfqpoint{0.741211in}{1.393667in}}%
\pgfpathlineto{\pgfqpoint{0.757335in}{1.373040in}}%
\pgfpathlineto{\pgfqpoint{0.769965in}{1.354454in}}%
\pgfpathlineto{\pgfqpoint{0.772898in}{1.352412in}}%
\pgfpathlineto{\pgfqpoint{0.796522in}{1.331784in}}%
\pgfpathlineto{\pgfqpoint{0.811011in}{1.314631in}}%
\pgfpathlineto{\pgfqpoint{0.852056in}{1.328910in}}%
\pgfpathlineto{\pgfqpoint{0.853335in}{1.331784in}}%
\pgfpathlineto{\pgfqpoint{0.864476in}{1.352412in}}%
\pgfpathlineto{\pgfqpoint{0.876391in}{1.373040in}}%
\pgfpathlineto{\pgfqpoint{0.888896in}{1.393667in}}%
\pgfpathlineto{\pgfqpoint{0.893101in}{1.400019in}}%
\pgfpathlineto{\pgfqpoint{0.899399in}{1.414295in}}%
\pgfpathlineto{\pgfqpoint{0.909035in}{1.434923in}}%
\pgfpathlineto{\pgfqpoint{0.918937in}{1.455551in}}%
\pgfpathlineto{\pgfqpoint{0.929058in}{1.476179in}}%
\pgfpathlineto{\pgfqpoint{0.934147in}{1.486076in}}%
\pgfpathlineto{\pgfqpoint{0.939145in}{1.496807in}}%
\pgfpathlineto{\pgfqpoint{0.949060in}{1.517434in}}%
\pgfpathlineto{\pgfqpoint{0.958976in}{1.538062in}}%
\pgfpathlineto{\pgfqpoint{0.968892in}{1.558690in}}%
\pgfpathlineto{\pgfqpoint{0.975192in}{1.571547in}}%
\pgfpathlineto{\pgfqpoint{0.979294in}{1.579318in}}%
\pgfpathlineto{\pgfqpoint{0.990364in}{1.599946in}}%
\pgfpathlineto{\pgfqpoint{1.001210in}{1.620573in}}%
\pgfpathlineto{\pgfqpoint{1.011867in}{1.641201in}}%
\pgfpathlineto{\pgfqpoint{1.016237in}{1.649535in}}%
\pgfpathlineto{\pgfqpoint{1.024069in}{1.661829in}}%
\pgfpathlineto{\pgfqpoint{1.037008in}{1.682457in}}%
\pgfpathlineto{\pgfqpoint{1.049473in}{1.703085in}}%
\pgfpathlineto{\pgfqpoint{1.057283in}{1.716168in}}%
\pgfpathlineto{\pgfqpoint{1.063291in}{1.723713in}}%
\pgfpathlineto{\pgfqpoint{1.079443in}{1.744340in}}%
\pgfpathlineto{\pgfqpoint{1.094702in}{1.764968in}}%
\pgfpathlineto{\pgfqpoint{1.098328in}{1.769879in}}%
\pgfpathlineto{\pgfqpoint{1.114518in}{1.785596in}}%
\pgfpathlineto{\pgfqpoint{1.134520in}{1.806224in}}%
\pgfpathlineto{\pgfqpoint{1.139373in}{1.811309in}}%
\pgfpathlineto{\pgfqpoint{1.159959in}{1.826852in}}%
\pgfpathlineto{\pgfqpoint{1.180419in}{1.843097in}}%
\pgfpathlineto{\pgfqpoint{1.187617in}{1.847480in}}%
\pgfpathlineto{\pgfqpoint{1.221023in}{1.868107in}}%
\pgfpathlineto{\pgfqpoint{1.221464in}{1.868379in}}%
\pgfpathlineto{\pgfqpoint{1.262509in}{1.888279in}}%
\pgfpathlineto{\pgfqpoint{1.263886in}{1.888735in}}%
\pgfpathlineto{\pgfqpoint{1.303555in}{1.901076in}}%
\pgfpathlineto{\pgfqpoint{1.344600in}{1.903054in}}%
\pgfpathlineto{\pgfqpoint{1.385645in}{1.889543in}}%
\pgfpathlineto{\pgfqpoint{1.386679in}{1.888735in}}%
\pgfpathlineto{\pgfqpoint{1.412287in}{1.868107in}}%
\pgfpathlineto{\pgfqpoint{1.426691in}{1.856377in}}%
\pgfpathlineto{\pgfqpoint{1.433583in}{1.847480in}}%
\pgfpathlineto{\pgfqpoint{1.448993in}{1.826852in}}%
\pgfpathlineto{\pgfqpoint{1.464066in}{1.806224in}}%
\pgfpathlineto{\pgfqpoint{1.467736in}{1.800949in}}%
\pgfpathlineto{\pgfqpoint{1.476007in}{1.785596in}}%
\pgfpathlineto{\pgfqpoint{1.486546in}{1.764968in}}%
\pgfpathlineto{\pgfqpoint{1.496539in}{1.744340in}}%
\pgfpathlineto{\pgfqpoint{1.505944in}{1.723713in}}%
\pgfpathlineto{\pgfqpoint{1.508781in}{1.716925in}}%
\pgfpathlineto{\pgfqpoint{1.514073in}{1.703085in}}%
\pgfpathlineto{\pgfqpoint{1.521190in}{1.682457in}}%
\pgfpathlineto{\pgfqpoint{1.527543in}{1.661829in}}%
\pgfpathlineto{\pgfqpoint{1.533088in}{1.641201in}}%
\pgfpathlineto{\pgfqpoint{1.537775in}{1.620573in}}%
\pgfpathlineto{\pgfqpoint{1.541553in}{1.599946in}}%
\pgfpathlineto{\pgfqpoint{1.544363in}{1.579318in}}%
\pgfpathlineto{\pgfqpoint{1.546145in}{1.558690in}}%
\pgfpathlineto{\pgfqpoint{1.546831in}{1.538062in}}%
\pgfpathlineto{\pgfqpoint{1.546348in}{1.517434in}}%
\pgfpathlineto{\pgfqpoint{1.544616in}{1.496807in}}%
\pgfpathlineto{\pgfqpoint{1.541547in}{1.476179in}}%
\pgfpathlineto{\pgfqpoint{1.537047in}{1.455551in}}%
\pgfpathlineto{\pgfqpoint{1.531010in}{1.434923in}}%
\pgfpathlineto{\pgfqpoint{1.523321in}{1.414295in}}%
\pgfpathlineto{\pgfqpoint{1.513852in}{1.393667in}}%
\pgfpathlineto{\pgfqpoint{1.508781in}{1.384351in}}%
\pgfpathlineto{\pgfqpoint{1.500265in}{1.373040in}}%
\pgfpathlineto{\pgfqpoint{1.481616in}{1.352412in}}%
\pgfpathlineto{\pgfqpoint{1.467736in}{1.339492in}}%
\pgfpathlineto{\pgfqpoint{1.447860in}{1.331784in}}%
\pgfpathlineto{\pgfqpoint{1.426691in}{1.324826in}}%
\pgfpathlineto{\pgfqpoint{1.385645in}{1.328018in}}%
\pgfpathlineto{\pgfqpoint{1.372849in}{1.331784in}}%
\pgfpathlineto{\pgfqpoint{1.344600in}{1.340326in}}%
\pgfpathlineto{\pgfqpoint{1.304609in}{1.352412in}}%
\pgfpathlineto{\pgfqpoint{1.303555in}{1.352749in}}%
\pgfpathlineto{\pgfqpoint{1.262509in}{1.357783in}}%
\pgfpathlineto{\pgfqpoint{1.237395in}{1.352412in}}%
\pgfpathlineto{\pgfqpoint{1.221464in}{1.349086in}}%
\pgfpathlineto{\pgfqpoint{1.191419in}{1.331784in}}%
\pgfpathlineto{\pgfqpoint{1.180419in}{1.325357in}}%
\pgfpathlineto{\pgfqpoint{1.163647in}{1.311156in}}%
\pgfpathlineto{\pgfqpoint{1.140417in}{1.290528in}}%
\pgfpathlineto{\pgfqpoint{1.139373in}{1.289604in}}%
\pgfpathlineto{\pgfqpoint{1.118747in}{1.269900in}}%
\pgfpathlineto{\pgfqpoint{1.098754in}{1.249273in}}%
\pgfpathlineto{\pgfqpoint{1.098328in}{1.248833in}}%
\pgfpathlineto{\pgfqpoint{1.075669in}{1.228645in}}%
\pgfpathlineto{\pgfqpoint{1.057283in}{1.210567in}}%
\pgfpathlineto{\pgfqpoint{1.053372in}{1.208017in}}%
\pgfpathlineto{\pgfqpoint{1.024090in}{1.187389in}}%
\pgfpathlineto{\pgfqpoint{1.016237in}{1.181328in}}%
\pgfpathlineto{\pgfqpoint{0.978103in}{1.166761in}}%
\pgfpathlineto{\pgfqpoint{0.975192in}{1.165512in}}%
\pgfpathlineto{\pgfqpoint{0.934147in}{1.165585in}}%
\pgfpathlineto{\pgfqpoint{0.929382in}{1.166761in}}%
\pgfpathlineto{\pgfqpoint{0.893101in}{1.180197in}}%
\pgfpathlineto{\pgfqpoint{0.870551in}{1.187389in}}%
\pgfpathlineto{\pgfqpoint{0.852056in}{1.197592in}}%
\pgfpathlineto{\pgfqpoint{0.831523in}{1.187389in}}%
\pgfpathlineto{\pgfqpoint{0.811011in}{1.177728in}}%
\pgfpathlineto{\pgfqpoint{0.804511in}{1.166761in}}%
\pgfpathlineto{\pgfqpoint{0.789256in}{1.146134in}}%
\pgfpathlineto{\pgfqpoint{0.771380in}{1.125506in}}%
\pgfpathlineto{\pgfqpoint{0.769965in}{1.124103in}}%
\pgfpathlineto{\pgfqpoint{0.755254in}{1.104878in}}%
\pgfpathlineto{\pgfqpoint{0.736539in}{1.084250in}}%
\pgfpathlineto{\pgfqpoint{0.728920in}{1.076977in}}%
\pgfpathlineto{\pgfqpoint{0.712494in}{1.063622in}}%
\pgfpathlineto{\pgfqpoint{0.687875in}{1.046783in}}%
\pgfpathlineto{\pgfqpoint{0.673279in}{1.042994in}}%
\pgfpathlineto{\pgfqpoint{0.646829in}{1.037216in}}%
\pgfpathclose%
\pgfusepath{stroke,fill}%
\end{pgfscope}%
\begin{pgfscope}%
\pgfpathrectangle{\pgfqpoint{0.605784in}{0.382904in}}{\pgfqpoint{4.063488in}{2.042155in}}%
\pgfusepath{clip}%
\pgfsetbuttcap%
\pgfsetroundjoin%
\definecolor{currentfill}{rgb}{0.277941,0.056324,0.381191}%
\pgfsetfillcolor{currentfill}%
\pgfsetlinewidth{1.003750pt}%
\definecolor{currentstroke}{rgb}{0.277941,0.056324,0.381191}%
\pgfsetstrokecolor{currentstroke}%
\pgfsetdash{}{0pt}%
\pgfpathmoveto{\pgfqpoint{1.833377in}{1.517434in}}%
\pgfpathlineto{\pgfqpoint{1.823641in}{1.538062in}}%
\pgfpathlineto{\pgfqpoint{1.816702in}{1.558690in}}%
\pgfpathlineto{\pgfqpoint{1.811994in}{1.579318in}}%
\pgfpathlineto{\pgfqpoint{1.809094in}{1.599946in}}%
\pgfpathlineto{\pgfqpoint{1.807680in}{1.620573in}}%
\pgfpathlineto{\pgfqpoint{1.807502in}{1.641201in}}%
\pgfpathlineto{\pgfqpoint{1.808364in}{1.661829in}}%
\pgfpathlineto{\pgfqpoint{1.810107in}{1.682457in}}%
\pgfpathlineto{\pgfqpoint{1.812606in}{1.703085in}}%
\pgfpathlineto{\pgfqpoint{1.815757in}{1.723713in}}%
\pgfpathlineto{\pgfqpoint{1.819473in}{1.744340in}}%
\pgfpathlineto{\pgfqpoint{1.823683in}{1.764968in}}%
\pgfpathlineto{\pgfqpoint{1.828329in}{1.785596in}}%
\pgfpathlineto{\pgfqpoint{1.833359in}{1.806224in}}%
\pgfpathlineto{\pgfqpoint{1.837144in}{1.820594in}}%
\pgfpathlineto{\pgfqpoint{1.838870in}{1.826852in}}%
\pgfpathlineto{\pgfqpoint{1.844997in}{1.847480in}}%
\pgfpathlineto{\pgfqpoint{1.851349in}{1.868107in}}%
\pgfpathlineto{\pgfqpoint{1.857900in}{1.888735in}}%
\pgfpathlineto{\pgfqpoint{1.864627in}{1.909363in}}%
\pgfpathlineto{\pgfqpoint{1.871511in}{1.929991in}}%
\pgfpathlineto{\pgfqpoint{1.878189in}{1.949562in}}%
\pgfpathlineto{\pgfqpoint{1.878604in}{1.950619in}}%
\pgfpathlineto{\pgfqpoint{1.887071in}{1.971247in}}%
\pgfpathlineto{\pgfqpoint{1.895568in}{1.991874in}}%
\pgfpathlineto{\pgfqpoint{1.904092in}{2.012502in}}%
\pgfpathlineto{\pgfqpoint{1.912641in}{2.033130in}}%
\pgfpathlineto{\pgfqpoint{1.919235in}{2.048871in}}%
\pgfpathlineto{\pgfqpoint{1.921761in}{2.053758in}}%
\pgfpathlineto{\pgfqpoint{1.932648in}{2.074386in}}%
\pgfpathlineto{\pgfqpoint{1.943423in}{2.095013in}}%
\pgfpathlineto{\pgfqpoint{1.954097in}{2.115641in}}%
\pgfpathlineto{\pgfqpoint{1.960280in}{2.127509in}}%
\pgfpathlineto{\pgfqpoint{1.966236in}{2.136269in}}%
\pgfpathlineto{\pgfqpoint{1.980322in}{2.156897in}}%
\pgfpathlineto{\pgfqpoint{1.994150in}{2.177525in}}%
\pgfpathlineto{\pgfqpoint{2.001325in}{2.188213in}}%
\pgfpathlineto{\pgfqpoint{2.010526in}{2.198153in}}%
\pgfpathlineto{\pgfqpoint{2.029567in}{2.218780in}}%
\pgfpathlineto{\pgfqpoint{2.042371in}{2.232798in}}%
\pgfpathlineto{\pgfqpoint{2.051494in}{2.239408in}}%
\pgfpathlineto{\pgfqpoint{2.080014in}{2.260036in}}%
\pgfpathlineto{\pgfqpoint{2.083416in}{2.262472in}}%
\pgfpathlineto{\pgfqpoint{2.124461in}{2.277177in}}%
\pgfpathlineto{\pgfqpoint{2.165507in}{2.275716in}}%
\pgfpathlineto{\pgfqpoint{2.197966in}{2.260036in}}%
\pgfpathlineto{\pgfqpoint{2.206552in}{2.255731in}}%
\pgfpathlineto{\pgfqpoint{2.222906in}{2.239408in}}%
\pgfpathlineto{\pgfqpoint{2.243116in}{2.218780in}}%
\pgfpathlineto{\pgfqpoint{2.247597in}{2.214036in}}%
\pgfpathlineto{\pgfqpoint{2.257657in}{2.198153in}}%
\pgfpathlineto{\pgfqpoint{2.270307in}{2.177525in}}%
\pgfpathlineto{\pgfqpoint{2.282579in}{2.156897in}}%
\pgfpathlineto{\pgfqpoint{2.288643in}{2.146257in}}%
\pgfpathlineto{\pgfqpoint{2.293113in}{2.136269in}}%
\pgfpathlineto{\pgfqpoint{2.301876in}{2.115641in}}%
\pgfpathlineto{\pgfqpoint{2.310229in}{2.095013in}}%
\pgfpathlineto{\pgfqpoint{2.318150in}{2.074386in}}%
\pgfpathlineto{\pgfqpoint{2.325614in}{2.053758in}}%
\pgfpathlineto{\pgfqpoint{2.329688in}{2.041637in}}%
\pgfpathlineto{\pgfqpoint{2.332236in}{2.033130in}}%
\pgfpathlineto{\pgfqpoint{2.337875in}{2.012502in}}%
\pgfpathlineto{\pgfqpoint{2.343006in}{1.991874in}}%
\pgfpathlineto{\pgfqpoint{2.347606in}{1.971247in}}%
\pgfpathlineto{\pgfqpoint{2.351649in}{1.950619in}}%
\pgfpathlineto{\pgfqpoint{2.355109in}{1.929991in}}%
\pgfpathlineto{\pgfqpoint{2.357959in}{1.909363in}}%
\pgfpathlineto{\pgfqpoint{2.360168in}{1.888735in}}%
\pgfpathlineto{\pgfqpoint{2.361704in}{1.868107in}}%
\pgfpathlineto{\pgfqpoint{2.362534in}{1.847480in}}%
\pgfpathlineto{\pgfqpoint{2.362620in}{1.826852in}}%
\pgfpathlineto{\pgfqpoint{2.361925in}{1.806224in}}%
\pgfpathlineto{\pgfqpoint{2.360405in}{1.785596in}}%
\pgfpathlineto{\pgfqpoint{2.358017in}{1.764968in}}%
\pgfpathlineto{\pgfqpoint{2.354713in}{1.744340in}}%
\pgfpathlineto{\pgfqpoint{2.350440in}{1.723713in}}%
\pgfpathlineto{\pgfqpoint{2.345143in}{1.703085in}}%
\pgfpathlineto{\pgfqpoint{2.338763in}{1.682457in}}%
\pgfpathlineto{\pgfqpoint{2.331235in}{1.661829in}}%
\pgfpathlineto{\pgfqpoint{2.329688in}{1.658121in}}%
\pgfpathlineto{\pgfqpoint{2.320610in}{1.641201in}}%
\pgfpathlineto{\pgfqpoint{2.307783in}{1.620573in}}%
\pgfpathlineto{\pgfqpoint{2.292980in}{1.599946in}}%
\pgfpathlineto{\pgfqpoint{2.288643in}{1.594552in}}%
\pgfpathlineto{\pgfqpoint{2.268911in}{1.579318in}}%
\pgfpathlineto{\pgfqpoint{2.247597in}{1.564899in}}%
\pgfpathlineto{\pgfqpoint{2.228328in}{1.558690in}}%
\pgfpathlineto{\pgfqpoint{2.206552in}{1.552368in}}%
\pgfpathlineto{\pgfqpoint{2.165507in}{1.547483in}}%
\pgfpathlineto{\pgfqpoint{2.124461in}{1.543152in}}%
\pgfpathlineto{\pgfqpoint{2.099705in}{1.538062in}}%
\pgfpathlineto{\pgfqpoint{2.083416in}{1.534795in}}%
\pgfpathlineto{\pgfqpoint{2.042371in}{1.520751in}}%
\pgfpathlineto{\pgfqpoint{2.034688in}{1.517434in}}%
\pgfpathlineto{\pgfqpoint{2.001325in}{1.502574in}}%
\pgfpathlineto{\pgfqpoint{1.987631in}{1.496807in}}%
\pgfpathlineto{\pgfqpoint{1.960280in}{1.484612in}}%
\pgfpathlineto{\pgfqpoint{1.929168in}{1.476179in}}%
\pgfpathlineto{\pgfqpoint{1.919235in}{1.473264in}}%
\pgfpathlineto{\pgfqpoint{1.886280in}{1.476179in}}%
\pgfpathlineto{\pgfqpoint{1.878189in}{1.477110in}}%
\pgfpathlineto{\pgfqpoint{1.851045in}{1.496807in}}%
\pgfpathlineto{\pgfqpoint{1.837144in}{1.511217in}}%
\pgfpathclose%
\pgfpathmoveto{\pgfqpoint{2.001325in}{1.760437in}}%
\pgfpathlineto{\pgfqpoint{2.042371in}{1.748620in}}%
\pgfpathlineto{\pgfqpoint{2.083416in}{1.750254in}}%
\pgfpathlineto{\pgfqpoint{2.124461in}{1.754160in}}%
\pgfpathlineto{\pgfqpoint{2.165507in}{1.762261in}}%
\pgfpathlineto{\pgfqpoint{2.170667in}{1.764968in}}%
\pgfpathlineto{\pgfqpoint{2.203094in}{1.785596in}}%
\pgfpathlineto{\pgfqpoint{2.206552in}{1.788350in}}%
\pgfpathlineto{\pgfqpoint{2.216661in}{1.806224in}}%
\pgfpathlineto{\pgfqpoint{2.225381in}{1.826852in}}%
\pgfpathlineto{\pgfqpoint{2.231412in}{1.847480in}}%
\pgfpathlineto{\pgfqpoint{2.235033in}{1.868107in}}%
\pgfpathlineto{\pgfqpoint{2.236483in}{1.888735in}}%
\pgfpathlineto{\pgfqpoint{2.235972in}{1.909363in}}%
\pgfpathlineto{\pgfqpoint{2.233683in}{1.929991in}}%
\pgfpathlineto{\pgfqpoint{2.229777in}{1.950619in}}%
\pgfpathlineto{\pgfqpoint{2.224397in}{1.971247in}}%
\pgfpathlineto{\pgfqpoint{2.217669in}{1.991874in}}%
\pgfpathlineto{\pgfqpoint{2.209704in}{2.012502in}}%
\pgfpathlineto{\pgfqpoint{2.206552in}{2.019521in}}%
\pgfpathlineto{\pgfqpoint{2.194269in}{2.033130in}}%
\pgfpathlineto{\pgfqpoint{2.173373in}{2.053758in}}%
\pgfpathlineto{\pgfqpoint{2.165507in}{2.060743in}}%
\pgfpathlineto{\pgfqpoint{2.124461in}{2.066156in}}%
\pgfpathlineto{\pgfqpoint{2.097367in}{2.053758in}}%
\pgfpathlineto{\pgfqpoint{2.083416in}{2.046973in}}%
\pgfpathlineto{\pgfqpoint{2.069643in}{2.033130in}}%
\pgfpathlineto{\pgfqpoint{2.049487in}{2.012502in}}%
\pgfpathlineto{\pgfqpoint{2.042371in}{2.004864in}}%
\pgfpathlineto{\pgfqpoint{2.034694in}{1.991874in}}%
\pgfpathlineto{\pgfqpoint{2.023073in}{1.971247in}}%
\pgfpathlineto{\pgfqpoint{2.011779in}{1.950619in}}%
\pgfpathlineto{\pgfqpoint{2.001325in}{1.930723in}}%
\pgfpathlineto{\pgfqpoint{2.001071in}{1.929991in}}%
\pgfpathlineto{\pgfqpoint{1.994996in}{1.909363in}}%
\pgfpathlineto{\pgfqpoint{1.989715in}{1.888735in}}%
\pgfpathlineto{\pgfqpoint{1.985449in}{1.868107in}}%
\pgfpathlineto{\pgfqpoint{1.982509in}{1.847480in}}%
\pgfpathlineto{\pgfqpoint{1.981347in}{1.826852in}}%
\pgfpathlineto{\pgfqpoint{1.982648in}{1.806224in}}%
\pgfpathlineto{\pgfqpoint{1.987498in}{1.785596in}}%
\pgfpathlineto{\pgfqpoint{1.997736in}{1.764968in}}%
\pgfpathclose%
\pgfusepath{stroke,fill}%
\end{pgfscope}%
\begin{pgfscope}%
\pgfpathrectangle{\pgfqpoint{0.605784in}{0.382904in}}{\pgfqpoint{4.063488in}{2.042155in}}%
\pgfusepath{clip}%
\pgfsetbuttcap%
\pgfsetroundjoin%
\definecolor{currentfill}{rgb}{0.277941,0.056324,0.381191}%
\pgfsetfillcolor{currentfill}%
\pgfsetlinewidth{1.003750pt}%
\definecolor{currentstroke}{rgb}{0.277941,0.056324,0.381191}%
\pgfsetstrokecolor{currentstroke}%
\pgfsetdash{}{0pt}%
\pgfpathmoveto{\pgfqpoint{2.534837in}{1.249273in}}%
\pgfpathlineto{\pgfqpoint{2.534270in}{1.269900in}}%
\pgfpathlineto{\pgfqpoint{2.533742in}{1.290528in}}%
\pgfpathlineto{\pgfqpoint{2.533257in}{1.311156in}}%
\pgfpathlineto{\pgfqpoint{2.532822in}{1.331784in}}%
\pgfpathlineto{\pgfqpoint{2.532445in}{1.352412in}}%
\pgfpathlineto{\pgfqpoint{2.532134in}{1.373040in}}%
\pgfpathlineto{\pgfqpoint{2.531899in}{1.393667in}}%
\pgfpathlineto{\pgfqpoint{2.531755in}{1.414295in}}%
\pgfpathlineto{\pgfqpoint{2.531716in}{1.434923in}}%
\pgfpathlineto{\pgfqpoint{2.531804in}{1.455551in}}%
\pgfpathlineto{\pgfqpoint{2.532044in}{1.476179in}}%
\pgfpathlineto{\pgfqpoint{2.532469in}{1.496807in}}%
\pgfpathlineto{\pgfqpoint{2.533124in}{1.517434in}}%
\pgfpathlineto{\pgfqpoint{2.534068in}{1.538062in}}%
\pgfpathlineto{\pgfqpoint{2.534915in}{1.551765in}}%
\pgfpathlineto{\pgfqpoint{2.575960in}{1.551765in}}%
\pgfpathlineto{\pgfqpoint{2.617005in}{1.551765in}}%
\pgfpathlineto{\pgfqpoint{2.658051in}{1.551765in}}%
\pgfpathlineto{\pgfqpoint{2.699096in}{1.551765in}}%
\pgfpathlineto{\pgfqpoint{2.740141in}{1.551765in}}%
\pgfpathlineto{\pgfqpoint{2.781187in}{1.551765in}}%
\pgfpathlineto{\pgfqpoint{2.822232in}{1.551765in}}%
\pgfpathlineto{\pgfqpoint{2.863277in}{1.551765in}}%
\pgfpathlineto{\pgfqpoint{2.904323in}{1.551765in}}%
\pgfpathlineto{\pgfqpoint{2.945368in}{1.551765in}}%
\pgfpathlineto{\pgfqpoint{2.986413in}{1.551765in}}%
\pgfpathlineto{\pgfqpoint{3.027459in}{1.551765in}}%
\pgfpathlineto{\pgfqpoint{3.068504in}{1.551765in}}%
\pgfpathlineto{\pgfqpoint{3.109549in}{1.551765in}}%
\pgfpathlineto{\pgfqpoint{3.150595in}{1.551765in}}%
\pgfpathlineto{\pgfqpoint{3.191640in}{1.551765in}}%
\pgfpathlineto{\pgfqpoint{3.232685in}{1.551765in}}%
\pgfpathlineto{\pgfqpoint{3.273731in}{1.551765in}}%
\pgfpathlineto{\pgfqpoint{3.314776in}{1.551765in}}%
\pgfpathlineto{\pgfqpoint{3.355821in}{1.551765in}}%
\pgfpathlineto{\pgfqpoint{3.396867in}{1.551765in}}%
\pgfpathlineto{\pgfqpoint{3.437912in}{1.551765in}}%
\pgfpathlineto{\pgfqpoint{3.478957in}{1.551765in}}%
\pgfpathlineto{\pgfqpoint{3.520003in}{1.551765in}}%
\pgfpathlineto{\pgfqpoint{3.561048in}{1.551765in}}%
\pgfpathlineto{\pgfqpoint{3.602093in}{1.551765in}}%
\pgfpathlineto{\pgfqpoint{3.643139in}{1.551765in}}%
\pgfpathlineto{\pgfqpoint{3.644695in}{1.538062in}}%
\pgfpathlineto{\pgfqpoint{3.647039in}{1.517434in}}%
\pgfpathlineto{\pgfqpoint{3.649462in}{1.496807in}}%
\pgfpathlineto{\pgfqpoint{3.651992in}{1.476179in}}%
\pgfpathlineto{\pgfqpoint{3.654668in}{1.455551in}}%
\pgfpathlineto{\pgfqpoint{3.657553in}{1.434923in}}%
\pgfpathlineto{\pgfqpoint{3.660750in}{1.414295in}}%
\pgfpathlineto{\pgfqpoint{3.664448in}{1.393667in}}%
\pgfpathlineto{\pgfqpoint{3.669022in}{1.373040in}}%
\pgfpathlineto{\pgfqpoint{3.675349in}{1.352412in}}%
\pgfpathlineto{\pgfqpoint{3.684184in}{1.334339in}}%
\pgfpathlineto{\pgfqpoint{3.725229in}{1.334339in}}%
\pgfpathlineto{\pgfqpoint{3.766275in}{1.334339in}}%
\pgfpathlineto{\pgfqpoint{3.807320in}{1.334339in}}%
\pgfpathlineto{\pgfqpoint{3.848365in}{1.334339in}}%
\pgfpathlineto{\pgfqpoint{3.889411in}{1.334339in}}%
\pgfpathlineto{\pgfqpoint{3.930456in}{1.334339in}}%
\pgfpathlineto{\pgfqpoint{3.971501in}{1.334339in}}%
\pgfpathlineto{\pgfqpoint{4.012547in}{1.334339in}}%
\pgfpathlineto{\pgfqpoint{4.053592in}{1.334339in}}%
\pgfpathlineto{\pgfqpoint{4.094637in}{1.334339in}}%
\pgfpathlineto{\pgfqpoint{4.135683in}{1.334339in}}%
\pgfpathlineto{\pgfqpoint{4.176728in}{1.334339in}}%
\pgfpathlineto{\pgfqpoint{4.217773in}{1.334339in}}%
\pgfpathlineto{\pgfqpoint{4.258819in}{1.334339in}}%
\pgfpathlineto{\pgfqpoint{4.299864in}{1.334339in}}%
\pgfpathlineto{\pgfqpoint{4.340909in}{1.334339in}}%
\pgfpathlineto{\pgfqpoint{4.381955in}{1.334339in}}%
\pgfpathlineto{\pgfqpoint{4.423000in}{1.334339in}}%
\pgfpathlineto{\pgfqpoint{4.464045in}{1.334339in}}%
\pgfpathlineto{\pgfqpoint{4.505091in}{1.334339in}}%
\pgfpathlineto{\pgfqpoint{4.546136in}{1.334339in}}%
\pgfpathlineto{\pgfqpoint{4.587181in}{1.334339in}}%
\pgfpathlineto{\pgfqpoint{4.628227in}{1.334339in}}%
\pgfpathlineto{\pgfqpoint{4.669272in}{1.334339in}}%
\pgfpathlineto{\pgfqpoint{4.669272in}{1.331784in}}%
\pgfpathlineto{\pgfqpoint{4.669272in}{1.311156in}}%
\pgfpathlineto{\pgfqpoint{4.669272in}{1.290528in}}%
\pgfpathlineto{\pgfqpoint{4.669272in}{1.269900in}}%
\pgfpathlineto{\pgfqpoint{4.669272in}{1.249273in}}%
\pgfpathlineto{\pgfqpoint{4.669272in}{1.228645in}}%
\pgfpathlineto{\pgfqpoint{4.669272in}{1.208017in}}%
\pgfpathlineto{\pgfqpoint{4.669272in}{1.187389in}}%
\pgfpathlineto{\pgfqpoint{4.669272in}{1.166761in}}%
\pgfpathlineto{\pgfqpoint{4.669272in}{1.146134in}}%
\pgfpathlineto{\pgfqpoint{4.669272in}{1.125506in}}%
\pgfpathlineto{\pgfqpoint{4.669272in}{1.104878in}}%
\pgfpathlineto{\pgfqpoint{4.669272in}{1.084250in}}%
\pgfpathlineto{\pgfqpoint{4.669272in}{1.063622in}}%
\pgfpathlineto{\pgfqpoint{4.669272in}{1.042994in}}%
\pgfpathlineto{\pgfqpoint{4.669272in}{1.029292in}}%
\pgfpathlineto{\pgfqpoint{4.628227in}{1.029292in}}%
\pgfpathlineto{\pgfqpoint{4.587181in}{1.029292in}}%
\pgfpathlineto{\pgfqpoint{4.546136in}{1.029292in}}%
\pgfpathlineto{\pgfqpoint{4.505091in}{1.029292in}}%
\pgfpathlineto{\pgfqpoint{4.464045in}{1.029292in}}%
\pgfpathlineto{\pgfqpoint{4.423000in}{1.029292in}}%
\pgfpathlineto{\pgfqpoint{4.381955in}{1.029292in}}%
\pgfpathlineto{\pgfqpoint{4.340909in}{1.029292in}}%
\pgfpathlineto{\pgfqpoint{4.299864in}{1.029292in}}%
\pgfpathlineto{\pgfqpoint{4.258819in}{1.029292in}}%
\pgfpathlineto{\pgfqpoint{4.217773in}{1.029292in}}%
\pgfpathlineto{\pgfqpoint{4.176728in}{1.029292in}}%
\pgfpathlineto{\pgfqpoint{4.135683in}{1.029292in}}%
\pgfpathlineto{\pgfqpoint{4.094637in}{1.029292in}}%
\pgfpathlineto{\pgfqpoint{4.053592in}{1.029292in}}%
\pgfpathlineto{\pgfqpoint{4.012547in}{1.029292in}}%
\pgfpathlineto{\pgfqpoint{3.971501in}{1.029292in}}%
\pgfpathlineto{\pgfqpoint{3.930456in}{1.029292in}}%
\pgfpathlineto{\pgfqpoint{3.889411in}{1.029292in}}%
\pgfpathlineto{\pgfqpoint{3.848365in}{1.029292in}}%
\pgfpathlineto{\pgfqpoint{3.807320in}{1.029292in}}%
\pgfpathlineto{\pgfqpoint{3.766275in}{1.029292in}}%
\pgfpathlineto{\pgfqpoint{3.725229in}{1.029292in}}%
\pgfpathlineto{\pgfqpoint{3.684184in}{1.029292in}}%
\pgfpathlineto{\pgfqpoint{3.682627in}{1.042994in}}%
\pgfpathlineto{\pgfqpoint{3.680284in}{1.063622in}}%
\pgfpathlineto{\pgfqpoint{3.677860in}{1.084250in}}%
\pgfpathlineto{\pgfqpoint{3.675330in}{1.104878in}}%
\pgfpathlineto{\pgfqpoint{3.672654in}{1.125506in}}%
\pgfpathlineto{\pgfqpoint{3.669770in}{1.146134in}}%
\pgfpathlineto{\pgfqpoint{3.666573in}{1.166761in}}%
\pgfpathlineto{\pgfqpoint{3.662874in}{1.187389in}}%
\pgfpathlineto{\pgfqpoint{3.658300in}{1.208017in}}%
\pgfpathlineto{\pgfqpoint{3.651973in}{1.228645in}}%
\pgfpathlineto{\pgfqpoint{3.643139in}{1.246718in}}%
\pgfpathlineto{\pgfqpoint{3.602093in}{1.246718in}}%
\pgfpathlineto{\pgfqpoint{3.561048in}{1.246718in}}%
\pgfpathlineto{\pgfqpoint{3.520003in}{1.246718in}}%
\pgfpathlineto{\pgfqpoint{3.478957in}{1.246718in}}%
\pgfpathlineto{\pgfqpoint{3.437912in}{1.246718in}}%
\pgfpathlineto{\pgfqpoint{3.396867in}{1.246718in}}%
\pgfpathlineto{\pgfqpoint{3.355821in}{1.246718in}}%
\pgfpathlineto{\pgfqpoint{3.314776in}{1.246718in}}%
\pgfpathlineto{\pgfqpoint{3.273731in}{1.246718in}}%
\pgfpathlineto{\pgfqpoint{3.232685in}{1.246718in}}%
\pgfpathlineto{\pgfqpoint{3.191640in}{1.246718in}}%
\pgfpathlineto{\pgfqpoint{3.150595in}{1.246718in}}%
\pgfpathlineto{\pgfqpoint{3.109549in}{1.246718in}}%
\pgfpathlineto{\pgfqpoint{3.068504in}{1.246718in}}%
\pgfpathlineto{\pgfqpoint{3.027459in}{1.246718in}}%
\pgfpathlineto{\pgfqpoint{2.986413in}{1.246718in}}%
\pgfpathlineto{\pgfqpoint{2.945368in}{1.246718in}}%
\pgfpathlineto{\pgfqpoint{2.904323in}{1.246718in}}%
\pgfpathlineto{\pgfqpoint{2.863277in}{1.246718in}}%
\pgfpathlineto{\pgfqpoint{2.822232in}{1.246718in}}%
\pgfpathlineto{\pgfqpoint{2.781187in}{1.246718in}}%
\pgfpathlineto{\pgfqpoint{2.740141in}{1.246718in}}%
\pgfpathlineto{\pgfqpoint{2.699096in}{1.246718in}}%
\pgfpathlineto{\pgfqpoint{2.658051in}{1.246718in}}%
\pgfpathlineto{\pgfqpoint{2.617005in}{1.246718in}}%
\pgfpathlineto{\pgfqpoint{2.575960in}{1.246718in}}%
\pgfpathlineto{\pgfqpoint{2.534915in}{1.246718in}}%
\pgfpathclose%
\pgfusepath{stroke,fill}%
\end{pgfscope}%
\begin{pgfscope}%
\pgfpathrectangle{\pgfqpoint{0.605784in}{0.382904in}}{\pgfqpoint{4.063488in}{2.042155in}}%
\pgfusepath{clip}%
\pgfsetbuttcap%
\pgfsetroundjoin%
\definecolor{currentfill}{rgb}{0.281887,0.150881,0.465405}%
\pgfsetfillcolor{currentfill}%
\pgfsetlinewidth{1.003750pt}%
\definecolor{currentstroke}{rgb}{0.281887,0.150881,0.465405}%
\pgfsetstrokecolor{currentstroke}%
\pgfsetdash{}{0pt}%
\pgfpathmoveto{\pgfqpoint{0.609374in}{0.877972in}}%
\pgfpathlineto{\pgfqpoint{0.605784in}{0.879256in}}%
\pgfpathlineto{\pgfqpoint{0.605784in}{0.898600in}}%
\pgfpathlineto{\pgfqpoint{0.605784in}{0.919227in}}%
\pgfpathlineto{\pgfqpoint{0.605784in}{0.939855in}}%
\pgfpathlineto{\pgfqpoint{0.605784in}{0.960483in}}%
\pgfpathlineto{\pgfqpoint{0.605784in}{0.981111in}}%
\pgfpathlineto{\pgfqpoint{0.605784in}{1.001739in}}%
\pgfpathlineto{\pgfqpoint{0.605784in}{1.022367in}}%
\pgfpathlineto{\pgfqpoint{0.605784in}{1.042994in}}%
\pgfpathlineto{\pgfqpoint{0.605784in}{1.048119in}}%
\pgfpathlineto{\pgfqpoint{0.625055in}{1.042994in}}%
\pgfpathlineto{\pgfqpoint{0.646829in}{1.037216in}}%
\pgfpathlineto{\pgfqpoint{0.673279in}{1.042994in}}%
\pgfpathlineto{\pgfqpoint{0.687875in}{1.046783in}}%
\pgfpathlineto{\pgfqpoint{0.712494in}{1.063622in}}%
\pgfpathlineto{\pgfqpoint{0.728920in}{1.076977in}}%
\pgfpathlineto{\pgfqpoint{0.736539in}{1.084250in}}%
\pgfpathlineto{\pgfqpoint{0.755254in}{1.104878in}}%
\pgfpathlineto{\pgfqpoint{0.769965in}{1.124103in}}%
\pgfpathlineto{\pgfqpoint{0.771380in}{1.125506in}}%
\pgfpathlineto{\pgfqpoint{0.789256in}{1.146134in}}%
\pgfpathlineto{\pgfqpoint{0.804511in}{1.166761in}}%
\pgfpathlineto{\pgfqpoint{0.811011in}{1.177728in}}%
\pgfpathlineto{\pgfqpoint{0.831523in}{1.187389in}}%
\pgfpathlineto{\pgfqpoint{0.852056in}{1.197592in}}%
\pgfpathlineto{\pgfqpoint{0.870551in}{1.187389in}}%
\pgfpathlineto{\pgfqpoint{0.893101in}{1.180197in}}%
\pgfpathlineto{\pgfqpoint{0.929382in}{1.166761in}}%
\pgfpathlineto{\pgfqpoint{0.934147in}{1.165585in}}%
\pgfpathlineto{\pgfqpoint{0.975192in}{1.165512in}}%
\pgfpathlineto{\pgfqpoint{0.978103in}{1.166761in}}%
\pgfpathlineto{\pgfqpoint{1.016237in}{1.181328in}}%
\pgfpathlineto{\pgfqpoint{1.024090in}{1.187389in}}%
\pgfpathlineto{\pgfqpoint{1.053372in}{1.208017in}}%
\pgfpathlineto{\pgfqpoint{1.057283in}{1.210567in}}%
\pgfpathlineto{\pgfqpoint{1.075669in}{1.228645in}}%
\pgfpathlineto{\pgfqpoint{1.098328in}{1.248833in}}%
\pgfpathlineto{\pgfqpoint{1.098754in}{1.249273in}}%
\pgfpathlineto{\pgfqpoint{1.118747in}{1.269900in}}%
\pgfpathlineto{\pgfqpoint{1.139373in}{1.289604in}}%
\pgfpathlineto{\pgfqpoint{1.140417in}{1.290528in}}%
\pgfpathlineto{\pgfqpoint{1.163647in}{1.311156in}}%
\pgfpathlineto{\pgfqpoint{1.180419in}{1.325357in}}%
\pgfpathlineto{\pgfqpoint{1.191419in}{1.331784in}}%
\pgfpathlineto{\pgfqpoint{1.221464in}{1.349086in}}%
\pgfpathlineto{\pgfqpoint{1.237395in}{1.352412in}}%
\pgfpathlineto{\pgfqpoint{1.262509in}{1.357783in}}%
\pgfpathlineto{\pgfqpoint{1.303555in}{1.352749in}}%
\pgfpathlineto{\pgfqpoint{1.304609in}{1.352412in}}%
\pgfpathlineto{\pgfqpoint{1.344600in}{1.340326in}}%
\pgfpathlineto{\pgfqpoint{1.372849in}{1.331784in}}%
\pgfpathlineto{\pgfqpoint{1.385645in}{1.328018in}}%
\pgfpathlineto{\pgfqpoint{1.426691in}{1.324826in}}%
\pgfpathlineto{\pgfqpoint{1.447860in}{1.331784in}}%
\pgfpathlineto{\pgfqpoint{1.467736in}{1.339492in}}%
\pgfpathlineto{\pgfqpoint{1.481616in}{1.352412in}}%
\pgfpathlineto{\pgfqpoint{1.500265in}{1.373040in}}%
\pgfpathlineto{\pgfqpoint{1.508781in}{1.384351in}}%
\pgfpathlineto{\pgfqpoint{1.513852in}{1.393667in}}%
\pgfpathlineto{\pgfqpoint{1.523321in}{1.414295in}}%
\pgfpathlineto{\pgfqpoint{1.531010in}{1.434923in}}%
\pgfpathlineto{\pgfqpoint{1.537047in}{1.455551in}}%
\pgfpathlineto{\pgfqpoint{1.541547in}{1.476179in}}%
\pgfpathlineto{\pgfqpoint{1.544616in}{1.496807in}}%
\pgfpathlineto{\pgfqpoint{1.546348in}{1.517434in}}%
\pgfpathlineto{\pgfqpoint{1.546831in}{1.538062in}}%
\pgfpathlineto{\pgfqpoint{1.546145in}{1.558690in}}%
\pgfpathlineto{\pgfqpoint{1.544363in}{1.579318in}}%
\pgfpathlineto{\pgfqpoint{1.541553in}{1.599946in}}%
\pgfpathlineto{\pgfqpoint{1.537775in}{1.620573in}}%
\pgfpathlineto{\pgfqpoint{1.533088in}{1.641201in}}%
\pgfpathlineto{\pgfqpoint{1.527543in}{1.661829in}}%
\pgfpathlineto{\pgfqpoint{1.521190in}{1.682457in}}%
\pgfpathlineto{\pgfqpoint{1.514073in}{1.703085in}}%
\pgfpathlineto{\pgfqpoint{1.508781in}{1.716925in}}%
\pgfpathlineto{\pgfqpoint{1.505944in}{1.723713in}}%
\pgfpathlineto{\pgfqpoint{1.496539in}{1.744340in}}%
\pgfpathlineto{\pgfqpoint{1.486546in}{1.764968in}}%
\pgfpathlineto{\pgfqpoint{1.476007in}{1.785596in}}%
\pgfpathlineto{\pgfqpoint{1.467736in}{1.800949in}}%
\pgfpathlineto{\pgfqpoint{1.464066in}{1.806224in}}%
\pgfpathlineto{\pgfqpoint{1.448993in}{1.826852in}}%
\pgfpathlineto{\pgfqpoint{1.433583in}{1.847480in}}%
\pgfpathlineto{\pgfqpoint{1.426691in}{1.856377in}}%
\pgfpathlineto{\pgfqpoint{1.412287in}{1.868107in}}%
\pgfpathlineto{\pgfqpoint{1.386679in}{1.888735in}}%
\pgfpathlineto{\pgfqpoint{1.385645in}{1.889543in}}%
\pgfpathlineto{\pgfqpoint{1.344600in}{1.903054in}}%
\pgfpathlineto{\pgfqpoint{1.303555in}{1.901076in}}%
\pgfpathlineto{\pgfqpoint{1.263886in}{1.888735in}}%
\pgfpathlineto{\pgfqpoint{1.262509in}{1.888279in}}%
\pgfpathlineto{\pgfqpoint{1.221464in}{1.868379in}}%
\pgfpathlineto{\pgfqpoint{1.221023in}{1.868107in}}%
\pgfpathlineto{\pgfqpoint{1.187617in}{1.847480in}}%
\pgfpathlineto{\pgfqpoint{1.180419in}{1.843097in}}%
\pgfpathlineto{\pgfqpoint{1.159959in}{1.826852in}}%
\pgfpathlineto{\pgfqpoint{1.139373in}{1.811309in}}%
\pgfpathlineto{\pgfqpoint{1.134520in}{1.806224in}}%
\pgfpathlineto{\pgfqpoint{1.114518in}{1.785596in}}%
\pgfpathlineto{\pgfqpoint{1.098328in}{1.769879in}}%
\pgfpathlineto{\pgfqpoint{1.094702in}{1.764968in}}%
\pgfpathlineto{\pgfqpoint{1.079443in}{1.744340in}}%
\pgfpathlineto{\pgfqpoint{1.063291in}{1.723713in}}%
\pgfpathlineto{\pgfqpoint{1.057283in}{1.716168in}}%
\pgfpathlineto{\pgfqpoint{1.049473in}{1.703085in}}%
\pgfpathlineto{\pgfqpoint{1.037008in}{1.682457in}}%
\pgfpathlineto{\pgfqpoint{1.024069in}{1.661829in}}%
\pgfpathlineto{\pgfqpoint{1.016237in}{1.649535in}}%
\pgfpathlineto{\pgfqpoint{1.011867in}{1.641201in}}%
\pgfpathlineto{\pgfqpoint{1.001210in}{1.620573in}}%
\pgfpathlineto{\pgfqpoint{0.990364in}{1.599946in}}%
\pgfpathlineto{\pgfqpoint{0.979294in}{1.579318in}}%
\pgfpathlineto{\pgfqpoint{0.975192in}{1.571547in}}%
\pgfpathlineto{\pgfqpoint{0.968892in}{1.558690in}}%
\pgfpathlineto{\pgfqpoint{0.958976in}{1.538062in}}%
\pgfpathlineto{\pgfqpoint{0.949060in}{1.517434in}}%
\pgfpathlineto{\pgfqpoint{0.939145in}{1.496807in}}%
\pgfpathlineto{\pgfqpoint{0.934147in}{1.486076in}}%
\pgfpathlineto{\pgfqpoint{0.929058in}{1.476179in}}%
\pgfpathlineto{\pgfqpoint{0.918937in}{1.455551in}}%
\pgfpathlineto{\pgfqpoint{0.909035in}{1.434923in}}%
\pgfpathlineto{\pgfqpoint{0.899399in}{1.414295in}}%
\pgfpathlineto{\pgfqpoint{0.893101in}{1.400019in}}%
\pgfpathlineto{\pgfqpoint{0.888896in}{1.393667in}}%
\pgfpathlineto{\pgfqpoint{0.876391in}{1.373040in}}%
\pgfpathlineto{\pgfqpoint{0.864476in}{1.352412in}}%
\pgfpathlineto{\pgfqpoint{0.853335in}{1.331784in}}%
\pgfpathlineto{\pgfqpoint{0.852056in}{1.328910in}}%
\pgfpathlineto{\pgfqpoint{0.811011in}{1.314631in}}%
\pgfpathlineto{\pgfqpoint{0.796522in}{1.331784in}}%
\pgfpathlineto{\pgfqpoint{0.772898in}{1.352412in}}%
\pgfpathlineto{\pgfqpoint{0.769965in}{1.354454in}}%
\pgfpathlineto{\pgfqpoint{0.757335in}{1.373040in}}%
\pgfpathlineto{\pgfqpoint{0.741211in}{1.393667in}}%
\pgfpathlineto{\pgfqpoint{0.728920in}{1.407586in}}%
\pgfpathlineto{\pgfqpoint{0.724243in}{1.414295in}}%
\pgfpathlineto{\pgfqpoint{0.708519in}{1.434923in}}%
\pgfpathlineto{\pgfqpoint{0.691769in}{1.455551in}}%
\pgfpathlineto{\pgfqpoint{0.687875in}{1.460008in}}%
\pgfpathlineto{\pgfqpoint{0.673037in}{1.476179in}}%
\pgfpathlineto{\pgfqpoint{0.653540in}{1.496807in}}%
\pgfpathlineto{\pgfqpoint{0.646829in}{1.503614in}}%
\pgfpathlineto{\pgfqpoint{0.627507in}{1.517434in}}%
\pgfpathlineto{\pgfqpoint{0.605784in}{1.532937in}}%
\pgfpathlineto{\pgfqpoint{0.605784in}{1.538062in}}%
\pgfpathlineto{\pgfqpoint{0.605784in}{1.558690in}}%
\pgfpathlineto{\pgfqpoint{0.605784in}{1.579318in}}%
\pgfpathlineto{\pgfqpoint{0.605784in}{1.599946in}}%
\pgfpathlineto{\pgfqpoint{0.605784in}{1.620573in}}%
\pgfpathlineto{\pgfqpoint{0.605784in}{1.641201in}}%
\pgfpathlineto{\pgfqpoint{0.605784in}{1.661829in}}%
\pgfpathlineto{\pgfqpoint{0.605784in}{1.682457in}}%
\pgfpathlineto{\pgfqpoint{0.605784in}{1.701800in}}%
\pgfpathlineto{\pgfqpoint{0.636891in}{1.682457in}}%
\pgfpathlineto{\pgfqpoint{0.646829in}{1.676269in}}%
\pgfpathlineto{\pgfqpoint{0.665631in}{1.661829in}}%
\pgfpathlineto{\pgfqpoint{0.687875in}{1.644591in}}%
\pgfpathlineto{\pgfqpoint{0.692473in}{1.641201in}}%
\pgfpathlineto{\pgfqpoint{0.719610in}{1.620573in}}%
\pgfpathlineto{\pgfqpoint{0.728920in}{1.613290in}}%
\pgfpathlineto{\pgfqpoint{0.753713in}{1.599946in}}%
\pgfpathlineto{\pgfqpoint{0.769965in}{1.590758in}}%
\pgfpathlineto{\pgfqpoint{0.811011in}{1.585142in}}%
\pgfpathlineto{\pgfqpoint{0.847684in}{1.599946in}}%
\pgfpathlineto{\pgfqpoint{0.852056in}{1.601692in}}%
\pgfpathlineto{\pgfqpoint{0.872444in}{1.620573in}}%
\pgfpathlineto{\pgfqpoint{0.893101in}{1.640066in}}%
\pgfpathlineto{\pgfqpoint{0.893949in}{1.641201in}}%
\pgfpathlineto{\pgfqpoint{0.909630in}{1.661829in}}%
\pgfpathlineto{\pgfqpoint{0.925028in}{1.682457in}}%
\pgfpathlineto{\pgfqpoint{0.934147in}{1.694700in}}%
\pgfpathlineto{\pgfqpoint{0.939648in}{1.703085in}}%
\pgfpathlineto{\pgfqpoint{0.953213in}{1.723713in}}%
\pgfpathlineto{\pgfqpoint{0.966518in}{1.744340in}}%
\pgfpathlineto{\pgfqpoint{0.975192in}{1.757866in}}%
\pgfpathlineto{\pgfqpoint{0.979828in}{1.764968in}}%
\pgfpathlineto{\pgfqpoint{0.993293in}{1.785596in}}%
\pgfpathlineto{\pgfqpoint{1.006431in}{1.806224in}}%
\pgfpathlineto{\pgfqpoint{1.016237in}{1.821833in}}%
\pgfpathlineto{\pgfqpoint{1.019810in}{1.826852in}}%
\pgfpathlineto{\pgfqpoint{1.034470in}{1.847480in}}%
\pgfpathlineto{\pgfqpoint{1.048670in}{1.868107in}}%
\pgfpathlineto{\pgfqpoint{1.057283in}{1.880801in}}%
\pgfpathlineto{\pgfqpoint{1.063883in}{1.888735in}}%
\pgfpathlineto{\pgfqpoint{1.080807in}{1.909363in}}%
\pgfpathlineto{\pgfqpoint{1.097076in}{1.929991in}}%
\pgfpathlineto{\pgfqpoint{1.098328in}{1.931575in}}%
\pgfpathlineto{\pgfqpoint{1.117713in}{1.950619in}}%
\pgfpathlineto{\pgfqpoint{1.137790in}{1.971247in}}%
\pgfpathlineto{\pgfqpoint{1.139373in}{1.972877in}}%
\pgfpathlineto{\pgfqpoint{1.163722in}{1.991874in}}%
\pgfpathlineto{\pgfqpoint{1.180419in}{2.005309in}}%
\pgfpathlineto{\pgfqpoint{1.192331in}{2.012502in}}%
\pgfpathlineto{\pgfqpoint{1.221464in}{2.030257in}}%
\pgfpathlineto{\pgfqpoint{1.228032in}{2.033130in}}%
\pgfpathlineto{\pgfqpoint{1.262509in}{2.047985in}}%
\pgfpathlineto{\pgfqpoint{1.286575in}{2.053758in}}%
\pgfpathlineto{\pgfqpoint{1.303555in}{2.057676in}}%
\pgfpathlineto{\pgfqpoint{1.344600in}{2.057033in}}%
\pgfpathlineto{\pgfqpoint{1.354726in}{2.053758in}}%
\pgfpathlineto{\pgfqpoint{1.385645in}{2.043582in}}%
\pgfpathlineto{\pgfqpoint{1.401286in}{2.033130in}}%
\pgfpathlineto{\pgfqpoint{1.426691in}{2.016055in}}%
\pgfpathlineto{\pgfqpoint{1.430287in}{2.012502in}}%
\pgfpathlineto{\pgfqpoint{1.450751in}{1.991874in}}%
\pgfpathlineto{\pgfqpoint{1.467736in}{1.974669in}}%
\pgfpathlineto{\pgfqpoint{1.470510in}{1.971247in}}%
\pgfpathlineto{\pgfqpoint{1.486760in}{1.950619in}}%
\pgfpathlineto{\pgfqpoint{1.502793in}{1.929991in}}%
\pgfpathlineto{\pgfqpoint{1.508781in}{1.922075in}}%
\pgfpathlineto{\pgfqpoint{1.518077in}{1.909363in}}%
\pgfpathlineto{\pgfqpoint{1.532691in}{1.888735in}}%
\pgfpathlineto{\pgfqpoint{1.546907in}{1.868107in}}%
\pgfpathlineto{\pgfqpoint{1.549827in}{1.863687in}}%
\pgfpathlineto{\pgfqpoint{1.562326in}{1.847480in}}%
\pgfpathlineto{\pgfqpoint{1.577522in}{1.826852in}}%
\pgfpathlineto{\pgfqpoint{1.590872in}{1.807843in}}%
\pgfpathlineto{\pgfqpoint{1.592674in}{1.806224in}}%
\pgfpathlineto{\pgfqpoint{1.614036in}{1.785596in}}%
\pgfpathlineto{\pgfqpoint{1.631917in}{1.766939in}}%
\pgfpathlineto{\pgfqpoint{1.640801in}{1.764968in}}%
\pgfpathlineto{\pgfqpoint{1.672963in}{1.756857in}}%
\pgfpathlineto{\pgfqpoint{1.683402in}{1.764968in}}%
\pgfpathlineto{\pgfqpoint{1.710271in}{1.785596in}}%
\pgfpathlineto{\pgfqpoint{1.714008in}{1.788395in}}%
\pgfpathlineto{\pgfqpoint{1.724834in}{1.806224in}}%
\pgfpathlineto{\pgfqpoint{1.737550in}{1.826852in}}%
\pgfpathlineto{\pgfqpoint{1.750378in}{1.847480in}}%
\pgfpathlineto{\pgfqpoint{1.755053in}{1.854777in}}%
\pgfpathlineto{\pgfqpoint{1.761249in}{1.868107in}}%
\pgfpathlineto{\pgfqpoint{1.771043in}{1.888735in}}%
\pgfpathlineto{\pgfqpoint{1.780920in}{1.909363in}}%
\pgfpathlineto{\pgfqpoint{1.790874in}{1.929991in}}%
\pgfpathlineto{\pgfqpoint{1.796099in}{1.940594in}}%
\pgfpathlineto{\pgfqpoint{1.800490in}{1.950619in}}%
\pgfpathlineto{\pgfqpoint{1.809699in}{1.971247in}}%
\pgfpathlineto{\pgfqpoint{1.818930in}{1.991874in}}%
\pgfpathlineto{\pgfqpoint{1.828183in}{2.012502in}}%
\pgfpathlineto{\pgfqpoint{1.837144in}{2.032419in}}%
\pgfpathlineto{\pgfqpoint{1.837467in}{2.033130in}}%
\pgfpathlineto{\pgfqpoint{1.847093in}{2.053758in}}%
\pgfpathlineto{\pgfqpoint{1.856679in}{2.074386in}}%
\pgfpathlineto{\pgfqpoint{1.866228in}{2.095013in}}%
\pgfpathlineto{\pgfqpoint{1.875744in}{2.115641in}}%
\pgfpathlineto{\pgfqpoint{1.878189in}{2.120855in}}%
\pgfpathlineto{\pgfqpoint{1.886199in}{2.136269in}}%
\pgfpathlineto{\pgfqpoint{1.896889in}{2.156897in}}%
\pgfpathlineto{\pgfqpoint{1.907474in}{2.177525in}}%
\pgfpathlineto{\pgfqpoint{1.917965in}{2.198153in}}%
\pgfpathlineto{\pgfqpoint{1.919235in}{2.200610in}}%
\pgfpathlineto{\pgfqpoint{1.930416in}{2.218780in}}%
\pgfpathlineto{\pgfqpoint{1.942957in}{2.239408in}}%
\pgfpathlineto{\pgfqpoint{1.955315in}{2.260036in}}%
\pgfpathlineto{\pgfqpoint{1.960280in}{2.268293in}}%
\pgfpathlineto{\pgfqpoint{1.969712in}{2.280664in}}%
\pgfpathlineto{\pgfqpoint{1.985325in}{2.301292in}}%
\pgfpathlineto{\pgfqpoint{2.000642in}{2.321920in}}%
\pgfpathlineto{\pgfqpoint{2.001325in}{2.322830in}}%
\pgfpathlineto{\pgfqpoint{2.021309in}{2.342547in}}%
\pgfpathlineto{\pgfqpoint{2.041772in}{2.363175in}}%
\pgfpathlineto{\pgfqpoint{2.042371in}{2.363773in}}%
\pgfpathlineto{\pgfqpoint{2.072334in}{2.383803in}}%
\pgfpathlineto{\pgfqpoint{2.083416in}{2.391229in}}%
\pgfpathlineto{\pgfqpoint{2.122143in}{2.404431in}}%
\pgfpathlineto{\pgfqpoint{2.124461in}{2.405207in}}%
\pgfpathlineto{\pgfqpoint{2.165507in}{2.404529in}}%
\pgfpathlineto{\pgfqpoint{2.165754in}{2.404431in}}%
\pgfpathlineto{\pgfqpoint{2.206552in}{2.387793in}}%
\pgfpathlineto{\pgfqpoint{2.211486in}{2.383803in}}%
\pgfpathlineto{\pgfqpoint{2.236474in}{2.363175in}}%
\pgfpathlineto{\pgfqpoint{2.247597in}{2.353838in}}%
\pgfpathlineto{\pgfqpoint{2.256766in}{2.342547in}}%
\pgfpathlineto{\pgfqpoint{2.273200in}{2.321920in}}%
\pgfpathlineto{\pgfqpoint{2.288643in}{2.302280in}}%
\pgfpathlineto{\pgfqpoint{2.289266in}{2.301292in}}%
\pgfpathlineto{\pgfqpoint{2.301864in}{2.280664in}}%
\pgfpathlineto{\pgfqpoint{2.314217in}{2.260036in}}%
\pgfpathlineto{\pgfqpoint{2.326315in}{2.239408in}}%
\pgfpathlineto{\pgfqpoint{2.329688in}{2.233460in}}%
\pgfpathlineto{\pgfqpoint{2.337299in}{2.218780in}}%
\pgfpathlineto{\pgfqpoint{2.347644in}{2.198153in}}%
\pgfpathlineto{\pgfqpoint{2.357667in}{2.177525in}}%
\pgfpathlineto{\pgfqpoint{2.367354in}{2.156897in}}%
\pgfpathlineto{\pgfqpoint{2.370733in}{2.149368in}}%
\pgfpathlineto{\pgfqpoint{2.376904in}{2.136269in}}%
\pgfpathlineto{\pgfqpoint{2.386150in}{2.115641in}}%
\pgfpathlineto{\pgfqpoint{2.394949in}{2.095013in}}%
\pgfpathlineto{\pgfqpoint{2.403285in}{2.074386in}}%
\pgfpathlineto{\pgfqpoint{2.411143in}{2.053758in}}%
\pgfpathlineto{\pgfqpoint{2.411779in}{2.051960in}}%
\pgfpathlineto{\pgfqpoint{2.420096in}{2.033130in}}%
\pgfpathlineto{\pgfqpoint{2.428493in}{2.012502in}}%
\pgfpathlineto{\pgfqpoint{2.436166in}{1.991874in}}%
\pgfpathlineto{\pgfqpoint{2.443100in}{1.971247in}}%
\pgfpathlineto{\pgfqpoint{2.449280in}{1.950619in}}%
\pgfpathlineto{\pgfqpoint{2.452824in}{1.937070in}}%
\pgfpathlineto{\pgfqpoint{2.455919in}{1.929991in}}%
\pgfpathlineto{\pgfqpoint{2.463476in}{1.909363in}}%
\pgfpathlineto{\pgfqpoint{2.469566in}{1.888735in}}%
\pgfpathlineto{\pgfqpoint{2.474202in}{1.868107in}}%
\pgfpathlineto{\pgfqpoint{2.477400in}{1.847480in}}%
\pgfpathlineto{\pgfqpoint{2.479175in}{1.826852in}}%
\pgfpathlineto{\pgfqpoint{2.479540in}{1.806224in}}%
\pgfpathlineto{\pgfqpoint{2.478510in}{1.785596in}}%
\pgfpathlineto{\pgfqpoint{2.476099in}{1.764968in}}%
\pgfpathlineto{\pgfqpoint{2.472320in}{1.744340in}}%
\pgfpathlineto{\pgfqpoint{2.467187in}{1.723713in}}%
\pgfpathlineto{\pgfqpoint{2.460714in}{1.703085in}}%
\pgfpathlineto{\pgfqpoint{2.452913in}{1.682457in}}%
\pgfpathlineto{\pgfqpoint{2.452824in}{1.682255in}}%
\pgfpathlineto{\pgfqpoint{2.446563in}{1.661829in}}%
\pgfpathlineto{\pgfqpoint{2.439187in}{1.641201in}}%
\pgfpathlineto{\pgfqpoint{2.430730in}{1.620573in}}%
\pgfpathlineto{\pgfqpoint{2.421166in}{1.599946in}}%
\pgfpathlineto{\pgfqpoint{2.411779in}{1.581825in}}%
\pgfpathlineto{\pgfqpoint{2.410584in}{1.579318in}}%
\pgfpathlineto{\pgfqpoint{2.399680in}{1.558690in}}%
\pgfpathlineto{\pgfqpoint{2.387520in}{1.538062in}}%
\pgfpathlineto{\pgfqpoint{2.374045in}{1.517434in}}%
\pgfpathlineto{\pgfqpoint{2.370733in}{1.512779in}}%
\pgfpathlineto{\pgfqpoint{2.357796in}{1.496807in}}%
\pgfpathlineto{\pgfqpoint{2.339279in}{1.476179in}}%
\pgfpathlineto{\pgfqpoint{2.329688in}{1.466426in}}%
\pgfpathlineto{\pgfqpoint{2.314615in}{1.455551in}}%
\pgfpathlineto{\pgfqpoint{2.288643in}{1.438623in}}%
\pgfpathlineto{\pgfqpoint{2.278144in}{1.434923in}}%
\pgfpathlineto{\pgfqpoint{2.247597in}{1.425076in}}%
\pgfpathlineto{\pgfqpoint{2.206552in}{1.420299in}}%
\pgfpathlineto{\pgfqpoint{2.165507in}{1.418702in}}%
\pgfpathlineto{\pgfqpoint{2.124461in}{1.415096in}}%
\pgfpathlineto{\pgfqpoint{2.120897in}{1.414295in}}%
\pgfpathlineto{\pgfqpoint{2.083416in}{1.406025in}}%
\pgfpathlineto{\pgfqpoint{2.051985in}{1.393667in}}%
\pgfpathlineto{\pgfqpoint{2.042371in}{1.389878in}}%
\pgfpathlineto{\pgfqpoint{2.010549in}{1.373040in}}%
\pgfpathlineto{\pgfqpoint{2.001325in}{1.368050in}}%
\pgfpathlineto{\pgfqpoint{1.974453in}{1.352412in}}%
\pgfpathlineto{\pgfqpoint{1.960280in}{1.343831in}}%
\pgfpathlineto{\pgfqpoint{1.937451in}{1.331784in}}%
\pgfpathlineto{\pgfqpoint{1.919235in}{1.321639in}}%
\pgfpathlineto{\pgfqpoint{1.891451in}{1.311156in}}%
\pgfpathlineto{\pgfqpoint{1.878189in}{1.305823in}}%
\pgfpathlineto{\pgfqpoint{1.837144in}{1.299516in}}%
\pgfpathlineto{\pgfqpoint{1.796099in}{1.303121in}}%
\pgfpathlineto{\pgfqpoint{1.761656in}{1.311156in}}%
\pgfpathlineto{\pgfqpoint{1.755053in}{1.313040in}}%
\pgfpathlineto{\pgfqpoint{1.714008in}{1.320561in}}%
\pgfpathlineto{\pgfqpoint{1.672963in}{1.312282in}}%
\pgfpathlineto{\pgfqpoint{1.671459in}{1.311156in}}%
\pgfpathlineto{\pgfqpoint{1.642167in}{1.290528in}}%
\pgfpathlineto{\pgfqpoint{1.631917in}{1.283651in}}%
\pgfpathlineto{\pgfqpoint{1.618674in}{1.269900in}}%
\pgfpathlineto{\pgfqpoint{1.597098in}{1.249273in}}%
\pgfpathlineto{\pgfqpoint{1.590872in}{1.243755in}}%
\pgfpathlineto{\pgfqpoint{1.575428in}{1.228645in}}%
\pgfpathlineto{\pgfqpoint{1.552193in}{1.208017in}}%
\pgfpathlineto{\pgfqpoint{1.549827in}{1.206080in}}%
\pgfpathlineto{\pgfqpoint{1.522228in}{1.187389in}}%
\pgfpathlineto{\pgfqpoint{1.508781in}{1.179169in}}%
\pgfpathlineto{\pgfqpoint{1.471335in}{1.166761in}}%
\pgfpathlineto{\pgfqpoint{1.467736in}{1.165685in}}%
\pgfpathlineto{\pgfqpoint{1.426691in}{1.165130in}}%
\pgfpathlineto{\pgfqpoint{1.419006in}{1.166761in}}%
\pgfpathlineto{\pgfqpoint{1.385645in}{1.173883in}}%
\pgfpathlineto{\pgfqpoint{1.344600in}{1.186284in}}%
\pgfpathlineto{\pgfqpoint{1.339958in}{1.187389in}}%
\pgfpathlineto{\pgfqpoint{1.303555in}{1.196368in}}%
\pgfpathlineto{\pgfqpoint{1.262509in}{1.197964in}}%
\pgfpathlineto{\pgfqpoint{1.222737in}{1.187389in}}%
\pgfpathlineto{\pgfqpoint{1.221464in}{1.187056in}}%
\pgfpathlineto{\pgfqpoint{1.186694in}{1.166761in}}%
\pgfpathlineto{\pgfqpoint{1.180419in}{1.163066in}}%
\pgfpathlineto{\pgfqpoint{1.160182in}{1.146134in}}%
\pgfpathlineto{\pgfqpoint{1.139373in}{1.128173in}}%
\pgfpathlineto{\pgfqpoint{1.136627in}{1.125506in}}%
\pgfpathlineto{\pgfqpoint{1.115553in}{1.104878in}}%
\pgfpathlineto{\pgfqpoint{1.098328in}{1.087297in}}%
\pgfpathlineto{\pgfqpoint{1.095190in}{1.084250in}}%
\pgfpathlineto{\pgfqpoint{1.074269in}{1.063622in}}%
\pgfpathlineto{\pgfqpoint{1.057283in}{1.045980in}}%
\pgfpathlineto{\pgfqpoint{1.053811in}{1.042994in}}%
\pgfpathlineto{\pgfqpoint{1.030455in}{1.022367in}}%
\pgfpathlineto{\pgfqpoint{1.016237in}{1.009085in}}%
\pgfpathlineto{\pgfqpoint{1.005628in}{1.001739in}}%
\pgfpathlineto{\pgfqpoint{0.977570in}{0.981111in}}%
\pgfpathlineto{\pgfqpoint{0.975192in}{0.979300in}}%
\pgfpathlineto{\pgfqpoint{0.940014in}{0.960483in}}%
\pgfpathlineto{\pgfqpoint{0.934147in}{0.957160in}}%
\pgfpathlineto{\pgfqpoint{0.893101in}{0.940099in}}%
\pgfpathlineto{\pgfqpoint{0.892450in}{0.939855in}}%
\pgfpathlineto{\pgfqpoint{0.852056in}{0.924459in}}%
\pgfpathlineto{\pgfqpoint{0.839916in}{0.919227in}}%
\pgfpathlineto{\pgfqpoint{0.811011in}{0.906903in}}%
\pgfpathlineto{\pgfqpoint{0.793531in}{0.898600in}}%
\pgfpathlineto{\pgfqpoint{0.769965in}{0.887849in}}%
\pgfpathlineto{\pgfqpoint{0.746179in}{0.877972in}}%
\pgfpathlineto{\pgfqpoint{0.728920in}{0.871256in}}%
\pgfpathlineto{\pgfqpoint{0.687875in}{0.862243in}}%
\pgfpathlineto{\pgfqpoint{0.646829in}{0.864588in}}%
\pgfpathclose%
\pgfpathmoveto{\pgfqpoint{1.837144in}{1.511217in}}%
\pgfpathlineto{\pgfqpoint{1.851045in}{1.496807in}}%
\pgfpathlineto{\pgfqpoint{1.878189in}{1.477110in}}%
\pgfpathlineto{\pgfqpoint{1.886280in}{1.476179in}}%
\pgfpathlineto{\pgfqpoint{1.919235in}{1.473264in}}%
\pgfpathlineto{\pgfqpoint{1.929168in}{1.476179in}}%
\pgfpathlineto{\pgfqpoint{1.960280in}{1.484612in}}%
\pgfpathlineto{\pgfqpoint{1.987631in}{1.496807in}}%
\pgfpathlineto{\pgfqpoint{2.001325in}{1.502574in}}%
\pgfpathlineto{\pgfqpoint{2.034688in}{1.517434in}}%
\pgfpathlineto{\pgfqpoint{2.042371in}{1.520751in}}%
\pgfpathlineto{\pgfqpoint{2.083416in}{1.534795in}}%
\pgfpathlineto{\pgfqpoint{2.099705in}{1.538062in}}%
\pgfpathlineto{\pgfqpoint{2.124461in}{1.543152in}}%
\pgfpathlineto{\pgfqpoint{2.165507in}{1.547483in}}%
\pgfpathlineto{\pgfqpoint{2.206552in}{1.552368in}}%
\pgfpathlineto{\pgfqpoint{2.228328in}{1.558690in}}%
\pgfpathlineto{\pgfqpoint{2.247597in}{1.564899in}}%
\pgfpathlineto{\pgfqpoint{2.268911in}{1.579318in}}%
\pgfpathlineto{\pgfqpoint{2.288643in}{1.594552in}}%
\pgfpathlineto{\pgfqpoint{2.292980in}{1.599946in}}%
\pgfpathlineto{\pgfqpoint{2.307783in}{1.620573in}}%
\pgfpathlineto{\pgfqpoint{2.320610in}{1.641201in}}%
\pgfpathlineto{\pgfqpoint{2.329688in}{1.658121in}}%
\pgfpathlineto{\pgfqpoint{2.331235in}{1.661829in}}%
\pgfpathlineto{\pgfqpoint{2.338763in}{1.682457in}}%
\pgfpathlineto{\pgfqpoint{2.345143in}{1.703085in}}%
\pgfpathlineto{\pgfqpoint{2.350440in}{1.723713in}}%
\pgfpathlineto{\pgfqpoint{2.354713in}{1.744340in}}%
\pgfpathlineto{\pgfqpoint{2.358017in}{1.764968in}}%
\pgfpathlineto{\pgfqpoint{2.360405in}{1.785596in}}%
\pgfpathlineto{\pgfqpoint{2.361925in}{1.806224in}}%
\pgfpathlineto{\pgfqpoint{2.362620in}{1.826852in}}%
\pgfpathlineto{\pgfqpoint{2.362534in}{1.847480in}}%
\pgfpathlineto{\pgfqpoint{2.361704in}{1.868107in}}%
\pgfpathlineto{\pgfqpoint{2.360168in}{1.888735in}}%
\pgfpathlineto{\pgfqpoint{2.357959in}{1.909363in}}%
\pgfpathlineto{\pgfqpoint{2.355109in}{1.929991in}}%
\pgfpathlineto{\pgfqpoint{2.351649in}{1.950619in}}%
\pgfpathlineto{\pgfqpoint{2.347606in}{1.971247in}}%
\pgfpathlineto{\pgfqpoint{2.343006in}{1.991874in}}%
\pgfpathlineto{\pgfqpoint{2.337875in}{2.012502in}}%
\pgfpathlineto{\pgfqpoint{2.332236in}{2.033130in}}%
\pgfpathlineto{\pgfqpoint{2.329688in}{2.041637in}}%
\pgfpathlineto{\pgfqpoint{2.325614in}{2.053758in}}%
\pgfpathlineto{\pgfqpoint{2.318150in}{2.074386in}}%
\pgfpathlineto{\pgfqpoint{2.310229in}{2.095013in}}%
\pgfpathlineto{\pgfqpoint{2.301876in}{2.115641in}}%
\pgfpathlineto{\pgfqpoint{2.293113in}{2.136269in}}%
\pgfpathlineto{\pgfqpoint{2.288643in}{2.146257in}}%
\pgfpathlineto{\pgfqpoint{2.282579in}{2.156897in}}%
\pgfpathlineto{\pgfqpoint{2.270307in}{2.177525in}}%
\pgfpathlineto{\pgfqpoint{2.257657in}{2.198153in}}%
\pgfpathlineto{\pgfqpoint{2.247597in}{2.214036in}}%
\pgfpathlineto{\pgfqpoint{2.243116in}{2.218780in}}%
\pgfpathlineto{\pgfqpoint{2.222906in}{2.239408in}}%
\pgfpathlineto{\pgfqpoint{2.206552in}{2.255731in}}%
\pgfpathlineto{\pgfqpoint{2.197966in}{2.260036in}}%
\pgfpathlineto{\pgfqpoint{2.165507in}{2.275716in}}%
\pgfpathlineto{\pgfqpoint{2.124461in}{2.277177in}}%
\pgfpathlineto{\pgfqpoint{2.083416in}{2.262472in}}%
\pgfpathlineto{\pgfqpoint{2.080014in}{2.260036in}}%
\pgfpathlineto{\pgfqpoint{2.051494in}{2.239408in}}%
\pgfpathlineto{\pgfqpoint{2.042371in}{2.232798in}}%
\pgfpathlineto{\pgfqpoint{2.029567in}{2.218780in}}%
\pgfpathlineto{\pgfqpoint{2.010526in}{2.198153in}}%
\pgfpathlineto{\pgfqpoint{2.001325in}{2.188213in}}%
\pgfpathlineto{\pgfqpoint{1.994150in}{2.177525in}}%
\pgfpathlineto{\pgfqpoint{1.980322in}{2.156897in}}%
\pgfpathlineto{\pgfqpoint{1.966236in}{2.136269in}}%
\pgfpathlineto{\pgfqpoint{1.960280in}{2.127509in}}%
\pgfpathlineto{\pgfqpoint{1.954097in}{2.115641in}}%
\pgfpathlineto{\pgfqpoint{1.943423in}{2.095013in}}%
\pgfpathlineto{\pgfqpoint{1.932648in}{2.074386in}}%
\pgfpathlineto{\pgfqpoint{1.921761in}{2.053758in}}%
\pgfpathlineto{\pgfqpoint{1.919235in}{2.048871in}}%
\pgfpathlineto{\pgfqpoint{1.912641in}{2.033130in}}%
\pgfpathlineto{\pgfqpoint{1.904092in}{2.012502in}}%
\pgfpathlineto{\pgfqpoint{1.895568in}{1.991874in}}%
\pgfpathlineto{\pgfqpoint{1.887071in}{1.971247in}}%
\pgfpathlineto{\pgfqpoint{1.878604in}{1.950619in}}%
\pgfpathlineto{\pgfqpoint{1.878189in}{1.949562in}}%
\pgfpathlineto{\pgfqpoint{1.871511in}{1.929991in}}%
\pgfpathlineto{\pgfqpoint{1.864627in}{1.909363in}}%
\pgfpathlineto{\pgfqpoint{1.857900in}{1.888735in}}%
\pgfpathlineto{\pgfqpoint{1.851349in}{1.868107in}}%
\pgfpathlineto{\pgfqpoint{1.844997in}{1.847480in}}%
\pgfpathlineto{\pgfqpoint{1.838870in}{1.826852in}}%
\pgfpathlineto{\pgfqpoint{1.837144in}{1.820594in}}%
\pgfpathlineto{\pgfqpoint{1.833359in}{1.806224in}}%
\pgfpathlineto{\pgfqpoint{1.828329in}{1.785596in}}%
\pgfpathlineto{\pgfqpoint{1.823683in}{1.764968in}}%
\pgfpathlineto{\pgfqpoint{1.819473in}{1.744340in}}%
\pgfpathlineto{\pgfqpoint{1.815757in}{1.723713in}}%
\pgfpathlineto{\pgfqpoint{1.812606in}{1.703085in}}%
\pgfpathlineto{\pgfqpoint{1.810107in}{1.682457in}}%
\pgfpathlineto{\pgfqpoint{1.808364in}{1.661829in}}%
\pgfpathlineto{\pgfqpoint{1.807502in}{1.641201in}}%
\pgfpathlineto{\pgfqpoint{1.807680in}{1.620573in}}%
\pgfpathlineto{\pgfqpoint{1.809094in}{1.599946in}}%
\pgfpathlineto{\pgfqpoint{1.811994in}{1.579318in}}%
\pgfpathlineto{\pgfqpoint{1.816702in}{1.558690in}}%
\pgfpathlineto{\pgfqpoint{1.823641in}{1.538062in}}%
\pgfpathlineto{\pgfqpoint{1.833377in}{1.517434in}}%
\pgfpathclose%
\pgfusepath{stroke,fill}%
\end{pgfscope}%
\begin{pgfscope}%
\pgfpathrectangle{\pgfqpoint{0.605784in}{0.382904in}}{\pgfqpoint{4.063488in}{2.042155in}}%
\pgfusepath{clip}%
\pgfsetbuttcap%
\pgfsetroundjoin%
\definecolor{currentfill}{rgb}{0.281887,0.150881,0.465405}%
\pgfsetfillcolor{currentfill}%
\pgfsetlinewidth{1.003750pt}%
\definecolor{currentstroke}{rgb}{0.281887,0.150881,0.465405}%
\pgfsetstrokecolor{currentstroke}%
\pgfsetdash{}{0pt}%
\pgfpathmoveto{\pgfqpoint{2.534462in}{1.042994in}}%
\pgfpathlineto{\pgfqpoint{2.533455in}{1.063622in}}%
\pgfpathlineto{\pgfqpoint{2.532449in}{1.084250in}}%
\pgfpathlineto{\pgfqpoint{2.531444in}{1.104878in}}%
\pgfpathlineto{\pgfqpoint{2.530441in}{1.125506in}}%
\pgfpathlineto{\pgfqpoint{2.529440in}{1.146134in}}%
\pgfpathlineto{\pgfqpoint{2.528439in}{1.166761in}}%
\pgfpathlineto{\pgfqpoint{2.527441in}{1.187389in}}%
\pgfpathlineto{\pgfqpoint{2.526445in}{1.208017in}}%
\pgfpathlineto{\pgfqpoint{2.525451in}{1.228645in}}%
\pgfpathlineto{\pgfqpoint{2.524460in}{1.249273in}}%
\pgfpathlineto{\pgfqpoint{2.523471in}{1.269900in}}%
\pgfpathlineto{\pgfqpoint{2.522486in}{1.290528in}}%
\pgfpathlineto{\pgfqpoint{2.521504in}{1.311156in}}%
\pgfpathlineto{\pgfqpoint{2.520526in}{1.331784in}}%
\pgfpathlineto{\pgfqpoint{2.519553in}{1.352412in}}%
\pgfpathlineto{\pgfqpoint{2.518585in}{1.373040in}}%
\pgfpathlineto{\pgfqpoint{2.517624in}{1.393667in}}%
\pgfpathlineto{\pgfqpoint{2.516670in}{1.414295in}}%
\pgfpathlineto{\pgfqpoint{2.515725in}{1.434923in}}%
\pgfpathlineto{\pgfqpoint{2.514790in}{1.455551in}}%
\pgfpathlineto{\pgfqpoint{2.513867in}{1.476179in}}%
\pgfpathlineto{\pgfqpoint{2.512960in}{1.496807in}}%
\pgfpathlineto{\pgfqpoint{2.512071in}{1.517434in}}%
\pgfpathlineto{\pgfqpoint{2.511206in}{1.538062in}}%
\pgfpathlineto{\pgfqpoint{2.510372in}{1.558690in}}%
\pgfpathlineto{\pgfqpoint{2.509577in}{1.579318in}}%
\pgfpathlineto{\pgfqpoint{2.508836in}{1.599946in}}%
\pgfpathlineto{\pgfqpoint{2.508169in}{1.620573in}}%
\pgfpathlineto{\pgfqpoint{2.507610in}{1.641201in}}%
\pgfpathlineto{\pgfqpoint{2.507218in}{1.661829in}}%
\pgfpathlineto{\pgfqpoint{2.507099in}{1.682457in}}%
\pgfpathlineto{\pgfqpoint{2.507478in}{1.703085in}}%
\pgfpathlineto{\pgfqpoint{2.508923in}{1.723713in}}%
\pgfpathlineto{\pgfqpoint{2.513384in}{1.744340in}}%
\pgfpathlineto{\pgfqpoint{2.534915in}{1.764918in}}%
\pgfpathlineto{\pgfqpoint{2.575960in}{1.764918in}}%
\pgfpathlineto{\pgfqpoint{2.617005in}{1.764918in}}%
\pgfpathlineto{\pgfqpoint{2.658051in}{1.764918in}}%
\pgfpathlineto{\pgfqpoint{2.699096in}{1.764918in}}%
\pgfpathlineto{\pgfqpoint{2.740141in}{1.764918in}}%
\pgfpathlineto{\pgfqpoint{2.781187in}{1.764918in}}%
\pgfpathlineto{\pgfqpoint{2.822232in}{1.764918in}}%
\pgfpathlineto{\pgfqpoint{2.863277in}{1.764918in}}%
\pgfpathlineto{\pgfqpoint{2.904323in}{1.764918in}}%
\pgfpathlineto{\pgfqpoint{2.945368in}{1.764918in}}%
\pgfpathlineto{\pgfqpoint{2.986413in}{1.764918in}}%
\pgfpathlineto{\pgfqpoint{3.027459in}{1.764918in}}%
\pgfpathlineto{\pgfqpoint{3.068504in}{1.764918in}}%
\pgfpathlineto{\pgfqpoint{3.109549in}{1.764918in}}%
\pgfpathlineto{\pgfqpoint{3.150595in}{1.764918in}}%
\pgfpathlineto{\pgfqpoint{3.191640in}{1.764918in}}%
\pgfpathlineto{\pgfqpoint{3.232685in}{1.764918in}}%
\pgfpathlineto{\pgfqpoint{3.273731in}{1.764918in}}%
\pgfpathlineto{\pgfqpoint{3.314776in}{1.764918in}}%
\pgfpathlineto{\pgfqpoint{3.355821in}{1.764918in}}%
\pgfpathlineto{\pgfqpoint{3.396867in}{1.764918in}}%
\pgfpathlineto{\pgfqpoint{3.437912in}{1.764918in}}%
\pgfpathlineto{\pgfqpoint{3.478957in}{1.764918in}}%
\pgfpathlineto{\pgfqpoint{3.520003in}{1.764918in}}%
\pgfpathlineto{\pgfqpoint{3.561048in}{1.764918in}}%
\pgfpathlineto{\pgfqpoint{3.602093in}{1.764918in}}%
\pgfpathlineto{\pgfqpoint{3.643139in}{1.764918in}}%
\pgfpathlineto{\pgfqpoint{3.646178in}{1.744340in}}%
\pgfpathlineto{\pgfqpoint{3.649329in}{1.723713in}}%
\pgfpathlineto{\pgfqpoint{3.652601in}{1.703085in}}%
\pgfpathlineto{\pgfqpoint{3.656013in}{1.682457in}}%
\pgfpathlineto{\pgfqpoint{3.659587in}{1.661829in}}%
\pgfpathlineto{\pgfqpoint{3.663353in}{1.641201in}}%
\pgfpathlineto{\pgfqpoint{3.667347in}{1.620573in}}%
\pgfpathlineto{\pgfqpoint{3.671613in}{1.599946in}}%
\pgfpathlineto{\pgfqpoint{3.676211in}{1.579318in}}%
\pgfpathlineto{\pgfqpoint{3.681218in}{1.558690in}}%
\pgfpathlineto{\pgfqpoint{3.684184in}{1.547190in}}%
\pgfpathlineto{\pgfqpoint{3.725229in}{1.547190in}}%
\pgfpathlineto{\pgfqpoint{3.766275in}{1.547190in}}%
\pgfpathlineto{\pgfqpoint{3.807320in}{1.547190in}}%
\pgfpathlineto{\pgfqpoint{3.848365in}{1.547190in}}%
\pgfpathlineto{\pgfqpoint{3.889411in}{1.547190in}}%
\pgfpathlineto{\pgfqpoint{3.930456in}{1.547190in}}%
\pgfpathlineto{\pgfqpoint{3.971501in}{1.547190in}}%
\pgfpathlineto{\pgfqpoint{4.012547in}{1.547190in}}%
\pgfpathlineto{\pgfqpoint{4.053592in}{1.547190in}}%
\pgfpathlineto{\pgfqpoint{4.094637in}{1.547190in}}%
\pgfpathlineto{\pgfqpoint{4.135683in}{1.547190in}}%
\pgfpathlineto{\pgfqpoint{4.176728in}{1.547190in}}%
\pgfpathlineto{\pgfqpoint{4.217773in}{1.547190in}}%
\pgfpathlineto{\pgfqpoint{4.258819in}{1.547190in}}%
\pgfpathlineto{\pgfqpoint{4.299864in}{1.547190in}}%
\pgfpathlineto{\pgfqpoint{4.340909in}{1.547190in}}%
\pgfpathlineto{\pgfqpoint{4.381955in}{1.547190in}}%
\pgfpathlineto{\pgfqpoint{4.423000in}{1.547190in}}%
\pgfpathlineto{\pgfqpoint{4.464045in}{1.547190in}}%
\pgfpathlineto{\pgfqpoint{4.505091in}{1.547190in}}%
\pgfpathlineto{\pgfqpoint{4.546136in}{1.547190in}}%
\pgfpathlineto{\pgfqpoint{4.587181in}{1.547190in}}%
\pgfpathlineto{\pgfqpoint{4.628227in}{1.547190in}}%
\pgfpathlineto{\pgfqpoint{4.669272in}{1.547190in}}%
\pgfpathlineto{\pgfqpoint{4.669272in}{1.538062in}}%
\pgfpathlineto{\pgfqpoint{4.669272in}{1.517434in}}%
\pgfpathlineto{\pgfqpoint{4.669272in}{1.496807in}}%
\pgfpathlineto{\pgfqpoint{4.669272in}{1.476179in}}%
\pgfpathlineto{\pgfqpoint{4.669272in}{1.455551in}}%
\pgfpathlineto{\pgfqpoint{4.669272in}{1.434923in}}%
\pgfpathlineto{\pgfqpoint{4.669272in}{1.414295in}}%
\pgfpathlineto{\pgfqpoint{4.669272in}{1.393667in}}%
\pgfpathlineto{\pgfqpoint{4.669272in}{1.373040in}}%
\pgfpathlineto{\pgfqpoint{4.669272in}{1.352412in}}%
\pgfpathlineto{\pgfqpoint{4.669272in}{1.334339in}}%
\pgfpathlineto{\pgfqpoint{4.628227in}{1.334339in}}%
\pgfpathlineto{\pgfqpoint{4.587181in}{1.334339in}}%
\pgfpathlineto{\pgfqpoint{4.546136in}{1.334339in}}%
\pgfpathlineto{\pgfqpoint{4.505091in}{1.334339in}}%
\pgfpathlineto{\pgfqpoint{4.464045in}{1.334339in}}%
\pgfpathlineto{\pgfqpoint{4.423000in}{1.334339in}}%
\pgfpathlineto{\pgfqpoint{4.381955in}{1.334339in}}%
\pgfpathlineto{\pgfqpoint{4.340909in}{1.334339in}}%
\pgfpathlineto{\pgfqpoint{4.299864in}{1.334339in}}%
\pgfpathlineto{\pgfqpoint{4.258819in}{1.334339in}}%
\pgfpathlineto{\pgfqpoint{4.217773in}{1.334339in}}%
\pgfpathlineto{\pgfqpoint{4.176728in}{1.334339in}}%
\pgfpathlineto{\pgfqpoint{4.135683in}{1.334339in}}%
\pgfpathlineto{\pgfqpoint{4.094637in}{1.334339in}}%
\pgfpathlineto{\pgfqpoint{4.053592in}{1.334339in}}%
\pgfpathlineto{\pgfqpoint{4.012547in}{1.334339in}}%
\pgfpathlineto{\pgfqpoint{3.971501in}{1.334339in}}%
\pgfpathlineto{\pgfqpoint{3.930456in}{1.334339in}}%
\pgfpathlineto{\pgfqpoint{3.889411in}{1.334339in}}%
\pgfpathlineto{\pgfqpoint{3.848365in}{1.334339in}}%
\pgfpathlineto{\pgfqpoint{3.807320in}{1.334339in}}%
\pgfpathlineto{\pgfqpoint{3.766275in}{1.334339in}}%
\pgfpathlineto{\pgfqpoint{3.725229in}{1.334339in}}%
\pgfpathlineto{\pgfqpoint{3.684184in}{1.334339in}}%
\pgfpathlineto{\pgfqpoint{3.675349in}{1.352412in}}%
\pgfpathlineto{\pgfqpoint{3.669022in}{1.373040in}}%
\pgfpathlineto{\pgfqpoint{3.664448in}{1.393667in}}%
\pgfpathlineto{\pgfqpoint{3.660750in}{1.414295in}}%
\pgfpathlineto{\pgfqpoint{3.657553in}{1.434923in}}%
\pgfpathlineto{\pgfqpoint{3.654668in}{1.455551in}}%
\pgfpathlineto{\pgfqpoint{3.651992in}{1.476179in}}%
\pgfpathlineto{\pgfqpoint{3.649462in}{1.496807in}}%
\pgfpathlineto{\pgfqpoint{3.647039in}{1.517434in}}%
\pgfpathlineto{\pgfqpoint{3.644695in}{1.538062in}}%
\pgfpathlineto{\pgfqpoint{3.643139in}{1.551765in}}%
\pgfpathlineto{\pgfqpoint{3.602093in}{1.551765in}}%
\pgfpathlineto{\pgfqpoint{3.561048in}{1.551765in}}%
\pgfpathlineto{\pgfqpoint{3.520003in}{1.551765in}}%
\pgfpathlineto{\pgfqpoint{3.478957in}{1.551765in}}%
\pgfpathlineto{\pgfqpoint{3.437912in}{1.551765in}}%
\pgfpathlineto{\pgfqpoint{3.396867in}{1.551765in}}%
\pgfpathlineto{\pgfqpoint{3.355821in}{1.551765in}}%
\pgfpathlineto{\pgfqpoint{3.314776in}{1.551765in}}%
\pgfpathlineto{\pgfqpoint{3.273731in}{1.551765in}}%
\pgfpathlineto{\pgfqpoint{3.232685in}{1.551765in}}%
\pgfpathlineto{\pgfqpoint{3.191640in}{1.551765in}}%
\pgfpathlineto{\pgfqpoint{3.150595in}{1.551765in}}%
\pgfpathlineto{\pgfqpoint{3.109549in}{1.551765in}}%
\pgfpathlineto{\pgfqpoint{3.068504in}{1.551765in}}%
\pgfpathlineto{\pgfqpoint{3.027459in}{1.551765in}}%
\pgfpathlineto{\pgfqpoint{2.986413in}{1.551765in}}%
\pgfpathlineto{\pgfqpoint{2.945368in}{1.551765in}}%
\pgfpathlineto{\pgfqpoint{2.904323in}{1.551765in}}%
\pgfpathlineto{\pgfqpoint{2.863277in}{1.551765in}}%
\pgfpathlineto{\pgfqpoint{2.822232in}{1.551765in}}%
\pgfpathlineto{\pgfqpoint{2.781187in}{1.551765in}}%
\pgfpathlineto{\pgfqpoint{2.740141in}{1.551765in}}%
\pgfpathlineto{\pgfqpoint{2.699096in}{1.551765in}}%
\pgfpathlineto{\pgfqpoint{2.658051in}{1.551765in}}%
\pgfpathlineto{\pgfqpoint{2.617005in}{1.551765in}}%
\pgfpathlineto{\pgfqpoint{2.575960in}{1.551765in}}%
\pgfpathlineto{\pgfqpoint{2.534915in}{1.551765in}}%
\pgfpathlineto{\pgfqpoint{2.534068in}{1.538062in}}%
\pgfpathlineto{\pgfqpoint{2.533124in}{1.517434in}}%
\pgfpathlineto{\pgfqpoint{2.532469in}{1.496807in}}%
\pgfpathlineto{\pgfqpoint{2.532044in}{1.476179in}}%
\pgfpathlineto{\pgfqpoint{2.531804in}{1.455551in}}%
\pgfpathlineto{\pgfqpoint{2.531716in}{1.434923in}}%
\pgfpathlineto{\pgfqpoint{2.531755in}{1.414295in}}%
\pgfpathlineto{\pgfqpoint{2.531899in}{1.393667in}}%
\pgfpathlineto{\pgfqpoint{2.532134in}{1.373040in}}%
\pgfpathlineto{\pgfqpoint{2.532445in}{1.352412in}}%
\pgfpathlineto{\pgfqpoint{2.532822in}{1.331784in}}%
\pgfpathlineto{\pgfqpoint{2.533257in}{1.311156in}}%
\pgfpathlineto{\pgfqpoint{2.533742in}{1.290528in}}%
\pgfpathlineto{\pgfqpoint{2.534270in}{1.269900in}}%
\pgfpathlineto{\pgfqpoint{2.534837in}{1.249273in}}%
\pgfpathlineto{\pgfqpoint{2.534915in}{1.246718in}}%
\pgfpathlineto{\pgfqpoint{2.575960in}{1.246718in}}%
\pgfpathlineto{\pgfqpoint{2.617005in}{1.246718in}}%
\pgfpathlineto{\pgfqpoint{2.658051in}{1.246718in}}%
\pgfpathlineto{\pgfqpoint{2.699096in}{1.246718in}}%
\pgfpathlineto{\pgfqpoint{2.740141in}{1.246718in}}%
\pgfpathlineto{\pgfqpoint{2.781187in}{1.246718in}}%
\pgfpathlineto{\pgfqpoint{2.822232in}{1.246718in}}%
\pgfpathlineto{\pgfqpoint{2.863277in}{1.246718in}}%
\pgfpathlineto{\pgfqpoint{2.904323in}{1.246718in}}%
\pgfpathlineto{\pgfqpoint{2.945368in}{1.246718in}}%
\pgfpathlineto{\pgfqpoint{2.986413in}{1.246718in}}%
\pgfpathlineto{\pgfqpoint{3.027459in}{1.246718in}}%
\pgfpathlineto{\pgfqpoint{3.068504in}{1.246718in}}%
\pgfpathlineto{\pgfqpoint{3.109549in}{1.246718in}}%
\pgfpathlineto{\pgfqpoint{3.150595in}{1.246718in}}%
\pgfpathlineto{\pgfqpoint{3.191640in}{1.246718in}}%
\pgfpathlineto{\pgfqpoint{3.232685in}{1.246718in}}%
\pgfpathlineto{\pgfqpoint{3.273731in}{1.246718in}}%
\pgfpathlineto{\pgfqpoint{3.314776in}{1.246718in}}%
\pgfpathlineto{\pgfqpoint{3.355821in}{1.246718in}}%
\pgfpathlineto{\pgfqpoint{3.396867in}{1.246718in}}%
\pgfpathlineto{\pgfqpoint{3.437912in}{1.246718in}}%
\pgfpathlineto{\pgfqpoint{3.478957in}{1.246718in}}%
\pgfpathlineto{\pgfqpoint{3.520003in}{1.246718in}}%
\pgfpathlineto{\pgfqpoint{3.561048in}{1.246718in}}%
\pgfpathlineto{\pgfqpoint{3.602093in}{1.246718in}}%
\pgfpathlineto{\pgfqpoint{3.643139in}{1.246718in}}%
\pgfpathlineto{\pgfqpoint{3.651973in}{1.228645in}}%
\pgfpathlineto{\pgfqpoint{3.658300in}{1.208017in}}%
\pgfpathlineto{\pgfqpoint{3.662874in}{1.187389in}}%
\pgfpathlineto{\pgfqpoint{3.666573in}{1.166761in}}%
\pgfpathlineto{\pgfqpoint{3.669770in}{1.146134in}}%
\pgfpathlineto{\pgfqpoint{3.672654in}{1.125506in}}%
\pgfpathlineto{\pgfqpoint{3.675330in}{1.104878in}}%
\pgfpathlineto{\pgfqpoint{3.677860in}{1.084250in}}%
\pgfpathlineto{\pgfqpoint{3.680284in}{1.063622in}}%
\pgfpathlineto{\pgfqpoint{3.682627in}{1.042994in}}%
\pgfpathlineto{\pgfqpoint{3.684184in}{1.029292in}}%
\pgfpathlineto{\pgfqpoint{3.725229in}{1.029292in}}%
\pgfpathlineto{\pgfqpoint{3.766275in}{1.029292in}}%
\pgfpathlineto{\pgfqpoint{3.807320in}{1.029292in}}%
\pgfpathlineto{\pgfqpoint{3.848365in}{1.029292in}}%
\pgfpathlineto{\pgfqpoint{3.889411in}{1.029292in}}%
\pgfpathlineto{\pgfqpoint{3.930456in}{1.029292in}}%
\pgfpathlineto{\pgfqpoint{3.971501in}{1.029292in}}%
\pgfpathlineto{\pgfqpoint{4.012547in}{1.029292in}}%
\pgfpathlineto{\pgfqpoint{4.053592in}{1.029292in}}%
\pgfpathlineto{\pgfqpoint{4.094637in}{1.029292in}}%
\pgfpathlineto{\pgfqpoint{4.135683in}{1.029292in}}%
\pgfpathlineto{\pgfqpoint{4.176728in}{1.029292in}}%
\pgfpathlineto{\pgfqpoint{4.217773in}{1.029292in}}%
\pgfpathlineto{\pgfqpoint{4.258819in}{1.029292in}}%
\pgfpathlineto{\pgfqpoint{4.299864in}{1.029292in}}%
\pgfpathlineto{\pgfqpoint{4.340909in}{1.029292in}}%
\pgfpathlineto{\pgfqpoint{4.381955in}{1.029292in}}%
\pgfpathlineto{\pgfqpoint{4.423000in}{1.029292in}}%
\pgfpathlineto{\pgfqpoint{4.464045in}{1.029292in}}%
\pgfpathlineto{\pgfqpoint{4.505091in}{1.029292in}}%
\pgfpathlineto{\pgfqpoint{4.546136in}{1.029292in}}%
\pgfpathlineto{\pgfqpoint{4.587181in}{1.029292in}}%
\pgfpathlineto{\pgfqpoint{4.628227in}{1.029292in}}%
\pgfpathlineto{\pgfqpoint{4.669272in}{1.029292in}}%
\pgfpathlineto{\pgfqpoint{4.669272in}{1.022367in}}%
\pgfpathlineto{\pgfqpoint{4.669272in}{1.001739in}}%
\pgfpathlineto{\pgfqpoint{4.669272in}{0.981111in}}%
\pgfpathlineto{\pgfqpoint{4.669272in}{0.960483in}}%
\pgfpathlineto{\pgfqpoint{4.669272in}{0.939855in}}%
\pgfpathlineto{\pgfqpoint{4.669272in}{0.919227in}}%
\pgfpathlineto{\pgfqpoint{4.669272in}{0.898600in}}%
\pgfpathlineto{\pgfqpoint{4.669272in}{0.877972in}}%
\pgfpathlineto{\pgfqpoint{4.669272in}{0.857344in}}%
\pgfpathlineto{\pgfqpoint{4.669272in}{0.836716in}}%
\pgfpathlineto{\pgfqpoint{4.669272in}{0.816139in}}%
\pgfpathlineto{\pgfqpoint{4.628227in}{0.816139in}}%
\pgfpathlineto{\pgfqpoint{4.587181in}{0.816139in}}%
\pgfpathlineto{\pgfqpoint{4.546136in}{0.816139in}}%
\pgfpathlineto{\pgfqpoint{4.505091in}{0.816139in}}%
\pgfpathlineto{\pgfqpoint{4.464045in}{0.816139in}}%
\pgfpathlineto{\pgfqpoint{4.423000in}{0.816139in}}%
\pgfpathlineto{\pgfqpoint{4.381955in}{0.816139in}}%
\pgfpathlineto{\pgfqpoint{4.340909in}{0.816139in}}%
\pgfpathlineto{\pgfqpoint{4.299864in}{0.816139in}}%
\pgfpathlineto{\pgfqpoint{4.258819in}{0.816139in}}%
\pgfpathlineto{\pgfqpoint{4.217773in}{0.816139in}}%
\pgfpathlineto{\pgfqpoint{4.176728in}{0.816139in}}%
\pgfpathlineto{\pgfqpoint{4.135683in}{0.816139in}}%
\pgfpathlineto{\pgfqpoint{4.094637in}{0.816139in}}%
\pgfpathlineto{\pgfqpoint{4.053592in}{0.816139in}}%
\pgfpathlineto{\pgfqpoint{4.012547in}{0.816139in}}%
\pgfpathlineto{\pgfqpoint{3.971501in}{0.816139in}}%
\pgfpathlineto{\pgfqpoint{3.930456in}{0.816139in}}%
\pgfpathlineto{\pgfqpoint{3.889411in}{0.816139in}}%
\pgfpathlineto{\pgfqpoint{3.848365in}{0.816139in}}%
\pgfpathlineto{\pgfqpoint{3.807320in}{0.816139in}}%
\pgfpathlineto{\pgfqpoint{3.766275in}{0.816139in}}%
\pgfpathlineto{\pgfqpoint{3.725229in}{0.816139in}}%
\pgfpathlineto{\pgfqpoint{3.684184in}{0.816139in}}%
\pgfpathlineto{\pgfqpoint{3.681145in}{0.836716in}}%
\pgfpathlineto{\pgfqpoint{3.677993in}{0.857344in}}%
\pgfpathlineto{\pgfqpoint{3.674721in}{0.877972in}}%
\pgfpathlineto{\pgfqpoint{3.671310in}{0.898600in}}%
\pgfpathlineto{\pgfqpoint{3.667735in}{0.919227in}}%
\pgfpathlineto{\pgfqpoint{3.663969in}{0.939855in}}%
\pgfpathlineto{\pgfqpoint{3.659976in}{0.960483in}}%
\pgfpathlineto{\pgfqpoint{3.655709in}{0.981111in}}%
\pgfpathlineto{\pgfqpoint{3.651111in}{1.001739in}}%
\pgfpathlineto{\pgfqpoint{3.646105in}{1.022367in}}%
\pgfpathlineto{\pgfqpoint{3.643139in}{1.033867in}}%
\pgfpathlineto{\pgfqpoint{3.602093in}{1.033867in}}%
\pgfpathlineto{\pgfqpoint{3.561048in}{1.033867in}}%
\pgfpathlineto{\pgfqpoint{3.520003in}{1.033867in}}%
\pgfpathlineto{\pgfqpoint{3.478957in}{1.033867in}}%
\pgfpathlineto{\pgfqpoint{3.437912in}{1.033867in}}%
\pgfpathlineto{\pgfqpoint{3.396867in}{1.033867in}}%
\pgfpathlineto{\pgfqpoint{3.355821in}{1.033867in}}%
\pgfpathlineto{\pgfqpoint{3.314776in}{1.033867in}}%
\pgfpathlineto{\pgfqpoint{3.273731in}{1.033867in}}%
\pgfpathlineto{\pgfqpoint{3.232685in}{1.033867in}}%
\pgfpathlineto{\pgfqpoint{3.191640in}{1.033867in}}%
\pgfpathlineto{\pgfqpoint{3.150595in}{1.033867in}}%
\pgfpathlineto{\pgfqpoint{3.109549in}{1.033867in}}%
\pgfpathlineto{\pgfqpoint{3.068504in}{1.033867in}}%
\pgfpathlineto{\pgfqpoint{3.027459in}{1.033867in}}%
\pgfpathlineto{\pgfqpoint{2.986413in}{1.033867in}}%
\pgfpathlineto{\pgfqpoint{2.945368in}{1.033867in}}%
\pgfpathlineto{\pgfqpoint{2.904323in}{1.033867in}}%
\pgfpathlineto{\pgfqpoint{2.863277in}{1.033867in}}%
\pgfpathlineto{\pgfqpoint{2.822232in}{1.033867in}}%
\pgfpathlineto{\pgfqpoint{2.781187in}{1.033867in}}%
\pgfpathlineto{\pgfqpoint{2.740141in}{1.033867in}}%
\pgfpathlineto{\pgfqpoint{2.699096in}{1.033867in}}%
\pgfpathlineto{\pgfqpoint{2.658051in}{1.033867in}}%
\pgfpathlineto{\pgfqpoint{2.617005in}{1.033867in}}%
\pgfpathlineto{\pgfqpoint{2.575960in}{1.033867in}}%
\pgfpathlineto{\pgfqpoint{2.534915in}{1.033867in}}%
\pgfpathclose%
\pgfusepath{stroke,fill}%
\end{pgfscope}%
\begin{pgfscope}%
\pgfpathrectangle{\pgfqpoint{0.605784in}{0.382904in}}{\pgfqpoint{4.063488in}{2.042155in}}%
\pgfusepath{clip}%
\pgfsetbuttcap%
\pgfsetroundjoin%
\definecolor{currentfill}{rgb}{0.263663,0.237631,0.518762}%
\pgfsetfillcolor{currentfill}%
\pgfsetlinewidth{1.003750pt}%
\definecolor{currentstroke}{rgb}{0.263663,0.237631,0.518762}%
\pgfsetstrokecolor{currentstroke}%
\pgfsetdash{}{0pt}%
\pgfsys@defobject{currentmarker}{\pgfqpoint{0.605784in}{0.687855in}}{\pgfqpoint{4.669272in}{2.425059in}}{%
\pgfpathmoveto{\pgfqpoint{0.625696in}{0.754205in}}%
\pgfpathlineto{\pgfqpoint{0.605784in}{0.761947in}}%
\pgfpathlineto{\pgfqpoint{0.605784in}{0.774833in}}%
\pgfpathlineto{\pgfqpoint{0.605784in}{0.795460in}}%
\pgfpathlineto{\pgfqpoint{0.605784in}{0.816088in}}%
\pgfpathlineto{\pgfqpoint{0.605784in}{0.836716in}}%
\pgfpathlineto{\pgfqpoint{0.605784in}{0.857344in}}%
\pgfpathlineto{\pgfqpoint{0.605784in}{0.877972in}}%
\pgfpathlineto{\pgfqpoint{0.605784in}{0.879256in}}%
\pgfpathlineto{\pgfqpoint{0.609374in}{0.877972in}}%
\pgfpathlineto{\pgfqpoint{0.646829in}{0.864588in}}%
\pgfpathlineto{\pgfqpoint{0.687875in}{0.862243in}}%
\pgfpathlineto{\pgfqpoint{0.728920in}{0.871256in}}%
\pgfpathlineto{\pgfqpoint{0.746179in}{0.877972in}}%
\pgfpathlineto{\pgfqpoint{0.769965in}{0.887849in}}%
\pgfpathlineto{\pgfqpoint{0.793531in}{0.898600in}}%
\pgfpathlineto{\pgfqpoint{0.811011in}{0.906903in}}%
\pgfpathlineto{\pgfqpoint{0.839916in}{0.919227in}}%
\pgfpathlineto{\pgfqpoint{0.852056in}{0.924459in}}%
\pgfpathlineto{\pgfqpoint{0.892450in}{0.939855in}}%
\pgfpathlineto{\pgfqpoint{0.893101in}{0.940099in}}%
\pgfpathlineto{\pgfqpoint{0.934147in}{0.957160in}}%
\pgfpathlineto{\pgfqpoint{0.940014in}{0.960483in}}%
\pgfpathlineto{\pgfqpoint{0.975192in}{0.979300in}}%
\pgfpathlineto{\pgfqpoint{0.977570in}{0.981111in}}%
\pgfpathlineto{\pgfqpoint{1.005628in}{1.001739in}}%
\pgfpathlineto{\pgfqpoint{1.016237in}{1.009085in}}%
\pgfpathlineto{\pgfqpoint{1.030455in}{1.022367in}}%
\pgfpathlineto{\pgfqpoint{1.053811in}{1.042994in}}%
\pgfpathlineto{\pgfqpoint{1.057283in}{1.045980in}}%
\pgfpathlineto{\pgfqpoint{1.074269in}{1.063622in}}%
\pgfpathlineto{\pgfqpoint{1.095190in}{1.084250in}}%
\pgfpathlineto{\pgfqpoint{1.098328in}{1.087297in}}%
\pgfpathlineto{\pgfqpoint{1.115553in}{1.104878in}}%
\pgfpathlineto{\pgfqpoint{1.136627in}{1.125506in}}%
\pgfpathlineto{\pgfqpoint{1.139373in}{1.128173in}}%
\pgfpathlineto{\pgfqpoint{1.160182in}{1.146134in}}%
\pgfpathlineto{\pgfqpoint{1.180419in}{1.163066in}}%
\pgfpathlineto{\pgfqpoint{1.186694in}{1.166761in}}%
\pgfpathlineto{\pgfqpoint{1.221464in}{1.187056in}}%
\pgfpathlineto{\pgfqpoint{1.222737in}{1.187389in}}%
\pgfpathlineto{\pgfqpoint{1.262509in}{1.197964in}}%
\pgfpathlineto{\pgfqpoint{1.303555in}{1.196368in}}%
\pgfpathlineto{\pgfqpoint{1.339958in}{1.187389in}}%
\pgfpathlineto{\pgfqpoint{1.344600in}{1.186284in}}%
\pgfpathlineto{\pgfqpoint{1.385645in}{1.173883in}}%
\pgfpathlineto{\pgfqpoint{1.419006in}{1.166761in}}%
\pgfpathlineto{\pgfqpoint{1.426691in}{1.165130in}}%
\pgfpathlineto{\pgfqpoint{1.467736in}{1.165685in}}%
\pgfpathlineto{\pgfqpoint{1.471335in}{1.166761in}}%
\pgfpathlineto{\pgfqpoint{1.508781in}{1.179169in}}%
\pgfpathlineto{\pgfqpoint{1.522228in}{1.187389in}}%
\pgfpathlineto{\pgfqpoint{1.549827in}{1.206080in}}%
\pgfpathlineto{\pgfqpoint{1.552193in}{1.208017in}}%
\pgfpathlineto{\pgfqpoint{1.575428in}{1.228645in}}%
\pgfpathlineto{\pgfqpoint{1.590872in}{1.243755in}}%
\pgfpathlineto{\pgfqpoint{1.597098in}{1.249273in}}%
\pgfpathlineto{\pgfqpoint{1.618674in}{1.269900in}}%
\pgfpathlineto{\pgfqpoint{1.631917in}{1.283651in}}%
\pgfpathlineto{\pgfqpoint{1.642167in}{1.290528in}}%
\pgfpathlineto{\pgfqpoint{1.671459in}{1.311156in}}%
\pgfpathlineto{\pgfqpoint{1.672963in}{1.312282in}}%
\pgfpathlineto{\pgfqpoint{1.714008in}{1.320561in}}%
\pgfpathlineto{\pgfqpoint{1.755053in}{1.313040in}}%
\pgfpathlineto{\pgfqpoint{1.761656in}{1.311156in}}%
\pgfpathlineto{\pgfqpoint{1.796099in}{1.303121in}}%
\pgfpathlineto{\pgfqpoint{1.837144in}{1.299516in}}%
\pgfpathlineto{\pgfqpoint{1.878189in}{1.305823in}}%
\pgfpathlineto{\pgfqpoint{1.891451in}{1.311156in}}%
\pgfpathlineto{\pgfqpoint{1.919235in}{1.321639in}}%
\pgfpathlineto{\pgfqpoint{1.937451in}{1.331784in}}%
\pgfpathlineto{\pgfqpoint{1.960280in}{1.343831in}}%
\pgfpathlineto{\pgfqpoint{1.974453in}{1.352412in}}%
\pgfpathlineto{\pgfqpoint{2.001325in}{1.368050in}}%
\pgfpathlineto{\pgfqpoint{2.010549in}{1.373040in}}%
\pgfpathlineto{\pgfqpoint{2.042371in}{1.389878in}}%
\pgfpathlineto{\pgfqpoint{2.051985in}{1.393667in}}%
\pgfpathlineto{\pgfqpoint{2.083416in}{1.406025in}}%
\pgfpathlineto{\pgfqpoint{2.120897in}{1.414295in}}%
\pgfpathlineto{\pgfqpoint{2.124461in}{1.415096in}}%
\pgfpathlineto{\pgfqpoint{2.165507in}{1.418702in}}%
\pgfpathlineto{\pgfqpoint{2.206552in}{1.420299in}}%
\pgfpathlineto{\pgfqpoint{2.247597in}{1.425076in}}%
\pgfpathlineto{\pgfqpoint{2.278144in}{1.434923in}}%
\pgfpathlineto{\pgfqpoint{2.288643in}{1.438623in}}%
\pgfpathlineto{\pgfqpoint{2.314615in}{1.455551in}}%
\pgfpathlineto{\pgfqpoint{2.329688in}{1.466426in}}%
\pgfpathlineto{\pgfqpoint{2.339279in}{1.476179in}}%
\pgfpathlineto{\pgfqpoint{2.357796in}{1.496807in}}%
\pgfpathlineto{\pgfqpoint{2.370733in}{1.512779in}}%
\pgfpathlineto{\pgfqpoint{2.374045in}{1.517434in}}%
\pgfpathlineto{\pgfqpoint{2.387520in}{1.538062in}}%
\pgfpathlineto{\pgfqpoint{2.399680in}{1.558690in}}%
\pgfpathlineto{\pgfqpoint{2.410584in}{1.579318in}}%
\pgfpathlineto{\pgfqpoint{2.411779in}{1.581825in}}%
\pgfpathlineto{\pgfqpoint{2.421166in}{1.599946in}}%
\pgfpathlineto{\pgfqpoint{2.430730in}{1.620573in}}%
\pgfpathlineto{\pgfqpoint{2.439187in}{1.641201in}}%
\pgfpathlineto{\pgfqpoint{2.446563in}{1.661829in}}%
\pgfpathlineto{\pgfqpoint{2.452824in}{1.682255in}}%
\pgfpathlineto{\pgfqpoint{2.452913in}{1.682457in}}%
\pgfpathlineto{\pgfqpoint{2.460714in}{1.703085in}}%
\pgfpathlineto{\pgfqpoint{2.467187in}{1.723713in}}%
\pgfpathlineto{\pgfqpoint{2.472320in}{1.744340in}}%
\pgfpathlineto{\pgfqpoint{2.476099in}{1.764968in}}%
\pgfpathlineto{\pgfqpoint{2.478510in}{1.785596in}}%
\pgfpathlineto{\pgfqpoint{2.479540in}{1.806224in}}%
\pgfpathlineto{\pgfqpoint{2.479175in}{1.826852in}}%
\pgfpathlineto{\pgfqpoint{2.477400in}{1.847480in}}%
\pgfpathlineto{\pgfqpoint{2.474202in}{1.868107in}}%
\pgfpathlineto{\pgfqpoint{2.469566in}{1.888735in}}%
\pgfpathlineto{\pgfqpoint{2.463476in}{1.909363in}}%
\pgfpathlineto{\pgfqpoint{2.455919in}{1.929991in}}%
\pgfpathlineto{\pgfqpoint{2.452824in}{1.937070in}}%
\pgfpathlineto{\pgfqpoint{2.449280in}{1.950619in}}%
\pgfpathlineto{\pgfqpoint{2.443100in}{1.971247in}}%
\pgfpathlineto{\pgfqpoint{2.436166in}{1.991874in}}%
\pgfpathlineto{\pgfqpoint{2.428493in}{2.012502in}}%
\pgfpathlineto{\pgfqpoint{2.420096in}{2.033130in}}%
\pgfpathlineto{\pgfqpoint{2.411779in}{2.051960in}}%
\pgfpathlineto{\pgfqpoint{2.411143in}{2.053758in}}%
\pgfpathlineto{\pgfqpoint{2.403285in}{2.074386in}}%
\pgfpathlineto{\pgfqpoint{2.394949in}{2.095013in}}%
\pgfpathlineto{\pgfqpoint{2.386150in}{2.115641in}}%
\pgfpathlineto{\pgfqpoint{2.376904in}{2.136269in}}%
\pgfpathlineto{\pgfqpoint{2.370733in}{2.149368in}}%
\pgfpathlineto{\pgfqpoint{2.367354in}{2.156897in}}%
\pgfpathlineto{\pgfqpoint{2.357667in}{2.177525in}}%
\pgfpathlineto{\pgfqpoint{2.347644in}{2.198153in}}%
\pgfpathlineto{\pgfqpoint{2.337299in}{2.218780in}}%
\pgfpathlineto{\pgfqpoint{2.329688in}{2.233460in}}%
\pgfpathlineto{\pgfqpoint{2.326315in}{2.239408in}}%
\pgfpathlineto{\pgfqpoint{2.314217in}{2.260036in}}%
\pgfpathlineto{\pgfqpoint{2.301864in}{2.280664in}}%
\pgfpathlineto{\pgfqpoint{2.289266in}{2.301292in}}%
\pgfpathlineto{\pgfqpoint{2.288643in}{2.302280in}}%
\pgfpathlineto{\pgfqpoint{2.273200in}{2.321920in}}%
\pgfpathlineto{\pgfqpoint{2.256766in}{2.342547in}}%
\pgfpathlineto{\pgfqpoint{2.247597in}{2.353838in}}%
\pgfpathlineto{\pgfqpoint{2.236474in}{2.363175in}}%
\pgfpathlineto{\pgfqpoint{2.211486in}{2.383803in}}%
\pgfpathlineto{\pgfqpoint{2.206552in}{2.387793in}}%
\pgfpathlineto{\pgfqpoint{2.165754in}{2.404431in}}%
\pgfpathlineto{\pgfqpoint{2.165507in}{2.404529in}}%
\pgfpathlineto{\pgfqpoint{2.124461in}{2.405207in}}%
\pgfpathlineto{\pgfqpoint{2.122143in}{2.404431in}}%
\pgfpathlineto{\pgfqpoint{2.083416in}{2.391229in}}%
\pgfpathlineto{\pgfqpoint{2.072334in}{2.383803in}}%
\pgfpathlineto{\pgfqpoint{2.042371in}{2.363773in}}%
\pgfpathlineto{\pgfqpoint{2.041772in}{2.363175in}}%
\pgfpathlineto{\pgfqpoint{2.021309in}{2.342547in}}%
\pgfpathlineto{\pgfqpoint{2.001325in}{2.322830in}}%
\pgfpathlineto{\pgfqpoint{2.000642in}{2.321920in}}%
\pgfpathlineto{\pgfqpoint{1.985325in}{2.301292in}}%
\pgfpathlineto{\pgfqpoint{1.969712in}{2.280664in}}%
\pgfpathlineto{\pgfqpoint{1.960280in}{2.268293in}}%
\pgfpathlineto{\pgfqpoint{1.955315in}{2.260036in}}%
\pgfpathlineto{\pgfqpoint{1.942957in}{2.239408in}}%
\pgfpathlineto{\pgfqpoint{1.930416in}{2.218780in}}%
\pgfpathlineto{\pgfqpoint{1.919235in}{2.200610in}}%
\pgfpathlineto{\pgfqpoint{1.917965in}{2.198153in}}%
\pgfpathlineto{\pgfqpoint{1.907474in}{2.177525in}}%
\pgfpathlineto{\pgfqpoint{1.896889in}{2.156897in}}%
\pgfpathlineto{\pgfqpoint{1.886199in}{2.136269in}}%
\pgfpathlineto{\pgfqpoint{1.878189in}{2.120855in}}%
\pgfpathlineto{\pgfqpoint{1.875744in}{2.115641in}}%
\pgfpathlineto{\pgfqpoint{1.866228in}{2.095013in}}%
\pgfpathlineto{\pgfqpoint{1.856679in}{2.074386in}}%
\pgfpathlineto{\pgfqpoint{1.847093in}{2.053758in}}%
\pgfpathlineto{\pgfqpoint{1.837467in}{2.033130in}}%
\pgfpathlineto{\pgfqpoint{1.837144in}{2.032419in}}%
\pgfpathlineto{\pgfqpoint{1.828183in}{2.012502in}}%
\pgfpathlineto{\pgfqpoint{1.818930in}{1.991874in}}%
\pgfpathlineto{\pgfqpoint{1.809699in}{1.971247in}}%
\pgfpathlineto{\pgfqpoint{1.800490in}{1.950619in}}%
\pgfpathlineto{\pgfqpoint{1.796099in}{1.940594in}}%
\pgfpathlineto{\pgfqpoint{1.790874in}{1.929991in}}%
\pgfpathlineto{\pgfqpoint{1.780920in}{1.909363in}}%
\pgfpathlineto{\pgfqpoint{1.771043in}{1.888735in}}%
\pgfpathlineto{\pgfqpoint{1.761249in}{1.868107in}}%
\pgfpathlineto{\pgfqpoint{1.755053in}{1.854777in}}%
\pgfpathlineto{\pgfqpoint{1.750378in}{1.847480in}}%
\pgfpathlineto{\pgfqpoint{1.737550in}{1.826852in}}%
\pgfpathlineto{\pgfqpoint{1.724834in}{1.806224in}}%
\pgfpathlineto{\pgfqpoint{1.714008in}{1.788395in}}%
\pgfpathlineto{\pgfqpoint{1.710271in}{1.785596in}}%
\pgfpathlineto{\pgfqpoint{1.683402in}{1.764968in}}%
\pgfpathlineto{\pgfqpoint{1.672963in}{1.756857in}}%
\pgfpathlineto{\pgfqpoint{1.640801in}{1.764968in}}%
\pgfpathlineto{\pgfqpoint{1.631917in}{1.766939in}}%
\pgfpathlineto{\pgfqpoint{1.614036in}{1.785596in}}%
\pgfpathlineto{\pgfqpoint{1.592674in}{1.806224in}}%
\pgfpathlineto{\pgfqpoint{1.590872in}{1.807843in}}%
\pgfpathlineto{\pgfqpoint{1.577522in}{1.826852in}}%
\pgfpathlineto{\pgfqpoint{1.562326in}{1.847480in}}%
\pgfpathlineto{\pgfqpoint{1.549827in}{1.863687in}}%
\pgfpathlineto{\pgfqpoint{1.546907in}{1.868107in}}%
\pgfpathlineto{\pgfqpoint{1.532691in}{1.888735in}}%
\pgfpathlineto{\pgfqpoint{1.518077in}{1.909363in}}%
\pgfpathlineto{\pgfqpoint{1.508781in}{1.922075in}}%
\pgfpathlineto{\pgfqpoint{1.502793in}{1.929991in}}%
\pgfpathlineto{\pgfqpoint{1.486760in}{1.950619in}}%
\pgfpathlineto{\pgfqpoint{1.470510in}{1.971247in}}%
\pgfpathlineto{\pgfqpoint{1.467736in}{1.974669in}}%
\pgfpathlineto{\pgfqpoint{1.450751in}{1.991874in}}%
\pgfpathlineto{\pgfqpoint{1.430287in}{2.012502in}}%
\pgfpathlineto{\pgfqpoint{1.426691in}{2.016055in}}%
\pgfpathlineto{\pgfqpoint{1.401286in}{2.033130in}}%
\pgfpathlineto{\pgfqpoint{1.385645in}{2.043582in}}%
\pgfpathlineto{\pgfqpoint{1.354726in}{2.053758in}}%
\pgfpathlineto{\pgfqpoint{1.344600in}{2.057033in}}%
\pgfpathlineto{\pgfqpoint{1.303555in}{2.057676in}}%
\pgfpathlineto{\pgfqpoint{1.286575in}{2.053758in}}%
\pgfpathlineto{\pgfqpoint{1.262509in}{2.047985in}}%
\pgfpathlineto{\pgfqpoint{1.228032in}{2.033130in}}%
\pgfpathlineto{\pgfqpoint{1.221464in}{2.030257in}}%
\pgfpathlineto{\pgfqpoint{1.192331in}{2.012502in}}%
\pgfpathlineto{\pgfqpoint{1.180419in}{2.005309in}}%
\pgfpathlineto{\pgfqpoint{1.163722in}{1.991874in}}%
\pgfpathlineto{\pgfqpoint{1.139373in}{1.972877in}}%
\pgfpathlineto{\pgfqpoint{1.137790in}{1.971247in}}%
\pgfpathlineto{\pgfqpoint{1.117713in}{1.950619in}}%
\pgfpathlineto{\pgfqpoint{1.098328in}{1.931575in}}%
\pgfpathlineto{\pgfqpoint{1.097076in}{1.929991in}}%
\pgfpathlineto{\pgfqpoint{1.080807in}{1.909363in}}%
\pgfpathlineto{\pgfqpoint{1.063883in}{1.888735in}}%
\pgfpathlineto{\pgfqpoint{1.057283in}{1.880801in}}%
\pgfpathlineto{\pgfqpoint{1.048670in}{1.868107in}}%
\pgfpathlineto{\pgfqpoint{1.034470in}{1.847480in}}%
\pgfpathlineto{\pgfqpoint{1.019810in}{1.826852in}}%
\pgfpathlineto{\pgfqpoint{1.016237in}{1.821833in}}%
\pgfpathlineto{\pgfqpoint{1.006431in}{1.806224in}}%
\pgfpathlineto{\pgfqpoint{0.993293in}{1.785596in}}%
\pgfpathlineto{\pgfqpoint{0.979828in}{1.764968in}}%
\pgfpathlineto{\pgfqpoint{0.975192in}{1.757866in}}%
\pgfpathlineto{\pgfqpoint{0.966518in}{1.744340in}}%
\pgfpathlineto{\pgfqpoint{0.953213in}{1.723713in}}%
\pgfpathlineto{\pgfqpoint{0.939648in}{1.703085in}}%
\pgfpathlineto{\pgfqpoint{0.934147in}{1.694700in}}%
\pgfpathlineto{\pgfqpoint{0.925028in}{1.682457in}}%
\pgfpathlineto{\pgfqpoint{0.909630in}{1.661829in}}%
\pgfpathlineto{\pgfqpoint{0.893949in}{1.641201in}}%
\pgfpathlineto{\pgfqpoint{0.893101in}{1.640066in}}%
\pgfpathlineto{\pgfqpoint{0.872444in}{1.620573in}}%
\pgfpathlineto{\pgfqpoint{0.852056in}{1.601692in}}%
\pgfpathlineto{\pgfqpoint{0.847684in}{1.599946in}}%
\pgfpathlineto{\pgfqpoint{0.811011in}{1.585142in}}%
\pgfpathlineto{\pgfqpoint{0.769965in}{1.590758in}}%
\pgfpathlineto{\pgfqpoint{0.753713in}{1.599946in}}%
\pgfpathlineto{\pgfqpoint{0.728920in}{1.613290in}}%
\pgfpathlineto{\pgfqpoint{0.719610in}{1.620573in}}%
\pgfpathlineto{\pgfqpoint{0.692473in}{1.641201in}}%
\pgfpathlineto{\pgfqpoint{0.687875in}{1.644591in}}%
\pgfpathlineto{\pgfqpoint{0.665631in}{1.661829in}}%
\pgfpathlineto{\pgfqpoint{0.646829in}{1.676269in}}%
\pgfpathlineto{\pgfqpoint{0.636891in}{1.682457in}}%
\pgfpathlineto{\pgfqpoint{0.605784in}{1.701800in}}%
\pgfpathlineto{\pgfqpoint{0.605784in}{1.703085in}}%
\pgfpathlineto{\pgfqpoint{0.605784in}{1.723713in}}%
\pgfpathlineto{\pgfqpoint{0.605784in}{1.744340in}}%
\pgfpathlineto{\pgfqpoint{0.605784in}{1.764968in}}%
\pgfpathlineto{\pgfqpoint{0.605784in}{1.785596in}}%
\pgfpathlineto{\pgfqpoint{0.605784in}{1.806224in}}%
\pgfpathlineto{\pgfqpoint{0.605784in}{1.819109in}}%
\pgfpathlineto{\pgfqpoint{0.627595in}{1.806224in}}%
\pgfpathlineto{\pgfqpoint{0.646829in}{1.794850in}}%
\pgfpathlineto{\pgfqpoint{0.660343in}{1.785596in}}%
\pgfpathlineto{\pgfqpoint{0.687875in}{1.766612in}}%
\pgfpathlineto{\pgfqpoint{0.690498in}{1.764968in}}%
\pgfpathlineto{\pgfqpoint{0.722710in}{1.744340in}}%
\pgfpathlineto{\pgfqpoint{0.728920in}{1.740281in}}%
\pgfpathlineto{\pgfqpoint{0.768276in}{1.723713in}}%
\pgfpathlineto{\pgfqpoint{0.769965in}{1.722975in}}%
\pgfpathlineto{\pgfqpoint{0.811011in}{1.720744in}}%
\pgfpathlineto{\pgfqpoint{0.818311in}{1.723713in}}%
\pgfpathlineto{\pgfqpoint{0.852056in}{1.737330in}}%
\pgfpathlineto{\pgfqpoint{0.860246in}{1.744340in}}%
\pgfpathlineto{\pgfqpoint{0.884313in}{1.764968in}}%
\pgfpathlineto{\pgfqpoint{0.893101in}{1.772517in}}%
\pgfpathlineto{\pgfqpoint{0.903994in}{1.785596in}}%
\pgfpathlineto{\pgfqpoint{0.921038in}{1.806224in}}%
\pgfpathlineto{\pgfqpoint{0.934147in}{1.822272in}}%
\pgfpathlineto{\pgfqpoint{0.937413in}{1.826852in}}%
\pgfpathlineto{\pgfqpoint{0.952185in}{1.847480in}}%
\pgfpathlineto{\pgfqpoint{0.966672in}{1.868107in}}%
\pgfpathlineto{\pgfqpoint{0.975192in}{1.880317in}}%
\pgfpathlineto{\pgfqpoint{0.981050in}{1.888735in}}%
\pgfpathlineto{\pgfqpoint{0.995339in}{1.909363in}}%
\pgfpathlineto{\pgfqpoint{1.009311in}{1.929991in}}%
\pgfpathlineto{\pgfqpoint{1.016237in}{1.940280in}}%
\pgfpathlineto{\pgfqpoint{1.023901in}{1.950619in}}%
\pgfpathlineto{\pgfqpoint{1.039017in}{1.971247in}}%
\pgfpathlineto{\pgfqpoint{1.053731in}{1.991874in}}%
\pgfpathlineto{\pgfqpoint{1.057283in}{1.996859in}}%
\pgfpathlineto{\pgfqpoint{1.070502in}{2.012502in}}%
\pgfpathlineto{\pgfqpoint{1.087505in}{2.033130in}}%
\pgfpathlineto{\pgfqpoint{1.098328in}{2.046520in}}%
\pgfpathlineto{\pgfqpoint{1.105669in}{2.053758in}}%
\pgfpathlineto{\pgfqpoint{1.126300in}{2.074386in}}%
\pgfpathlineto{\pgfqpoint{1.139373in}{2.087778in}}%
\pgfpathlineto{\pgfqpoint{1.148554in}{2.095013in}}%
\pgfpathlineto{\pgfqpoint{1.174334in}{2.115641in}}%
\pgfpathlineto{\pgfqpoint{1.180419in}{2.120560in}}%
\pgfpathlineto{\pgfqpoint{1.206509in}{2.136269in}}%
\pgfpathlineto{\pgfqpoint{1.221464in}{2.145338in}}%
\pgfpathlineto{\pgfqpoint{1.249193in}{2.156897in}}%
\pgfpathlineto{\pgfqpoint{1.262509in}{2.162383in}}%
\pgfpathlineto{\pgfqpoint{1.303555in}{2.170859in}}%
\pgfpathlineto{\pgfqpoint{1.344600in}{2.169233in}}%
\pgfpathlineto{\pgfqpoint{1.382926in}{2.156897in}}%
\pgfpathlineto{\pgfqpoint{1.385645in}{2.156009in}}%
\pgfpathlineto{\pgfqpoint{1.417240in}{2.136269in}}%
\pgfpathlineto{\pgfqpoint{1.426691in}{2.130337in}}%
\pgfpathlineto{\pgfqpoint{1.443141in}{2.115641in}}%
\pgfpathlineto{\pgfqpoint{1.466190in}{2.095013in}}%
\pgfpathlineto{\pgfqpoint{1.467736in}{2.093607in}}%
\pgfpathlineto{\pgfqpoint{1.485500in}{2.074386in}}%
\pgfpathlineto{\pgfqpoint{1.504448in}{2.053758in}}%
\pgfpathlineto{\pgfqpoint{1.508781in}{2.048949in}}%
\pgfpathlineto{\pgfqpoint{1.522902in}{2.033130in}}%
\pgfpathlineto{\pgfqpoint{1.541006in}{2.012502in}}%
\pgfpathlineto{\pgfqpoint{1.549827in}{2.002229in}}%
\pgfpathlineto{\pgfqpoint{1.560558in}{1.991874in}}%
\pgfpathlineto{\pgfqpoint{1.581314in}{1.971247in}}%
\pgfpathlineto{\pgfqpoint{1.590872in}{1.961444in}}%
\pgfpathlineto{\pgfqpoint{1.608787in}{1.950619in}}%
\pgfpathlineto{\pgfqpoint{1.631917in}{1.935964in}}%
\pgfpathlineto{\pgfqpoint{1.672963in}{1.934322in}}%
\pgfpathlineto{\pgfqpoint{1.698357in}{1.950619in}}%
\pgfpathlineto{\pgfqpoint{1.714008in}{1.960641in}}%
\pgfpathlineto{\pgfqpoint{1.722355in}{1.971247in}}%
\pgfpathlineto{\pgfqpoint{1.738704in}{1.991874in}}%
\pgfpathlineto{\pgfqpoint{1.754943in}{2.012502in}}%
\pgfpathlineto{\pgfqpoint{1.755053in}{2.012640in}}%
\pgfpathlineto{\pgfqpoint{1.767181in}{2.033130in}}%
\pgfpathlineto{\pgfqpoint{1.779340in}{2.053758in}}%
\pgfpathlineto{\pgfqpoint{1.791450in}{2.074386in}}%
\pgfpathlineto{\pgfqpoint{1.796099in}{2.082214in}}%
\pgfpathlineto{\pgfqpoint{1.802816in}{2.095013in}}%
\pgfpathlineto{\pgfqpoint{1.813684in}{2.115641in}}%
\pgfpathlineto{\pgfqpoint{1.824492in}{2.136269in}}%
\pgfpathlineto{\pgfqpoint{1.835244in}{2.156897in}}%
\pgfpathlineto{\pgfqpoint{1.837144in}{2.160490in}}%
\pgfpathlineto{\pgfqpoint{1.846121in}{2.177525in}}%
\pgfpathlineto{\pgfqpoint{1.856943in}{2.198153in}}%
\pgfpathlineto{\pgfqpoint{1.867675in}{2.218780in}}%
\pgfpathlineto{\pgfqpoint{1.878189in}{2.239145in}}%
\pgfpathlineto{\pgfqpoint{1.878338in}{2.239408in}}%
\pgfpathlineto{\pgfqpoint{1.890129in}{2.260036in}}%
\pgfpathlineto{\pgfqpoint{1.901775in}{2.280664in}}%
\pgfpathlineto{\pgfqpoint{1.913285in}{2.301292in}}%
\pgfpathlineto{\pgfqpoint{1.919235in}{2.311948in}}%
\pgfpathlineto{\pgfqpoint{1.925770in}{2.321920in}}%
\pgfpathlineto{\pgfqpoint{1.939256in}{2.342547in}}%
\pgfpathlineto{\pgfqpoint{1.952535in}{2.363175in}}%
\pgfpathlineto{\pgfqpoint{1.960280in}{2.375258in}}%
\pgfpathlineto{\pgfqpoint{1.967130in}{2.383803in}}%
\pgfpathlineto{\pgfqpoint{1.983616in}{2.404431in}}%
\pgfpathlineto{\pgfqpoint{1.999798in}{2.425059in}}%
\pgfpathlineto{\pgfqpoint{2.001325in}{2.425059in}}%
\pgfpathlineto{\pgfqpoint{2.042371in}{2.425059in}}%
\pgfpathlineto{\pgfqpoint{2.083416in}{2.425059in}}%
\pgfpathlineto{\pgfqpoint{2.124461in}{2.425059in}}%
\pgfpathlineto{\pgfqpoint{2.165507in}{2.425059in}}%
\pgfpathlineto{\pgfqpoint{2.206552in}{2.425059in}}%
\pgfpathlineto{\pgfqpoint{2.247597in}{2.425059in}}%
\pgfpathlineto{\pgfqpoint{2.279797in}{2.425059in}}%
\pgfpathlineto{\pgfqpoint{2.288643in}{2.415163in}}%
\pgfpathlineto{\pgfqpoint{2.296422in}{2.404431in}}%
\pgfpathlineto{\pgfqpoint{2.311116in}{2.383803in}}%
\pgfpathlineto{\pgfqpoint{2.325642in}{2.363175in}}%
\pgfpathlineto{\pgfqpoint{2.329688in}{2.357300in}}%
\pgfpathlineto{\pgfqpoint{2.339102in}{2.342547in}}%
\pgfpathlineto{\pgfqpoint{2.351993in}{2.321920in}}%
\pgfpathlineto{\pgfqpoint{2.364648in}{2.301292in}}%
\pgfpathlineto{\pgfqpoint{2.370733in}{2.291114in}}%
\pgfpathlineto{\pgfqpoint{2.377410in}{2.280664in}}%
\pgfpathlineto{\pgfqpoint{2.390193in}{2.260036in}}%
\pgfpathlineto{\pgfqpoint{2.402631in}{2.239408in}}%
\pgfpathlineto{\pgfqpoint{2.411779in}{2.223754in}}%
\pgfpathlineto{\pgfqpoint{2.415542in}{2.218780in}}%
\pgfpathlineto{\pgfqpoint{2.430491in}{2.198153in}}%
\pgfpathlineto{\pgfqpoint{2.444846in}{2.177525in}}%
\pgfpathlineto{\pgfqpoint{2.452824in}{2.165521in}}%
\pgfpathlineto{\pgfqpoint{2.463557in}{2.156897in}}%
\pgfpathlineto{\pgfqpoint{2.487625in}{2.136269in}}%
\pgfpathlineto{\pgfqpoint{2.493869in}{2.130540in}}%
\pgfpathlineto{\pgfqpoint{2.495249in}{2.115641in}}%
\pgfpathlineto{\pgfqpoint{2.497239in}{2.095013in}}%
\pgfpathlineto{\pgfqpoint{2.499364in}{2.074386in}}%
\pgfpathlineto{\pgfqpoint{2.501653in}{2.053758in}}%
\pgfpathlineto{\pgfqpoint{2.504146in}{2.033130in}}%
\pgfpathlineto{\pgfqpoint{2.506897in}{2.012502in}}%
\pgfpathlineto{\pgfqpoint{2.509982in}{1.991874in}}%
\pgfpathlineto{\pgfqpoint{2.513506in}{1.971247in}}%
\pgfpathlineto{\pgfqpoint{2.517628in}{1.950619in}}%
\pgfpathlineto{\pgfqpoint{2.522591in}{1.929991in}}%
\pgfpathlineto{\pgfqpoint{2.528792in}{1.909363in}}%
\pgfpathlineto{\pgfqpoint{2.534915in}{1.893201in}}%
\pgfpathlineto{\pgfqpoint{2.575960in}{1.893201in}}%
\pgfpathlineto{\pgfqpoint{2.617005in}{1.893201in}}%
\pgfpathlineto{\pgfqpoint{2.658051in}{1.893201in}}%
\pgfpathlineto{\pgfqpoint{2.699096in}{1.893201in}}%
\pgfpathlineto{\pgfqpoint{2.740141in}{1.893201in}}%
\pgfpathlineto{\pgfqpoint{2.781187in}{1.893201in}}%
\pgfpathlineto{\pgfqpoint{2.822232in}{1.893201in}}%
\pgfpathlineto{\pgfqpoint{2.863277in}{1.893201in}}%
\pgfpathlineto{\pgfqpoint{2.904323in}{1.893201in}}%
\pgfpathlineto{\pgfqpoint{2.945368in}{1.893201in}}%
\pgfpathlineto{\pgfqpoint{2.986413in}{1.893201in}}%
\pgfpathlineto{\pgfqpoint{3.027459in}{1.893201in}}%
\pgfpathlineto{\pgfqpoint{3.068504in}{1.893201in}}%
\pgfpathlineto{\pgfqpoint{3.109549in}{1.893201in}}%
\pgfpathlineto{\pgfqpoint{3.150595in}{1.893201in}}%
\pgfpathlineto{\pgfqpoint{3.191640in}{1.893201in}}%
\pgfpathlineto{\pgfqpoint{3.232685in}{1.893201in}}%
\pgfpathlineto{\pgfqpoint{3.273731in}{1.893201in}}%
\pgfpathlineto{\pgfqpoint{3.314776in}{1.893201in}}%
\pgfpathlineto{\pgfqpoint{3.355821in}{1.893201in}}%
\pgfpathlineto{\pgfqpoint{3.396867in}{1.893201in}}%
\pgfpathlineto{\pgfqpoint{3.437912in}{1.893201in}}%
\pgfpathlineto{\pgfqpoint{3.478957in}{1.893201in}}%
\pgfpathlineto{\pgfqpoint{3.520003in}{1.893201in}}%
\pgfpathlineto{\pgfqpoint{3.561048in}{1.893201in}}%
\pgfpathlineto{\pgfqpoint{3.602093in}{1.893201in}}%
\pgfpathlineto{\pgfqpoint{3.643139in}{1.893201in}}%
\pgfpathlineto{\pgfqpoint{3.643842in}{1.888735in}}%
\pgfpathlineto{\pgfqpoint{3.647095in}{1.868107in}}%
\pgfpathlineto{\pgfqpoint{3.650445in}{1.847480in}}%
\pgfpathlineto{\pgfqpoint{3.653903in}{1.826852in}}%
\pgfpathlineto{\pgfqpoint{3.657481in}{1.806224in}}%
\pgfpathlineto{\pgfqpoint{3.661196in}{1.785596in}}%
\pgfpathlineto{\pgfqpoint{3.665065in}{1.764968in}}%
\pgfpathlineto{\pgfqpoint{3.669108in}{1.744340in}}%
\pgfpathlineto{\pgfqpoint{3.673352in}{1.723713in}}%
\pgfpathlineto{\pgfqpoint{3.677825in}{1.703085in}}%
\pgfpathlineto{\pgfqpoint{3.682564in}{1.682457in}}%
\pgfpathlineto{\pgfqpoint{3.684184in}{1.675593in}}%
\pgfpathlineto{\pgfqpoint{3.725229in}{1.675593in}}%
\pgfpathlineto{\pgfqpoint{3.766275in}{1.675593in}}%
\pgfpathlineto{\pgfqpoint{3.807320in}{1.675593in}}%
\pgfpathlineto{\pgfqpoint{3.848365in}{1.675593in}}%
\pgfpathlineto{\pgfqpoint{3.889411in}{1.675593in}}%
\pgfpathlineto{\pgfqpoint{3.930456in}{1.675593in}}%
\pgfpathlineto{\pgfqpoint{3.971501in}{1.675593in}}%
\pgfpathlineto{\pgfqpoint{4.012547in}{1.675593in}}%
\pgfpathlineto{\pgfqpoint{4.053592in}{1.675593in}}%
\pgfpathlineto{\pgfqpoint{4.094637in}{1.675593in}}%
\pgfpathlineto{\pgfqpoint{4.135683in}{1.675593in}}%
\pgfpathlineto{\pgfqpoint{4.176728in}{1.675593in}}%
\pgfpathlineto{\pgfqpoint{4.217773in}{1.675593in}}%
\pgfpathlineto{\pgfqpoint{4.258819in}{1.675593in}}%
\pgfpathlineto{\pgfqpoint{4.299864in}{1.675593in}}%
\pgfpathlineto{\pgfqpoint{4.340909in}{1.675593in}}%
\pgfpathlineto{\pgfqpoint{4.381955in}{1.675593in}}%
\pgfpathlineto{\pgfqpoint{4.423000in}{1.675593in}}%
\pgfpathlineto{\pgfqpoint{4.464045in}{1.675593in}}%
\pgfpathlineto{\pgfqpoint{4.505091in}{1.675593in}}%
\pgfpathlineto{\pgfqpoint{4.546136in}{1.675593in}}%
\pgfpathlineto{\pgfqpoint{4.587181in}{1.675593in}}%
\pgfpathlineto{\pgfqpoint{4.628227in}{1.675593in}}%
\pgfpathlineto{\pgfqpoint{4.669272in}{1.675593in}}%
\pgfpathlineto{\pgfqpoint{4.669272in}{1.661829in}}%
\pgfpathlineto{\pgfqpoint{4.669272in}{1.641201in}}%
\pgfpathlineto{\pgfqpoint{4.669272in}{1.620573in}}%
\pgfpathlineto{\pgfqpoint{4.669272in}{1.599946in}}%
\pgfpathlineto{\pgfqpoint{4.669272in}{1.579318in}}%
\pgfpathlineto{\pgfqpoint{4.669272in}{1.558690in}}%
\pgfpathlineto{\pgfqpoint{4.669272in}{1.547190in}}%
\pgfpathlineto{\pgfqpoint{4.628227in}{1.547190in}}%
\pgfpathlineto{\pgfqpoint{4.587181in}{1.547190in}}%
\pgfpathlineto{\pgfqpoint{4.546136in}{1.547190in}}%
\pgfpathlineto{\pgfqpoint{4.505091in}{1.547190in}}%
\pgfpathlineto{\pgfqpoint{4.464045in}{1.547190in}}%
\pgfpathlineto{\pgfqpoint{4.423000in}{1.547190in}}%
\pgfpathlineto{\pgfqpoint{4.381955in}{1.547190in}}%
\pgfpathlineto{\pgfqpoint{4.340909in}{1.547190in}}%
\pgfpathlineto{\pgfqpoint{4.299864in}{1.547190in}}%
\pgfpathlineto{\pgfqpoint{4.258819in}{1.547190in}}%
\pgfpathlineto{\pgfqpoint{4.217773in}{1.547190in}}%
\pgfpathlineto{\pgfqpoint{4.176728in}{1.547190in}}%
\pgfpathlineto{\pgfqpoint{4.135683in}{1.547190in}}%
\pgfpathlineto{\pgfqpoint{4.094637in}{1.547190in}}%
\pgfpathlineto{\pgfqpoint{4.053592in}{1.547190in}}%
\pgfpathlineto{\pgfqpoint{4.012547in}{1.547190in}}%
\pgfpathlineto{\pgfqpoint{3.971501in}{1.547190in}}%
\pgfpathlineto{\pgfqpoint{3.930456in}{1.547190in}}%
\pgfpathlineto{\pgfqpoint{3.889411in}{1.547190in}}%
\pgfpathlineto{\pgfqpoint{3.848365in}{1.547190in}}%
\pgfpathlineto{\pgfqpoint{3.807320in}{1.547190in}}%
\pgfpathlineto{\pgfqpoint{3.766275in}{1.547190in}}%
\pgfpathlineto{\pgfqpoint{3.725229in}{1.547190in}}%
\pgfpathlineto{\pgfqpoint{3.684184in}{1.547190in}}%
\pgfpathlineto{\pgfqpoint{3.681218in}{1.558690in}}%
\pgfpathlineto{\pgfqpoint{3.676211in}{1.579318in}}%
\pgfpathlineto{\pgfqpoint{3.671613in}{1.599946in}}%
\pgfpathlineto{\pgfqpoint{3.667347in}{1.620573in}}%
\pgfpathlineto{\pgfqpoint{3.663353in}{1.641201in}}%
\pgfpathlineto{\pgfqpoint{3.659587in}{1.661829in}}%
\pgfpathlineto{\pgfqpoint{3.656013in}{1.682457in}}%
\pgfpathlineto{\pgfqpoint{3.652601in}{1.703085in}}%
\pgfpathlineto{\pgfqpoint{3.649329in}{1.723713in}}%
\pgfpathlineto{\pgfqpoint{3.646178in}{1.744340in}}%
\pgfpathlineto{\pgfqpoint{3.643139in}{1.764918in}}%
\pgfpathlineto{\pgfqpoint{3.602093in}{1.764918in}}%
\pgfpathlineto{\pgfqpoint{3.561048in}{1.764918in}}%
\pgfpathlineto{\pgfqpoint{3.520003in}{1.764918in}}%
\pgfpathlineto{\pgfqpoint{3.478957in}{1.764918in}}%
\pgfpathlineto{\pgfqpoint{3.437912in}{1.764918in}}%
\pgfpathlineto{\pgfqpoint{3.396867in}{1.764918in}}%
\pgfpathlineto{\pgfqpoint{3.355821in}{1.764918in}}%
\pgfpathlineto{\pgfqpoint{3.314776in}{1.764918in}}%
\pgfpathlineto{\pgfqpoint{3.273731in}{1.764918in}}%
\pgfpathlineto{\pgfqpoint{3.232685in}{1.764918in}}%
\pgfpathlineto{\pgfqpoint{3.191640in}{1.764918in}}%
\pgfpathlineto{\pgfqpoint{3.150595in}{1.764918in}}%
\pgfpathlineto{\pgfqpoint{3.109549in}{1.764918in}}%
\pgfpathlineto{\pgfqpoint{3.068504in}{1.764918in}}%
\pgfpathlineto{\pgfqpoint{3.027459in}{1.764918in}}%
\pgfpathlineto{\pgfqpoint{2.986413in}{1.764918in}}%
\pgfpathlineto{\pgfqpoint{2.945368in}{1.764918in}}%
\pgfpathlineto{\pgfqpoint{2.904323in}{1.764918in}}%
\pgfpathlineto{\pgfqpoint{2.863277in}{1.764918in}}%
\pgfpathlineto{\pgfqpoint{2.822232in}{1.764918in}}%
\pgfpathlineto{\pgfqpoint{2.781187in}{1.764918in}}%
\pgfpathlineto{\pgfqpoint{2.740141in}{1.764918in}}%
\pgfpathlineto{\pgfqpoint{2.699096in}{1.764918in}}%
\pgfpathlineto{\pgfqpoint{2.658051in}{1.764918in}}%
\pgfpathlineto{\pgfqpoint{2.617005in}{1.764918in}}%
\pgfpathlineto{\pgfqpoint{2.575960in}{1.764918in}}%
\pgfpathlineto{\pgfqpoint{2.534915in}{1.764918in}}%
\pgfpathlineto{\pgfqpoint{2.513384in}{1.744340in}}%
\pgfpathlineto{\pgfqpoint{2.508923in}{1.723713in}}%
\pgfpathlineto{\pgfqpoint{2.507478in}{1.703085in}}%
\pgfpathlineto{\pgfqpoint{2.507099in}{1.682457in}}%
\pgfpathlineto{\pgfqpoint{2.507218in}{1.661829in}}%
\pgfpathlineto{\pgfqpoint{2.507610in}{1.641201in}}%
\pgfpathlineto{\pgfqpoint{2.508169in}{1.620573in}}%
\pgfpathlineto{\pgfqpoint{2.508836in}{1.599946in}}%
\pgfpathlineto{\pgfqpoint{2.509577in}{1.579318in}}%
\pgfpathlineto{\pgfqpoint{2.510372in}{1.558690in}}%
\pgfpathlineto{\pgfqpoint{2.511206in}{1.538062in}}%
\pgfpathlineto{\pgfqpoint{2.512071in}{1.517434in}}%
\pgfpathlineto{\pgfqpoint{2.512960in}{1.496807in}}%
\pgfpathlineto{\pgfqpoint{2.513867in}{1.476179in}}%
\pgfpathlineto{\pgfqpoint{2.514790in}{1.455551in}}%
\pgfpathlineto{\pgfqpoint{2.515725in}{1.434923in}}%
\pgfpathlineto{\pgfqpoint{2.516670in}{1.414295in}}%
\pgfpathlineto{\pgfqpoint{2.517624in}{1.393667in}}%
\pgfpathlineto{\pgfqpoint{2.518585in}{1.373040in}}%
\pgfpathlineto{\pgfqpoint{2.519553in}{1.352412in}}%
\pgfpathlineto{\pgfqpoint{2.520526in}{1.331784in}}%
\pgfpathlineto{\pgfqpoint{2.521504in}{1.311156in}}%
\pgfpathlineto{\pgfqpoint{2.522486in}{1.290528in}}%
\pgfpathlineto{\pgfqpoint{2.523471in}{1.269900in}}%
\pgfpathlineto{\pgfqpoint{2.524460in}{1.249273in}}%
\pgfpathlineto{\pgfqpoint{2.525451in}{1.228645in}}%
\pgfpathlineto{\pgfqpoint{2.526445in}{1.208017in}}%
\pgfpathlineto{\pgfqpoint{2.527441in}{1.187389in}}%
\pgfpathlineto{\pgfqpoint{2.528439in}{1.166761in}}%
\pgfpathlineto{\pgfqpoint{2.529440in}{1.146134in}}%
\pgfpathlineto{\pgfqpoint{2.530441in}{1.125506in}}%
\pgfpathlineto{\pgfqpoint{2.531444in}{1.104878in}}%
\pgfpathlineto{\pgfqpoint{2.532449in}{1.084250in}}%
\pgfpathlineto{\pgfqpoint{2.533455in}{1.063622in}}%
\pgfpathlineto{\pgfqpoint{2.534462in}{1.042994in}}%
\pgfpathlineto{\pgfqpoint{2.534915in}{1.033867in}}%
\pgfpathlineto{\pgfqpoint{2.575960in}{1.033867in}}%
\pgfpathlineto{\pgfqpoint{2.617005in}{1.033867in}}%
\pgfpathlineto{\pgfqpoint{2.658051in}{1.033867in}}%
\pgfpathlineto{\pgfqpoint{2.699096in}{1.033867in}}%
\pgfpathlineto{\pgfqpoint{2.740141in}{1.033867in}}%
\pgfpathlineto{\pgfqpoint{2.781187in}{1.033867in}}%
\pgfpathlineto{\pgfqpoint{2.822232in}{1.033867in}}%
\pgfpathlineto{\pgfqpoint{2.863277in}{1.033867in}}%
\pgfpathlineto{\pgfqpoint{2.904323in}{1.033867in}}%
\pgfpathlineto{\pgfqpoint{2.945368in}{1.033867in}}%
\pgfpathlineto{\pgfqpoint{2.986413in}{1.033867in}}%
\pgfpathlineto{\pgfqpoint{3.027459in}{1.033867in}}%
\pgfpathlineto{\pgfqpoint{3.068504in}{1.033867in}}%
\pgfpathlineto{\pgfqpoint{3.109549in}{1.033867in}}%
\pgfpathlineto{\pgfqpoint{3.150595in}{1.033867in}}%
\pgfpathlineto{\pgfqpoint{3.191640in}{1.033867in}}%
\pgfpathlineto{\pgfqpoint{3.232685in}{1.033867in}}%
\pgfpathlineto{\pgfqpoint{3.273731in}{1.033867in}}%
\pgfpathlineto{\pgfqpoint{3.314776in}{1.033867in}}%
\pgfpathlineto{\pgfqpoint{3.355821in}{1.033867in}}%
\pgfpathlineto{\pgfqpoint{3.396867in}{1.033867in}}%
\pgfpathlineto{\pgfqpoint{3.437912in}{1.033867in}}%
\pgfpathlineto{\pgfqpoint{3.478957in}{1.033867in}}%
\pgfpathlineto{\pgfqpoint{3.520003in}{1.033867in}}%
\pgfpathlineto{\pgfqpoint{3.561048in}{1.033867in}}%
\pgfpathlineto{\pgfqpoint{3.602093in}{1.033867in}}%
\pgfpathlineto{\pgfqpoint{3.643139in}{1.033867in}}%
\pgfpathlineto{\pgfqpoint{3.646105in}{1.022367in}}%
\pgfpathlineto{\pgfqpoint{3.651111in}{1.001739in}}%
\pgfpathlineto{\pgfqpoint{3.655709in}{0.981111in}}%
\pgfpathlineto{\pgfqpoint{3.659976in}{0.960483in}}%
\pgfpathlineto{\pgfqpoint{3.663969in}{0.939855in}}%
\pgfpathlineto{\pgfqpoint{3.667735in}{0.919227in}}%
\pgfpathlineto{\pgfqpoint{3.671310in}{0.898600in}}%
\pgfpathlineto{\pgfqpoint{3.674721in}{0.877972in}}%
\pgfpathlineto{\pgfqpoint{3.677993in}{0.857344in}}%
\pgfpathlineto{\pgfqpoint{3.681145in}{0.836716in}}%
\pgfpathlineto{\pgfqpoint{3.684184in}{0.816139in}}%
\pgfpathlineto{\pgfqpoint{3.725229in}{0.816139in}}%
\pgfpathlineto{\pgfqpoint{3.766275in}{0.816139in}}%
\pgfpathlineto{\pgfqpoint{3.807320in}{0.816139in}}%
\pgfpathlineto{\pgfqpoint{3.848365in}{0.816139in}}%
\pgfpathlineto{\pgfqpoint{3.889411in}{0.816139in}}%
\pgfpathlineto{\pgfqpoint{3.930456in}{0.816139in}}%
\pgfpathlineto{\pgfqpoint{3.971501in}{0.816139in}}%
\pgfpathlineto{\pgfqpoint{4.012547in}{0.816139in}}%
\pgfpathlineto{\pgfqpoint{4.053592in}{0.816139in}}%
\pgfpathlineto{\pgfqpoint{4.094637in}{0.816139in}}%
\pgfpathlineto{\pgfqpoint{4.135683in}{0.816139in}}%
\pgfpathlineto{\pgfqpoint{4.176728in}{0.816139in}}%
\pgfpathlineto{\pgfqpoint{4.217773in}{0.816139in}}%
\pgfpathlineto{\pgfqpoint{4.258819in}{0.816139in}}%
\pgfpathlineto{\pgfqpoint{4.299864in}{0.816139in}}%
\pgfpathlineto{\pgfqpoint{4.340909in}{0.816139in}}%
\pgfpathlineto{\pgfqpoint{4.381955in}{0.816139in}}%
\pgfpathlineto{\pgfqpoint{4.423000in}{0.816139in}}%
\pgfpathlineto{\pgfqpoint{4.464045in}{0.816139in}}%
\pgfpathlineto{\pgfqpoint{4.505091in}{0.816139in}}%
\pgfpathlineto{\pgfqpoint{4.546136in}{0.816139in}}%
\pgfpathlineto{\pgfqpoint{4.587181in}{0.816139in}}%
\pgfpathlineto{\pgfqpoint{4.628227in}{0.816139in}}%
\pgfpathlineto{\pgfqpoint{4.669272in}{0.816139in}}%
\pgfpathlineto{\pgfqpoint{4.669272in}{0.816088in}}%
\pgfpathlineto{\pgfqpoint{4.669272in}{0.795460in}}%
\pgfpathlineto{\pgfqpoint{4.669272in}{0.774833in}}%
\pgfpathlineto{\pgfqpoint{4.669272in}{0.754205in}}%
\pgfpathlineto{\pgfqpoint{4.669272in}{0.733577in}}%
\pgfpathlineto{\pgfqpoint{4.669272in}{0.712949in}}%
\pgfpathlineto{\pgfqpoint{4.669272in}{0.692321in}}%
\pgfpathlineto{\pgfqpoint{4.669272in}{0.687855in}}%
\pgfpathlineto{\pgfqpoint{4.628227in}{0.687855in}}%
\pgfpathlineto{\pgfqpoint{4.587181in}{0.687855in}}%
\pgfpathlineto{\pgfqpoint{4.546136in}{0.687855in}}%
\pgfpathlineto{\pgfqpoint{4.505091in}{0.687855in}}%
\pgfpathlineto{\pgfqpoint{4.464045in}{0.687855in}}%
\pgfpathlineto{\pgfqpoint{4.423000in}{0.687855in}}%
\pgfpathlineto{\pgfqpoint{4.381955in}{0.687855in}}%
\pgfpathlineto{\pgfqpoint{4.340909in}{0.687855in}}%
\pgfpathlineto{\pgfqpoint{4.299864in}{0.687855in}}%
\pgfpathlineto{\pgfqpoint{4.258819in}{0.687855in}}%
\pgfpathlineto{\pgfqpoint{4.217773in}{0.687855in}}%
\pgfpathlineto{\pgfqpoint{4.176728in}{0.687855in}}%
\pgfpathlineto{\pgfqpoint{4.135683in}{0.687855in}}%
\pgfpathlineto{\pgfqpoint{4.094637in}{0.687855in}}%
\pgfpathlineto{\pgfqpoint{4.053592in}{0.687855in}}%
\pgfpathlineto{\pgfqpoint{4.012547in}{0.687855in}}%
\pgfpathlineto{\pgfqpoint{3.971501in}{0.687855in}}%
\pgfpathlineto{\pgfqpoint{3.930456in}{0.687855in}}%
\pgfpathlineto{\pgfqpoint{3.889411in}{0.687855in}}%
\pgfpathlineto{\pgfqpoint{3.848365in}{0.687855in}}%
\pgfpathlineto{\pgfqpoint{3.807320in}{0.687855in}}%
\pgfpathlineto{\pgfqpoint{3.766275in}{0.687855in}}%
\pgfpathlineto{\pgfqpoint{3.725229in}{0.687855in}}%
\pgfpathlineto{\pgfqpoint{3.684184in}{0.687855in}}%
\pgfpathlineto{\pgfqpoint{3.683480in}{0.692321in}}%
\pgfpathlineto{\pgfqpoint{3.680227in}{0.712949in}}%
\pgfpathlineto{\pgfqpoint{3.676877in}{0.733577in}}%
\pgfpathlineto{\pgfqpoint{3.673420in}{0.754205in}}%
\pgfpathlineto{\pgfqpoint{3.669841in}{0.774833in}}%
\pgfpathlineto{\pgfqpoint{3.666126in}{0.795460in}}%
\pgfpathlineto{\pgfqpoint{3.662258in}{0.816088in}}%
\pgfpathlineto{\pgfqpoint{3.658214in}{0.836716in}}%
\pgfpathlineto{\pgfqpoint{3.653971in}{0.857344in}}%
\pgfpathlineto{\pgfqpoint{3.649498in}{0.877972in}}%
\pgfpathlineto{\pgfqpoint{3.644759in}{0.898600in}}%
\pgfpathlineto{\pgfqpoint{3.643139in}{0.905464in}}%
\pgfpathlineto{\pgfqpoint{3.602093in}{0.905464in}}%
\pgfpathlineto{\pgfqpoint{3.561048in}{0.905464in}}%
\pgfpathlineto{\pgfqpoint{3.520003in}{0.905464in}}%
\pgfpathlineto{\pgfqpoint{3.478957in}{0.905464in}}%
\pgfpathlineto{\pgfqpoint{3.437912in}{0.905464in}}%
\pgfpathlineto{\pgfqpoint{3.396867in}{0.905464in}}%
\pgfpathlineto{\pgfqpoint{3.355821in}{0.905464in}}%
\pgfpathlineto{\pgfqpoint{3.314776in}{0.905464in}}%
\pgfpathlineto{\pgfqpoint{3.273731in}{0.905464in}}%
\pgfpathlineto{\pgfqpoint{3.232685in}{0.905464in}}%
\pgfpathlineto{\pgfqpoint{3.191640in}{0.905464in}}%
\pgfpathlineto{\pgfqpoint{3.150595in}{0.905464in}}%
\pgfpathlineto{\pgfqpoint{3.109549in}{0.905464in}}%
\pgfpathlineto{\pgfqpoint{3.068504in}{0.905464in}}%
\pgfpathlineto{\pgfqpoint{3.027459in}{0.905464in}}%
\pgfpathlineto{\pgfqpoint{2.986413in}{0.905464in}}%
\pgfpathlineto{\pgfqpoint{2.945368in}{0.905464in}}%
\pgfpathlineto{\pgfqpoint{2.904323in}{0.905464in}}%
\pgfpathlineto{\pgfqpoint{2.863277in}{0.905464in}}%
\pgfpathlineto{\pgfqpoint{2.822232in}{0.905464in}}%
\pgfpathlineto{\pgfqpoint{2.781187in}{0.905464in}}%
\pgfpathlineto{\pgfqpoint{2.740141in}{0.905464in}}%
\pgfpathlineto{\pgfqpoint{2.699096in}{0.905464in}}%
\pgfpathlineto{\pgfqpoint{2.658051in}{0.905464in}}%
\pgfpathlineto{\pgfqpoint{2.617005in}{0.905464in}}%
\pgfpathlineto{\pgfqpoint{2.575960in}{0.905464in}}%
\pgfpathlineto{\pgfqpoint{2.534915in}{0.905464in}}%
\pgfpathlineto{\pgfqpoint{2.534133in}{0.919227in}}%
\pgfpathlineto{\pgfqpoint{2.532963in}{0.939855in}}%
\pgfpathlineto{\pgfqpoint{2.531787in}{0.960483in}}%
\pgfpathlineto{\pgfqpoint{2.530602in}{0.981111in}}%
\pgfpathlineto{\pgfqpoint{2.529410in}{1.001739in}}%
\pgfpathlineto{\pgfqpoint{2.528208in}{1.022367in}}%
\pgfpathlineto{\pgfqpoint{2.526996in}{1.042994in}}%
\pgfpathlineto{\pgfqpoint{2.525773in}{1.063622in}}%
\pgfpathlineto{\pgfqpoint{2.524539in}{1.084250in}}%
\pgfpathlineto{\pgfqpoint{2.523292in}{1.104878in}}%
\pgfpathlineto{\pgfqpoint{2.522032in}{1.125506in}}%
\pgfpathlineto{\pgfqpoint{2.520755in}{1.146134in}}%
\pgfpathlineto{\pgfqpoint{2.519462in}{1.166761in}}%
\pgfpathlineto{\pgfqpoint{2.518151in}{1.187389in}}%
\pgfpathlineto{\pgfqpoint{2.516818in}{1.208017in}}%
\pgfpathlineto{\pgfqpoint{2.515463in}{1.228645in}}%
\pgfpathlineto{\pgfqpoint{2.514082in}{1.249273in}}%
\pgfpathlineto{\pgfqpoint{2.512672in}{1.269900in}}%
\pgfpathlineto{\pgfqpoint{2.511230in}{1.290528in}}%
\pgfpathlineto{\pgfqpoint{2.509751in}{1.311156in}}%
\pgfpathlineto{\pgfqpoint{2.508230in}{1.331784in}}%
\pgfpathlineto{\pgfqpoint{2.506661in}{1.352412in}}%
\pgfpathlineto{\pgfqpoint{2.505037in}{1.373040in}}%
\pgfpathlineto{\pgfqpoint{2.503349in}{1.393667in}}%
\pgfpathlineto{\pgfqpoint{2.501585in}{1.414295in}}%
\pgfpathlineto{\pgfqpoint{2.499733in}{1.434923in}}%
\pgfpathlineto{\pgfqpoint{2.497776in}{1.455551in}}%
\pgfpathlineto{\pgfqpoint{2.495691in}{1.476179in}}%
\pgfpathlineto{\pgfqpoint{2.493869in}{1.493170in}}%
\pgfpathlineto{\pgfqpoint{2.476636in}{1.476179in}}%
\pgfpathlineto{\pgfqpoint{2.454514in}{1.455551in}}%
\pgfpathlineto{\pgfqpoint{2.452824in}{1.454059in}}%
\pgfpathlineto{\pgfqpoint{2.435822in}{1.434923in}}%
\pgfpathlineto{\pgfqpoint{2.416141in}{1.414295in}}%
\pgfpathlineto{\pgfqpoint{2.411779in}{1.409997in}}%
\pgfpathlineto{\pgfqpoint{2.395437in}{1.393667in}}%
\pgfpathlineto{\pgfqpoint{2.373093in}{1.373040in}}%
\pgfpathlineto{\pgfqpoint{2.370733in}{1.370997in}}%
\pgfpathlineto{\pgfqpoint{2.344648in}{1.352412in}}%
\pgfpathlineto{\pgfqpoint{2.329688in}{1.342581in}}%
\pgfpathlineto{\pgfqpoint{2.304222in}{1.331784in}}%
\pgfpathlineto{\pgfqpoint{2.288643in}{1.325673in}}%
\pgfpathlineto{\pgfqpoint{2.247597in}{1.318505in}}%
\pgfpathlineto{\pgfqpoint{2.206552in}{1.317241in}}%
\pgfpathlineto{\pgfqpoint{2.165507in}{1.317240in}}%
\pgfpathlineto{\pgfqpoint{2.124461in}{1.314020in}}%
\pgfpathlineto{\pgfqpoint{2.112253in}{1.311156in}}%
\pgfpathlineto{\pgfqpoint{2.083416in}{1.304494in}}%
\pgfpathlineto{\pgfqpoint{2.049932in}{1.290528in}}%
\pgfpathlineto{\pgfqpoint{2.042371in}{1.287368in}}%
\pgfpathlineto{\pgfqpoint{2.011719in}{1.269900in}}%
\pgfpathlineto{\pgfqpoint{2.001325in}{1.263868in}}%
\pgfpathlineto{\pgfqpoint{1.978810in}{1.249273in}}%
\pgfpathlineto{\pgfqpoint{1.960280in}{1.236863in}}%
\pgfpathlineto{\pgfqpoint{1.947255in}{1.228645in}}%
\pgfpathlineto{\pgfqpoint{1.919235in}{1.210171in}}%
\pgfpathlineto{\pgfqpoint{1.915145in}{1.208017in}}%
\pgfpathlineto{\pgfqpoint{1.878189in}{1.187529in}}%
\pgfpathlineto{\pgfqpoint{1.877815in}{1.187389in}}%
\pgfpathlineto{\pgfqpoint{1.837144in}{1.171335in}}%
\pgfpathlineto{\pgfqpoint{1.817577in}{1.166761in}}%
\pgfpathlineto{\pgfqpoint{1.796099in}{1.161488in}}%
\pgfpathlineto{\pgfqpoint{1.755053in}{1.155412in}}%
\pgfpathlineto{\pgfqpoint{1.714008in}{1.148111in}}%
\pgfpathlineto{\pgfqpoint{1.707811in}{1.146134in}}%
\pgfpathlineto{\pgfqpoint{1.672963in}{1.135040in}}%
\pgfpathlineto{\pgfqpoint{1.654166in}{1.125506in}}%
\pgfpathlineto{\pgfqpoint{1.631917in}{1.114529in}}%
\pgfpathlineto{\pgfqpoint{1.616233in}{1.104878in}}%
\pgfpathlineto{\pgfqpoint{1.590872in}{1.090037in}}%
\pgfpathlineto{\pgfqpoint{1.580826in}{1.084250in}}%
\pgfpathlineto{\pgfqpoint{1.549827in}{1.067547in}}%
\pgfpathlineto{\pgfqpoint{1.539829in}{1.063622in}}%
\pgfpathlineto{\pgfqpoint{1.508781in}{1.052323in}}%
\pgfpathlineto{\pgfqpoint{1.467736in}{1.046829in}}%
\pgfpathlineto{\pgfqpoint{1.426691in}{1.050896in}}%
\pgfpathlineto{\pgfqpoint{1.385645in}{1.061487in}}%
\pgfpathlineto{\pgfqpoint{1.378631in}{1.063622in}}%
\pgfpathlineto{\pgfqpoint{1.344600in}{1.074128in}}%
\pgfpathlineto{\pgfqpoint{1.303555in}{1.083075in}}%
\pgfpathlineto{\pgfqpoint{1.262509in}{1.083479in}}%
\pgfpathlineto{\pgfqpoint{1.221464in}{1.072025in}}%
\pgfpathlineto{\pgfqpoint{1.207127in}{1.063622in}}%
\pgfpathlineto{\pgfqpoint{1.180419in}{1.047855in}}%
\pgfpathlineto{\pgfqpoint{1.174588in}{1.042994in}}%
\pgfpathlineto{\pgfqpoint{1.150045in}{1.022367in}}%
\pgfpathlineto{\pgfqpoint{1.139373in}{1.013259in}}%
\pgfpathlineto{\pgfqpoint{1.127595in}{1.001739in}}%
\pgfpathlineto{\pgfqpoint{1.106968in}{0.981111in}}%
\pgfpathlineto{\pgfqpoint{1.098328in}{0.972331in}}%
\pgfpathlineto{\pgfqpoint{1.086544in}{0.960483in}}%
\pgfpathlineto{\pgfqpoint{1.066595in}{0.939855in}}%
\pgfpathlineto{\pgfqpoint{1.057283in}{0.930006in}}%
\pgfpathlineto{\pgfqpoint{1.045713in}{0.919227in}}%
\pgfpathlineto{\pgfqpoint{1.024314in}{0.898600in}}%
\pgfpathlineto{\pgfqpoint{1.016237in}{0.890609in}}%
\pgfpathlineto{\pgfqpoint{1.000394in}{0.877972in}}%
\pgfpathlineto{\pgfqpoint{0.975786in}{0.857344in}}%
\pgfpathlineto{\pgfqpoint{0.975192in}{0.856838in}}%
\pgfpathlineto{\pgfqpoint{0.944454in}{0.836716in}}%
\pgfpathlineto{\pgfqpoint{0.934147in}{0.829671in}}%
\pgfpathlineto{\pgfqpoint{0.908500in}{0.816088in}}%
\pgfpathlineto{\pgfqpoint{0.893101in}{0.807677in}}%
\pgfpathlineto{\pgfqpoint{0.866485in}{0.795460in}}%
\pgfpathlineto{\pgfqpoint{0.852056in}{0.788749in}}%
\pgfpathlineto{\pgfqpoint{0.819398in}{0.774833in}}%
\pgfpathlineto{\pgfqpoint{0.811011in}{0.771285in}}%
\pgfpathlineto{\pgfqpoint{0.769965in}{0.755644in}}%
\pgfpathlineto{\pgfqpoint{0.764972in}{0.754205in}}%
\pgfpathlineto{\pgfqpoint{0.728920in}{0.744309in}}%
\pgfpathlineto{\pgfqpoint{0.687875in}{0.740259in}}%
\pgfpathlineto{\pgfqpoint{0.646829in}{0.745995in}}%
\pgfpathclose%
\pgfusepath{stroke,fill}%
}%
\begin{pgfscope}%
\pgfsys@transformshift{0.000000in}{0.000000in}%
\pgfsys@useobject{currentmarker}{}%
\end{pgfscope}%
\end{pgfscope}%
\begin{pgfscope}%
\pgfpathrectangle{\pgfqpoint{0.605784in}{0.382904in}}{\pgfqpoint{4.063488in}{2.042155in}}%
\pgfusepath{clip}%
\pgfsetbuttcap%
\pgfsetroundjoin%
\definecolor{currentfill}{rgb}{0.229739,0.322361,0.545706}%
\pgfsetfillcolor{currentfill}%
\pgfsetlinewidth{1.003750pt}%
\definecolor{currentstroke}{rgb}{0.229739,0.322361,0.545706}%
\pgfsetstrokecolor{currentstroke}%
\pgfsetdash{}{0pt}%
\pgfpathmoveto{\pgfqpoint{0.643201in}{0.651066in}}%
\pgfpathlineto{\pgfqpoint{0.605784in}{0.666235in}}%
\pgfpathlineto{\pgfqpoint{0.605784in}{0.671694in}}%
\pgfpathlineto{\pgfqpoint{0.605784in}{0.692321in}}%
\pgfpathlineto{\pgfqpoint{0.605784in}{0.712949in}}%
\pgfpathlineto{\pgfqpoint{0.605784in}{0.733577in}}%
\pgfpathlineto{\pgfqpoint{0.605784in}{0.754205in}}%
\pgfpathlineto{\pgfqpoint{0.605784in}{0.761947in}}%
\pgfpathlineto{\pgfqpoint{0.625696in}{0.754205in}}%
\pgfpathlineto{\pgfqpoint{0.646829in}{0.745995in}}%
\pgfpathlineto{\pgfqpoint{0.687875in}{0.740259in}}%
\pgfpathlineto{\pgfqpoint{0.728920in}{0.744309in}}%
\pgfpathlineto{\pgfqpoint{0.764972in}{0.754205in}}%
\pgfpathlineto{\pgfqpoint{0.769965in}{0.755644in}}%
\pgfpathlineto{\pgfqpoint{0.811011in}{0.771285in}}%
\pgfpathlineto{\pgfqpoint{0.819398in}{0.774833in}}%
\pgfpathlineto{\pgfqpoint{0.852056in}{0.788749in}}%
\pgfpathlineto{\pgfqpoint{0.866485in}{0.795460in}}%
\pgfpathlineto{\pgfqpoint{0.893101in}{0.807677in}}%
\pgfpathlineto{\pgfqpoint{0.908500in}{0.816088in}}%
\pgfpathlineto{\pgfqpoint{0.934147in}{0.829671in}}%
\pgfpathlineto{\pgfqpoint{0.944454in}{0.836716in}}%
\pgfpathlineto{\pgfqpoint{0.975192in}{0.856838in}}%
\pgfpathlineto{\pgfqpoint{0.975786in}{0.857344in}}%
\pgfpathlineto{\pgfqpoint{1.000394in}{0.877972in}}%
\pgfpathlineto{\pgfqpoint{1.016237in}{0.890609in}}%
\pgfpathlineto{\pgfqpoint{1.024314in}{0.898600in}}%
\pgfpathlineto{\pgfqpoint{1.045713in}{0.919227in}}%
\pgfpathlineto{\pgfqpoint{1.057283in}{0.930006in}}%
\pgfpathlineto{\pgfqpoint{1.066595in}{0.939855in}}%
\pgfpathlineto{\pgfqpoint{1.086544in}{0.960483in}}%
\pgfpathlineto{\pgfqpoint{1.098328in}{0.972331in}}%
\pgfpathlineto{\pgfqpoint{1.106968in}{0.981111in}}%
\pgfpathlineto{\pgfqpoint{1.127595in}{1.001739in}}%
\pgfpathlineto{\pgfqpoint{1.139373in}{1.013259in}}%
\pgfpathlineto{\pgfqpoint{1.150045in}{1.022367in}}%
\pgfpathlineto{\pgfqpoint{1.174588in}{1.042994in}}%
\pgfpathlineto{\pgfqpoint{1.180419in}{1.047855in}}%
\pgfpathlineto{\pgfqpoint{1.207127in}{1.063622in}}%
\pgfpathlineto{\pgfqpoint{1.221464in}{1.072025in}}%
\pgfpathlineto{\pgfqpoint{1.262509in}{1.083479in}}%
\pgfpathlineto{\pgfqpoint{1.303555in}{1.083075in}}%
\pgfpathlineto{\pgfqpoint{1.344600in}{1.074128in}}%
\pgfpathlineto{\pgfqpoint{1.378631in}{1.063622in}}%
\pgfpathlineto{\pgfqpoint{1.385645in}{1.061487in}}%
\pgfpathlineto{\pgfqpoint{1.426691in}{1.050896in}}%
\pgfpathlineto{\pgfqpoint{1.467736in}{1.046829in}}%
\pgfpathlineto{\pgfqpoint{1.508781in}{1.052323in}}%
\pgfpathlineto{\pgfqpoint{1.539829in}{1.063622in}}%
\pgfpathlineto{\pgfqpoint{1.549827in}{1.067547in}}%
\pgfpathlineto{\pgfqpoint{1.580826in}{1.084250in}}%
\pgfpathlineto{\pgfqpoint{1.590872in}{1.090037in}}%
\pgfpathlineto{\pgfqpoint{1.616233in}{1.104878in}}%
\pgfpathlineto{\pgfqpoint{1.631917in}{1.114529in}}%
\pgfpathlineto{\pgfqpoint{1.654166in}{1.125506in}}%
\pgfpathlineto{\pgfqpoint{1.672963in}{1.135040in}}%
\pgfpathlineto{\pgfqpoint{1.707811in}{1.146134in}}%
\pgfpathlineto{\pgfqpoint{1.714008in}{1.148111in}}%
\pgfpathlineto{\pgfqpoint{1.755053in}{1.155412in}}%
\pgfpathlineto{\pgfqpoint{1.796099in}{1.161488in}}%
\pgfpathlineto{\pgfqpoint{1.817577in}{1.166761in}}%
\pgfpathlineto{\pgfqpoint{1.837144in}{1.171335in}}%
\pgfpathlineto{\pgfqpoint{1.877815in}{1.187389in}}%
\pgfpathlineto{\pgfqpoint{1.878189in}{1.187529in}}%
\pgfpathlineto{\pgfqpoint{1.915145in}{1.208017in}}%
\pgfpathlineto{\pgfqpoint{1.919235in}{1.210171in}}%
\pgfpathlineto{\pgfqpoint{1.947255in}{1.228645in}}%
\pgfpathlineto{\pgfqpoint{1.960280in}{1.236863in}}%
\pgfpathlineto{\pgfqpoint{1.978810in}{1.249273in}}%
\pgfpathlineto{\pgfqpoint{2.001325in}{1.263868in}}%
\pgfpathlineto{\pgfqpoint{2.011719in}{1.269900in}}%
\pgfpathlineto{\pgfqpoint{2.042371in}{1.287368in}}%
\pgfpathlineto{\pgfqpoint{2.049932in}{1.290528in}}%
\pgfpathlineto{\pgfqpoint{2.083416in}{1.304494in}}%
\pgfpathlineto{\pgfqpoint{2.112253in}{1.311156in}}%
\pgfpathlineto{\pgfqpoint{2.124461in}{1.314020in}}%
\pgfpathlineto{\pgfqpoint{2.165507in}{1.317240in}}%
\pgfpathlineto{\pgfqpoint{2.206552in}{1.317241in}}%
\pgfpathlineto{\pgfqpoint{2.247597in}{1.318505in}}%
\pgfpathlineto{\pgfqpoint{2.288643in}{1.325673in}}%
\pgfpathlineto{\pgfqpoint{2.304222in}{1.331784in}}%
\pgfpathlineto{\pgfqpoint{2.329688in}{1.342581in}}%
\pgfpathlineto{\pgfqpoint{2.344648in}{1.352412in}}%
\pgfpathlineto{\pgfqpoint{2.370733in}{1.370997in}}%
\pgfpathlineto{\pgfqpoint{2.373093in}{1.373040in}}%
\pgfpathlineto{\pgfqpoint{2.395437in}{1.393667in}}%
\pgfpathlineto{\pgfqpoint{2.411779in}{1.409997in}}%
\pgfpathlineto{\pgfqpoint{2.416141in}{1.414295in}}%
\pgfpathlineto{\pgfqpoint{2.435822in}{1.434923in}}%
\pgfpathlineto{\pgfqpoint{2.452824in}{1.454059in}}%
\pgfpathlineto{\pgfqpoint{2.454514in}{1.455551in}}%
\pgfpathlineto{\pgfqpoint{2.476636in}{1.476179in}}%
\pgfpathlineto{\pgfqpoint{2.493869in}{1.493170in}}%
\pgfpathlineto{\pgfqpoint{2.495691in}{1.476179in}}%
\pgfpathlineto{\pgfqpoint{2.497776in}{1.455551in}}%
\pgfpathlineto{\pgfqpoint{2.499733in}{1.434923in}}%
\pgfpathlineto{\pgfqpoint{2.501585in}{1.414295in}}%
\pgfpathlineto{\pgfqpoint{2.503349in}{1.393667in}}%
\pgfpathlineto{\pgfqpoint{2.505037in}{1.373040in}}%
\pgfpathlineto{\pgfqpoint{2.506661in}{1.352412in}}%
\pgfpathlineto{\pgfqpoint{2.508230in}{1.331784in}}%
\pgfpathlineto{\pgfqpoint{2.509751in}{1.311156in}}%
\pgfpathlineto{\pgfqpoint{2.511230in}{1.290528in}}%
\pgfpathlineto{\pgfqpoint{2.512672in}{1.269900in}}%
\pgfpathlineto{\pgfqpoint{2.514082in}{1.249273in}}%
\pgfpathlineto{\pgfqpoint{2.515463in}{1.228645in}}%
\pgfpathlineto{\pgfqpoint{2.516818in}{1.208017in}}%
\pgfpathlineto{\pgfqpoint{2.518151in}{1.187389in}}%
\pgfpathlineto{\pgfqpoint{2.519462in}{1.166761in}}%
\pgfpathlineto{\pgfqpoint{2.520755in}{1.146134in}}%
\pgfpathlineto{\pgfqpoint{2.522032in}{1.125506in}}%
\pgfpathlineto{\pgfqpoint{2.523292in}{1.104878in}}%
\pgfpathlineto{\pgfqpoint{2.524539in}{1.084250in}}%
\pgfpathlineto{\pgfqpoint{2.525773in}{1.063622in}}%
\pgfpathlineto{\pgfqpoint{2.526996in}{1.042994in}}%
\pgfpathlineto{\pgfqpoint{2.528208in}{1.022367in}}%
\pgfpathlineto{\pgfqpoint{2.529410in}{1.001739in}}%
\pgfpathlineto{\pgfqpoint{2.530602in}{0.981111in}}%
\pgfpathlineto{\pgfqpoint{2.531787in}{0.960483in}}%
\pgfpathlineto{\pgfqpoint{2.532963in}{0.939855in}}%
\pgfpathlineto{\pgfqpoint{2.534133in}{0.919227in}}%
\pgfpathlineto{\pgfqpoint{2.534915in}{0.905464in}}%
\pgfpathlineto{\pgfqpoint{2.575960in}{0.905464in}}%
\pgfpathlineto{\pgfqpoint{2.617005in}{0.905464in}}%
\pgfpathlineto{\pgfqpoint{2.658051in}{0.905464in}}%
\pgfpathlineto{\pgfqpoint{2.699096in}{0.905464in}}%
\pgfpathlineto{\pgfqpoint{2.740141in}{0.905464in}}%
\pgfpathlineto{\pgfqpoint{2.781187in}{0.905464in}}%
\pgfpathlineto{\pgfqpoint{2.822232in}{0.905464in}}%
\pgfpathlineto{\pgfqpoint{2.863277in}{0.905464in}}%
\pgfpathlineto{\pgfqpoint{2.904323in}{0.905464in}}%
\pgfpathlineto{\pgfqpoint{2.945368in}{0.905464in}}%
\pgfpathlineto{\pgfqpoint{2.986413in}{0.905464in}}%
\pgfpathlineto{\pgfqpoint{3.027459in}{0.905464in}}%
\pgfpathlineto{\pgfqpoint{3.068504in}{0.905464in}}%
\pgfpathlineto{\pgfqpoint{3.109549in}{0.905464in}}%
\pgfpathlineto{\pgfqpoint{3.150595in}{0.905464in}}%
\pgfpathlineto{\pgfqpoint{3.191640in}{0.905464in}}%
\pgfpathlineto{\pgfqpoint{3.232685in}{0.905464in}}%
\pgfpathlineto{\pgfqpoint{3.273731in}{0.905464in}}%
\pgfpathlineto{\pgfqpoint{3.314776in}{0.905464in}}%
\pgfpathlineto{\pgfqpoint{3.355821in}{0.905464in}}%
\pgfpathlineto{\pgfqpoint{3.396867in}{0.905464in}}%
\pgfpathlineto{\pgfqpoint{3.437912in}{0.905464in}}%
\pgfpathlineto{\pgfqpoint{3.478957in}{0.905464in}}%
\pgfpathlineto{\pgfqpoint{3.520003in}{0.905464in}}%
\pgfpathlineto{\pgfqpoint{3.561048in}{0.905464in}}%
\pgfpathlineto{\pgfqpoint{3.602093in}{0.905464in}}%
\pgfpathlineto{\pgfqpoint{3.643139in}{0.905464in}}%
\pgfpathlineto{\pgfqpoint{3.644759in}{0.898600in}}%
\pgfpathlineto{\pgfqpoint{3.649498in}{0.877972in}}%
\pgfpathlineto{\pgfqpoint{3.653971in}{0.857344in}}%
\pgfpathlineto{\pgfqpoint{3.658214in}{0.836716in}}%
\pgfpathlineto{\pgfqpoint{3.662258in}{0.816088in}}%
\pgfpathlineto{\pgfqpoint{3.666126in}{0.795460in}}%
\pgfpathlineto{\pgfqpoint{3.669841in}{0.774833in}}%
\pgfpathlineto{\pgfqpoint{3.673420in}{0.754205in}}%
\pgfpathlineto{\pgfqpoint{3.676877in}{0.733577in}}%
\pgfpathlineto{\pgfqpoint{3.680227in}{0.712949in}}%
\pgfpathlineto{\pgfqpoint{3.683480in}{0.692321in}}%
\pgfpathlineto{\pgfqpoint{3.684184in}{0.687855in}}%
\pgfpathlineto{\pgfqpoint{3.725229in}{0.687855in}}%
\pgfpathlineto{\pgfqpoint{3.766275in}{0.687855in}}%
\pgfpathlineto{\pgfqpoint{3.807320in}{0.687855in}}%
\pgfpathlineto{\pgfqpoint{3.848365in}{0.687855in}}%
\pgfpathlineto{\pgfqpoint{3.889411in}{0.687855in}}%
\pgfpathlineto{\pgfqpoint{3.930456in}{0.687855in}}%
\pgfpathlineto{\pgfqpoint{3.971501in}{0.687855in}}%
\pgfpathlineto{\pgfqpoint{4.012547in}{0.687855in}}%
\pgfpathlineto{\pgfqpoint{4.053592in}{0.687855in}}%
\pgfpathlineto{\pgfqpoint{4.094637in}{0.687855in}}%
\pgfpathlineto{\pgfqpoint{4.135683in}{0.687855in}}%
\pgfpathlineto{\pgfqpoint{4.176728in}{0.687855in}}%
\pgfpathlineto{\pgfqpoint{4.217773in}{0.687855in}}%
\pgfpathlineto{\pgfqpoint{4.258819in}{0.687855in}}%
\pgfpathlineto{\pgfqpoint{4.299864in}{0.687855in}}%
\pgfpathlineto{\pgfqpoint{4.340909in}{0.687855in}}%
\pgfpathlineto{\pgfqpoint{4.381955in}{0.687855in}}%
\pgfpathlineto{\pgfqpoint{4.423000in}{0.687855in}}%
\pgfpathlineto{\pgfqpoint{4.464045in}{0.687855in}}%
\pgfpathlineto{\pgfqpoint{4.505091in}{0.687855in}}%
\pgfpathlineto{\pgfqpoint{4.546136in}{0.687855in}}%
\pgfpathlineto{\pgfqpoint{4.587181in}{0.687855in}}%
\pgfpathlineto{\pgfqpoint{4.628227in}{0.687855in}}%
\pgfpathlineto{\pgfqpoint{4.669272in}{0.687855in}}%
\pgfpathlineto{\pgfqpoint{4.669272in}{0.671694in}}%
\pgfpathlineto{\pgfqpoint{4.669272in}{0.651066in}}%
\pgfpathlineto{\pgfqpoint{4.669272in}{0.630438in}}%
\pgfpathlineto{\pgfqpoint{4.669272in}{0.609810in}}%
\pgfpathlineto{\pgfqpoint{4.669272in}{0.589182in}}%
\pgfpathlineto{\pgfqpoint{4.669272in}{0.586530in}}%
\pgfpathlineto{\pgfqpoint{4.628227in}{0.586530in}}%
\pgfpathlineto{\pgfqpoint{4.587181in}{0.586530in}}%
\pgfpathlineto{\pgfqpoint{4.546136in}{0.586530in}}%
\pgfpathlineto{\pgfqpoint{4.505091in}{0.586530in}}%
\pgfpathlineto{\pgfqpoint{4.464045in}{0.586530in}}%
\pgfpathlineto{\pgfqpoint{4.423000in}{0.586530in}}%
\pgfpathlineto{\pgfqpoint{4.381955in}{0.586530in}}%
\pgfpathlineto{\pgfqpoint{4.340909in}{0.586530in}}%
\pgfpathlineto{\pgfqpoint{4.299864in}{0.586530in}}%
\pgfpathlineto{\pgfqpoint{4.258819in}{0.586530in}}%
\pgfpathlineto{\pgfqpoint{4.217773in}{0.586530in}}%
\pgfpathlineto{\pgfqpoint{4.176728in}{0.586530in}}%
\pgfpathlineto{\pgfqpoint{4.135683in}{0.586530in}}%
\pgfpathlineto{\pgfqpoint{4.094637in}{0.586530in}}%
\pgfpathlineto{\pgfqpoint{4.053592in}{0.586530in}}%
\pgfpathlineto{\pgfqpoint{4.012547in}{0.586530in}}%
\pgfpathlineto{\pgfqpoint{3.971501in}{0.586530in}}%
\pgfpathlineto{\pgfqpoint{3.930456in}{0.586530in}}%
\pgfpathlineto{\pgfqpoint{3.889411in}{0.586530in}}%
\pgfpathlineto{\pgfqpoint{3.848365in}{0.586530in}}%
\pgfpathlineto{\pgfqpoint{3.807320in}{0.586530in}}%
\pgfpathlineto{\pgfqpoint{3.766275in}{0.586530in}}%
\pgfpathlineto{\pgfqpoint{3.725229in}{0.586530in}}%
\pgfpathlineto{\pgfqpoint{3.684184in}{0.586530in}}%
\pgfpathlineto{\pgfqpoint{3.683754in}{0.589182in}}%
\pgfpathlineto{\pgfqpoint{3.680412in}{0.609810in}}%
\pgfpathlineto{\pgfqpoint{3.676984in}{0.630438in}}%
\pgfpathlineto{\pgfqpoint{3.673460in}{0.651066in}}%
\pgfpathlineto{\pgfqpoint{3.669830in}{0.671694in}}%
\pgfpathlineto{\pgfqpoint{3.666085in}{0.692321in}}%
\pgfpathlineto{\pgfqpoint{3.662210in}{0.712949in}}%
\pgfpathlineto{\pgfqpoint{3.658193in}{0.733577in}}%
\pgfpathlineto{\pgfqpoint{3.654017in}{0.754205in}}%
\pgfpathlineto{\pgfqpoint{3.649662in}{0.774833in}}%
\pgfpathlineto{\pgfqpoint{3.645107in}{0.795460in}}%
\pgfpathlineto{\pgfqpoint{3.643139in}{0.804162in}}%
\pgfpathlineto{\pgfqpoint{3.602093in}{0.804162in}}%
\pgfpathlineto{\pgfqpoint{3.561048in}{0.804162in}}%
\pgfpathlineto{\pgfqpoint{3.520003in}{0.804162in}}%
\pgfpathlineto{\pgfqpoint{3.478957in}{0.804162in}}%
\pgfpathlineto{\pgfqpoint{3.437912in}{0.804162in}}%
\pgfpathlineto{\pgfqpoint{3.396867in}{0.804162in}}%
\pgfpathlineto{\pgfqpoint{3.355821in}{0.804162in}}%
\pgfpathlineto{\pgfqpoint{3.314776in}{0.804162in}}%
\pgfpathlineto{\pgfqpoint{3.273731in}{0.804162in}}%
\pgfpathlineto{\pgfqpoint{3.232685in}{0.804162in}}%
\pgfpathlineto{\pgfqpoint{3.191640in}{0.804162in}}%
\pgfpathlineto{\pgfqpoint{3.150595in}{0.804162in}}%
\pgfpathlineto{\pgfqpoint{3.109549in}{0.804162in}}%
\pgfpathlineto{\pgfqpoint{3.068504in}{0.804162in}}%
\pgfpathlineto{\pgfqpoint{3.027459in}{0.804162in}}%
\pgfpathlineto{\pgfqpoint{2.986413in}{0.804162in}}%
\pgfpathlineto{\pgfqpoint{2.945368in}{0.804162in}}%
\pgfpathlineto{\pgfqpoint{2.904323in}{0.804162in}}%
\pgfpathlineto{\pgfqpoint{2.863277in}{0.804162in}}%
\pgfpathlineto{\pgfqpoint{2.822232in}{0.804162in}}%
\pgfpathlineto{\pgfqpoint{2.781187in}{0.804162in}}%
\pgfpathlineto{\pgfqpoint{2.740141in}{0.804162in}}%
\pgfpathlineto{\pgfqpoint{2.699096in}{0.804162in}}%
\pgfpathlineto{\pgfqpoint{2.658051in}{0.804162in}}%
\pgfpathlineto{\pgfqpoint{2.617005in}{0.804162in}}%
\pgfpathlineto{\pgfqpoint{2.575960in}{0.804162in}}%
\pgfpathlineto{\pgfqpoint{2.534915in}{0.804162in}}%
\pgfpathlineto{\pgfqpoint{2.534183in}{0.816088in}}%
\pgfpathlineto{\pgfqpoint{2.532917in}{0.836716in}}%
\pgfpathlineto{\pgfqpoint{2.531642in}{0.857344in}}%
\pgfpathlineto{\pgfqpoint{2.530355in}{0.877972in}}%
\pgfpathlineto{\pgfqpoint{2.529055in}{0.898600in}}%
\pgfpathlineto{\pgfqpoint{2.527743in}{0.919227in}}%
\pgfpathlineto{\pgfqpoint{2.526416in}{0.939855in}}%
\pgfpathlineto{\pgfqpoint{2.525074in}{0.960483in}}%
\pgfpathlineto{\pgfqpoint{2.523716in}{0.981111in}}%
\pgfpathlineto{\pgfqpoint{2.522340in}{1.001739in}}%
\pgfpathlineto{\pgfqpoint{2.520946in}{1.022367in}}%
\pgfpathlineto{\pgfqpoint{2.519530in}{1.042994in}}%
\pgfpathlineto{\pgfqpoint{2.518092in}{1.063622in}}%
\pgfpathlineto{\pgfqpoint{2.516630in}{1.084250in}}%
\pgfpathlineto{\pgfqpoint{2.515140in}{1.104878in}}%
\pgfpathlineto{\pgfqpoint{2.513622in}{1.125506in}}%
\pgfpathlineto{\pgfqpoint{2.512071in}{1.146134in}}%
\pgfpathlineto{\pgfqpoint{2.510485in}{1.166761in}}%
\pgfpathlineto{\pgfqpoint{2.508860in}{1.187389in}}%
\pgfpathlineto{\pgfqpoint{2.507192in}{1.208017in}}%
\pgfpathlineto{\pgfqpoint{2.505475in}{1.228645in}}%
\pgfpathlineto{\pgfqpoint{2.503704in}{1.249273in}}%
\pgfpathlineto{\pgfqpoint{2.501873in}{1.269900in}}%
\pgfpathlineto{\pgfqpoint{2.499974in}{1.290528in}}%
\pgfpathlineto{\pgfqpoint{2.497998in}{1.311156in}}%
\pgfpathlineto{\pgfqpoint{2.495934in}{1.331784in}}%
\pgfpathlineto{\pgfqpoint{2.493869in}{1.351500in}}%
\pgfpathlineto{\pgfqpoint{2.465628in}{1.331784in}}%
\pgfpathlineto{\pgfqpoint{2.452824in}{1.323185in}}%
\pgfpathlineto{\pgfqpoint{2.437727in}{1.311156in}}%
\pgfpathlineto{\pgfqpoint{2.411779in}{1.291588in}}%
\pgfpathlineto{\pgfqpoint{2.410327in}{1.290528in}}%
\pgfpathlineto{\pgfqpoint{2.380358in}{1.269900in}}%
\pgfpathlineto{\pgfqpoint{2.370733in}{1.263674in}}%
\pgfpathlineto{\pgfqpoint{2.341992in}{1.249273in}}%
\pgfpathlineto{\pgfqpoint{2.329688in}{1.243505in}}%
\pgfpathlineto{\pgfqpoint{2.288643in}{1.232375in}}%
\pgfpathlineto{\pgfqpoint{2.247597in}{1.228845in}}%
\pgfpathlineto{\pgfqpoint{2.206552in}{1.229703in}}%
\pgfpathlineto{\pgfqpoint{2.165507in}{1.230685in}}%
\pgfpathlineto{\pgfqpoint{2.137653in}{1.228645in}}%
\pgfpathlineto{\pgfqpoint{2.124461in}{1.227705in}}%
\pgfpathlineto{\pgfqpoint{2.083416in}{1.217959in}}%
\pgfpathlineto{\pgfqpoint{2.060348in}{1.208017in}}%
\pgfpathlineto{\pgfqpoint{2.042371in}{1.200255in}}%
\pgfpathlineto{\pgfqpoint{2.020695in}{1.187389in}}%
\pgfpathlineto{\pgfqpoint{2.001325in}{1.175707in}}%
\pgfpathlineto{\pgfqpoint{1.988296in}{1.166761in}}%
\pgfpathlineto{\pgfqpoint{1.960280in}{1.146971in}}%
\pgfpathlineto{\pgfqpoint{1.959066in}{1.146134in}}%
\pgfpathlineto{\pgfqpoint{1.929760in}{1.125506in}}%
\pgfpathlineto{\pgfqpoint{1.919235in}{1.117833in}}%
\pgfpathlineto{\pgfqpoint{1.898618in}{1.104878in}}%
\pgfpathlineto{\pgfqpoint{1.878189in}{1.091479in}}%
\pgfpathlineto{\pgfqpoint{1.863859in}{1.084250in}}%
\pgfpathlineto{\pgfqpoint{1.837144in}{1.070163in}}%
\pgfpathlineto{\pgfqpoint{1.819930in}{1.063622in}}%
\pgfpathlineto{\pgfqpoint{1.796099in}{1.054194in}}%
\pgfpathlineto{\pgfqpoint{1.759228in}{1.042994in}}%
\pgfpathlineto{\pgfqpoint{1.755053in}{1.041687in}}%
\pgfpathlineto{\pgfqpoint{1.714008in}{1.029677in}}%
\pgfpathlineto{\pgfqpoint{1.693562in}{1.022367in}}%
\pgfpathlineto{\pgfqpoint{1.672963in}{1.015013in}}%
\pgfpathlineto{\pgfqpoint{1.642894in}{1.001739in}}%
\pgfpathlineto{\pgfqpoint{1.631917in}{0.996995in}}%
\pgfpathlineto{\pgfqpoint{1.598105in}{0.981111in}}%
\pgfpathlineto{\pgfqpoint{1.590872in}{0.977843in}}%
\pgfpathlineto{\pgfqpoint{1.549827in}{0.961498in}}%
\pgfpathlineto{\pgfqpoint{1.545781in}{0.960483in}}%
\pgfpathlineto{\pgfqpoint{1.508781in}{0.951763in}}%
\pgfpathlineto{\pgfqpoint{1.467736in}{0.950359in}}%
\pgfpathlineto{\pgfqpoint{1.426691in}{0.956845in}}%
\pgfpathlineto{\pgfqpoint{1.413976in}{0.960483in}}%
\pgfpathlineto{\pgfqpoint{1.385645in}{0.968621in}}%
\pgfpathlineto{\pgfqpoint{1.344807in}{0.981111in}}%
\pgfpathlineto{\pgfqpoint{1.344600in}{0.981175in}}%
\pgfpathlineto{\pgfqpoint{1.303555in}{0.989739in}}%
\pgfpathlineto{\pgfqpoint{1.262509in}{0.989493in}}%
\pgfpathlineto{\pgfqpoint{1.233869in}{0.981111in}}%
\pgfpathlineto{\pgfqpoint{1.221464in}{0.977516in}}%
\pgfpathlineto{\pgfqpoint{1.192483in}{0.960483in}}%
\pgfpathlineto{\pgfqpoint{1.180419in}{0.953349in}}%
\pgfpathlineto{\pgfqpoint{1.164192in}{0.939855in}}%
\pgfpathlineto{\pgfqpoint{1.139881in}{0.919227in}}%
\pgfpathlineto{\pgfqpoint{1.139373in}{0.918797in}}%
\pgfpathlineto{\pgfqpoint{1.118802in}{0.898600in}}%
\pgfpathlineto{\pgfqpoint{1.098458in}{0.877972in}}%
\pgfpathlineto{\pgfqpoint{1.098328in}{0.877839in}}%
\pgfpathlineto{\pgfqpoint{1.078311in}{0.857344in}}%
\pgfpathlineto{\pgfqpoint{1.058905in}{0.836716in}}%
\pgfpathlineto{\pgfqpoint{1.057283in}{0.834984in}}%
\pgfpathlineto{\pgfqpoint{1.037803in}{0.816088in}}%
\pgfpathlineto{\pgfqpoint{1.017406in}{0.795460in}}%
\pgfpathlineto{\pgfqpoint{1.016237in}{0.794270in}}%
\pgfpathlineto{\pgfqpoint{0.993472in}{0.774833in}}%
\pgfpathlineto{\pgfqpoint{0.975192in}{0.758548in}}%
\pgfpathlineto{\pgfqpoint{0.969050in}{0.754205in}}%
\pgfpathlineto{\pgfqpoint{0.940620in}{0.733577in}}%
\pgfpathlineto{\pgfqpoint{0.934147in}{0.728758in}}%
\pgfpathlineto{\pgfqpoint{0.907284in}{0.712949in}}%
\pgfpathlineto{\pgfqpoint{0.893101in}{0.704386in}}%
\pgfpathlineto{\pgfqpoint{0.868684in}{0.692321in}}%
\pgfpathlineto{\pgfqpoint{0.852056in}{0.684014in}}%
\pgfpathlineto{\pgfqpoint{0.823007in}{0.671694in}}%
\pgfpathlineto{\pgfqpoint{0.811011in}{0.666637in}}%
\pgfpathlineto{\pgfqpoint{0.769965in}{0.652572in}}%
\pgfpathlineto{\pgfqpoint{0.763234in}{0.651066in}}%
\pgfpathlineto{\pgfqpoint{0.728920in}{0.643688in}}%
\pgfpathlineto{\pgfqpoint{0.687875in}{0.642148in}}%
\pgfpathlineto{\pgfqpoint{0.646829in}{0.649596in}}%
\pgfpathclose%
\pgfusepath{stroke,fill}%
\end{pgfscope}%
\begin{pgfscope}%
\pgfpathrectangle{\pgfqpoint{0.605784in}{0.382904in}}{\pgfqpoint{4.063488in}{2.042155in}}%
\pgfusepath{clip}%
\pgfsetbuttcap%
\pgfsetroundjoin%
\definecolor{currentfill}{rgb}{0.229739,0.322361,0.545706}%
\pgfsetfillcolor{currentfill}%
\pgfsetlinewidth{1.003750pt}%
\definecolor{currentstroke}{rgb}{0.229739,0.322361,0.545706}%
\pgfsetstrokecolor{currentstroke}%
\pgfsetdash{}{0pt}%
\pgfpathmoveto{\pgfqpoint{0.627595in}{1.806224in}}%
\pgfpathlineto{\pgfqpoint{0.605784in}{1.819109in}}%
\pgfpathlineto{\pgfqpoint{0.605784in}{1.826852in}}%
\pgfpathlineto{\pgfqpoint{0.605784in}{1.847480in}}%
\pgfpathlineto{\pgfqpoint{0.605784in}{1.868107in}}%
\pgfpathlineto{\pgfqpoint{0.605784in}{1.888735in}}%
\pgfpathlineto{\pgfqpoint{0.605784in}{1.909363in}}%
\pgfpathlineto{\pgfqpoint{0.605784in}{1.914822in}}%
\pgfpathlineto{\pgfqpoint{0.615291in}{1.909363in}}%
\pgfpathlineto{\pgfqpoint{0.646829in}{1.891238in}}%
\pgfpathlineto{\pgfqpoint{0.650737in}{1.888735in}}%
\pgfpathlineto{\pgfqpoint{0.682649in}{1.868107in}}%
\pgfpathlineto{\pgfqpoint{0.687875in}{1.864701in}}%
\pgfpathlineto{\pgfqpoint{0.717662in}{1.847480in}}%
\pgfpathlineto{\pgfqpoint{0.728920in}{1.840867in}}%
\pgfpathlineto{\pgfqpoint{0.767792in}{1.826852in}}%
\pgfpathlineto{\pgfqpoint{0.769965in}{1.826044in}}%
\pgfpathlineto{\pgfqpoint{0.811011in}{1.825428in}}%
\pgfpathlineto{\pgfqpoint{0.814499in}{1.826852in}}%
\pgfpathlineto{\pgfqpoint{0.852056in}{1.842084in}}%
\pgfpathlineto{\pgfqpoint{0.858631in}{1.847480in}}%
\pgfpathlineto{\pgfqpoint{0.883757in}{1.868107in}}%
\pgfpathlineto{\pgfqpoint{0.893101in}{1.875806in}}%
\pgfpathlineto{\pgfqpoint{0.904408in}{1.888735in}}%
\pgfpathlineto{\pgfqpoint{0.922298in}{1.909363in}}%
\pgfpathlineto{\pgfqpoint{0.934147in}{1.923148in}}%
\pgfpathlineto{\pgfqpoint{0.939258in}{1.929991in}}%
\pgfpathlineto{\pgfqpoint{0.954647in}{1.950619in}}%
\pgfpathlineto{\pgfqpoint{0.969757in}{1.971247in}}%
\pgfpathlineto{\pgfqpoint{0.975192in}{1.978675in}}%
\pgfpathlineto{\pgfqpoint{0.984707in}{1.991874in}}%
\pgfpathlineto{\pgfqpoint{0.999408in}{2.012502in}}%
\pgfpathlineto{\pgfqpoint{1.013813in}{2.033130in}}%
\pgfpathlineto{\pgfqpoint{1.016237in}{2.036594in}}%
\pgfpathlineto{\pgfqpoint{1.029245in}{2.053758in}}%
\pgfpathlineto{\pgfqpoint{1.044568in}{2.074386in}}%
\pgfpathlineto{\pgfqpoint{1.057283in}{2.091837in}}%
\pgfpathlineto{\pgfqpoint{1.059986in}{2.095013in}}%
\pgfpathlineto{\pgfqpoint{1.077484in}{2.115641in}}%
\pgfpathlineto{\pgfqpoint{1.094488in}{2.136269in}}%
\pgfpathlineto{\pgfqpoint{1.098328in}{2.140947in}}%
\pgfpathlineto{\pgfqpoint{1.114463in}{2.156897in}}%
\pgfpathlineto{\pgfqpoint{1.134811in}{2.177525in}}%
\pgfpathlineto{\pgfqpoint{1.139373in}{2.182181in}}%
\pgfpathlineto{\pgfqpoint{1.159504in}{2.198153in}}%
\pgfpathlineto{\pgfqpoint{1.180419in}{2.215093in}}%
\pgfpathlineto{\pgfqpoint{1.186551in}{2.218780in}}%
\pgfpathlineto{\pgfqpoint{1.220668in}{2.239408in}}%
\pgfpathlineto{\pgfqpoint{1.221464in}{2.239889in}}%
\pgfpathlineto{\pgfqpoint{1.262509in}{2.256465in}}%
\pgfpathlineto{\pgfqpoint{1.280986in}{2.260036in}}%
\pgfpathlineto{\pgfqpoint{1.303555in}{2.264286in}}%
\pgfpathlineto{\pgfqpoint{1.344600in}{2.262152in}}%
\pgfpathlineto{\pgfqpoint{1.351197in}{2.260036in}}%
\pgfpathlineto{\pgfqpoint{1.385645in}{2.248852in}}%
\pgfpathlineto{\pgfqpoint{1.401508in}{2.239408in}}%
\pgfpathlineto{\pgfqpoint{1.426691in}{2.224358in}}%
\pgfpathlineto{\pgfqpoint{1.433416in}{2.218780in}}%
\pgfpathlineto{\pgfqpoint{1.458093in}{2.198153in}}%
\pgfpathlineto{\pgfqpoint{1.467736in}{2.190046in}}%
\pgfpathlineto{\pgfqpoint{1.480499in}{2.177525in}}%
\pgfpathlineto{\pgfqpoint{1.501370in}{2.156897in}}%
\pgfpathlineto{\pgfqpoint{1.508781in}{2.149484in}}%
\pgfpathlineto{\pgfqpoint{1.522094in}{2.136269in}}%
\pgfpathlineto{\pgfqpoint{1.542607in}{2.115641in}}%
\pgfpathlineto{\pgfqpoint{1.549827in}{2.108254in}}%
\pgfpathlineto{\pgfqpoint{1.565764in}{2.095013in}}%
\pgfpathlineto{\pgfqpoint{1.590087in}{2.074386in}}%
\pgfpathlineto{\pgfqpoint{1.590872in}{2.073701in}}%
\pgfpathlineto{\pgfqpoint{1.631496in}{2.053758in}}%
\pgfpathlineto{\pgfqpoint{1.631917in}{2.053543in}}%
\pgfpathlineto{\pgfqpoint{1.642200in}{2.053758in}}%
\pgfpathlineto{\pgfqpoint{1.672963in}{2.054387in}}%
\pgfpathlineto{\pgfqpoint{1.706060in}{2.074386in}}%
\pgfpathlineto{\pgfqpoint{1.714008in}{2.079180in}}%
\pgfpathlineto{\pgfqpoint{1.727872in}{2.095013in}}%
\pgfpathlineto{\pgfqpoint{1.745856in}{2.115641in}}%
\pgfpathlineto{\pgfqpoint{1.755053in}{2.126170in}}%
\pgfpathlineto{\pgfqpoint{1.761595in}{2.136269in}}%
\pgfpathlineto{\pgfqpoint{1.775011in}{2.156897in}}%
\pgfpathlineto{\pgfqpoint{1.788330in}{2.177525in}}%
\pgfpathlineto{\pgfqpoint{1.796099in}{2.189532in}}%
\pgfpathlineto{\pgfqpoint{1.801015in}{2.198153in}}%
\pgfpathlineto{\pgfqpoint{1.812833in}{2.218780in}}%
\pgfpathlineto{\pgfqpoint{1.824558in}{2.239408in}}%
\pgfpathlineto{\pgfqpoint{1.836194in}{2.260036in}}%
\pgfpathlineto{\pgfqpoint{1.837144in}{2.261700in}}%
\pgfpathlineto{\pgfqpoint{1.847852in}{2.280664in}}%
\pgfpathlineto{\pgfqpoint{1.859402in}{2.301292in}}%
\pgfpathlineto{\pgfqpoint{1.870840in}{2.321920in}}%
\pgfpathlineto{\pgfqpoint{1.878189in}{2.335197in}}%
\pgfpathlineto{\pgfqpoint{1.882588in}{2.342547in}}%
\pgfpathlineto{\pgfqpoint{1.894957in}{2.363175in}}%
\pgfpathlineto{\pgfqpoint{1.907170in}{2.383803in}}%
\pgfpathlineto{\pgfqpoint{1.919235in}{2.404424in}}%
\pgfpathlineto{\pgfqpoint{1.919239in}{2.404431in}}%
\pgfpathlineto{\pgfqpoint{1.933387in}{2.425059in}}%
\pgfpathlineto{\pgfqpoint{1.960280in}{2.425059in}}%
\pgfpathlineto{\pgfqpoint{1.999798in}{2.425059in}}%
\pgfpathlineto{\pgfqpoint{1.983616in}{2.404431in}}%
\pgfpathlineto{\pgfqpoint{1.967130in}{2.383803in}}%
\pgfpathlineto{\pgfqpoint{1.960280in}{2.375258in}}%
\pgfpathlineto{\pgfqpoint{1.952535in}{2.363175in}}%
\pgfpathlineto{\pgfqpoint{1.939256in}{2.342547in}}%
\pgfpathlineto{\pgfqpoint{1.925770in}{2.321920in}}%
\pgfpathlineto{\pgfqpoint{1.919235in}{2.311948in}}%
\pgfpathlineto{\pgfqpoint{1.913285in}{2.301292in}}%
\pgfpathlineto{\pgfqpoint{1.901775in}{2.280664in}}%
\pgfpathlineto{\pgfqpoint{1.890129in}{2.260036in}}%
\pgfpathlineto{\pgfqpoint{1.878338in}{2.239408in}}%
\pgfpathlineto{\pgfqpoint{1.878189in}{2.239145in}}%
\pgfpathlineto{\pgfqpoint{1.867675in}{2.218780in}}%
\pgfpathlineto{\pgfqpoint{1.856943in}{2.198153in}}%
\pgfpathlineto{\pgfqpoint{1.846121in}{2.177525in}}%
\pgfpathlineto{\pgfqpoint{1.837144in}{2.160490in}}%
\pgfpathlineto{\pgfqpoint{1.835244in}{2.156897in}}%
\pgfpathlineto{\pgfqpoint{1.824492in}{2.136269in}}%
\pgfpathlineto{\pgfqpoint{1.813684in}{2.115641in}}%
\pgfpathlineto{\pgfqpoint{1.802816in}{2.095013in}}%
\pgfpathlineto{\pgfqpoint{1.796099in}{2.082214in}}%
\pgfpathlineto{\pgfqpoint{1.791450in}{2.074386in}}%
\pgfpathlineto{\pgfqpoint{1.779340in}{2.053758in}}%
\pgfpathlineto{\pgfqpoint{1.767181in}{2.033130in}}%
\pgfpathlineto{\pgfqpoint{1.755053in}{2.012640in}}%
\pgfpathlineto{\pgfqpoint{1.754943in}{2.012502in}}%
\pgfpathlineto{\pgfqpoint{1.738704in}{1.991874in}}%
\pgfpathlineto{\pgfqpoint{1.722355in}{1.971247in}}%
\pgfpathlineto{\pgfqpoint{1.714008in}{1.960641in}}%
\pgfpathlineto{\pgfqpoint{1.698357in}{1.950619in}}%
\pgfpathlineto{\pgfqpoint{1.672963in}{1.934322in}}%
\pgfpathlineto{\pgfqpoint{1.631917in}{1.935964in}}%
\pgfpathlineto{\pgfqpoint{1.608787in}{1.950619in}}%
\pgfpathlineto{\pgfqpoint{1.590872in}{1.961444in}}%
\pgfpathlineto{\pgfqpoint{1.581314in}{1.971247in}}%
\pgfpathlineto{\pgfqpoint{1.560558in}{1.991874in}}%
\pgfpathlineto{\pgfqpoint{1.549827in}{2.002229in}}%
\pgfpathlineto{\pgfqpoint{1.541006in}{2.012502in}}%
\pgfpathlineto{\pgfqpoint{1.522902in}{2.033130in}}%
\pgfpathlineto{\pgfqpoint{1.508781in}{2.048949in}}%
\pgfpathlineto{\pgfqpoint{1.504448in}{2.053758in}}%
\pgfpathlineto{\pgfqpoint{1.485500in}{2.074386in}}%
\pgfpathlineto{\pgfqpoint{1.467736in}{2.093607in}}%
\pgfpathlineto{\pgfqpoint{1.466190in}{2.095013in}}%
\pgfpathlineto{\pgfqpoint{1.443141in}{2.115641in}}%
\pgfpathlineto{\pgfqpoint{1.426691in}{2.130337in}}%
\pgfpathlineto{\pgfqpoint{1.417240in}{2.136269in}}%
\pgfpathlineto{\pgfqpoint{1.385645in}{2.156009in}}%
\pgfpathlineto{\pgfqpoint{1.382926in}{2.156897in}}%
\pgfpathlineto{\pgfqpoint{1.344600in}{2.169233in}}%
\pgfpathlineto{\pgfqpoint{1.303555in}{2.170859in}}%
\pgfpathlineto{\pgfqpoint{1.262509in}{2.162383in}}%
\pgfpathlineto{\pgfqpoint{1.249193in}{2.156897in}}%
\pgfpathlineto{\pgfqpoint{1.221464in}{2.145338in}}%
\pgfpathlineto{\pgfqpoint{1.206509in}{2.136269in}}%
\pgfpathlineto{\pgfqpoint{1.180419in}{2.120560in}}%
\pgfpathlineto{\pgfqpoint{1.174334in}{2.115641in}}%
\pgfpathlineto{\pgfqpoint{1.148554in}{2.095013in}}%
\pgfpathlineto{\pgfqpoint{1.139373in}{2.087778in}}%
\pgfpathlineto{\pgfqpoint{1.126300in}{2.074386in}}%
\pgfpathlineto{\pgfqpoint{1.105669in}{2.053758in}}%
\pgfpathlineto{\pgfqpoint{1.098328in}{2.046520in}}%
\pgfpathlineto{\pgfqpoint{1.087505in}{2.033130in}}%
\pgfpathlineto{\pgfqpoint{1.070502in}{2.012502in}}%
\pgfpathlineto{\pgfqpoint{1.057283in}{1.996859in}}%
\pgfpathlineto{\pgfqpoint{1.053731in}{1.991874in}}%
\pgfpathlineto{\pgfqpoint{1.039017in}{1.971247in}}%
\pgfpathlineto{\pgfqpoint{1.023901in}{1.950619in}}%
\pgfpathlineto{\pgfqpoint{1.016237in}{1.940280in}}%
\pgfpathlineto{\pgfqpoint{1.009311in}{1.929991in}}%
\pgfpathlineto{\pgfqpoint{0.995339in}{1.909363in}}%
\pgfpathlineto{\pgfqpoint{0.981050in}{1.888735in}}%
\pgfpathlineto{\pgfqpoint{0.975192in}{1.880317in}}%
\pgfpathlineto{\pgfqpoint{0.966672in}{1.868107in}}%
\pgfpathlineto{\pgfqpoint{0.952185in}{1.847480in}}%
\pgfpathlineto{\pgfqpoint{0.937413in}{1.826852in}}%
\pgfpathlineto{\pgfqpoint{0.934147in}{1.822272in}}%
\pgfpathlineto{\pgfqpoint{0.921038in}{1.806224in}}%
\pgfpathlineto{\pgfqpoint{0.903994in}{1.785596in}}%
\pgfpathlineto{\pgfqpoint{0.893101in}{1.772517in}}%
\pgfpathlineto{\pgfqpoint{0.884313in}{1.764968in}}%
\pgfpathlineto{\pgfqpoint{0.860246in}{1.744340in}}%
\pgfpathlineto{\pgfqpoint{0.852056in}{1.737330in}}%
\pgfpathlineto{\pgfqpoint{0.818311in}{1.723713in}}%
\pgfpathlineto{\pgfqpoint{0.811011in}{1.720744in}}%
\pgfpathlineto{\pgfqpoint{0.769965in}{1.722975in}}%
\pgfpathlineto{\pgfqpoint{0.768276in}{1.723713in}}%
\pgfpathlineto{\pgfqpoint{0.728920in}{1.740281in}}%
\pgfpathlineto{\pgfqpoint{0.722710in}{1.744340in}}%
\pgfpathlineto{\pgfqpoint{0.690498in}{1.764968in}}%
\pgfpathlineto{\pgfqpoint{0.687875in}{1.766612in}}%
\pgfpathlineto{\pgfqpoint{0.660343in}{1.785596in}}%
\pgfpathlineto{\pgfqpoint{0.646829in}{1.794850in}}%
\pgfpathclose%
\pgfusepath{stroke,fill}%
\end{pgfscope}%
\begin{pgfscope}%
\pgfpathrectangle{\pgfqpoint{0.605784in}{0.382904in}}{\pgfqpoint{4.063488in}{2.042155in}}%
\pgfusepath{clip}%
\pgfsetbuttcap%
\pgfsetroundjoin%
\definecolor{currentfill}{rgb}{0.229739,0.322361,0.545706}%
\pgfsetfillcolor{currentfill}%
\pgfsetlinewidth{1.003750pt}%
\definecolor{currentstroke}{rgb}{0.229739,0.322361,0.545706}%
\pgfsetstrokecolor{currentstroke}%
\pgfsetdash{}{0pt}%
\pgfpathmoveto{\pgfqpoint{2.279797in}{2.425059in}}%
\pgfpathlineto{\pgfqpoint{2.288643in}{2.425059in}}%
\pgfpathlineto{\pgfqpoint{2.329688in}{2.425059in}}%
\pgfpathlineto{\pgfqpoint{2.352125in}{2.425059in}}%
\pgfpathlineto{\pgfqpoint{2.366653in}{2.404431in}}%
\pgfpathlineto{\pgfqpoint{2.370733in}{2.398514in}}%
\pgfpathlineto{\pgfqpoint{2.381697in}{2.383803in}}%
\pgfpathlineto{\pgfqpoint{2.396754in}{2.363175in}}%
\pgfpathlineto{\pgfqpoint{2.411526in}{2.342547in}}%
\pgfpathlineto{\pgfqpoint{2.411779in}{2.342185in}}%
\pgfpathlineto{\pgfqpoint{2.430347in}{2.321920in}}%
\pgfpathlineto{\pgfqpoint{2.448723in}{2.301292in}}%
\pgfpathlineto{\pgfqpoint{2.452824in}{2.296528in}}%
\pgfpathlineto{\pgfqpoint{2.479986in}{2.280664in}}%
\pgfpathlineto{\pgfqpoint{2.493869in}{2.272154in}}%
\pgfpathlineto{\pgfqpoint{2.495016in}{2.260036in}}%
\pgfpathlineto{\pgfqpoint{2.497016in}{2.239408in}}%
\pgfpathlineto{\pgfqpoint{2.499106in}{2.218780in}}%
\pgfpathlineto{\pgfqpoint{2.501301in}{2.198153in}}%
\pgfpathlineto{\pgfqpoint{2.503617in}{2.177525in}}%
\pgfpathlineto{\pgfqpoint{2.506073in}{2.156897in}}%
\pgfpathlineto{\pgfqpoint{2.508694in}{2.136269in}}%
\pgfpathlineto{\pgfqpoint{2.511510in}{2.115641in}}%
\pgfpathlineto{\pgfqpoint{2.514559in}{2.095013in}}%
\pgfpathlineto{\pgfqpoint{2.517889in}{2.074386in}}%
\pgfpathlineto{\pgfqpoint{2.521565in}{2.053758in}}%
\pgfpathlineto{\pgfqpoint{2.525669in}{2.033130in}}%
\pgfpathlineto{\pgfqpoint{2.530314in}{2.012502in}}%
\pgfpathlineto{\pgfqpoint{2.534915in}{1.994527in}}%
\pgfpathlineto{\pgfqpoint{2.575960in}{1.994527in}}%
\pgfpathlineto{\pgfqpoint{2.617005in}{1.994527in}}%
\pgfpathlineto{\pgfqpoint{2.658051in}{1.994527in}}%
\pgfpathlineto{\pgfqpoint{2.699096in}{1.994527in}}%
\pgfpathlineto{\pgfqpoint{2.740141in}{1.994527in}}%
\pgfpathlineto{\pgfqpoint{2.781187in}{1.994527in}}%
\pgfpathlineto{\pgfqpoint{2.822232in}{1.994527in}}%
\pgfpathlineto{\pgfqpoint{2.863277in}{1.994527in}}%
\pgfpathlineto{\pgfqpoint{2.904323in}{1.994527in}}%
\pgfpathlineto{\pgfqpoint{2.945368in}{1.994527in}}%
\pgfpathlineto{\pgfqpoint{2.986413in}{1.994527in}}%
\pgfpathlineto{\pgfqpoint{3.027459in}{1.994527in}}%
\pgfpathlineto{\pgfqpoint{3.068504in}{1.994527in}}%
\pgfpathlineto{\pgfqpoint{3.109549in}{1.994527in}}%
\pgfpathlineto{\pgfqpoint{3.150595in}{1.994527in}}%
\pgfpathlineto{\pgfqpoint{3.191640in}{1.994527in}}%
\pgfpathlineto{\pgfqpoint{3.232685in}{1.994527in}}%
\pgfpathlineto{\pgfqpoint{3.273731in}{1.994527in}}%
\pgfpathlineto{\pgfqpoint{3.314776in}{1.994527in}}%
\pgfpathlineto{\pgfqpoint{3.355821in}{1.994527in}}%
\pgfpathlineto{\pgfqpoint{3.396867in}{1.994527in}}%
\pgfpathlineto{\pgfqpoint{3.437912in}{1.994527in}}%
\pgfpathlineto{\pgfqpoint{3.478957in}{1.994527in}}%
\pgfpathlineto{\pgfqpoint{3.520003in}{1.994527in}}%
\pgfpathlineto{\pgfqpoint{3.561048in}{1.994527in}}%
\pgfpathlineto{\pgfqpoint{3.602093in}{1.994527in}}%
\pgfpathlineto{\pgfqpoint{3.643139in}{1.994527in}}%
\pgfpathlineto{\pgfqpoint{3.643569in}{1.991874in}}%
\pgfpathlineto{\pgfqpoint{3.646910in}{1.971247in}}%
\pgfpathlineto{\pgfqpoint{3.650339in}{1.950619in}}%
\pgfpathlineto{\pgfqpoint{3.653863in}{1.929991in}}%
\pgfpathlineto{\pgfqpoint{3.657492in}{1.909363in}}%
\pgfpathlineto{\pgfqpoint{3.661238in}{1.888735in}}%
\pgfpathlineto{\pgfqpoint{3.665112in}{1.868107in}}%
\pgfpathlineto{\pgfqpoint{3.669129in}{1.847480in}}%
\pgfpathlineto{\pgfqpoint{3.673305in}{1.826852in}}%
\pgfpathlineto{\pgfqpoint{3.677660in}{1.806224in}}%
\pgfpathlineto{\pgfqpoint{3.682216in}{1.785596in}}%
\pgfpathlineto{\pgfqpoint{3.684184in}{1.776894in}}%
\pgfpathlineto{\pgfqpoint{3.725229in}{1.776894in}}%
\pgfpathlineto{\pgfqpoint{3.766275in}{1.776894in}}%
\pgfpathlineto{\pgfqpoint{3.807320in}{1.776894in}}%
\pgfpathlineto{\pgfqpoint{3.848365in}{1.776894in}}%
\pgfpathlineto{\pgfqpoint{3.889411in}{1.776894in}}%
\pgfpathlineto{\pgfqpoint{3.930456in}{1.776894in}}%
\pgfpathlineto{\pgfqpoint{3.971501in}{1.776894in}}%
\pgfpathlineto{\pgfqpoint{4.012547in}{1.776894in}}%
\pgfpathlineto{\pgfqpoint{4.053592in}{1.776894in}}%
\pgfpathlineto{\pgfqpoint{4.094637in}{1.776894in}}%
\pgfpathlineto{\pgfqpoint{4.135683in}{1.776894in}}%
\pgfpathlineto{\pgfqpoint{4.176728in}{1.776894in}}%
\pgfpathlineto{\pgfqpoint{4.217773in}{1.776894in}}%
\pgfpathlineto{\pgfqpoint{4.258819in}{1.776894in}}%
\pgfpathlineto{\pgfqpoint{4.299864in}{1.776894in}}%
\pgfpathlineto{\pgfqpoint{4.340909in}{1.776894in}}%
\pgfpathlineto{\pgfqpoint{4.381955in}{1.776894in}}%
\pgfpathlineto{\pgfqpoint{4.423000in}{1.776894in}}%
\pgfpathlineto{\pgfqpoint{4.464045in}{1.776894in}}%
\pgfpathlineto{\pgfqpoint{4.505091in}{1.776894in}}%
\pgfpathlineto{\pgfqpoint{4.546136in}{1.776894in}}%
\pgfpathlineto{\pgfqpoint{4.587181in}{1.776894in}}%
\pgfpathlineto{\pgfqpoint{4.628227in}{1.776894in}}%
\pgfpathlineto{\pgfqpoint{4.669272in}{1.776894in}}%
\pgfpathlineto{\pgfqpoint{4.669272in}{1.764968in}}%
\pgfpathlineto{\pgfqpoint{4.669272in}{1.744340in}}%
\pgfpathlineto{\pgfqpoint{4.669272in}{1.723713in}}%
\pgfpathlineto{\pgfqpoint{4.669272in}{1.703085in}}%
\pgfpathlineto{\pgfqpoint{4.669272in}{1.682457in}}%
\pgfpathlineto{\pgfqpoint{4.669272in}{1.675593in}}%
\pgfpathlineto{\pgfqpoint{4.628227in}{1.675593in}}%
\pgfpathlineto{\pgfqpoint{4.587181in}{1.675593in}}%
\pgfpathlineto{\pgfqpoint{4.546136in}{1.675593in}}%
\pgfpathlineto{\pgfqpoint{4.505091in}{1.675593in}}%
\pgfpathlineto{\pgfqpoint{4.464045in}{1.675593in}}%
\pgfpathlineto{\pgfqpoint{4.423000in}{1.675593in}}%
\pgfpathlineto{\pgfqpoint{4.381955in}{1.675593in}}%
\pgfpathlineto{\pgfqpoint{4.340909in}{1.675593in}}%
\pgfpathlineto{\pgfqpoint{4.299864in}{1.675593in}}%
\pgfpathlineto{\pgfqpoint{4.258819in}{1.675593in}}%
\pgfpathlineto{\pgfqpoint{4.217773in}{1.675593in}}%
\pgfpathlineto{\pgfqpoint{4.176728in}{1.675593in}}%
\pgfpathlineto{\pgfqpoint{4.135683in}{1.675593in}}%
\pgfpathlineto{\pgfqpoint{4.094637in}{1.675593in}}%
\pgfpathlineto{\pgfqpoint{4.053592in}{1.675593in}}%
\pgfpathlineto{\pgfqpoint{4.012547in}{1.675593in}}%
\pgfpathlineto{\pgfqpoint{3.971501in}{1.675593in}}%
\pgfpathlineto{\pgfqpoint{3.930456in}{1.675593in}}%
\pgfpathlineto{\pgfqpoint{3.889411in}{1.675593in}}%
\pgfpathlineto{\pgfqpoint{3.848365in}{1.675593in}}%
\pgfpathlineto{\pgfqpoint{3.807320in}{1.675593in}}%
\pgfpathlineto{\pgfqpoint{3.766275in}{1.675593in}}%
\pgfpathlineto{\pgfqpoint{3.725229in}{1.675593in}}%
\pgfpathlineto{\pgfqpoint{3.684184in}{1.675593in}}%
\pgfpathlineto{\pgfqpoint{3.682564in}{1.682457in}}%
\pgfpathlineto{\pgfqpoint{3.677825in}{1.703085in}}%
\pgfpathlineto{\pgfqpoint{3.673352in}{1.723713in}}%
\pgfpathlineto{\pgfqpoint{3.669108in}{1.744340in}}%
\pgfpathlineto{\pgfqpoint{3.665065in}{1.764968in}}%
\pgfpathlineto{\pgfqpoint{3.661196in}{1.785596in}}%
\pgfpathlineto{\pgfqpoint{3.657481in}{1.806224in}}%
\pgfpathlineto{\pgfqpoint{3.653903in}{1.826852in}}%
\pgfpathlineto{\pgfqpoint{3.650445in}{1.847480in}}%
\pgfpathlineto{\pgfqpoint{3.647095in}{1.868107in}}%
\pgfpathlineto{\pgfqpoint{3.643842in}{1.888735in}}%
\pgfpathlineto{\pgfqpoint{3.643139in}{1.893201in}}%
\pgfpathlineto{\pgfqpoint{3.602093in}{1.893201in}}%
\pgfpathlineto{\pgfqpoint{3.561048in}{1.893201in}}%
\pgfpathlineto{\pgfqpoint{3.520003in}{1.893201in}}%
\pgfpathlineto{\pgfqpoint{3.478957in}{1.893201in}}%
\pgfpathlineto{\pgfqpoint{3.437912in}{1.893201in}}%
\pgfpathlineto{\pgfqpoint{3.396867in}{1.893201in}}%
\pgfpathlineto{\pgfqpoint{3.355821in}{1.893201in}}%
\pgfpathlineto{\pgfqpoint{3.314776in}{1.893201in}}%
\pgfpathlineto{\pgfqpoint{3.273731in}{1.893201in}}%
\pgfpathlineto{\pgfqpoint{3.232685in}{1.893201in}}%
\pgfpathlineto{\pgfqpoint{3.191640in}{1.893201in}}%
\pgfpathlineto{\pgfqpoint{3.150595in}{1.893201in}}%
\pgfpathlineto{\pgfqpoint{3.109549in}{1.893201in}}%
\pgfpathlineto{\pgfqpoint{3.068504in}{1.893201in}}%
\pgfpathlineto{\pgfqpoint{3.027459in}{1.893201in}}%
\pgfpathlineto{\pgfqpoint{2.986413in}{1.893201in}}%
\pgfpathlineto{\pgfqpoint{2.945368in}{1.893201in}}%
\pgfpathlineto{\pgfqpoint{2.904323in}{1.893201in}}%
\pgfpathlineto{\pgfqpoint{2.863277in}{1.893201in}}%
\pgfpathlineto{\pgfqpoint{2.822232in}{1.893201in}}%
\pgfpathlineto{\pgfqpoint{2.781187in}{1.893201in}}%
\pgfpathlineto{\pgfqpoint{2.740141in}{1.893201in}}%
\pgfpathlineto{\pgfqpoint{2.699096in}{1.893201in}}%
\pgfpathlineto{\pgfqpoint{2.658051in}{1.893201in}}%
\pgfpathlineto{\pgfqpoint{2.617005in}{1.893201in}}%
\pgfpathlineto{\pgfqpoint{2.575960in}{1.893201in}}%
\pgfpathlineto{\pgfqpoint{2.534915in}{1.893201in}}%
\pgfpathlineto{\pgfqpoint{2.528792in}{1.909363in}}%
\pgfpathlineto{\pgfqpoint{2.522591in}{1.929991in}}%
\pgfpathlineto{\pgfqpoint{2.517628in}{1.950619in}}%
\pgfpathlineto{\pgfqpoint{2.513506in}{1.971247in}}%
\pgfpathlineto{\pgfqpoint{2.509982in}{1.991874in}}%
\pgfpathlineto{\pgfqpoint{2.506897in}{2.012502in}}%
\pgfpathlineto{\pgfqpoint{2.504146in}{2.033130in}}%
\pgfpathlineto{\pgfqpoint{2.501653in}{2.053758in}}%
\pgfpathlineto{\pgfqpoint{2.499364in}{2.074386in}}%
\pgfpathlineto{\pgfqpoint{2.497239in}{2.095013in}}%
\pgfpathlineto{\pgfqpoint{2.495249in}{2.115641in}}%
\pgfpathlineto{\pgfqpoint{2.493869in}{2.130540in}}%
\pgfpathlineto{\pgfqpoint{2.487625in}{2.136269in}}%
\pgfpathlineto{\pgfqpoint{2.463557in}{2.156897in}}%
\pgfpathlineto{\pgfqpoint{2.452824in}{2.165521in}}%
\pgfpathlineto{\pgfqpoint{2.444846in}{2.177525in}}%
\pgfpathlineto{\pgfqpoint{2.430491in}{2.198153in}}%
\pgfpathlineto{\pgfqpoint{2.415542in}{2.218780in}}%
\pgfpathlineto{\pgfqpoint{2.411779in}{2.223754in}}%
\pgfpathlineto{\pgfqpoint{2.402631in}{2.239408in}}%
\pgfpathlineto{\pgfqpoint{2.390193in}{2.260036in}}%
\pgfpathlineto{\pgfqpoint{2.377410in}{2.280664in}}%
\pgfpathlineto{\pgfqpoint{2.370733in}{2.291114in}}%
\pgfpathlineto{\pgfqpoint{2.364648in}{2.301292in}}%
\pgfpathlineto{\pgfqpoint{2.351993in}{2.321920in}}%
\pgfpathlineto{\pgfqpoint{2.339102in}{2.342547in}}%
\pgfpathlineto{\pgfqpoint{2.329688in}{2.357300in}}%
\pgfpathlineto{\pgfqpoint{2.325642in}{2.363175in}}%
\pgfpathlineto{\pgfqpoint{2.311116in}{2.383803in}}%
\pgfpathlineto{\pgfqpoint{2.296422in}{2.404431in}}%
\pgfpathlineto{\pgfqpoint{2.288643in}{2.415163in}}%
\pgfpathclose%
\pgfusepath{stroke,fill}%
\end{pgfscope}%
\begin{pgfscope}%
\pgfpathrectangle{\pgfqpoint{0.605784in}{0.382904in}}{\pgfqpoint{4.063488in}{2.042155in}}%
\pgfusepath{clip}%
\pgfsetbuttcap%
\pgfsetroundjoin%
\definecolor{currentfill}{rgb}{0.195860,0.395433,0.555276}%
\pgfsetfillcolor{currentfill}%
\pgfsetlinewidth{1.003750pt}%
\definecolor{currentstroke}{rgb}{0.195860,0.395433,0.555276}%
\pgfsetstrokecolor{currentstroke}%
\pgfsetdash{}{0pt}%
\pgfpathmoveto{\pgfqpoint{0.641478in}{0.568554in}}%
\pgfpathlineto{\pgfqpoint{0.605784in}{0.583374in}}%
\pgfpathlineto{\pgfqpoint{0.605784in}{0.589182in}}%
\pgfpathlineto{\pgfqpoint{0.605784in}{0.609810in}}%
\pgfpathlineto{\pgfqpoint{0.605784in}{0.630438in}}%
\pgfpathlineto{\pgfqpoint{0.605784in}{0.651066in}}%
\pgfpathlineto{\pgfqpoint{0.605784in}{0.666235in}}%
\pgfpathlineto{\pgfqpoint{0.643201in}{0.651066in}}%
\pgfpathlineto{\pgfqpoint{0.646829in}{0.649596in}}%
\pgfpathlineto{\pgfqpoint{0.687875in}{0.642148in}}%
\pgfpathlineto{\pgfqpoint{0.728920in}{0.643688in}}%
\pgfpathlineto{\pgfqpoint{0.763234in}{0.651066in}}%
\pgfpathlineto{\pgfqpoint{0.769965in}{0.652572in}}%
\pgfpathlineto{\pgfqpoint{0.811011in}{0.666637in}}%
\pgfpathlineto{\pgfqpoint{0.823007in}{0.671694in}}%
\pgfpathlineto{\pgfqpoint{0.852056in}{0.684014in}}%
\pgfpathlineto{\pgfqpoint{0.868684in}{0.692321in}}%
\pgfpathlineto{\pgfqpoint{0.893101in}{0.704386in}}%
\pgfpathlineto{\pgfqpoint{0.907284in}{0.712949in}}%
\pgfpathlineto{\pgfqpoint{0.934147in}{0.728758in}}%
\pgfpathlineto{\pgfqpoint{0.940620in}{0.733577in}}%
\pgfpathlineto{\pgfqpoint{0.969050in}{0.754205in}}%
\pgfpathlineto{\pgfqpoint{0.975192in}{0.758548in}}%
\pgfpathlineto{\pgfqpoint{0.993472in}{0.774833in}}%
\pgfpathlineto{\pgfqpoint{1.016237in}{0.794270in}}%
\pgfpathlineto{\pgfqpoint{1.017406in}{0.795460in}}%
\pgfpathlineto{\pgfqpoint{1.037803in}{0.816088in}}%
\pgfpathlineto{\pgfqpoint{1.057283in}{0.834984in}}%
\pgfpathlineto{\pgfqpoint{1.058905in}{0.836716in}}%
\pgfpathlineto{\pgfqpoint{1.078311in}{0.857344in}}%
\pgfpathlineto{\pgfqpoint{1.098328in}{0.877839in}}%
\pgfpathlineto{\pgfqpoint{1.098458in}{0.877972in}}%
\pgfpathlineto{\pgfqpoint{1.118802in}{0.898600in}}%
\pgfpathlineto{\pgfqpoint{1.139373in}{0.918797in}}%
\pgfpathlineto{\pgfqpoint{1.139881in}{0.919227in}}%
\pgfpathlineto{\pgfqpoint{1.164192in}{0.939855in}}%
\pgfpathlineto{\pgfqpoint{1.180419in}{0.953349in}}%
\pgfpathlineto{\pgfqpoint{1.192483in}{0.960483in}}%
\pgfpathlineto{\pgfqpoint{1.221464in}{0.977516in}}%
\pgfpathlineto{\pgfqpoint{1.233869in}{0.981111in}}%
\pgfpathlineto{\pgfqpoint{1.262509in}{0.989493in}}%
\pgfpathlineto{\pgfqpoint{1.303555in}{0.989739in}}%
\pgfpathlineto{\pgfqpoint{1.344600in}{0.981175in}}%
\pgfpathlineto{\pgfqpoint{1.344807in}{0.981111in}}%
\pgfpathlineto{\pgfqpoint{1.385645in}{0.968621in}}%
\pgfpathlineto{\pgfqpoint{1.413976in}{0.960483in}}%
\pgfpathlineto{\pgfqpoint{1.426691in}{0.956845in}}%
\pgfpathlineto{\pgfqpoint{1.467736in}{0.950359in}}%
\pgfpathlineto{\pgfqpoint{1.508781in}{0.951763in}}%
\pgfpathlineto{\pgfqpoint{1.545781in}{0.960483in}}%
\pgfpathlineto{\pgfqpoint{1.549827in}{0.961498in}}%
\pgfpathlineto{\pgfqpoint{1.590872in}{0.977843in}}%
\pgfpathlineto{\pgfqpoint{1.598105in}{0.981111in}}%
\pgfpathlineto{\pgfqpoint{1.631917in}{0.996995in}}%
\pgfpathlineto{\pgfqpoint{1.642894in}{1.001739in}}%
\pgfpathlineto{\pgfqpoint{1.672963in}{1.015013in}}%
\pgfpathlineto{\pgfqpoint{1.693562in}{1.022367in}}%
\pgfpathlineto{\pgfqpoint{1.714008in}{1.029677in}}%
\pgfpathlineto{\pgfqpoint{1.755053in}{1.041687in}}%
\pgfpathlineto{\pgfqpoint{1.759228in}{1.042994in}}%
\pgfpathlineto{\pgfqpoint{1.796099in}{1.054194in}}%
\pgfpathlineto{\pgfqpoint{1.819930in}{1.063622in}}%
\pgfpathlineto{\pgfqpoint{1.837144in}{1.070163in}}%
\pgfpathlineto{\pgfqpoint{1.863859in}{1.084250in}}%
\pgfpathlineto{\pgfqpoint{1.878189in}{1.091479in}}%
\pgfpathlineto{\pgfqpoint{1.898618in}{1.104878in}}%
\pgfpathlineto{\pgfqpoint{1.919235in}{1.117833in}}%
\pgfpathlineto{\pgfqpoint{1.929760in}{1.125506in}}%
\pgfpathlineto{\pgfqpoint{1.959066in}{1.146134in}}%
\pgfpathlineto{\pgfqpoint{1.960280in}{1.146971in}}%
\pgfpathlineto{\pgfqpoint{1.988296in}{1.166761in}}%
\pgfpathlineto{\pgfqpoint{2.001325in}{1.175707in}}%
\pgfpathlineto{\pgfqpoint{2.020695in}{1.187389in}}%
\pgfpathlineto{\pgfqpoint{2.042371in}{1.200255in}}%
\pgfpathlineto{\pgfqpoint{2.060348in}{1.208017in}}%
\pgfpathlineto{\pgfqpoint{2.083416in}{1.217959in}}%
\pgfpathlineto{\pgfqpoint{2.124461in}{1.227705in}}%
\pgfpathlineto{\pgfqpoint{2.137653in}{1.228645in}}%
\pgfpathlineto{\pgfqpoint{2.165507in}{1.230685in}}%
\pgfpathlineto{\pgfqpoint{2.206552in}{1.229703in}}%
\pgfpathlineto{\pgfqpoint{2.247597in}{1.228845in}}%
\pgfpathlineto{\pgfqpoint{2.288643in}{1.232375in}}%
\pgfpathlineto{\pgfqpoint{2.329688in}{1.243505in}}%
\pgfpathlineto{\pgfqpoint{2.341992in}{1.249273in}}%
\pgfpathlineto{\pgfqpoint{2.370733in}{1.263674in}}%
\pgfpathlineto{\pgfqpoint{2.380358in}{1.269900in}}%
\pgfpathlineto{\pgfqpoint{2.410327in}{1.290528in}}%
\pgfpathlineto{\pgfqpoint{2.411779in}{1.291588in}}%
\pgfpathlineto{\pgfqpoint{2.437727in}{1.311156in}}%
\pgfpathlineto{\pgfqpoint{2.452824in}{1.323185in}}%
\pgfpathlineto{\pgfqpoint{2.465628in}{1.331784in}}%
\pgfpathlineto{\pgfqpoint{2.493869in}{1.351500in}}%
\pgfpathlineto{\pgfqpoint{2.495934in}{1.331784in}}%
\pgfpathlineto{\pgfqpoint{2.497998in}{1.311156in}}%
\pgfpathlineto{\pgfqpoint{2.499974in}{1.290528in}}%
\pgfpathlineto{\pgfqpoint{2.501873in}{1.269900in}}%
\pgfpathlineto{\pgfqpoint{2.503704in}{1.249273in}}%
\pgfpathlineto{\pgfqpoint{2.505475in}{1.228645in}}%
\pgfpathlineto{\pgfqpoint{2.507192in}{1.208017in}}%
\pgfpathlineto{\pgfqpoint{2.508860in}{1.187389in}}%
\pgfpathlineto{\pgfqpoint{2.510485in}{1.166761in}}%
\pgfpathlineto{\pgfqpoint{2.512071in}{1.146134in}}%
\pgfpathlineto{\pgfqpoint{2.513622in}{1.125506in}}%
\pgfpathlineto{\pgfqpoint{2.515140in}{1.104878in}}%
\pgfpathlineto{\pgfqpoint{2.516630in}{1.084250in}}%
\pgfpathlineto{\pgfqpoint{2.518092in}{1.063622in}}%
\pgfpathlineto{\pgfqpoint{2.519530in}{1.042994in}}%
\pgfpathlineto{\pgfqpoint{2.520946in}{1.022367in}}%
\pgfpathlineto{\pgfqpoint{2.522340in}{1.001739in}}%
\pgfpathlineto{\pgfqpoint{2.523716in}{0.981111in}}%
\pgfpathlineto{\pgfqpoint{2.525074in}{0.960483in}}%
\pgfpathlineto{\pgfqpoint{2.526416in}{0.939855in}}%
\pgfpathlineto{\pgfqpoint{2.527743in}{0.919227in}}%
\pgfpathlineto{\pgfqpoint{2.529055in}{0.898600in}}%
\pgfpathlineto{\pgfqpoint{2.530355in}{0.877972in}}%
\pgfpathlineto{\pgfqpoint{2.531642in}{0.857344in}}%
\pgfpathlineto{\pgfqpoint{2.532917in}{0.836716in}}%
\pgfpathlineto{\pgfqpoint{2.534183in}{0.816088in}}%
\pgfpathlineto{\pgfqpoint{2.534915in}{0.804162in}}%
\pgfpathlineto{\pgfqpoint{2.575960in}{0.804162in}}%
\pgfpathlineto{\pgfqpoint{2.617005in}{0.804162in}}%
\pgfpathlineto{\pgfqpoint{2.658051in}{0.804162in}}%
\pgfpathlineto{\pgfqpoint{2.699096in}{0.804162in}}%
\pgfpathlineto{\pgfqpoint{2.740141in}{0.804162in}}%
\pgfpathlineto{\pgfqpoint{2.781187in}{0.804162in}}%
\pgfpathlineto{\pgfqpoint{2.822232in}{0.804162in}}%
\pgfpathlineto{\pgfqpoint{2.863277in}{0.804162in}}%
\pgfpathlineto{\pgfqpoint{2.904323in}{0.804162in}}%
\pgfpathlineto{\pgfqpoint{2.945368in}{0.804162in}}%
\pgfpathlineto{\pgfqpoint{2.986413in}{0.804162in}}%
\pgfpathlineto{\pgfqpoint{3.027459in}{0.804162in}}%
\pgfpathlineto{\pgfqpoint{3.068504in}{0.804162in}}%
\pgfpathlineto{\pgfqpoint{3.109549in}{0.804162in}}%
\pgfpathlineto{\pgfqpoint{3.150595in}{0.804162in}}%
\pgfpathlineto{\pgfqpoint{3.191640in}{0.804162in}}%
\pgfpathlineto{\pgfqpoint{3.232685in}{0.804162in}}%
\pgfpathlineto{\pgfqpoint{3.273731in}{0.804162in}}%
\pgfpathlineto{\pgfqpoint{3.314776in}{0.804162in}}%
\pgfpathlineto{\pgfqpoint{3.355821in}{0.804162in}}%
\pgfpathlineto{\pgfqpoint{3.396867in}{0.804162in}}%
\pgfpathlineto{\pgfqpoint{3.437912in}{0.804162in}}%
\pgfpathlineto{\pgfqpoint{3.478957in}{0.804162in}}%
\pgfpathlineto{\pgfqpoint{3.520003in}{0.804162in}}%
\pgfpathlineto{\pgfqpoint{3.561048in}{0.804162in}}%
\pgfpathlineto{\pgfqpoint{3.602093in}{0.804162in}}%
\pgfpathlineto{\pgfqpoint{3.643139in}{0.804162in}}%
\pgfpathlineto{\pgfqpoint{3.645107in}{0.795460in}}%
\pgfpathlineto{\pgfqpoint{3.649662in}{0.774833in}}%
\pgfpathlineto{\pgfqpoint{3.654017in}{0.754205in}}%
\pgfpathlineto{\pgfqpoint{3.658193in}{0.733577in}}%
\pgfpathlineto{\pgfqpoint{3.662210in}{0.712949in}}%
\pgfpathlineto{\pgfqpoint{3.666085in}{0.692321in}}%
\pgfpathlineto{\pgfqpoint{3.669830in}{0.671694in}}%
\pgfpathlineto{\pgfqpoint{3.673460in}{0.651066in}}%
\pgfpathlineto{\pgfqpoint{3.676984in}{0.630438in}}%
\pgfpathlineto{\pgfqpoint{3.680412in}{0.609810in}}%
\pgfpathlineto{\pgfqpoint{3.683754in}{0.589182in}}%
\pgfpathlineto{\pgfqpoint{3.684184in}{0.586530in}}%
\pgfpathlineto{\pgfqpoint{3.725229in}{0.586530in}}%
\pgfpathlineto{\pgfqpoint{3.766275in}{0.586530in}}%
\pgfpathlineto{\pgfqpoint{3.807320in}{0.586530in}}%
\pgfpathlineto{\pgfqpoint{3.848365in}{0.586530in}}%
\pgfpathlineto{\pgfqpoint{3.889411in}{0.586530in}}%
\pgfpathlineto{\pgfqpoint{3.930456in}{0.586530in}}%
\pgfpathlineto{\pgfqpoint{3.971501in}{0.586530in}}%
\pgfpathlineto{\pgfqpoint{4.012547in}{0.586530in}}%
\pgfpathlineto{\pgfqpoint{4.053592in}{0.586530in}}%
\pgfpathlineto{\pgfqpoint{4.094637in}{0.586530in}}%
\pgfpathlineto{\pgfqpoint{4.135683in}{0.586530in}}%
\pgfpathlineto{\pgfqpoint{4.176728in}{0.586530in}}%
\pgfpathlineto{\pgfqpoint{4.217773in}{0.586530in}}%
\pgfpathlineto{\pgfqpoint{4.258819in}{0.586530in}}%
\pgfpathlineto{\pgfqpoint{4.299864in}{0.586530in}}%
\pgfpathlineto{\pgfqpoint{4.340909in}{0.586530in}}%
\pgfpathlineto{\pgfqpoint{4.381955in}{0.586530in}}%
\pgfpathlineto{\pgfqpoint{4.423000in}{0.586530in}}%
\pgfpathlineto{\pgfqpoint{4.464045in}{0.586530in}}%
\pgfpathlineto{\pgfqpoint{4.505091in}{0.586530in}}%
\pgfpathlineto{\pgfqpoint{4.546136in}{0.586530in}}%
\pgfpathlineto{\pgfqpoint{4.587181in}{0.586530in}}%
\pgfpathlineto{\pgfqpoint{4.628227in}{0.586530in}}%
\pgfpathlineto{\pgfqpoint{4.669272in}{0.586530in}}%
\pgfpathlineto{\pgfqpoint{4.669272in}{0.568554in}}%
\pgfpathlineto{\pgfqpoint{4.669272in}{0.547927in}}%
\pgfpathlineto{\pgfqpoint{4.669272in}{0.527299in}}%
\pgfpathlineto{\pgfqpoint{4.669272in}{0.506671in}}%
\pgfpathlineto{\pgfqpoint{4.669272in}{0.500154in}}%
\pgfpathlineto{\pgfqpoint{4.628227in}{0.500154in}}%
\pgfpathlineto{\pgfqpoint{4.587181in}{0.500154in}}%
\pgfpathlineto{\pgfqpoint{4.546136in}{0.500154in}}%
\pgfpathlineto{\pgfqpoint{4.505091in}{0.500154in}}%
\pgfpathlineto{\pgfqpoint{4.464045in}{0.500154in}}%
\pgfpathlineto{\pgfqpoint{4.423000in}{0.500154in}}%
\pgfpathlineto{\pgfqpoint{4.381955in}{0.500154in}}%
\pgfpathlineto{\pgfqpoint{4.340909in}{0.500154in}}%
\pgfpathlineto{\pgfqpoint{4.299864in}{0.500154in}}%
\pgfpathlineto{\pgfqpoint{4.258819in}{0.500154in}}%
\pgfpathlineto{\pgfqpoint{4.217773in}{0.500154in}}%
\pgfpathlineto{\pgfqpoint{4.176728in}{0.500154in}}%
\pgfpathlineto{\pgfqpoint{4.135683in}{0.500154in}}%
\pgfpathlineto{\pgfqpoint{4.094637in}{0.500154in}}%
\pgfpathlineto{\pgfqpoint{4.053592in}{0.500154in}}%
\pgfpathlineto{\pgfqpoint{4.012547in}{0.500154in}}%
\pgfpathlineto{\pgfqpoint{3.971501in}{0.500154in}}%
\pgfpathlineto{\pgfqpoint{3.930456in}{0.500154in}}%
\pgfpathlineto{\pgfqpoint{3.889411in}{0.500154in}}%
\pgfpathlineto{\pgfqpoint{3.848365in}{0.500154in}}%
\pgfpathlineto{\pgfqpoint{3.807320in}{0.500154in}}%
\pgfpathlineto{\pgfqpoint{3.766275in}{0.500154in}}%
\pgfpathlineto{\pgfqpoint{3.725229in}{0.500154in}}%
\pgfpathlineto{\pgfqpoint{3.684184in}{0.500154in}}%
\pgfpathlineto{\pgfqpoint{3.683109in}{0.506671in}}%
\pgfpathlineto{\pgfqpoint{3.679691in}{0.527299in}}%
\pgfpathlineto{\pgfqpoint{3.676191in}{0.547927in}}%
\pgfpathlineto{\pgfqpoint{3.672602in}{0.568554in}}%
\pgfpathlineto{\pgfqpoint{3.668916in}{0.589182in}}%
\pgfpathlineto{\pgfqpoint{3.665125in}{0.609810in}}%
\pgfpathlineto{\pgfqpoint{3.661219in}{0.630438in}}%
\pgfpathlineto{\pgfqpoint{3.657186in}{0.651066in}}%
\pgfpathlineto{\pgfqpoint{3.653015in}{0.671694in}}%
\pgfpathlineto{\pgfqpoint{3.648689in}{0.692321in}}%
\pgfpathlineto{\pgfqpoint{3.644194in}{0.712949in}}%
\pgfpathlineto{\pgfqpoint{3.643139in}{0.717728in}}%
\pgfpathlineto{\pgfqpoint{3.602093in}{0.717728in}}%
\pgfpathlineto{\pgfqpoint{3.561048in}{0.717728in}}%
\pgfpathlineto{\pgfqpoint{3.520003in}{0.717728in}}%
\pgfpathlineto{\pgfqpoint{3.478957in}{0.717728in}}%
\pgfpathlineto{\pgfqpoint{3.437912in}{0.717728in}}%
\pgfpathlineto{\pgfqpoint{3.396867in}{0.717728in}}%
\pgfpathlineto{\pgfqpoint{3.355821in}{0.717728in}}%
\pgfpathlineto{\pgfqpoint{3.314776in}{0.717728in}}%
\pgfpathlineto{\pgfqpoint{3.273731in}{0.717728in}}%
\pgfpathlineto{\pgfqpoint{3.232685in}{0.717728in}}%
\pgfpathlineto{\pgfqpoint{3.191640in}{0.717728in}}%
\pgfpathlineto{\pgfqpoint{3.150595in}{0.717728in}}%
\pgfpathlineto{\pgfqpoint{3.109549in}{0.717728in}}%
\pgfpathlineto{\pgfqpoint{3.068504in}{0.717728in}}%
\pgfpathlineto{\pgfqpoint{3.027459in}{0.717728in}}%
\pgfpathlineto{\pgfqpoint{2.986413in}{0.717728in}}%
\pgfpathlineto{\pgfqpoint{2.945368in}{0.717728in}}%
\pgfpathlineto{\pgfqpoint{2.904323in}{0.717728in}}%
\pgfpathlineto{\pgfqpoint{2.863277in}{0.717728in}}%
\pgfpathlineto{\pgfqpoint{2.822232in}{0.717728in}}%
\pgfpathlineto{\pgfqpoint{2.781187in}{0.717728in}}%
\pgfpathlineto{\pgfqpoint{2.740141in}{0.717728in}}%
\pgfpathlineto{\pgfqpoint{2.699096in}{0.717728in}}%
\pgfpathlineto{\pgfqpoint{2.658051in}{0.717728in}}%
\pgfpathlineto{\pgfqpoint{2.617005in}{0.717728in}}%
\pgfpathlineto{\pgfqpoint{2.575960in}{0.717728in}}%
\pgfpathlineto{\pgfqpoint{2.534915in}{0.717728in}}%
\pgfpathlineto{\pgfqpoint{2.533894in}{0.733577in}}%
\pgfpathlineto{\pgfqpoint{2.532560in}{0.754205in}}%
\pgfpathlineto{\pgfqpoint{2.531213in}{0.774833in}}%
\pgfpathlineto{\pgfqpoint{2.529852in}{0.795460in}}%
\pgfpathlineto{\pgfqpoint{2.528478in}{0.816088in}}%
\pgfpathlineto{\pgfqpoint{2.527088in}{0.836716in}}%
\pgfpathlineto{\pgfqpoint{2.525681in}{0.857344in}}%
\pgfpathlineto{\pgfqpoint{2.524258in}{0.877972in}}%
\pgfpathlineto{\pgfqpoint{2.522815in}{0.898600in}}%
\pgfpathlineto{\pgfqpoint{2.521353in}{0.919227in}}%
\pgfpathlineto{\pgfqpoint{2.519869in}{0.939855in}}%
\pgfpathlineto{\pgfqpoint{2.518362in}{0.960483in}}%
\pgfpathlineto{\pgfqpoint{2.516830in}{0.981111in}}%
\pgfpathlineto{\pgfqpoint{2.515271in}{1.001739in}}%
\pgfpathlineto{\pgfqpoint{2.513684in}{1.022367in}}%
\pgfpathlineto{\pgfqpoint{2.512064in}{1.042994in}}%
\pgfpathlineto{\pgfqpoint{2.510411in}{1.063622in}}%
\pgfpathlineto{\pgfqpoint{2.508720in}{1.084250in}}%
\pgfpathlineto{\pgfqpoint{2.506988in}{1.104878in}}%
\pgfpathlineto{\pgfqpoint{2.505212in}{1.125506in}}%
\pgfpathlineto{\pgfqpoint{2.503387in}{1.146134in}}%
\pgfpathlineto{\pgfqpoint{2.501508in}{1.166761in}}%
\pgfpathlineto{\pgfqpoint{2.499569in}{1.187389in}}%
\pgfpathlineto{\pgfqpoint{2.497565in}{1.208017in}}%
\pgfpathlineto{\pgfqpoint{2.495487in}{1.228645in}}%
\pgfpathlineto{\pgfqpoint{2.493869in}{1.244237in}}%
\pgfpathlineto{\pgfqpoint{2.467049in}{1.228645in}}%
\pgfpathlineto{\pgfqpoint{2.452824in}{1.220635in}}%
\pgfpathlineto{\pgfqpoint{2.432731in}{1.208017in}}%
\pgfpathlineto{\pgfqpoint{2.411779in}{1.195447in}}%
\pgfpathlineto{\pgfqpoint{2.397154in}{1.187389in}}%
\pgfpathlineto{\pgfqpoint{2.370733in}{1.173607in}}%
\pgfpathlineto{\pgfqpoint{2.352716in}{1.166761in}}%
\pgfpathlineto{\pgfqpoint{2.329688in}{1.158510in}}%
\pgfpathlineto{\pgfqpoint{2.288643in}{1.151105in}}%
\pgfpathlineto{\pgfqpoint{2.247597in}{1.150083in}}%
\pgfpathlineto{\pgfqpoint{2.206552in}{1.152382in}}%
\pgfpathlineto{\pgfqpoint{2.165507in}{1.154029in}}%
\pgfpathlineto{\pgfqpoint{2.124461in}{1.151205in}}%
\pgfpathlineto{\pgfqpoint{2.103739in}{1.146134in}}%
\pgfpathlineto{\pgfqpoint{2.083416in}{1.141219in}}%
\pgfpathlineto{\pgfqpoint{2.047856in}{1.125506in}}%
\pgfpathlineto{\pgfqpoint{2.042371in}{1.123078in}}%
\pgfpathlineto{\pgfqpoint{2.012637in}{1.104878in}}%
\pgfpathlineto{\pgfqpoint{2.001325in}{1.097854in}}%
\pgfpathlineto{\pgfqpoint{1.982338in}{1.084250in}}%
\pgfpathlineto{\pgfqpoint{1.960280in}{1.068042in}}%
\pgfpathlineto{\pgfqpoint{1.954273in}{1.063622in}}%
\pgfpathlineto{\pgfqpoint{1.926848in}{1.042994in}}%
\pgfpathlineto{\pgfqpoint{1.919235in}{1.037128in}}%
\pgfpathlineto{\pgfqpoint{1.897809in}{1.022367in}}%
\pgfpathlineto{\pgfqpoint{1.878189in}{1.008327in}}%
\pgfpathlineto{\pgfqpoint{1.866851in}{1.001739in}}%
\pgfpathlineto{\pgfqpoint{1.837144in}{0.983790in}}%
\pgfpathlineto{\pgfqpoint{1.831379in}{0.981111in}}%
\pgfpathlineto{\pgfqpoint{1.796099in}{0.964124in}}%
\pgfpathlineto{\pgfqpoint{1.786573in}{0.960483in}}%
\pgfpathlineto{\pgfqpoint{1.755053in}{0.948106in}}%
\pgfpathlineto{\pgfqpoint{1.731911in}{0.939855in}}%
\pgfpathlineto{\pgfqpoint{1.714008in}{0.933381in}}%
\pgfpathlineto{\pgfqpoint{1.676563in}{0.919227in}}%
\pgfpathlineto{\pgfqpoint{1.672963in}{0.917868in}}%
\pgfpathlineto{\pgfqpoint{1.631917in}{0.901221in}}%
\pgfpathlineto{\pgfqpoint{1.625449in}{0.898600in}}%
\pgfpathlineto{\pgfqpoint{1.590872in}{0.885055in}}%
\pgfpathlineto{\pgfqpoint{1.568559in}{0.877972in}}%
\pgfpathlineto{\pgfqpoint{1.549827in}{0.872291in}}%
\pgfpathlineto{\pgfqpoint{1.508781in}{0.865815in}}%
\pgfpathlineto{\pgfqpoint{1.467736in}{0.866971in}}%
\pgfpathlineto{\pgfqpoint{1.426691in}{0.875071in}}%
\pgfpathlineto{\pgfqpoint{1.417145in}{0.877972in}}%
\pgfpathlineto{\pgfqpoint{1.385645in}{0.887577in}}%
\pgfpathlineto{\pgfqpoint{1.349731in}{0.898600in}}%
\pgfpathlineto{\pgfqpoint{1.344600in}{0.900191in}}%
\pgfpathlineto{\pgfqpoint{1.303555in}{0.908372in}}%
\pgfpathlineto{\pgfqpoint{1.262509in}{0.907694in}}%
\pgfpathlineto{\pgfqpoint{1.232099in}{0.898600in}}%
\pgfpathlineto{\pgfqpoint{1.221464in}{0.895447in}}%
\pgfpathlineto{\pgfqpoint{1.191781in}{0.877972in}}%
\pgfpathlineto{\pgfqpoint{1.180419in}{0.871247in}}%
\pgfpathlineto{\pgfqpoint{1.163675in}{0.857344in}}%
\pgfpathlineto{\pgfqpoint{1.139373in}{0.836790in}}%
\pgfpathlineto{\pgfqpoint{1.139298in}{0.836716in}}%
\pgfpathlineto{\pgfqpoint{1.118337in}{0.816088in}}%
\pgfpathlineto{\pgfqpoint{1.098328in}{0.795826in}}%
\pgfpathlineto{\pgfqpoint{1.097974in}{0.795460in}}%
\pgfpathlineto{\pgfqpoint{1.078054in}{0.774833in}}%
\pgfpathlineto{\pgfqpoint{1.058775in}{0.754205in}}%
\pgfpathlineto{\pgfqpoint{1.057283in}{0.752602in}}%
\pgfpathlineto{\pgfqpoint{1.038120in}{0.733577in}}%
\pgfpathlineto{\pgfqpoint{1.018071in}{0.712949in}}%
\pgfpathlineto{\pgfqpoint{1.016237in}{0.711049in}}%
\pgfpathlineto{\pgfqpoint{0.995124in}{0.692321in}}%
\pgfpathlineto{\pgfqpoint{0.975192in}{0.673966in}}%
\pgfpathlineto{\pgfqpoint{0.972150in}{0.671694in}}%
\pgfpathlineto{\pgfqpoint{0.944973in}{0.651066in}}%
\pgfpathlineto{\pgfqpoint{0.934147in}{0.642622in}}%
\pgfpathlineto{\pgfqpoint{0.914586in}{0.630438in}}%
\pgfpathlineto{\pgfqpoint{0.893101in}{0.616752in}}%
\pgfpathlineto{\pgfqpoint{0.879607in}{0.609810in}}%
\pgfpathlineto{\pgfqpoint{0.852056in}{0.595499in}}%
\pgfpathlineto{\pgfqpoint{0.837121in}{0.589182in}}%
\pgfpathlineto{\pgfqpoint{0.811011in}{0.578197in}}%
\pgfpathlineto{\pgfqpoint{0.780958in}{0.568554in}}%
\pgfpathlineto{\pgfqpoint{0.769965in}{0.565099in}}%
\pgfpathlineto{\pgfqpoint{0.728920in}{0.557713in}}%
\pgfpathlineto{\pgfqpoint{0.687875in}{0.557731in}}%
\pgfpathlineto{\pgfqpoint{0.646829in}{0.566334in}}%
\pgfpathclose%
\pgfusepath{stroke,fill}%
\end{pgfscope}%
\begin{pgfscope}%
\pgfpathrectangle{\pgfqpoint{0.605784in}{0.382904in}}{\pgfqpoint{4.063488in}{2.042155in}}%
\pgfusepath{clip}%
\pgfsetbuttcap%
\pgfsetroundjoin%
\definecolor{currentfill}{rgb}{0.195860,0.395433,0.555276}%
\pgfsetfillcolor{currentfill}%
\pgfsetlinewidth{1.003750pt}%
\definecolor{currentstroke}{rgb}{0.195860,0.395433,0.555276}%
\pgfsetstrokecolor{currentstroke}%
\pgfsetdash{}{0pt}%
\pgfpathmoveto{\pgfqpoint{0.615291in}{1.909363in}}%
\pgfpathlineto{\pgfqpoint{0.605784in}{1.914822in}}%
\pgfpathlineto{\pgfqpoint{0.605784in}{1.929991in}}%
\pgfpathlineto{\pgfqpoint{0.605784in}{1.950619in}}%
\pgfpathlineto{\pgfqpoint{0.605784in}{1.971247in}}%
\pgfpathlineto{\pgfqpoint{0.605784in}{1.991874in}}%
\pgfpathlineto{\pgfqpoint{0.605784in}{1.997683in}}%
\pgfpathlineto{\pgfqpoint{0.616074in}{1.991874in}}%
\pgfpathlineto{\pgfqpoint{0.646829in}{1.974503in}}%
\pgfpathlineto{\pgfqpoint{0.652131in}{1.971247in}}%
\pgfpathlineto{\pgfqpoint{0.685495in}{1.950619in}}%
\pgfpathlineto{\pgfqpoint{0.687875in}{1.949135in}}%
\pgfpathlineto{\pgfqpoint{0.723245in}{1.929991in}}%
\pgfpathlineto{\pgfqpoint{0.728920in}{1.926876in}}%
\pgfpathlineto{\pgfqpoint{0.769965in}{1.913499in}}%
\pgfpathlineto{\pgfqpoint{0.811011in}{1.913848in}}%
\pgfpathlineto{\pgfqpoint{0.850462in}{1.929991in}}%
\pgfpathlineto{\pgfqpoint{0.852056in}{1.930640in}}%
\pgfpathlineto{\pgfqpoint{0.877155in}{1.950619in}}%
\pgfpathlineto{\pgfqpoint{0.893101in}{1.963434in}}%
\pgfpathlineto{\pgfqpoint{0.900138in}{1.971247in}}%
\pgfpathlineto{\pgfqpoint{0.918669in}{1.991874in}}%
\pgfpathlineto{\pgfqpoint{0.934147in}{2.009334in}}%
\pgfpathlineto{\pgfqpoint{0.936579in}{2.012502in}}%
\pgfpathlineto{\pgfqpoint{0.952456in}{2.033130in}}%
\pgfpathlineto{\pgfqpoint{0.968056in}{2.053758in}}%
\pgfpathlineto{\pgfqpoint{0.975192in}{2.063234in}}%
\pgfpathlineto{\pgfqpoint{0.983405in}{2.074386in}}%
\pgfpathlineto{\pgfqpoint{0.998466in}{2.095013in}}%
\pgfpathlineto{\pgfqpoint{1.013242in}{2.115641in}}%
\pgfpathlineto{\pgfqpoint{1.016237in}{2.119821in}}%
\pgfpathlineto{\pgfqpoint{1.028870in}{2.136269in}}%
\pgfpathlineto{\pgfqpoint{1.044439in}{2.156897in}}%
\pgfpathlineto{\pgfqpoint{1.057283in}{2.174218in}}%
\pgfpathlineto{\pgfqpoint{1.060112in}{2.177525in}}%
\pgfpathlineto{\pgfqpoint{1.077706in}{2.198153in}}%
\pgfpathlineto{\pgfqpoint{1.094853in}{2.218780in}}%
\pgfpathlineto{\pgfqpoint{1.098328in}{2.222975in}}%
\pgfpathlineto{\pgfqpoint{1.114928in}{2.239408in}}%
\pgfpathlineto{\pgfqpoint{1.135292in}{2.260036in}}%
\pgfpathlineto{\pgfqpoint{1.139373in}{2.264193in}}%
\pgfpathlineto{\pgfqpoint{1.160052in}{2.280664in}}%
\pgfpathlineto{\pgfqpoint{1.180419in}{2.297186in}}%
\pgfpathlineto{\pgfqpoint{1.187252in}{2.301292in}}%
\pgfpathlineto{\pgfqpoint{1.221404in}{2.321920in}}%
\pgfpathlineto{\pgfqpoint{1.221464in}{2.321956in}}%
\pgfpathlineto{\pgfqpoint{1.262509in}{2.338255in}}%
\pgfpathlineto{\pgfqpoint{1.286060in}{2.342547in}}%
\pgfpathlineto{\pgfqpoint{1.303555in}{2.345663in}}%
\pgfpathlineto{\pgfqpoint{1.344600in}{2.343178in}}%
\pgfpathlineto{\pgfqpoint{1.346569in}{2.342547in}}%
\pgfpathlineto{\pgfqpoint{1.385645in}{2.329902in}}%
\pgfpathlineto{\pgfqpoint{1.399461in}{2.321920in}}%
\pgfpathlineto{\pgfqpoint{1.426691in}{2.306132in}}%
\pgfpathlineto{\pgfqpoint{1.432807in}{2.301292in}}%
\pgfpathlineto{\pgfqpoint{1.458704in}{2.280664in}}%
\pgfpathlineto{\pgfqpoint{1.467736in}{2.273436in}}%
\pgfpathlineto{\pgfqpoint{1.482276in}{2.260036in}}%
\pgfpathlineto{\pgfqpoint{1.504560in}{2.239408in}}%
\pgfpathlineto{\pgfqpoint{1.508781in}{2.235455in}}%
\pgfpathlineto{\pgfqpoint{1.526936in}{2.218780in}}%
\pgfpathlineto{\pgfqpoint{1.549224in}{2.198153in}}%
\pgfpathlineto{\pgfqpoint{1.549827in}{2.197585in}}%
\pgfpathlineto{\pgfqpoint{1.576432in}{2.177525in}}%
\pgfpathlineto{\pgfqpoint{1.590872in}{2.166442in}}%
\pgfpathlineto{\pgfqpoint{1.613928in}{2.156897in}}%
\pgfpathlineto{\pgfqpoint{1.631917in}{2.149208in}}%
\pgfpathlineto{\pgfqpoint{1.672963in}{2.151404in}}%
\pgfpathlineto{\pgfqpoint{1.682310in}{2.156897in}}%
\pgfpathlineto{\pgfqpoint{1.714008in}{2.175499in}}%
\pgfpathlineto{\pgfqpoint{1.715875in}{2.177525in}}%
\pgfpathlineto{\pgfqpoint{1.735053in}{2.198153in}}%
\pgfpathlineto{\pgfqpoint{1.754024in}{2.218780in}}%
\pgfpathlineto{\pgfqpoint{1.755053in}{2.219889in}}%
\pgfpathlineto{\pgfqpoint{1.768566in}{2.239408in}}%
\pgfpathlineto{\pgfqpoint{1.782736in}{2.260036in}}%
\pgfpathlineto{\pgfqpoint{1.796099in}{2.279631in}}%
\pgfpathlineto{\pgfqpoint{1.796718in}{2.280664in}}%
\pgfpathlineto{\pgfqpoint{1.809217in}{2.301292in}}%
\pgfpathlineto{\pgfqpoint{1.821603in}{2.321920in}}%
\pgfpathlineto{\pgfqpoint{1.833882in}{2.342547in}}%
\pgfpathlineto{\pgfqpoint{1.837144in}{2.347996in}}%
\pgfpathlineto{\pgfqpoint{1.846088in}{2.363175in}}%
\pgfpathlineto{\pgfqpoint{1.858178in}{2.383803in}}%
\pgfpathlineto{\pgfqpoint{1.870144in}{2.404431in}}%
\pgfpathlineto{\pgfqpoint{1.878189in}{2.418350in}}%
\pgfpathlineto{\pgfqpoint{1.882354in}{2.425059in}}%
\pgfpathlineto{\pgfqpoint{1.919235in}{2.425059in}}%
\pgfpathlineto{\pgfqpoint{1.933387in}{2.425059in}}%
\pgfpathlineto{\pgfqpoint{1.919239in}{2.404431in}}%
\pgfpathlineto{\pgfqpoint{1.919235in}{2.404424in}}%
\pgfpathlineto{\pgfqpoint{1.907170in}{2.383803in}}%
\pgfpathlineto{\pgfqpoint{1.894957in}{2.363175in}}%
\pgfpathlineto{\pgfqpoint{1.882588in}{2.342547in}}%
\pgfpathlineto{\pgfqpoint{1.878189in}{2.335197in}}%
\pgfpathlineto{\pgfqpoint{1.870840in}{2.321920in}}%
\pgfpathlineto{\pgfqpoint{1.859402in}{2.301292in}}%
\pgfpathlineto{\pgfqpoint{1.847852in}{2.280664in}}%
\pgfpathlineto{\pgfqpoint{1.837144in}{2.261700in}}%
\pgfpathlineto{\pgfqpoint{1.836194in}{2.260036in}}%
\pgfpathlineto{\pgfqpoint{1.824558in}{2.239408in}}%
\pgfpathlineto{\pgfqpoint{1.812833in}{2.218780in}}%
\pgfpathlineto{\pgfqpoint{1.801015in}{2.198153in}}%
\pgfpathlineto{\pgfqpoint{1.796099in}{2.189532in}}%
\pgfpathlineto{\pgfqpoint{1.788330in}{2.177525in}}%
\pgfpathlineto{\pgfqpoint{1.775011in}{2.156897in}}%
\pgfpathlineto{\pgfqpoint{1.761595in}{2.136269in}}%
\pgfpathlineto{\pgfqpoint{1.755053in}{2.126170in}}%
\pgfpathlineto{\pgfqpoint{1.745856in}{2.115641in}}%
\pgfpathlineto{\pgfqpoint{1.727872in}{2.095013in}}%
\pgfpathlineto{\pgfqpoint{1.714008in}{2.079180in}}%
\pgfpathlineto{\pgfqpoint{1.706060in}{2.074386in}}%
\pgfpathlineto{\pgfqpoint{1.672963in}{2.054387in}}%
\pgfpathlineto{\pgfqpoint{1.642200in}{2.053758in}}%
\pgfpathlineto{\pgfqpoint{1.631917in}{2.053543in}}%
\pgfpathlineto{\pgfqpoint{1.631496in}{2.053758in}}%
\pgfpathlineto{\pgfqpoint{1.590872in}{2.073701in}}%
\pgfpathlineto{\pgfqpoint{1.590087in}{2.074386in}}%
\pgfpathlineto{\pgfqpoint{1.565764in}{2.095013in}}%
\pgfpathlineto{\pgfqpoint{1.549827in}{2.108254in}}%
\pgfpathlineto{\pgfqpoint{1.542607in}{2.115641in}}%
\pgfpathlineto{\pgfqpoint{1.522094in}{2.136269in}}%
\pgfpathlineto{\pgfqpoint{1.508781in}{2.149484in}}%
\pgfpathlineto{\pgfqpoint{1.501370in}{2.156897in}}%
\pgfpathlineto{\pgfqpoint{1.480499in}{2.177525in}}%
\pgfpathlineto{\pgfqpoint{1.467736in}{2.190046in}}%
\pgfpathlineto{\pgfqpoint{1.458093in}{2.198153in}}%
\pgfpathlineto{\pgfqpoint{1.433416in}{2.218780in}}%
\pgfpathlineto{\pgfqpoint{1.426691in}{2.224358in}}%
\pgfpathlineto{\pgfqpoint{1.401508in}{2.239408in}}%
\pgfpathlineto{\pgfqpoint{1.385645in}{2.248852in}}%
\pgfpathlineto{\pgfqpoint{1.351197in}{2.260036in}}%
\pgfpathlineto{\pgfqpoint{1.344600in}{2.262152in}}%
\pgfpathlineto{\pgfqpoint{1.303555in}{2.264286in}}%
\pgfpathlineto{\pgfqpoint{1.280986in}{2.260036in}}%
\pgfpathlineto{\pgfqpoint{1.262509in}{2.256465in}}%
\pgfpathlineto{\pgfqpoint{1.221464in}{2.239889in}}%
\pgfpathlineto{\pgfqpoint{1.220668in}{2.239408in}}%
\pgfpathlineto{\pgfqpoint{1.186551in}{2.218780in}}%
\pgfpathlineto{\pgfqpoint{1.180419in}{2.215093in}}%
\pgfpathlineto{\pgfqpoint{1.159504in}{2.198153in}}%
\pgfpathlineto{\pgfqpoint{1.139373in}{2.182181in}}%
\pgfpathlineto{\pgfqpoint{1.134811in}{2.177525in}}%
\pgfpathlineto{\pgfqpoint{1.114463in}{2.156897in}}%
\pgfpathlineto{\pgfqpoint{1.098328in}{2.140947in}}%
\pgfpathlineto{\pgfqpoint{1.094488in}{2.136269in}}%
\pgfpathlineto{\pgfqpoint{1.077484in}{2.115641in}}%
\pgfpathlineto{\pgfqpoint{1.059986in}{2.095013in}}%
\pgfpathlineto{\pgfqpoint{1.057283in}{2.091837in}}%
\pgfpathlineto{\pgfqpoint{1.044568in}{2.074386in}}%
\pgfpathlineto{\pgfqpoint{1.029245in}{2.053758in}}%
\pgfpathlineto{\pgfqpoint{1.016237in}{2.036594in}}%
\pgfpathlineto{\pgfqpoint{1.013813in}{2.033130in}}%
\pgfpathlineto{\pgfqpoint{0.999408in}{2.012502in}}%
\pgfpathlineto{\pgfqpoint{0.984707in}{1.991874in}}%
\pgfpathlineto{\pgfqpoint{0.975192in}{1.978675in}}%
\pgfpathlineto{\pgfqpoint{0.969757in}{1.971247in}}%
\pgfpathlineto{\pgfqpoint{0.954647in}{1.950619in}}%
\pgfpathlineto{\pgfqpoint{0.939258in}{1.929991in}}%
\pgfpathlineto{\pgfqpoint{0.934147in}{1.923148in}}%
\pgfpathlineto{\pgfqpoint{0.922298in}{1.909363in}}%
\pgfpathlineto{\pgfqpoint{0.904408in}{1.888735in}}%
\pgfpathlineto{\pgfqpoint{0.893101in}{1.875806in}}%
\pgfpathlineto{\pgfqpoint{0.883757in}{1.868107in}}%
\pgfpathlineto{\pgfqpoint{0.858631in}{1.847480in}}%
\pgfpathlineto{\pgfqpoint{0.852056in}{1.842084in}}%
\pgfpathlineto{\pgfqpoint{0.814499in}{1.826852in}}%
\pgfpathlineto{\pgfqpoint{0.811011in}{1.825428in}}%
\pgfpathlineto{\pgfqpoint{0.769965in}{1.826044in}}%
\pgfpathlineto{\pgfqpoint{0.767792in}{1.826852in}}%
\pgfpathlineto{\pgfqpoint{0.728920in}{1.840867in}}%
\pgfpathlineto{\pgfqpoint{0.717662in}{1.847480in}}%
\pgfpathlineto{\pgfqpoint{0.687875in}{1.864701in}}%
\pgfpathlineto{\pgfqpoint{0.682649in}{1.868107in}}%
\pgfpathlineto{\pgfqpoint{0.650737in}{1.888735in}}%
\pgfpathlineto{\pgfqpoint{0.646829in}{1.891238in}}%
\pgfpathclose%
\pgfusepath{stroke,fill}%
\end{pgfscope}%
\begin{pgfscope}%
\pgfpathrectangle{\pgfqpoint{0.605784in}{0.382904in}}{\pgfqpoint{4.063488in}{2.042155in}}%
\pgfusepath{clip}%
\pgfsetbuttcap%
\pgfsetroundjoin%
\definecolor{currentfill}{rgb}{0.195860,0.395433,0.555276}%
\pgfsetfillcolor{currentfill}%
\pgfsetlinewidth{1.003750pt}%
\definecolor{currentstroke}{rgb}{0.195860,0.395433,0.555276}%
\pgfsetstrokecolor{currentstroke}%
\pgfsetdash{}{0pt}%
\pgfpathmoveto{\pgfqpoint{2.366653in}{2.404431in}}%
\pgfpathlineto{\pgfqpoint{2.352125in}{2.425059in}}%
\pgfpathlineto{\pgfqpoint{2.370733in}{2.425059in}}%
\pgfpathlineto{\pgfqpoint{2.411779in}{2.425059in}}%
\pgfpathlineto{\pgfqpoint{2.425912in}{2.425059in}}%
\pgfpathlineto{\pgfqpoint{2.447388in}{2.404431in}}%
\pgfpathlineto{\pgfqpoint{2.452824in}{2.399066in}}%
\pgfpathlineto{\pgfqpoint{2.485146in}{2.383803in}}%
\pgfpathlineto{\pgfqpoint{2.493869in}{2.379517in}}%
\pgfpathlineto{\pgfqpoint{2.495430in}{2.363175in}}%
\pgfpathlineto{\pgfqpoint{2.497454in}{2.342547in}}%
\pgfpathlineto{\pgfqpoint{2.499554in}{2.321920in}}%
\pgfpathlineto{\pgfqpoint{2.501738in}{2.301292in}}%
\pgfpathlineto{\pgfqpoint{2.504018in}{2.280664in}}%
\pgfpathlineto{\pgfqpoint{2.506405in}{2.260036in}}%
\pgfpathlineto{\pgfqpoint{2.508914in}{2.239408in}}%
\pgfpathlineto{\pgfqpoint{2.511561in}{2.218780in}}%
\pgfpathlineto{\pgfqpoint{2.514368in}{2.198153in}}%
\pgfpathlineto{\pgfqpoint{2.517358in}{2.177525in}}%
\pgfpathlineto{\pgfqpoint{2.520563in}{2.156897in}}%
\pgfpathlineto{\pgfqpoint{2.524018in}{2.136269in}}%
\pgfpathlineto{\pgfqpoint{2.527771in}{2.115641in}}%
\pgfpathlineto{\pgfqpoint{2.531878in}{2.095013in}}%
\pgfpathlineto{\pgfqpoint{2.534915in}{2.080903in}}%
\pgfpathlineto{\pgfqpoint{2.575960in}{2.080903in}}%
\pgfpathlineto{\pgfqpoint{2.617005in}{2.080903in}}%
\pgfpathlineto{\pgfqpoint{2.658051in}{2.080903in}}%
\pgfpathlineto{\pgfqpoint{2.699096in}{2.080903in}}%
\pgfpathlineto{\pgfqpoint{2.740141in}{2.080903in}}%
\pgfpathlineto{\pgfqpoint{2.781187in}{2.080903in}}%
\pgfpathlineto{\pgfqpoint{2.822232in}{2.080903in}}%
\pgfpathlineto{\pgfqpoint{2.863277in}{2.080903in}}%
\pgfpathlineto{\pgfqpoint{2.904323in}{2.080903in}}%
\pgfpathlineto{\pgfqpoint{2.945368in}{2.080903in}}%
\pgfpathlineto{\pgfqpoint{2.986413in}{2.080903in}}%
\pgfpathlineto{\pgfqpoint{3.027459in}{2.080903in}}%
\pgfpathlineto{\pgfqpoint{3.068504in}{2.080903in}}%
\pgfpathlineto{\pgfqpoint{3.109549in}{2.080903in}}%
\pgfpathlineto{\pgfqpoint{3.150595in}{2.080903in}}%
\pgfpathlineto{\pgfqpoint{3.191640in}{2.080903in}}%
\pgfpathlineto{\pgfqpoint{3.232685in}{2.080903in}}%
\pgfpathlineto{\pgfqpoint{3.273731in}{2.080903in}}%
\pgfpathlineto{\pgfqpoint{3.314776in}{2.080903in}}%
\pgfpathlineto{\pgfqpoint{3.355821in}{2.080903in}}%
\pgfpathlineto{\pgfqpoint{3.396867in}{2.080903in}}%
\pgfpathlineto{\pgfqpoint{3.437912in}{2.080903in}}%
\pgfpathlineto{\pgfqpoint{3.478957in}{2.080903in}}%
\pgfpathlineto{\pgfqpoint{3.520003in}{2.080903in}}%
\pgfpathlineto{\pgfqpoint{3.561048in}{2.080903in}}%
\pgfpathlineto{\pgfqpoint{3.602093in}{2.080903in}}%
\pgfpathlineto{\pgfqpoint{3.643139in}{2.080903in}}%
\pgfpathlineto{\pgfqpoint{3.644214in}{2.074386in}}%
\pgfpathlineto{\pgfqpoint{3.647632in}{2.053758in}}%
\pgfpathlineto{\pgfqpoint{3.651132in}{2.033130in}}%
\pgfpathlineto{\pgfqpoint{3.654721in}{2.012502in}}%
\pgfpathlineto{\pgfqpoint{3.658406in}{1.991874in}}%
\pgfpathlineto{\pgfqpoint{3.662197in}{1.971247in}}%
\pgfpathlineto{\pgfqpoint{3.666103in}{1.950619in}}%
\pgfpathlineto{\pgfqpoint{3.670136in}{1.929991in}}%
\pgfpathlineto{\pgfqpoint{3.674308in}{1.909363in}}%
\pgfpathlineto{\pgfqpoint{3.678633in}{1.888735in}}%
\pgfpathlineto{\pgfqpoint{3.683129in}{1.868107in}}%
\pgfpathlineto{\pgfqpoint{3.684184in}{1.863329in}}%
\pgfpathlineto{\pgfqpoint{3.725229in}{1.863329in}}%
\pgfpathlineto{\pgfqpoint{3.766275in}{1.863329in}}%
\pgfpathlineto{\pgfqpoint{3.807320in}{1.863329in}}%
\pgfpathlineto{\pgfqpoint{3.848365in}{1.863329in}}%
\pgfpathlineto{\pgfqpoint{3.889411in}{1.863329in}}%
\pgfpathlineto{\pgfqpoint{3.930456in}{1.863329in}}%
\pgfpathlineto{\pgfqpoint{3.971501in}{1.863329in}}%
\pgfpathlineto{\pgfqpoint{4.012547in}{1.863329in}}%
\pgfpathlineto{\pgfqpoint{4.053592in}{1.863329in}}%
\pgfpathlineto{\pgfqpoint{4.094637in}{1.863329in}}%
\pgfpathlineto{\pgfqpoint{4.135683in}{1.863329in}}%
\pgfpathlineto{\pgfqpoint{4.176728in}{1.863329in}}%
\pgfpathlineto{\pgfqpoint{4.217773in}{1.863329in}}%
\pgfpathlineto{\pgfqpoint{4.258819in}{1.863329in}}%
\pgfpathlineto{\pgfqpoint{4.299864in}{1.863329in}}%
\pgfpathlineto{\pgfqpoint{4.340909in}{1.863329in}}%
\pgfpathlineto{\pgfqpoint{4.381955in}{1.863329in}}%
\pgfpathlineto{\pgfqpoint{4.423000in}{1.863329in}}%
\pgfpathlineto{\pgfqpoint{4.464045in}{1.863329in}}%
\pgfpathlineto{\pgfqpoint{4.505091in}{1.863329in}}%
\pgfpathlineto{\pgfqpoint{4.546136in}{1.863329in}}%
\pgfpathlineto{\pgfqpoint{4.587181in}{1.863329in}}%
\pgfpathlineto{\pgfqpoint{4.628227in}{1.863329in}}%
\pgfpathlineto{\pgfqpoint{4.669272in}{1.863329in}}%
\pgfpathlineto{\pgfqpoint{4.669272in}{1.847480in}}%
\pgfpathlineto{\pgfqpoint{4.669272in}{1.826852in}}%
\pgfpathlineto{\pgfqpoint{4.669272in}{1.806224in}}%
\pgfpathlineto{\pgfqpoint{4.669272in}{1.785596in}}%
\pgfpathlineto{\pgfqpoint{4.669272in}{1.776894in}}%
\pgfpathlineto{\pgfqpoint{4.628227in}{1.776894in}}%
\pgfpathlineto{\pgfqpoint{4.587181in}{1.776894in}}%
\pgfpathlineto{\pgfqpoint{4.546136in}{1.776894in}}%
\pgfpathlineto{\pgfqpoint{4.505091in}{1.776894in}}%
\pgfpathlineto{\pgfqpoint{4.464045in}{1.776894in}}%
\pgfpathlineto{\pgfqpoint{4.423000in}{1.776894in}}%
\pgfpathlineto{\pgfqpoint{4.381955in}{1.776894in}}%
\pgfpathlineto{\pgfqpoint{4.340909in}{1.776894in}}%
\pgfpathlineto{\pgfqpoint{4.299864in}{1.776894in}}%
\pgfpathlineto{\pgfqpoint{4.258819in}{1.776894in}}%
\pgfpathlineto{\pgfqpoint{4.217773in}{1.776894in}}%
\pgfpathlineto{\pgfqpoint{4.176728in}{1.776894in}}%
\pgfpathlineto{\pgfqpoint{4.135683in}{1.776894in}}%
\pgfpathlineto{\pgfqpoint{4.094637in}{1.776894in}}%
\pgfpathlineto{\pgfqpoint{4.053592in}{1.776894in}}%
\pgfpathlineto{\pgfqpoint{4.012547in}{1.776894in}}%
\pgfpathlineto{\pgfqpoint{3.971501in}{1.776894in}}%
\pgfpathlineto{\pgfqpoint{3.930456in}{1.776894in}}%
\pgfpathlineto{\pgfqpoint{3.889411in}{1.776894in}}%
\pgfpathlineto{\pgfqpoint{3.848365in}{1.776894in}}%
\pgfpathlineto{\pgfqpoint{3.807320in}{1.776894in}}%
\pgfpathlineto{\pgfqpoint{3.766275in}{1.776894in}}%
\pgfpathlineto{\pgfqpoint{3.725229in}{1.776894in}}%
\pgfpathlineto{\pgfqpoint{3.684184in}{1.776894in}}%
\pgfpathlineto{\pgfqpoint{3.682216in}{1.785596in}}%
\pgfpathlineto{\pgfqpoint{3.677660in}{1.806224in}}%
\pgfpathlineto{\pgfqpoint{3.673305in}{1.826852in}}%
\pgfpathlineto{\pgfqpoint{3.669129in}{1.847480in}}%
\pgfpathlineto{\pgfqpoint{3.665112in}{1.868107in}}%
\pgfpathlineto{\pgfqpoint{3.661238in}{1.888735in}}%
\pgfpathlineto{\pgfqpoint{3.657492in}{1.909363in}}%
\pgfpathlineto{\pgfqpoint{3.653863in}{1.929991in}}%
\pgfpathlineto{\pgfqpoint{3.650339in}{1.950619in}}%
\pgfpathlineto{\pgfqpoint{3.646910in}{1.971247in}}%
\pgfpathlineto{\pgfqpoint{3.643569in}{1.991874in}}%
\pgfpathlineto{\pgfqpoint{3.643139in}{1.994527in}}%
\pgfpathlineto{\pgfqpoint{3.602093in}{1.994527in}}%
\pgfpathlineto{\pgfqpoint{3.561048in}{1.994527in}}%
\pgfpathlineto{\pgfqpoint{3.520003in}{1.994527in}}%
\pgfpathlineto{\pgfqpoint{3.478957in}{1.994527in}}%
\pgfpathlineto{\pgfqpoint{3.437912in}{1.994527in}}%
\pgfpathlineto{\pgfqpoint{3.396867in}{1.994527in}}%
\pgfpathlineto{\pgfqpoint{3.355821in}{1.994527in}}%
\pgfpathlineto{\pgfqpoint{3.314776in}{1.994527in}}%
\pgfpathlineto{\pgfqpoint{3.273731in}{1.994527in}}%
\pgfpathlineto{\pgfqpoint{3.232685in}{1.994527in}}%
\pgfpathlineto{\pgfqpoint{3.191640in}{1.994527in}}%
\pgfpathlineto{\pgfqpoint{3.150595in}{1.994527in}}%
\pgfpathlineto{\pgfqpoint{3.109549in}{1.994527in}}%
\pgfpathlineto{\pgfqpoint{3.068504in}{1.994527in}}%
\pgfpathlineto{\pgfqpoint{3.027459in}{1.994527in}}%
\pgfpathlineto{\pgfqpoint{2.986413in}{1.994527in}}%
\pgfpathlineto{\pgfqpoint{2.945368in}{1.994527in}}%
\pgfpathlineto{\pgfqpoint{2.904323in}{1.994527in}}%
\pgfpathlineto{\pgfqpoint{2.863277in}{1.994527in}}%
\pgfpathlineto{\pgfqpoint{2.822232in}{1.994527in}}%
\pgfpathlineto{\pgfqpoint{2.781187in}{1.994527in}}%
\pgfpathlineto{\pgfqpoint{2.740141in}{1.994527in}}%
\pgfpathlineto{\pgfqpoint{2.699096in}{1.994527in}}%
\pgfpathlineto{\pgfqpoint{2.658051in}{1.994527in}}%
\pgfpathlineto{\pgfqpoint{2.617005in}{1.994527in}}%
\pgfpathlineto{\pgfqpoint{2.575960in}{1.994527in}}%
\pgfpathlineto{\pgfqpoint{2.534915in}{1.994527in}}%
\pgfpathlineto{\pgfqpoint{2.530314in}{2.012502in}}%
\pgfpathlineto{\pgfqpoint{2.525669in}{2.033130in}}%
\pgfpathlineto{\pgfqpoint{2.521565in}{2.053758in}}%
\pgfpathlineto{\pgfqpoint{2.517889in}{2.074386in}}%
\pgfpathlineto{\pgfqpoint{2.514559in}{2.095013in}}%
\pgfpathlineto{\pgfqpoint{2.511510in}{2.115641in}}%
\pgfpathlineto{\pgfqpoint{2.508694in}{2.136269in}}%
\pgfpathlineto{\pgfqpoint{2.506073in}{2.156897in}}%
\pgfpathlineto{\pgfqpoint{2.503617in}{2.177525in}}%
\pgfpathlineto{\pgfqpoint{2.501301in}{2.198153in}}%
\pgfpathlineto{\pgfqpoint{2.499106in}{2.218780in}}%
\pgfpathlineto{\pgfqpoint{2.497016in}{2.239408in}}%
\pgfpathlineto{\pgfqpoint{2.495016in}{2.260036in}}%
\pgfpathlineto{\pgfqpoint{2.493869in}{2.272154in}}%
\pgfpathlineto{\pgfqpoint{2.479986in}{2.280664in}}%
\pgfpathlineto{\pgfqpoint{2.452824in}{2.296528in}}%
\pgfpathlineto{\pgfqpoint{2.448723in}{2.301292in}}%
\pgfpathlineto{\pgfqpoint{2.430347in}{2.321920in}}%
\pgfpathlineto{\pgfqpoint{2.411779in}{2.342185in}}%
\pgfpathlineto{\pgfqpoint{2.411526in}{2.342547in}}%
\pgfpathlineto{\pgfqpoint{2.396754in}{2.363175in}}%
\pgfpathlineto{\pgfqpoint{2.381697in}{2.383803in}}%
\pgfpathlineto{\pgfqpoint{2.370733in}{2.398514in}}%
\pgfpathclose%
\pgfusepath{stroke,fill}%
\end{pgfscope}%
\begin{pgfscope}%
\pgfpathrectangle{\pgfqpoint{0.605784in}{0.382904in}}{\pgfqpoint{4.063488in}{2.042155in}}%
\pgfusepath{clip}%
\pgfsetbuttcap%
\pgfsetroundjoin%
\definecolor{currentfill}{rgb}{0.166617,0.463708,0.558119}%
\pgfsetfillcolor{currentfill}%
\pgfsetlinewidth{1.003750pt}%
\definecolor{currentstroke}{rgb}{0.166617,0.463708,0.558119}%
\pgfsetstrokecolor{currentstroke}%
\pgfsetdash{}{0pt}%
\pgfpathmoveto{\pgfqpoint{0.611848in}{0.506671in}}%
\pgfpathlineto{\pgfqpoint{0.605784in}{0.509225in}}%
\pgfpathlineto{\pgfqpoint{0.605784in}{0.527299in}}%
\pgfpathlineto{\pgfqpoint{0.605784in}{0.547927in}}%
\pgfpathlineto{\pgfqpoint{0.605784in}{0.568554in}}%
\pgfpathlineto{\pgfqpoint{0.605784in}{0.583374in}}%
\pgfpathlineto{\pgfqpoint{0.641478in}{0.568554in}}%
\pgfpathlineto{\pgfqpoint{0.646829in}{0.566334in}}%
\pgfpathlineto{\pgfqpoint{0.687875in}{0.557731in}}%
\pgfpathlineto{\pgfqpoint{0.728920in}{0.557713in}}%
\pgfpathlineto{\pgfqpoint{0.769965in}{0.565099in}}%
\pgfpathlineto{\pgfqpoint{0.780958in}{0.568554in}}%
\pgfpathlineto{\pgfqpoint{0.811011in}{0.578197in}}%
\pgfpathlineto{\pgfqpoint{0.837121in}{0.589182in}}%
\pgfpathlineto{\pgfqpoint{0.852056in}{0.595499in}}%
\pgfpathlineto{\pgfqpoint{0.879607in}{0.609810in}}%
\pgfpathlineto{\pgfqpoint{0.893101in}{0.616752in}}%
\pgfpathlineto{\pgfqpoint{0.914586in}{0.630438in}}%
\pgfpathlineto{\pgfqpoint{0.934147in}{0.642622in}}%
\pgfpathlineto{\pgfqpoint{0.944973in}{0.651066in}}%
\pgfpathlineto{\pgfqpoint{0.972150in}{0.671694in}}%
\pgfpathlineto{\pgfqpoint{0.975192in}{0.673966in}}%
\pgfpathlineto{\pgfqpoint{0.995124in}{0.692321in}}%
\pgfpathlineto{\pgfqpoint{1.016237in}{0.711049in}}%
\pgfpathlineto{\pgfqpoint{1.018071in}{0.712949in}}%
\pgfpathlineto{\pgfqpoint{1.038120in}{0.733577in}}%
\pgfpathlineto{\pgfqpoint{1.057283in}{0.752602in}}%
\pgfpathlineto{\pgfqpoint{1.058775in}{0.754205in}}%
\pgfpathlineto{\pgfqpoint{1.078054in}{0.774833in}}%
\pgfpathlineto{\pgfqpoint{1.097974in}{0.795460in}}%
\pgfpathlineto{\pgfqpoint{1.098328in}{0.795826in}}%
\pgfpathlineto{\pgfqpoint{1.118337in}{0.816088in}}%
\pgfpathlineto{\pgfqpoint{1.139298in}{0.836716in}}%
\pgfpathlineto{\pgfqpoint{1.139373in}{0.836790in}}%
\pgfpathlineto{\pgfqpoint{1.163675in}{0.857344in}}%
\pgfpathlineto{\pgfqpoint{1.180419in}{0.871247in}}%
\pgfpathlineto{\pgfqpoint{1.191781in}{0.877972in}}%
\pgfpathlineto{\pgfqpoint{1.221464in}{0.895447in}}%
\pgfpathlineto{\pgfqpoint{1.232099in}{0.898600in}}%
\pgfpathlineto{\pgfqpoint{1.262509in}{0.907694in}}%
\pgfpathlineto{\pgfqpoint{1.303555in}{0.908372in}}%
\pgfpathlineto{\pgfqpoint{1.344600in}{0.900191in}}%
\pgfpathlineto{\pgfqpoint{1.349731in}{0.898600in}}%
\pgfpathlineto{\pgfqpoint{1.385645in}{0.887577in}}%
\pgfpathlineto{\pgfqpoint{1.417145in}{0.877972in}}%
\pgfpathlineto{\pgfqpoint{1.426691in}{0.875071in}}%
\pgfpathlineto{\pgfqpoint{1.467736in}{0.866971in}}%
\pgfpathlineto{\pgfqpoint{1.508781in}{0.865815in}}%
\pgfpathlineto{\pgfqpoint{1.549827in}{0.872291in}}%
\pgfpathlineto{\pgfqpoint{1.568559in}{0.877972in}}%
\pgfpathlineto{\pgfqpoint{1.590872in}{0.885055in}}%
\pgfpathlineto{\pgfqpoint{1.625449in}{0.898600in}}%
\pgfpathlineto{\pgfqpoint{1.631917in}{0.901221in}}%
\pgfpathlineto{\pgfqpoint{1.672963in}{0.917868in}}%
\pgfpathlineto{\pgfqpoint{1.676563in}{0.919227in}}%
\pgfpathlineto{\pgfqpoint{1.714008in}{0.933381in}}%
\pgfpathlineto{\pgfqpoint{1.731911in}{0.939855in}}%
\pgfpathlineto{\pgfqpoint{1.755053in}{0.948106in}}%
\pgfpathlineto{\pgfqpoint{1.786573in}{0.960483in}}%
\pgfpathlineto{\pgfqpoint{1.796099in}{0.964124in}}%
\pgfpathlineto{\pgfqpoint{1.831379in}{0.981111in}}%
\pgfpathlineto{\pgfqpoint{1.837144in}{0.983790in}}%
\pgfpathlineto{\pgfqpoint{1.866851in}{1.001739in}}%
\pgfpathlineto{\pgfqpoint{1.878189in}{1.008327in}}%
\pgfpathlineto{\pgfqpoint{1.897809in}{1.022367in}}%
\pgfpathlineto{\pgfqpoint{1.919235in}{1.037128in}}%
\pgfpathlineto{\pgfqpoint{1.926848in}{1.042994in}}%
\pgfpathlineto{\pgfqpoint{1.954273in}{1.063622in}}%
\pgfpathlineto{\pgfqpoint{1.960280in}{1.068042in}}%
\pgfpathlineto{\pgfqpoint{1.982338in}{1.084250in}}%
\pgfpathlineto{\pgfqpoint{2.001325in}{1.097854in}}%
\pgfpathlineto{\pgfqpoint{2.012637in}{1.104878in}}%
\pgfpathlineto{\pgfqpoint{2.042371in}{1.123078in}}%
\pgfpathlineto{\pgfqpoint{2.047856in}{1.125506in}}%
\pgfpathlineto{\pgfqpoint{2.083416in}{1.141219in}}%
\pgfpathlineto{\pgfqpoint{2.103739in}{1.146134in}}%
\pgfpathlineto{\pgfqpoint{2.124461in}{1.151205in}}%
\pgfpathlineto{\pgfqpoint{2.165507in}{1.154029in}}%
\pgfpathlineto{\pgfqpoint{2.206552in}{1.152382in}}%
\pgfpathlineto{\pgfqpoint{2.247597in}{1.150083in}}%
\pgfpathlineto{\pgfqpoint{2.288643in}{1.151105in}}%
\pgfpathlineto{\pgfqpoint{2.329688in}{1.158510in}}%
\pgfpathlineto{\pgfqpoint{2.352716in}{1.166761in}}%
\pgfpathlineto{\pgfqpoint{2.370733in}{1.173607in}}%
\pgfpathlineto{\pgfqpoint{2.397154in}{1.187389in}}%
\pgfpathlineto{\pgfqpoint{2.411779in}{1.195447in}}%
\pgfpathlineto{\pgfqpoint{2.432731in}{1.208017in}}%
\pgfpathlineto{\pgfqpoint{2.452824in}{1.220635in}}%
\pgfpathlineto{\pgfqpoint{2.467049in}{1.228645in}}%
\pgfpathlineto{\pgfqpoint{2.493869in}{1.244237in}}%
\pgfpathlineto{\pgfqpoint{2.495487in}{1.228645in}}%
\pgfpathlineto{\pgfqpoint{2.497565in}{1.208017in}}%
\pgfpathlineto{\pgfqpoint{2.499569in}{1.187389in}}%
\pgfpathlineto{\pgfqpoint{2.501508in}{1.166761in}}%
\pgfpathlineto{\pgfqpoint{2.503387in}{1.146134in}}%
\pgfpathlineto{\pgfqpoint{2.505212in}{1.125506in}}%
\pgfpathlineto{\pgfqpoint{2.506988in}{1.104878in}}%
\pgfpathlineto{\pgfqpoint{2.508720in}{1.084250in}}%
\pgfpathlineto{\pgfqpoint{2.510411in}{1.063622in}}%
\pgfpathlineto{\pgfqpoint{2.512064in}{1.042994in}}%
\pgfpathlineto{\pgfqpoint{2.513684in}{1.022367in}}%
\pgfpathlineto{\pgfqpoint{2.515271in}{1.001739in}}%
\pgfpathlineto{\pgfqpoint{2.516830in}{0.981111in}}%
\pgfpathlineto{\pgfqpoint{2.518362in}{0.960483in}}%
\pgfpathlineto{\pgfqpoint{2.519869in}{0.939855in}}%
\pgfpathlineto{\pgfqpoint{2.521353in}{0.919227in}}%
\pgfpathlineto{\pgfqpoint{2.522815in}{0.898600in}}%
\pgfpathlineto{\pgfqpoint{2.524258in}{0.877972in}}%
\pgfpathlineto{\pgfqpoint{2.525681in}{0.857344in}}%
\pgfpathlineto{\pgfqpoint{2.527088in}{0.836716in}}%
\pgfpathlineto{\pgfqpoint{2.528478in}{0.816088in}}%
\pgfpathlineto{\pgfqpoint{2.529852in}{0.795460in}}%
\pgfpathlineto{\pgfqpoint{2.531213in}{0.774833in}}%
\pgfpathlineto{\pgfqpoint{2.532560in}{0.754205in}}%
\pgfpathlineto{\pgfqpoint{2.533894in}{0.733577in}}%
\pgfpathlineto{\pgfqpoint{2.534915in}{0.717728in}}%
\pgfpathlineto{\pgfqpoint{2.575960in}{0.717728in}}%
\pgfpathlineto{\pgfqpoint{2.617005in}{0.717728in}}%
\pgfpathlineto{\pgfqpoint{2.658051in}{0.717728in}}%
\pgfpathlineto{\pgfqpoint{2.699096in}{0.717728in}}%
\pgfpathlineto{\pgfqpoint{2.740141in}{0.717728in}}%
\pgfpathlineto{\pgfqpoint{2.781187in}{0.717728in}}%
\pgfpathlineto{\pgfqpoint{2.822232in}{0.717728in}}%
\pgfpathlineto{\pgfqpoint{2.863277in}{0.717728in}}%
\pgfpathlineto{\pgfqpoint{2.904323in}{0.717728in}}%
\pgfpathlineto{\pgfqpoint{2.945368in}{0.717728in}}%
\pgfpathlineto{\pgfqpoint{2.986413in}{0.717728in}}%
\pgfpathlineto{\pgfqpoint{3.027459in}{0.717728in}}%
\pgfpathlineto{\pgfqpoint{3.068504in}{0.717728in}}%
\pgfpathlineto{\pgfqpoint{3.109549in}{0.717728in}}%
\pgfpathlineto{\pgfqpoint{3.150595in}{0.717728in}}%
\pgfpathlineto{\pgfqpoint{3.191640in}{0.717728in}}%
\pgfpathlineto{\pgfqpoint{3.232685in}{0.717728in}}%
\pgfpathlineto{\pgfqpoint{3.273731in}{0.717728in}}%
\pgfpathlineto{\pgfqpoint{3.314776in}{0.717728in}}%
\pgfpathlineto{\pgfqpoint{3.355821in}{0.717728in}}%
\pgfpathlineto{\pgfqpoint{3.396867in}{0.717728in}}%
\pgfpathlineto{\pgfqpoint{3.437912in}{0.717728in}}%
\pgfpathlineto{\pgfqpoint{3.478957in}{0.717728in}}%
\pgfpathlineto{\pgfqpoint{3.520003in}{0.717728in}}%
\pgfpathlineto{\pgfqpoint{3.561048in}{0.717728in}}%
\pgfpathlineto{\pgfqpoint{3.602093in}{0.717728in}}%
\pgfpathlineto{\pgfqpoint{3.643139in}{0.717728in}}%
\pgfpathlineto{\pgfqpoint{3.644194in}{0.712949in}}%
\pgfpathlineto{\pgfqpoint{3.648689in}{0.692321in}}%
\pgfpathlineto{\pgfqpoint{3.653015in}{0.671694in}}%
\pgfpathlineto{\pgfqpoint{3.657186in}{0.651066in}}%
\pgfpathlineto{\pgfqpoint{3.661219in}{0.630438in}}%
\pgfpathlineto{\pgfqpoint{3.665125in}{0.609810in}}%
\pgfpathlineto{\pgfqpoint{3.668916in}{0.589182in}}%
\pgfpathlineto{\pgfqpoint{3.672602in}{0.568554in}}%
\pgfpathlineto{\pgfqpoint{3.676191in}{0.547927in}}%
\pgfpathlineto{\pgfqpoint{3.679691in}{0.527299in}}%
\pgfpathlineto{\pgfqpoint{3.683109in}{0.506671in}}%
\pgfpathlineto{\pgfqpoint{3.684184in}{0.500154in}}%
\pgfpathlineto{\pgfqpoint{3.725229in}{0.500154in}}%
\pgfpathlineto{\pgfqpoint{3.766275in}{0.500154in}}%
\pgfpathlineto{\pgfqpoint{3.807320in}{0.500154in}}%
\pgfpathlineto{\pgfqpoint{3.848365in}{0.500154in}}%
\pgfpathlineto{\pgfqpoint{3.889411in}{0.500154in}}%
\pgfpathlineto{\pgfqpoint{3.930456in}{0.500154in}}%
\pgfpathlineto{\pgfqpoint{3.971501in}{0.500154in}}%
\pgfpathlineto{\pgfqpoint{4.012547in}{0.500154in}}%
\pgfpathlineto{\pgfqpoint{4.053592in}{0.500154in}}%
\pgfpathlineto{\pgfqpoint{4.094637in}{0.500154in}}%
\pgfpathlineto{\pgfqpoint{4.135683in}{0.500154in}}%
\pgfpathlineto{\pgfqpoint{4.176728in}{0.500154in}}%
\pgfpathlineto{\pgfqpoint{4.217773in}{0.500154in}}%
\pgfpathlineto{\pgfqpoint{4.258819in}{0.500154in}}%
\pgfpathlineto{\pgfqpoint{4.299864in}{0.500154in}}%
\pgfpathlineto{\pgfqpoint{4.340909in}{0.500154in}}%
\pgfpathlineto{\pgfqpoint{4.381955in}{0.500154in}}%
\pgfpathlineto{\pgfqpoint{4.423000in}{0.500154in}}%
\pgfpathlineto{\pgfqpoint{4.464045in}{0.500154in}}%
\pgfpathlineto{\pgfqpoint{4.505091in}{0.500154in}}%
\pgfpathlineto{\pgfqpoint{4.546136in}{0.500154in}}%
\pgfpathlineto{\pgfqpoint{4.587181in}{0.500154in}}%
\pgfpathlineto{\pgfqpoint{4.628227in}{0.500154in}}%
\pgfpathlineto{\pgfqpoint{4.669272in}{0.500154in}}%
\pgfpathlineto{\pgfqpoint{4.669272in}{0.486043in}}%
\pgfpathlineto{\pgfqpoint{4.669272in}{0.465415in}}%
\pgfpathlineto{\pgfqpoint{4.669272in}{0.444787in}}%
\pgfpathlineto{\pgfqpoint{4.669272in}{0.424160in}}%
\pgfpathlineto{\pgfqpoint{4.669272in}{0.423474in}}%
\pgfpathlineto{\pgfqpoint{4.628227in}{0.423474in}}%
\pgfpathlineto{\pgfqpoint{4.587181in}{0.423474in}}%
\pgfpathlineto{\pgfqpoint{4.546136in}{0.423474in}}%
\pgfpathlineto{\pgfqpoint{4.505091in}{0.423474in}}%
\pgfpathlineto{\pgfqpoint{4.464045in}{0.423474in}}%
\pgfpathlineto{\pgfqpoint{4.423000in}{0.423474in}}%
\pgfpathlineto{\pgfqpoint{4.381955in}{0.423474in}}%
\pgfpathlineto{\pgfqpoint{4.340909in}{0.423474in}}%
\pgfpathlineto{\pgfqpoint{4.299864in}{0.423474in}}%
\pgfpathlineto{\pgfqpoint{4.258819in}{0.423474in}}%
\pgfpathlineto{\pgfqpoint{4.217773in}{0.423474in}}%
\pgfpathlineto{\pgfqpoint{4.176728in}{0.423474in}}%
\pgfpathlineto{\pgfqpoint{4.135683in}{0.423474in}}%
\pgfpathlineto{\pgfqpoint{4.094637in}{0.423474in}}%
\pgfpathlineto{\pgfqpoint{4.053592in}{0.423474in}}%
\pgfpathlineto{\pgfqpoint{4.012547in}{0.423474in}}%
\pgfpathlineto{\pgfqpoint{3.971501in}{0.423474in}}%
\pgfpathlineto{\pgfqpoint{3.930456in}{0.423474in}}%
\pgfpathlineto{\pgfqpoint{3.889411in}{0.423474in}}%
\pgfpathlineto{\pgfqpoint{3.848365in}{0.423474in}}%
\pgfpathlineto{\pgfqpoint{3.807320in}{0.423474in}}%
\pgfpathlineto{\pgfqpoint{3.766275in}{0.423474in}}%
\pgfpathlineto{\pgfqpoint{3.725229in}{0.423474in}}%
\pgfpathlineto{\pgfqpoint{3.684184in}{0.423474in}}%
\pgfpathlineto{\pgfqpoint{3.684069in}{0.424160in}}%
\pgfpathlineto{\pgfqpoint{3.680628in}{0.444787in}}%
\pgfpathlineto{\pgfqpoint{3.677113in}{0.465415in}}%
\pgfpathlineto{\pgfqpoint{3.673516in}{0.486043in}}%
\pgfpathlineto{\pgfqpoint{3.669833in}{0.506671in}}%
\pgfpathlineto{\pgfqpoint{3.666056in}{0.527299in}}%
\pgfpathlineto{\pgfqpoint{3.662178in}{0.547927in}}%
\pgfpathlineto{\pgfqpoint{3.658188in}{0.568554in}}%
\pgfpathlineto{\pgfqpoint{3.654079in}{0.589182in}}%
\pgfpathlineto{\pgfqpoint{3.649838in}{0.609810in}}%
\pgfpathlineto{\pgfqpoint{3.645454in}{0.630438in}}%
\pgfpathlineto{\pgfqpoint{3.643139in}{0.641121in}}%
\pgfpathlineto{\pgfqpoint{3.602093in}{0.641121in}}%
\pgfpathlineto{\pgfqpoint{3.561048in}{0.641121in}}%
\pgfpathlineto{\pgfqpoint{3.520003in}{0.641121in}}%
\pgfpathlineto{\pgfqpoint{3.478957in}{0.641121in}}%
\pgfpathlineto{\pgfqpoint{3.437912in}{0.641121in}}%
\pgfpathlineto{\pgfqpoint{3.396867in}{0.641121in}}%
\pgfpathlineto{\pgfqpoint{3.355821in}{0.641121in}}%
\pgfpathlineto{\pgfqpoint{3.314776in}{0.641121in}}%
\pgfpathlineto{\pgfqpoint{3.273731in}{0.641121in}}%
\pgfpathlineto{\pgfqpoint{3.232685in}{0.641121in}}%
\pgfpathlineto{\pgfqpoint{3.191640in}{0.641121in}}%
\pgfpathlineto{\pgfqpoint{3.150595in}{0.641121in}}%
\pgfpathlineto{\pgfqpoint{3.109549in}{0.641121in}}%
\pgfpathlineto{\pgfqpoint{3.068504in}{0.641121in}}%
\pgfpathlineto{\pgfqpoint{3.027459in}{0.641121in}}%
\pgfpathlineto{\pgfqpoint{2.986413in}{0.641121in}}%
\pgfpathlineto{\pgfqpoint{2.945368in}{0.641121in}}%
\pgfpathlineto{\pgfqpoint{2.904323in}{0.641121in}}%
\pgfpathlineto{\pgfqpoint{2.863277in}{0.641121in}}%
\pgfpathlineto{\pgfqpoint{2.822232in}{0.641121in}}%
\pgfpathlineto{\pgfqpoint{2.781187in}{0.641121in}}%
\pgfpathlineto{\pgfqpoint{2.740141in}{0.641121in}}%
\pgfpathlineto{\pgfqpoint{2.699096in}{0.641121in}}%
\pgfpathlineto{\pgfqpoint{2.658051in}{0.641121in}}%
\pgfpathlineto{\pgfqpoint{2.617005in}{0.641121in}}%
\pgfpathlineto{\pgfqpoint{2.575960in}{0.641121in}}%
\pgfpathlineto{\pgfqpoint{2.534915in}{0.641121in}}%
\pgfpathlineto{\pgfqpoint{2.534249in}{0.651066in}}%
\pgfpathlineto{\pgfqpoint{2.532867in}{0.671694in}}%
\pgfpathlineto{\pgfqpoint{2.531473in}{0.692321in}}%
\pgfpathlineto{\pgfqpoint{2.530064in}{0.712949in}}%
\pgfpathlineto{\pgfqpoint{2.528640in}{0.733577in}}%
\pgfpathlineto{\pgfqpoint{2.527200in}{0.754205in}}%
\pgfpathlineto{\pgfqpoint{2.525742in}{0.774833in}}%
\pgfpathlineto{\pgfqpoint{2.524267in}{0.795460in}}%
\pgfpathlineto{\pgfqpoint{2.522773in}{0.816088in}}%
\pgfpathlineto{\pgfqpoint{2.521258in}{0.836716in}}%
\pgfpathlineto{\pgfqpoint{2.519721in}{0.857344in}}%
\pgfpathlineto{\pgfqpoint{2.518160in}{0.877972in}}%
\pgfpathlineto{\pgfqpoint{2.516575in}{0.898600in}}%
\pgfpathlineto{\pgfqpoint{2.514963in}{0.919227in}}%
\pgfpathlineto{\pgfqpoint{2.513322in}{0.939855in}}%
\pgfpathlineto{\pgfqpoint{2.511650in}{0.960483in}}%
\pgfpathlineto{\pgfqpoint{2.509944in}{0.981111in}}%
\pgfpathlineto{\pgfqpoint{2.508202in}{1.001739in}}%
\pgfpathlineto{\pgfqpoint{2.506422in}{1.022367in}}%
\pgfpathlineto{\pgfqpoint{2.504599in}{1.042994in}}%
\pgfpathlineto{\pgfqpoint{2.502730in}{1.063622in}}%
\pgfpathlineto{\pgfqpoint{2.500810in}{1.084250in}}%
\pgfpathlineto{\pgfqpoint{2.498836in}{1.104878in}}%
\pgfpathlineto{\pgfqpoint{2.496803in}{1.125506in}}%
\pgfpathlineto{\pgfqpoint{2.494703in}{1.146134in}}%
\pgfpathlineto{\pgfqpoint{2.493869in}{1.154214in}}%
\pgfpathlineto{\pgfqpoint{2.478161in}{1.146134in}}%
\pgfpathlineto{\pgfqpoint{2.452824in}{1.133460in}}%
\pgfpathlineto{\pgfqpoint{2.437820in}{1.125506in}}%
\pgfpathlineto{\pgfqpoint{2.411779in}{1.112245in}}%
\pgfpathlineto{\pgfqpoint{2.395241in}{1.104878in}}%
\pgfpathlineto{\pgfqpoint{2.370733in}{1.094477in}}%
\pgfpathlineto{\pgfqpoint{2.334972in}{1.084250in}}%
\pgfpathlineto{\pgfqpoint{2.329688in}{1.082816in}}%
\pgfpathlineto{\pgfqpoint{2.288643in}{1.078126in}}%
\pgfpathlineto{\pgfqpoint{2.247597in}{1.078910in}}%
\pgfpathlineto{\pgfqpoint{2.206552in}{1.082231in}}%
\pgfpathlineto{\pgfqpoint{2.167413in}{1.084250in}}%
\pgfpathlineto{\pgfqpoint{2.165507in}{1.084350in}}%
\pgfpathlineto{\pgfqpoint{2.164003in}{1.084250in}}%
\pgfpathlineto{\pgfqpoint{2.124461in}{1.081683in}}%
\pgfpathlineto{\pgfqpoint{2.083416in}{1.071602in}}%
\pgfpathlineto{\pgfqpoint{2.065630in}{1.063622in}}%
\pgfpathlineto{\pgfqpoint{2.042371in}{1.053172in}}%
\pgfpathlineto{\pgfqpoint{2.026049in}{1.042994in}}%
\pgfpathlineto{\pgfqpoint{2.001325in}{1.027373in}}%
\pgfpathlineto{\pgfqpoint{1.994513in}{1.022367in}}%
\pgfpathlineto{\pgfqpoint{1.966873in}{1.001739in}}%
\pgfpathlineto{\pgfqpoint{1.960280in}{0.996744in}}%
\pgfpathlineto{\pgfqpoint{1.940034in}{0.981111in}}%
\pgfpathlineto{\pgfqpoint{1.919235in}{0.964555in}}%
\pgfpathlineto{\pgfqpoint{1.913627in}{0.960483in}}%
\pgfpathlineto{\pgfqpoint{1.885872in}{0.939855in}}%
\pgfpathlineto{\pgfqpoint{1.878189in}{0.933996in}}%
\pgfpathlineto{\pgfqpoint{1.855170in}{0.919227in}}%
\pgfpathlineto{\pgfqpoint{1.837144in}{0.907253in}}%
\pgfpathlineto{\pgfqpoint{1.820849in}{0.898600in}}%
\pgfpathlineto{\pgfqpoint{1.796099in}{0.885036in}}%
\pgfpathlineto{\pgfqpoint{1.780157in}{0.877972in}}%
\pgfpathlineto{\pgfqpoint{1.755053in}{0.866580in}}%
\pgfpathlineto{\pgfqpoint{1.731869in}{0.857344in}}%
\pgfpathlineto{\pgfqpoint{1.714008in}{0.850139in}}%
\pgfpathlineto{\pgfqpoint{1.679596in}{0.836716in}}%
\pgfpathlineto{\pgfqpoint{1.672963in}{0.834132in}}%
\pgfpathlineto{\pgfqpoint{1.631917in}{0.818321in}}%
\pgfpathlineto{\pgfqpoint{1.625644in}{0.816088in}}%
\pgfpathlineto{\pgfqpoint{1.590872in}{0.804078in}}%
\pgfpathlineto{\pgfqpoint{1.557142in}{0.795460in}}%
\pgfpathlineto{\pgfqpoint{1.549827in}{0.793666in}}%
\pgfpathlineto{\pgfqpoint{1.508781in}{0.789501in}}%
\pgfpathlineto{\pgfqpoint{1.467736in}{0.792439in}}%
\pgfpathlineto{\pgfqpoint{1.454428in}{0.795460in}}%
\pgfpathlineto{\pgfqpoint{1.426691in}{0.801759in}}%
\pgfpathlineto{\pgfqpoint{1.385645in}{0.814710in}}%
\pgfpathlineto{\pgfqpoint{1.381167in}{0.816088in}}%
\pgfpathlineto{\pgfqpoint{1.344600in}{0.827451in}}%
\pgfpathlineto{\pgfqpoint{1.303555in}{0.835272in}}%
\pgfpathlineto{\pgfqpoint{1.262509in}{0.834294in}}%
\pgfpathlineto{\pgfqpoint{1.221464in}{0.821916in}}%
\pgfpathlineto{\pgfqpoint{1.211575in}{0.816088in}}%
\pgfpathlineto{\pgfqpoint{1.180419in}{0.797636in}}%
\pgfpathlineto{\pgfqpoint{1.177796in}{0.795460in}}%
\pgfpathlineto{\pgfqpoint{1.152991in}{0.774833in}}%
\pgfpathlineto{\pgfqpoint{1.139373in}{0.763351in}}%
\pgfpathlineto{\pgfqpoint{1.130091in}{0.754205in}}%
\pgfpathlineto{\pgfqpoint{1.109407in}{0.733577in}}%
\pgfpathlineto{\pgfqpoint{1.098328in}{0.722371in}}%
\pgfpathlineto{\pgfqpoint{1.089283in}{0.712949in}}%
\pgfpathlineto{\pgfqpoint{1.069768in}{0.692321in}}%
\pgfpathlineto{\pgfqpoint{1.057283in}{0.678866in}}%
\pgfpathlineto{\pgfqpoint{1.050158in}{0.671694in}}%
\pgfpathlineto{\pgfqpoint{1.029965in}{0.651066in}}%
\pgfpathlineto{\pgfqpoint{1.016237in}{0.636698in}}%
\pgfpathlineto{\pgfqpoint{1.009337in}{0.630438in}}%
\pgfpathlineto{\pgfqpoint{0.986976in}{0.609810in}}%
\pgfpathlineto{\pgfqpoint{0.975192in}{0.598680in}}%
\pgfpathlineto{\pgfqpoint{0.962991in}{0.589182in}}%
\pgfpathlineto{\pgfqpoint{0.937145in}{0.568554in}}%
\pgfpathlineto{\pgfqpoint{0.934147in}{0.566132in}}%
\pgfpathlineto{\pgfqpoint{0.906130in}{0.547927in}}%
\pgfpathlineto{\pgfqpoint{0.893101in}{0.539288in}}%
\pgfpathlineto{\pgfqpoint{0.870492in}{0.527299in}}%
\pgfpathlineto{\pgfqpoint{0.852056in}{0.517438in}}%
\pgfpathlineto{\pgfqpoint{0.826548in}{0.506671in}}%
\pgfpathlineto{\pgfqpoint{0.811011in}{0.500144in}}%
\pgfpathlineto{\pgfqpoint{0.769965in}{0.487700in}}%
\pgfpathlineto{\pgfqpoint{0.759476in}{0.486043in}}%
\pgfpathlineto{\pgfqpoint{0.728920in}{0.481364in}}%
\pgfpathlineto{\pgfqpoint{0.687875in}{0.482474in}}%
\pgfpathlineto{\pgfqpoint{0.672367in}{0.486043in}}%
\pgfpathlineto{\pgfqpoint{0.646829in}{0.491949in}}%
\pgfpathclose%
\pgfusepath{stroke,fill}%
\end{pgfscope}%
\begin{pgfscope}%
\pgfpathrectangle{\pgfqpoint{0.605784in}{0.382904in}}{\pgfqpoint{4.063488in}{2.042155in}}%
\pgfusepath{clip}%
\pgfsetbuttcap%
\pgfsetroundjoin%
\definecolor{currentfill}{rgb}{0.166617,0.463708,0.558119}%
\pgfsetfillcolor{currentfill}%
\pgfsetlinewidth{1.003750pt}%
\definecolor{currentstroke}{rgb}{0.166617,0.463708,0.558119}%
\pgfsetstrokecolor{currentstroke}%
\pgfsetdash{}{0pt}%
\pgfpathmoveto{\pgfqpoint{0.616074in}{1.991874in}}%
\pgfpathlineto{\pgfqpoint{0.605784in}{1.997683in}}%
\pgfpathlineto{\pgfqpoint{0.605784in}{2.012502in}}%
\pgfpathlineto{\pgfqpoint{0.605784in}{2.033130in}}%
\pgfpathlineto{\pgfqpoint{0.605784in}{2.053758in}}%
\pgfpathlineto{\pgfqpoint{0.605784in}{2.071832in}}%
\pgfpathlineto{\pgfqpoint{0.638148in}{2.053758in}}%
\pgfpathlineto{\pgfqpoint{0.646829in}{2.048907in}}%
\pgfpathlineto{\pgfqpoint{0.673199in}{2.033130in}}%
\pgfpathlineto{\pgfqpoint{0.687875in}{2.024309in}}%
\pgfpathlineto{\pgfqpoint{0.710922in}{2.012502in}}%
\pgfpathlineto{\pgfqpoint{0.728920in}{2.003167in}}%
\pgfpathlineto{\pgfqpoint{0.766815in}{1.991874in}}%
\pgfpathlineto{\pgfqpoint{0.769965in}{1.990913in}}%
\pgfpathlineto{\pgfqpoint{0.808451in}{1.991874in}}%
\pgfpathlineto{\pgfqpoint{0.811011in}{1.991937in}}%
\pgfpathlineto{\pgfqpoint{0.852056in}{2.008671in}}%
\pgfpathlineto{\pgfqpoint{0.856939in}{2.012502in}}%
\pgfpathlineto{\pgfqpoint{0.883234in}{2.033130in}}%
\pgfpathlineto{\pgfqpoint{0.893101in}{2.040903in}}%
\pgfpathlineto{\pgfqpoint{0.904955in}{2.053758in}}%
\pgfpathlineto{\pgfqpoint{0.923821in}{2.074386in}}%
\pgfpathlineto{\pgfqpoint{0.934147in}{2.085748in}}%
\pgfpathlineto{\pgfqpoint{0.941421in}{2.095013in}}%
\pgfpathlineto{\pgfqpoint{0.957537in}{2.115641in}}%
\pgfpathlineto{\pgfqpoint{0.973390in}{2.136269in}}%
\pgfpathlineto{\pgfqpoint{0.975192in}{2.138607in}}%
\pgfpathlineto{\pgfqpoint{0.988894in}{2.156897in}}%
\pgfpathlineto{\pgfqpoint{1.004101in}{2.177525in}}%
\pgfpathlineto{\pgfqpoint{1.016237in}{2.194211in}}%
\pgfpathlineto{\pgfqpoint{1.019289in}{2.198153in}}%
\pgfpathlineto{\pgfqpoint{1.035232in}{2.218780in}}%
\pgfpathlineto{\pgfqpoint{1.050845in}{2.239408in}}%
\pgfpathlineto{\pgfqpoint{1.057283in}{2.247968in}}%
\pgfpathlineto{\pgfqpoint{1.067653in}{2.260036in}}%
\pgfpathlineto{\pgfqpoint{1.085141in}{2.280664in}}%
\pgfpathlineto{\pgfqpoint{1.098328in}{2.296492in}}%
\pgfpathlineto{\pgfqpoint{1.103172in}{2.301292in}}%
\pgfpathlineto{\pgfqpoint{1.123860in}{2.321920in}}%
\pgfpathlineto{\pgfqpoint{1.139373in}{2.337697in}}%
\pgfpathlineto{\pgfqpoint{1.145448in}{2.342547in}}%
\pgfpathlineto{\pgfqpoint{1.171103in}{2.363175in}}%
\pgfpathlineto{\pgfqpoint{1.180419in}{2.370742in}}%
\pgfpathlineto{\pgfqpoint{1.202177in}{2.383803in}}%
\pgfpathlineto{\pgfqpoint{1.221464in}{2.395437in}}%
\pgfpathlineto{\pgfqpoint{1.244225in}{2.404431in}}%
\pgfpathlineto{\pgfqpoint{1.262509in}{2.411597in}}%
\pgfpathlineto{\pgfqpoint{1.303555in}{2.418687in}}%
\pgfpathlineto{\pgfqpoint{1.344600in}{2.415906in}}%
\pgfpathlineto{\pgfqpoint{1.380480in}{2.404431in}}%
\pgfpathlineto{\pgfqpoint{1.385645in}{2.402763in}}%
\pgfpathlineto{\pgfqpoint{1.419080in}{2.383803in}}%
\pgfpathlineto{\pgfqpoint{1.426691in}{2.379474in}}%
\pgfpathlineto{\pgfqpoint{1.447910in}{2.363175in}}%
\pgfpathlineto{\pgfqpoint{1.467736in}{2.347943in}}%
\pgfpathlineto{\pgfqpoint{1.473904in}{2.342547in}}%
\pgfpathlineto{\pgfqpoint{1.497288in}{2.321920in}}%
\pgfpathlineto{\pgfqpoint{1.508781in}{2.311728in}}%
\pgfpathlineto{\pgfqpoint{1.520912in}{2.301292in}}%
\pgfpathlineto{\pgfqpoint{1.544670in}{2.280664in}}%
\pgfpathlineto{\pgfqpoint{1.549827in}{2.276128in}}%
\pgfpathlineto{\pgfqpoint{1.573000in}{2.260036in}}%
\pgfpathlineto{\pgfqpoint{1.590872in}{2.247428in}}%
\pgfpathlineto{\pgfqpoint{1.612650in}{2.239408in}}%
\pgfpathlineto{\pgfqpoint{1.631917in}{2.232110in}}%
\pgfpathlineto{\pgfqpoint{1.672963in}{2.235170in}}%
\pgfpathlineto{\pgfqpoint{1.680332in}{2.239408in}}%
\pgfpathlineto{\pgfqpoint{1.714008in}{2.258749in}}%
\pgfpathlineto{\pgfqpoint{1.715243in}{2.260036in}}%
\pgfpathlineto{\pgfqpoint{1.735195in}{2.280664in}}%
\pgfpathlineto{\pgfqpoint{1.754926in}{2.301292in}}%
\pgfpathlineto{\pgfqpoint{1.755053in}{2.301423in}}%
\pgfpathlineto{\pgfqpoint{1.769847in}{2.321920in}}%
\pgfpathlineto{\pgfqpoint{1.784600in}{2.342547in}}%
\pgfpathlineto{\pgfqpoint{1.796099in}{2.358704in}}%
\pgfpathlineto{\pgfqpoint{1.798893in}{2.363175in}}%
\pgfpathlineto{\pgfqpoint{1.811852in}{2.383803in}}%
\pgfpathlineto{\pgfqpoint{1.824688in}{2.404431in}}%
\pgfpathlineto{\pgfqpoint{1.837144in}{2.424622in}}%
\pgfpathlineto{\pgfqpoint{1.837408in}{2.425059in}}%
\pgfpathlineto{\pgfqpoint{1.878189in}{2.425059in}}%
\pgfpathlineto{\pgfqpoint{1.882354in}{2.425059in}}%
\pgfpathlineto{\pgfqpoint{1.878189in}{2.418350in}}%
\pgfpathlineto{\pgfqpoint{1.870144in}{2.404431in}}%
\pgfpathlineto{\pgfqpoint{1.858178in}{2.383803in}}%
\pgfpathlineto{\pgfqpoint{1.846088in}{2.363175in}}%
\pgfpathlineto{\pgfqpoint{1.837144in}{2.347996in}}%
\pgfpathlineto{\pgfqpoint{1.833882in}{2.342547in}}%
\pgfpathlineto{\pgfqpoint{1.821603in}{2.321920in}}%
\pgfpathlineto{\pgfqpoint{1.809217in}{2.301292in}}%
\pgfpathlineto{\pgfqpoint{1.796718in}{2.280664in}}%
\pgfpathlineto{\pgfqpoint{1.796099in}{2.279631in}}%
\pgfpathlineto{\pgfqpoint{1.782736in}{2.260036in}}%
\pgfpathlineto{\pgfqpoint{1.768566in}{2.239408in}}%
\pgfpathlineto{\pgfqpoint{1.755053in}{2.219889in}}%
\pgfpathlineto{\pgfqpoint{1.754024in}{2.218780in}}%
\pgfpathlineto{\pgfqpoint{1.735053in}{2.198153in}}%
\pgfpathlineto{\pgfqpoint{1.715875in}{2.177525in}}%
\pgfpathlineto{\pgfqpoint{1.714008in}{2.175499in}}%
\pgfpathlineto{\pgfqpoint{1.682310in}{2.156897in}}%
\pgfpathlineto{\pgfqpoint{1.672963in}{2.151404in}}%
\pgfpathlineto{\pgfqpoint{1.631917in}{2.149208in}}%
\pgfpathlineto{\pgfqpoint{1.613928in}{2.156897in}}%
\pgfpathlineto{\pgfqpoint{1.590872in}{2.166442in}}%
\pgfpathlineto{\pgfqpoint{1.576432in}{2.177525in}}%
\pgfpathlineto{\pgfqpoint{1.549827in}{2.197585in}}%
\pgfpathlineto{\pgfqpoint{1.549224in}{2.198153in}}%
\pgfpathlineto{\pgfqpoint{1.526936in}{2.218780in}}%
\pgfpathlineto{\pgfqpoint{1.508781in}{2.235455in}}%
\pgfpathlineto{\pgfqpoint{1.504560in}{2.239408in}}%
\pgfpathlineto{\pgfqpoint{1.482276in}{2.260036in}}%
\pgfpathlineto{\pgfqpoint{1.467736in}{2.273436in}}%
\pgfpathlineto{\pgfqpoint{1.458704in}{2.280664in}}%
\pgfpathlineto{\pgfqpoint{1.432807in}{2.301292in}}%
\pgfpathlineto{\pgfqpoint{1.426691in}{2.306132in}}%
\pgfpathlineto{\pgfqpoint{1.399461in}{2.321920in}}%
\pgfpathlineto{\pgfqpoint{1.385645in}{2.329902in}}%
\pgfpathlineto{\pgfqpoint{1.346569in}{2.342547in}}%
\pgfpathlineto{\pgfqpoint{1.344600in}{2.343178in}}%
\pgfpathlineto{\pgfqpoint{1.303555in}{2.345663in}}%
\pgfpathlineto{\pgfqpoint{1.286060in}{2.342547in}}%
\pgfpathlineto{\pgfqpoint{1.262509in}{2.338255in}}%
\pgfpathlineto{\pgfqpoint{1.221464in}{2.321956in}}%
\pgfpathlineto{\pgfqpoint{1.221404in}{2.321920in}}%
\pgfpathlineto{\pgfqpoint{1.187252in}{2.301292in}}%
\pgfpathlineto{\pgfqpoint{1.180419in}{2.297186in}}%
\pgfpathlineto{\pgfqpoint{1.160052in}{2.280664in}}%
\pgfpathlineto{\pgfqpoint{1.139373in}{2.264193in}}%
\pgfpathlineto{\pgfqpoint{1.135292in}{2.260036in}}%
\pgfpathlineto{\pgfqpoint{1.114928in}{2.239408in}}%
\pgfpathlineto{\pgfqpoint{1.098328in}{2.222975in}}%
\pgfpathlineto{\pgfqpoint{1.094853in}{2.218780in}}%
\pgfpathlineto{\pgfqpoint{1.077706in}{2.198153in}}%
\pgfpathlineto{\pgfqpoint{1.060112in}{2.177525in}}%
\pgfpathlineto{\pgfqpoint{1.057283in}{2.174218in}}%
\pgfpathlineto{\pgfqpoint{1.044439in}{2.156897in}}%
\pgfpathlineto{\pgfqpoint{1.028870in}{2.136269in}}%
\pgfpathlineto{\pgfqpoint{1.016237in}{2.119821in}}%
\pgfpathlineto{\pgfqpoint{1.013242in}{2.115641in}}%
\pgfpathlineto{\pgfqpoint{0.998466in}{2.095013in}}%
\pgfpathlineto{\pgfqpoint{0.983405in}{2.074386in}}%
\pgfpathlineto{\pgfqpoint{0.975192in}{2.063234in}}%
\pgfpathlineto{\pgfqpoint{0.968056in}{2.053758in}}%
\pgfpathlineto{\pgfqpoint{0.952456in}{2.033130in}}%
\pgfpathlineto{\pgfqpoint{0.936579in}{2.012502in}}%
\pgfpathlineto{\pgfqpoint{0.934147in}{2.009334in}}%
\pgfpathlineto{\pgfqpoint{0.918669in}{1.991874in}}%
\pgfpathlineto{\pgfqpoint{0.900138in}{1.971247in}}%
\pgfpathlineto{\pgfqpoint{0.893101in}{1.963434in}}%
\pgfpathlineto{\pgfqpoint{0.877155in}{1.950619in}}%
\pgfpathlineto{\pgfqpoint{0.852056in}{1.930640in}}%
\pgfpathlineto{\pgfqpoint{0.850462in}{1.929991in}}%
\pgfpathlineto{\pgfqpoint{0.811011in}{1.913848in}}%
\pgfpathlineto{\pgfqpoint{0.769965in}{1.913499in}}%
\pgfpathlineto{\pgfqpoint{0.728920in}{1.926876in}}%
\pgfpathlineto{\pgfqpoint{0.723245in}{1.929991in}}%
\pgfpathlineto{\pgfqpoint{0.687875in}{1.949135in}}%
\pgfpathlineto{\pgfqpoint{0.685495in}{1.950619in}}%
\pgfpathlineto{\pgfqpoint{0.652131in}{1.971247in}}%
\pgfpathlineto{\pgfqpoint{0.646829in}{1.974503in}}%
\pgfpathclose%
\pgfusepath{stroke,fill}%
\end{pgfscope}%
\begin{pgfscope}%
\pgfpathrectangle{\pgfqpoint{0.605784in}{0.382904in}}{\pgfqpoint{4.063488in}{2.042155in}}%
\pgfusepath{clip}%
\pgfsetbuttcap%
\pgfsetroundjoin%
\definecolor{currentfill}{rgb}{0.166617,0.463708,0.558119}%
\pgfsetfillcolor{currentfill}%
\pgfsetlinewidth{1.003750pt}%
\definecolor{currentstroke}{rgb}{0.166617,0.463708,0.558119}%
\pgfsetstrokecolor{currentstroke}%
\pgfsetdash{}{0pt}%
\pgfpathmoveto{\pgfqpoint{2.447388in}{2.404431in}}%
\pgfpathlineto{\pgfqpoint{2.425912in}{2.425059in}}%
\pgfpathlineto{\pgfqpoint{2.452824in}{2.425059in}}%
\pgfpathlineto{\pgfqpoint{2.493869in}{2.425059in}}%
\pgfpathlineto{\pgfqpoint{2.498223in}{2.425059in}}%
\pgfpathlineto{\pgfqpoint{2.500342in}{2.404431in}}%
\pgfpathlineto{\pgfqpoint{2.502535in}{2.383803in}}%
\pgfpathlineto{\pgfqpoint{2.504811in}{2.363175in}}%
\pgfpathlineto{\pgfqpoint{2.507178in}{2.342547in}}%
\pgfpathlineto{\pgfqpoint{2.509647in}{2.321920in}}%
\pgfpathlineto{\pgfqpoint{2.512229in}{2.301292in}}%
\pgfpathlineto{\pgfqpoint{2.514939in}{2.280664in}}%
\pgfpathlineto{\pgfqpoint{2.517794in}{2.260036in}}%
\pgfpathlineto{\pgfqpoint{2.520812in}{2.239408in}}%
\pgfpathlineto{\pgfqpoint{2.524016in}{2.218780in}}%
\pgfpathlineto{\pgfqpoint{2.527434in}{2.198153in}}%
\pgfpathlineto{\pgfqpoint{2.531100in}{2.177525in}}%
\pgfpathlineto{\pgfqpoint{2.534915in}{2.157582in}}%
\pgfpathlineto{\pgfqpoint{2.575960in}{2.157582in}}%
\pgfpathlineto{\pgfqpoint{2.617005in}{2.157582in}}%
\pgfpathlineto{\pgfqpoint{2.658051in}{2.157582in}}%
\pgfpathlineto{\pgfqpoint{2.699096in}{2.157582in}}%
\pgfpathlineto{\pgfqpoint{2.740141in}{2.157582in}}%
\pgfpathlineto{\pgfqpoint{2.781187in}{2.157582in}}%
\pgfpathlineto{\pgfqpoint{2.822232in}{2.157582in}}%
\pgfpathlineto{\pgfqpoint{2.863277in}{2.157582in}}%
\pgfpathlineto{\pgfqpoint{2.904323in}{2.157582in}}%
\pgfpathlineto{\pgfqpoint{2.945368in}{2.157582in}}%
\pgfpathlineto{\pgfqpoint{2.986413in}{2.157582in}}%
\pgfpathlineto{\pgfqpoint{3.027459in}{2.157582in}}%
\pgfpathlineto{\pgfqpoint{3.068504in}{2.157582in}}%
\pgfpathlineto{\pgfqpoint{3.109549in}{2.157582in}}%
\pgfpathlineto{\pgfqpoint{3.150595in}{2.157582in}}%
\pgfpathlineto{\pgfqpoint{3.191640in}{2.157582in}}%
\pgfpathlineto{\pgfqpoint{3.232685in}{2.157582in}}%
\pgfpathlineto{\pgfqpoint{3.273731in}{2.157582in}}%
\pgfpathlineto{\pgfqpoint{3.314776in}{2.157582in}}%
\pgfpathlineto{\pgfqpoint{3.355821in}{2.157582in}}%
\pgfpathlineto{\pgfqpoint{3.396867in}{2.157582in}}%
\pgfpathlineto{\pgfqpoint{3.437912in}{2.157582in}}%
\pgfpathlineto{\pgfqpoint{3.478957in}{2.157582in}}%
\pgfpathlineto{\pgfqpoint{3.520003in}{2.157582in}}%
\pgfpathlineto{\pgfqpoint{3.561048in}{2.157582in}}%
\pgfpathlineto{\pgfqpoint{3.602093in}{2.157582in}}%
\pgfpathlineto{\pgfqpoint{3.643139in}{2.157582in}}%
\pgfpathlineto{\pgfqpoint{3.643253in}{2.156897in}}%
\pgfpathlineto{\pgfqpoint{3.646694in}{2.136269in}}%
\pgfpathlineto{\pgfqpoint{3.650210in}{2.115641in}}%
\pgfpathlineto{\pgfqpoint{3.653806in}{2.095013in}}%
\pgfpathlineto{\pgfqpoint{3.657489in}{2.074386in}}%
\pgfpathlineto{\pgfqpoint{3.661266in}{2.053758in}}%
\pgfpathlineto{\pgfqpoint{3.665145in}{2.033130in}}%
\pgfpathlineto{\pgfqpoint{3.669134in}{2.012502in}}%
\pgfpathlineto{\pgfqpoint{3.673243in}{1.991874in}}%
\pgfpathlineto{\pgfqpoint{3.677484in}{1.971247in}}%
\pgfpathlineto{\pgfqpoint{3.681868in}{1.950619in}}%
\pgfpathlineto{\pgfqpoint{3.684184in}{1.939935in}}%
\pgfpathlineto{\pgfqpoint{3.725229in}{1.939935in}}%
\pgfpathlineto{\pgfqpoint{3.766275in}{1.939935in}}%
\pgfpathlineto{\pgfqpoint{3.807320in}{1.939935in}}%
\pgfpathlineto{\pgfqpoint{3.848365in}{1.939935in}}%
\pgfpathlineto{\pgfqpoint{3.889411in}{1.939935in}}%
\pgfpathlineto{\pgfqpoint{3.930456in}{1.939935in}}%
\pgfpathlineto{\pgfqpoint{3.971501in}{1.939935in}}%
\pgfpathlineto{\pgfqpoint{4.012547in}{1.939935in}}%
\pgfpathlineto{\pgfqpoint{4.053592in}{1.939935in}}%
\pgfpathlineto{\pgfqpoint{4.094637in}{1.939935in}}%
\pgfpathlineto{\pgfqpoint{4.135683in}{1.939935in}}%
\pgfpathlineto{\pgfqpoint{4.176728in}{1.939935in}}%
\pgfpathlineto{\pgfqpoint{4.217773in}{1.939935in}}%
\pgfpathlineto{\pgfqpoint{4.258819in}{1.939935in}}%
\pgfpathlineto{\pgfqpoint{4.299864in}{1.939935in}}%
\pgfpathlineto{\pgfqpoint{4.340909in}{1.939935in}}%
\pgfpathlineto{\pgfqpoint{4.381955in}{1.939935in}}%
\pgfpathlineto{\pgfqpoint{4.423000in}{1.939935in}}%
\pgfpathlineto{\pgfqpoint{4.464045in}{1.939935in}}%
\pgfpathlineto{\pgfqpoint{4.505091in}{1.939935in}}%
\pgfpathlineto{\pgfqpoint{4.546136in}{1.939935in}}%
\pgfpathlineto{\pgfqpoint{4.587181in}{1.939935in}}%
\pgfpathlineto{\pgfqpoint{4.628227in}{1.939935in}}%
\pgfpathlineto{\pgfqpoint{4.669272in}{1.939935in}}%
\pgfpathlineto{\pgfqpoint{4.669272in}{1.929991in}}%
\pgfpathlineto{\pgfqpoint{4.669272in}{1.909363in}}%
\pgfpathlineto{\pgfqpoint{4.669272in}{1.888735in}}%
\pgfpathlineto{\pgfqpoint{4.669272in}{1.868107in}}%
\pgfpathlineto{\pgfqpoint{4.669272in}{1.863329in}}%
\pgfpathlineto{\pgfqpoint{4.628227in}{1.863329in}}%
\pgfpathlineto{\pgfqpoint{4.587181in}{1.863329in}}%
\pgfpathlineto{\pgfqpoint{4.546136in}{1.863329in}}%
\pgfpathlineto{\pgfqpoint{4.505091in}{1.863329in}}%
\pgfpathlineto{\pgfqpoint{4.464045in}{1.863329in}}%
\pgfpathlineto{\pgfqpoint{4.423000in}{1.863329in}}%
\pgfpathlineto{\pgfqpoint{4.381955in}{1.863329in}}%
\pgfpathlineto{\pgfqpoint{4.340909in}{1.863329in}}%
\pgfpathlineto{\pgfqpoint{4.299864in}{1.863329in}}%
\pgfpathlineto{\pgfqpoint{4.258819in}{1.863329in}}%
\pgfpathlineto{\pgfqpoint{4.217773in}{1.863329in}}%
\pgfpathlineto{\pgfqpoint{4.176728in}{1.863329in}}%
\pgfpathlineto{\pgfqpoint{4.135683in}{1.863329in}}%
\pgfpathlineto{\pgfqpoint{4.094637in}{1.863329in}}%
\pgfpathlineto{\pgfqpoint{4.053592in}{1.863329in}}%
\pgfpathlineto{\pgfqpoint{4.012547in}{1.863329in}}%
\pgfpathlineto{\pgfqpoint{3.971501in}{1.863329in}}%
\pgfpathlineto{\pgfqpoint{3.930456in}{1.863329in}}%
\pgfpathlineto{\pgfqpoint{3.889411in}{1.863329in}}%
\pgfpathlineto{\pgfqpoint{3.848365in}{1.863329in}}%
\pgfpathlineto{\pgfqpoint{3.807320in}{1.863329in}}%
\pgfpathlineto{\pgfqpoint{3.766275in}{1.863329in}}%
\pgfpathlineto{\pgfqpoint{3.725229in}{1.863329in}}%
\pgfpathlineto{\pgfqpoint{3.684184in}{1.863329in}}%
\pgfpathlineto{\pgfqpoint{3.683129in}{1.868107in}}%
\pgfpathlineto{\pgfqpoint{3.678633in}{1.888735in}}%
\pgfpathlineto{\pgfqpoint{3.674308in}{1.909363in}}%
\pgfpathlineto{\pgfqpoint{3.670136in}{1.929991in}}%
\pgfpathlineto{\pgfqpoint{3.666103in}{1.950619in}}%
\pgfpathlineto{\pgfqpoint{3.662197in}{1.971247in}}%
\pgfpathlineto{\pgfqpoint{3.658406in}{1.991874in}}%
\pgfpathlineto{\pgfqpoint{3.654721in}{2.012502in}}%
\pgfpathlineto{\pgfqpoint{3.651132in}{2.033130in}}%
\pgfpathlineto{\pgfqpoint{3.647632in}{2.053758in}}%
\pgfpathlineto{\pgfqpoint{3.644214in}{2.074386in}}%
\pgfpathlineto{\pgfqpoint{3.643139in}{2.080903in}}%
\pgfpathlineto{\pgfqpoint{3.602093in}{2.080903in}}%
\pgfpathlineto{\pgfqpoint{3.561048in}{2.080903in}}%
\pgfpathlineto{\pgfqpoint{3.520003in}{2.080903in}}%
\pgfpathlineto{\pgfqpoint{3.478957in}{2.080903in}}%
\pgfpathlineto{\pgfqpoint{3.437912in}{2.080903in}}%
\pgfpathlineto{\pgfqpoint{3.396867in}{2.080903in}}%
\pgfpathlineto{\pgfqpoint{3.355821in}{2.080903in}}%
\pgfpathlineto{\pgfqpoint{3.314776in}{2.080903in}}%
\pgfpathlineto{\pgfqpoint{3.273731in}{2.080903in}}%
\pgfpathlineto{\pgfqpoint{3.232685in}{2.080903in}}%
\pgfpathlineto{\pgfqpoint{3.191640in}{2.080903in}}%
\pgfpathlineto{\pgfqpoint{3.150595in}{2.080903in}}%
\pgfpathlineto{\pgfqpoint{3.109549in}{2.080903in}}%
\pgfpathlineto{\pgfqpoint{3.068504in}{2.080903in}}%
\pgfpathlineto{\pgfqpoint{3.027459in}{2.080903in}}%
\pgfpathlineto{\pgfqpoint{2.986413in}{2.080903in}}%
\pgfpathlineto{\pgfqpoint{2.945368in}{2.080903in}}%
\pgfpathlineto{\pgfqpoint{2.904323in}{2.080903in}}%
\pgfpathlineto{\pgfqpoint{2.863277in}{2.080903in}}%
\pgfpathlineto{\pgfqpoint{2.822232in}{2.080903in}}%
\pgfpathlineto{\pgfqpoint{2.781187in}{2.080903in}}%
\pgfpathlineto{\pgfqpoint{2.740141in}{2.080903in}}%
\pgfpathlineto{\pgfqpoint{2.699096in}{2.080903in}}%
\pgfpathlineto{\pgfqpoint{2.658051in}{2.080903in}}%
\pgfpathlineto{\pgfqpoint{2.617005in}{2.080903in}}%
\pgfpathlineto{\pgfqpoint{2.575960in}{2.080903in}}%
\pgfpathlineto{\pgfqpoint{2.534915in}{2.080903in}}%
\pgfpathlineto{\pgfqpoint{2.531878in}{2.095013in}}%
\pgfpathlineto{\pgfqpoint{2.527771in}{2.115641in}}%
\pgfpathlineto{\pgfqpoint{2.524018in}{2.136269in}}%
\pgfpathlineto{\pgfqpoint{2.520563in}{2.156897in}}%
\pgfpathlineto{\pgfqpoint{2.517358in}{2.177525in}}%
\pgfpathlineto{\pgfqpoint{2.514368in}{2.198153in}}%
\pgfpathlineto{\pgfqpoint{2.511561in}{2.218780in}}%
\pgfpathlineto{\pgfqpoint{2.508914in}{2.239408in}}%
\pgfpathlineto{\pgfqpoint{2.506405in}{2.260036in}}%
\pgfpathlineto{\pgfqpoint{2.504018in}{2.280664in}}%
\pgfpathlineto{\pgfqpoint{2.501738in}{2.301292in}}%
\pgfpathlineto{\pgfqpoint{2.499554in}{2.321920in}}%
\pgfpathlineto{\pgfqpoint{2.497454in}{2.342547in}}%
\pgfpathlineto{\pgfqpoint{2.495430in}{2.363175in}}%
\pgfpathlineto{\pgfqpoint{2.493869in}{2.379517in}}%
\pgfpathlineto{\pgfqpoint{2.485146in}{2.383803in}}%
\pgfpathlineto{\pgfqpoint{2.452824in}{2.399066in}}%
\pgfpathclose%
\pgfusepath{stroke,fill}%
\end{pgfscope}%
\begin{pgfscope}%
\pgfpathrectangle{\pgfqpoint{0.605784in}{0.382904in}}{\pgfqpoint{4.063488in}{2.042155in}}%
\pgfusepath{clip}%
\pgfsetbuttcap%
\pgfsetroundjoin%
\definecolor{currentfill}{rgb}{0.140536,0.530132,0.555659}%
\pgfsetfillcolor{currentfill}%
\pgfsetlinewidth{1.003750pt}%
\definecolor{currentstroke}{rgb}{0.140536,0.530132,0.555659}%
\pgfsetstrokecolor{currentstroke}%
\pgfsetdash{}{0pt}%
\pgfpathmoveto{\pgfqpoint{0.646453in}{0.424160in}}%
\pgfpathlineto{\pgfqpoint{0.605784in}{0.441557in}}%
\pgfpathlineto{\pgfqpoint{0.605784in}{0.444787in}}%
\pgfpathlineto{\pgfqpoint{0.605784in}{0.465415in}}%
\pgfpathlineto{\pgfqpoint{0.605784in}{0.486043in}}%
\pgfpathlineto{\pgfqpoint{0.605784in}{0.506671in}}%
\pgfpathlineto{\pgfqpoint{0.605784in}{0.509225in}}%
\pgfpathlineto{\pgfqpoint{0.611848in}{0.506671in}}%
\pgfpathlineto{\pgfqpoint{0.646829in}{0.491949in}}%
\pgfpathlineto{\pgfqpoint{0.672367in}{0.486043in}}%
\pgfpathlineto{\pgfqpoint{0.687875in}{0.482474in}}%
\pgfpathlineto{\pgfqpoint{0.728920in}{0.481364in}}%
\pgfpathlineto{\pgfqpoint{0.759476in}{0.486043in}}%
\pgfpathlineto{\pgfqpoint{0.769965in}{0.487700in}}%
\pgfpathlineto{\pgfqpoint{0.811011in}{0.500144in}}%
\pgfpathlineto{\pgfqpoint{0.826548in}{0.506671in}}%
\pgfpathlineto{\pgfqpoint{0.852056in}{0.517438in}}%
\pgfpathlineto{\pgfqpoint{0.870492in}{0.527299in}}%
\pgfpathlineto{\pgfqpoint{0.893101in}{0.539288in}}%
\pgfpathlineto{\pgfqpoint{0.906130in}{0.547927in}}%
\pgfpathlineto{\pgfqpoint{0.934147in}{0.566132in}}%
\pgfpathlineto{\pgfqpoint{0.937145in}{0.568554in}}%
\pgfpathlineto{\pgfqpoint{0.962991in}{0.589182in}}%
\pgfpathlineto{\pgfqpoint{0.975192in}{0.598680in}}%
\pgfpathlineto{\pgfqpoint{0.986976in}{0.609810in}}%
\pgfpathlineto{\pgfqpoint{1.009337in}{0.630438in}}%
\pgfpathlineto{\pgfqpoint{1.016237in}{0.636698in}}%
\pgfpathlineto{\pgfqpoint{1.029965in}{0.651066in}}%
\pgfpathlineto{\pgfqpoint{1.050158in}{0.671694in}}%
\pgfpathlineto{\pgfqpoint{1.057283in}{0.678866in}}%
\pgfpathlineto{\pgfqpoint{1.069768in}{0.692321in}}%
\pgfpathlineto{\pgfqpoint{1.089283in}{0.712949in}}%
\pgfpathlineto{\pgfqpoint{1.098328in}{0.722371in}}%
\pgfpathlineto{\pgfqpoint{1.109407in}{0.733577in}}%
\pgfpathlineto{\pgfqpoint{1.130091in}{0.754205in}}%
\pgfpathlineto{\pgfqpoint{1.139373in}{0.763351in}}%
\pgfpathlineto{\pgfqpoint{1.152991in}{0.774833in}}%
\pgfpathlineto{\pgfqpoint{1.177796in}{0.795460in}}%
\pgfpathlineto{\pgfqpoint{1.180419in}{0.797636in}}%
\pgfpathlineto{\pgfqpoint{1.211575in}{0.816088in}}%
\pgfpathlineto{\pgfqpoint{1.221464in}{0.821916in}}%
\pgfpathlineto{\pgfqpoint{1.262509in}{0.834294in}}%
\pgfpathlineto{\pgfqpoint{1.303555in}{0.835272in}}%
\pgfpathlineto{\pgfqpoint{1.344600in}{0.827451in}}%
\pgfpathlineto{\pgfqpoint{1.381167in}{0.816088in}}%
\pgfpathlineto{\pgfqpoint{1.385645in}{0.814710in}}%
\pgfpathlineto{\pgfqpoint{1.426691in}{0.801759in}}%
\pgfpathlineto{\pgfqpoint{1.454428in}{0.795460in}}%
\pgfpathlineto{\pgfqpoint{1.467736in}{0.792439in}}%
\pgfpathlineto{\pgfqpoint{1.508781in}{0.789501in}}%
\pgfpathlineto{\pgfqpoint{1.549827in}{0.793666in}}%
\pgfpathlineto{\pgfqpoint{1.557142in}{0.795460in}}%
\pgfpathlineto{\pgfqpoint{1.590872in}{0.804078in}}%
\pgfpathlineto{\pgfqpoint{1.625644in}{0.816088in}}%
\pgfpathlineto{\pgfqpoint{1.631917in}{0.818321in}}%
\pgfpathlineto{\pgfqpoint{1.672963in}{0.834132in}}%
\pgfpathlineto{\pgfqpoint{1.679596in}{0.836716in}}%
\pgfpathlineto{\pgfqpoint{1.714008in}{0.850139in}}%
\pgfpathlineto{\pgfqpoint{1.731869in}{0.857344in}}%
\pgfpathlineto{\pgfqpoint{1.755053in}{0.866580in}}%
\pgfpathlineto{\pgfqpoint{1.780157in}{0.877972in}}%
\pgfpathlineto{\pgfqpoint{1.796099in}{0.885036in}}%
\pgfpathlineto{\pgfqpoint{1.820849in}{0.898600in}}%
\pgfpathlineto{\pgfqpoint{1.837144in}{0.907253in}}%
\pgfpathlineto{\pgfqpoint{1.855170in}{0.919227in}}%
\pgfpathlineto{\pgfqpoint{1.878189in}{0.933996in}}%
\pgfpathlineto{\pgfqpoint{1.885872in}{0.939855in}}%
\pgfpathlineto{\pgfqpoint{1.913627in}{0.960483in}}%
\pgfpathlineto{\pgfqpoint{1.919235in}{0.964555in}}%
\pgfpathlineto{\pgfqpoint{1.940034in}{0.981111in}}%
\pgfpathlineto{\pgfqpoint{1.960280in}{0.996744in}}%
\pgfpathlineto{\pgfqpoint{1.966873in}{1.001739in}}%
\pgfpathlineto{\pgfqpoint{1.994513in}{1.022367in}}%
\pgfpathlineto{\pgfqpoint{2.001325in}{1.027373in}}%
\pgfpathlineto{\pgfqpoint{2.026049in}{1.042994in}}%
\pgfpathlineto{\pgfqpoint{2.042371in}{1.053172in}}%
\pgfpathlineto{\pgfqpoint{2.065630in}{1.063622in}}%
\pgfpathlineto{\pgfqpoint{2.083416in}{1.071602in}}%
\pgfpathlineto{\pgfqpoint{2.124461in}{1.081683in}}%
\pgfpathlineto{\pgfqpoint{2.164003in}{1.084250in}}%
\pgfpathlineto{\pgfqpoint{2.165507in}{1.084350in}}%
\pgfpathlineto{\pgfqpoint{2.167413in}{1.084250in}}%
\pgfpathlineto{\pgfqpoint{2.206552in}{1.082231in}}%
\pgfpathlineto{\pgfqpoint{2.247597in}{1.078910in}}%
\pgfpathlineto{\pgfqpoint{2.288643in}{1.078126in}}%
\pgfpathlineto{\pgfqpoint{2.329688in}{1.082816in}}%
\pgfpathlineto{\pgfqpoint{2.334972in}{1.084250in}}%
\pgfpathlineto{\pgfqpoint{2.370733in}{1.094477in}}%
\pgfpathlineto{\pgfqpoint{2.395241in}{1.104878in}}%
\pgfpathlineto{\pgfqpoint{2.411779in}{1.112245in}}%
\pgfpathlineto{\pgfqpoint{2.437820in}{1.125506in}}%
\pgfpathlineto{\pgfqpoint{2.452824in}{1.133460in}}%
\pgfpathlineto{\pgfqpoint{2.478161in}{1.146134in}}%
\pgfpathlineto{\pgfqpoint{2.493869in}{1.154214in}}%
\pgfpathlineto{\pgfqpoint{2.494703in}{1.146134in}}%
\pgfpathlineto{\pgfqpoint{2.496803in}{1.125506in}}%
\pgfpathlineto{\pgfqpoint{2.498836in}{1.104878in}}%
\pgfpathlineto{\pgfqpoint{2.500810in}{1.084250in}}%
\pgfpathlineto{\pgfqpoint{2.502730in}{1.063622in}}%
\pgfpathlineto{\pgfqpoint{2.504599in}{1.042994in}}%
\pgfpathlineto{\pgfqpoint{2.506422in}{1.022367in}}%
\pgfpathlineto{\pgfqpoint{2.508202in}{1.001739in}}%
\pgfpathlineto{\pgfqpoint{2.509944in}{0.981111in}}%
\pgfpathlineto{\pgfqpoint{2.511650in}{0.960483in}}%
\pgfpathlineto{\pgfqpoint{2.513322in}{0.939855in}}%
\pgfpathlineto{\pgfqpoint{2.514963in}{0.919227in}}%
\pgfpathlineto{\pgfqpoint{2.516575in}{0.898600in}}%
\pgfpathlineto{\pgfqpoint{2.518160in}{0.877972in}}%
\pgfpathlineto{\pgfqpoint{2.519721in}{0.857344in}}%
\pgfpathlineto{\pgfqpoint{2.521258in}{0.836716in}}%
\pgfpathlineto{\pgfqpoint{2.522773in}{0.816088in}}%
\pgfpathlineto{\pgfqpoint{2.524267in}{0.795460in}}%
\pgfpathlineto{\pgfqpoint{2.525742in}{0.774833in}}%
\pgfpathlineto{\pgfqpoint{2.527200in}{0.754205in}}%
\pgfpathlineto{\pgfqpoint{2.528640in}{0.733577in}}%
\pgfpathlineto{\pgfqpoint{2.530064in}{0.712949in}}%
\pgfpathlineto{\pgfqpoint{2.531473in}{0.692321in}}%
\pgfpathlineto{\pgfqpoint{2.532867in}{0.671694in}}%
\pgfpathlineto{\pgfqpoint{2.534249in}{0.651066in}}%
\pgfpathlineto{\pgfqpoint{2.534915in}{0.641121in}}%
\pgfpathlineto{\pgfqpoint{2.575960in}{0.641121in}}%
\pgfpathlineto{\pgfqpoint{2.617005in}{0.641121in}}%
\pgfpathlineto{\pgfqpoint{2.658051in}{0.641121in}}%
\pgfpathlineto{\pgfqpoint{2.699096in}{0.641121in}}%
\pgfpathlineto{\pgfqpoint{2.740141in}{0.641121in}}%
\pgfpathlineto{\pgfqpoint{2.781187in}{0.641121in}}%
\pgfpathlineto{\pgfqpoint{2.822232in}{0.641121in}}%
\pgfpathlineto{\pgfqpoint{2.863277in}{0.641121in}}%
\pgfpathlineto{\pgfqpoint{2.904323in}{0.641121in}}%
\pgfpathlineto{\pgfqpoint{2.945368in}{0.641121in}}%
\pgfpathlineto{\pgfqpoint{2.986413in}{0.641121in}}%
\pgfpathlineto{\pgfqpoint{3.027459in}{0.641121in}}%
\pgfpathlineto{\pgfqpoint{3.068504in}{0.641121in}}%
\pgfpathlineto{\pgfqpoint{3.109549in}{0.641121in}}%
\pgfpathlineto{\pgfqpoint{3.150595in}{0.641121in}}%
\pgfpathlineto{\pgfqpoint{3.191640in}{0.641121in}}%
\pgfpathlineto{\pgfqpoint{3.232685in}{0.641121in}}%
\pgfpathlineto{\pgfqpoint{3.273731in}{0.641121in}}%
\pgfpathlineto{\pgfqpoint{3.314776in}{0.641121in}}%
\pgfpathlineto{\pgfqpoint{3.355821in}{0.641121in}}%
\pgfpathlineto{\pgfqpoint{3.396867in}{0.641121in}}%
\pgfpathlineto{\pgfqpoint{3.437912in}{0.641121in}}%
\pgfpathlineto{\pgfqpoint{3.478957in}{0.641121in}}%
\pgfpathlineto{\pgfqpoint{3.520003in}{0.641121in}}%
\pgfpathlineto{\pgfqpoint{3.561048in}{0.641121in}}%
\pgfpathlineto{\pgfqpoint{3.602093in}{0.641121in}}%
\pgfpathlineto{\pgfqpoint{3.643139in}{0.641121in}}%
\pgfpathlineto{\pgfqpoint{3.645454in}{0.630438in}}%
\pgfpathlineto{\pgfqpoint{3.649838in}{0.609810in}}%
\pgfpathlineto{\pgfqpoint{3.654079in}{0.589182in}}%
\pgfpathlineto{\pgfqpoint{3.658188in}{0.568554in}}%
\pgfpathlineto{\pgfqpoint{3.662178in}{0.547927in}}%
\pgfpathlineto{\pgfqpoint{3.666056in}{0.527299in}}%
\pgfpathlineto{\pgfqpoint{3.669833in}{0.506671in}}%
\pgfpathlineto{\pgfqpoint{3.673516in}{0.486043in}}%
\pgfpathlineto{\pgfqpoint{3.677113in}{0.465415in}}%
\pgfpathlineto{\pgfqpoint{3.680628in}{0.444787in}}%
\pgfpathlineto{\pgfqpoint{3.684069in}{0.424160in}}%
\pgfpathlineto{\pgfqpoint{3.684184in}{0.423474in}}%
\pgfpathlineto{\pgfqpoint{3.725229in}{0.423474in}}%
\pgfpathlineto{\pgfqpoint{3.766275in}{0.423474in}}%
\pgfpathlineto{\pgfqpoint{3.807320in}{0.423474in}}%
\pgfpathlineto{\pgfqpoint{3.848365in}{0.423474in}}%
\pgfpathlineto{\pgfqpoint{3.889411in}{0.423474in}}%
\pgfpathlineto{\pgfqpoint{3.930456in}{0.423474in}}%
\pgfpathlineto{\pgfqpoint{3.971501in}{0.423474in}}%
\pgfpathlineto{\pgfqpoint{4.012547in}{0.423474in}}%
\pgfpathlineto{\pgfqpoint{4.053592in}{0.423474in}}%
\pgfpathlineto{\pgfqpoint{4.094637in}{0.423474in}}%
\pgfpathlineto{\pgfqpoint{4.135683in}{0.423474in}}%
\pgfpathlineto{\pgfqpoint{4.176728in}{0.423474in}}%
\pgfpathlineto{\pgfqpoint{4.217773in}{0.423474in}}%
\pgfpathlineto{\pgfqpoint{4.258819in}{0.423474in}}%
\pgfpathlineto{\pgfqpoint{4.299864in}{0.423474in}}%
\pgfpathlineto{\pgfqpoint{4.340909in}{0.423474in}}%
\pgfpathlineto{\pgfqpoint{4.381955in}{0.423474in}}%
\pgfpathlineto{\pgfqpoint{4.423000in}{0.423474in}}%
\pgfpathlineto{\pgfqpoint{4.464045in}{0.423474in}}%
\pgfpathlineto{\pgfqpoint{4.505091in}{0.423474in}}%
\pgfpathlineto{\pgfqpoint{4.546136in}{0.423474in}}%
\pgfpathlineto{\pgfqpoint{4.587181in}{0.423474in}}%
\pgfpathlineto{\pgfqpoint{4.628227in}{0.423474in}}%
\pgfpathlineto{\pgfqpoint{4.669272in}{0.423474in}}%
\pgfpathlineto{\pgfqpoint{4.669272in}{0.403532in}}%
\pgfpathlineto{\pgfqpoint{4.669272in}{0.382904in}}%
\pgfpathlineto{\pgfqpoint{4.628227in}{0.382904in}}%
\pgfpathlineto{\pgfqpoint{4.587181in}{0.382904in}}%
\pgfpathlineto{\pgfqpoint{4.546136in}{0.382904in}}%
\pgfpathlineto{\pgfqpoint{4.505091in}{0.382904in}}%
\pgfpathlineto{\pgfqpoint{4.464045in}{0.382904in}}%
\pgfpathlineto{\pgfqpoint{4.423000in}{0.382904in}}%
\pgfpathlineto{\pgfqpoint{4.381955in}{0.382904in}}%
\pgfpathlineto{\pgfqpoint{4.340909in}{0.382904in}}%
\pgfpathlineto{\pgfqpoint{4.299864in}{0.382904in}}%
\pgfpathlineto{\pgfqpoint{4.258819in}{0.382904in}}%
\pgfpathlineto{\pgfqpoint{4.217773in}{0.382904in}}%
\pgfpathlineto{\pgfqpoint{4.176728in}{0.382904in}}%
\pgfpathlineto{\pgfqpoint{4.135683in}{0.382904in}}%
\pgfpathlineto{\pgfqpoint{4.094637in}{0.382904in}}%
\pgfpathlineto{\pgfqpoint{4.053592in}{0.382904in}}%
\pgfpathlineto{\pgfqpoint{4.012547in}{0.382904in}}%
\pgfpathlineto{\pgfqpoint{3.971501in}{0.382904in}}%
\pgfpathlineto{\pgfqpoint{3.930456in}{0.382904in}}%
\pgfpathlineto{\pgfqpoint{3.889411in}{0.382904in}}%
\pgfpathlineto{\pgfqpoint{3.848365in}{0.382904in}}%
\pgfpathlineto{\pgfqpoint{3.807320in}{0.382904in}}%
\pgfpathlineto{\pgfqpoint{3.766275in}{0.382904in}}%
\pgfpathlineto{\pgfqpoint{3.725229in}{0.382904in}}%
\pgfpathlineto{\pgfqpoint{3.684184in}{0.382904in}}%
\pgfpathlineto{\pgfqpoint{3.679282in}{0.382904in}}%
\pgfpathlineto{\pgfqpoint{3.675709in}{0.403532in}}%
\pgfpathlineto{\pgfqpoint{3.672058in}{0.424160in}}%
\pgfpathlineto{\pgfqpoint{3.668324in}{0.444787in}}%
\pgfpathlineto{\pgfqpoint{3.664501in}{0.465415in}}%
\pgfpathlineto{\pgfqpoint{3.660581in}{0.486043in}}%
\pgfpathlineto{\pgfqpoint{3.656558in}{0.506671in}}%
\pgfpathlineto{\pgfqpoint{3.652422in}{0.527299in}}%
\pgfpathlineto{\pgfqpoint{3.648165in}{0.547927in}}%
\pgfpathlineto{\pgfqpoint{3.643775in}{0.568554in}}%
\pgfpathlineto{\pgfqpoint{3.643139in}{0.571523in}}%
\pgfpathlineto{\pgfqpoint{3.602093in}{0.571523in}}%
\pgfpathlineto{\pgfqpoint{3.561048in}{0.571523in}}%
\pgfpathlineto{\pgfqpoint{3.520003in}{0.571523in}}%
\pgfpathlineto{\pgfqpoint{3.478957in}{0.571523in}}%
\pgfpathlineto{\pgfqpoint{3.437912in}{0.571523in}}%
\pgfpathlineto{\pgfqpoint{3.396867in}{0.571523in}}%
\pgfpathlineto{\pgfqpoint{3.355821in}{0.571523in}}%
\pgfpathlineto{\pgfqpoint{3.314776in}{0.571523in}}%
\pgfpathlineto{\pgfqpoint{3.273731in}{0.571523in}}%
\pgfpathlineto{\pgfqpoint{3.232685in}{0.571523in}}%
\pgfpathlineto{\pgfqpoint{3.191640in}{0.571523in}}%
\pgfpathlineto{\pgfqpoint{3.150595in}{0.571523in}}%
\pgfpathlineto{\pgfqpoint{3.109549in}{0.571523in}}%
\pgfpathlineto{\pgfqpoint{3.068504in}{0.571523in}}%
\pgfpathlineto{\pgfqpoint{3.027459in}{0.571523in}}%
\pgfpathlineto{\pgfqpoint{2.986413in}{0.571523in}}%
\pgfpathlineto{\pgfqpoint{2.945368in}{0.571523in}}%
\pgfpathlineto{\pgfqpoint{2.904323in}{0.571523in}}%
\pgfpathlineto{\pgfqpoint{2.863277in}{0.571523in}}%
\pgfpathlineto{\pgfqpoint{2.822232in}{0.571523in}}%
\pgfpathlineto{\pgfqpoint{2.781187in}{0.571523in}}%
\pgfpathlineto{\pgfqpoint{2.740141in}{0.571523in}}%
\pgfpathlineto{\pgfqpoint{2.699096in}{0.571523in}}%
\pgfpathlineto{\pgfqpoint{2.658051in}{0.571523in}}%
\pgfpathlineto{\pgfqpoint{2.617005in}{0.571523in}}%
\pgfpathlineto{\pgfqpoint{2.575960in}{0.571523in}}%
\pgfpathlineto{\pgfqpoint{2.534915in}{0.571523in}}%
\pgfpathlineto{\pgfqpoint{2.533702in}{0.589182in}}%
\pgfpathlineto{\pgfqpoint{2.532276in}{0.609810in}}%
\pgfpathlineto{\pgfqpoint{2.530835in}{0.630438in}}%
\pgfpathlineto{\pgfqpoint{2.529379in}{0.651066in}}%
\pgfpathlineto{\pgfqpoint{2.527907in}{0.671694in}}%
\pgfpathlineto{\pgfqpoint{2.526418in}{0.692321in}}%
\pgfpathlineto{\pgfqpoint{2.524911in}{0.712949in}}%
\pgfpathlineto{\pgfqpoint{2.523385in}{0.733577in}}%
\pgfpathlineto{\pgfqpoint{2.521839in}{0.754205in}}%
\pgfpathlineto{\pgfqpoint{2.520272in}{0.774833in}}%
\pgfpathlineto{\pgfqpoint{2.518682in}{0.795460in}}%
\pgfpathlineto{\pgfqpoint{2.517068in}{0.816088in}}%
\pgfpathlineto{\pgfqpoint{2.515428in}{0.836716in}}%
\pgfpathlineto{\pgfqpoint{2.513760in}{0.857344in}}%
\pgfpathlineto{\pgfqpoint{2.512063in}{0.877972in}}%
\pgfpathlineto{\pgfqpoint{2.510335in}{0.898600in}}%
\pgfpathlineto{\pgfqpoint{2.508573in}{0.919227in}}%
\pgfpathlineto{\pgfqpoint{2.506775in}{0.939855in}}%
\pgfpathlineto{\pgfqpoint{2.504937in}{0.960483in}}%
\pgfpathlineto{\pgfqpoint{2.503058in}{0.981111in}}%
\pgfpathlineto{\pgfqpoint{2.501133in}{1.001739in}}%
\pgfpathlineto{\pgfqpoint{2.499160in}{1.022367in}}%
\pgfpathlineto{\pgfqpoint{2.497133in}{1.042994in}}%
\pgfpathlineto{\pgfqpoint{2.495048in}{1.063622in}}%
\pgfpathlineto{\pgfqpoint{2.493869in}{1.075096in}}%
\pgfpathlineto{\pgfqpoint{2.469056in}{1.063622in}}%
\pgfpathlineto{\pgfqpoint{2.452824in}{1.056301in}}%
\pgfpathlineto{\pgfqpoint{2.423488in}{1.042994in}}%
\pgfpathlineto{\pgfqpoint{2.411779in}{1.037870in}}%
\pgfpathlineto{\pgfqpoint{2.370733in}{1.022993in}}%
\pgfpathlineto{\pgfqpoint{2.367989in}{1.022367in}}%
\pgfpathlineto{\pgfqpoint{2.329688in}{1.014037in}}%
\pgfpathlineto{\pgfqpoint{2.288643in}{1.011327in}}%
\pgfpathlineto{\pgfqpoint{2.247597in}{1.013505in}}%
\pgfpathlineto{\pgfqpoint{2.206552in}{1.017665in}}%
\pgfpathlineto{\pgfqpoint{2.165507in}{1.020177in}}%
\pgfpathlineto{\pgfqpoint{2.124461in}{1.017594in}}%
\pgfpathlineto{\pgfqpoint{2.083416in}{1.007383in}}%
\pgfpathlineto{\pgfqpoint{2.070995in}{1.001739in}}%
\pgfpathlineto{\pgfqpoint{2.042371in}{0.988713in}}%
\pgfpathlineto{\pgfqpoint{2.030369in}{0.981111in}}%
\pgfpathlineto{\pgfqpoint{2.001325in}{0.962487in}}%
\pgfpathlineto{\pgfqpoint{1.998657in}{0.960483in}}%
\pgfpathlineto{\pgfqpoint{1.971433in}{0.939855in}}%
\pgfpathlineto{\pgfqpoint{1.960280in}{0.931249in}}%
\pgfpathlineto{\pgfqpoint{1.945159in}{0.919227in}}%
\pgfpathlineto{\pgfqpoint{1.919911in}{0.898600in}}%
\pgfpathlineto{\pgfqpoint{1.919235in}{0.898042in}}%
\pgfpathlineto{\pgfqpoint{1.893083in}{0.877972in}}%
\pgfpathlineto{\pgfqpoint{1.878189in}{0.866175in}}%
\pgfpathlineto{\pgfqpoint{1.865186in}{0.857344in}}%
\pgfpathlineto{\pgfqpoint{1.837144in}{0.837675in}}%
\pgfpathlineto{\pgfqpoint{1.835468in}{0.836716in}}%
\pgfpathlineto{\pgfqpoint{1.800242in}{0.816088in}}%
\pgfpathlineto{\pgfqpoint{1.796099in}{0.813597in}}%
\pgfpathlineto{\pgfqpoint{1.759209in}{0.795460in}}%
\pgfpathlineto{\pgfqpoint{1.755053in}{0.793373in}}%
\pgfpathlineto{\pgfqpoint{1.714008in}{0.775690in}}%
\pgfpathlineto{\pgfqpoint{1.711850in}{0.774833in}}%
\pgfpathlineto{\pgfqpoint{1.672963in}{0.759396in}}%
\pgfpathlineto{\pgfqpoint{1.659059in}{0.754205in}}%
\pgfpathlineto{\pgfqpoint{1.631917in}{0.744216in}}%
\pgfpathlineto{\pgfqpoint{1.598350in}{0.733577in}}%
\pgfpathlineto{\pgfqpoint{1.590872in}{0.731269in}}%
\pgfpathlineto{\pgfqpoint{1.549827in}{0.722692in}}%
\pgfpathlineto{\pgfqpoint{1.508781in}{0.720187in}}%
\pgfpathlineto{\pgfqpoint{1.467736in}{0.724512in}}%
\pgfpathlineto{\pgfqpoint{1.431002in}{0.733577in}}%
\pgfpathlineto{\pgfqpoint{1.426691in}{0.734641in}}%
\pgfpathlineto{\pgfqpoint{1.385645in}{0.748072in}}%
\pgfpathlineto{\pgfqpoint{1.365753in}{0.754205in}}%
\pgfpathlineto{\pgfqpoint{1.344600in}{0.760785in}}%
\pgfpathlineto{\pgfqpoint{1.303555in}{0.768456in}}%
\pgfpathlineto{\pgfqpoint{1.262509in}{0.767234in}}%
\pgfpathlineto{\pgfqpoint{1.221464in}{0.754629in}}%
\pgfpathlineto{\pgfqpoint{1.220745in}{0.754205in}}%
\pgfpathlineto{\pgfqpoint{1.185799in}{0.733577in}}%
\pgfpathlineto{\pgfqpoint{1.180419in}{0.730389in}}%
\pgfpathlineto{\pgfqpoint{1.159376in}{0.712949in}}%
\pgfpathlineto{\pgfqpoint{1.139373in}{0.696117in}}%
\pgfpathlineto{\pgfqpoint{1.135526in}{0.692321in}}%
\pgfpathlineto{\pgfqpoint{1.114717in}{0.671694in}}%
\pgfpathlineto{\pgfqpoint{1.098328in}{0.655129in}}%
\pgfpathlineto{\pgfqpoint{1.094448in}{0.651066in}}%
\pgfpathlineto{\pgfqpoint{1.074887in}{0.630438in}}%
\pgfpathlineto{\pgfqpoint{1.057283in}{0.611412in}}%
\pgfpathlineto{\pgfqpoint{1.055710in}{0.609810in}}%
\pgfpathlineto{\pgfqpoint{1.035574in}{0.589182in}}%
\pgfpathlineto{\pgfqpoint{1.016237in}{0.568769in}}%
\pgfpathlineto{\pgfqpoint{1.016005in}{0.568554in}}%
\pgfpathlineto{\pgfqpoint{0.993826in}{0.547927in}}%
\pgfpathlineto{\pgfqpoint{0.975192in}{0.530055in}}%
\pgfpathlineto{\pgfqpoint{0.971738in}{0.527299in}}%
\pgfpathlineto{\pgfqpoint{0.946181in}{0.506671in}}%
\pgfpathlineto{\pgfqpoint{0.934147in}{0.496741in}}%
\pgfpathlineto{\pgfqpoint{0.918096in}{0.486043in}}%
\pgfpathlineto{\pgfqpoint{0.893101in}{0.469071in}}%
\pgfpathlineto{\pgfqpoint{0.886332in}{0.465415in}}%
\pgfpathlineto{\pgfqpoint{0.852056in}{0.446756in}}%
\pgfpathlineto{\pgfqpoint{0.847386in}{0.444787in}}%
\pgfpathlineto{\pgfqpoint{0.811011in}{0.429521in}}%
\pgfpathlineto{\pgfqpoint{0.792702in}{0.424160in}}%
\pgfpathlineto{\pgfqpoint{0.769965in}{0.417613in}}%
\pgfpathlineto{\pgfqpoint{0.728920in}{0.412044in}}%
\pgfpathlineto{\pgfqpoint{0.687875in}{0.413990in}}%
\pgfpathlineto{\pgfqpoint{0.646829in}{0.423999in}}%
\pgfpathclose%
\pgfusepath{stroke,fill}%
\end{pgfscope}%
\begin{pgfscope}%
\pgfpathrectangle{\pgfqpoint{0.605784in}{0.382904in}}{\pgfqpoint{4.063488in}{2.042155in}}%
\pgfusepath{clip}%
\pgfsetbuttcap%
\pgfsetroundjoin%
\definecolor{currentfill}{rgb}{0.140536,0.530132,0.555659}%
\pgfsetfillcolor{currentfill}%
\pgfsetlinewidth{1.003750pt}%
\definecolor{currentstroke}{rgb}{0.140536,0.530132,0.555659}%
\pgfsetstrokecolor{currentstroke}%
\pgfsetdash{}{0pt}%
\pgfpathmoveto{\pgfqpoint{0.638148in}{2.053758in}}%
\pgfpathlineto{\pgfqpoint{0.605784in}{2.071832in}}%
\pgfpathlineto{\pgfqpoint{0.605784in}{2.074386in}}%
\pgfpathlineto{\pgfqpoint{0.605784in}{2.095013in}}%
\pgfpathlineto{\pgfqpoint{0.605784in}{2.115641in}}%
\pgfpathlineto{\pgfqpoint{0.605784in}{2.136269in}}%
\pgfpathlineto{\pgfqpoint{0.605784in}{2.139500in}}%
\pgfpathlineto{\pgfqpoint{0.611639in}{2.136269in}}%
\pgfpathlineto{\pgfqpoint{0.646829in}{2.116838in}}%
\pgfpathlineto{\pgfqpoint{0.648890in}{2.115641in}}%
\pgfpathlineto{\pgfqpoint{0.684208in}{2.095013in}}%
\pgfpathlineto{\pgfqpoint{0.687875in}{2.092860in}}%
\pgfpathlineto{\pgfqpoint{0.725247in}{2.074386in}}%
\pgfpathlineto{\pgfqpoint{0.728920in}{2.072549in}}%
\pgfpathlineto{\pgfqpoint{0.769965in}{2.060989in}}%
\pgfpathlineto{\pgfqpoint{0.811011in}{2.062486in}}%
\pgfpathlineto{\pgfqpoint{0.840021in}{2.074386in}}%
\pgfpathlineto{\pgfqpoint{0.852056in}{2.079300in}}%
\pgfpathlineto{\pgfqpoint{0.872411in}{2.095013in}}%
\pgfpathlineto{\pgfqpoint{0.893101in}{2.111111in}}%
\pgfpathlineto{\pgfqpoint{0.897341in}{2.115641in}}%
\pgfpathlineto{\pgfqpoint{0.916649in}{2.136269in}}%
\pgfpathlineto{\pgfqpoint{0.934147in}{2.155225in}}%
\pgfpathlineto{\pgfqpoint{0.935478in}{2.156897in}}%
\pgfpathlineto{\pgfqpoint{0.951943in}{2.177525in}}%
\pgfpathlineto{\pgfqpoint{0.968148in}{2.198153in}}%
\pgfpathlineto{\pgfqpoint{0.975192in}{2.207158in}}%
\pgfpathlineto{\pgfqpoint{0.983993in}{2.218780in}}%
\pgfpathlineto{\pgfqpoint{0.999482in}{2.239408in}}%
\pgfpathlineto{\pgfqpoint{1.014708in}{2.260036in}}%
\pgfpathlineto{\pgfqpoint{1.016237in}{2.262105in}}%
\pgfpathlineto{\pgfqpoint{1.030743in}{2.280664in}}%
\pgfpathlineto{\pgfqpoint{1.046585in}{2.301292in}}%
\pgfpathlineto{\pgfqpoint{1.057283in}{2.315391in}}%
\pgfpathlineto{\pgfqpoint{1.062908in}{2.321920in}}%
\pgfpathlineto{\pgfqpoint{1.080568in}{2.342547in}}%
\pgfpathlineto{\pgfqpoint{1.097849in}{2.363175in}}%
\pgfpathlineto{\pgfqpoint{1.098328in}{2.363747in}}%
\pgfpathlineto{\pgfqpoint{1.118550in}{2.383803in}}%
\pgfpathlineto{\pgfqpoint{1.138867in}{2.404431in}}%
\pgfpathlineto{\pgfqpoint{1.139373in}{2.404945in}}%
\pgfpathlineto{\pgfqpoint{1.164497in}{2.425059in}}%
\pgfpathlineto{\pgfqpoint{1.180419in}{2.425059in}}%
\pgfpathlineto{\pgfqpoint{1.221464in}{2.425059in}}%
\pgfpathlineto{\pgfqpoint{1.262509in}{2.425059in}}%
\pgfpathlineto{\pgfqpoint{1.303555in}{2.425059in}}%
\pgfpathlineto{\pgfqpoint{1.344600in}{2.425059in}}%
\pgfpathlineto{\pgfqpoint{1.385645in}{2.425059in}}%
\pgfpathlineto{\pgfqpoint{1.426691in}{2.425059in}}%
\pgfpathlineto{\pgfqpoint{1.455483in}{2.425059in}}%
\pgfpathlineto{\pgfqpoint{1.467736in}{2.415888in}}%
\pgfpathlineto{\pgfqpoint{1.481285in}{2.404431in}}%
\pgfpathlineto{\pgfqpoint{1.505605in}{2.383803in}}%
\pgfpathlineto{\pgfqpoint{1.508781in}{2.381085in}}%
\pgfpathlineto{\pgfqpoint{1.530504in}{2.363175in}}%
\pgfpathlineto{\pgfqpoint{1.549827in}{2.347149in}}%
\pgfpathlineto{\pgfqpoint{1.556940in}{2.342547in}}%
\pgfpathlineto{\pgfqpoint{1.588358in}{2.321920in}}%
\pgfpathlineto{\pgfqpoint{1.590872in}{2.320242in}}%
\pgfpathlineto{\pgfqpoint{1.631917in}{2.306273in}}%
\pgfpathlineto{\pgfqpoint{1.672963in}{2.309907in}}%
\pgfpathlineto{\pgfqpoint{1.694219in}{2.321920in}}%
\pgfpathlineto{\pgfqpoint{1.714008in}{2.333090in}}%
\pgfpathlineto{\pgfqpoint{1.723393in}{2.342547in}}%
\pgfpathlineto{\pgfqpoint{1.743841in}{2.363175in}}%
\pgfpathlineto{\pgfqpoint{1.755053in}{2.374517in}}%
\pgfpathlineto{\pgfqpoint{1.761936in}{2.383803in}}%
\pgfpathlineto{\pgfqpoint{1.777228in}{2.404431in}}%
\pgfpathlineto{\pgfqpoint{1.792377in}{2.425059in}}%
\pgfpathlineto{\pgfqpoint{1.796099in}{2.425059in}}%
\pgfpathlineto{\pgfqpoint{1.837144in}{2.425059in}}%
\pgfpathlineto{\pgfqpoint{1.837408in}{2.425059in}}%
\pgfpathlineto{\pgfqpoint{1.837144in}{2.424622in}}%
\pgfpathlineto{\pgfqpoint{1.824688in}{2.404431in}}%
\pgfpathlineto{\pgfqpoint{1.811852in}{2.383803in}}%
\pgfpathlineto{\pgfqpoint{1.798893in}{2.363175in}}%
\pgfpathlineto{\pgfqpoint{1.796099in}{2.358704in}}%
\pgfpathlineto{\pgfqpoint{1.784600in}{2.342547in}}%
\pgfpathlineto{\pgfqpoint{1.769847in}{2.321920in}}%
\pgfpathlineto{\pgfqpoint{1.755053in}{2.301423in}}%
\pgfpathlineto{\pgfqpoint{1.754926in}{2.301292in}}%
\pgfpathlineto{\pgfqpoint{1.735195in}{2.280664in}}%
\pgfpathlineto{\pgfqpoint{1.715243in}{2.260036in}}%
\pgfpathlineto{\pgfqpoint{1.714008in}{2.258749in}}%
\pgfpathlineto{\pgfqpoint{1.680332in}{2.239408in}}%
\pgfpathlineto{\pgfqpoint{1.672963in}{2.235170in}}%
\pgfpathlineto{\pgfqpoint{1.631917in}{2.232110in}}%
\pgfpathlineto{\pgfqpoint{1.612650in}{2.239408in}}%
\pgfpathlineto{\pgfqpoint{1.590872in}{2.247428in}}%
\pgfpathlineto{\pgfqpoint{1.573000in}{2.260036in}}%
\pgfpathlineto{\pgfqpoint{1.549827in}{2.276128in}}%
\pgfpathlineto{\pgfqpoint{1.544670in}{2.280664in}}%
\pgfpathlineto{\pgfqpoint{1.520912in}{2.301292in}}%
\pgfpathlineto{\pgfqpoint{1.508781in}{2.311728in}}%
\pgfpathlineto{\pgfqpoint{1.497288in}{2.321920in}}%
\pgfpathlineto{\pgfqpoint{1.473904in}{2.342547in}}%
\pgfpathlineto{\pgfqpoint{1.467736in}{2.347943in}}%
\pgfpathlineto{\pgfqpoint{1.447910in}{2.363175in}}%
\pgfpathlineto{\pgfqpoint{1.426691in}{2.379474in}}%
\pgfpathlineto{\pgfqpoint{1.419080in}{2.383803in}}%
\pgfpathlineto{\pgfqpoint{1.385645in}{2.402763in}}%
\pgfpathlineto{\pgfqpoint{1.380480in}{2.404431in}}%
\pgfpathlineto{\pgfqpoint{1.344600in}{2.415906in}}%
\pgfpathlineto{\pgfqpoint{1.303555in}{2.418687in}}%
\pgfpathlineto{\pgfqpoint{1.262509in}{2.411597in}}%
\pgfpathlineto{\pgfqpoint{1.244225in}{2.404431in}}%
\pgfpathlineto{\pgfqpoint{1.221464in}{2.395437in}}%
\pgfpathlineto{\pgfqpoint{1.202177in}{2.383803in}}%
\pgfpathlineto{\pgfqpoint{1.180419in}{2.370742in}}%
\pgfpathlineto{\pgfqpoint{1.171103in}{2.363175in}}%
\pgfpathlineto{\pgfqpoint{1.145448in}{2.342547in}}%
\pgfpathlineto{\pgfqpoint{1.139373in}{2.337697in}}%
\pgfpathlineto{\pgfqpoint{1.123860in}{2.321920in}}%
\pgfpathlineto{\pgfqpoint{1.103172in}{2.301292in}}%
\pgfpathlineto{\pgfqpoint{1.098328in}{2.296492in}}%
\pgfpathlineto{\pgfqpoint{1.085141in}{2.280664in}}%
\pgfpathlineto{\pgfqpoint{1.067653in}{2.260036in}}%
\pgfpathlineto{\pgfqpoint{1.057283in}{2.247968in}}%
\pgfpathlineto{\pgfqpoint{1.050845in}{2.239408in}}%
\pgfpathlineto{\pgfqpoint{1.035232in}{2.218780in}}%
\pgfpathlineto{\pgfqpoint{1.019289in}{2.198153in}}%
\pgfpathlineto{\pgfqpoint{1.016237in}{2.194211in}}%
\pgfpathlineto{\pgfqpoint{1.004101in}{2.177525in}}%
\pgfpathlineto{\pgfqpoint{0.988894in}{2.156897in}}%
\pgfpathlineto{\pgfqpoint{0.975192in}{2.138607in}}%
\pgfpathlineto{\pgfqpoint{0.973390in}{2.136269in}}%
\pgfpathlineto{\pgfqpoint{0.957537in}{2.115641in}}%
\pgfpathlineto{\pgfqpoint{0.941421in}{2.095013in}}%
\pgfpathlineto{\pgfqpoint{0.934147in}{2.085748in}}%
\pgfpathlineto{\pgfqpoint{0.923821in}{2.074386in}}%
\pgfpathlineto{\pgfqpoint{0.904955in}{2.053758in}}%
\pgfpathlineto{\pgfqpoint{0.893101in}{2.040903in}}%
\pgfpathlineto{\pgfqpoint{0.883234in}{2.033130in}}%
\pgfpathlineto{\pgfqpoint{0.856939in}{2.012502in}}%
\pgfpathlineto{\pgfqpoint{0.852056in}{2.008671in}}%
\pgfpathlineto{\pgfqpoint{0.811011in}{1.991937in}}%
\pgfpathlineto{\pgfqpoint{0.808451in}{1.991874in}}%
\pgfpathlineto{\pgfqpoint{0.769965in}{1.990913in}}%
\pgfpathlineto{\pgfqpoint{0.766815in}{1.991874in}}%
\pgfpathlineto{\pgfqpoint{0.728920in}{2.003167in}}%
\pgfpathlineto{\pgfqpoint{0.710922in}{2.012502in}}%
\pgfpathlineto{\pgfqpoint{0.687875in}{2.024309in}}%
\pgfpathlineto{\pgfqpoint{0.673199in}{2.033130in}}%
\pgfpathlineto{\pgfqpoint{0.646829in}{2.048907in}}%
\pgfpathclose%
\pgfusepath{stroke,fill}%
\end{pgfscope}%
\begin{pgfscope}%
\pgfpathrectangle{\pgfqpoint{0.605784in}{0.382904in}}{\pgfqpoint{4.063488in}{2.042155in}}%
\pgfusepath{clip}%
\pgfsetbuttcap%
\pgfsetroundjoin%
\definecolor{currentfill}{rgb}{0.140536,0.530132,0.555659}%
\pgfsetfillcolor{currentfill}%
\pgfsetlinewidth{1.003750pt}%
\definecolor{currentstroke}{rgb}{0.140536,0.530132,0.555659}%
\pgfsetstrokecolor{currentstroke}%
\pgfsetdash{}{0pt}%
\pgfpathmoveto{\pgfqpoint{2.531100in}{2.177525in}}%
\pgfpathlineto{\pgfqpoint{2.527434in}{2.198153in}}%
\pgfpathlineto{\pgfqpoint{2.524016in}{2.218780in}}%
\pgfpathlineto{\pgfqpoint{2.520812in}{2.239408in}}%
\pgfpathlineto{\pgfqpoint{2.517794in}{2.260036in}}%
\pgfpathlineto{\pgfqpoint{2.514939in}{2.280664in}}%
\pgfpathlineto{\pgfqpoint{2.512229in}{2.301292in}}%
\pgfpathlineto{\pgfqpoint{2.509647in}{2.321920in}}%
\pgfpathlineto{\pgfqpoint{2.507178in}{2.342547in}}%
\pgfpathlineto{\pgfqpoint{2.504811in}{2.363175in}}%
\pgfpathlineto{\pgfqpoint{2.502535in}{2.383803in}}%
\pgfpathlineto{\pgfqpoint{2.500342in}{2.404431in}}%
\pgfpathlineto{\pgfqpoint{2.498223in}{2.425059in}}%
\pgfpathlineto{\pgfqpoint{2.506707in}{2.425059in}}%
\pgfpathlineto{\pgfqpoint{2.509105in}{2.404431in}}%
\pgfpathlineto{\pgfqpoint{2.511597in}{2.383803in}}%
\pgfpathlineto{\pgfqpoint{2.514192in}{2.363175in}}%
\pgfpathlineto{\pgfqpoint{2.516902in}{2.342547in}}%
\pgfpathlineto{\pgfqpoint{2.519739in}{2.321920in}}%
\pgfpathlineto{\pgfqpoint{2.522720in}{2.301292in}}%
\pgfpathlineto{\pgfqpoint{2.525861in}{2.280664in}}%
\pgfpathlineto{\pgfqpoint{2.529183in}{2.260036in}}%
\pgfpathlineto{\pgfqpoint{2.532710in}{2.239408in}}%
\pgfpathlineto{\pgfqpoint{2.534915in}{2.227089in}}%
\pgfpathlineto{\pgfqpoint{2.575960in}{2.227089in}}%
\pgfpathlineto{\pgfqpoint{2.617005in}{2.227089in}}%
\pgfpathlineto{\pgfqpoint{2.658051in}{2.227089in}}%
\pgfpathlineto{\pgfqpoint{2.699096in}{2.227089in}}%
\pgfpathlineto{\pgfqpoint{2.740141in}{2.227089in}}%
\pgfpathlineto{\pgfqpoint{2.781187in}{2.227089in}}%
\pgfpathlineto{\pgfqpoint{2.822232in}{2.227089in}}%
\pgfpathlineto{\pgfqpoint{2.863277in}{2.227089in}}%
\pgfpathlineto{\pgfqpoint{2.904323in}{2.227089in}}%
\pgfpathlineto{\pgfqpoint{2.945368in}{2.227089in}}%
\pgfpathlineto{\pgfqpoint{2.986413in}{2.227089in}}%
\pgfpathlineto{\pgfqpoint{3.027459in}{2.227089in}}%
\pgfpathlineto{\pgfqpoint{3.068504in}{2.227089in}}%
\pgfpathlineto{\pgfqpoint{3.109549in}{2.227089in}}%
\pgfpathlineto{\pgfqpoint{3.150595in}{2.227089in}}%
\pgfpathlineto{\pgfqpoint{3.191640in}{2.227089in}}%
\pgfpathlineto{\pgfqpoint{3.232685in}{2.227089in}}%
\pgfpathlineto{\pgfqpoint{3.273731in}{2.227089in}}%
\pgfpathlineto{\pgfqpoint{3.314776in}{2.227089in}}%
\pgfpathlineto{\pgfqpoint{3.355821in}{2.227089in}}%
\pgfpathlineto{\pgfqpoint{3.396867in}{2.227089in}}%
\pgfpathlineto{\pgfqpoint{3.437912in}{2.227089in}}%
\pgfpathlineto{\pgfqpoint{3.478957in}{2.227089in}}%
\pgfpathlineto{\pgfqpoint{3.520003in}{2.227089in}}%
\pgfpathlineto{\pgfqpoint{3.561048in}{2.227089in}}%
\pgfpathlineto{\pgfqpoint{3.602093in}{2.227089in}}%
\pgfpathlineto{\pgfqpoint{3.643139in}{2.227089in}}%
\pgfpathlineto{\pgfqpoint{3.644540in}{2.218780in}}%
\pgfpathlineto{\pgfqpoint{3.648040in}{2.198153in}}%
\pgfpathlineto{\pgfqpoint{3.651614in}{2.177525in}}%
\pgfpathlineto{\pgfqpoint{3.655264in}{2.156897in}}%
\pgfpathlineto{\pgfqpoint{3.658998in}{2.136269in}}%
\pgfpathlineto{\pgfqpoint{3.662822in}{2.115641in}}%
\pgfpathlineto{\pgfqpoint{3.666741in}{2.095013in}}%
\pgfpathlineto{\pgfqpoint{3.670765in}{2.074386in}}%
\pgfpathlineto{\pgfqpoint{3.674900in}{2.053758in}}%
\pgfpathlineto{\pgfqpoint{3.679158in}{2.033130in}}%
\pgfpathlineto{\pgfqpoint{3.683547in}{2.012502in}}%
\pgfpathlineto{\pgfqpoint{3.684184in}{2.009533in}}%
\pgfpathlineto{\pgfqpoint{3.725229in}{2.009533in}}%
\pgfpathlineto{\pgfqpoint{3.766275in}{2.009533in}}%
\pgfpathlineto{\pgfqpoint{3.807320in}{2.009533in}}%
\pgfpathlineto{\pgfqpoint{3.848365in}{2.009533in}}%
\pgfpathlineto{\pgfqpoint{3.889411in}{2.009533in}}%
\pgfpathlineto{\pgfqpoint{3.930456in}{2.009533in}}%
\pgfpathlineto{\pgfqpoint{3.971501in}{2.009533in}}%
\pgfpathlineto{\pgfqpoint{4.012547in}{2.009533in}}%
\pgfpathlineto{\pgfqpoint{4.053592in}{2.009533in}}%
\pgfpathlineto{\pgfqpoint{4.094637in}{2.009533in}}%
\pgfpathlineto{\pgfqpoint{4.135683in}{2.009533in}}%
\pgfpathlineto{\pgfqpoint{4.176728in}{2.009533in}}%
\pgfpathlineto{\pgfqpoint{4.217773in}{2.009533in}}%
\pgfpathlineto{\pgfqpoint{4.258819in}{2.009533in}}%
\pgfpathlineto{\pgfqpoint{4.299864in}{2.009533in}}%
\pgfpathlineto{\pgfqpoint{4.340909in}{2.009533in}}%
\pgfpathlineto{\pgfqpoint{4.381955in}{2.009533in}}%
\pgfpathlineto{\pgfqpoint{4.423000in}{2.009533in}}%
\pgfpathlineto{\pgfqpoint{4.464045in}{2.009533in}}%
\pgfpathlineto{\pgfqpoint{4.505091in}{2.009533in}}%
\pgfpathlineto{\pgfqpoint{4.546136in}{2.009533in}}%
\pgfpathlineto{\pgfqpoint{4.587181in}{2.009533in}}%
\pgfpathlineto{\pgfqpoint{4.628227in}{2.009533in}}%
\pgfpathlineto{\pgfqpoint{4.669272in}{2.009533in}}%
\pgfpathlineto{\pgfqpoint{4.669272in}{1.991874in}}%
\pgfpathlineto{\pgfqpoint{4.669272in}{1.971247in}}%
\pgfpathlineto{\pgfqpoint{4.669272in}{1.950619in}}%
\pgfpathlineto{\pgfqpoint{4.669272in}{1.939935in}}%
\pgfpathlineto{\pgfqpoint{4.628227in}{1.939935in}}%
\pgfpathlineto{\pgfqpoint{4.587181in}{1.939935in}}%
\pgfpathlineto{\pgfqpoint{4.546136in}{1.939935in}}%
\pgfpathlineto{\pgfqpoint{4.505091in}{1.939935in}}%
\pgfpathlineto{\pgfqpoint{4.464045in}{1.939935in}}%
\pgfpathlineto{\pgfqpoint{4.423000in}{1.939935in}}%
\pgfpathlineto{\pgfqpoint{4.381955in}{1.939935in}}%
\pgfpathlineto{\pgfqpoint{4.340909in}{1.939935in}}%
\pgfpathlineto{\pgfqpoint{4.299864in}{1.939935in}}%
\pgfpathlineto{\pgfqpoint{4.258819in}{1.939935in}}%
\pgfpathlineto{\pgfqpoint{4.217773in}{1.939935in}}%
\pgfpathlineto{\pgfqpoint{4.176728in}{1.939935in}}%
\pgfpathlineto{\pgfqpoint{4.135683in}{1.939935in}}%
\pgfpathlineto{\pgfqpoint{4.094637in}{1.939935in}}%
\pgfpathlineto{\pgfqpoint{4.053592in}{1.939935in}}%
\pgfpathlineto{\pgfqpoint{4.012547in}{1.939935in}}%
\pgfpathlineto{\pgfqpoint{3.971501in}{1.939935in}}%
\pgfpathlineto{\pgfqpoint{3.930456in}{1.939935in}}%
\pgfpathlineto{\pgfqpoint{3.889411in}{1.939935in}}%
\pgfpathlineto{\pgfqpoint{3.848365in}{1.939935in}}%
\pgfpathlineto{\pgfqpoint{3.807320in}{1.939935in}}%
\pgfpathlineto{\pgfqpoint{3.766275in}{1.939935in}}%
\pgfpathlineto{\pgfqpoint{3.725229in}{1.939935in}}%
\pgfpathlineto{\pgfqpoint{3.684184in}{1.939935in}}%
\pgfpathlineto{\pgfqpoint{3.681868in}{1.950619in}}%
\pgfpathlineto{\pgfqpoint{3.677484in}{1.971247in}}%
\pgfpathlineto{\pgfqpoint{3.673243in}{1.991874in}}%
\pgfpathlineto{\pgfqpoint{3.669134in}{2.012502in}}%
\pgfpathlineto{\pgfqpoint{3.665145in}{2.033130in}}%
\pgfpathlineto{\pgfqpoint{3.661266in}{2.053758in}}%
\pgfpathlineto{\pgfqpoint{3.657489in}{2.074386in}}%
\pgfpathlineto{\pgfqpoint{3.653806in}{2.095013in}}%
\pgfpathlineto{\pgfqpoint{3.650210in}{2.115641in}}%
\pgfpathlineto{\pgfqpoint{3.646694in}{2.136269in}}%
\pgfpathlineto{\pgfqpoint{3.643253in}{2.156897in}}%
\pgfpathlineto{\pgfqpoint{3.643139in}{2.157582in}}%
\pgfpathlineto{\pgfqpoint{3.602093in}{2.157582in}}%
\pgfpathlineto{\pgfqpoint{3.561048in}{2.157582in}}%
\pgfpathlineto{\pgfqpoint{3.520003in}{2.157582in}}%
\pgfpathlineto{\pgfqpoint{3.478957in}{2.157582in}}%
\pgfpathlineto{\pgfqpoint{3.437912in}{2.157582in}}%
\pgfpathlineto{\pgfqpoint{3.396867in}{2.157582in}}%
\pgfpathlineto{\pgfqpoint{3.355821in}{2.157582in}}%
\pgfpathlineto{\pgfqpoint{3.314776in}{2.157582in}}%
\pgfpathlineto{\pgfqpoint{3.273731in}{2.157582in}}%
\pgfpathlineto{\pgfqpoint{3.232685in}{2.157582in}}%
\pgfpathlineto{\pgfqpoint{3.191640in}{2.157582in}}%
\pgfpathlineto{\pgfqpoint{3.150595in}{2.157582in}}%
\pgfpathlineto{\pgfqpoint{3.109549in}{2.157582in}}%
\pgfpathlineto{\pgfqpoint{3.068504in}{2.157582in}}%
\pgfpathlineto{\pgfqpoint{3.027459in}{2.157582in}}%
\pgfpathlineto{\pgfqpoint{2.986413in}{2.157582in}}%
\pgfpathlineto{\pgfqpoint{2.945368in}{2.157582in}}%
\pgfpathlineto{\pgfqpoint{2.904323in}{2.157582in}}%
\pgfpathlineto{\pgfqpoint{2.863277in}{2.157582in}}%
\pgfpathlineto{\pgfqpoint{2.822232in}{2.157582in}}%
\pgfpathlineto{\pgfqpoint{2.781187in}{2.157582in}}%
\pgfpathlineto{\pgfqpoint{2.740141in}{2.157582in}}%
\pgfpathlineto{\pgfqpoint{2.699096in}{2.157582in}}%
\pgfpathlineto{\pgfqpoint{2.658051in}{2.157582in}}%
\pgfpathlineto{\pgfqpoint{2.617005in}{2.157582in}}%
\pgfpathlineto{\pgfqpoint{2.575960in}{2.157582in}}%
\pgfpathlineto{\pgfqpoint{2.534915in}{2.157582in}}%
\pgfpathclose%
\pgfusepath{stroke,fill}%
\end{pgfscope}%
\begin{pgfscope}%
\pgfpathrectangle{\pgfqpoint{0.605784in}{0.382904in}}{\pgfqpoint{4.063488in}{2.042155in}}%
\pgfusepath{clip}%
\pgfsetbuttcap%
\pgfsetroundjoin%
\definecolor{currentfill}{rgb}{0.120092,0.600104,0.542530}%
\pgfsetfillcolor{currentfill}%
\pgfsetlinewidth{1.003750pt}%
\definecolor{currentstroke}{rgb}{0.120092,0.600104,0.542530}%
\pgfsetstrokecolor{currentstroke}%
\pgfsetdash{}{0pt}%
\pgfpathmoveto{\pgfqpoint{0.605784in}{0.403532in}}%
\pgfpathlineto{\pgfqpoint{0.605784in}{0.424160in}}%
\pgfpathlineto{\pgfqpoint{0.605784in}{0.441557in}}%
\pgfpathlineto{\pgfqpoint{0.646453in}{0.424160in}}%
\pgfpathlineto{\pgfqpoint{0.646829in}{0.423999in}}%
\pgfpathlineto{\pgfqpoint{0.687875in}{0.413990in}}%
\pgfpathlineto{\pgfqpoint{0.728920in}{0.412044in}}%
\pgfpathlineto{\pgfqpoint{0.769965in}{0.417613in}}%
\pgfpathlineto{\pgfqpoint{0.792702in}{0.424160in}}%
\pgfpathlineto{\pgfqpoint{0.811011in}{0.429521in}}%
\pgfpathlineto{\pgfqpoint{0.847386in}{0.444787in}}%
\pgfpathlineto{\pgfqpoint{0.852056in}{0.446756in}}%
\pgfpathlineto{\pgfqpoint{0.886332in}{0.465415in}}%
\pgfpathlineto{\pgfqpoint{0.893101in}{0.469071in}}%
\pgfpathlineto{\pgfqpoint{0.918096in}{0.486043in}}%
\pgfpathlineto{\pgfqpoint{0.934147in}{0.496741in}}%
\pgfpathlineto{\pgfqpoint{0.946181in}{0.506671in}}%
\pgfpathlineto{\pgfqpoint{0.971738in}{0.527299in}}%
\pgfpathlineto{\pgfqpoint{0.975192in}{0.530055in}}%
\pgfpathlineto{\pgfqpoint{0.993826in}{0.547927in}}%
\pgfpathlineto{\pgfqpoint{1.016005in}{0.568554in}}%
\pgfpathlineto{\pgfqpoint{1.016237in}{0.568769in}}%
\pgfpathlineto{\pgfqpoint{1.035574in}{0.589182in}}%
\pgfpathlineto{\pgfqpoint{1.055710in}{0.609810in}}%
\pgfpathlineto{\pgfqpoint{1.057283in}{0.611412in}}%
\pgfpathlineto{\pgfqpoint{1.074887in}{0.630438in}}%
\pgfpathlineto{\pgfqpoint{1.094448in}{0.651066in}}%
\pgfpathlineto{\pgfqpoint{1.098328in}{0.655129in}}%
\pgfpathlineto{\pgfqpoint{1.114717in}{0.671694in}}%
\pgfpathlineto{\pgfqpoint{1.135526in}{0.692321in}}%
\pgfpathlineto{\pgfqpoint{1.139373in}{0.696117in}}%
\pgfpathlineto{\pgfqpoint{1.159376in}{0.712949in}}%
\pgfpathlineto{\pgfqpoint{1.180419in}{0.730389in}}%
\pgfpathlineto{\pgfqpoint{1.185799in}{0.733577in}}%
\pgfpathlineto{\pgfqpoint{1.220745in}{0.754205in}}%
\pgfpathlineto{\pgfqpoint{1.221464in}{0.754629in}}%
\pgfpathlineto{\pgfqpoint{1.262509in}{0.767234in}}%
\pgfpathlineto{\pgfqpoint{1.303555in}{0.768456in}}%
\pgfpathlineto{\pgfqpoint{1.344600in}{0.760785in}}%
\pgfpathlineto{\pgfqpoint{1.365753in}{0.754205in}}%
\pgfpathlineto{\pgfqpoint{1.385645in}{0.748072in}}%
\pgfpathlineto{\pgfqpoint{1.426691in}{0.734641in}}%
\pgfpathlineto{\pgfqpoint{1.431002in}{0.733577in}}%
\pgfpathlineto{\pgfqpoint{1.467736in}{0.724512in}}%
\pgfpathlineto{\pgfqpoint{1.508781in}{0.720187in}}%
\pgfpathlineto{\pgfqpoint{1.549827in}{0.722692in}}%
\pgfpathlineto{\pgfqpoint{1.590872in}{0.731269in}}%
\pgfpathlineto{\pgfqpoint{1.598350in}{0.733577in}}%
\pgfpathlineto{\pgfqpoint{1.631917in}{0.744216in}}%
\pgfpathlineto{\pgfqpoint{1.659059in}{0.754205in}}%
\pgfpathlineto{\pgfqpoint{1.672963in}{0.759396in}}%
\pgfpathlineto{\pgfqpoint{1.711850in}{0.774833in}}%
\pgfpathlineto{\pgfqpoint{1.714008in}{0.775690in}}%
\pgfpathlineto{\pgfqpoint{1.755053in}{0.793373in}}%
\pgfpathlineto{\pgfqpoint{1.759209in}{0.795460in}}%
\pgfpathlineto{\pgfqpoint{1.796099in}{0.813597in}}%
\pgfpathlineto{\pgfqpoint{1.800242in}{0.816088in}}%
\pgfpathlineto{\pgfqpoint{1.835468in}{0.836716in}}%
\pgfpathlineto{\pgfqpoint{1.837144in}{0.837675in}}%
\pgfpathlineto{\pgfqpoint{1.865186in}{0.857344in}}%
\pgfpathlineto{\pgfqpoint{1.878189in}{0.866175in}}%
\pgfpathlineto{\pgfqpoint{1.893083in}{0.877972in}}%
\pgfpathlineto{\pgfqpoint{1.919235in}{0.898042in}}%
\pgfpathlineto{\pgfqpoint{1.919911in}{0.898600in}}%
\pgfpathlineto{\pgfqpoint{1.945159in}{0.919227in}}%
\pgfpathlineto{\pgfqpoint{1.960280in}{0.931249in}}%
\pgfpathlineto{\pgfqpoint{1.971433in}{0.939855in}}%
\pgfpathlineto{\pgfqpoint{1.998657in}{0.960483in}}%
\pgfpathlineto{\pgfqpoint{2.001325in}{0.962487in}}%
\pgfpathlineto{\pgfqpoint{2.030369in}{0.981111in}}%
\pgfpathlineto{\pgfqpoint{2.042371in}{0.988713in}}%
\pgfpathlineto{\pgfqpoint{2.070995in}{1.001739in}}%
\pgfpathlineto{\pgfqpoint{2.083416in}{1.007383in}}%
\pgfpathlineto{\pgfqpoint{2.124461in}{1.017594in}}%
\pgfpathlineto{\pgfqpoint{2.165507in}{1.020177in}}%
\pgfpathlineto{\pgfqpoint{2.206552in}{1.017665in}}%
\pgfpathlineto{\pgfqpoint{2.247597in}{1.013505in}}%
\pgfpathlineto{\pgfqpoint{2.288643in}{1.011327in}}%
\pgfpathlineto{\pgfqpoint{2.329688in}{1.014037in}}%
\pgfpathlineto{\pgfqpoint{2.367989in}{1.022367in}}%
\pgfpathlineto{\pgfqpoint{2.370733in}{1.022993in}}%
\pgfpathlineto{\pgfqpoint{2.411779in}{1.037870in}}%
\pgfpathlineto{\pgfqpoint{2.423488in}{1.042994in}}%
\pgfpathlineto{\pgfqpoint{2.452824in}{1.056301in}}%
\pgfpathlineto{\pgfqpoint{2.469056in}{1.063622in}}%
\pgfpathlineto{\pgfqpoint{2.493869in}{1.075096in}}%
\pgfpathlineto{\pgfqpoint{2.495048in}{1.063622in}}%
\pgfpathlineto{\pgfqpoint{2.497133in}{1.042994in}}%
\pgfpathlineto{\pgfqpoint{2.499160in}{1.022367in}}%
\pgfpathlineto{\pgfqpoint{2.501133in}{1.001739in}}%
\pgfpathlineto{\pgfqpoint{2.503058in}{0.981111in}}%
\pgfpathlineto{\pgfqpoint{2.504937in}{0.960483in}}%
\pgfpathlineto{\pgfqpoint{2.506775in}{0.939855in}}%
\pgfpathlineto{\pgfqpoint{2.508573in}{0.919227in}}%
\pgfpathlineto{\pgfqpoint{2.510335in}{0.898600in}}%
\pgfpathlineto{\pgfqpoint{2.512063in}{0.877972in}}%
\pgfpathlineto{\pgfqpoint{2.513760in}{0.857344in}}%
\pgfpathlineto{\pgfqpoint{2.515428in}{0.836716in}}%
\pgfpathlineto{\pgfqpoint{2.517068in}{0.816088in}}%
\pgfpathlineto{\pgfqpoint{2.518682in}{0.795460in}}%
\pgfpathlineto{\pgfqpoint{2.520272in}{0.774833in}}%
\pgfpathlineto{\pgfqpoint{2.521839in}{0.754205in}}%
\pgfpathlineto{\pgfqpoint{2.523385in}{0.733577in}}%
\pgfpathlineto{\pgfqpoint{2.524911in}{0.712949in}}%
\pgfpathlineto{\pgfqpoint{2.526418in}{0.692321in}}%
\pgfpathlineto{\pgfqpoint{2.527907in}{0.671694in}}%
\pgfpathlineto{\pgfqpoint{2.529379in}{0.651066in}}%
\pgfpathlineto{\pgfqpoint{2.530835in}{0.630438in}}%
\pgfpathlineto{\pgfqpoint{2.532276in}{0.609810in}}%
\pgfpathlineto{\pgfqpoint{2.533702in}{0.589182in}}%
\pgfpathlineto{\pgfqpoint{2.534915in}{0.571523in}}%
\pgfpathlineto{\pgfqpoint{2.575960in}{0.571523in}}%
\pgfpathlineto{\pgfqpoint{2.617005in}{0.571523in}}%
\pgfpathlineto{\pgfqpoint{2.658051in}{0.571523in}}%
\pgfpathlineto{\pgfqpoint{2.699096in}{0.571523in}}%
\pgfpathlineto{\pgfqpoint{2.740141in}{0.571523in}}%
\pgfpathlineto{\pgfqpoint{2.781187in}{0.571523in}}%
\pgfpathlineto{\pgfqpoint{2.822232in}{0.571523in}}%
\pgfpathlineto{\pgfqpoint{2.863277in}{0.571523in}}%
\pgfpathlineto{\pgfqpoint{2.904323in}{0.571523in}}%
\pgfpathlineto{\pgfqpoint{2.945368in}{0.571523in}}%
\pgfpathlineto{\pgfqpoint{2.986413in}{0.571523in}}%
\pgfpathlineto{\pgfqpoint{3.027459in}{0.571523in}}%
\pgfpathlineto{\pgfqpoint{3.068504in}{0.571523in}}%
\pgfpathlineto{\pgfqpoint{3.109549in}{0.571523in}}%
\pgfpathlineto{\pgfqpoint{3.150595in}{0.571523in}}%
\pgfpathlineto{\pgfqpoint{3.191640in}{0.571523in}}%
\pgfpathlineto{\pgfqpoint{3.232685in}{0.571523in}}%
\pgfpathlineto{\pgfqpoint{3.273731in}{0.571523in}}%
\pgfpathlineto{\pgfqpoint{3.314776in}{0.571523in}}%
\pgfpathlineto{\pgfqpoint{3.355821in}{0.571523in}}%
\pgfpathlineto{\pgfqpoint{3.396867in}{0.571523in}}%
\pgfpathlineto{\pgfqpoint{3.437912in}{0.571523in}}%
\pgfpathlineto{\pgfqpoint{3.478957in}{0.571523in}}%
\pgfpathlineto{\pgfqpoint{3.520003in}{0.571523in}}%
\pgfpathlineto{\pgfqpoint{3.561048in}{0.571523in}}%
\pgfpathlineto{\pgfqpoint{3.602093in}{0.571523in}}%
\pgfpathlineto{\pgfqpoint{3.643139in}{0.571523in}}%
\pgfpathlineto{\pgfqpoint{3.643775in}{0.568554in}}%
\pgfpathlineto{\pgfqpoint{3.648165in}{0.547927in}}%
\pgfpathlineto{\pgfqpoint{3.652422in}{0.527299in}}%
\pgfpathlineto{\pgfqpoint{3.656558in}{0.506671in}}%
\pgfpathlineto{\pgfqpoint{3.660581in}{0.486043in}}%
\pgfpathlineto{\pgfqpoint{3.664501in}{0.465415in}}%
\pgfpathlineto{\pgfqpoint{3.668324in}{0.444787in}}%
\pgfpathlineto{\pgfqpoint{3.672058in}{0.424160in}}%
\pgfpathlineto{\pgfqpoint{3.675709in}{0.403532in}}%
\pgfpathlineto{\pgfqpoint{3.679282in}{0.382904in}}%
\pgfpathlineto{\pgfqpoint{3.667817in}{0.382904in}}%
\pgfpathlineto{\pgfqpoint{3.663977in}{0.403532in}}%
\pgfpathlineto{\pgfqpoint{3.660047in}{0.424160in}}%
\pgfpathlineto{\pgfqpoint{3.656020in}{0.444787in}}%
\pgfpathlineto{\pgfqpoint{3.651889in}{0.465415in}}%
\pgfpathlineto{\pgfqpoint{3.647646in}{0.486043in}}%
\pgfpathlineto{\pgfqpoint{3.643282in}{0.506671in}}%
\pgfpathlineto{\pgfqpoint{3.643139in}{0.507348in}}%
\pgfpathlineto{\pgfqpoint{3.602093in}{0.507348in}}%
\pgfpathlineto{\pgfqpoint{3.561048in}{0.507348in}}%
\pgfpathlineto{\pgfqpoint{3.520003in}{0.507348in}}%
\pgfpathlineto{\pgfqpoint{3.478957in}{0.507348in}}%
\pgfpathlineto{\pgfqpoint{3.437912in}{0.507348in}}%
\pgfpathlineto{\pgfqpoint{3.396867in}{0.507348in}}%
\pgfpathlineto{\pgfqpoint{3.355821in}{0.507348in}}%
\pgfpathlineto{\pgfqpoint{3.314776in}{0.507348in}}%
\pgfpathlineto{\pgfqpoint{3.273731in}{0.507348in}}%
\pgfpathlineto{\pgfqpoint{3.232685in}{0.507348in}}%
\pgfpathlineto{\pgfqpoint{3.191640in}{0.507348in}}%
\pgfpathlineto{\pgfqpoint{3.150595in}{0.507348in}}%
\pgfpathlineto{\pgfqpoint{3.109549in}{0.507348in}}%
\pgfpathlineto{\pgfqpoint{3.068504in}{0.507348in}}%
\pgfpathlineto{\pgfqpoint{3.027459in}{0.507348in}}%
\pgfpathlineto{\pgfqpoint{2.986413in}{0.507348in}}%
\pgfpathlineto{\pgfqpoint{2.945368in}{0.507348in}}%
\pgfpathlineto{\pgfqpoint{2.904323in}{0.507348in}}%
\pgfpathlineto{\pgfqpoint{2.863277in}{0.507348in}}%
\pgfpathlineto{\pgfqpoint{2.822232in}{0.507348in}}%
\pgfpathlineto{\pgfqpoint{2.781187in}{0.507348in}}%
\pgfpathlineto{\pgfqpoint{2.740141in}{0.507348in}}%
\pgfpathlineto{\pgfqpoint{2.699096in}{0.507348in}}%
\pgfpathlineto{\pgfqpoint{2.658051in}{0.507348in}}%
\pgfpathlineto{\pgfqpoint{2.617005in}{0.507348in}}%
\pgfpathlineto{\pgfqpoint{2.575960in}{0.507348in}}%
\pgfpathlineto{\pgfqpoint{2.534915in}{0.507348in}}%
\pgfpathlineto{\pgfqpoint{2.533514in}{0.527299in}}%
\pgfpathlineto{\pgfqpoint{2.532053in}{0.547927in}}%
\pgfpathlineto{\pgfqpoint{2.530578in}{0.568554in}}%
\pgfpathlineto{\pgfqpoint{2.529086in}{0.589182in}}%
\pgfpathlineto{\pgfqpoint{2.527578in}{0.609810in}}%
\pgfpathlineto{\pgfqpoint{2.526053in}{0.630438in}}%
\pgfpathlineto{\pgfqpoint{2.524509in}{0.651066in}}%
\pgfpathlineto{\pgfqpoint{2.522947in}{0.671694in}}%
\pgfpathlineto{\pgfqpoint{2.521364in}{0.692321in}}%
\pgfpathlineto{\pgfqpoint{2.519759in}{0.712949in}}%
\pgfpathlineto{\pgfqpoint{2.518131in}{0.733577in}}%
\pgfpathlineto{\pgfqpoint{2.516479in}{0.754205in}}%
\pgfpathlineto{\pgfqpoint{2.514802in}{0.774833in}}%
\pgfpathlineto{\pgfqpoint{2.513097in}{0.795460in}}%
\pgfpathlineto{\pgfqpoint{2.511363in}{0.816088in}}%
\pgfpathlineto{\pgfqpoint{2.509598in}{0.836716in}}%
\pgfpathlineto{\pgfqpoint{2.507800in}{0.857344in}}%
\pgfpathlineto{\pgfqpoint{2.505966in}{0.877972in}}%
\pgfpathlineto{\pgfqpoint{2.504095in}{0.898600in}}%
\pgfpathlineto{\pgfqpoint{2.502183in}{0.919227in}}%
\pgfpathlineto{\pgfqpoint{2.500227in}{0.939855in}}%
\pgfpathlineto{\pgfqpoint{2.498225in}{0.960483in}}%
\pgfpathlineto{\pgfqpoint{2.496172in}{0.981111in}}%
\pgfpathlineto{\pgfqpoint{2.494064in}{1.001739in}}%
\pgfpathlineto{\pgfqpoint{2.493869in}{1.003642in}}%
\pgfpathlineto{\pgfqpoint{2.489447in}{1.001739in}}%
\pgfpathlineto{\pgfqpoint{2.452824in}{0.986332in}}%
\pgfpathlineto{\pgfqpoint{2.439975in}{0.981111in}}%
\pgfpathlineto{\pgfqpoint{2.411779in}{0.970028in}}%
\pgfpathlineto{\pgfqpoint{2.381024in}{0.960483in}}%
\pgfpathlineto{\pgfqpoint{2.370733in}{0.957416in}}%
\pgfpathlineto{\pgfqpoint{2.329688in}{0.950463in}}%
\pgfpathlineto{\pgfqpoint{2.288643in}{0.949351in}}%
\pgfpathlineto{\pgfqpoint{2.247597in}{0.952641in}}%
\pgfpathlineto{\pgfqpoint{2.206552in}{0.957461in}}%
\pgfpathlineto{\pgfqpoint{2.165507in}{0.960284in}}%
\pgfpathlineto{\pgfqpoint{2.124461in}{0.957778in}}%
\pgfpathlineto{\pgfqpoint{2.083416in}{0.947507in}}%
\pgfpathlineto{\pgfqpoint{2.066762in}{0.939855in}}%
\pgfpathlineto{\pgfqpoint{2.042371in}{0.928633in}}%
\pgfpathlineto{\pgfqpoint{2.027720in}{0.919227in}}%
\pgfpathlineto{\pgfqpoint{2.001325in}{0.902086in}}%
\pgfpathlineto{\pgfqpoint{1.996764in}{0.898600in}}%
\pgfpathlineto{\pgfqpoint{1.970051in}{0.877972in}}%
\pgfpathlineto{\pgfqpoint{1.960280in}{0.870311in}}%
\pgfpathlineto{\pgfqpoint{1.944366in}{0.857344in}}%
\pgfpathlineto{\pgfqpoint{1.919695in}{0.836716in}}%
\pgfpathlineto{\pgfqpoint{1.919235in}{0.836329in}}%
\pgfpathlineto{\pgfqpoint{1.893732in}{0.816088in}}%
\pgfpathlineto{\pgfqpoint{1.878189in}{0.803384in}}%
\pgfpathlineto{\pgfqpoint{1.867055in}{0.795460in}}%
\pgfpathlineto{\pgfqpoint{1.838862in}{0.774833in}}%
\pgfpathlineto{\pgfqpoint{1.837144in}{0.773556in}}%
\pgfpathlineto{\pgfqpoint{1.805826in}{0.754205in}}%
\pgfpathlineto{\pgfqpoint{1.796099in}{0.748034in}}%
\pgfpathlineto{\pgfqpoint{1.768408in}{0.733577in}}%
\pgfpathlineto{\pgfqpoint{1.755053in}{0.726465in}}%
\pgfpathlineto{\pgfqpoint{1.725171in}{0.712949in}}%
\pgfpathlineto{\pgfqpoint{1.714008in}{0.707847in}}%
\pgfpathlineto{\pgfqpoint{1.675724in}{0.692321in}}%
\pgfpathlineto{\pgfqpoint{1.672963in}{0.691202in}}%
\pgfpathlineto{\pgfqpoint{1.631917in}{0.676501in}}%
\pgfpathlineto{\pgfqpoint{1.615495in}{0.671694in}}%
\pgfpathlineto{\pgfqpoint{1.590872in}{0.664661in}}%
\pgfpathlineto{\pgfqpoint{1.549827in}{0.657389in}}%
\pgfpathlineto{\pgfqpoint{1.508781in}{0.656212in}}%
\pgfpathlineto{\pgfqpoint{1.467736in}{0.661622in}}%
\pgfpathlineto{\pgfqpoint{1.429560in}{0.671694in}}%
\pgfpathlineto{\pgfqpoint{1.426691in}{0.672451in}}%
\pgfpathlineto{\pgfqpoint{1.385645in}{0.686201in}}%
\pgfpathlineto{\pgfqpoint{1.365825in}{0.692321in}}%
\pgfpathlineto{\pgfqpoint{1.344600in}{0.698930in}}%
\pgfpathlineto{\pgfqpoint{1.303555in}{0.706443in}}%
\pgfpathlineto{\pgfqpoint{1.262509in}{0.705032in}}%
\pgfpathlineto{\pgfqpoint{1.221509in}{0.692321in}}%
\pgfpathlineto{\pgfqpoint{1.221464in}{0.692307in}}%
\pgfpathlineto{\pgfqpoint{1.186549in}{0.671694in}}%
\pgfpathlineto{\pgfqpoint{1.180419in}{0.668059in}}%
\pgfpathlineto{\pgfqpoint{1.159901in}{0.651066in}}%
\pgfpathlineto{\pgfqpoint{1.139373in}{0.633820in}}%
\pgfpathlineto{\pgfqpoint{1.135948in}{0.630438in}}%
\pgfpathlineto{\pgfqpoint{1.115143in}{0.609810in}}%
\pgfpathlineto{\pgfqpoint{1.098328in}{0.592826in}}%
\pgfpathlineto{\pgfqpoint{1.094866in}{0.589182in}}%
\pgfpathlineto{\pgfqpoint{1.075376in}{0.568554in}}%
\pgfpathlineto{\pgfqpoint{1.057283in}{0.548949in}}%
\pgfpathlineto{\pgfqpoint{1.056289in}{0.547927in}}%
\pgfpathlineto{\pgfqpoint{1.036326in}{0.527299in}}%
\pgfpathlineto{\pgfqpoint{1.016912in}{0.506671in}}%
\pgfpathlineto{\pgfqpoint{1.016237in}{0.505952in}}%
\pgfpathlineto{\pgfqpoint{0.995147in}{0.486043in}}%
\pgfpathlineto{\pgfqpoint{0.975192in}{0.466653in}}%
\pgfpathlineto{\pgfqpoint{0.973673in}{0.465415in}}%
\pgfpathlineto{\pgfqpoint{0.948551in}{0.444787in}}%
\pgfpathlineto{\pgfqpoint{0.934147in}{0.432688in}}%
\pgfpathlineto{\pgfqpoint{0.921622in}{0.424160in}}%
\pgfpathlineto{\pgfqpoint{0.893101in}{0.404401in}}%
\pgfpathlineto{\pgfqpoint{0.891516in}{0.403532in}}%
\pgfpathlineto{\pgfqpoint{0.854125in}{0.382904in}}%
\pgfpathlineto{\pgfqpoint{0.852056in}{0.382904in}}%
\pgfpathlineto{\pgfqpoint{0.811011in}{0.382904in}}%
\pgfpathlineto{\pgfqpoint{0.769965in}{0.382904in}}%
\pgfpathlineto{\pgfqpoint{0.728920in}{0.382904in}}%
\pgfpathlineto{\pgfqpoint{0.687875in}{0.382904in}}%
\pgfpathlineto{\pgfqpoint{0.646829in}{0.382904in}}%
\pgfpathlineto{\pgfqpoint{0.605784in}{0.382904in}}%
\pgfpathclose%
\pgfusepath{stroke,fill}%
\end{pgfscope}%
\begin{pgfscope}%
\pgfpathrectangle{\pgfqpoint{0.605784in}{0.382904in}}{\pgfqpoint{4.063488in}{2.042155in}}%
\pgfusepath{clip}%
\pgfsetbuttcap%
\pgfsetroundjoin%
\definecolor{currentfill}{rgb}{0.120092,0.600104,0.542530}%
\pgfsetfillcolor{currentfill}%
\pgfsetlinewidth{1.003750pt}%
\definecolor{currentstroke}{rgb}{0.120092,0.600104,0.542530}%
\pgfsetstrokecolor{currentstroke}%
\pgfsetdash{}{0pt}%
\pgfpathmoveto{\pgfqpoint{0.611639in}{2.136269in}}%
\pgfpathlineto{\pgfqpoint{0.605784in}{2.139500in}}%
\pgfpathlineto{\pgfqpoint{0.605784in}{2.156897in}}%
\pgfpathlineto{\pgfqpoint{0.605784in}{2.177525in}}%
\pgfpathlineto{\pgfqpoint{0.605784in}{2.198153in}}%
\pgfpathlineto{\pgfqpoint{0.605784in}{2.202159in}}%
\pgfpathlineto{\pgfqpoint{0.613099in}{2.198153in}}%
\pgfpathlineto{\pgfqpoint{0.646829in}{2.179670in}}%
\pgfpathlineto{\pgfqpoint{0.650597in}{2.177525in}}%
\pgfpathlineto{\pgfqpoint{0.686663in}{2.156897in}}%
\pgfpathlineto{\pgfqpoint{0.687875in}{2.156200in}}%
\pgfpathlineto{\pgfqpoint{0.728920in}{2.136541in}}%
\pgfpathlineto{\pgfqpoint{0.729959in}{2.136269in}}%
\pgfpathlineto{\pgfqpoint{0.769965in}{2.125580in}}%
\pgfpathlineto{\pgfqpoint{0.811011in}{2.127464in}}%
\pgfpathlineto{\pgfqpoint{0.832460in}{2.136269in}}%
\pgfpathlineto{\pgfqpoint{0.852056in}{2.144281in}}%
\pgfpathlineto{\pgfqpoint{0.868568in}{2.156897in}}%
\pgfpathlineto{\pgfqpoint{0.893101in}{2.175779in}}%
\pgfpathlineto{\pgfqpoint{0.894757in}{2.177525in}}%
\pgfpathlineto{\pgfqpoint{0.914361in}{2.198153in}}%
\pgfpathlineto{\pgfqpoint{0.933672in}{2.218780in}}%
\pgfpathlineto{\pgfqpoint{0.934147in}{2.219285in}}%
\pgfpathlineto{\pgfqpoint{0.950424in}{2.239408in}}%
\pgfpathlineto{\pgfqpoint{0.966863in}{2.260036in}}%
\pgfpathlineto{\pgfqpoint{0.975192in}{2.270547in}}%
\pgfpathlineto{\pgfqpoint{0.982925in}{2.280664in}}%
\pgfpathlineto{\pgfqpoint{0.998587in}{2.301292in}}%
\pgfpathlineto{\pgfqpoint{1.013994in}{2.321920in}}%
\pgfpathlineto{\pgfqpoint{1.016237in}{2.324921in}}%
\pgfpathlineto{\pgfqpoint{1.030095in}{2.342547in}}%
\pgfpathlineto{\pgfqpoint{1.046060in}{2.363175in}}%
\pgfpathlineto{\pgfqpoint{1.057283in}{2.377853in}}%
\pgfpathlineto{\pgfqpoint{1.062422in}{2.383803in}}%
\pgfpathlineto{\pgfqpoint{1.080140in}{2.404431in}}%
\pgfpathlineto{\pgfqpoint{1.097501in}{2.425059in}}%
\pgfpathlineto{\pgfqpoint{1.098328in}{2.425059in}}%
\pgfpathlineto{\pgfqpoint{1.139373in}{2.425059in}}%
\pgfpathlineto{\pgfqpoint{1.164497in}{2.425059in}}%
\pgfpathlineto{\pgfqpoint{1.139373in}{2.404945in}}%
\pgfpathlineto{\pgfqpoint{1.138867in}{2.404431in}}%
\pgfpathlineto{\pgfqpoint{1.118550in}{2.383803in}}%
\pgfpathlineto{\pgfqpoint{1.098328in}{2.363747in}}%
\pgfpathlineto{\pgfqpoint{1.097849in}{2.363175in}}%
\pgfpathlineto{\pgfqpoint{1.080568in}{2.342547in}}%
\pgfpathlineto{\pgfqpoint{1.062908in}{2.321920in}}%
\pgfpathlineto{\pgfqpoint{1.057283in}{2.315391in}}%
\pgfpathlineto{\pgfqpoint{1.046585in}{2.301292in}}%
\pgfpathlineto{\pgfqpoint{1.030743in}{2.280664in}}%
\pgfpathlineto{\pgfqpoint{1.016237in}{2.262105in}}%
\pgfpathlineto{\pgfqpoint{1.014708in}{2.260036in}}%
\pgfpathlineto{\pgfqpoint{0.999482in}{2.239408in}}%
\pgfpathlineto{\pgfqpoint{0.983993in}{2.218780in}}%
\pgfpathlineto{\pgfqpoint{0.975192in}{2.207158in}}%
\pgfpathlineto{\pgfqpoint{0.968148in}{2.198153in}}%
\pgfpathlineto{\pgfqpoint{0.951943in}{2.177525in}}%
\pgfpathlineto{\pgfqpoint{0.935478in}{2.156897in}}%
\pgfpathlineto{\pgfqpoint{0.934147in}{2.155225in}}%
\pgfpathlineto{\pgfqpoint{0.916649in}{2.136269in}}%
\pgfpathlineto{\pgfqpoint{0.897341in}{2.115641in}}%
\pgfpathlineto{\pgfqpoint{0.893101in}{2.111111in}}%
\pgfpathlineto{\pgfqpoint{0.872411in}{2.095013in}}%
\pgfpathlineto{\pgfqpoint{0.852056in}{2.079300in}}%
\pgfpathlineto{\pgfqpoint{0.840021in}{2.074386in}}%
\pgfpathlineto{\pgfqpoint{0.811011in}{2.062486in}}%
\pgfpathlineto{\pgfqpoint{0.769965in}{2.060989in}}%
\pgfpathlineto{\pgfqpoint{0.728920in}{2.072549in}}%
\pgfpathlineto{\pgfqpoint{0.725247in}{2.074386in}}%
\pgfpathlineto{\pgfqpoint{0.687875in}{2.092860in}}%
\pgfpathlineto{\pgfqpoint{0.684208in}{2.095013in}}%
\pgfpathlineto{\pgfqpoint{0.648890in}{2.115641in}}%
\pgfpathlineto{\pgfqpoint{0.646829in}{2.116838in}}%
\pgfpathclose%
\pgfusepath{stroke,fill}%
\end{pgfscope}%
\begin{pgfscope}%
\pgfpathrectangle{\pgfqpoint{0.605784in}{0.382904in}}{\pgfqpoint{4.063488in}{2.042155in}}%
\pgfusepath{clip}%
\pgfsetbuttcap%
\pgfsetroundjoin%
\definecolor{currentfill}{rgb}{0.120092,0.600104,0.542530}%
\pgfsetfillcolor{currentfill}%
\pgfsetlinewidth{1.003750pt}%
\definecolor{currentstroke}{rgb}{0.120092,0.600104,0.542530}%
\pgfsetstrokecolor{currentstroke}%
\pgfsetdash{}{0pt}%
\pgfpathmoveto{\pgfqpoint{1.455483in}{2.425059in}}%
\pgfpathlineto{\pgfqpoint{1.467736in}{2.425059in}}%
\pgfpathlineto{\pgfqpoint{1.508781in}{2.425059in}}%
\pgfpathlineto{\pgfqpoint{1.533995in}{2.425059in}}%
\pgfpathlineto{\pgfqpoint{1.549827in}{2.412425in}}%
\pgfpathlineto{\pgfqpoint{1.562781in}{2.404431in}}%
\pgfpathlineto{\pgfqpoint{1.590872in}{2.386868in}}%
\pgfpathlineto{\pgfqpoint{1.600786in}{2.383803in}}%
\pgfpathlineto{\pgfqpoint{1.631917in}{2.373951in}}%
\pgfpathlineto{\pgfqpoint{1.672963in}{2.378078in}}%
\pgfpathlineto{\pgfqpoint{1.683205in}{2.383803in}}%
\pgfpathlineto{\pgfqpoint{1.714008in}{2.401002in}}%
\pgfpathlineto{\pgfqpoint{1.717484in}{2.404431in}}%
\pgfpathlineto{\pgfqpoint{1.738470in}{2.425059in}}%
\pgfpathlineto{\pgfqpoint{1.755053in}{2.425059in}}%
\pgfpathlineto{\pgfqpoint{1.792377in}{2.425059in}}%
\pgfpathlineto{\pgfqpoint{1.777228in}{2.404431in}}%
\pgfpathlineto{\pgfqpoint{1.761936in}{2.383803in}}%
\pgfpathlineto{\pgfqpoint{1.755053in}{2.374517in}}%
\pgfpathlineto{\pgfqpoint{1.743841in}{2.363175in}}%
\pgfpathlineto{\pgfqpoint{1.723393in}{2.342547in}}%
\pgfpathlineto{\pgfqpoint{1.714008in}{2.333090in}}%
\pgfpathlineto{\pgfqpoint{1.694219in}{2.321920in}}%
\pgfpathlineto{\pgfqpoint{1.672963in}{2.309907in}}%
\pgfpathlineto{\pgfqpoint{1.631917in}{2.306273in}}%
\pgfpathlineto{\pgfqpoint{1.590872in}{2.320242in}}%
\pgfpathlineto{\pgfqpoint{1.588358in}{2.321920in}}%
\pgfpathlineto{\pgfqpoint{1.556940in}{2.342547in}}%
\pgfpathlineto{\pgfqpoint{1.549827in}{2.347149in}}%
\pgfpathlineto{\pgfqpoint{1.530504in}{2.363175in}}%
\pgfpathlineto{\pgfqpoint{1.508781in}{2.381085in}}%
\pgfpathlineto{\pgfqpoint{1.505605in}{2.383803in}}%
\pgfpathlineto{\pgfqpoint{1.481285in}{2.404431in}}%
\pgfpathlineto{\pgfqpoint{1.467736in}{2.415888in}}%
\pgfpathclose%
\pgfusepath{stroke,fill}%
\end{pgfscope}%
\begin{pgfscope}%
\pgfpathrectangle{\pgfqpoint{0.605784in}{0.382904in}}{\pgfqpoint{4.063488in}{2.042155in}}%
\pgfusepath{clip}%
\pgfsetbuttcap%
\pgfsetroundjoin%
\definecolor{currentfill}{rgb}{0.120092,0.600104,0.542530}%
\pgfsetfillcolor{currentfill}%
\pgfsetlinewidth{1.003750pt}%
\definecolor{currentstroke}{rgb}{0.120092,0.600104,0.542530}%
\pgfsetstrokecolor{currentstroke}%
\pgfsetdash{}{0pt}%
\pgfpathmoveto{\pgfqpoint{2.532710in}{2.239408in}}%
\pgfpathlineto{\pgfqpoint{2.529183in}{2.260036in}}%
\pgfpathlineto{\pgfqpoint{2.525861in}{2.280664in}}%
\pgfpathlineto{\pgfqpoint{2.522720in}{2.301292in}}%
\pgfpathlineto{\pgfqpoint{2.519739in}{2.321920in}}%
\pgfpathlineto{\pgfqpoint{2.516902in}{2.342547in}}%
\pgfpathlineto{\pgfqpoint{2.514192in}{2.363175in}}%
\pgfpathlineto{\pgfqpoint{2.511597in}{2.383803in}}%
\pgfpathlineto{\pgfqpoint{2.509105in}{2.404431in}}%
\pgfpathlineto{\pgfqpoint{2.506707in}{2.425059in}}%
\pgfpathlineto{\pgfqpoint{2.515190in}{2.425059in}}%
\pgfpathlineto{\pgfqpoint{2.517868in}{2.404431in}}%
\pgfpathlineto{\pgfqpoint{2.520658in}{2.383803in}}%
\pgfpathlineto{\pgfqpoint{2.523573in}{2.363175in}}%
\pgfpathlineto{\pgfqpoint{2.526626in}{2.342547in}}%
\pgfpathlineto{\pgfqpoint{2.529832in}{2.321920in}}%
\pgfpathlineto{\pgfqpoint{2.533210in}{2.301292in}}%
\pgfpathlineto{\pgfqpoint{2.534915in}{2.291242in}}%
\pgfpathlineto{\pgfqpoint{2.575960in}{2.291242in}}%
\pgfpathlineto{\pgfqpoint{2.617005in}{2.291242in}}%
\pgfpathlineto{\pgfqpoint{2.658051in}{2.291242in}}%
\pgfpathlineto{\pgfqpoint{2.699096in}{2.291242in}}%
\pgfpathlineto{\pgfqpoint{2.740141in}{2.291242in}}%
\pgfpathlineto{\pgfqpoint{2.781187in}{2.291242in}}%
\pgfpathlineto{\pgfqpoint{2.822232in}{2.291242in}}%
\pgfpathlineto{\pgfqpoint{2.863277in}{2.291242in}}%
\pgfpathlineto{\pgfqpoint{2.904323in}{2.291242in}}%
\pgfpathlineto{\pgfqpoint{2.945368in}{2.291242in}}%
\pgfpathlineto{\pgfqpoint{2.986413in}{2.291242in}}%
\pgfpathlineto{\pgfqpoint{3.027459in}{2.291242in}}%
\pgfpathlineto{\pgfqpoint{3.068504in}{2.291242in}}%
\pgfpathlineto{\pgfqpoint{3.109549in}{2.291242in}}%
\pgfpathlineto{\pgfqpoint{3.150595in}{2.291242in}}%
\pgfpathlineto{\pgfqpoint{3.191640in}{2.291242in}}%
\pgfpathlineto{\pgfqpoint{3.232685in}{2.291242in}}%
\pgfpathlineto{\pgfqpoint{3.273731in}{2.291242in}}%
\pgfpathlineto{\pgfqpoint{3.314776in}{2.291242in}}%
\pgfpathlineto{\pgfqpoint{3.355821in}{2.291242in}}%
\pgfpathlineto{\pgfqpoint{3.396867in}{2.291242in}}%
\pgfpathlineto{\pgfqpoint{3.437912in}{2.291242in}}%
\pgfpathlineto{\pgfqpoint{3.478957in}{2.291242in}}%
\pgfpathlineto{\pgfqpoint{3.520003in}{2.291242in}}%
\pgfpathlineto{\pgfqpoint{3.561048in}{2.291242in}}%
\pgfpathlineto{\pgfqpoint{3.602093in}{2.291242in}}%
\pgfpathlineto{\pgfqpoint{3.643139in}{2.291242in}}%
\pgfpathlineto{\pgfqpoint{3.644936in}{2.280664in}}%
\pgfpathlineto{\pgfqpoint{3.648470in}{2.260036in}}%
\pgfpathlineto{\pgfqpoint{3.652073in}{2.239408in}}%
\pgfpathlineto{\pgfqpoint{3.655750in}{2.218780in}}%
\pgfpathlineto{\pgfqpoint{3.659506in}{2.198153in}}%
\pgfpathlineto{\pgfqpoint{3.663345in}{2.177525in}}%
\pgfpathlineto{\pgfqpoint{3.667276in}{2.156897in}}%
\pgfpathlineto{\pgfqpoint{3.671302in}{2.136269in}}%
\pgfpathlineto{\pgfqpoint{3.675433in}{2.115641in}}%
\pgfpathlineto{\pgfqpoint{3.679676in}{2.095013in}}%
\pgfpathlineto{\pgfqpoint{3.684040in}{2.074386in}}%
\pgfpathlineto{\pgfqpoint{3.684184in}{2.073708in}}%
\pgfpathlineto{\pgfqpoint{3.725229in}{2.073708in}}%
\pgfpathlineto{\pgfqpoint{3.766275in}{2.073708in}}%
\pgfpathlineto{\pgfqpoint{3.807320in}{2.073708in}}%
\pgfpathlineto{\pgfqpoint{3.848365in}{2.073708in}}%
\pgfpathlineto{\pgfqpoint{3.889411in}{2.073708in}}%
\pgfpathlineto{\pgfqpoint{3.930456in}{2.073708in}}%
\pgfpathlineto{\pgfqpoint{3.971501in}{2.073708in}}%
\pgfpathlineto{\pgfqpoint{4.012547in}{2.073708in}}%
\pgfpathlineto{\pgfqpoint{4.053592in}{2.073708in}}%
\pgfpathlineto{\pgfqpoint{4.094637in}{2.073708in}}%
\pgfpathlineto{\pgfqpoint{4.135683in}{2.073708in}}%
\pgfpathlineto{\pgfqpoint{4.176728in}{2.073708in}}%
\pgfpathlineto{\pgfqpoint{4.217773in}{2.073708in}}%
\pgfpathlineto{\pgfqpoint{4.258819in}{2.073708in}}%
\pgfpathlineto{\pgfqpoint{4.299864in}{2.073708in}}%
\pgfpathlineto{\pgfqpoint{4.340909in}{2.073708in}}%
\pgfpathlineto{\pgfqpoint{4.381955in}{2.073708in}}%
\pgfpathlineto{\pgfqpoint{4.423000in}{2.073708in}}%
\pgfpathlineto{\pgfqpoint{4.464045in}{2.073708in}}%
\pgfpathlineto{\pgfqpoint{4.505091in}{2.073708in}}%
\pgfpathlineto{\pgfqpoint{4.546136in}{2.073708in}}%
\pgfpathlineto{\pgfqpoint{4.587181in}{2.073708in}}%
\pgfpathlineto{\pgfqpoint{4.628227in}{2.073708in}}%
\pgfpathlineto{\pgfqpoint{4.669272in}{2.073708in}}%
\pgfpathlineto{\pgfqpoint{4.669272in}{2.053758in}}%
\pgfpathlineto{\pgfqpoint{4.669272in}{2.033130in}}%
\pgfpathlineto{\pgfqpoint{4.669272in}{2.012502in}}%
\pgfpathlineto{\pgfqpoint{4.669272in}{2.009533in}}%
\pgfpathlineto{\pgfqpoint{4.628227in}{2.009533in}}%
\pgfpathlineto{\pgfqpoint{4.587181in}{2.009533in}}%
\pgfpathlineto{\pgfqpoint{4.546136in}{2.009533in}}%
\pgfpathlineto{\pgfqpoint{4.505091in}{2.009533in}}%
\pgfpathlineto{\pgfqpoint{4.464045in}{2.009533in}}%
\pgfpathlineto{\pgfqpoint{4.423000in}{2.009533in}}%
\pgfpathlineto{\pgfqpoint{4.381955in}{2.009533in}}%
\pgfpathlineto{\pgfqpoint{4.340909in}{2.009533in}}%
\pgfpathlineto{\pgfqpoint{4.299864in}{2.009533in}}%
\pgfpathlineto{\pgfqpoint{4.258819in}{2.009533in}}%
\pgfpathlineto{\pgfqpoint{4.217773in}{2.009533in}}%
\pgfpathlineto{\pgfqpoint{4.176728in}{2.009533in}}%
\pgfpathlineto{\pgfqpoint{4.135683in}{2.009533in}}%
\pgfpathlineto{\pgfqpoint{4.094637in}{2.009533in}}%
\pgfpathlineto{\pgfqpoint{4.053592in}{2.009533in}}%
\pgfpathlineto{\pgfqpoint{4.012547in}{2.009533in}}%
\pgfpathlineto{\pgfqpoint{3.971501in}{2.009533in}}%
\pgfpathlineto{\pgfqpoint{3.930456in}{2.009533in}}%
\pgfpathlineto{\pgfqpoint{3.889411in}{2.009533in}}%
\pgfpathlineto{\pgfqpoint{3.848365in}{2.009533in}}%
\pgfpathlineto{\pgfqpoint{3.807320in}{2.009533in}}%
\pgfpathlineto{\pgfqpoint{3.766275in}{2.009533in}}%
\pgfpathlineto{\pgfqpoint{3.725229in}{2.009533in}}%
\pgfpathlineto{\pgfqpoint{3.684184in}{2.009533in}}%
\pgfpathlineto{\pgfqpoint{3.683547in}{2.012502in}}%
\pgfpathlineto{\pgfqpoint{3.679158in}{2.033130in}}%
\pgfpathlineto{\pgfqpoint{3.674900in}{2.053758in}}%
\pgfpathlineto{\pgfqpoint{3.670765in}{2.074386in}}%
\pgfpathlineto{\pgfqpoint{3.666741in}{2.095013in}}%
\pgfpathlineto{\pgfqpoint{3.662822in}{2.115641in}}%
\pgfpathlineto{\pgfqpoint{3.658998in}{2.136269in}}%
\pgfpathlineto{\pgfqpoint{3.655264in}{2.156897in}}%
\pgfpathlineto{\pgfqpoint{3.651614in}{2.177525in}}%
\pgfpathlineto{\pgfqpoint{3.648040in}{2.198153in}}%
\pgfpathlineto{\pgfqpoint{3.644540in}{2.218780in}}%
\pgfpathlineto{\pgfqpoint{3.643139in}{2.227089in}}%
\pgfpathlineto{\pgfqpoint{3.602093in}{2.227089in}}%
\pgfpathlineto{\pgfqpoint{3.561048in}{2.227089in}}%
\pgfpathlineto{\pgfqpoint{3.520003in}{2.227089in}}%
\pgfpathlineto{\pgfqpoint{3.478957in}{2.227089in}}%
\pgfpathlineto{\pgfqpoint{3.437912in}{2.227089in}}%
\pgfpathlineto{\pgfqpoint{3.396867in}{2.227089in}}%
\pgfpathlineto{\pgfqpoint{3.355821in}{2.227089in}}%
\pgfpathlineto{\pgfqpoint{3.314776in}{2.227089in}}%
\pgfpathlineto{\pgfqpoint{3.273731in}{2.227089in}}%
\pgfpathlineto{\pgfqpoint{3.232685in}{2.227089in}}%
\pgfpathlineto{\pgfqpoint{3.191640in}{2.227089in}}%
\pgfpathlineto{\pgfqpoint{3.150595in}{2.227089in}}%
\pgfpathlineto{\pgfqpoint{3.109549in}{2.227089in}}%
\pgfpathlineto{\pgfqpoint{3.068504in}{2.227089in}}%
\pgfpathlineto{\pgfqpoint{3.027459in}{2.227089in}}%
\pgfpathlineto{\pgfqpoint{2.986413in}{2.227089in}}%
\pgfpathlineto{\pgfqpoint{2.945368in}{2.227089in}}%
\pgfpathlineto{\pgfqpoint{2.904323in}{2.227089in}}%
\pgfpathlineto{\pgfqpoint{2.863277in}{2.227089in}}%
\pgfpathlineto{\pgfqpoint{2.822232in}{2.227089in}}%
\pgfpathlineto{\pgfqpoint{2.781187in}{2.227089in}}%
\pgfpathlineto{\pgfqpoint{2.740141in}{2.227089in}}%
\pgfpathlineto{\pgfqpoint{2.699096in}{2.227089in}}%
\pgfpathlineto{\pgfqpoint{2.658051in}{2.227089in}}%
\pgfpathlineto{\pgfqpoint{2.617005in}{2.227089in}}%
\pgfpathlineto{\pgfqpoint{2.575960in}{2.227089in}}%
\pgfpathlineto{\pgfqpoint{2.534915in}{2.227089in}}%
\pgfpathclose%
\pgfusepath{stroke,fill}%
\end{pgfscope}%
\begin{pgfscope}%
\pgfpathrectangle{\pgfqpoint{0.605784in}{0.382904in}}{\pgfqpoint{4.063488in}{2.042155in}}%
\pgfusepath{clip}%
\pgfsetbuttcap%
\pgfsetroundjoin%
\definecolor{currentfill}{rgb}{0.140210,0.665859,0.513427}%
\pgfsetfillcolor{currentfill}%
\pgfsetlinewidth{1.003750pt}%
\definecolor{currentstroke}{rgb}{0.140210,0.665859,0.513427}%
\pgfsetstrokecolor{currentstroke}%
\pgfsetdash{}{0pt}%
\pgfpathmoveto{\pgfqpoint{0.613099in}{2.198153in}}%
\pgfpathlineto{\pgfqpoint{0.605784in}{2.202159in}}%
\pgfpathlineto{\pgfqpoint{0.605784in}{2.218780in}}%
\pgfpathlineto{\pgfqpoint{0.605784in}{2.239408in}}%
\pgfpathlineto{\pgfqpoint{0.605784in}{2.260036in}}%
\pgfpathlineto{\pgfqpoint{0.605784in}{2.260812in}}%
\pgfpathlineto{\pgfqpoint{0.607209in}{2.260036in}}%
\pgfpathlineto{\pgfqpoint{0.645075in}{2.239408in}}%
\pgfpathlineto{\pgfqpoint{0.646829in}{2.238451in}}%
\pgfpathlineto{\pgfqpoint{0.681786in}{2.218780in}}%
\pgfpathlineto{\pgfqpoint{0.687875in}{2.215341in}}%
\pgfpathlineto{\pgfqpoint{0.724807in}{2.198153in}}%
\pgfpathlineto{\pgfqpoint{0.728920in}{2.196219in}}%
\pgfpathlineto{\pgfqpoint{0.769965in}{2.185780in}}%
\pgfpathlineto{\pgfqpoint{0.811011in}{2.187969in}}%
\pgfpathlineto{\pgfqpoint{0.835801in}{2.198153in}}%
\pgfpathlineto{\pgfqpoint{0.852056in}{2.204805in}}%
\pgfpathlineto{\pgfqpoint{0.870513in}{2.218780in}}%
\pgfpathlineto{\pgfqpoint{0.893101in}{2.236001in}}%
\pgfpathlineto{\pgfqpoint{0.896369in}{2.239408in}}%
\pgfpathlineto{\pgfqpoint{0.916161in}{2.260036in}}%
\pgfpathlineto{\pgfqpoint{0.934147in}{2.279027in}}%
\pgfpathlineto{\pgfqpoint{0.935480in}{2.280664in}}%
\pgfpathlineto{\pgfqpoint{0.952314in}{2.301292in}}%
\pgfpathlineto{\pgfqpoint{0.968901in}{2.321920in}}%
\pgfpathlineto{\pgfqpoint{0.975192in}{2.329769in}}%
\pgfpathlineto{\pgfqpoint{0.985041in}{2.342547in}}%
\pgfpathlineto{\pgfqpoint{1.000797in}{2.363175in}}%
\pgfpathlineto{\pgfqpoint{1.016237in}{2.383704in}}%
\pgfpathlineto{\pgfqpoint{1.016315in}{2.383803in}}%
\pgfpathlineto{\pgfqpoint{1.032615in}{2.404431in}}%
\pgfpathlineto{\pgfqpoint{1.048628in}{2.425059in}}%
\pgfpathlineto{\pgfqpoint{1.057283in}{2.425059in}}%
\pgfpathlineto{\pgfqpoint{1.097501in}{2.425059in}}%
\pgfpathlineto{\pgfqpoint{1.080140in}{2.404431in}}%
\pgfpathlineto{\pgfqpoint{1.062422in}{2.383803in}}%
\pgfpathlineto{\pgfqpoint{1.057283in}{2.377853in}}%
\pgfpathlineto{\pgfqpoint{1.046060in}{2.363175in}}%
\pgfpathlineto{\pgfqpoint{1.030095in}{2.342547in}}%
\pgfpathlineto{\pgfqpoint{1.016237in}{2.324921in}}%
\pgfpathlineto{\pgfqpoint{1.013994in}{2.321920in}}%
\pgfpathlineto{\pgfqpoint{0.998587in}{2.301292in}}%
\pgfpathlineto{\pgfqpoint{0.982925in}{2.280664in}}%
\pgfpathlineto{\pgfqpoint{0.975192in}{2.270547in}}%
\pgfpathlineto{\pgfqpoint{0.966863in}{2.260036in}}%
\pgfpathlineto{\pgfqpoint{0.950424in}{2.239408in}}%
\pgfpathlineto{\pgfqpoint{0.934147in}{2.219285in}}%
\pgfpathlineto{\pgfqpoint{0.933672in}{2.218780in}}%
\pgfpathlineto{\pgfqpoint{0.914361in}{2.198153in}}%
\pgfpathlineto{\pgfqpoint{0.894757in}{2.177525in}}%
\pgfpathlineto{\pgfqpoint{0.893101in}{2.175779in}}%
\pgfpathlineto{\pgfqpoint{0.868568in}{2.156897in}}%
\pgfpathlineto{\pgfqpoint{0.852056in}{2.144281in}}%
\pgfpathlineto{\pgfqpoint{0.832460in}{2.136269in}}%
\pgfpathlineto{\pgfqpoint{0.811011in}{2.127464in}}%
\pgfpathlineto{\pgfqpoint{0.769965in}{2.125580in}}%
\pgfpathlineto{\pgfqpoint{0.729959in}{2.136269in}}%
\pgfpathlineto{\pgfqpoint{0.728920in}{2.136541in}}%
\pgfpathlineto{\pgfqpoint{0.687875in}{2.156200in}}%
\pgfpathlineto{\pgfqpoint{0.686663in}{2.156897in}}%
\pgfpathlineto{\pgfqpoint{0.650597in}{2.177525in}}%
\pgfpathlineto{\pgfqpoint{0.646829in}{2.179670in}}%
\pgfpathclose%
\pgfusepath{stroke,fill}%
\end{pgfscope}%
\begin{pgfscope}%
\pgfpathrectangle{\pgfqpoint{0.605784in}{0.382904in}}{\pgfqpoint{4.063488in}{2.042155in}}%
\pgfusepath{clip}%
\pgfsetbuttcap%
\pgfsetroundjoin%
\definecolor{currentfill}{rgb}{0.140210,0.665859,0.513427}%
\pgfsetfillcolor{currentfill}%
\pgfsetlinewidth{1.003750pt}%
\definecolor{currentstroke}{rgb}{0.140210,0.665859,0.513427}%
\pgfsetstrokecolor{currentstroke}%
\pgfsetdash{}{0pt}%
\pgfpathmoveto{\pgfqpoint{0.891516in}{0.403532in}}%
\pgfpathlineto{\pgfqpoint{0.893101in}{0.404401in}}%
\pgfpathlineto{\pgfqpoint{0.921622in}{0.424160in}}%
\pgfpathlineto{\pgfqpoint{0.934147in}{0.432688in}}%
\pgfpathlineto{\pgfqpoint{0.948551in}{0.444787in}}%
\pgfpathlineto{\pgfqpoint{0.973673in}{0.465415in}}%
\pgfpathlineto{\pgfqpoint{0.975192in}{0.466653in}}%
\pgfpathlineto{\pgfqpoint{0.995147in}{0.486043in}}%
\pgfpathlineto{\pgfqpoint{1.016237in}{0.505952in}}%
\pgfpathlineto{\pgfqpoint{1.016912in}{0.506671in}}%
\pgfpathlineto{\pgfqpoint{1.036326in}{0.527299in}}%
\pgfpathlineto{\pgfqpoint{1.056289in}{0.547927in}}%
\pgfpathlineto{\pgfqpoint{1.057283in}{0.548949in}}%
\pgfpathlineto{\pgfqpoint{1.075376in}{0.568554in}}%
\pgfpathlineto{\pgfqpoint{1.094866in}{0.589182in}}%
\pgfpathlineto{\pgfqpoint{1.098328in}{0.592826in}}%
\pgfpathlineto{\pgfqpoint{1.115143in}{0.609810in}}%
\pgfpathlineto{\pgfqpoint{1.135948in}{0.630438in}}%
\pgfpathlineto{\pgfqpoint{1.139373in}{0.633820in}}%
\pgfpathlineto{\pgfqpoint{1.159901in}{0.651066in}}%
\pgfpathlineto{\pgfqpoint{1.180419in}{0.668059in}}%
\pgfpathlineto{\pgfqpoint{1.186549in}{0.671694in}}%
\pgfpathlineto{\pgfqpoint{1.221464in}{0.692307in}}%
\pgfpathlineto{\pgfqpoint{1.221509in}{0.692321in}}%
\pgfpathlineto{\pgfqpoint{1.262509in}{0.705032in}}%
\pgfpathlineto{\pgfqpoint{1.303555in}{0.706443in}}%
\pgfpathlineto{\pgfqpoint{1.344600in}{0.698930in}}%
\pgfpathlineto{\pgfqpoint{1.365825in}{0.692321in}}%
\pgfpathlineto{\pgfqpoint{1.385645in}{0.686201in}}%
\pgfpathlineto{\pgfqpoint{1.426691in}{0.672451in}}%
\pgfpathlineto{\pgfqpoint{1.429560in}{0.671694in}}%
\pgfpathlineto{\pgfqpoint{1.467736in}{0.661622in}}%
\pgfpathlineto{\pgfqpoint{1.508781in}{0.656212in}}%
\pgfpathlineto{\pgfqpoint{1.549827in}{0.657389in}}%
\pgfpathlineto{\pgfqpoint{1.590872in}{0.664661in}}%
\pgfpathlineto{\pgfqpoint{1.615495in}{0.671694in}}%
\pgfpathlineto{\pgfqpoint{1.631917in}{0.676501in}}%
\pgfpathlineto{\pgfqpoint{1.672963in}{0.691202in}}%
\pgfpathlineto{\pgfqpoint{1.675724in}{0.692321in}}%
\pgfpathlineto{\pgfqpoint{1.714008in}{0.707847in}}%
\pgfpathlineto{\pgfqpoint{1.725171in}{0.712949in}}%
\pgfpathlineto{\pgfqpoint{1.755053in}{0.726465in}}%
\pgfpathlineto{\pgfqpoint{1.768408in}{0.733577in}}%
\pgfpathlineto{\pgfqpoint{1.796099in}{0.748034in}}%
\pgfpathlineto{\pgfqpoint{1.805826in}{0.754205in}}%
\pgfpathlineto{\pgfqpoint{1.837144in}{0.773556in}}%
\pgfpathlineto{\pgfqpoint{1.838862in}{0.774833in}}%
\pgfpathlineto{\pgfqpoint{1.867055in}{0.795460in}}%
\pgfpathlineto{\pgfqpoint{1.878189in}{0.803384in}}%
\pgfpathlineto{\pgfqpoint{1.893732in}{0.816088in}}%
\pgfpathlineto{\pgfqpoint{1.919235in}{0.836329in}}%
\pgfpathlineto{\pgfqpoint{1.919695in}{0.836716in}}%
\pgfpathlineto{\pgfqpoint{1.944366in}{0.857344in}}%
\pgfpathlineto{\pgfqpoint{1.960280in}{0.870311in}}%
\pgfpathlineto{\pgfqpoint{1.970051in}{0.877972in}}%
\pgfpathlineto{\pgfqpoint{1.996764in}{0.898600in}}%
\pgfpathlineto{\pgfqpoint{2.001325in}{0.902086in}}%
\pgfpathlineto{\pgfqpoint{2.027720in}{0.919227in}}%
\pgfpathlineto{\pgfqpoint{2.042371in}{0.928633in}}%
\pgfpathlineto{\pgfqpoint{2.066762in}{0.939855in}}%
\pgfpathlineto{\pgfqpoint{2.083416in}{0.947507in}}%
\pgfpathlineto{\pgfqpoint{2.124461in}{0.957778in}}%
\pgfpathlineto{\pgfqpoint{2.165507in}{0.960284in}}%
\pgfpathlineto{\pgfqpoint{2.206552in}{0.957461in}}%
\pgfpathlineto{\pgfqpoint{2.247597in}{0.952641in}}%
\pgfpathlineto{\pgfqpoint{2.288643in}{0.949351in}}%
\pgfpathlineto{\pgfqpoint{2.329688in}{0.950463in}}%
\pgfpathlineto{\pgfqpoint{2.370733in}{0.957416in}}%
\pgfpathlineto{\pgfqpoint{2.381024in}{0.960483in}}%
\pgfpathlineto{\pgfqpoint{2.411779in}{0.970028in}}%
\pgfpathlineto{\pgfqpoint{2.439975in}{0.981111in}}%
\pgfpathlineto{\pgfqpoint{2.452824in}{0.986332in}}%
\pgfpathlineto{\pgfqpoint{2.489447in}{1.001739in}}%
\pgfpathlineto{\pgfqpoint{2.493869in}{1.003642in}}%
\pgfpathlineto{\pgfqpoint{2.494064in}{1.001739in}}%
\pgfpathlineto{\pgfqpoint{2.496172in}{0.981111in}}%
\pgfpathlineto{\pgfqpoint{2.498225in}{0.960483in}}%
\pgfpathlineto{\pgfqpoint{2.500227in}{0.939855in}}%
\pgfpathlineto{\pgfqpoint{2.502183in}{0.919227in}}%
\pgfpathlineto{\pgfqpoint{2.504095in}{0.898600in}}%
\pgfpathlineto{\pgfqpoint{2.505966in}{0.877972in}}%
\pgfpathlineto{\pgfqpoint{2.507800in}{0.857344in}}%
\pgfpathlineto{\pgfqpoint{2.509598in}{0.836716in}}%
\pgfpathlineto{\pgfqpoint{2.511363in}{0.816088in}}%
\pgfpathlineto{\pgfqpoint{2.513097in}{0.795460in}}%
\pgfpathlineto{\pgfqpoint{2.514802in}{0.774833in}}%
\pgfpathlineto{\pgfqpoint{2.516479in}{0.754205in}}%
\pgfpathlineto{\pgfqpoint{2.518131in}{0.733577in}}%
\pgfpathlineto{\pgfqpoint{2.519759in}{0.712949in}}%
\pgfpathlineto{\pgfqpoint{2.521364in}{0.692321in}}%
\pgfpathlineto{\pgfqpoint{2.522947in}{0.671694in}}%
\pgfpathlineto{\pgfqpoint{2.524509in}{0.651066in}}%
\pgfpathlineto{\pgfqpoint{2.526053in}{0.630438in}}%
\pgfpathlineto{\pgfqpoint{2.527578in}{0.609810in}}%
\pgfpathlineto{\pgfqpoint{2.529086in}{0.589182in}}%
\pgfpathlineto{\pgfqpoint{2.530578in}{0.568554in}}%
\pgfpathlineto{\pgfqpoint{2.532053in}{0.547927in}}%
\pgfpathlineto{\pgfqpoint{2.533514in}{0.527299in}}%
\pgfpathlineto{\pgfqpoint{2.534915in}{0.507348in}}%
\pgfpathlineto{\pgfqpoint{2.575960in}{0.507348in}}%
\pgfpathlineto{\pgfqpoint{2.617005in}{0.507348in}}%
\pgfpathlineto{\pgfqpoint{2.658051in}{0.507348in}}%
\pgfpathlineto{\pgfqpoint{2.699096in}{0.507348in}}%
\pgfpathlineto{\pgfqpoint{2.740141in}{0.507348in}}%
\pgfpathlineto{\pgfqpoint{2.781187in}{0.507348in}}%
\pgfpathlineto{\pgfqpoint{2.822232in}{0.507348in}}%
\pgfpathlineto{\pgfqpoint{2.863277in}{0.507348in}}%
\pgfpathlineto{\pgfqpoint{2.904323in}{0.507348in}}%
\pgfpathlineto{\pgfqpoint{2.945368in}{0.507348in}}%
\pgfpathlineto{\pgfqpoint{2.986413in}{0.507348in}}%
\pgfpathlineto{\pgfqpoint{3.027459in}{0.507348in}}%
\pgfpathlineto{\pgfqpoint{3.068504in}{0.507348in}}%
\pgfpathlineto{\pgfqpoint{3.109549in}{0.507348in}}%
\pgfpathlineto{\pgfqpoint{3.150595in}{0.507348in}}%
\pgfpathlineto{\pgfqpoint{3.191640in}{0.507348in}}%
\pgfpathlineto{\pgfqpoint{3.232685in}{0.507348in}}%
\pgfpathlineto{\pgfqpoint{3.273731in}{0.507348in}}%
\pgfpathlineto{\pgfqpoint{3.314776in}{0.507348in}}%
\pgfpathlineto{\pgfqpoint{3.355821in}{0.507348in}}%
\pgfpathlineto{\pgfqpoint{3.396867in}{0.507348in}}%
\pgfpathlineto{\pgfqpoint{3.437912in}{0.507348in}}%
\pgfpathlineto{\pgfqpoint{3.478957in}{0.507348in}}%
\pgfpathlineto{\pgfqpoint{3.520003in}{0.507348in}}%
\pgfpathlineto{\pgfqpoint{3.561048in}{0.507348in}}%
\pgfpathlineto{\pgfqpoint{3.602093in}{0.507348in}}%
\pgfpathlineto{\pgfqpoint{3.643139in}{0.507348in}}%
\pgfpathlineto{\pgfqpoint{3.643282in}{0.506671in}}%
\pgfpathlineto{\pgfqpoint{3.647646in}{0.486043in}}%
\pgfpathlineto{\pgfqpoint{3.651889in}{0.465415in}}%
\pgfpathlineto{\pgfqpoint{3.656020in}{0.444787in}}%
\pgfpathlineto{\pgfqpoint{3.660047in}{0.424160in}}%
\pgfpathlineto{\pgfqpoint{3.663977in}{0.403532in}}%
\pgfpathlineto{\pgfqpoint{3.667817in}{0.382904in}}%
\pgfpathlineto{\pgfqpoint{3.656352in}{0.382904in}}%
\pgfpathlineto{\pgfqpoint{3.652245in}{0.403532in}}%
\pgfpathlineto{\pgfqpoint{3.648036in}{0.424160in}}%
\pgfpathlineto{\pgfqpoint{3.643716in}{0.444787in}}%
\pgfpathlineto{\pgfqpoint{3.643139in}{0.447529in}}%
\pgfpathlineto{\pgfqpoint{3.602093in}{0.447529in}}%
\pgfpathlineto{\pgfqpoint{3.561048in}{0.447529in}}%
\pgfpathlineto{\pgfqpoint{3.520003in}{0.447529in}}%
\pgfpathlineto{\pgfqpoint{3.478957in}{0.447529in}}%
\pgfpathlineto{\pgfqpoint{3.437912in}{0.447529in}}%
\pgfpathlineto{\pgfqpoint{3.396867in}{0.447529in}}%
\pgfpathlineto{\pgfqpoint{3.355821in}{0.447529in}}%
\pgfpathlineto{\pgfqpoint{3.314776in}{0.447529in}}%
\pgfpathlineto{\pgfqpoint{3.273731in}{0.447529in}}%
\pgfpathlineto{\pgfqpoint{3.232685in}{0.447529in}}%
\pgfpathlineto{\pgfqpoint{3.191640in}{0.447529in}}%
\pgfpathlineto{\pgfqpoint{3.150595in}{0.447529in}}%
\pgfpathlineto{\pgfqpoint{3.109549in}{0.447529in}}%
\pgfpathlineto{\pgfqpoint{3.068504in}{0.447529in}}%
\pgfpathlineto{\pgfqpoint{3.027459in}{0.447529in}}%
\pgfpathlineto{\pgfqpoint{2.986413in}{0.447529in}}%
\pgfpathlineto{\pgfqpoint{2.945368in}{0.447529in}}%
\pgfpathlineto{\pgfqpoint{2.904323in}{0.447529in}}%
\pgfpathlineto{\pgfqpoint{2.863277in}{0.447529in}}%
\pgfpathlineto{\pgfqpoint{2.822232in}{0.447529in}}%
\pgfpathlineto{\pgfqpoint{2.781187in}{0.447529in}}%
\pgfpathlineto{\pgfqpoint{2.740141in}{0.447529in}}%
\pgfpathlineto{\pgfqpoint{2.699096in}{0.447529in}}%
\pgfpathlineto{\pgfqpoint{2.658051in}{0.447529in}}%
\pgfpathlineto{\pgfqpoint{2.617005in}{0.447529in}}%
\pgfpathlineto{\pgfqpoint{2.575960in}{0.447529in}}%
\pgfpathlineto{\pgfqpoint{2.534915in}{0.447529in}}%
\pgfpathlineto{\pgfqpoint{2.533635in}{0.465415in}}%
\pgfpathlineto{\pgfqpoint{2.532147in}{0.486043in}}%
\pgfpathlineto{\pgfqpoint{2.530645in}{0.506671in}}%
\pgfpathlineto{\pgfqpoint{2.529127in}{0.527299in}}%
\pgfpathlineto{\pgfqpoint{2.527592in}{0.547927in}}%
\pgfpathlineto{\pgfqpoint{2.526040in}{0.568554in}}%
\pgfpathlineto{\pgfqpoint{2.524470in}{0.589182in}}%
\pgfpathlineto{\pgfqpoint{2.522881in}{0.609810in}}%
\pgfpathlineto{\pgfqpoint{2.521271in}{0.630438in}}%
\pgfpathlineto{\pgfqpoint{2.519640in}{0.651066in}}%
\pgfpathlineto{\pgfqpoint{2.517986in}{0.671694in}}%
\pgfpathlineto{\pgfqpoint{2.516309in}{0.692321in}}%
\pgfpathlineto{\pgfqpoint{2.514606in}{0.712949in}}%
\pgfpathlineto{\pgfqpoint{2.512877in}{0.733577in}}%
\pgfpathlineto{\pgfqpoint{2.511119in}{0.754205in}}%
\pgfpathlineto{\pgfqpoint{2.509332in}{0.774833in}}%
\pgfpathlineto{\pgfqpoint{2.507512in}{0.795460in}}%
\pgfpathlineto{\pgfqpoint{2.505658in}{0.816088in}}%
\pgfpathlineto{\pgfqpoint{2.503768in}{0.836716in}}%
\pgfpathlineto{\pgfqpoint{2.501839in}{0.857344in}}%
\pgfpathlineto{\pgfqpoint{2.499869in}{0.877972in}}%
\pgfpathlineto{\pgfqpoint{2.497855in}{0.898600in}}%
\pgfpathlineto{\pgfqpoint{2.495793in}{0.919227in}}%
\pgfpathlineto{\pgfqpoint{2.493869in}{0.938050in}}%
\pgfpathlineto{\pgfqpoint{2.452824in}{0.921850in}}%
\pgfpathlineto{\pgfqpoint{2.445654in}{0.919227in}}%
\pgfpathlineto{\pgfqpoint{2.411779in}{0.907216in}}%
\pgfpathlineto{\pgfqpoint{2.379338in}{0.898600in}}%
\pgfpathlineto{\pgfqpoint{2.370733in}{0.896399in}}%
\pgfpathlineto{\pgfqpoint{2.329688in}{0.891089in}}%
\pgfpathlineto{\pgfqpoint{2.288643in}{0.891286in}}%
\pgfpathlineto{\pgfqpoint{2.247597in}{0.895477in}}%
\pgfpathlineto{\pgfqpoint{2.223766in}{0.898600in}}%
\pgfpathlineto{\pgfqpoint{2.206552in}{0.900869in}}%
\pgfpathlineto{\pgfqpoint{2.165507in}{0.904002in}}%
\pgfpathlineto{\pgfqpoint{2.124461in}{0.901519in}}%
\pgfpathlineto{\pgfqpoint{2.112951in}{0.898600in}}%
\pgfpathlineto{\pgfqpoint{2.083416in}{0.891176in}}%
\pgfpathlineto{\pgfqpoint{2.054954in}{0.877972in}}%
\pgfpathlineto{\pgfqpoint{2.042371in}{0.872127in}}%
\pgfpathlineto{\pgfqpoint{2.019611in}{0.857344in}}%
\pgfpathlineto{\pgfqpoint{2.001325in}{0.845338in}}%
\pgfpathlineto{\pgfqpoint{1.990222in}{0.836716in}}%
\pgfpathlineto{\pgfqpoint{1.964059in}{0.816088in}}%
\pgfpathlineto{\pgfqpoint{1.960280in}{0.813084in}}%
\pgfpathlineto{\pgfqpoint{1.939103in}{0.795460in}}%
\pgfpathlineto{\pgfqpoint{1.919235in}{0.778513in}}%
\pgfpathlineto{\pgfqpoint{1.914687in}{0.774833in}}%
\pgfpathlineto{\pgfqpoint{1.889561in}{0.754205in}}%
\pgfpathlineto{\pgfqpoint{1.878189in}{0.744659in}}%
\pgfpathlineto{\pgfqpoint{1.863213in}{0.733577in}}%
\pgfpathlineto{\pgfqpoint{1.837144in}{0.713737in}}%
\pgfpathlineto{\pgfqpoint{1.835909in}{0.712949in}}%
\pgfpathlineto{\pgfqpoint{1.804077in}{0.692321in}}%
\pgfpathlineto{\pgfqpoint{1.796099in}{0.687030in}}%
\pgfpathlineto{\pgfqpoint{1.768138in}{0.671694in}}%
\pgfpathlineto{\pgfqpoint{1.755053in}{0.664382in}}%
\pgfpathlineto{\pgfqpoint{1.726736in}{0.651066in}}%
\pgfpathlineto{\pgfqpoint{1.714008in}{0.645022in}}%
\pgfpathlineto{\pgfqpoint{1.678496in}{0.630438in}}%
\pgfpathlineto{\pgfqpoint{1.672963in}{0.628167in}}%
\pgfpathlineto{\pgfqpoint{1.631917in}{0.613815in}}%
\pgfpathlineto{\pgfqpoint{1.617186in}{0.609810in}}%
\pgfpathlineto{\pgfqpoint{1.590872in}{0.602819in}}%
\pgfpathlineto{\pgfqpoint{1.549827in}{0.596634in}}%
\pgfpathlineto{\pgfqpoint{1.508781in}{0.596537in}}%
\pgfpathlineto{\pgfqpoint{1.467736in}{0.602805in}}%
\pgfpathlineto{\pgfqpoint{1.442602in}{0.609810in}}%
\pgfpathlineto{\pgfqpoint{1.426691in}{0.614246in}}%
\pgfpathlineto{\pgfqpoint{1.385645in}{0.628203in}}%
\pgfpathlineto{\pgfqpoint{1.378416in}{0.630438in}}%
\pgfpathlineto{\pgfqpoint{1.344600in}{0.640975in}}%
\pgfpathlineto{\pgfqpoint{1.303555in}{0.648329in}}%
\pgfpathlineto{\pgfqpoint{1.262509in}{0.646768in}}%
\pgfpathlineto{\pgfqpoint{1.221464in}{0.633998in}}%
\pgfpathlineto{\pgfqpoint{1.215437in}{0.630438in}}%
\pgfpathlineto{\pgfqpoint{1.180648in}{0.609810in}}%
\pgfpathlineto{\pgfqpoint{1.180419in}{0.609674in}}%
\pgfpathlineto{\pgfqpoint{1.155663in}{0.589182in}}%
\pgfpathlineto{\pgfqpoint{1.139373in}{0.575517in}}%
\pgfpathlineto{\pgfqpoint{1.132329in}{0.568554in}}%
\pgfpathlineto{\pgfqpoint{1.111613in}{0.547927in}}%
\pgfpathlineto{\pgfqpoint{1.098328in}{0.534516in}}%
\pgfpathlineto{\pgfqpoint{1.091499in}{0.527299in}}%
\pgfpathlineto{\pgfqpoint{1.072155in}{0.506671in}}%
\pgfpathlineto{\pgfqpoint{1.057283in}{0.490521in}}%
\pgfpathlineto{\pgfqpoint{1.052968in}{0.486043in}}%
\pgfpathlineto{\pgfqpoint{1.033246in}{0.465415in}}%
\pgfpathlineto{\pgfqpoint{1.016237in}{0.447212in}}%
\pgfpathlineto{\pgfqpoint{1.013692in}{0.444787in}}%
\pgfpathlineto{\pgfqpoint{0.992187in}{0.424160in}}%
\pgfpathlineto{\pgfqpoint{0.975192in}{0.407459in}}%
\pgfpathlineto{\pgfqpoint{0.970456in}{0.403532in}}%
\pgfpathlineto{\pgfqpoint{0.945837in}{0.382904in}}%
\pgfpathlineto{\pgfqpoint{0.934147in}{0.382904in}}%
\pgfpathlineto{\pgfqpoint{0.893101in}{0.382904in}}%
\pgfpathlineto{\pgfqpoint{0.854125in}{0.382904in}}%
\pgfpathclose%
\pgfusepath{stroke,fill}%
\end{pgfscope}%
\begin{pgfscope}%
\pgfpathrectangle{\pgfqpoint{0.605784in}{0.382904in}}{\pgfqpoint{4.063488in}{2.042155in}}%
\pgfusepath{clip}%
\pgfsetbuttcap%
\pgfsetroundjoin%
\definecolor{currentfill}{rgb}{0.140210,0.665859,0.513427}%
\pgfsetfillcolor{currentfill}%
\pgfsetlinewidth{1.003750pt}%
\definecolor{currentstroke}{rgb}{0.140210,0.665859,0.513427}%
\pgfsetstrokecolor{currentstroke}%
\pgfsetdash{}{0pt}%
\pgfpathmoveto{\pgfqpoint{1.533995in}{2.425059in}}%
\pgfpathlineto{\pgfqpoint{1.549827in}{2.425059in}}%
\pgfpathlineto{\pgfqpoint{1.590872in}{2.425059in}}%
\pgfpathlineto{\pgfqpoint{1.631917in}{2.425059in}}%
\pgfpathlineto{\pgfqpoint{1.672963in}{2.425059in}}%
\pgfpathlineto{\pgfqpoint{1.714008in}{2.425059in}}%
\pgfpathlineto{\pgfqpoint{1.738470in}{2.425059in}}%
\pgfpathlineto{\pgfqpoint{1.717484in}{2.404431in}}%
\pgfpathlineto{\pgfqpoint{1.714008in}{2.401002in}}%
\pgfpathlineto{\pgfqpoint{1.683205in}{2.383803in}}%
\pgfpathlineto{\pgfqpoint{1.672963in}{2.378078in}}%
\pgfpathlineto{\pgfqpoint{1.631917in}{2.373951in}}%
\pgfpathlineto{\pgfqpoint{1.600786in}{2.383803in}}%
\pgfpathlineto{\pgfqpoint{1.590872in}{2.386868in}}%
\pgfpathlineto{\pgfqpoint{1.562781in}{2.404431in}}%
\pgfpathlineto{\pgfqpoint{1.549827in}{2.412425in}}%
\pgfpathclose%
\pgfusepath{stroke,fill}%
\end{pgfscope}%
\begin{pgfscope}%
\pgfpathrectangle{\pgfqpoint{0.605784in}{0.382904in}}{\pgfqpoint{4.063488in}{2.042155in}}%
\pgfusepath{clip}%
\pgfsetbuttcap%
\pgfsetroundjoin%
\definecolor{currentfill}{rgb}{0.140210,0.665859,0.513427}%
\pgfsetfillcolor{currentfill}%
\pgfsetlinewidth{1.003750pt}%
\definecolor{currentstroke}{rgb}{0.140210,0.665859,0.513427}%
\pgfsetstrokecolor{currentstroke}%
\pgfsetdash{}{0pt}%
\pgfpathmoveto{\pgfqpoint{2.533210in}{2.301292in}}%
\pgfpathlineto{\pgfqpoint{2.529832in}{2.321920in}}%
\pgfpathlineto{\pgfqpoint{2.526626in}{2.342547in}}%
\pgfpathlineto{\pgfqpoint{2.523573in}{2.363175in}}%
\pgfpathlineto{\pgfqpoint{2.520658in}{2.383803in}}%
\pgfpathlineto{\pgfqpoint{2.517868in}{2.404431in}}%
\pgfpathlineto{\pgfqpoint{2.515190in}{2.425059in}}%
\pgfpathlineto{\pgfqpoint{2.523674in}{2.425059in}}%
\pgfpathlineto{\pgfqpoint{2.526631in}{2.404431in}}%
\pgfpathlineto{\pgfqpoint{2.529720in}{2.383803in}}%
\pgfpathlineto{\pgfqpoint{2.532954in}{2.363175in}}%
\pgfpathlineto{\pgfqpoint{2.534915in}{2.351085in}}%
\pgfpathlineto{\pgfqpoint{2.575960in}{2.351085in}}%
\pgfpathlineto{\pgfqpoint{2.617005in}{2.351085in}}%
\pgfpathlineto{\pgfqpoint{2.658051in}{2.351085in}}%
\pgfpathlineto{\pgfqpoint{2.699096in}{2.351085in}}%
\pgfpathlineto{\pgfqpoint{2.740141in}{2.351085in}}%
\pgfpathlineto{\pgfqpoint{2.781187in}{2.351085in}}%
\pgfpathlineto{\pgfqpoint{2.822232in}{2.351085in}}%
\pgfpathlineto{\pgfqpoint{2.863277in}{2.351085in}}%
\pgfpathlineto{\pgfqpoint{2.904323in}{2.351085in}}%
\pgfpathlineto{\pgfqpoint{2.945368in}{2.351085in}}%
\pgfpathlineto{\pgfqpoint{2.986413in}{2.351085in}}%
\pgfpathlineto{\pgfqpoint{3.027459in}{2.351085in}}%
\pgfpathlineto{\pgfqpoint{3.068504in}{2.351085in}}%
\pgfpathlineto{\pgfqpoint{3.109549in}{2.351085in}}%
\pgfpathlineto{\pgfqpoint{3.150595in}{2.351085in}}%
\pgfpathlineto{\pgfqpoint{3.191640in}{2.351085in}}%
\pgfpathlineto{\pgfqpoint{3.232685in}{2.351085in}}%
\pgfpathlineto{\pgfqpoint{3.273731in}{2.351085in}}%
\pgfpathlineto{\pgfqpoint{3.314776in}{2.351085in}}%
\pgfpathlineto{\pgfqpoint{3.355821in}{2.351085in}}%
\pgfpathlineto{\pgfqpoint{3.396867in}{2.351085in}}%
\pgfpathlineto{\pgfqpoint{3.437912in}{2.351085in}}%
\pgfpathlineto{\pgfqpoint{3.478957in}{2.351085in}}%
\pgfpathlineto{\pgfqpoint{3.520003in}{2.351085in}}%
\pgfpathlineto{\pgfqpoint{3.561048in}{2.351085in}}%
\pgfpathlineto{\pgfqpoint{3.602093in}{2.351085in}}%
\pgfpathlineto{\pgfqpoint{3.643139in}{2.351085in}}%
\pgfpathlineto{\pgfqpoint{3.644598in}{2.342547in}}%
\pgfpathlineto{\pgfqpoint{3.648147in}{2.321920in}}%
\pgfpathlineto{\pgfqpoint{3.651762in}{2.301292in}}%
\pgfpathlineto{\pgfqpoint{3.655445in}{2.280664in}}%
\pgfpathlineto{\pgfqpoint{3.659203in}{2.260036in}}%
\pgfpathlineto{\pgfqpoint{3.663040in}{2.239408in}}%
\pgfpathlineto{\pgfqpoint{3.666960in}{2.218780in}}%
\pgfpathlineto{\pgfqpoint{3.670971in}{2.198153in}}%
\pgfpathlineto{\pgfqpoint{3.675077in}{2.177525in}}%
\pgfpathlineto{\pgfqpoint{3.679287in}{2.156897in}}%
\pgfpathlineto{\pgfqpoint{3.683607in}{2.136269in}}%
\pgfpathlineto{\pgfqpoint{3.684184in}{2.133528in}}%
\pgfpathlineto{\pgfqpoint{3.725229in}{2.133528in}}%
\pgfpathlineto{\pgfqpoint{3.766275in}{2.133528in}}%
\pgfpathlineto{\pgfqpoint{3.807320in}{2.133528in}}%
\pgfpathlineto{\pgfqpoint{3.848365in}{2.133528in}}%
\pgfpathlineto{\pgfqpoint{3.889411in}{2.133528in}}%
\pgfpathlineto{\pgfqpoint{3.930456in}{2.133528in}}%
\pgfpathlineto{\pgfqpoint{3.971501in}{2.133528in}}%
\pgfpathlineto{\pgfqpoint{4.012547in}{2.133528in}}%
\pgfpathlineto{\pgfqpoint{4.053592in}{2.133528in}}%
\pgfpathlineto{\pgfqpoint{4.094637in}{2.133528in}}%
\pgfpathlineto{\pgfqpoint{4.135683in}{2.133528in}}%
\pgfpathlineto{\pgfqpoint{4.176728in}{2.133528in}}%
\pgfpathlineto{\pgfqpoint{4.217773in}{2.133528in}}%
\pgfpathlineto{\pgfqpoint{4.258819in}{2.133528in}}%
\pgfpathlineto{\pgfqpoint{4.299864in}{2.133528in}}%
\pgfpathlineto{\pgfqpoint{4.340909in}{2.133528in}}%
\pgfpathlineto{\pgfqpoint{4.381955in}{2.133528in}}%
\pgfpathlineto{\pgfqpoint{4.423000in}{2.133528in}}%
\pgfpathlineto{\pgfqpoint{4.464045in}{2.133528in}}%
\pgfpathlineto{\pgfqpoint{4.505091in}{2.133528in}}%
\pgfpathlineto{\pgfqpoint{4.546136in}{2.133528in}}%
\pgfpathlineto{\pgfqpoint{4.587181in}{2.133528in}}%
\pgfpathlineto{\pgfqpoint{4.628227in}{2.133528in}}%
\pgfpathlineto{\pgfqpoint{4.669272in}{2.133528in}}%
\pgfpathlineto{\pgfqpoint{4.669272in}{2.115641in}}%
\pgfpathlineto{\pgfqpoint{4.669272in}{2.095013in}}%
\pgfpathlineto{\pgfqpoint{4.669272in}{2.074386in}}%
\pgfpathlineto{\pgfqpoint{4.669272in}{2.073708in}}%
\pgfpathlineto{\pgfqpoint{4.628227in}{2.073708in}}%
\pgfpathlineto{\pgfqpoint{4.587181in}{2.073708in}}%
\pgfpathlineto{\pgfqpoint{4.546136in}{2.073708in}}%
\pgfpathlineto{\pgfqpoint{4.505091in}{2.073708in}}%
\pgfpathlineto{\pgfqpoint{4.464045in}{2.073708in}}%
\pgfpathlineto{\pgfqpoint{4.423000in}{2.073708in}}%
\pgfpathlineto{\pgfqpoint{4.381955in}{2.073708in}}%
\pgfpathlineto{\pgfqpoint{4.340909in}{2.073708in}}%
\pgfpathlineto{\pgfqpoint{4.299864in}{2.073708in}}%
\pgfpathlineto{\pgfqpoint{4.258819in}{2.073708in}}%
\pgfpathlineto{\pgfqpoint{4.217773in}{2.073708in}}%
\pgfpathlineto{\pgfqpoint{4.176728in}{2.073708in}}%
\pgfpathlineto{\pgfqpoint{4.135683in}{2.073708in}}%
\pgfpathlineto{\pgfqpoint{4.094637in}{2.073708in}}%
\pgfpathlineto{\pgfqpoint{4.053592in}{2.073708in}}%
\pgfpathlineto{\pgfqpoint{4.012547in}{2.073708in}}%
\pgfpathlineto{\pgfqpoint{3.971501in}{2.073708in}}%
\pgfpathlineto{\pgfqpoint{3.930456in}{2.073708in}}%
\pgfpathlineto{\pgfqpoint{3.889411in}{2.073708in}}%
\pgfpathlineto{\pgfqpoint{3.848365in}{2.073708in}}%
\pgfpathlineto{\pgfqpoint{3.807320in}{2.073708in}}%
\pgfpathlineto{\pgfqpoint{3.766275in}{2.073708in}}%
\pgfpathlineto{\pgfqpoint{3.725229in}{2.073708in}}%
\pgfpathlineto{\pgfqpoint{3.684184in}{2.073708in}}%
\pgfpathlineto{\pgfqpoint{3.684040in}{2.074386in}}%
\pgfpathlineto{\pgfqpoint{3.679676in}{2.095013in}}%
\pgfpathlineto{\pgfqpoint{3.675433in}{2.115641in}}%
\pgfpathlineto{\pgfqpoint{3.671302in}{2.136269in}}%
\pgfpathlineto{\pgfqpoint{3.667276in}{2.156897in}}%
\pgfpathlineto{\pgfqpoint{3.663345in}{2.177525in}}%
\pgfpathlineto{\pgfqpoint{3.659506in}{2.198153in}}%
\pgfpathlineto{\pgfqpoint{3.655750in}{2.218780in}}%
\pgfpathlineto{\pgfqpoint{3.652073in}{2.239408in}}%
\pgfpathlineto{\pgfqpoint{3.648470in}{2.260036in}}%
\pgfpathlineto{\pgfqpoint{3.644936in}{2.280664in}}%
\pgfpathlineto{\pgfqpoint{3.643139in}{2.291242in}}%
\pgfpathlineto{\pgfqpoint{3.602093in}{2.291242in}}%
\pgfpathlineto{\pgfqpoint{3.561048in}{2.291242in}}%
\pgfpathlineto{\pgfqpoint{3.520003in}{2.291242in}}%
\pgfpathlineto{\pgfqpoint{3.478957in}{2.291242in}}%
\pgfpathlineto{\pgfqpoint{3.437912in}{2.291242in}}%
\pgfpathlineto{\pgfqpoint{3.396867in}{2.291242in}}%
\pgfpathlineto{\pgfqpoint{3.355821in}{2.291242in}}%
\pgfpathlineto{\pgfqpoint{3.314776in}{2.291242in}}%
\pgfpathlineto{\pgfqpoint{3.273731in}{2.291242in}}%
\pgfpathlineto{\pgfqpoint{3.232685in}{2.291242in}}%
\pgfpathlineto{\pgfqpoint{3.191640in}{2.291242in}}%
\pgfpathlineto{\pgfqpoint{3.150595in}{2.291242in}}%
\pgfpathlineto{\pgfqpoint{3.109549in}{2.291242in}}%
\pgfpathlineto{\pgfqpoint{3.068504in}{2.291242in}}%
\pgfpathlineto{\pgfqpoint{3.027459in}{2.291242in}}%
\pgfpathlineto{\pgfqpoint{2.986413in}{2.291242in}}%
\pgfpathlineto{\pgfqpoint{2.945368in}{2.291242in}}%
\pgfpathlineto{\pgfqpoint{2.904323in}{2.291242in}}%
\pgfpathlineto{\pgfqpoint{2.863277in}{2.291242in}}%
\pgfpathlineto{\pgfqpoint{2.822232in}{2.291242in}}%
\pgfpathlineto{\pgfqpoint{2.781187in}{2.291242in}}%
\pgfpathlineto{\pgfqpoint{2.740141in}{2.291242in}}%
\pgfpathlineto{\pgfqpoint{2.699096in}{2.291242in}}%
\pgfpathlineto{\pgfqpoint{2.658051in}{2.291242in}}%
\pgfpathlineto{\pgfqpoint{2.617005in}{2.291242in}}%
\pgfpathlineto{\pgfqpoint{2.575960in}{2.291242in}}%
\pgfpathlineto{\pgfqpoint{2.534915in}{2.291242in}}%
\pgfpathclose%
\pgfusepath{stroke,fill}%
\end{pgfscope}%
\begin{pgfscope}%
\pgfpathrectangle{\pgfqpoint{0.605784in}{0.382904in}}{\pgfqpoint{4.063488in}{2.042155in}}%
\pgfusepath{clip}%
\pgfsetbuttcap%
\pgfsetroundjoin%
\definecolor{currentfill}{rgb}{0.226397,0.728888,0.462789}%
\pgfsetfillcolor{currentfill}%
\pgfsetlinewidth{1.003750pt}%
\definecolor{currentstroke}{rgb}{0.226397,0.728888,0.462789}%
\pgfsetstrokecolor{currentstroke}%
\pgfsetdash{}{0pt}%
\pgfpathmoveto{\pgfqpoint{0.645075in}{2.239408in}}%
\pgfpathlineto{\pgfqpoint{0.607209in}{2.260036in}}%
\pgfpathlineto{\pgfqpoint{0.605784in}{2.260812in}}%
\pgfpathlineto{\pgfqpoint{0.605784in}{2.280664in}}%
\pgfpathlineto{\pgfqpoint{0.605784in}{2.301292in}}%
\pgfpathlineto{\pgfqpoint{0.605784in}{2.316056in}}%
\pgfpathlineto{\pgfqpoint{0.633033in}{2.301292in}}%
\pgfpathlineto{\pgfqpoint{0.646829in}{2.293813in}}%
\pgfpathlineto{\pgfqpoint{0.670575in}{2.280664in}}%
\pgfpathlineto{\pgfqpoint{0.687875in}{2.271051in}}%
\pgfpathlineto{\pgfqpoint{0.712158in}{2.260036in}}%
\pgfpathlineto{\pgfqpoint{0.728920in}{2.252361in}}%
\pgfpathlineto{\pgfqpoint{0.769965in}{2.242396in}}%
\pgfpathlineto{\pgfqpoint{0.811011in}{2.244828in}}%
\pgfpathlineto{\pgfqpoint{0.848007in}{2.260036in}}%
\pgfpathlineto{\pgfqpoint{0.852056in}{2.261694in}}%
\pgfpathlineto{\pgfqpoint{0.877309in}{2.280664in}}%
\pgfpathlineto{\pgfqpoint{0.893101in}{2.292602in}}%
\pgfpathlineto{\pgfqpoint{0.901519in}{2.301292in}}%
\pgfpathlineto{\pgfqpoint{0.921419in}{2.321920in}}%
\pgfpathlineto{\pgfqpoint{0.934147in}{2.335216in}}%
\pgfpathlineto{\pgfqpoint{0.940175in}{2.342547in}}%
\pgfpathlineto{\pgfqpoint{0.957084in}{2.363175in}}%
\pgfpathlineto{\pgfqpoint{0.973756in}{2.383803in}}%
\pgfpathlineto{\pgfqpoint{0.975192in}{2.385577in}}%
\pgfpathlineto{\pgfqpoint{0.989830in}{2.404431in}}%
\pgfpathlineto{\pgfqpoint{1.005625in}{2.425059in}}%
\pgfpathlineto{\pgfqpoint{1.016237in}{2.425059in}}%
\pgfpathlineto{\pgfqpoint{1.048628in}{2.425059in}}%
\pgfpathlineto{\pgfqpoint{1.032615in}{2.404431in}}%
\pgfpathlineto{\pgfqpoint{1.016315in}{2.383803in}}%
\pgfpathlineto{\pgfqpoint{1.016237in}{2.383704in}}%
\pgfpathlineto{\pgfqpoint{1.000797in}{2.363175in}}%
\pgfpathlineto{\pgfqpoint{0.985041in}{2.342547in}}%
\pgfpathlineto{\pgfqpoint{0.975192in}{2.329769in}}%
\pgfpathlineto{\pgfqpoint{0.968901in}{2.321920in}}%
\pgfpathlineto{\pgfqpoint{0.952314in}{2.301292in}}%
\pgfpathlineto{\pgfqpoint{0.935480in}{2.280664in}}%
\pgfpathlineto{\pgfqpoint{0.934147in}{2.279027in}}%
\pgfpathlineto{\pgfqpoint{0.916161in}{2.260036in}}%
\pgfpathlineto{\pgfqpoint{0.896369in}{2.239408in}}%
\pgfpathlineto{\pgfqpoint{0.893101in}{2.236001in}}%
\pgfpathlineto{\pgfqpoint{0.870513in}{2.218780in}}%
\pgfpathlineto{\pgfqpoint{0.852056in}{2.204805in}}%
\pgfpathlineto{\pgfqpoint{0.835801in}{2.198153in}}%
\pgfpathlineto{\pgfqpoint{0.811011in}{2.187969in}}%
\pgfpathlineto{\pgfqpoint{0.769965in}{2.185780in}}%
\pgfpathlineto{\pgfqpoint{0.728920in}{2.196219in}}%
\pgfpathlineto{\pgfqpoint{0.724807in}{2.198153in}}%
\pgfpathlineto{\pgfqpoint{0.687875in}{2.215341in}}%
\pgfpathlineto{\pgfqpoint{0.681786in}{2.218780in}}%
\pgfpathlineto{\pgfqpoint{0.646829in}{2.238451in}}%
\pgfpathclose%
\pgfusepath{stroke,fill}%
\end{pgfscope}%
\begin{pgfscope}%
\pgfpathrectangle{\pgfqpoint{0.605784in}{0.382904in}}{\pgfqpoint{4.063488in}{2.042155in}}%
\pgfusepath{clip}%
\pgfsetbuttcap%
\pgfsetroundjoin%
\definecolor{currentfill}{rgb}{0.226397,0.728888,0.462789}%
\pgfsetfillcolor{currentfill}%
\pgfsetlinewidth{1.003750pt}%
\definecolor{currentstroke}{rgb}{0.226397,0.728888,0.462789}%
\pgfsetstrokecolor{currentstroke}%
\pgfsetdash{}{0pt}%
\pgfpathmoveto{\pgfqpoint{0.970456in}{0.403532in}}%
\pgfpathlineto{\pgfqpoint{0.975192in}{0.407459in}}%
\pgfpathlineto{\pgfqpoint{0.992187in}{0.424160in}}%
\pgfpathlineto{\pgfqpoint{1.013692in}{0.444787in}}%
\pgfpathlineto{\pgfqpoint{1.016237in}{0.447212in}}%
\pgfpathlineto{\pgfqpoint{1.033246in}{0.465415in}}%
\pgfpathlineto{\pgfqpoint{1.052968in}{0.486043in}}%
\pgfpathlineto{\pgfqpoint{1.057283in}{0.490521in}}%
\pgfpathlineto{\pgfqpoint{1.072155in}{0.506671in}}%
\pgfpathlineto{\pgfqpoint{1.091499in}{0.527299in}}%
\pgfpathlineto{\pgfqpoint{1.098328in}{0.534516in}}%
\pgfpathlineto{\pgfqpoint{1.111613in}{0.547927in}}%
\pgfpathlineto{\pgfqpoint{1.132329in}{0.568554in}}%
\pgfpathlineto{\pgfqpoint{1.139373in}{0.575517in}}%
\pgfpathlineto{\pgfqpoint{1.155663in}{0.589182in}}%
\pgfpathlineto{\pgfqpoint{1.180419in}{0.609674in}}%
\pgfpathlineto{\pgfqpoint{1.180648in}{0.609810in}}%
\pgfpathlineto{\pgfqpoint{1.215437in}{0.630438in}}%
\pgfpathlineto{\pgfqpoint{1.221464in}{0.633998in}}%
\pgfpathlineto{\pgfqpoint{1.262509in}{0.646768in}}%
\pgfpathlineto{\pgfqpoint{1.303555in}{0.648329in}}%
\pgfpathlineto{\pgfqpoint{1.344600in}{0.640975in}}%
\pgfpathlineto{\pgfqpoint{1.378416in}{0.630438in}}%
\pgfpathlineto{\pgfqpoint{1.385645in}{0.628203in}}%
\pgfpathlineto{\pgfqpoint{1.426691in}{0.614246in}}%
\pgfpathlineto{\pgfqpoint{1.442602in}{0.609810in}}%
\pgfpathlineto{\pgfqpoint{1.467736in}{0.602805in}}%
\pgfpathlineto{\pgfqpoint{1.508781in}{0.596537in}}%
\pgfpathlineto{\pgfqpoint{1.549827in}{0.596634in}}%
\pgfpathlineto{\pgfqpoint{1.590872in}{0.602819in}}%
\pgfpathlineto{\pgfqpoint{1.617186in}{0.609810in}}%
\pgfpathlineto{\pgfqpoint{1.631917in}{0.613815in}}%
\pgfpathlineto{\pgfqpoint{1.672963in}{0.628167in}}%
\pgfpathlineto{\pgfqpoint{1.678496in}{0.630438in}}%
\pgfpathlineto{\pgfqpoint{1.714008in}{0.645022in}}%
\pgfpathlineto{\pgfqpoint{1.726736in}{0.651066in}}%
\pgfpathlineto{\pgfqpoint{1.755053in}{0.664382in}}%
\pgfpathlineto{\pgfqpoint{1.768138in}{0.671694in}}%
\pgfpathlineto{\pgfqpoint{1.796099in}{0.687030in}}%
\pgfpathlineto{\pgfqpoint{1.804077in}{0.692321in}}%
\pgfpathlineto{\pgfqpoint{1.835909in}{0.712949in}}%
\pgfpathlineto{\pgfqpoint{1.837144in}{0.713737in}}%
\pgfpathlineto{\pgfqpoint{1.863213in}{0.733577in}}%
\pgfpathlineto{\pgfqpoint{1.878189in}{0.744659in}}%
\pgfpathlineto{\pgfqpoint{1.889561in}{0.754205in}}%
\pgfpathlineto{\pgfqpoint{1.914687in}{0.774833in}}%
\pgfpathlineto{\pgfqpoint{1.919235in}{0.778513in}}%
\pgfpathlineto{\pgfqpoint{1.939103in}{0.795460in}}%
\pgfpathlineto{\pgfqpoint{1.960280in}{0.813084in}}%
\pgfpathlineto{\pgfqpoint{1.964059in}{0.816088in}}%
\pgfpathlineto{\pgfqpoint{1.990222in}{0.836716in}}%
\pgfpathlineto{\pgfqpoint{2.001325in}{0.845338in}}%
\pgfpathlineto{\pgfqpoint{2.019611in}{0.857344in}}%
\pgfpathlineto{\pgfqpoint{2.042371in}{0.872127in}}%
\pgfpathlineto{\pgfqpoint{2.054954in}{0.877972in}}%
\pgfpathlineto{\pgfqpoint{2.083416in}{0.891176in}}%
\pgfpathlineto{\pgfqpoint{2.112951in}{0.898600in}}%
\pgfpathlineto{\pgfqpoint{2.124461in}{0.901519in}}%
\pgfpathlineto{\pgfqpoint{2.165507in}{0.904002in}}%
\pgfpathlineto{\pgfqpoint{2.206552in}{0.900869in}}%
\pgfpathlineto{\pgfqpoint{2.223766in}{0.898600in}}%
\pgfpathlineto{\pgfqpoint{2.247597in}{0.895477in}}%
\pgfpathlineto{\pgfqpoint{2.288643in}{0.891286in}}%
\pgfpathlineto{\pgfqpoint{2.329688in}{0.891089in}}%
\pgfpathlineto{\pgfqpoint{2.370733in}{0.896399in}}%
\pgfpathlineto{\pgfqpoint{2.379338in}{0.898600in}}%
\pgfpathlineto{\pgfqpoint{2.411779in}{0.907216in}}%
\pgfpathlineto{\pgfqpoint{2.445654in}{0.919227in}}%
\pgfpathlineto{\pgfqpoint{2.452824in}{0.921850in}}%
\pgfpathlineto{\pgfqpoint{2.493869in}{0.938050in}}%
\pgfpathlineto{\pgfqpoint{2.495793in}{0.919227in}}%
\pgfpathlineto{\pgfqpoint{2.497855in}{0.898600in}}%
\pgfpathlineto{\pgfqpoint{2.499869in}{0.877972in}}%
\pgfpathlineto{\pgfqpoint{2.501839in}{0.857344in}}%
\pgfpathlineto{\pgfqpoint{2.503768in}{0.836716in}}%
\pgfpathlineto{\pgfqpoint{2.505658in}{0.816088in}}%
\pgfpathlineto{\pgfqpoint{2.507512in}{0.795460in}}%
\pgfpathlineto{\pgfqpoint{2.509332in}{0.774833in}}%
\pgfpathlineto{\pgfqpoint{2.511119in}{0.754205in}}%
\pgfpathlineto{\pgfqpoint{2.512877in}{0.733577in}}%
\pgfpathlineto{\pgfqpoint{2.514606in}{0.712949in}}%
\pgfpathlineto{\pgfqpoint{2.516309in}{0.692321in}}%
\pgfpathlineto{\pgfqpoint{2.517986in}{0.671694in}}%
\pgfpathlineto{\pgfqpoint{2.519640in}{0.651066in}}%
\pgfpathlineto{\pgfqpoint{2.521271in}{0.630438in}}%
\pgfpathlineto{\pgfqpoint{2.522881in}{0.609810in}}%
\pgfpathlineto{\pgfqpoint{2.524470in}{0.589182in}}%
\pgfpathlineto{\pgfqpoint{2.526040in}{0.568554in}}%
\pgfpathlineto{\pgfqpoint{2.527592in}{0.547927in}}%
\pgfpathlineto{\pgfqpoint{2.529127in}{0.527299in}}%
\pgfpathlineto{\pgfqpoint{2.530645in}{0.506671in}}%
\pgfpathlineto{\pgfqpoint{2.532147in}{0.486043in}}%
\pgfpathlineto{\pgfqpoint{2.533635in}{0.465415in}}%
\pgfpathlineto{\pgfqpoint{2.534915in}{0.447529in}}%
\pgfpathlineto{\pgfqpoint{2.575960in}{0.447529in}}%
\pgfpathlineto{\pgfqpoint{2.617005in}{0.447529in}}%
\pgfpathlineto{\pgfqpoint{2.658051in}{0.447529in}}%
\pgfpathlineto{\pgfqpoint{2.699096in}{0.447529in}}%
\pgfpathlineto{\pgfqpoint{2.740141in}{0.447529in}}%
\pgfpathlineto{\pgfqpoint{2.781187in}{0.447529in}}%
\pgfpathlineto{\pgfqpoint{2.822232in}{0.447529in}}%
\pgfpathlineto{\pgfqpoint{2.863277in}{0.447529in}}%
\pgfpathlineto{\pgfqpoint{2.904323in}{0.447529in}}%
\pgfpathlineto{\pgfqpoint{2.945368in}{0.447529in}}%
\pgfpathlineto{\pgfqpoint{2.986413in}{0.447529in}}%
\pgfpathlineto{\pgfqpoint{3.027459in}{0.447529in}}%
\pgfpathlineto{\pgfqpoint{3.068504in}{0.447529in}}%
\pgfpathlineto{\pgfqpoint{3.109549in}{0.447529in}}%
\pgfpathlineto{\pgfqpoint{3.150595in}{0.447529in}}%
\pgfpathlineto{\pgfqpoint{3.191640in}{0.447529in}}%
\pgfpathlineto{\pgfqpoint{3.232685in}{0.447529in}}%
\pgfpathlineto{\pgfqpoint{3.273731in}{0.447529in}}%
\pgfpathlineto{\pgfqpoint{3.314776in}{0.447529in}}%
\pgfpathlineto{\pgfqpoint{3.355821in}{0.447529in}}%
\pgfpathlineto{\pgfqpoint{3.396867in}{0.447529in}}%
\pgfpathlineto{\pgfqpoint{3.437912in}{0.447529in}}%
\pgfpathlineto{\pgfqpoint{3.478957in}{0.447529in}}%
\pgfpathlineto{\pgfqpoint{3.520003in}{0.447529in}}%
\pgfpathlineto{\pgfqpoint{3.561048in}{0.447529in}}%
\pgfpathlineto{\pgfqpoint{3.602093in}{0.447529in}}%
\pgfpathlineto{\pgfqpoint{3.643139in}{0.447529in}}%
\pgfpathlineto{\pgfqpoint{3.643716in}{0.444787in}}%
\pgfpathlineto{\pgfqpoint{3.648036in}{0.424160in}}%
\pgfpathlineto{\pgfqpoint{3.652245in}{0.403532in}}%
\pgfpathlineto{\pgfqpoint{3.656352in}{0.382904in}}%
\pgfpathlineto{\pgfqpoint{3.644886in}{0.382904in}}%
\pgfpathlineto{\pgfqpoint{3.643139in}{0.391262in}}%
\pgfpathlineto{\pgfqpoint{3.602093in}{0.391262in}}%
\pgfpathlineto{\pgfqpoint{3.561048in}{0.391262in}}%
\pgfpathlineto{\pgfqpoint{3.520003in}{0.391262in}}%
\pgfpathlineto{\pgfqpoint{3.478957in}{0.391262in}}%
\pgfpathlineto{\pgfqpoint{3.437912in}{0.391262in}}%
\pgfpathlineto{\pgfqpoint{3.396867in}{0.391262in}}%
\pgfpathlineto{\pgfqpoint{3.355821in}{0.391262in}}%
\pgfpathlineto{\pgfqpoint{3.314776in}{0.391262in}}%
\pgfpathlineto{\pgfqpoint{3.273731in}{0.391262in}}%
\pgfpathlineto{\pgfqpoint{3.232685in}{0.391262in}}%
\pgfpathlineto{\pgfqpoint{3.191640in}{0.391262in}}%
\pgfpathlineto{\pgfqpoint{3.150595in}{0.391262in}}%
\pgfpathlineto{\pgfqpoint{3.109549in}{0.391262in}}%
\pgfpathlineto{\pgfqpoint{3.068504in}{0.391262in}}%
\pgfpathlineto{\pgfqpoint{3.027459in}{0.391262in}}%
\pgfpathlineto{\pgfqpoint{2.986413in}{0.391262in}}%
\pgfpathlineto{\pgfqpoint{2.945368in}{0.391262in}}%
\pgfpathlineto{\pgfqpoint{2.904323in}{0.391262in}}%
\pgfpathlineto{\pgfqpoint{2.863277in}{0.391262in}}%
\pgfpathlineto{\pgfqpoint{2.822232in}{0.391262in}}%
\pgfpathlineto{\pgfqpoint{2.781187in}{0.391262in}}%
\pgfpathlineto{\pgfqpoint{2.740141in}{0.391262in}}%
\pgfpathlineto{\pgfqpoint{2.699096in}{0.391262in}}%
\pgfpathlineto{\pgfqpoint{2.658051in}{0.391262in}}%
\pgfpathlineto{\pgfqpoint{2.617005in}{0.391262in}}%
\pgfpathlineto{\pgfqpoint{2.575960in}{0.391262in}}%
\pgfpathlineto{\pgfqpoint{2.534915in}{0.391262in}}%
\pgfpathlineto{\pgfqpoint{2.534021in}{0.403532in}}%
\pgfpathlineto{\pgfqpoint{2.532514in}{0.424160in}}%
\pgfpathlineto{\pgfqpoint{2.530991in}{0.444787in}}%
\pgfpathlineto{\pgfqpoint{2.529454in}{0.465415in}}%
\pgfpathlineto{\pgfqpoint{2.527899in}{0.486043in}}%
\pgfpathlineto{\pgfqpoint{2.526328in}{0.506671in}}%
\pgfpathlineto{\pgfqpoint{2.524739in}{0.527299in}}%
\pgfpathlineto{\pgfqpoint{2.523131in}{0.547927in}}%
\pgfpathlineto{\pgfqpoint{2.521503in}{0.568554in}}%
\pgfpathlineto{\pgfqpoint{2.519854in}{0.589182in}}%
\pgfpathlineto{\pgfqpoint{2.518183in}{0.609810in}}%
\pgfpathlineto{\pgfqpoint{2.516489in}{0.630438in}}%
\pgfpathlineto{\pgfqpoint{2.514770in}{0.651066in}}%
\pgfpathlineto{\pgfqpoint{2.513026in}{0.671694in}}%
\pgfpathlineto{\pgfqpoint{2.511254in}{0.692321in}}%
\pgfpathlineto{\pgfqpoint{2.509454in}{0.712949in}}%
\pgfpathlineto{\pgfqpoint{2.507623in}{0.733577in}}%
\pgfpathlineto{\pgfqpoint{2.505759in}{0.754205in}}%
\pgfpathlineto{\pgfqpoint{2.503861in}{0.774833in}}%
\pgfpathlineto{\pgfqpoint{2.501927in}{0.795460in}}%
\pgfpathlineto{\pgfqpoint{2.499953in}{0.816088in}}%
\pgfpathlineto{\pgfqpoint{2.497938in}{0.836716in}}%
\pgfpathlineto{\pgfqpoint{2.495879in}{0.857344in}}%
\pgfpathlineto{\pgfqpoint{2.493869in}{0.877041in}}%
\pgfpathlineto{\pgfqpoint{2.452824in}{0.861774in}}%
\pgfpathlineto{\pgfqpoint{2.439418in}{0.857344in}}%
\pgfpathlineto{\pgfqpoint{2.411779in}{0.848471in}}%
\pgfpathlineto{\pgfqpoint{2.370733in}{0.839136in}}%
\pgfpathlineto{\pgfqpoint{2.346184in}{0.836716in}}%
\pgfpathlineto{\pgfqpoint{2.329688in}{0.835153in}}%
\pgfpathlineto{\pgfqpoint{2.288643in}{0.836437in}}%
\pgfpathlineto{\pgfqpoint{2.286353in}{0.836716in}}%
\pgfpathlineto{\pgfqpoint{2.247597in}{0.841457in}}%
\pgfpathlineto{\pgfqpoint{2.206552in}{0.847340in}}%
\pgfpathlineto{\pgfqpoint{2.165507in}{0.850670in}}%
\pgfpathlineto{\pgfqpoint{2.124461in}{0.848260in}}%
\pgfpathlineto{\pgfqpoint{2.083416in}{0.837796in}}%
\pgfpathlineto{\pgfqpoint{2.081102in}{0.836716in}}%
\pgfpathlineto{\pgfqpoint{2.042371in}{0.818624in}}%
\pgfpathlineto{\pgfqpoint{2.038493in}{0.816088in}}%
\pgfpathlineto{\pgfqpoint{2.007146in}{0.795460in}}%
\pgfpathlineto{\pgfqpoint{2.001325in}{0.791602in}}%
\pgfpathlineto{\pgfqpoint{1.980021in}{0.774833in}}%
\pgfpathlineto{\pgfqpoint{1.960280in}{0.759014in}}%
\pgfpathlineto{\pgfqpoint{1.954573in}{0.754205in}}%
\pgfpathlineto{\pgfqpoint{1.930378in}{0.733577in}}%
\pgfpathlineto{\pgfqpoint{1.919235in}{0.723910in}}%
\pgfpathlineto{\pgfqpoint{1.906035in}{0.712949in}}%
\pgfpathlineto{\pgfqpoint{1.881744in}{0.692321in}}%
\pgfpathlineto{\pgfqpoint{1.878189in}{0.689268in}}%
\pgfpathlineto{\pgfqpoint{1.855214in}{0.671694in}}%
\pgfpathlineto{\pgfqpoint{1.837144in}{0.657502in}}%
\pgfpathlineto{\pgfqpoint{1.827459in}{0.651066in}}%
\pgfpathlineto{\pgfqpoint{1.797113in}{0.630438in}}%
\pgfpathlineto{\pgfqpoint{1.796099in}{0.629740in}}%
\pgfpathlineto{\pgfqpoint{1.761235in}{0.609810in}}%
\pgfpathlineto{\pgfqpoint{1.755053in}{0.606214in}}%
\pgfpathlineto{\pgfqpoint{1.720004in}{0.589182in}}%
\pgfpathlineto{\pgfqpoint{1.714008in}{0.586242in}}%
\pgfpathlineto{\pgfqpoint{1.672963in}{0.569214in}}%
\pgfpathlineto{\pgfqpoint{1.671052in}{0.568554in}}%
\pgfpathlineto{\pgfqpoint{1.631917in}{0.555192in}}%
\pgfpathlineto{\pgfqpoint{1.603277in}{0.547927in}}%
\pgfpathlineto{\pgfqpoint{1.590872in}{0.544847in}}%
\pgfpathlineto{\pgfqpoint{1.549827in}{0.539578in}}%
\pgfpathlineto{\pgfqpoint{1.508781in}{0.540372in}}%
\pgfpathlineto{\pgfqpoint{1.467736in}{0.547329in}}%
\pgfpathlineto{\pgfqpoint{1.465688in}{0.547927in}}%
\pgfpathlineto{\pgfqpoint{1.426691in}{0.559309in}}%
\pgfpathlineto{\pgfqpoint{1.399906in}{0.568554in}}%
\pgfpathlineto{\pgfqpoint{1.385645in}{0.573489in}}%
\pgfpathlineto{\pgfqpoint{1.344600in}{0.586226in}}%
\pgfpathlineto{\pgfqpoint{1.327795in}{0.589182in}}%
\pgfpathlineto{\pgfqpoint{1.303555in}{0.593510in}}%
\pgfpathlineto{\pgfqpoint{1.262509in}{0.591795in}}%
\pgfpathlineto{\pgfqpoint{1.254184in}{0.589182in}}%
\pgfpathlineto{\pgfqpoint{1.221464in}{0.578977in}}%
\pgfpathlineto{\pgfqpoint{1.203828in}{0.568554in}}%
\pgfpathlineto{\pgfqpoint{1.180419in}{0.554668in}}%
\pgfpathlineto{\pgfqpoint{1.172272in}{0.547927in}}%
\pgfpathlineto{\pgfqpoint{1.147499in}{0.527299in}}%
\pgfpathlineto{\pgfqpoint{1.139373in}{0.520491in}}%
\pgfpathlineto{\pgfqpoint{1.125402in}{0.506671in}}%
\pgfpathlineto{\pgfqpoint{1.104830in}{0.486043in}}%
\pgfpathlineto{\pgfqpoint{1.098328in}{0.479483in}}%
\pgfpathlineto{\pgfqpoint{1.085065in}{0.465415in}}%
\pgfpathlineto{\pgfqpoint{1.065912in}{0.444787in}}%
\pgfpathlineto{\pgfqpoint{1.057283in}{0.435399in}}%
\pgfpathlineto{\pgfqpoint{1.046535in}{0.424160in}}%
\pgfpathlineto{\pgfqpoint{1.027090in}{0.403532in}}%
\pgfpathlineto{\pgfqpoint{1.016237in}{0.391852in}}%
\pgfpathlineto{\pgfqpoint{1.006954in}{0.382904in}}%
\pgfpathlineto{\pgfqpoint{0.975192in}{0.382904in}}%
\pgfpathlineto{\pgfqpoint{0.945837in}{0.382904in}}%
\pgfpathclose%
\pgfusepath{stroke,fill}%
\end{pgfscope}%
\begin{pgfscope}%
\pgfpathrectangle{\pgfqpoint{0.605784in}{0.382904in}}{\pgfqpoint{4.063488in}{2.042155in}}%
\pgfusepath{clip}%
\pgfsetbuttcap%
\pgfsetroundjoin%
\definecolor{currentfill}{rgb}{0.226397,0.728888,0.462789}%
\pgfsetfillcolor{currentfill}%
\pgfsetlinewidth{1.003750pt}%
\definecolor{currentstroke}{rgb}{0.226397,0.728888,0.462789}%
\pgfsetstrokecolor{currentstroke}%
\pgfsetdash{}{0pt}%
\pgfpathmoveto{\pgfqpoint{2.532954in}{2.363175in}}%
\pgfpathlineto{\pgfqpoint{2.529720in}{2.383803in}}%
\pgfpathlineto{\pgfqpoint{2.526631in}{2.404431in}}%
\pgfpathlineto{\pgfqpoint{2.523674in}{2.425059in}}%
\pgfpathlineto{\pgfqpoint{2.532158in}{2.425059in}}%
\pgfpathlineto{\pgfqpoint{2.534915in}{2.407405in}}%
\pgfpathlineto{\pgfqpoint{2.575960in}{2.407405in}}%
\pgfpathlineto{\pgfqpoint{2.617005in}{2.407405in}}%
\pgfpathlineto{\pgfqpoint{2.658051in}{2.407405in}}%
\pgfpathlineto{\pgfqpoint{2.699096in}{2.407405in}}%
\pgfpathlineto{\pgfqpoint{2.740141in}{2.407405in}}%
\pgfpathlineto{\pgfqpoint{2.781187in}{2.407405in}}%
\pgfpathlineto{\pgfqpoint{2.822232in}{2.407405in}}%
\pgfpathlineto{\pgfqpoint{2.863277in}{2.407405in}}%
\pgfpathlineto{\pgfqpoint{2.904323in}{2.407405in}}%
\pgfpathlineto{\pgfqpoint{2.945368in}{2.407405in}}%
\pgfpathlineto{\pgfqpoint{2.986413in}{2.407405in}}%
\pgfpathlineto{\pgfqpoint{3.027459in}{2.407405in}}%
\pgfpathlineto{\pgfqpoint{3.068504in}{2.407405in}}%
\pgfpathlineto{\pgfqpoint{3.109549in}{2.407405in}}%
\pgfpathlineto{\pgfqpoint{3.150595in}{2.407405in}}%
\pgfpathlineto{\pgfqpoint{3.191640in}{2.407405in}}%
\pgfpathlineto{\pgfqpoint{3.232685in}{2.407405in}}%
\pgfpathlineto{\pgfqpoint{3.273731in}{2.407405in}}%
\pgfpathlineto{\pgfqpoint{3.314776in}{2.407405in}}%
\pgfpathlineto{\pgfqpoint{3.355821in}{2.407405in}}%
\pgfpathlineto{\pgfqpoint{3.396867in}{2.407405in}}%
\pgfpathlineto{\pgfqpoint{3.437912in}{2.407405in}}%
\pgfpathlineto{\pgfqpoint{3.478957in}{2.407405in}}%
\pgfpathlineto{\pgfqpoint{3.520003in}{2.407405in}}%
\pgfpathlineto{\pgfqpoint{3.561048in}{2.407405in}}%
\pgfpathlineto{\pgfqpoint{3.602093in}{2.407405in}}%
\pgfpathlineto{\pgfqpoint{3.643139in}{2.407405in}}%
\pgfpathlineto{\pgfqpoint{3.643650in}{2.404431in}}%
\pgfpathlineto{\pgfqpoint{3.647200in}{2.383803in}}%
\pgfpathlineto{\pgfqpoint{3.650812in}{2.363175in}}%
\pgfpathlineto{\pgfqpoint{3.654490in}{2.342547in}}%
\pgfpathlineto{\pgfqpoint{3.658237in}{2.321920in}}%
\pgfpathlineto{\pgfqpoint{3.662057in}{2.301292in}}%
\pgfpathlineto{\pgfqpoint{3.665955in}{2.280664in}}%
\pgfpathlineto{\pgfqpoint{3.669937in}{2.260036in}}%
\pgfpathlineto{\pgfqpoint{3.674007in}{2.239408in}}%
\pgfpathlineto{\pgfqpoint{3.678171in}{2.218780in}}%
\pgfpathlineto{\pgfqpoint{3.682436in}{2.198153in}}%
\pgfpathlineto{\pgfqpoint{3.684184in}{2.189794in}}%
\pgfpathlineto{\pgfqpoint{3.725229in}{2.189794in}}%
\pgfpathlineto{\pgfqpoint{3.766275in}{2.189794in}}%
\pgfpathlineto{\pgfqpoint{3.807320in}{2.189794in}}%
\pgfpathlineto{\pgfqpoint{3.848365in}{2.189794in}}%
\pgfpathlineto{\pgfqpoint{3.889411in}{2.189794in}}%
\pgfpathlineto{\pgfqpoint{3.930456in}{2.189794in}}%
\pgfpathlineto{\pgfqpoint{3.971501in}{2.189794in}}%
\pgfpathlineto{\pgfqpoint{4.012547in}{2.189794in}}%
\pgfpathlineto{\pgfqpoint{4.053592in}{2.189794in}}%
\pgfpathlineto{\pgfqpoint{4.094637in}{2.189794in}}%
\pgfpathlineto{\pgfqpoint{4.135683in}{2.189794in}}%
\pgfpathlineto{\pgfqpoint{4.176728in}{2.189794in}}%
\pgfpathlineto{\pgfqpoint{4.217773in}{2.189794in}}%
\pgfpathlineto{\pgfqpoint{4.258819in}{2.189794in}}%
\pgfpathlineto{\pgfqpoint{4.299864in}{2.189794in}}%
\pgfpathlineto{\pgfqpoint{4.340909in}{2.189794in}}%
\pgfpathlineto{\pgfqpoint{4.381955in}{2.189794in}}%
\pgfpathlineto{\pgfqpoint{4.423000in}{2.189794in}}%
\pgfpathlineto{\pgfqpoint{4.464045in}{2.189794in}}%
\pgfpathlineto{\pgfqpoint{4.505091in}{2.189794in}}%
\pgfpathlineto{\pgfqpoint{4.546136in}{2.189794in}}%
\pgfpathlineto{\pgfqpoint{4.587181in}{2.189794in}}%
\pgfpathlineto{\pgfqpoint{4.628227in}{2.189794in}}%
\pgfpathlineto{\pgfqpoint{4.669272in}{2.189794in}}%
\pgfpathlineto{\pgfqpoint{4.669272in}{2.177525in}}%
\pgfpathlineto{\pgfqpoint{4.669272in}{2.156897in}}%
\pgfpathlineto{\pgfqpoint{4.669272in}{2.136269in}}%
\pgfpathlineto{\pgfqpoint{4.669272in}{2.133528in}}%
\pgfpathlineto{\pgfqpoint{4.628227in}{2.133528in}}%
\pgfpathlineto{\pgfqpoint{4.587181in}{2.133528in}}%
\pgfpathlineto{\pgfqpoint{4.546136in}{2.133528in}}%
\pgfpathlineto{\pgfqpoint{4.505091in}{2.133528in}}%
\pgfpathlineto{\pgfqpoint{4.464045in}{2.133528in}}%
\pgfpathlineto{\pgfqpoint{4.423000in}{2.133528in}}%
\pgfpathlineto{\pgfqpoint{4.381955in}{2.133528in}}%
\pgfpathlineto{\pgfqpoint{4.340909in}{2.133528in}}%
\pgfpathlineto{\pgfqpoint{4.299864in}{2.133528in}}%
\pgfpathlineto{\pgfqpoint{4.258819in}{2.133528in}}%
\pgfpathlineto{\pgfqpoint{4.217773in}{2.133528in}}%
\pgfpathlineto{\pgfqpoint{4.176728in}{2.133528in}}%
\pgfpathlineto{\pgfqpoint{4.135683in}{2.133528in}}%
\pgfpathlineto{\pgfqpoint{4.094637in}{2.133528in}}%
\pgfpathlineto{\pgfqpoint{4.053592in}{2.133528in}}%
\pgfpathlineto{\pgfqpoint{4.012547in}{2.133528in}}%
\pgfpathlineto{\pgfqpoint{3.971501in}{2.133528in}}%
\pgfpathlineto{\pgfqpoint{3.930456in}{2.133528in}}%
\pgfpathlineto{\pgfqpoint{3.889411in}{2.133528in}}%
\pgfpathlineto{\pgfqpoint{3.848365in}{2.133528in}}%
\pgfpathlineto{\pgfqpoint{3.807320in}{2.133528in}}%
\pgfpathlineto{\pgfqpoint{3.766275in}{2.133528in}}%
\pgfpathlineto{\pgfqpoint{3.725229in}{2.133528in}}%
\pgfpathlineto{\pgfqpoint{3.684184in}{2.133528in}}%
\pgfpathlineto{\pgfqpoint{3.683607in}{2.136269in}}%
\pgfpathlineto{\pgfqpoint{3.679287in}{2.156897in}}%
\pgfpathlineto{\pgfqpoint{3.675077in}{2.177525in}}%
\pgfpathlineto{\pgfqpoint{3.670971in}{2.198153in}}%
\pgfpathlineto{\pgfqpoint{3.666960in}{2.218780in}}%
\pgfpathlineto{\pgfqpoint{3.663040in}{2.239408in}}%
\pgfpathlineto{\pgfqpoint{3.659203in}{2.260036in}}%
\pgfpathlineto{\pgfqpoint{3.655445in}{2.280664in}}%
\pgfpathlineto{\pgfqpoint{3.651762in}{2.301292in}}%
\pgfpathlineto{\pgfqpoint{3.648147in}{2.321920in}}%
\pgfpathlineto{\pgfqpoint{3.644598in}{2.342547in}}%
\pgfpathlineto{\pgfqpoint{3.643139in}{2.351085in}}%
\pgfpathlineto{\pgfqpoint{3.602093in}{2.351085in}}%
\pgfpathlineto{\pgfqpoint{3.561048in}{2.351085in}}%
\pgfpathlineto{\pgfqpoint{3.520003in}{2.351085in}}%
\pgfpathlineto{\pgfqpoint{3.478957in}{2.351085in}}%
\pgfpathlineto{\pgfqpoint{3.437912in}{2.351085in}}%
\pgfpathlineto{\pgfqpoint{3.396867in}{2.351085in}}%
\pgfpathlineto{\pgfqpoint{3.355821in}{2.351085in}}%
\pgfpathlineto{\pgfqpoint{3.314776in}{2.351085in}}%
\pgfpathlineto{\pgfqpoint{3.273731in}{2.351085in}}%
\pgfpathlineto{\pgfqpoint{3.232685in}{2.351085in}}%
\pgfpathlineto{\pgfqpoint{3.191640in}{2.351085in}}%
\pgfpathlineto{\pgfqpoint{3.150595in}{2.351085in}}%
\pgfpathlineto{\pgfqpoint{3.109549in}{2.351085in}}%
\pgfpathlineto{\pgfqpoint{3.068504in}{2.351085in}}%
\pgfpathlineto{\pgfqpoint{3.027459in}{2.351085in}}%
\pgfpathlineto{\pgfqpoint{2.986413in}{2.351085in}}%
\pgfpathlineto{\pgfqpoint{2.945368in}{2.351085in}}%
\pgfpathlineto{\pgfqpoint{2.904323in}{2.351085in}}%
\pgfpathlineto{\pgfqpoint{2.863277in}{2.351085in}}%
\pgfpathlineto{\pgfqpoint{2.822232in}{2.351085in}}%
\pgfpathlineto{\pgfqpoint{2.781187in}{2.351085in}}%
\pgfpathlineto{\pgfqpoint{2.740141in}{2.351085in}}%
\pgfpathlineto{\pgfqpoint{2.699096in}{2.351085in}}%
\pgfpathlineto{\pgfqpoint{2.658051in}{2.351085in}}%
\pgfpathlineto{\pgfqpoint{2.617005in}{2.351085in}}%
\pgfpathlineto{\pgfqpoint{2.575960in}{2.351085in}}%
\pgfpathlineto{\pgfqpoint{2.534915in}{2.351085in}}%
\pgfpathclose%
\pgfusepath{stroke,fill}%
\end{pgfscope}%
\begin{pgfscope}%
\pgfpathrectangle{\pgfqpoint{0.605784in}{0.382904in}}{\pgfqpoint{4.063488in}{2.042155in}}%
\pgfusepath{clip}%
\pgfsetbuttcap%
\pgfsetroundjoin%
\definecolor{currentfill}{rgb}{0.369214,0.788888,0.382914}%
\pgfsetfillcolor{currentfill}%
\pgfsetlinewidth{1.003750pt}%
\definecolor{currentstroke}{rgb}{0.369214,0.788888,0.382914}%
\pgfsetstrokecolor{currentstroke}%
\pgfsetdash{}{0pt}%
\pgfpathmoveto{\pgfqpoint{0.633033in}{2.301292in}}%
\pgfpathlineto{\pgfqpoint{0.605784in}{2.316056in}}%
\pgfpathlineto{\pgfqpoint{0.605784in}{2.321920in}}%
\pgfpathlineto{\pgfqpoint{0.605784in}{2.342547in}}%
\pgfpathlineto{\pgfqpoint{0.605784in}{2.363175in}}%
\pgfpathlineto{\pgfqpoint{0.605784in}{2.368508in}}%
\pgfpathlineto{\pgfqpoint{0.615680in}{2.363175in}}%
\pgfpathlineto{\pgfqpoint{0.646829in}{2.346383in}}%
\pgfpathlineto{\pgfqpoint{0.653856in}{2.342547in}}%
\pgfpathlineto{\pgfqpoint{0.687875in}{2.323919in}}%
\pgfpathlineto{\pgfqpoint{0.692386in}{2.321920in}}%
\pgfpathlineto{\pgfqpoint{0.728920in}{2.305587in}}%
\pgfpathlineto{\pgfqpoint{0.747375in}{2.301292in}}%
\pgfpathlineto{\pgfqpoint{0.769965in}{2.295946in}}%
\pgfpathlineto{\pgfqpoint{0.811011in}{2.298626in}}%
\pgfpathlineto{\pgfqpoint{0.817493in}{2.301292in}}%
\pgfpathlineto{\pgfqpoint{0.852056in}{2.315456in}}%
\pgfpathlineto{\pgfqpoint{0.860703in}{2.321920in}}%
\pgfpathlineto{\pgfqpoint{0.888213in}{2.342547in}}%
\pgfpathlineto{\pgfqpoint{0.893101in}{2.346215in}}%
\pgfpathlineto{\pgfqpoint{0.909678in}{2.363175in}}%
\pgfpathlineto{\pgfqpoint{0.929627in}{2.383803in}}%
\pgfpathlineto{\pgfqpoint{0.934147in}{2.388480in}}%
\pgfpathlineto{\pgfqpoint{0.947374in}{2.404431in}}%
\pgfpathlineto{\pgfqpoint{0.964309in}{2.425059in}}%
\pgfpathlineto{\pgfqpoint{0.975192in}{2.425059in}}%
\pgfpathlineto{\pgfqpoint{1.005625in}{2.425059in}}%
\pgfpathlineto{\pgfqpoint{0.989830in}{2.404431in}}%
\pgfpathlineto{\pgfqpoint{0.975192in}{2.385577in}}%
\pgfpathlineto{\pgfqpoint{0.973756in}{2.383803in}}%
\pgfpathlineto{\pgfqpoint{0.957084in}{2.363175in}}%
\pgfpathlineto{\pgfqpoint{0.940175in}{2.342547in}}%
\pgfpathlineto{\pgfqpoint{0.934147in}{2.335216in}}%
\pgfpathlineto{\pgfqpoint{0.921419in}{2.321920in}}%
\pgfpathlineto{\pgfqpoint{0.901519in}{2.301292in}}%
\pgfpathlineto{\pgfqpoint{0.893101in}{2.292602in}}%
\pgfpathlineto{\pgfqpoint{0.877309in}{2.280664in}}%
\pgfpathlineto{\pgfqpoint{0.852056in}{2.261694in}}%
\pgfpathlineto{\pgfqpoint{0.848007in}{2.260036in}}%
\pgfpathlineto{\pgfqpoint{0.811011in}{2.244828in}}%
\pgfpathlineto{\pgfqpoint{0.769965in}{2.242396in}}%
\pgfpathlineto{\pgfqpoint{0.728920in}{2.252361in}}%
\pgfpathlineto{\pgfqpoint{0.712158in}{2.260036in}}%
\pgfpathlineto{\pgfqpoint{0.687875in}{2.271051in}}%
\pgfpathlineto{\pgfqpoint{0.670575in}{2.280664in}}%
\pgfpathlineto{\pgfqpoint{0.646829in}{2.293813in}}%
\pgfpathclose%
\pgfusepath{stroke,fill}%
\end{pgfscope}%
\begin{pgfscope}%
\pgfpathrectangle{\pgfqpoint{0.605784in}{0.382904in}}{\pgfqpoint{4.063488in}{2.042155in}}%
\pgfusepath{clip}%
\pgfsetbuttcap%
\pgfsetroundjoin%
\definecolor{currentfill}{rgb}{0.369214,0.788888,0.382914}%
\pgfsetfillcolor{currentfill}%
\pgfsetlinewidth{1.003750pt}%
\definecolor{currentstroke}{rgb}{0.369214,0.788888,0.382914}%
\pgfsetstrokecolor{currentstroke}%
\pgfsetdash{}{0pt}%
\pgfpathmoveto{\pgfqpoint{1.016237in}{0.391852in}}%
\pgfpathlineto{\pgfqpoint{1.027090in}{0.403532in}}%
\pgfpathlineto{\pgfqpoint{1.046535in}{0.424160in}}%
\pgfpathlineto{\pgfqpoint{1.057283in}{0.435399in}}%
\pgfpathlineto{\pgfqpoint{1.065912in}{0.444787in}}%
\pgfpathlineto{\pgfqpoint{1.085065in}{0.465415in}}%
\pgfpathlineto{\pgfqpoint{1.098328in}{0.479483in}}%
\pgfpathlineto{\pgfqpoint{1.104830in}{0.486043in}}%
\pgfpathlineto{\pgfqpoint{1.125402in}{0.506671in}}%
\pgfpathlineto{\pgfqpoint{1.139373in}{0.520491in}}%
\pgfpathlineto{\pgfqpoint{1.147499in}{0.527299in}}%
\pgfpathlineto{\pgfqpoint{1.172272in}{0.547927in}}%
\pgfpathlineto{\pgfqpoint{1.180419in}{0.554668in}}%
\pgfpathlineto{\pgfqpoint{1.203828in}{0.568554in}}%
\pgfpathlineto{\pgfqpoint{1.221464in}{0.578977in}}%
\pgfpathlineto{\pgfqpoint{1.254184in}{0.589182in}}%
\pgfpathlineto{\pgfqpoint{1.262509in}{0.591795in}}%
\pgfpathlineto{\pgfqpoint{1.303555in}{0.593510in}}%
\pgfpathlineto{\pgfqpoint{1.327795in}{0.589182in}}%
\pgfpathlineto{\pgfqpoint{1.344600in}{0.586226in}}%
\pgfpathlineto{\pgfqpoint{1.385645in}{0.573489in}}%
\pgfpathlineto{\pgfqpoint{1.399906in}{0.568554in}}%
\pgfpathlineto{\pgfqpoint{1.426691in}{0.559309in}}%
\pgfpathlineto{\pgfqpoint{1.465688in}{0.547927in}}%
\pgfpathlineto{\pgfqpoint{1.467736in}{0.547329in}}%
\pgfpathlineto{\pgfqpoint{1.508781in}{0.540372in}}%
\pgfpathlineto{\pgfqpoint{1.549827in}{0.539578in}}%
\pgfpathlineto{\pgfqpoint{1.590872in}{0.544847in}}%
\pgfpathlineto{\pgfqpoint{1.603277in}{0.547927in}}%
\pgfpathlineto{\pgfqpoint{1.631917in}{0.555192in}}%
\pgfpathlineto{\pgfqpoint{1.671052in}{0.568554in}}%
\pgfpathlineto{\pgfqpoint{1.672963in}{0.569214in}}%
\pgfpathlineto{\pgfqpoint{1.714008in}{0.586242in}}%
\pgfpathlineto{\pgfqpoint{1.720004in}{0.589182in}}%
\pgfpathlineto{\pgfqpoint{1.755053in}{0.606214in}}%
\pgfpathlineto{\pgfqpoint{1.761235in}{0.609810in}}%
\pgfpathlineto{\pgfqpoint{1.796099in}{0.629740in}}%
\pgfpathlineto{\pgfqpoint{1.797113in}{0.630438in}}%
\pgfpathlineto{\pgfqpoint{1.827459in}{0.651066in}}%
\pgfpathlineto{\pgfqpoint{1.837144in}{0.657502in}}%
\pgfpathlineto{\pgfqpoint{1.855214in}{0.671694in}}%
\pgfpathlineto{\pgfqpoint{1.878189in}{0.689268in}}%
\pgfpathlineto{\pgfqpoint{1.881744in}{0.692321in}}%
\pgfpathlineto{\pgfqpoint{1.906035in}{0.712949in}}%
\pgfpathlineto{\pgfqpoint{1.919235in}{0.723910in}}%
\pgfpathlineto{\pgfqpoint{1.930378in}{0.733577in}}%
\pgfpathlineto{\pgfqpoint{1.954573in}{0.754205in}}%
\pgfpathlineto{\pgfqpoint{1.960280in}{0.759014in}}%
\pgfpathlineto{\pgfqpoint{1.980021in}{0.774833in}}%
\pgfpathlineto{\pgfqpoint{2.001325in}{0.791602in}}%
\pgfpathlineto{\pgfqpoint{2.007146in}{0.795460in}}%
\pgfpathlineto{\pgfqpoint{2.038493in}{0.816088in}}%
\pgfpathlineto{\pgfqpoint{2.042371in}{0.818624in}}%
\pgfpathlineto{\pgfqpoint{2.081102in}{0.836716in}}%
\pgfpathlineto{\pgfqpoint{2.083416in}{0.837796in}}%
\pgfpathlineto{\pgfqpoint{2.124461in}{0.848260in}}%
\pgfpathlineto{\pgfqpoint{2.165507in}{0.850670in}}%
\pgfpathlineto{\pgfqpoint{2.206552in}{0.847340in}}%
\pgfpathlineto{\pgfqpoint{2.247597in}{0.841457in}}%
\pgfpathlineto{\pgfqpoint{2.286353in}{0.836716in}}%
\pgfpathlineto{\pgfqpoint{2.288643in}{0.836437in}}%
\pgfpathlineto{\pgfqpoint{2.329688in}{0.835153in}}%
\pgfpathlineto{\pgfqpoint{2.346184in}{0.836716in}}%
\pgfpathlineto{\pgfqpoint{2.370733in}{0.839136in}}%
\pgfpathlineto{\pgfqpoint{2.411779in}{0.848471in}}%
\pgfpathlineto{\pgfqpoint{2.439418in}{0.857344in}}%
\pgfpathlineto{\pgfqpoint{2.452824in}{0.861774in}}%
\pgfpathlineto{\pgfqpoint{2.493869in}{0.877041in}}%
\pgfpathlineto{\pgfqpoint{2.495879in}{0.857344in}}%
\pgfpathlineto{\pgfqpoint{2.497938in}{0.836716in}}%
\pgfpathlineto{\pgfqpoint{2.499953in}{0.816088in}}%
\pgfpathlineto{\pgfqpoint{2.501927in}{0.795460in}}%
\pgfpathlineto{\pgfqpoint{2.503861in}{0.774833in}}%
\pgfpathlineto{\pgfqpoint{2.505759in}{0.754205in}}%
\pgfpathlineto{\pgfqpoint{2.507623in}{0.733577in}}%
\pgfpathlineto{\pgfqpoint{2.509454in}{0.712949in}}%
\pgfpathlineto{\pgfqpoint{2.511254in}{0.692321in}}%
\pgfpathlineto{\pgfqpoint{2.513026in}{0.671694in}}%
\pgfpathlineto{\pgfqpoint{2.514770in}{0.651066in}}%
\pgfpathlineto{\pgfqpoint{2.516489in}{0.630438in}}%
\pgfpathlineto{\pgfqpoint{2.518183in}{0.609810in}}%
\pgfpathlineto{\pgfqpoint{2.519854in}{0.589182in}}%
\pgfpathlineto{\pgfqpoint{2.521503in}{0.568554in}}%
\pgfpathlineto{\pgfqpoint{2.523131in}{0.547927in}}%
\pgfpathlineto{\pgfqpoint{2.524739in}{0.527299in}}%
\pgfpathlineto{\pgfqpoint{2.526328in}{0.506671in}}%
\pgfpathlineto{\pgfqpoint{2.527899in}{0.486043in}}%
\pgfpathlineto{\pgfqpoint{2.529454in}{0.465415in}}%
\pgfpathlineto{\pgfqpoint{2.530991in}{0.444787in}}%
\pgfpathlineto{\pgfqpoint{2.532514in}{0.424160in}}%
\pgfpathlineto{\pgfqpoint{2.534021in}{0.403532in}}%
\pgfpathlineto{\pgfqpoint{2.534915in}{0.391262in}}%
\pgfpathlineto{\pgfqpoint{2.575960in}{0.391262in}}%
\pgfpathlineto{\pgfqpoint{2.617005in}{0.391262in}}%
\pgfpathlineto{\pgfqpoint{2.658051in}{0.391262in}}%
\pgfpathlineto{\pgfqpoint{2.699096in}{0.391262in}}%
\pgfpathlineto{\pgfqpoint{2.740141in}{0.391262in}}%
\pgfpathlineto{\pgfqpoint{2.781187in}{0.391262in}}%
\pgfpathlineto{\pgfqpoint{2.822232in}{0.391262in}}%
\pgfpathlineto{\pgfqpoint{2.863277in}{0.391262in}}%
\pgfpathlineto{\pgfqpoint{2.904323in}{0.391262in}}%
\pgfpathlineto{\pgfqpoint{2.945368in}{0.391262in}}%
\pgfpathlineto{\pgfqpoint{2.986413in}{0.391262in}}%
\pgfpathlineto{\pgfqpoint{3.027459in}{0.391262in}}%
\pgfpathlineto{\pgfqpoint{3.068504in}{0.391262in}}%
\pgfpathlineto{\pgfqpoint{3.109549in}{0.391262in}}%
\pgfpathlineto{\pgfqpoint{3.150595in}{0.391262in}}%
\pgfpathlineto{\pgfqpoint{3.191640in}{0.391262in}}%
\pgfpathlineto{\pgfqpoint{3.232685in}{0.391262in}}%
\pgfpathlineto{\pgfqpoint{3.273731in}{0.391262in}}%
\pgfpathlineto{\pgfqpoint{3.314776in}{0.391262in}}%
\pgfpathlineto{\pgfqpoint{3.355821in}{0.391262in}}%
\pgfpathlineto{\pgfqpoint{3.396867in}{0.391262in}}%
\pgfpathlineto{\pgfqpoint{3.437912in}{0.391262in}}%
\pgfpathlineto{\pgfqpoint{3.478957in}{0.391262in}}%
\pgfpathlineto{\pgfqpoint{3.520003in}{0.391262in}}%
\pgfpathlineto{\pgfqpoint{3.561048in}{0.391262in}}%
\pgfpathlineto{\pgfqpoint{3.602093in}{0.391262in}}%
\pgfpathlineto{\pgfqpoint{3.643139in}{0.391262in}}%
\pgfpathlineto{\pgfqpoint{3.644886in}{0.382904in}}%
\pgfpathlineto{\pgfqpoint{3.643139in}{0.382904in}}%
\pgfpathlineto{\pgfqpoint{3.602093in}{0.382904in}}%
\pgfpathlineto{\pgfqpoint{3.561048in}{0.382904in}}%
\pgfpathlineto{\pgfqpoint{3.520003in}{0.382904in}}%
\pgfpathlineto{\pgfqpoint{3.478957in}{0.382904in}}%
\pgfpathlineto{\pgfqpoint{3.437912in}{0.382904in}}%
\pgfpathlineto{\pgfqpoint{3.396867in}{0.382904in}}%
\pgfpathlineto{\pgfqpoint{3.355821in}{0.382904in}}%
\pgfpathlineto{\pgfqpoint{3.314776in}{0.382904in}}%
\pgfpathlineto{\pgfqpoint{3.273731in}{0.382904in}}%
\pgfpathlineto{\pgfqpoint{3.232685in}{0.382904in}}%
\pgfpathlineto{\pgfqpoint{3.191640in}{0.382904in}}%
\pgfpathlineto{\pgfqpoint{3.150595in}{0.382904in}}%
\pgfpathlineto{\pgfqpoint{3.109549in}{0.382904in}}%
\pgfpathlineto{\pgfqpoint{3.068504in}{0.382904in}}%
\pgfpathlineto{\pgfqpoint{3.027459in}{0.382904in}}%
\pgfpathlineto{\pgfqpoint{2.986413in}{0.382904in}}%
\pgfpathlineto{\pgfqpoint{2.945368in}{0.382904in}}%
\pgfpathlineto{\pgfqpoint{2.904323in}{0.382904in}}%
\pgfpathlineto{\pgfqpoint{2.863277in}{0.382904in}}%
\pgfpathlineto{\pgfqpoint{2.822232in}{0.382904in}}%
\pgfpathlineto{\pgfqpoint{2.781187in}{0.382904in}}%
\pgfpathlineto{\pgfqpoint{2.740141in}{0.382904in}}%
\pgfpathlineto{\pgfqpoint{2.699096in}{0.382904in}}%
\pgfpathlineto{\pgfqpoint{2.658051in}{0.382904in}}%
\pgfpathlineto{\pgfqpoint{2.617005in}{0.382904in}}%
\pgfpathlineto{\pgfqpoint{2.575960in}{0.382904in}}%
\pgfpathlineto{\pgfqpoint{2.534915in}{0.382904in}}%
\pgfpathlineto{\pgfqpoint{2.531581in}{0.382904in}}%
\pgfpathlineto{\pgfqpoint{2.530028in}{0.403532in}}%
\pgfpathlineto{\pgfqpoint{2.528460in}{0.424160in}}%
\pgfpathlineto{\pgfqpoint{2.526875in}{0.444787in}}%
\pgfpathlineto{\pgfqpoint{2.525273in}{0.465415in}}%
\pgfpathlineto{\pgfqpoint{2.523652in}{0.486043in}}%
\pgfpathlineto{\pgfqpoint{2.522012in}{0.506671in}}%
\pgfpathlineto{\pgfqpoint{2.520351in}{0.527299in}}%
\pgfpathlineto{\pgfqpoint{2.518669in}{0.547927in}}%
\pgfpathlineto{\pgfqpoint{2.516965in}{0.568554in}}%
\pgfpathlineto{\pgfqpoint{2.515238in}{0.589182in}}%
\pgfpathlineto{\pgfqpoint{2.513485in}{0.609810in}}%
\pgfpathlineto{\pgfqpoint{2.511707in}{0.630438in}}%
\pgfpathlineto{\pgfqpoint{2.509901in}{0.651066in}}%
\pgfpathlineto{\pgfqpoint{2.508066in}{0.671694in}}%
\pgfpathlineto{\pgfqpoint{2.506200in}{0.692321in}}%
\pgfpathlineto{\pgfqpoint{2.504301in}{0.712949in}}%
\pgfpathlineto{\pgfqpoint{2.502368in}{0.733577in}}%
\pgfpathlineto{\pgfqpoint{2.500399in}{0.754205in}}%
\pgfpathlineto{\pgfqpoint{2.498391in}{0.774833in}}%
\pgfpathlineto{\pgfqpoint{2.496342in}{0.795460in}}%
\pgfpathlineto{\pgfqpoint{2.494248in}{0.816088in}}%
\pgfpathlineto{\pgfqpoint{2.493869in}{0.819808in}}%
\pgfpathlineto{\pgfqpoint{2.483500in}{0.816088in}}%
\pgfpathlineto{\pgfqpoint{2.452824in}{0.805289in}}%
\pgfpathlineto{\pgfqpoint{2.420008in}{0.795460in}}%
\pgfpathlineto{\pgfqpoint{2.411779in}{0.793062in}}%
\pgfpathlineto{\pgfqpoint{2.370733in}{0.785032in}}%
\pgfpathlineto{\pgfqpoint{2.329688in}{0.782207in}}%
\pgfpathlineto{\pgfqpoint{2.288643in}{0.784442in}}%
\pgfpathlineto{\pgfqpoint{2.247597in}{0.790082in}}%
\pgfpathlineto{\pgfqpoint{2.212207in}{0.795460in}}%
\pgfpathlineto{\pgfqpoint{2.206552in}{0.796325in}}%
\pgfpathlineto{\pgfqpoint{2.165507in}{0.799879in}}%
\pgfpathlineto{\pgfqpoint{2.124461in}{0.797493in}}%
\pgfpathlineto{\pgfqpoint{2.116528in}{0.795460in}}%
\pgfpathlineto{\pgfqpoint{2.083416in}{0.787047in}}%
\pgfpathlineto{\pgfqpoint{2.057438in}{0.774833in}}%
\pgfpathlineto{\pgfqpoint{2.042371in}{0.767740in}}%
\pgfpathlineto{\pgfqpoint{2.021870in}{0.754205in}}%
\pgfpathlineto{\pgfqpoint{2.001325in}{0.740508in}}%
\pgfpathlineto{\pgfqpoint{1.992591in}{0.733577in}}%
\pgfpathlineto{\pgfqpoint{1.966890in}{0.712949in}}%
\pgfpathlineto{\pgfqpoint{1.960280in}{0.707594in}}%
\pgfpathlineto{\pgfqpoint{1.942455in}{0.692321in}}%
\pgfpathlineto{\pgfqpoint{1.919235in}{0.671976in}}%
\pgfpathlineto{\pgfqpoint{1.918900in}{0.671694in}}%
\pgfpathlineto{\pgfqpoint{1.894616in}{0.651066in}}%
\pgfpathlineto{\pgfqpoint{1.878189in}{0.636765in}}%
\pgfpathlineto{\pgfqpoint{1.870077in}{0.630438in}}%
\pgfpathlineto{\pgfqpoint{1.844093in}{0.609810in}}%
\pgfpathlineto{\pgfqpoint{1.837144in}{0.604203in}}%
\pgfpathlineto{\pgfqpoint{1.815322in}{0.589182in}}%
\pgfpathlineto{\pgfqpoint{1.796099in}{0.575653in}}%
\pgfpathlineto{\pgfqpoint{1.783975in}{0.568554in}}%
\pgfpathlineto{\pgfqpoint{1.755053in}{0.551334in}}%
\pgfpathlineto{\pgfqpoint{1.748175in}{0.547927in}}%
\pgfpathlineto{\pgfqpoint{1.714008in}{0.530853in}}%
\pgfpathlineto{\pgfqpoint{1.705501in}{0.527299in}}%
\pgfpathlineto{\pgfqpoint{1.672963in}{0.513714in}}%
\pgfpathlineto{\pgfqpoint{1.652150in}{0.506671in}}%
\pgfpathlineto{\pgfqpoint{1.631917in}{0.499899in}}%
\pgfpathlineto{\pgfqpoint{1.590872in}{0.490142in}}%
\pgfpathlineto{\pgfqpoint{1.554155in}{0.486043in}}%
\pgfpathlineto{\pgfqpoint{1.549827in}{0.485574in}}%
\pgfpathlineto{\pgfqpoint{1.537419in}{0.486043in}}%
\pgfpathlineto{\pgfqpoint{1.508781in}{0.487132in}}%
\pgfpathlineto{\pgfqpoint{1.467736in}{0.494793in}}%
\pgfpathlineto{\pgfqpoint{1.428103in}{0.506671in}}%
\pgfpathlineto{\pgfqpoint{1.426691in}{0.507094in}}%
\pgfpathlineto{\pgfqpoint{1.385645in}{0.521512in}}%
\pgfpathlineto{\pgfqpoint{1.366963in}{0.527299in}}%
\pgfpathlineto{\pgfqpoint{1.344600in}{0.534275in}}%
\pgfpathlineto{\pgfqpoint{1.303555in}{0.541454in}}%
\pgfpathlineto{\pgfqpoint{1.262509in}{0.539644in}}%
\pgfpathlineto{\pgfqpoint{1.223414in}{0.527299in}}%
\pgfpathlineto{\pgfqpoint{1.221464in}{0.526687in}}%
\pgfpathlineto{\pgfqpoint{1.187614in}{0.506671in}}%
\pgfpathlineto{\pgfqpoint{1.180419in}{0.502401in}}%
\pgfpathlineto{\pgfqpoint{1.160641in}{0.486043in}}%
\pgfpathlineto{\pgfqpoint{1.139373in}{0.468239in}}%
\pgfpathlineto{\pgfqpoint{1.136520in}{0.465415in}}%
\pgfpathlineto{\pgfqpoint{1.115735in}{0.444787in}}%
\pgfpathlineto{\pgfqpoint{1.098328in}{0.427229in}}%
\pgfpathlineto{\pgfqpoint{1.095440in}{0.424160in}}%
\pgfpathlineto{\pgfqpoint{1.076111in}{0.403532in}}%
\pgfpathlineto{\pgfqpoint{1.057283in}{0.383023in}}%
\pgfpathlineto{\pgfqpoint{1.057169in}{0.382904in}}%
\pgfpathlineto{\pgfqpoint{1.016237in}{0.382904in}}%
\pgfpathlineto{\pgfqpoint{1.006954in}{0.382904in}}%
\pgfpathclose%
\pgfusepath{stroke,fill}%
\end{pgfscope}%
\begin{pgfscope}%
\pgfpathrectangle{\pgfqpoint{0.605784in}{0.382904in}}{\pgfqpoint{4.063488in}{2.042155in}}%
\pgfusepath{clip}%
\pgfsetbuttcap%
\pgfsetroundjoin%
\definecolor{currentfill}{rgb}{0.369214,0.788888,0.382914}%
\pgfsetfillcolor{currentfill}%
\pgfsetlinewidth{1.003750pt}%
\definecolor{currentstroke}{rgb}{0.369214,0.788888,0.382914}%
\pgfsetstrokecolor{currentstroke}%
\pgfsetdash{}{0pt}%
\pgfpathmoveto{\pgfqpoint{2.532158in}{2.425059in}}%
\pgfpathlineto{\pgfqpoint{2.534915in}{2.425059in}}%
\pgfpathlineto{\pgfqpoint{2.575960in}{2.425059in}}%
\pgfpathlineto{\pgfqpoint{2.617005in}{2.425059in}}%
\pgfpathlineto{\pgfqpoint{2.658051in}{2.425059in}}%
\pgfpathlineto{\pgfqpoint{2.699096in}{2.425059in}}%
\pgfpathlineto{\pgfqpoint{2.740141in}{2.425059in}}%
\pgfpathlineto{\pgfqpoint{2.781187in}{2.425059in}}%
\pgfpathlineto{\pgfqpoint{2.822232in}{2.425059in}}%
\pgfpathlineto{\pgfqpoint{2.863277in}{2.425059in}}%
\pgfpathlineto{\pgfqpoint{2.904323in}{2.425059in}}%
\pgfpathlineto{\pgfqpoint{2.945368in}{2.425059in}}%
\pgfpathlineto{\pgfqpoint{2.986413in}{2.425059in}}%
\pgfpathlineto{\pgfqpoint{3.027459in}{2.425059in}}%
\pgfpathlineto{\pgfqpoint{3.068504in}{2.425059in}}%
\pgfpathlineto{\pgfqpoint{3.109549in}{2.425059in}}%
\pgfpathlineto{\pgfqpoint{3.150595in}{2.425059in}}%
\pgfpathlineto{\pgfqpoint{3.191640in}{2.425059in}}%
\pgfpathlineto{\pgfqpoint{3.232685in}{2.425059in}}%
\pgfpathlineto{\pgfqpoint{3.273731in}{2.425059in}}%
\pgfpathlineto{\pgfqpoint{3.314776in}{2.425059in}}%
\pgfpathlineto{\pgfqpoint{3.355821in}{2.425059in}}%
\pgfpathlineto{\pgfqpoint{3.396867in}{2.425059in}}%
\pgfpathlineto{\pgfqpoint{3.437912in}{2.425059in}}%
\pgfpathlineto{\pgfqpoint{3.478957in}{2.425059in}}%
\pgfpathlineto{\pgfqpoint{3.520003in}{2.425059in}}%
\pgfpathlineto{\pgfqpoint{3.561048in}{2.425059in}}%
\pgfpathlineto{\pgfqpoint{3.602093in}{2.425059in}}%
\pgfpathlineto{\pgfqpoint{3.643139in}{2.425059in}}%
\pgfpathlineto{\pgfqpoint{3.649330in}{2.425059in}}%
\pgfpathlineto{\pgfqpoint{3.652992in}{2.404431in}}%
\pgfpathlineto{\pgfqpoint{3.656719in}{2.383803in}}%
\pgfpathlineto{\pgfqpoint{3.660514in}{2.363175in}}%
\pgfpathlineto{\pgfqpoint{3.664381in}{2.342547in}}%
\pgfpathlineto{\pgfqpoint{3.668326in}{2.321920in}}%
\pgfpathlineto{\pgfqpoint{3.672352in}{2.301292in}}%
\pgfpathlineto{\pgfqpoint{3.676465in}{2.280664in}}%
\pgfpathlineto{\pgfqpoint{3.680670in}{2.260036in}}%
\pgfpathlineto{\pgfqpoint{3.684184in}{2.243126in}}%
\pgfpathlineto{\pgfqpoint{3.725229in}{2.243126in}}%
\pgfpathlineto{\pgfqpoint{3.766275in}{2.243126in}}%
\pgfpathlineto{\pgfqpoint{3.807320in}{2.243126in}}%
\pgfpathlineto{\pgfqpoint{3.848365in}{2.243126in}}%
\pgfpathlineto{\pgfqpoint{3.889411in}{2.243126in}}%
\pgfpathlineto{\pgfqpoint{3.930456in}{2.243126in}}%
\pgfpathlineto{\pgfqpoint{3.971501in}{2.243126in}}%
\pgfpathlineto{\pgfqpoint{4.012547in}{2.243126in}}%
\pgfpathlineto{\pgfqpoint{4.053592in}{2.243126in}}%
\pgfpathlineto{\pgfqpoint{4.094637in}{2.243126in}}%
\pgfpathlineto{\pgfqpoint{4.135683in}{2.243126in}}%
\pgfpathlineto{\pgfqpoint{4.176728in}{2.243126in}}%
\pgfpathlineto{\pgfqpoint{4.217773in}{2.243126in}}%
\pgfpathlineto{\pgfqpoint{4.258819in}{2.243126in}}%
\pgfpathlineto{\pgfqpoint{4.299864in}{2.243126in}}%
\pgfpathlineto{\pgfqpoint{4.340909in}{2.243126in}}%
\pgfpathlineto{\pgfqpoint{4.381955in}{2.243126in}}%
\pgfpathlineto{\pgfqpoint{4.423000in}{2.243126in}}%
\pgfpathlineto{\pgfqpoint{4.464045in}{2.243126in}}%
\pgfpathlineto{\pgfqpoint{4.505091in}{2.243126in}}%
\pgfpathlineto{\pgfqpoint{4.546136in}{2.243126in}}%
\pgfpathlineto{\pgfqpoint{4.587181in}{2.243126in}}%
\pgfpathlineto{\pgfqpoint{4.628227in}{2.243126in}}%
\pgfpathlineto{\pgfqpoint{4.669272in}{2.243126in}}%
\pgfpathlineto{\pgfqpoint{4.669272in}{2.239408in}}%
\pgfpathlineto{\pgfqpoint{4.669272in}{2.218780in}}%
\pgfpathlineto{\pgfqpoint{4.669272in}{2.198153in}}%
\pgfpathlineto{\pgfqpoint{4.669272in}{2.189794in}}%
\pgfpathlineto{\pgfqpoint{4.628227in}{2.189794in}}%
\pgfpathlineto{\pgfqpoint{4.587181in}{2.189794in}}%
\pgfpathlineto{\pgfqpoint{4.546136in}{2.189794in}}%
\pgfpathlineto{\pgfqpoint{4.505091in}{2.189794in}}%
\pgfpathlineto{\pgfqpoint{4.464045in}{2.189794in}}%
\pgfpathlineto{\pgfqpoint{4.423000in}{2.189794in}}%
\pgfpathlineto{\pgfqpoint{4.381955in}{2.189794in}}%
\pgfpathlineto{\pgfqpoint{4.340909in}{2.189794in}}%
\pgfpathlineto{\pgfqpoint{4.299864in}{2.189794in}}%
\pgfpathlineto{\pgfqpoint{4.258819in}{2.189794in}}%
\pgfpathlineto{\pgfqpoint{4.217773in}{2.189794in}}%
\pgfpathlineto{\pgfqpoint{4.176728in}{2.189794in}}%
\pgfpathlineto{\pgfqpoint{4.135683in}{2.189794in}}%
\pgfpathlineto{\pgfqpoint{4.094637in}{2.189794in}}%
\pgfpathlineto{\pgfqpoint{4.053592in}{2.189794in}}%
\pgfpathlineto{\pgfqpoint{4.012547in}{2.189794in}}%
\pgfpathlineto{\pgfqpoint{3.971501in}{2.189794in}}%
\pgfpathlineto{\pgfqpoint{3.930456in}{2.189794in}}%
\pgfpathlineto{\pgfqpoint{3.889411in}{2.189794in}}%
\pgfpathlineto{\pgfqpoint{3.848365in}{2.189794in}}%
\pgfpathlineto{\pgfqpoint{3.807320in}{2.189794in}}%
\pgfpathlineto{\pgfqpoint{3.766275in}{2.189794in}}%
\pgfpathlineto{\pgfqpoint{3.725229in}{2.189794in}}%
\pgfpathlineto{\pgfqpoint{3.684184in}{2.189794in}}%
\pgfpathlineto{\pgfqpoint{3.682436in}{2.198153in}}%
\pgfpathlineto{\pgfqpoint{3.678171in}{2.218780in}}%
\pgfpathlineto{\pgfqpoint{3.674007in}{2.239408in}}%
\pgfpathlineto{\pgfqpoint{3.669937in}{2.260036in}}%
\pgfpathlineto{\pgfqpoint{3.665955in}{2.280664in}}%
\pgfpathlineto{\pgfqpoint{3.662057in}{2.301292in}}%
\pgfpathlineto{\pgfqpoint{3.658237in}{2.321920in}}%
\pgfpathlineto{\pgfqpoint{3.654490in}{2.342547in}}%
\pgfpathlineto{\pgfqpoint{3.650812in}{2.363175in}}%
\pgfpathlineto{\pgfqpoint{3.647200in}{2.383803in}}%
\pgfpathlineto{\pgfqpoint{3.643650in}{2.404431in}}%
\pgfpathlineto{\pgfqpoint{3.643139in}{2.407405in}}%
\pgfpathlineto{\pgfqpoint{3.602093in}{2.407405in}}%
\pgfpathlineto{\pgfqpoint{3.561048in}{2.407405in}}%
\pgfpathlineto{\pgfqpoint{3.520003in}{2.407405in}}%
\pgfpathlineto{\pgfqpoint{3.478957in}{2.407405in}}%
\pgfpathlineto{\pgfqpoint{3.437912in}{2.407405in}}%
\pgfpathlineto{\pgfqpoint{3.396867in}{2.407405in}}%
\pgfpathlineto{\pgfqpoint{3.355821in}{2.407405in}}%
\pgfpathlineto{\pgfqpoint{3.314776in}{2.407405in}}%
\pgfpathlineto{\pgfqpoint{3.273731in}{2.407405in}}%
\pgfpathlineto{\pgfqpoint{3.232685in}{2.407405in}}%
\pgfpathlineto{\pgfqpoint{3.191640in}{2.407405in}}%
\pgfpathlineto{\pgfqpoint{3.150595in}{2.407405in}}%
\pgfpathlineto{\pgfqpoint{3.109549in}{2.407405in}}%
\pgfpathlineto{\pgfqpoint{3.068504in}{2.407405in}}%
\pgfpathlineto{\pgfqpoint{3.027459in}{2.407405in}}%
\pgfpathlineto{\pgfqpoint{2.986413in}{2.407405in}}%
\pgfpathlineto{\pgfqpoint{2.945368in}{2.407405in}}%
\pgfpathlineto{\pgfqpoint{2.904323in}{2.407405in}}%
\pgfpathlineto{\pgfqpoint{2.863277in}{2.407405in}}%
\pgfpathlineto{\pgfqpoint{2.822232in}{2.407405in}}%
\pgfpathlineto{\pgfqpoint{2.781187in}{2.407405in}}%
\pgfpathlineto{\pgfqpoint{2.740141in}{2.407405in}}%
\pgfpathlineto{\pgfqpoint{2.699096in}{2.407405in}}%
\pgfpathlineto{\pgfqpoint{2.658051in}{2.407405in}}%
\pgfpathlineto{\pgfqpoint{2.617005in}{2.407405in}}%
\pgfpathlineto{\pgfqpoint{2.575960in}{2.407405in}}%
\pgfpathlineto{\pgfqpoint{2.534915in}{2.407405in}}%
\pgfpathclose%
\pgfusepath{stroke,fill}%
\end{pgfscope}%
\begin{pgfscope}%
\pgfpathrectangle{\pgfqpoint{0.605784in}{0.382904in}}{\pgfqpoint{4.063488in}{2.042155in}}%
\pgfusepath{clip}%
\pgfsetbuttcap%
\pgfsetroundjoin%
\definecolor{currentfill}{rgb}{0.535621,0.835785,0.281908}%
\pgfsetfillcolor{currentfill}%
\pgfsetlinewidth{1.003750pt}%
\definecolor{currentstroke}{rgb}{0.535621,0.835785,0.281908}%
\pgfsetstrokecolor{currentstroke}%
\pgfsetdash{}{0pt}%
\pgfpathmoveto{\pgfqpoint{0.615680in}{2.363175in}}%
\pgfpathlineto{\pgfqpoint{0.605784in}{2.368508in}}%
\pgfpathlineto{\pgfqpoint{0.605784in}{2.383803in}}%
\pgfpathlineto{\pgfqpoint{0.605784in}{2.404431in}}%
\pgfpathlineto{\pgfqpoint{0.605784in}{2.418516in}}%
\pgfpathlineto{\pgfqpoint{0.632012in}{2.404431in}}%
\pgfpathlineto{\pgfqpoint{0.646829in}{2.396470in}}%
\pgfpathlineto{\pgfqpoint{0.670244in}{2.383803in}}%
\pgfpathlineto{\pgfqpoint{0.687875in}{2.374235in}}%
\pgfpathlineto{\pgfqpoint{0.713194in}{2.363175in}}%
\pgfpathlineto{\pgfqpoint{0.728920in}{2.356247in}}%
\pgfpathlineto{\pgfqpoint{0.769965in}{2.346928in}}%
\pgfpathlineto{\pgfqpoint{0.811011in}{2.349767in}}%
\pgfpathlineto{\pgfqpoint{0.843601in}{2.363175in}}%
\pgfpathlineto{\pgfqpoint{0.852056in}{2.366643in}}%
\pgfpathlineto{\pgfqpoint{0.875164in}{2.383803in}}%
\pgfpathlineto{\pgfqpoint{0.893101in}{2.397201in}}%
\pgfpathlineto{\pgfqpoint{0.900207in}{2.404431in}}%
\pgfpathlineto{\pgfqpoint{0.920420in}{2.425059in}}%
\pgfpathlineto{\pgfqpoint{0.934147in}{2.425059in}}%
\pgfpathlineto{\pgfqpoint{0.964309in}{2.425059in}}%
\pgfpathlineto{\pgfqpoint{0.947374in}{2.404431in}}%
\pgfpathlineto{\pgfqpoint{0.934147in}{2.388480in}}%
\pgfpathlineto{\pgfqpoint{0.929627in}{2.383803in}}%
\pgfpathlineto{\pgfqpoint{0.909678in}{2.363175in}}%
\pgfpathlineto{\pgfqpoint{0.893101in}{2.346215in}}%
\pgfpathlineto{\pgfqpoint{0.888213in}{2.342547in}}%
\pgfpathlineto{\pgfqpoint{0.860703in}{2.321920in}}%
\pgfpathlineto{\pgfqpoint{0.852056in}{2.315456in}}%
\pgfpathlineto{\pgfqpoint{0.817493in}{2.301292in}}%
\pgfpathlineto{\pgfqpoint{0.811011in}{2.298626in}}%
\pgfpathlineto{\pgfqpoint{0.769965in}{2.295946in}}%
\pgfpathlineto{\pgfqpoint{0.747375in}{2.301292in}}%
\pgfpathlineto{\pgfqpoint{0.728920in}{2.305587in}}%
\pgfpathlineto{\pgfqpoint{0.692386in}{2.321920in}}%
\pgfpathlineto{\pgfqpoint{0.687875in}{2.323919in}}%
\pgfpathlineto{\pgfqpoint{0.653856in}{2.342547in}}%
\pgfpathlineto{\pgfqpoint{0.646829in}{2.346383in}}%
\pgfpathclose%
\pgfusepath{stroke,fill}%
\end{pgfscope}%
\begin{pgfscope}%
\pgfpathrectangle{\pgfqpoint{0.605784in}{0.382904in}}{\pgfqpoint{4.063488in}{2.042155in}}%
\pgfusepath{clip}%
\pgfsetbuttcap%
\pgfsetroundjoin%
\definecolor{currentfill}{rgb}{0.535621,0.835785,0.281908}%
\pgfsetfillcolor{currentfill}%
\pgfsetlinewidth{1.003750pt}%
\definecolor{currentstroke}{rgb}{0.535621,0.835785,0.281908}%
\pgfsetstrokecolor{currentstroke}%
\pgfsetdash{}{0pt}%
\pgfpathmoveto{\pgfqpoint{1.057283in}{0.383023in}}%
\pgfpathlineto{\pgfqpoint{1.076111in}{0.403532in}}%
\pgfpathlineto{\pgfqpoint{1.095440in}{0.424160in}}%
\pgfpathlineto{\pgfqpoint{1.098328in}{0.427229in}}%
\pgfpathlineto{\pgfqpoint{1.115735in}{0.444787in}}%
\pgfpathlineto{\pgfqpoint{1.136520in}{0.465415in}}%
\pgfpathlineto{\pgfqpoint{1.139373in}{0.468239in}}%
\pgfpathlineto{\pgfqpoint{1.160641in}{0.486043in}}%
\pgfpathlineto{\pgfqpoint{1.180419in}{0.502401in}}%
\pgfpathlineto{\pgfqpoint{1.187614in}{0.506671in}}%
\pgfpathlineto{\pgfqpoint{1.221464in}{0.526687in}}%
\pgfpathlineto{\pgfqpoint{1.223414in}{0.527299in}}%
\pgfpathlineto{\pgfqpoint{1.262509in}{0.539644in}}%
\pgfpathlineto{\pgfqpoint{1.303555in}{0.541454in}}%
\pgfpathlineto{\pgfqpoint{1.344600in}{0.534275in}}%
\pgfpathlineto{\pgfqpoint{1.366963in}{0.527299in}}%
\pgfpathlineto{\pgfqpoint{1.385645in}{0.521512in}}%
\pgfpathlineto{\pgfqpoint{1.426691in}{0.507094in}}%
\pgfpathlineto{\pgfqpoint{1.428103in}{0.506671in}}%
\pgfpathlineto{\pgfqpoint{1.467736in}{0.494793in}}%
\pgfpathlineto{\pgfqpoint{1.508781in}{0.487132in}}%
\pgfpathlineto{\pgfqpoint{1.537419in}{0.486043in}}%
\pgfpathlineto{\pgfqpoint{1.549827in}{0.485574in}}%
\pgfpathlineto{\pgfqpoint{1.554155in}{0.486043in}}%
\pgfpathlineto{\pgfqpoint{1.590872in}{0.490142in}}%
\pgfpathlineto{\pgfqpoint{1.631917in}{0.499899in}}%
\pgfpathlineto{\pgfqpoint{1.652150in}{0.506671in}}%
\pgfpathlineto{\pgfqpoint{1.672963in}{0.513714in}}%
\pgfpathlineto{\pgfqpoint{1.705501in}{0.527299in}}%
\pgfpathlineto{\pgfqpoint{1.714008in}{0.530853in}}%
\pgfpathlineto{\pgfqpoint{1.748175in}{0.547927in}}%
\pgfpathlineto{\pgfqpoint{1.755053in}{0.551334in}}%
\pgfpathlineto{\pgfqpoint{1.783975in}{0.568554in}}%
\pgfpathlineto{\pgfqpoint{1.796099in}{0.575653in}}%
\pgfpathlineto{\pgfqpoint{1.815322in}{0.589182in}}%
\pgfpathlineto{\pgfqpoint{1.837144in}{0.604203in}}%
\pgfpathlineto{\pgfqpoint{1.844093in}{0.609810in}}%
\pgfpathlineto{\pgfqpoint{1.870077in}{0.630438in}}%
\pgfpathlineto{\pgfqpoint{1.878189in}{0.636765in}}%
\pgfpathlineto{\pgfqpoint{1.894616in}{0.651066in}}%
\pgfpathlineto{\pgfqpoint{1.918900in}{0.671694in}}%
\pgfpathlineto{\pgfqpoint{1.919235in}{0.671976in}}%
\pgfpathlineto{\pgfqpoint{1.942455in}{0.692321in}}%
\pgfpathlineto{\pgfqpoint{1.960280in}{0.707594in}}%
\pgfpathlineto{\pgfqpoint{1.966890in}{0.712949in}}%
\pgfpathlineto{\pgfqpoint{1.992591in}{0.733577in}}%
\pgfpathlineto{\pgfqpoint{2.001325in}{0.740508in}}%
\pgfpathlineto{\pgfqpoint{2.021870in}{0.754205in}}%
\pgfpathlineto{\pgfqpoint{2.042371in}{0.767740in}}%
\pgfpathlineto{\pgfqpoint{2.057438in}{0.774833in}}%
\pgfpathlineto{\pgfqpoint{2.083416in}{0.787047in}}%
\pgfpathlineto{\pgfqpoint{2.116528in}{0.795460in}}%
\pgfpathlineto{\pgfqpoint{2.124461in}{0.797493in}}%
\pgfpathlineto{\pgfqpoint{2.165507in}{0.799879in}}%
\pgfpathlineto{\pgfqpoint{2.206552in}{0.796325in}}%
\pgfpathlineto{\pgfqpoint{2.212207in}{0.795460in}}%
\pgfpathlineto{\pgfqpoint{2.247597in}{0.790082in}}%
\pgfpathlineto{\pgfqpoint{2.288643in}{0.784442in}}%
\pgfpathlineto{\pgfqpoint{2.329688in}{0.782207in}}%
\pgfpathlineto{\pgfqpoint{2.370733in}{0.785032in}}%
\pgfpathlineto{\pgfqpoint{2.411779in}{0.793062in}}%
\pgfpathlineto{\pgfqpoint{2.420008in}{0.795460in}}%
\pgfpathlineto{\pgfqpoint{2.452824in}{0.805289in}}%
\pgfpathlineto{\pgfqpoint{2.483500in}{0.816088in}}%
\pgfpathlineto{\pgfqpoint{2.493869in}{0.819808in}}%
\pgfpathlineto{\pgfqpoint{2.494248in}{0.816088in}}%
\pgfpathlineto{\pgfqpoint{2.496342in}{0.795460in}}%
\pgfpathlineto{\pgfqpoint{2.498391in}{0.774833in}}%
\pgfpathlineto{\pgfqpoint{2.500399in}{0.754205in}}%
\pgfpathlineto{\pgfqpoint{2.502368in}{0.733577in}}%
\pgfpathlineto{\pgfqpoint{2.504301in}{0.712949in}}%
\pgfpathlineto{\pgfqpoint{2.506200in}{0.692321in}}%
\pgfpathlineto{\pgfqpoint{2.508066in}{0.671694in}}%
\pgfpathlineto{\pgfqpoint{2.509901in}{0.651066in}}%
\pgfpathlineto{\pgfqpoint{2.511707in}{0.630438in}}%
\pgfpathlineto{\pgfqpoint{2.513485in}{0.609810in}}%
\pgfpathlineto{\pgfqpoint{2.515238in}{0.589182in}}%
\pgfpathlineto{\pgfqpoint{2.516965in}{0.568554in}}%
\pgfpathlineto{\pgfqpoint{2.518669in}{0.547927in}}%
\pgfpathlineto{\pgfqpoint{2.520351in}{0.527299in}}%
\pgfpathlineto{\pgfqpoint{2.522012in}{0.506671in}}%
\pgfpathlineto{\pgfqpoint{2.523652in}{0.486043in}}%
\pgfpathlineto{\pgfqpoint{2.525273in}{0.465415in}}%
\pgfpathlineto{\pgfqpoint{2.526875in}{0.444787in}}%
\pgfpathlineto{\pgfqpoint{2.528460in}{0.424160in}}%
\pgfpathlineto{\pgfqpoint{2.530028in}{0.403532in}}%
\pgfpathlineto{\pgfqpoint{2.531581in}{0.382904in}}%
\pgfpathlineto{\pgfqpoint{2.527647in}{0.382904in}}%
\pgfpathlineto{\pgfqpoint{2.526036in}{0.403532in}}%
\pgfpathlineto{\pgfqpoint{2.524407in}{0.424160in}}%
\pgfpathlineto{\pgfqpoint{2.522759in}{0.444787in}}%
\pgfpathlineto{\pgfqpoint{2.521092in}{0.465415in}}%
\pgfpathlineto{\pgfqpoint{2.519404in}{0.486043in}}%
\pgfpathlineto{\pgfqpoint{2.517695in}{0.506671in}}%
\pgfpathlineto{\pgfqpoint{2.515963in}{0.527299in}}%
\pgfpathlineto{\pgfqpoint{2.514208in}{0.547927in}}%
\pgfpathlineto{\pgfqpoint{2.512428in}{0.568554in}}%
\pgfpathlineto{\pgfqpoint{2.510622in}{0.589182in}}%
\pgfpathlineto{\pgfqpoint{2.508788in}{0.609810in}}%
\pgfpathlineto{\pgfqpoint{2.506925in}{0.630438in}}%
\pgfpathlineto{\pgfqpoint{2.505031in}{0.651066in}}%
\pgfpathlineto{\pgfqpoint{2.503105in}{0.671694in}}%
\pgfpathlineto{\pgfqpoint{2.501145in}{0.692321in}}%
\pgfpathlineto{\pgfqpoint{2.499149in}{0.712949in}}%
\pgfpathlineto{\pgfqpoint{2.497114in}{0.733577in}}%
\pgfpathlineto{\pgfqpoint{2.495039in}{0.754205in}}%
\pgfpathlineto{\pgfqpoint{2.493869in}{0.765699in}}%
\pgfpathlineto{\pgfqpoint{2.460104in}{0.754205in}}%
\pgfpathlineto{\pgfqpoint{2.452824in}{0.751770in}}%
\pgfpathlineto{\pgfqpoint{2.411779in}{0.740553in}}%
\pgfpathlineto{\pgfqpoint{2.371105in}{0.733577in}}%
\pgfpathlineto{\pgfqpoint{2.370733in}{0.733515in}}%
\pgfpathlineto{\pgfqpoint{2.329688in}{0.731707in}}%
\pgfpathlineto{\pgfqpoint{2.304413in}{0.733577in}}%
\pgfpathlineto{\pgfqpoint{2.288643in}{0.734744in}}%
\pgfpathlineto{\pgfqpoint{2.247597in}{0.741007in}}%
\pgfpathlineto{\pgfqpoint{2.206552in}{0.747629in}}%
\pgfpathlineto{\pgfqpoint{2.165507in}{0.751310in}}%
\pgfpathlineto{\pgfqpoint{2.124461in}{0.748991in}}%
\pgfpathlineto{\pgfqpoint{2.083416in}{0.738470in}}%
\pgfpathlineto{\pgfqpoint{2.073057in}{0.733577in}}%
\pgfpathlineto{\pgfqpoint{2.042371in}{0.719066in}}%
\pgfpathlineto{\pgfqpoint{2.033159in}{0.712949in}}%
\pgfpathlineto{\pgfqpoint{2.002360in}{0.692321in}}%
\pgfpathlineto{\pgfqpoint{2.001325in}{0.691626in}}%
\pgfpathlineto{\pgfqpoint{1.976483in}{0.671694in}}%
\pgfpathlineto{\pgfqpoint{1.960280in}{0.658480in}}%
\pgfpathlineto{\pgfqpoint{1.951712in}{0.651066in}}%
\pgfpathlineto{\pgfqpoint{1.928180in}{0.630438in}}%
\pgfpathlineto{\pgfqpoint{1.919235in}{0.622493in}}%
\pgfpathlineto{\pgfqpoint{1.904495in}{0.609810in}}%
\pgfpathlineto{\pgfqpoint{1.881024in}{0.589182in}}%
\pgfpathlineto{\pgfqpoint{1.878189in}{0.586669in}}%
\pgfpathlineto{\pgfqpoint{1.855559in}{0.568554in}}%
\pgfpathlineto{\pgfqpoint{1.837144in}{0.553458in}}%
\pgfpathlineto{\pgfqpoint{1.829272in}{0.547927in}}%
\pgfpathlineto{\pgfqpoint{1.800414in}{0.527299in}}%
\pgfpathlineto{\pgfqpoint{1.796099in}{0.524173in}}%
\pgfpathlineto{\pgfqpoint{1.767145in}{0.506671in}}%
\pgfpathlineto{\pgfqpoint{1.755053in}{0.499246in}}%
\pgfpathlineto{\pgfqpoint{1.729070in}{0.486043in}}%
\pgfpathlineto{\pgfqpoint{1.714008in}{0.478326in}}%
\pgfpathlineto{\pgfqpoint{1.683377in}{0.465415in}}%
\pgfpathlineto{\pgfqpoint{1.672963in}{0.461029in}}%
\pgfpathlineto{\pgfqpoint{1.631917in}{0.447418in}}%
\pgfpathlineto{\pgfqpoint{1.620380in}{0.444787in}}%
\pgfpathlineto{\pgfqpoint{1.590872in}{0.438190in}}%
\pgfpathlineto{\pgfqpoint{1.549827in}{0.434302in}}%
\pgfpathlineto{\pgfqpoint{1.508781in}{0.436504in}}%
\pgfpathlineto{\pgfqpoint{1.467736in}{0.444620in}}%
\pgfpathlineto{\pgfqpoint{1.467198in}{0.444787in}}%
\pgfpathlineto{\pgfqpoint{1.426691in}{0.457366in}}%
\pgfpathlineto{\pgfqpoint{1.403949in}{0.465415in}}%
\pgfpathlineto{\pgfqpoint{1.385645in}{0.471908in}}%
\pgfpathlineto{\pgfqpoint{1.344600in}{0.484646in}}%
\pgfpathlineto{\pgfqpoint{1.336465in}{0.486043in}}%
\pgfpathlineto{\pgfqpoint{1.303555in}{0.491772in}}%
\pgfpathlineto{\pgfqpoint{1.262509in}{0.489849in}}%
\pgfpathlineto{\pgfqpoint{1.250500in}{0.486043in}}%
\pgfpathlineto{\pgfqpoint{1.221464in}{0.476892in}}%
\pgfpathlineto{\pgfqpoint{1.202061in}{0.465415in}}%
\pgfpathlineto{\pgfqpoint{1.180419in}{0.452571in}}%
\pgfpathlineto{\pgfqpoint{1.171006in}{0.444787in}}%
\pgfpathlineto{\pgfqpoint{1.146224in}{0.424160in}}%
\pgfpathlineto{\pgfqpoint{1.139373in}{0.418431in}}%
\pgfpathlineto{\pgfqpoint{1.124327in}{0.403532in}}%
\pgfpathlineto{\pgfqpoint{1.103772in}{0.382904in}}%
\pgfpathlineto{\pgfqpoint{1.098328in}{0.382904in}}%
\pgfpathlineto{\pgfqpoint{1.057283in}{0.382904in}}%
\pgfpathlineto{\pgfqpoint{1.057169in}{0.382904in}}%
\pgfpathclose%
\pgfusepath{stroke,fill}%
\end{pgfscope}%
\begin{pgfscope}%
\pgfpathrectangle{\pgfqpoint{0.605784in}{0.382904in}}{\pgfqpoint{4.063488in}{2.042155in}}%
\pgfusepath{clip}%
\pgfsetbuttcap%
\pgfsetroundjoin%
\definecolor{currentfill}{rgb}{0.535621,0.835785,0.281908}%
\pgfsetfillcolor{currentfill}%
\pgfsetlinewidth{1.003750pt}%
\definecolor{currentstroke}{rgb}{0.535621,0.835785,0.281908}%
\pgfsetstrokecolor{currentstroke}%
\pgfsetdash{}{0pt}%
\pgfpathmoveto{\pgfqpoint{3.680670in}{2.260036in}}%
\pgfpathlineto{\pgfqpoint{3.676465in}{2.280664in}}%
\pgfpathlineto{\pgfqpoint{3.672352in}{2.301292in}}%
\pgfpathlineto{\pgfqpoint{3.668326in}{2.321920in}}%
\pgfpathlineto{\pgfqpoint{3.664381in}{2.342547in}}%
\pgfpathlineto{\pgfqpoint{3.660514in}{2.363175in}}%
\pgfpathlineto{\pgfqpoint{3.656719in}{2.383803in}}%
\pgfpathlineto{\pgfqpoint{3.652992in}{2.404431in}}%
\pgfpathlineto{\pgfqpoint{3.649330in}{2.425059in}}%
\pgfpathlineto{\pgfqpoint{3.658502in}{2.425059in}}%
\pgfpathlineto{\pgfqpoint{3.662334in}{2.404431in}}%
\pgfpathlineto{\pgfqpoint{3.666237in}{2.383803in}}%
\pgfpathlineto{\pgfqpoint{3.670215in}{2.363175in}}%
\pgfpathlineto{\pgfqpoint{3.674273in}{2.342547in}}%
\pgfpathlineto{\pgfqpoint{3.678415in}{2.321920in}}%
\pgfpathlineto{\pgfqpoint{3.682647in}{2.301292in}}%
\pgfpathlineto{\pgfqpoint{3.684184in}{2.293870in}}%
\pgfpathlineto{\pgfqpoint{3.725229in}{2.293870in}}%
\pgfpathlineto{\pgfqpoint{3.766275in}{2.293870in}}%
\pgfpathlineto{\pgfqpoint{3.807320in}{2.293870in}}%
\pgfpathlineto{\pgfqpoint{3.848365in}{2.293870in}}%
\pgfpathlineto{\pgfqpoint{3.889411in}{2.293870in}}%
\pgfpathlineto{\pgfqpoint{3.930456in}{2.293870in}}%
\pgfpathlineto{\pgfqpoint{3.971501in}{2.293870in}}%
\pgfpathlineto{\pgfqpoint{4.012547in}{2.293870in}}%
\pgfpathlineto{\pgfqpoint{4.053592in}{2.293870in}}%
\pgfpathlineto{\pgfqpoint{4.094637in}{2.293870in}}%
\pgfpathlineto{\pgfqpoint{4.135683in}{2.293870in}}%
\pgfpathlineto{\pgfqpoint{4.176728in}{2.293870in}}%
\pgfpathlineto{\pgfqpoint{4.217773in}{2.293870in}}%
\pgfpathlineto{\pgfqpoint{4.258819in}{2.293870in}}%
\pgfpathlineto{\pgfqpoint{4.299864in}{2.293870in}}%
\pgfpathlineto{\pgfqpoint{4.340909in}{2.293870in}}%
\pgfpathlineto{\pgfqpoint{4.381955in}{2.293870in}}%
\pgfpathlineto{\pgfqpoint{4.423000in}{2.293870in}}%
\pgfpathlineto{\pgfqpoint{4.464045in}{2.293870in}}%
\pgfpathlineto{\pgfqpoint{4.505091in}{2.293870in}}%
\pgfpathlineto{\pgfqpoint{4.546136in}{2.293870in}}%
\pgfpathlineto{\pgfqpoint{4.587181in}{2.293870in}}%
\pgfpathlineto{\pgfqpoint{4.628227in}{2.293870in}}%
\pgfpathlineto{\pgfqpoint{4.669272in}{2.293870in}}%
\pgfpathlineto{\pgfqpoint{4.669272in}{2.280664in}}%
\pgfpathlineto{\pgfqpoint{4.669272in}{2.260036in}}%
\pgfpathlineto{\pgfqpoint{4.669272in}{2.243126in}}%
\pgfpathlineto{\pgfqpoint{4.628227in}{2.243126in}}%
\pgfpathlineto{\pgfqpoint{4.587181in}{2.243126in}}%
\pgfpathlineto{\pgfqpoint{4.546136in}{2.243126in}}%
\pgfpathlineto{\pgfqpoint{4.505091in}{2.243126in}}%
\pgfpathlineto{\pgfqpoint{4.464045in}{2.243126in}}%
\pgfpathlineto{\pgfqpoint{4.423000in}{2.243126in}}%
\pgfpathlineto{\pgfqpoint{4.381955in}{2.243126in}}%
\pgfpathlineto{\pgfqpoint{4.340909in}{2.243126in}}%
\pgfpathlineto{\pgfqpoint{4.299864in}{2.243126in}}%
\pgfpathlineto{\pgfqpoint{4.258819in}{2.243126in}}%
\pgfpathlineto{\pgfqpoint{4.217773in}{2.243126in}}%
\pgfpathlineto{\pgfqpoint{4.176728in}{2.243126in}}%
\pgfpathlineto{\pgfqpoint{4.135683in}{2.243126in}}%
\pgfpathlineto{\pgfqpoint{4.094637in}{2.243126in}}%
\pgfpathlineto{\pgfqpoint{4.053592in}{2.243126in}}%
\pgfpathlineto{\pgfqpoint{4.012547in}{2.243126in}}%
\pgfpathlineto{\pgfqpoint{3.971501in}{2.243126in}}%
\pgfpathlineto{\pgfqpoint{3.930456in}{2.243126in}}%
\pgfpathlineto{\pgfqpoint{3.889411in}{2.243126in}}%
\pgfpathlineto{\pgfqpoint{3.848365in}{2.243126in}}%
\pgfpathlineto{\pgfqpoint{3.807320in}{2.243126in}}%
\pgfpathlineto{\pgfqpoint{3.766275in}{2.243126in}}%
\pgfpathlineto{\pgfqpoint{3.725229in}{2.243126in}}%
\pgfpathlineto{\pgfqpoint{3.684184in}{2.243126in}}%
\pgfpathclose%
\pgfusepath{stroke,fill}%
\end{pgfscope}%
\begin{pgfscope}%
\pgfpathrectangle{\pgfqpoint{0.605784in}{0.382904in}}{\pgfqpoint{4.063488in}{2.042155in}}%
\pgfusepath{clip}%
\pgfsetbuttcap%
\pgfsetroundjoin%
\definecolor{currentfill}{rgb}{0.720391,0.870350,0.162603}%
\pgfsetfillcolor{currentfill}%
\pgfsetlinewidth{1.003750pt}%
\definecolor{currentstroke}{rgb}{0.720391,0.870350,0.162603}%
\pgfsetstrokecolor{currentstroke}%
\pgfsetdash{}{0pt}%
\pgfpathmoveto{\pgfqpoint{0.632012in}{2.404431in}}%
\pgfpathlineto{\pgfqpoint{0.605784in}{2.418516in}}%
\pgfpathlineto{\pgfqpoint{0.605784in}{2.425059in}}%
\pgfpathlineto{\pgfqpoint{0.646829in}{2.425059in}}%
\pgfpathlineto{\pgfqpoint{0.683011in}{2.425059in}}%
\pgfpathlineto{\pgfqpoint{0.687875in}{2.422441in}}%
\pgfpathlineto{\pgfqpoint{0.728920in}{2.404751in}}%
\pgfpathlineto{\pgfqpoint{0.730386in}{2.404431in}}%
\pgfpathlineto{\pgfqpoint{0.769965in}{2.395646in}}%
\pgfpathlineto{\pgfqpoint{0.811011in}{2.398667in}}%
\pgfpathlineto{\pgfqpoint{0.825015in}{2.404431in}}%
\pgfpathlineto{\pgfqpoint{0.852056in}{2.415524in}}%
\pgfpathlineto{\pgfqpoint{0.864946in}{2.425059in}}%
\pgfpathlineto{\pgfqpoint{0.893101in}{2.425059in}}%
\pgfpathlineto{\pgfqpoint{0.920420in}{2.425059in}}%
\pgfpathlineto{\pgfqpoint{0.900207in}{2.404431in}}%
\pgfpathlineto{\pgfqpoint{0.893101in}{2.397201in}}%
\pgfpathlineto{\pgfqpoint{0.875164in}{2.383803in}}%
\pgfpathlineto{\pgfqpoint{0.852056in}{2.366643in}}%
\pgfpathlineto{\pgfqpoint{0.843601in}{2.363175in}}%
\pgfpathlineto{\pgfqpoint{0.811011in}{2.349767in}}%
\pgfpathlineto{\pgfqpoint{0.769965in}{2.346928in}}%
\pgfpathlineto{\pgfqpoint{0.728920in}{2.356247in}}%
\pgfpathlineto{\pgfqpoint{0.713194in}{2.363175in}}%
\pgfpathlineto{\pgfqpoint{0.687875in}{2.374235in}}%
\pgfpathlineto{\pgfqpoint{0.670244in}{2.383803in}}%
\pgfpathlineto{\pgfqpoint{0.646829in}{2.396470in}}%
\pgfpathclose%
\pgfusepath{stroke,fill}%
\end{pgfscope}%
\begin{pgfscope}%
\pgfpathrectangle{\pgfqpoint{0.605784in}{0.382904in}}{\pgfqpoint{4.063488in}{2.042155in}}%
\pgfusepath{clip}%
\pgfsetbuttcap%
\pgfsetroundjoin%
\definecolor{currentfill}{rgb}{0.720391,0.870350,0.162603}%
\pgfsetfillcolor{currentfill}%
\pgfsetlinewidth{1.003750pt}%
\definecolor{currentstroke}{rgb}{0.720391,0.870350,0.162603}%
\pgfsetstrokecolor{currentstroke}%
\pgfsetdash{}{0pt}%
\pgfpathmoveto{\pgfqpoint{1.124327in}{0.403532in}}%
\pgfpathlineto{\pgfqpoint{1.139373in}{0.418431in}}%
\pgfpathlineto{\pgfqpoint{1.146224in}{0.424160in}}%
\pgfpathlineto{\pgfqpoint{1.171006in}{0.444787in}}%
\pgfpathlineto{\pgfqpoint{1.180419in}{0.452571in}}%
\pgfpathlineto{\pgfqpoint{1.202061in}{0.465415in}}%
\pgfpathlineto{\pgfqpoint{1.221464in}{0.476892in}}%
\pgfpathlineto{\pgfqpoint{1.250500in}{0.486043in}}%
\pgfpathlineto{\pgfqpoint{1.262509in}{0.489849in}}%
\pgfpathlineto{\pgfqpoint{1.303555in}{0.491772in}}%
\pgfpathlineto{\pgfqpoint{1.336465in}{0.486043in}}%
\pgfpathlineto{\pgfqpoint{1.344600in}{0.484646in}}%
\pgfpathlineto{\pgfqpoint{1.385645in}{0.471908in}}%
\pgfpathlineto{\pgfqpoint{1.403949in}{0.465415in}}%
\pgfpathlineto{\pgfqpoint{1.426691in}{0.457366in}}%
\pgfpathlineto{\pgfqpoint{1.467198in}{0.444787in}}%
\pgfpathlineto{\pgfqpoint{1.467736in}{0.444620in}}%
\pgfpathlineto{\pgfqpoint{1.508781in}{0.436504in}}%
\pgfpathlineto{\pgfqpoint{1.549827in}{0.434302in}}%
\pgfpathlineto{\pgfqpoint{1.590872in}{0.438190in}}%
\pgfpathlineto{\pgfqpoint{1.620380in}{0.444787in}}%
\pgfpathlineto{\pgfqpoint{1.631917in}{0.447418in}}%
\pgfpathlineto{\pgfqpoint{1.672963in}{0.461029in}}%
\pgfpathlineto{\pgfqpoint{1.683377in}{0.465415in}}%
\pgfpathlineto{\pgfqpoint{1.714008in}{0.478326in}}%
\pgfpathlineto{\pgfqpoint{1.729070in}{0.486043in}}%
\pgfpathlineto{\pgfqpoint{1.755053in}{0.499246in}}%
\pgfpathlineto{\pgfqpoint{1.767145in}{0.506671in}}%
\pgfpathlineto{\pgfqpoint{1.796099in}{0.524173in}}%
\pgfpathlineto{\pgfqpoint{1.800414in}{0.527299in}}%
\pgfpathlineto{\pgfqpoint{1.829272in}{0.547927in}}%
\pgfpathlineto{\pgfqpoint{1.837144in}{0.553458in}}%
\pgfpathlineto{\pgfqpoint{1.855559in}{0.568554in}}%
\pgfpathlineto{\pgfqpoint{1.878189in}{0.586669in}}%
\pgfpathlineto{\pgfqpoint{1.881024in}{0.589182in}}%
\pgfpathlineto{\pgfqpoint{1.904495in}{0.609810in}}%
\pgfpathlineto{\pgfqpoint{1.919235in}{0.622493in}}%
\pgfpathlineto{\pgfqpoint{1.928180in}{0.630438in}}%
\pgfpathlineto{\pgfqpoint{1.951712in}{0.651066in}}%
\pgfpathlineto{\pgfqpoint{1.960280in}{0.658480in}}%
\pgfpathlineto{\pgfqpoint{1.976483in}{0.671694in}}%
\pgfpathlineto{\pgfqpoint{2.001325in}{0.691626in}}%
\pgfpathlineto{\pgfqpoint{2.002360in}{0.692321in}}%
\pgfpathlineto{\pgfqpoint{2.033159in}{0.712949in}}%
\pgfpathlineto{\pgfqpoint{2.042371in}{0.719066in}}%
\pgfpathlineto{\pgfqpoint{2.073057in}{0.733577in}}%
\pgfpathlineto{\pgfqpoint{2.083416in}{0.738470in}}%
\pgfpathlineto{\pgfqpoint{2.124461in}{0.748991in}}%
\pgfpathlineto{\pgfqpoint{2.165507in}{0.751310in}}%
\pgfpathlineto{\pgfqpoint{2.206552in}{0.747629in}}%
\pgfpathlineto{\pgfqpoint{2.247597in}{0.741007in}}%
\pgfpathlineto{\pgfqpoint{2.288643in}{0.734744in}}%
\pgfpathlineto{\pgfqpoint{2.304413in}{0.733577in}}%
\pgfpathlineto{\pgfqpoint{2.329688in}{0.731707in}}%
\pgfpathlineto{\pgfqpoint{2.370733in}{0.733515in}}%
\pgfpathlineto{\pgfqpoint{2.371105in}{0.733577in}}%
\pgfpathlineto{\pgfqpoint{2.411779in}{0.740553in}}%
\pgfpathlineto{\pgfqpoint{2.452824in}{0.751770in}}%
\pgfpathlineto{\pgfqpoint{2.460104in}{0.754205in}}%
\pgfpathlineto{\pgfqpoint{2.493869in}{0.765699in}}%
\pgfpathlineto{\pgfqpoint{2.495039in}{0.754205in}}%
\pgfpathlineto{\pgfqpoint{2.497114in}{0.733577in}}%
\pgfpathlineto{\pgfqpoint{2.499149in}{0.712949in}}%
\pgfpathlineto{\pgfqpoint{2.501145in}{0.692321in}}%
\pgfpathlineto{\pgfqpoint{2.503105in}{0.671694in}}%
\pgfpathlineto{\pgfqpoint{2.505031in}{0.651066in}}%
\pgfpathlineto{\pgfqpoint{2.506925in}{0.630438in}}%
\pgfpathlineto{\pgfqpoint{2.508788in}{0.609810in}}%
\pgfpathlineto{\pgfqpoint{2.510622in}{0.589182in}}%
\pgfpathlineto{\pgfqpoint{2.512428in}{0.568554in}}%
\pgfpathlineto{\pgfqpoint{2.514208in}{0.547927in}}%
\pgfpathlineto{\pgfqpoint{2.515963in}{0.527299in}}%
\pgfpathlineto{\pgfqpoint{2.517695in}{0.506671in}}%
\pgfpathlineto{\pgfqpoint{2.519404in}{0.486043in}}%
\pgfpathlineto{\pgfqpoint{2.521092in}{0.465415in}}%
\pgfpathlineto{\pgfqpoint{2.522759in}{0.444787in}}%
\pgfpathlineto{\pgfqpoint{2.524407in}{0.424160in}}%
\pgfpathlineto{\pgfqpoint{2.526036in}{0.403532in}}%
\pgfpathlineto{\pgfqpoint{2.527647in}{0.382904in}}%
\pgfpathlineto{\pgfqpoint{2.523713in}{0.382904in}}%
\pgfpathlineto{\pgfqpoint{2.522043in}{0.403532in}}%
\pgfpathlineto{\pgfqpoint{2.520353in}{0.424160in}}%
\pgfpathlineto{\pgfqpoint{2.518643in}{0.444787in}}%
\pgfpathlineto{\pgfqpoint{2.516911in}{0.465415in}}%
\pgfpathlineto{\pgfqpoint{2.515157in}{0.486043in}}%
\pgfpathlineto{\pgfqpoint{2.513378in}{0.506671in}}%
\pgfpathlineto{\pgfqpoint{2.511576in}{0.527299in}}%
\pgfpathlineto{\pgfqpoint{2.509747in}{0.547927in}}%
\pgfpathlineto{\pgfqpoint{2.507890in}{0.568554in}}%
\pgfpathlineto{\pgfqpoint{2.506005in}{0.589182in}}%
\pgfpathlineto{\pgfqpoint{2.504090in}{0.609810in}}%
\pgfpathlineto{\pgfqpoint{2.502143in}{0.630438in}}%
\pgfpathlineto{\pgfqpoint{2.500162in}{0.651066in}}%
\pgfpathlineto{\pgfqpoint{2.498145in}{0.671694in}}%
\pgfpathlineto{\pgfqpoint{2.496090in}{0.692321in}}%
\pgfpathlineto{\pgfqpoint{2.493996in}{0.712949in}}%
\pgfpathlineto{\pgfqpoint{2.493869in}{0.714199in}}%
\pgfpathlineto{\pgfqpoint{2.490074in}{0.712949in}}%
\pgfpathlineto{\pgfqpoint{2.452824in}{0.700885in}}%
\pgfpathlineto{\pgfqpoint{2.419324in}{0.692321in}}%
\pgfpathlineto{\pgfqpoint{2.411779in}{0.690440in}}%
\pgfpathlineto{\pgfqpoint{2.370733in}{0.684400in}}%
\pgfpathlineto{\pgfqpoint{2.329688in}{0.683447in}}%
\pgfpathlineto{\pgfqpoint{2.288643in}{0.687191in}}%
\pgfpathlineto{\pgfqpoint{2.257291in}{0.692321in}}%
\pgfpathlineto{\pgfqpoint{2.247597in}{0.693910in}}%
\pgfpathlineto{\pgfqpoint{2.206552in}{0.700876in}}%
\pgfpathlineto{\pgfqpoint{2.165507in}{0.704723in}}%
\pgfpathlineto{\pgfqpoint{2.124461in}{0.702428in}}%
\pgfpathlineto{\pgfqpoint{2.085332in}{0.692321in}}%
\pgfpathlineto{\pgfqpoint{2.083416in}{0.691830in}}%
\pgfpathlineto{\pgfqpoint{2.042371in}{0.672328in}}%
\pgfpathlineto{\pgfqpoint{2.041420in}{0.671694in}}%
\pgfpathlineto{\pgfqpoint{2.010610in}{0.651066in}}%
\pgfpathlineto{\pgfqpoint{2.001325in}{0.644798in}}%
\pgfpathlineto{\pgfqpoint{1.983547in}{0.630438in}}%
\pgfpathlineto{\pgfqpoint{1.960280in}{0.611349in}}%
\pgfpathlineto{\pgfqpoint{1.958518in}{0.609810in}}%
\pgfpathlineto{\pgfqpoint{1.935042in}{0.589182in}}%
\pgfpathlineto{\pgfqpoint{1.919235in}{0.575029in}}%
\pgfpathlineto{\pgfqpoint{1.911801in}{0.568554in}}%
\pgfpathlineto{\pgfqpoint{1.888425in}{0.547927in}}%
\pgfpathlineto{\pgfqpoint{1.878189in}{0.538752in}}%
\pgfpathlineto{\pgfqpoint{1.864100in}{0.527299in}}%
\pgfpathlineto{\pgfqpoint{1.839250in}{0.506671in}}%
\pgfpathlineto{\pgfqpoint{1.837144in}{0.504907in}}%
\pgfpathlineto{\pgfqpoint{1.811027in}{0.486043in}}%
\pgfpathlineto{\pgfqpoint{1.796099in}{0.475041in}}%
\pgfpathlineto{\pgfqpoint{1.780471in}{0.465415in}}%
\pgfpathlineto{\pgfqpoint{1.755053in}{0.449520in}}%
\pgfpathlineto{\pgfqpoint{1.745880in}{0.444787in}}%
\pgfpathlineto{\pgfqpoint{1.714008in}{0.428215in}}%
\pgfpathlineto{\pgfqpoint{1.704438in}{0.424160in}}%
\pgfpathlineto{\pgfqpoint{1.672963in}{0.410834in}}%
\pgfpathlineto{\pgfqpoint{1.650727in}{0.403532in}}%
\pgfpathlineto{\pgfqpoint{1.631917in}{0.397416in}}%
\pgfpathlineto{\pgfqpoint{1.590872in}{0.388594in}}%
\pgfpathlineto{\pgfqpoint{1.549827in}{0.385245in}}%
\pgfpathlineto{\pgfqpoint{1.508781in}{0.388035in}}%
\pgfpathlineto{\pgfqpoint{1.467736in}{0.396673in}}%
\pgfpathlineto{\pgfqpoint{1.446100in}{0.403532in}}%
\pgfpathlineto{\pgfqpoint{1.426691in}{0.409686in}}%
\pgfpathlineto{\pgfqpoint{1.386149in}{0.424160in}}%
\pgfpathlineto{\pgfqpoint{1.385645in}{0.424340in}}%
\pgfpathlineto{\pgfqpoint{1.344600in}{0.437154in}}%
\pgfpathlineto{\pgfqpoint{1.303555in}{0.444146in}}%
\pgfpathlineto{\pgfqpoint{1.262509in}{0.442162in}}%
\pgfpathlineto{\pgfqpoint{1.221464in}{0.429146in}}%
\pgfpathlineto{\pgfqpoint{1.213035in}{0.424160in}}%
\pgfpathlineto{\pgfqpoint{1.180419in}{0.404799in}}%
\pgfpathlineto{\pgfqpoint{1.178886in}{0.403532in}}%
\pgfpathlineto{\pgfqpoint{1.153956in}{0.382904in}}%
\pgfpathlineto{\pgfqpoint{1.139373in}{0.382904in}}%
\pgfpathlineto{\pgfqpoint{1.103772in}{0.382904in}}%
\pgfpathclose%
\pgfusepath{stroke,fill}%
\end{pgfscope}%
\begin{pgfscope}%
\pgfpathrectangle{\pgfqpoint{0.605784in}{0.382904in}}{\pgfqpoint{4.063488in}{2.042155in}}%
\pgfusepath{clip}%
\pgfsetbuttcap%
\pgfsetroundjoin%
\definecolor{currentfill}{rgb}{0.720391,0.870350,0.162603}%
\pgfsetfillcolor{currentfill}%
\pgfsetlinewidth{1.003750pt}%
\definecolor{currentstroke}{rgb}{0.720391,0.870350,0.162603}%
\pgfsetstrokecolor{currentstroke}%
\pgfsetdash{}{0pt}%
\pgfpathmoveto{\pgfqpoint{3.682647in}{2.301292in}}%
\pgfpathlineto{\pgfqpoint{3.678415in}{2.321920in}}%
\pgfpathlineto{\pgfqpoint{3.674273in}{2.342547in}}%
\pgfpathlineto{\pgfqpoint{3.670215in}{2.363175in}}%
\pgfpathlineto{\pgfqpoint{3.666237in}{2.383803in}}%
\pgfpathlineto{\pgfqpoint{3.662334in}{2.404431in}}%
\pgfpathlineto{\pgfqpoint{3.658502in}{2.425059in}}%
\pgfpathlineto{\pgfqpoint{3.667675in}{2.425059in}}%
\pgfpathlineto{\pgfqpoint{3.671676in}{2.404431in}}%
\pgfpathlineto{\pgfqpoint{3.675755in}{2.383803in}}%
\pgfpathlineto{\pgfqpoint{3.679916in}{2.363175in}}%
\pgfpathlineto{\pgfqpoint{3.684165in}{2.342547in}}%
\pgfpathlineto{\pgfqpoint{3.684184in}{2.342454in}}%
\pgfpathlineto{\pgfqpoint{3.725229in}{2.342454in}}%
\pgfpathlineto{\pgfqpoint{3.766275in}{2.342454in}}%
\pgfpathlineto{\pgfqpoint{3.807320in}{2.342454in}}%
\pgfpathlineto{\pgfqpoint{3.848365in}{2.342454in}}%
\pgfpathlineto{\pgfqpoint{3.889411in}{2.342454in}}%
\pgfpathlineto{\pgfqpoint{3.930456in}{2.342454in}}%
\pgfpathlineto{\pgfqpoint{3.971501in}{2.342454in}}%
\pgfpathlineto{\pgfqpoint{4.012547in}{2.342454in}}%
\pgfpathlineto{\pgfqpoint{4.053592in}{2.342454in}}%
\pgfpathlineto{\pgfqpoint{4.094637in}{2.342454in}}%
\pgfpathlineto{\pgfqpoint{4.135683in}{2.342454in}}%
\pgfpathlineto{\pgfqpoint{4.176728in}{2.342454in}}%
\pgfpathlineto{\pgfqpoint{4.217773in}{2.342454in}}%
\pgfpathlineto{\pgfqpoint{4.258819in}{2.342454in}}%
\pgfpathlineto{\pgfqpoint{4.299864in}{2.342454in}}%
\pgfpathlineto{\pgfqpoint{4.340909in}{2.342454in}}%
\pgfpathlineto{\pgfqpoint{4.381955in}{2.342454in}}%
\pgfpathlineto{\pgfqpoint{4.423000in}{2.342454in}}%
\pgfpathlineto{\pgfqpoint{4.464045in}{2.342454in}}%
\pgfpathlineto{\pgfqpoint{4.505091in}{2.342454in}}%
\pgfpathlineto{\pgfqpoint{4.546136in}{2.342454in}}%
\pgfpathlineto{\pgfqpoint{4.587181in}{2.342454in}}%
\pgfpathlineto{\pgfqpoint{4.628227in}{2.342454in}}%
\pgfpathlineto{\pgfqpoint{4.669272in}{2.342454in}}%
\pgfpathlineto{\pgfqpoint{4.669272in}{2.321920in}}%
\pgfpathlineto{\pgfqpoint{4.669272in}{2.301292in}}%
\pgfpathlineto{\pgfqpoint{4.669272in}{2.293870in}}%
\pgfpathlineto{\pgfqpoint{4.628227in}{2.293870in}}%
\pgfpathlineto{\pgfqpoint{4.587181in}{2.293870in}}%
\pgfpathlineto{\pgfqpoint{4.546136in}{2.293870in}}%
\pgfpathlineto{\pgfqpoint{4.505091in}{2.293870in}}%
\pgfpathlineto{\pgfqpoint{4.464045in}{2.293870in}}%
\pgfpathlineto{\pgfqpoint{4.423000in}{2.293870in}}%
\pgfpathlineto{\pgfqpoint{4.381955in}{2.293870in}}%
\pgfpathlineto{\pgfqpoint{4.340909in}{2.293870in}}%
\pgfpathlineto{\pgfqpoint{4.299864in}{2.293870in}}%
\pgfpathlineto{\pgfqpoint{4.258819in}{2.293870in}}%
\pgfpathlineto{\pgfqpoint{4.217773in}{2.293870in}}%
\pgfpathlineto{\pgfqpoint{4.176728in}{2.293870in}}%
\pgfpathlineto{\pgfqpoint{4.135683in}{2.293870in}}%
\pgfpathlineto{\pgfqpoint{4.094637in}{2.293870in}}%
\pgfpathlineto{\pgfqpoint{4.053592in}{2.293870in}}%
\pgfpathlineto{\pgfqpoint{4.012547in}{2.293870in}}%
\pgfpathlineto{\pgfqpoint{3.971501in}{2.293870in}}%
\pgfpathlineto{\pgfqpoint{3.930456in}{2.293870in}}%
\pgfpathlineto{\pgfqpoint{3.889411in}{2.293870in}}%
\pgfpathlineto{\pgfqpoint{3.848365in}{2.293870in}}%
\pgfpathlineto{\pgfqpoint{3.807320in}{2.293870in}}%
\pgfpathlineto{\pgfqpoint{3.766275in}{2.293870in}}%
\pgfpathlineto{\pgfqpoint{3.725229in}{2.293870in}}%
\pgfpathlineto{\pgfqpoint{3.684184in}{2.293870in}}%
\pgfpathclose%
\pgfusepath{stroke,fill}%
\end{pgfscope}%
\begin{pgfscope}%
\pgfpathrectangle{\pgfqpoint{0.605784in}{0.382904in}}{\pgfqpoint{4.063488in}{2.042155in}}%
\pgfusepath{clip}%
\pgfsetbuttcap%
\pgfsetroundjoin%
\definecolor{currentfill}{rgb}{0.906311,0.894855,0.098125}%
\pgfsetfillcolor{currentfill}%
\pgfsetlinewidth{1.003750pt}%
\definecolor{currentstroke}{rgb}{0.906311,0.894855,0.098125}%
\pgfsetstrokecolor{currentstroke}%
\pgfsetdash{}{0pt}%
\pgfpathmoveto{\pgfqpoint{0.683011in}{2.425059in}}%
\pgfpathlineto{\pgfqpoint{0.687875in}{2.425059in}}%
\pgfpathlineto{\pgfqpoint{0.728920in}{2.425059in}}%
\pgfpathlineto{\pgfqpoint{0.769965in}{2.425059in}}%
\pgfpathlineto{\pgfqpoint{0.811011in}{2.425059in}}%
\pgfpathlineto{\pgfqpoint{0.852056in}{2.425059in}}%
\pgfpathlineto{\pgfqpoint{0.864946in}{2.425059in}}%
\pgfpathlineto{\pgfqpoint{0.852056in}{2.415524in}}%
\pgfpathlineto{\pgfqpoint{0.825015in}{2.404431in}}%
\pgfpathlineto{\pgfqpoint{0.811011in}{2.398667in}}%
\pgfpathlineto{\pgfqpoint{0.769965in}{2.395646in}}%
\pgfpathlineto{\pgfqpoint{0.730386in}{2.404431in}}%
\pgfpathlineto{\pgfqpoint{0.728920in}{2.404751in}}%
\pgfpathlineto{\pgfqpoint{0.687875in}{2.422441in}}%
\pgfpathclose%
\pgfusepath{stroke,fill}%
\end{pgfscope}%
\begin{pgfscope}%
\pgfpathrectangle{\pgfqpoint{0.605784in}{0.382904in}}{\pgfqpoint{4.063488in}{2.042155in}}%
\pgfusepath{clip}%
\pgfsetbuttcap%
\pgfsetroundjoin%
\definecolor{currentfill}{rgb}{0.906311,0.894855,0.098125}%
\pgfsetfillcolor{currentfill}%
\pgfsetlinewidth{1.003750pt}%
\definecolor{currentstroke}{rgb}{0.906311,0.894855,0.098125}%
\pgfsetstrokecolor{currentstroke}%
\pgfsetdash{}{0pt}%
\pgfpathmoveto{\pgfqpoint{1.178886in}{0.403532in}}%
\pgfpathlineto{\pgfqpoint{1.180419in}{0.404799in}}%
\pgfpathlineto{\pgfqpoint{1.213035in}{0.424160in}}%
\pgfpathlineto{\pgfqpoint{1.221464in}{0.429146in}}%
\pgfpathlineto{\pgfqpoint{1.262509in}{0.442162in}}%
\pgfpathlineto{\pgfqpoint{1.303555in}{0.444146in}}%
\pgfpathlineto{\pgfqpoint{1.344600in}{0.437154in}}%
\pgfpathlineto{\pgfqpoint{1.385645in}{0.424340in}}%
\pgfpathlineto{\pgfqpoint{1.386149in}{0.424160in}}%
\pgfpathlineto{\pgfqpoint{1.426691in}{0.409686in}}%
\pgfpathlineto{\pgfqpoint{1.446100in}{0.403532in}}%
\pgfpathlineto{\pgfqpoint{1.467736in}{0.396673in}}%
\pgfpathlineto{\pgfqpoint{1.508781in}{0.388035in}}%
\pgfpathlineto{\pgfqpoint{1.549827in}{0.385245in}}%
\pgfpathlineto{\pgfqpoint{1.590872in}{0.388594in}}%
\pgfpathlineto{\pgfqpoint{1.631917in}{0.397416in}}%
\pgfpathlineto{\pgfqpoint{1.650727in}{0.403532in}}%
\pgfpathlineto{\pgfqpoint{1.672963in}{0.410834in}}%
\pgfpathlineto{\pgfqpoint{1.704438in}{0.424160in}}%
\pgfpathlineto{\pgfqpoint{1.714008in}{0.428215in}}%
\pgfpathlineto{\pgfqpoint{1.745880in}{0.444787in}}%
\pgfpathlineto{\pgfqpoint{1.755053in}{0.449520in}}%
\pgfpathlineto{\pgfqpoint{1.780471in}{0.465415in}}%
\pgfpathlineto{\pgfqpoint{1.796099in}{0.475041in}}%
\pgfpathlineto{\pgfqpoint{1.811027in}{0.486043in}}%
\pgfpathlineto{\pgfqpoint{1.837144in}{0.504907in}}%
\pgfpathlineto{\pgfqpoint{1.839250in}{0.506671in}}%
\pgfpathlineto{\pgfqpoint{1.864100in}{0.527299in}}%
\pgfpathlineto{\pgfqpoint{1.878189in}{0.538752in}}%
\pgfpathlineto{\pgfqpoint{1.888425in}{0.547927in}}%
\pgfpathlineto{\pgfqpoint{1.911801in}{0.568554in}}%
\pgfpathlineto{\pgfqpoint{1.919235in}{0.575029in}}%
\pgfpathlineto{\pgfqpoint{1.935042in}{0.589182in}}%
\pgfpathlineto{\pgfqpoint{1.958518in}{0.609810in}}%
\pgfpathlineto{\pgfqpoint{1.960280in}{0.611349in}}%
\pgfpathlineto{\pgfqpoint{1.983547in}{0.630438in}}%
\pgfpathlineto{\pgfqpoint{2.001325in}{0.644798in}}%
\pgfpathlineto{\pgfqpoint{2.010610in}{0.651066in}}%
\pgfpathlineto{\pgfqpoint{2.041420in}{0.671694in}}%
\pgfpathlineto{\pgfqpoint{2.042371in}{0.672328in}}%
\pgfpathlineto{\pgfqpoint{2.083416in}{0.691830in}}%
\pgfpathlineto{\pgfqpoint{2.085332in}{0.692321in}}%
\pgfpathlineto{\pgfqpoint{2.124461in}{0.702428in}}%
\pgfpathlineto{\pgfqpoint{2.165507in}{0.704723in}}%
\pgfpathlineto{\pgfqpoint{2.206552in}{0.700876in}}%
\pgfpathlineto{\pgfqpoint{2.247597in}{0.693910in}}%
\pgfpathlineto{\pgfqpoint{2.257291in}{0.692321in}}%
\pgfpathlineto{\pgfqpoint{2.288643in}{0.687191in}}%
\pgfpathlineto{\pgfqpoint{2.329688in}{0.683447in}}%
\pgfpathlineto{\pgfqpoint{2.370733in}{0.684400in}}%
\pgfpathlineto{\pgfqpoint{2.411779in}{0.690440in}}%
\pgfpathlineto{\pgfqpoint{2.419324in}{0.692321in}}%
\pgfpathlineto{\pgfqpoint{2.452824in}{0.700885in}}%
\pgfpathlineto{\pgfqpoint{2.490074in}{0.712949in}}%
\pgfpathlineto{\pgfqpoint{2.493869in}{0.714199in}}%
\pgfpathlineto{\pgfqpoint{2.493996in}{0.712949in}}%
\pgfpathlineto{\pgfqpoint{2.496090in}{0.692321in}}%
\pgfpathlineto{\pgfqpoint{2.498145in}{0.671694in}}%
\pgfpathlineto{\pgfqpoint{2.500162in}{0.651066in}}%
\pgfpathlineto{\pgfqpoint{2.502143in}{0.630438in}}%
\pgfpathlineto{\pgfqpoint{2.504090in}{0.609810in}}%
\pgfpathlineto{\pgfqpoint{2.506005in}{0.589182in}}%
\pgfpathlineto{\pgfqpoint{2.507890in}{0.568554in}}%
\pgfpathlineto{\pgfqpoint{2.509747in}{0.547927in}}%
\pgfpathlineto{\pgfqpoint{2.511576in}{0.527299in}}%
\pgfpathlineto{\pgfqpoint{2.513378in}{0.506671in}}%
\pgfpathlineto{\pgfqpoint{2.515157in}{0.486043in}}%
\pgfpathlineto{\pgfqpoint{2.516911in}{0.465415in}}%
\pgfpathlineto{\pgfqpoint{2.518643in}{0.444787in}}%
\pgfpathlineto{\pgfqpoint{2.520353in}{0.424160in}}%
\pgfpathlineto{\pgfqpoint{2.522043in}{0.403532in}}%
\pgfpathlineto{\pgfqpoint{2.523713in}{0.382904in}}%
\pgfpathlineto{\pgfqpoint{2.519780in}{0.382904in}}%
\pgfpathlineto{\pgfqpoint{2.518050in}{0.403532in}}%
\pgfpathlineto{\pgfqpoint{2.516300in}{0.424160in}}%
\pgfpathlineto{\pgfqpoint{2.514527in}{0.444787in}}%
\pgfpathlineto{\pgfqpoint{2.512730in}{0.465415in}}%
\pgfpathlineto{\pgfqpoint{2.510909in}{0.486043in}}%
\pgfpathlineto{\pgfqpoint{2.509062in}{0.506671in}}%
\pgfpathlineto{\pgfqpoint{2.507188in}{0.527299in}}%
\pgfpathlineto{\pgfqpoint{2.505285in}{0.547927in}}%
\pgfpathlineto{\pgfqpoint{2.503353in}{0.568554in}}%
\pgfpathlineto{\pgfqpoint{2.501389in}{0.589182in}}%
\pgfpathlineto{\pgfqpoint{2.499392in}{0.609810in}}%
\pgfpathlineto{\pgfqpoint{2.497361in}{0.630438in}}%
\pgfpathlineto{\pgfqpoint{2.495292in}{0.651066in}}%
\pgfpathlineto{\pgfqpoint{2.493869in}{0.665074in}}%
\pgfpathlineto{\pgfqpoint{2.452824in}{0.652174in}}%
\pgfpathlineto{\pgfqpoint{2.448212in}{0.651066in}}%
\pgfpathlineto{\pgfqpoint{2.411779in}{0.642517in}}%
\pgfpathlineto{\pgfqpoint{2.370733in}{0.637259in}}%
\pgfpathlineto{\pgfqpoint{2.329688in}{0.637073in}}%
\pgfpathlineto{\pgfqpoint{2.288643in}{0.641465in}}%
\pgfpathlineto{\pgfqpoint{2.247597in}{0.648632in}}%
\pgfpathlineto{\pgfqpoint{2.233727in}{0.651066in}}%
\pgfpathlineto{\pgfqpoint{2.206552in}{0.655858in}}%
\pgfpathlineto{\pgfqpoint{2.165507in}{0.659849in}}%
\pgfpathlineto{\pgfqpoint{2.124461in}{0.657579in}}%
\pgfpathlineto{\pgfqpoint{2.099318in}{0.651066in}}%
\pgfpathlineto{\pgfqpoint{2.083416in}{0.646976in}}%
\pgfpathlineto{\pgfqpoint{2.048763in}{0.630438in}}%
\pgfpathlineto{\pgfqpoint{2.042371in}{0.627384in}}%
\pgfpathlineto{\pgfqpoint{2.016228in}{0.609810in}}%
\pgfpathlineto{\pgfqpoint{2.001325in}{0.599705in}}%
\pgfpathlineto{\pgfqpoint{1.988378in}{0.589182in}}%
\pgfpathlineto{\pgfqpoint{1.963318in}{0.568554in}}%
\pgfpathlineto{\pgfqpoint{1.960280in}{0.566041in}}%
\pgfpathlineto{\pgfqpoint{1.939799in}{0.547927in}}%
\pgfpathlineto{\pgfqpoint{1.919235in}{0.529375in}}%
\pgfpathlineto{\pgfqpoint{1.916877in}{0.527299in}}%
\pgfpathlineto{\pgfqpoint{1.893636in}{0.506671in}}%
\pgfpathlineto{\pgfqpoint{1.878189in}{0.492686in}}%
\pgfpathlineto{\pgfqpoint{1.870133in}{0.486043in}}%
\pgfpathlineto{\pgfqpoint{1.845468in}{0.465415in}}%
\pgfpathlineto{\pgfqpoint{1.837144in}{0.458353in}}%
\pgfpathlineto{\pgfqpoint{1.818668in}{0.444787in}}%
\pgfpathlineto{\pgfqpoint{1.796099in}{0.427890in}}%
\pgfpathlineto{\pgfqpoint{1.790146in}{0.424160in}}%
\pgfpathlineto{\pgfqpoint{1.757639in}{0.403532in}}%
\pgfpathlineto{\pgfqpoint{1.755053in}{0.401875in}}%
\pgfpathlineto{\pgfqpoint{1.719020in}{0.382904in}}%
\pgfpathlineto{\pgfqpoint{1.714008in}{0.382904in}}%
\pgfpathlineto{\pgfqpoint{1.672963in}{0.382904in}}%
\pgfpathlineto{\pgfqpoint{1.631917in}{0.382904in}}%
\pgfpathlineto{\pgfqpoint{1.590872in}{0.382904in}}%
\pgfpathlineto{\pgfqpoint{1.549827in}{0.382904in}}%
\pgfpathlineto{\pgfqpoint{1.508781in}{0.382904in}}%
\pgfpathlineto{\pgfqpoint{1.467736in}{0.382904in}}%
\pgfpathlineto{\pgfqpoint{1.426691in}{0.382904in}}%
\pgfpathlineto{\pgfqpoint{1.385645in}{0.382904in}}%
\pgfpathlineto{\pgfqpoint{1.372013in}{0.382904in}}%
\pgfpathlineto{\pgfqpoint{1.344600in}{0.391466in}}%
\pgfpathlineto{\pgfqpoint{1.303555in}{0.398421in}}%
\pgfpathlineto{\pgfqpoint{1.262509in}{0.396354in}}%
\pgfpathlineto{\pgfqpoint{1.221464in}{0.383242in}}%
\pgfpathlineto{\pgfqpoint{1.220893in}{0.382904in}}%
\pgfpathlineto{\pgfqpoint{1.180419in}{0.382904in}}%
\pgfpathlineto{\pgfqpoint{1.153956in}{0.382904in}}%
\pgfpathclose%
\pgfusepath{stroke,fill}%
\end{pgfscope}%
\begin{pgfscope}%
\pgfpathrectangle{\pgfqpoint{0.605784in}{0.382904in}}{\pgfqpoint{4.063488in}{2.042155in}}%
\pgfusepath{clip}%
\pgfsetbuttcap%
\pgfsetroundjoin%
\definecolor{currentfill}{rgb}{0.906311,0.894855,0.098125}%
\pgfsetfillcolor{currentfill}%
\pgfsetlinewidth{1.003750pt}%
\definecolor{currentstroke}{rgb}{0.906311,0.894855,0.098125}%
\pgfsetstrokecolor{currentstroke}%
\pgfsetdash{}{0pt}%
\pgfpathmoveto{\pgfqpoint{3.684165in}{2.342547in}}%
\pgfpathlineto{\pgfqpoint{3.679916in}{2.363175in}}%
\pgfpathlineto{\pgfqpoint{3.675755in}{2.383803in}}%
\pgfpathlineto{\pgfqpoint{3.671676in}{2.404431in}}%
\pgfpathlineto{\pgfqpoint{3.667675in}{2.425059in}}%
\pgfpathlineto{\pgfqpoint{3.676847in}{2.425059in}}%
\pgfpathlineto{\pgfqpoint{3.681018in}{2.404431in}}%
\pgfpathlineto{\pgfqpoint{3.684184in}{2.389012in}}%
\pgfpathlineto{\pgfqpoint{3.725229in}{2.389012in}}%
\pgfpathlineto{\pgfqpoint{3.766275in}{2.389012in}}%
\pgfpathlineto{\pgfqpoint{3.807320in}{2.389012in}}%
\pgfpathlineto{\pgfqpoint{3.848365in}{2.389012in}}%
\pgfpathlineto{\pgfqpoint{3.889411in}{2.389012in}}%
\pgfpathlineto{\pgfqpoint{3.930456in}{2.389012in}}%
\pgfpathlineto{\pgfqpoint{3.971501in}{2.389012in}}%
\pgfpathlineto{\pgfqpoint{4.012547in}{2.389012in}}%
\pgfpathlineto{\pgfqpoint{4.053592in}{2.389012in}}%
\pgfpathlineto{\pgfqpoint{4.094637in}{2.389012in}}%
\pgfpathlineto{\pgfqpoint{4.135683in}{2.389012in}}%
\pgfpathlineto{\pgfqpoint{4.176728in}{2.389012in}}%
\pgfpathlineto{\pgfqpoint{4.217773in}{2.389012in}}%
\pgfpathlineto{\pgfqpoint{4.258819in}{2.389012in}}%
\pgfpathlineto{\pgfqpoint{4.299864in}{2.389012in}}%
\pgfpathlineto{\pgfqpoint{4.340909in}{2.389012in}}%
\pgfpathlineto{\pgfqpoint{4.381955in}{2.389012in}}%
\pgfpathlineto{\pgfqpoint{4.423000in}{2.389012in}}%
\pgfpathlineto{\pgfqpoint{4.464045in}{2.389012in}}%
\pgfpathlineto{\pgfqpoint{4.505091in}{2.389012in}}%
\pgfpathlineto{\pgfqpoint{4.546136in}{2.389012in}}%
\pgfpathlineto{\pgfqpoint{4.587181in}{2.389012in}}%
\pgfpathlineto{\pgfqpoint{4.628227in}{2.389012in}}%
\pgfpathlineto{\pgfqpoint{4.669272in}{2.389012in}}%
\pgfpathlineto{\pgfqpoint{4.669272in}{2.383803in}}%
\pgfpathlineto{\pgfqpoint{4.669272in}{2.363175in}}%
\pgfpathlineto{\pgfqpoint{4.669272in}{2.342547in}}%
\pgfpathlineto{\pgfqpoint{4.669272in}{2.342454in}}%
\pgfpathlineto{\pgfqpoint{4.628227in}{2.342454in}}%
\pgfpathlineto{\pgfqpoint{4.587181in}{2.342454in}}%
\pgfpathlineto{\pgfqpoint{4.546136in}{2.342454in}}%
\pgfpathlineto{\pgfqpoint{4.505091in}{2.342454in}}%
\pgfpathlineto{\pgfqpoint{4.464045in}{2.342454in}}%
\pgfpathlineto{\pgfqpoint{4.423000in}{2.342454in}}%
\pgfpathlineto{\pgfqpoint{4.381955in}{2.342454in}}%
\pgfpathlineto{\pgfqpoint{4.340909in}{2.342454in}}%
\pgfpathlineto{\pgfqpoint{4.299864in}{2.342454in}}%
\pgfpathlineto{\pgfqpoint{4.258819in}{2.342454in}}%
\pgfpathlineto{\pgfqpoint{4.217773in}{2.342454in}}%
\pgfpathlineto{\pgfqpoint{4.176728in}{2.342454in}}%
\pgfpathlineto{\pgfqpoint{4.135683in}{2.342454in}}%
\pgfpathlineto{\pgfqpoint{4.094637in}{2.342454in}}%
\pgfpathlineto{\pgfqpoint{4.053592in}{2.342454in}}%
\pgfpathlineto{\pgfqpoint{4.012547in}{2.342454in}}%
\pgfpathlineto{\pgfqpoint{3.971501in}{2.342454in}}%
\pgfpathlineto{\pgfqpoint{3.930456in}{2.342454in}}%
\pgfpathlineto{\pgfqpoint{3.889411in}{2.342454in}}%
\pgfpathlineto{\pgfqpoint{3.848365in}{2.342454in}}%
\pgfpathlineto{\pgfqpoint{3.807320in}{2.342454in}}%
\pgfpathlineto{\pgfqpoint{3.766275in}{2.342454in}}%
\pgfpathlineto{\pgfqpoint{3.725229in}{2.342454in}}%
\pgfpathlineto{\pgfqpoint{3.684184in}{2.342454in}}%
\pgfpathclose%
\pgfusepath{stroke,fill}%
\end{pgfscope}%
\begin{pgfscope}%
\pgfpathrectangle{\pgfqpoint{0.605784in}{0.382904in}}{\pgfqpoint{4.063488in}{2.042155in}}%
\pgfusepath{clip}%
\pgfsetbuttcap%
\pgfsetroundjoin%
\definecolor{currentfill}{rgb}{0.993248,0.906157,0.143936}%
\pgfsetfillcolor{currentfill}%
\pgfsetlinewidth{1.003750pt}%
\definecolor{currentstroke}{rgb}{0.993248,0.906157,0.143936}%
\pgfsetstrokecolor{currentstroke}%
\pgfsetdash{}{0pt}%
\pgfpathmoveto{\pgfqpoint{1.221464in}{0.383242in}}%
\pgfpathlineto{\pgfqpoint{1.262509in}{0.396354in}}%
\pgfpathlineto{\pgfqpoint{1.303555in}{0.398421in}}%
\pgfpathlineto{\pgfqpoint{1.344600in}{0.391466in}}%
\pgfpathlineto{\pgfqpoint{1.372013in}{0.382904in}}%
\pgfpathlineto{\pgfqpoint{1.344600in}{0.382904in}}%
\pgfpathlineto{\pgfqpoint{1.303555in}{0.382904in}}%
\pgfpathlineto{\pgfqpoint{1.262509in}{0.382904in}}%
\pgfpathlineto{\pgfqpoint{1.221464in}{0.382904in}}%
\pgfpathlineto{\pgfqpoint{1.220893in}{0.382904in}}%
\pgfpathclose%
\pgfusepath{stroke,fill}%
\end{pgfscope}%
\begin{pgfscope}%
\pgfpathrectangle{\pgfqpoint{0.605784in}{0.382904in}}{\pgfqpoint{4.063488in}{2.042155in}}%
\pgfusepath{clip}%
\pgfsetbuttcap%
\pgfsetroundjoin%
\definecolor{currentfill}{rgb}{0.993248,0.906157,0.143936}%
\pgfsetfillcolor{currentfill}%
\pgfsetlinewidth{1.003750pt}%
\definecolor{currentstroke}{rgb}{0.993248,0.906157,0.143936}%
\pgfsetstrokecolor{currentstroke}%
\pgfsetdash{}{0pt}%
\pgfpathmoveto{\pgfqpoint{1.755053in}{0.401875in}}%
\pgfpathlineto{\pgfqpoint{1.757639in}{0.403532in}}%
\pgfpathlineto{\pgfqpoint{1.790146in}{0.424160in}}%
\pgfpathlineto{\pgfqpoint{1.796099in}{0.427890in}}%
\pgfpathlineto{\pgfqpoint{1.818668in}{0.444787in}}%
\pgfpathlineto{\pgfqpoint{1.837144in}{0.458353in}}%
\pgfpathlineto{\pgfqpoint{1.845468in}{0.465415in}}%
\pgfpathlineto{\pgfqpoint{1.870133in}{0.486043in}}%
\pgfpathlineto{\pgfqpoint{1.878189in}{0.492686in}}%
\pgfpathlineto{\pgfqpoint{1.893636in}{0.506671in}}%
\pgfpathlineto{\pgfqpoint{1.916877in}{0.527299in}}%
\pgfpathlineto{\pgfqpoint{1.919235in}{0.529375in}}%
\pgfpathlineto{\pgfqpoint{1.939799in}{0.547927in}}%
\pgfpathlineto{\pgfqpoint{1.960280in}{0.566041in}}%
\pgfpathlineto{\pgfqpoint{1.963318in}{0.568554in}}%
\pgfpathlineto{\pgfqpoint{1.988378in}{0.589182in}}%
\pgfpathlineto{\pgfqpoint{2.001325in}{0.599705in}}%
\pgfpathlineto{\pgfqpoint{2.016228in}{0.609810in}}%
\pgfpathlineto{\pgfqpoint{2.042371in}{0.627384in}}%
\pgfpathlineto{\pgfqpoint{2.048763in}{0.630438in}}%
\pgfpathlineto{\pgfqpoint{2.083416in}{0.646976in}}%
\pgfpathlineto{\pgfqpoint{2.099318in}{0.651066in}}%
\pgfpathlineto{\pgfqpoint{2.124461in}{0.657579in}}%
\pgfpathlineto{\pgfqpoint{2.165507in}{0.659849in}}%
\pgfpathlineto{\pgfqpoint{2.206552in}{0.655858in}}%
\pgfpathlineto{\pgfqpoint{2.233727in}{0.651066in}}%
\pgfpathlineto{\pgfqpoint{2.247597in}{0.648632in}}%
\pgfpathlineto{\pgfqpoint{2.288643in}{0.641465in}}%
\pgfpathlineto{\pgfqpoint{2.329688in}{0.637073in}}%
\pgfpathlineto{\pgfqpoint{2.370733in}{0.637259in}}%
\pgfpathlineto{\pgfqpoint{2.411779in}{0.642517in}}%
\pgfpathlineto{\pgfqpoint{2.448212in}{0.651066in}}%
\pgfpathlineto{\pgfqpoint{2.452824in}{0.652174in}}%
\pgfpathlineto{\pgfqpoint{2.493869in}{0.665074in}}%
\pgfpathlineto{\pgfqpoint{2.495292in}{0.651066in}}%
\pgfpathlineto{\pgfqpoint{2.497361in}{0.630438in}}%
\pgfpathlineto{\pgfqpoint{2.499392in}{0.609810in}}%
\pgfpathlineto{\pgfqpoint{2.501389in}{0.589182in}}%
\pgfpathlineto{\pgfqpoint{2.503353in}{0.568554in}}%
\pgfpathlineto{\pgfqpoint{2.505285in}{0.547927in}}%
\pgfpathlineto{\pgfqpoint{2.507188in}{0.527299in}}%
\pgfpathlineto{\pgfqpoint{2.509062in}{0.506671in}}%
\pgfpathlineto{\pgfqpoint{2.510909in}{0.486043in}}%
\pgfpathlineto{\pgfqpoint{2.512730in}{0.465415in}}%
\pgfpathlineto{\pgfqpoint{2.514527in}{0.444787in}}%
\pgfpathlineto{\pgfqpoint{2.516300in}{0.424160in}}%
\pgfpathlineto{\pgfqpoint{2.518050in}{0.403532in}}%
\pgfpathlineto{\pgfqpoint{2.519780in}{0.382904in}}%
\pgfpathlineto{\pgfqpoint{2.493869in}{0.382904in}}%
\pgfpathlineto{\pgfqpoint{2.452824in}{0.382904in}}%
\pgfpathlineto{\pgfqpoint{2.411779in}{0.382904in}}%
\pgfpathlineto{\pgfqpoint{2.370733in}{0.382904in}}%
\pgfpathlineto{\pgfqpoint{2.329688in}{0.382904in}}%
\pgfpathlineto{\pgfqpoint{2.288643in}{0.382904in}}%
\pgfpathlineto{\pgfqpoint{2.247597in}{0.382904in}}%
\pgfpathlineto{\pgfqpoint{2.206552in}{0.382904in}}%
\pgfpathlineto{\pgfqpoint{2.165507in}{0.382904in}}%
\pgfpathlineto{\pgfqpoint{2.124461in}{0.382904in}}%
\pgfpathlineto{\pgfqpoint{2.083416in}{0.382904in}}%
\pgfpathlineto{\pgfqpoint{2.042371in}{0.382904in}}%
\pgfpathlineto{\pgfqpoint{2.001325in}{0.382904in}}%
\pgfpathlineto{\pgfqpoint{1.960280in}{0.382904in}}%
\pgfpathlineto{\pgfqpoint{1.919235in}{0.382904in}}%
\pgfpathlineto{\pgfqpoint{1.878189in}{0.382904in}}%
\pgfpathlineto{\pgfqpoint{1.837144in}{0.382904in}}%
\pgfpathlineto{\pgfqpoint{1.796099in}{0.382904in}}%
\pgfpathlineto{\pgfqpoint{1.755053in}{0.382904in}}%
\pgfpathlineto{\pgfqpoint{1.719020in}{0.382904in}}%
\pgfpathclose%
\pgfusepath{stroke,fill}%
\end{pgfscope}%
\begin{pgfscope}%
\pgfpathrectangle{\pgfqpoint{0.605784in}{0.382904in}}{\pgfqpoint{4.063488in}{2.042155in}}%
\pgfusepath{clip}%
\pgfsetbuttcap%
\pgfsetroundjoin%
\definecolor{currentfill}{rgb}{0.993248,0.906157,0.143936}%
\pgfsetfillcolor{currentfill}%
\pgfsetlinewidth{1.003750pt}%
\definecolor{currentstroke}{rgb}{0.993248,0.906157,0.143936}%
\pgfsetstrokecolor{currentstroke}%
\pgfsetdash{}{0pt}%
\pgfpathmoveto{\pgfqpoint{3.681018in}{2.404431in}}%
\pgfpathlineto{\pgfqpoint{3.676847in}{2.425059in}}%
\pgfpathlineto{\pgfqpoint{3.684184in}{2.425059in}}%
\pgfpathlineto{\pgfqpoint{3.725229in}{2.425059in}}%
\pgfpathlineto{\pgfqpoint{3.766275in}{2.425059in}}%
\pgfpathlineto{\pgfqpoint{3.807320in}{2.425059in}}%
\pgfpathlineto{\pgfqpoint{3.848365in}{2.425059in}}%
\pgfpathlineto{\pgfqpoint{3.889411in}{2.425059in}}%
\pgfpathlineto{\pgfqpoint{3.930456in}{2.425059in}}%
\pgfpathlineto{\pgfqpoint{3.971501in}{2.425059in}}%
\pgfpathlineto{\pgfqpoint{4.012547in}{2.425059in}}%
\pgfpathlineto{\pgfqpoint{4.053592in}{2.425059in}}%
\pgfpathlineto{\pgfqpoint{4.094637in}{2.425059in}}%
\pgfpathlineto{\pgfqpoint{4.135683in}{2.425059in}}%
\pgfpathlineto{\pgfqpoint{4.176728in}{2.425059in}}%
\pgfpathlineto{\pgfqpoint{4.217773in}{2.425059in}}%
\pgfpathlineto{\pgfqpoint{4.258819in}{2.425059in}}%
\pgfpathlineto{\pgfqpoint{4.299864in}{2.425059in}}%
\pgfpathlineto{\pgfqpoint{4.340909in}{2.425059in}}%
\pgfpathlineto{\pgfqpoint{4.381955in}{2.425059in}}%
\pgfpathlineto{\pgfqpoint{4.423000in}{2.425059in}}%
\pgfpathlineto{\pgfqpoint{4.464045in}{2.425059in}}%
\pgfpathlineto{\pgfqpoint{4.505091in}{2.425059in}}%
\pgfpathlineto{\pgfqpoint{4.546136in}{2.425059in}}%
\pgfpathlineto{\pgfqpoint{4.587181in}{2.425059in}}%
\pgfpathlineto{\pgfqpoint{4.628227in}{2.425059in}}%
\pgfpathlineto{\pgfqpoint{4.669272in}{2.425059in}}%
\pgfpathlineto{\pgfqpoint{4.669272in}{2.404431in}}%
\pgfpathlineto{\pgfqpoint{4.669272in}{2.389012in}}%
\pgfpathlineto{\pgfqpoint{4.628227in}{2.389012in}}%
\pgfpathlineto{\pgfqpoint{4.587181in}{2.389012in}}%
\pgfpathlineto{\pgfqpoint{4.546136in}{2.389012in}}%
\pgfpathlineto{\pgfqpoint{4.505091in}{2.389012in}}%
\pgfpathlineto{\pgfqpoint{4.464045in}{2.389012in}}%
\pgfpathlineto{\pgfqpoint{4.423000in}{2.389012in}}%
\pgfpathlineto{\pgfqpoint{4.381955in}{2.389012in}}%
\pgfpathlineto{\pgfqpoint{4.340909in}{2.389012in}}%
\pgfpathlineto{\pgfqpoint{4.299864in}{2.389012in}}%
\pgfpathlineto{\pgfqpoint{4.258819in}{2.389012in}}%
\pgfpathlineto{\pgfqpoint{4.217773in}{2.389012in}}%
\pgfpathlineto{\pgfqpoint{4.176728in}{2.389012in}}%
\pgfpathlineto{\pgfqpoint{4.135683in}{2.389012in}}%
\pgfpathlineto{\pgfqpoint{4.094637in}{2.389012in}}%
\pgfpathlineto{\pgfqpoint{4.053592in}{2.389012in}}%
\pgfpathlineto{\pgfqpoint{4.012547in}{2.389012in}}%
\pgfpathlineto{\pgfqpoint{3.971501in}{2.389012in}}%
\pgfpathlineto{\pgfqpoint{3.930456in}{2.389012in}}%
\pgfpathlineto{\pgfqpoint{3.889411in}{2.389012in}}%
\pgfpathlineto{\pgfqpoint{3.848365in}{2.389012in}}%
\pgfpathlineto{\pgfqpoint{3.807320in}{2.389012in}}%
\pgfpathlineto{\pgfqpoint{3.766275in}{2.389012in}}%
\pgfpathlineto{\pgfqpoint{3.725229in}{2.389012in}}%
\pgfpathlineto{\pgfqpoint{3.684184in}{2.389012in}}%
\pgfpathclose%
\pgfusepath{stroke,fill}%
\end{pgfscope}%
\begin{pgfscope}%
\pgfsetbuttcap%
\pgfsetroundjoin%
\definecolor{currentfill}{rgb}{0.000000,0.000000,0.000000}%
\pgfsetfillcolor{currentfill}%
\pgfsetlinewidth{0.803000pt}%
\definecolor{currentstroke}{rgb}{0.000000,0.000000,0.000000}%
\pgfsetstrokecolor{currentstroke}%
\pgfsetdash{}{0pt}%
\pgfsys@defobject{currentmarker}{\pgfqpoint{0.000000in}{-0.048611in}}{\pgfqpoint{0.000000in}{0.000000in}}{%
\pgfpathmoveto{\pgfqpoint{0.000000in}{0.000000in}}%
\pgfpathlineto{\pgfqpoint{0.000000in}{-0.048611in}}%
\pgfusepath{stroke,fill}%
}%
\begin{pgfscope}%
\pgfsys@transformshift{0.605784in}{0.382904in}%
\pgfsys@useobject{currentmarker}{}%
\end{pgfscope}%
\end{pgfscope}%
\begin{pgfscope}%
\definecolor{textcolor}{rgb}{0.000000,0.000000,0.000000}%
\pgfsetstrokecolor{textcolor}%
\pgfsetfillcolor{textcolor}%
\pgftext[x=0.605784in,y=0.285682in,,top]{\color{textcolor}\rmfamily\fontsize{10.000000}{12.000000}\selectfont \(\displaystyle {0}\)}%
\end{pgfscope}%
\begin{pgfscope}%
\pgfsetbuttcap%
\pgfsetroundjoin%
\definecolor{currentfill}{rgb}{0.000000,0.000000,0.000000}%
\pgfsetfillcolor{currentfill}%
\pgfsetlinewidth{0.803000pt}%
\definecolor{currentstroke}{rgb}{0.000000,0.000000,0.000000}%
\pgfsetstrokecolor{currentstroke}%
\pgfsetdash{}{0pt}%
\pgfsys@defobject{currentmarker}{\pgfqpoint{0.000000in}{-0.048611in}}{\pgfqpoint{0.000000in}{0.000000in}}{%
\pgfpathmoveto{\pgfqpoint{0.000000in}{0.000000in}}%
\pgfpathlineto{\pgfqpoint{0.000000in}{-0.048611in}}%
\pgfusepath{stroke,fill}%
}%
\begin{pgfscope}%
\pgfsys@transformshift{1.283032in}{0.382904in}%
\pgfsys@useobject{currentmarker}{}%
\end{pgfscope}%
\end{pgfscope}%
\begin{pgfscope}%
\definecolor{textcolor}{rgb}{0.000000,0.000000,0.000000}%
\pgfsetstrokecolor{textcolor}%
\pgfsetfillcolor{textcolor}%
\pgftext[x=1.283032in,y=0.285682in,,top]{\color{textcolor}\rmfamily\fontsize{10.000000}{12.000000}\selectfont \(\displaystyle {50}\)}%
\end{pgfscope}%
\begin{pgfscope}%
\pgfsetbuttcap%
\pgfsetroundjoin%
\definecolor{currentfill}{rgb}{0.000000,0.000000,0.000000}%
\pgfsetfillcolor{currentfill}%
\pgfsetlinewidth{0.803000pt}%
\definecolor{currentstroke}{rgb}{0.000000,0.000000,0.000000}%
\pgfsetstrokecolor{currentstroke}%
\pgfsetdash{}{0pt}%
\pgfsys@defobject{currentmarker}{\pgfqpoint{0.000000in}{-0.048611in}}{\pgfqpoint{0.000000in}{0.000000in}}{%
\pgfpathmoveto{\pgfqpoint{0.000000in}{0.000000in}}%
\pgfpathlineto{\pgfqpoint{0.000000in}{-0.048611in}}%
\pgfusepath{stroke,fill}%
}%
\begin{pgfscope}%
\pgfsys@transformshift{1.960280in}{0.382904in}%
\pgfsys@useobject{currentmarker}{}%
\end{pgfscope}%
\end{pgfscope}%
\begin{pgfscope}%
\definecolor{textcolor}{rgb}{0.000000,0.000000,0.000000}%
\pgfsetstrokecolor{textcolor}%
\pgfsetfillcolor{textcolor}%
\pgftext[x=1.960280in,y=0.285682in,,top]{\color{textcolor}\rmfamily\fontsize{10.000000}{12.000000}\selectfont \(\displaystyle {100}\)}%
\end{pgfscope}%
\begin{pgfscope}%
\pgfsetbuttcap%
\pgfsetroundjoin%
\definecolor{currentfill}{rgb}{0.000000,0.000000,0.000000}%
\pgfsetfillcolor{currentfill}%
\pgfsetlinewidth{0.803000pt}%
\definecolor{currentstroke}{rgb}{0.000000,0.000000,0.000000}%
\pgfsetstrokecolor{currentstroke}%
\pgfsetdash{}{0pt}%
\pgfsys@defobject{currentmarker}{\pgfqpoint{0.000000in}{-0.048611in}}{\pgfqpoint{0.000000in}{0.000000in}}{%
\pgfpathmoveto{\pgfqpoint{0.000000in}{0.000000in}}%
\pgfpathlineto{\pgfqpoint{0.000000in}{-0.048611in}}%
\pgfusepath{stroke,fill}%
}%
\begin{pgfscope}%
\pgfsys@transformshift{2.637528in}{0.382904in}%
\pgfsys@useobject{currentmarker}{}%
\end{pgfscope}%
\end{pgfscope}%
\begin{pgfscope}%
\definecolor{textcolor}{rgb}{0.000000,0.000000,0.000000}%
\pgfsetstrokecolor{textcolor}%
\pgfsetfillcolor{textcolor}%
\pgftext[x=2.637528in,y=0.285682in,,top]{\color{textcolor}\rmfamily\fontsize{10.000000}{12.000000}\selectfont \(\displaystyle {150}\)}%
\end{pgfscope}%
\begin{pgfscope}%
\pgfsetbuttcap%
\pgfsetroundjoin%
\definecolor{currentfill}{rgb}{0.000000,0.000000,0.000000}%
\pgfsetfillcolor{currentfill}%
\pgfsetlinewidth{0.803000pt}%
\definecolor{currentstroke}{rgb}{0.000000,0.000000,0.000000}%
\pgfsetstrokecolor{currentstroke}%
\pgfsetdash{}{0pt}%
\pgfsys@defobject{currentmarker}{\pgfqpoint{0.000000in}{-0.048611in}}{\pgfqpoint{0.000000in}{0.000000in}}{%
\pgfpathmoveto{\pgfqpoint{0.000000in}{0.000000in}}%
\pgfpathlineto{\pgfqpoint{0.000000in}{-0.048611in}}%
\pgfusepath{stroke,fill}%
}%
\begin{pgfscope}%
\pgfsys@transformshift{3.314776in}{0.382904in}%
\pgfsys@useobject{currentmarker}{}%
\end{pgfscope}%
\end{pgfscope}%
\begin{pgfscope}%
\definecolor{textcolor}{rgb}{0.000000,0.000000,0.000000}%
\pgfsetstrokecolor{textcolor}%
\pgfsetfillcolor{textcolor}%
\pgftext[x=3.314776in,y=0.285682in,,top]{\color{textcolor}\rmfamily\fontsize{10.000000}{12.000000}\selectfont \(\displaystyle {200}\)}%
\end{pgfscope}%
\begin{pgfscope}%
\pgfsetbuttcap%
\pgfsetroundjoin%
\definecolor{currentfill}{rgb}{0.000000,0.000000,0.000000}%
\pgfsetfillcolor{currentfill}%
\pgfsetlinewidth{0.803000pt}%
\definecolor{currentstroke}{rgb}{0.000000,0.000000,0.000000}%
\pgfsetstrokecolor{currentstroke}%
\pgfsetdash{}{0pt}%
\pgfsys@defobject{currentmarker}{\pgfqpoint{0.000000in}{-0.048611in}}{\pgfqpoint{0.000000in}{0.000000in}}{%
\pgfpathmoveto{\pgfqpoint{0.000000in}{0.000000in}}%
\pgfpathlineto{\pgfqpoint{0.000000in}{-0.048611in}}%
\pgfusepath{stroke,fill}%
}%
\begin{pgfscope}%
\pgfsys@transformshift{3.992024in}{0.382904in}%
\pgfsys@useobject{currentmarker}{}%
\end{pgfscope}%
\end{pgfscope}%
\begin{pgfscope}%
\definecolor{textcolor}{rgb}{0.000000,0.000000,0.000000}%
\pgfsetstrokecolor{textcolor}%
\pgfsetfillcolor{textcolor}%
\pgftext[x=3.992024in,y=0.285682in,,top]{\color{textcolor}\rmfamily\fontsize{10.000000}{12.000000}\selectfont \(\displaystyle {250}\)}%
\end{pgfscope}%
\begin{pgfscope}%
\pgfsetbuttcap%
\pgfsetroundjoin%
\definecolor{currentfill}{rgb}{0.000000,0.000000,0.000000}%
\pgfsetfillcolor{currentfill}%
\pgfsetlinewidth{0.803000pt}%
\definecolor{currentstroke}{rgb}{0.000000,0.000000,0.000000}%
\pgfsetstrokecolor{currentstroke}%
\pgfsetdash{}{0pt}%
\pgfsys@defobject{currentmarker}{\pgfqpoint{0.000000in}{-0.048611in}}{\pgfqpoint{0.000000in}{0.000000in}}{%
\pgfpathmoveto{\pgfqpoint{0.000000in}{0.000000in}}%
\pgfpathlineto{\pgfqpoint{0.000000in}{-0.048611in}}%
\pgfusepath{stroke,fill}%
}%
\begin{pgfscope}%
\pgfsys@transformshift{4.669272in}{0.382904in}%
\pgfsys@useobject{currentmarker}{}%
\end{pgfscope}%
\end{pgfscope}%
\begin{pgfscope}%
\definecolor{textcolor}{rgb}{0.000000,0.000000,0.000000}%
\pgfsetstrokecolor{textcolor}%
\pgfsetfillcolor{textcolor}%
\pgftext[x=4.669272in,y=0.285682in,,top]{\color{textcolor}\rmfamily\fontsize{10.000000}{12.000000}\selectfont \(\displaystyle {300}\)}%
\end{pgfscope}%
\begin{pgfscope}%
\definecolor{textcolor}{rgb}{0.000000,0.000000,0.000000}%
\pgfsetstrokecolor{textcolor}%
\pgfsetfillcolor{textcolor}%
\pgftext[x=2.637528in,y=0.106793in,,top]{\color{textcolor}\rmfamily\fontsize{10.000000}{12.000000}\selectfont \(\displaystyle t\)}%
\end{pgfscope}%
\begin{pgfscope}%
\pgfsetbuttcap%
\pgfsetroundjoin%
\definecolor{currentfill}{rgb}{0.000000,0.000000,0.000000}%
\pgfsetfillcolor{currentfill}%
\pgfsetlinewidth{0.803000pt}%
\definecolor{currentstroke}{rgb}{0.000000,0.000000,0.000000}%
\pgfsetstrokecolor{currentstroke}%
\pgfsetdash{}{0pt}%
\pgfsys@defobject{currentmarker}{\pgfqpoint{-0.048611in}{0.000000in}}{\pgfqpoint{-0.000000in}{0.000000in}}{%
\pgfpathmoveto{\pgfqpoint{-0.000000in}{0.000000in}}%
\pgfpathlineto{\pgfqpoint{-0.048611in}{0.000000in}}%
\pgfusepath{stroke,fill}%
}%
\begin{pgfscope}%
\pgfsys@transformshift{0.605784in}{0.723263in}%
\pgfsys@useobject{currentmarker}{}%
\end{pgfscope}%
\end{pgfscope}%
\begin{pgfscope}%
\definecolor{textcolor}{rgb}{0.000000,0.000000,0.000000}%
\pgfsetstrokecolor{textcolor}%
\pgfsetfillcolor{textcolor}%
\pgftext[x=0.331092in, y=0.675069in, left, base]{\color{textcolor}\rmfamily\fontsize{10.000000}{12.000000}\selectfont \(\displaystyle {\ensuremath{-}5}\)}%
\end{pgfscope}%
\begin{pgfscope}%
\pgfsetbuttcap%
\pgfsetroundjoin%
\definecolor{currentfill}{rgb}{0.000000,0.000000,0.000000}%
\pgfsetfillcolor{currentfill}%
\pgfsetlinewidth{0.803000pt}%
\definecolor{currentstroke}{rgb}{0.000000,0.000000,0.000000}%
\pgfsetstrokecolor{currentstroke}%
\pgfsetdash{}{0pt}%
\pgfsys@defobject{currentmarker}{\pgfqpoint{-0.048611in}{0.000000in}}{\pgfqpoint{-0.000000in}{0.000000in}}{%
\pgfpathmoveto{\pgfqpoint{-0.000000in}{0.000000in}}%
\pgfpathlineto{\pgfqpoint{-0.048611in}{0.000000in}}%
\pgfusepath{stroke,fill}%
}%
\begin{pgfscope}%
\pgfsys@transformshift{0.605784in}{1.290528in}%
\pgfsys@useobject{currentmarker}{}%
\end{pgfscope}%
\end{pgfscope}%
\begin{pgfscope}%
\definecolor{textcolor}{rgb}{0.000000,0.000000,0.000000}%
\pgfsetstrokecolor{textcolor}%
\pgfsetfillcolor{textcolor}%
\pgftext[x=0.439117in, y=1.242334in, left, base]{\color{textcolor}\rmfamily\fontsize{10.000000}{12.000000}\selectfont \(\displaystyle {0}\)}%
\end{pgfscope}%
\begin{pgfscope}%
\pgfsetbuttcap%
\pgfsetroundjoin%
\definecolor{currentfill}{rgb}{0.000000,0.000000,0.000000}%
\pgfsetfillcolor{currentfill}%
\pgfsetlinewidth{0.803000pt}%
\definecolor{currentstroke}{rgb}{0.000000,0.000000,0.000000}%
\pgfsetstrokecolor{currentstroke}%
\pgfsetdash{}{0pt}%
\pgfsys@defobject{currentmarker}{\pgfqpoint{-0.048611in}{0.000000in}}{\pgfqpoint{-0.000000in}{0.000000in}}{%
\pgfpathmoveto{\pgfqpoint{-0.000000in}{0.000000in}}%
\pgfpathlineto{\pgfqpoint{-0.048611in}{0.000000in}}%
\pgfusepath{stroke,fill}%
}%
\begin{pgfscope}%
\pgfsys@transformshift{0.605784in}{1.857793in}%
\pgfsys@useobject{currentmarker}{}%
\end{pgfscope}%
\end{pgfscope}%
\begin{pgfscope}%
\definecolor{textcolor}{rgb}{0.000000,0.000000,0.000000}%
\pgfsetstrokecolor{textcolor}%
\pgfsetfillcolor{textcolor}%
\pgftext[x=0.439117in, y=1.809599in, left, base]{\color{textcolor}\rmfamily\fontsize{10.000000}{12.000000}\selectfont \(\displaystyle {5}\)}%
\end{pgfscope}%
\begin{pgfscope}%
\pgfsetbuttcap%
\pgfsetroundjoin%
\definecolor{currentfill}{rgb}{0.000000,0.000000,0.000000}%
\pgfsetfillcolor{currentfill}%
\pgfsetlinewidth{0.803000pt}%
\definecolor{currentstroke}{rgb}{0.000000,0.000000,0.000000}%
\pgfsetstrokecolor{currentstroke}%
\pgfsetdash{}{0pt}%
\pgfsys@defobject{currentmarker}{\pgfqpoint{-0.048611in}{0.000000in}}{\pgfqpoint{-0.000000in}{0.000000in}}{%
\pgfpathmoveto{\pgfqpoint{-0.000000in}{0.000000in}}%
\pgfpathlineto{\pgfqpoint{-0.048611in}{0.000000in}}%
\pgfusepath{stroke,fill}%
}%
\begin{pgfscope}%
\pgfsys@transformshift{0.605784in}{2.425059in}%
\pgfsys@useobject{currentmarker}{}%
\end{pgfscope}%
\end{pgfscope}%
\begin{pgfscope}%
\definecolor{textcolor}{rgb}{0.000000,0.000000,0.000000}%
\pgfsetstrokecolor{textcolor}%
\pgfsetfillcolor{textcolor}%
\pgftext[x=0.369672in, y=2.376864in, left, base]{\color{textcolor}\rmfamily\fontsize{10.000000}{12.000000}\selectfont \(\displaystyle {10}\)}%
\end{pgfscope}%
\begin{pgfscope}%
\definecolor{textcolor}{rgb}{0.000000,0.000000,0.000000}%
\pgfsetstrokecolor{textcolor}%
\pgfsetfillcolor{textcolor}%
\pgftext[x=0.275536in,y=1.403981in,,bottom,rotate=90.000000]{\color{textcolor}\rmfamily\fontsize{10.000000}{12.000000}\selectfont \(\displaystyle x\)}%
\end{pgfscope}%
\begin{pgfscope}%
\pgfpathrectangle{\pgfqpoint{0.605784in}{0.382904in}}{\pgfqpoint{4.063488in}{2.042155in}}%
\pgfusepath{clip}%
\pgfsetrectcap%
\pgfsetroundjoin%
\pgfsetlinewidth{1.505625pt}%
\definecolor{currentstroke}{rgb}{0.000000,0.000000,0.000000}%
\pgfsetstrokecolor{currentstroke}%
\pgfsetdash{}{0pt}%
\pgfpathmoveto{\pgfqpoint{0.605784in}{0.723263in}}%
\pgfpathlineto{\pgfqpoint{0.646829in}{0.723263in}}%
\pgfpathlineto{\pgfqpoint{0.687875in}{0.723263in}}%
\pgfpathlineto{\pgfqpoint{0.728920in}{0.723263in}}%
\pgfpathlineto{\pgfqpoint{0.769965in}{0.723263in}}%
\pgfpathlineto{\pgfqpoint{0.811011in}{0.723263in}}%
\pgfpathlineto{\pgfqpoint{0.852056in}{0.723263in}}%
\pgfpathlineto{\pgfqpoint{0.893101in}{0.723263in}}%
\pgfpathlineto{\pgfqpoint{0.934147in}{0.723263in}}%
\pgfpathlineto{\pgfqpoint{0.975192in}{0.723263in}}%
\pgfpathlineto{\pgfqpoint{1.016237in}{0.723263in}}%
\pgfpathlineto{\pgfqpoint{1.057283in}{0.723263in}}%
\pgfpathlineto{\pgfqpoint{1.098328in}{0.723263in}}%
\pgfpathlineto{\pgfqpoint{1.139373in}{0.723263in}}%
\pgfpathlineto{\pgfqpoint{1.180419in}{0.723263in}}%
\pgfpathlineto{\pgfqpoint{1.221464in}{0.723263in}}%
\pgfpathlineto{\pgfqpoint{1.262509in}{0.723263in}}%
\pgfpathlineto{\pgfqpoint{1.303555in}{0.723263in}}%
\pgfpathlineto{\pgfqpoint{1.344600in}{0.723263in}}%
\pgfpathlineto{\pgfqpoint{1.385645in}{0.723263in}}%
\pgfpathlineto{\pgfqpoint{1.426691in}{0.723263in}}%
\pgfpathlineto{\pgfqpoint{1.467736in}{0.723263in}}%
\pgfpathlineto{\pgfqpoint{1.508781in}{0.723263in}}%
\pgfpathlineto{\pgfqpoint{1.549827in}{0.723263in}}%
\pgfpathlineto{\pgfqpoint{1.590872in}{0.723263in}}%
\pgfpathlineto{\pgfqpoint{1.631917in}{0.723263in}}%
\pgfpathlineto{\pgfqpoint{1.672963in}{0.723263in}}%
\pgfpathlineto{\pgfqpoint{1.714008in}{0.723263in}}%
\pgfpathlineto{\pgfqpoint{1.755053in}{0.723263in}}%
\pgfpathlineto{\pgfqpoint{1.796099in}{0.723263in}}%
\pgfpathlineto{\pgfqpoint{1.837144in}{0.723263in}}%
\pgfpathlineto{\pgfqpoint{1.878189in}{0.723263in}}%
\pgfpathlineto{\pgfqpoint{1.919235in}{0.723263in}}%
\pgfpathlineto{\pgfqpoint{1.960280in}{0.723263in}}%
\pgfpathlineto{\pgfqpoint{2.001325in}{0.723263in}}%
\pgfpathlineto{\pgfqpoint{2.042371in}{0.723263in}}%
\pgfpathlineto{\pgfqpoint{2.083416in}{0.723263in}}%
\pgfpathlineto{\pgfqpoint{2.124461in}{0.723263in}}%
\pgfpathlineto{\pgfqpoint{2.165507in}{0.723263in}}%
\pgfpathlineto{\pgfqpoint{2.206552in}{0.723263in}}%
\pgfpathlineto{\pgfqpoint{2.247597in}{0.723263in}}%
\pgfpathlineto{\pgfqpoint{2.288643in}{0.723263in}}%
\pgfpathlineto{\pgfqpoint{2.329688in}{0.723263in}}%
\pgfpathlineto{\pgfqpoint{2.370733in}{0.723263in}}%
\pgfpathlineto{\pgfqpoint{2.411779in}{0.723263in}}%
\pgfpathlineto{\pgfqpoint{2.452824in}{0.723263in}}%
\pgfpathlineto{\pgfqpoint{2.493869in}{0.723263in}}%
\pgfpathlineto{\pgfqpoint{2.534915in}{0.723263in}}%
\pgfpathlineto{\pgfqpoint{2.575960in}{0.723263in}}%
\pgfpathlineto{\pgfqpoint{2.617005in}{0.723263in}}%
\pgfpathlineto{\pgfqpoint{2.658051in}{0.723263in}}%
\pgfpathlineto{\pgfqpoint{2.699096in}{0.723263in}}%
\pgfpathlineto{\pgfqpoint{2.740141in}{0.723263in}}%
\pgfpathlineto{\pgfqpoint{2.781187in}{0.723263in}}%
\pgfpathlineto{\pgfqpoint{2.822232in}{0.723263in}}%
\pgfpathlineto{\pgfqpoint{2.863277in}{0.723263in}}%
\pgfpathlineto{\pgfqpoint{2.904323in}{0.723263in}}%
\pgfpathlineto{\pgfqpoint{2.945368in}{0.723263in}}%
\pgfpathlineto{\pgfqpoint{2.986413in}{0.723263in}}%
\pgfpathlineto{\pgfqpoint{3.027459in}{0.723263in}}%
\pgfpathlineto{\pgfqpoint{3.068504in}{0.723263in}}%
\pgfpathlineto{\pgfqpoint{3.109549in}{0.723263in}}%
\pgfpathlineto{\pgfqpoint{3.150595in}{0.723263in}}%
\pgfpathlineto{\pgfqpoint{3.191640in}{0.723263in}}%
\pgfpathlineto{\pgfqpoint{3.232685in}{0.723263in}}%
\pgfpathlineto{\pgfqpoint{3.273731in}{0.723263in}}%
\pgfpathlineto{\pgfqpoint{3.314776in}{0.723263in}}%
\pgfpathlineto{\pgfqpoint{3.355821in}{0.723263in}}%
\pgfpathlineto{\pgfqpoint{3.396867in}{0.723263in}}%
\pgfpathlineto{\pgfqpoint{3.437912in}{0.723263in}}%
\pgfpathlineto{\pgfqpoint{3.478957in}{0.723263in}}%
\pgfpathlineto{\pgfqpoint{3.520003in}{0.723263in}}%
\pgfpathlineto{\pgfqpoint{3.561048in}{0.723263in}}%
\pgfpathlineto{\pgfqpoint{3.602093in}{0.723263in}}%
\pgfpathlineto{\pgfqpoint{3.643139in}{0.723263in}}%
\pgfpathlineto{\pgfqpoint{3.684184in}{0.723263in}}%
\pgfpathlineto{\pgfqpoint{3.725229in}{0.723263in}}%
\pgfpathlineto{\pgfqpoint{3.766275in}{0.723263in}}%
\pgfpathlineto{\pgfqpoint{3.807320in}{0.723263in}}%
\pgfpathlineto{\pgfqpoint{3.848365in}{0.723263in}}%
\pgfpathlineto{\pgfqpoint{3.889411in}{0.723263in}}%
\pgfpathlineto{\pgfqpoint{3.930456in}{0.723263in}}%
\pgfpathlineto{\pgfqpoint{3.971501in}{0.723263in}}%
\pgfpathlineto{\pgfqpoint{4.012547in}{0.723263in}}%
\pgfpathlineto{\pgfqpoint{4.053592in}{0.723263in}}%
\pgfpathlineto{\pgfqpoint{4.094637in}{0.723263in}}%
\pgfpathlineto{\pgfqpoint{4.135683in}{0.723263in}}%
\pgfpathlineto{\pgfqpoint{4.176728in}{0.723263in}}%
\pgfpathlineto{\pgfqpoint{4.217773in}{0.723263in}}%
\pgfpathlineto{\pgfqpoint{4.258819in}{0.723263in}}%
\pgfpathlineto{\pgfqpoint{4.299864in}{0.723263in}}%
\pgfpathlineto{\pgfqpoint{4.340909in}{0.723263in}}%
\pgfpathlineto{\pgfqpoint{4.381955in}{0.723263in}}%
\pgfpathlineto{\pgfqpoint{4.423000in}{0.723263in}}%
\pgfpathlineto{\pgfqpoint{4.464045in}{0.723263in}}%
\pgfpathlineto{\pgfqpoint{4.505091in}{0.723263in}}%
\pgfpathlineto{\pgfqpoint{4.546136in}{0.723263in}}%
\pgfpathlineto{\pgfqpoint{4.587181in}{0.723263in}}%
\pgfpathlineto{\pgfqpoint{4.628227in}{0.723263in}}%
\pgfpathlineto{\pgfqpoint{4.669272in}{0.723263in}}%
\pgfusepath{stroke}%
\end{pgfscope}%
\begin{pgfscope}%
\pgfpathrectangle{\pgfqpoint{0.605784in}{0.382904in}}{\pgfqpoint{4.063488in}{2.042155in}}%
\pgfusepath{clip}%
\pgfsetrectcap%
\pgfsetroundjoin%
\pgfsetlinewidth{1.505625pt}%
\definecolor{currentstroke}{rgb}{0.000000,0.000000,0.000000}%
\pgfsetstrokecolor{currentstroke}%
\pgfsetdash{}{0pt}%
\pgfpathmoveto{\pgfqpoint{0.605784in}{2.311606in}}%
\pgfpathlineto{\pgfqpoint{0.646829in}{2.311606in}}%
\pgfpathlineto{\pgfqpoint{0.687875in}{2.311606in}}%
\pgfpathlineto{\pgfqpoint{0.728920in}{2.311606in}}%
\pgfpathlineto{\pgfqpoint{0.769965in}{2.311606in}}%
\pgfpathlineto{\pgfqpoint{0.811011in}{2.311606in}}%
\pgfpathlineto{\pgfqpoint{0.852056in}{2.311606in}}%
\pgfpathlineto{\pgfqpoint{0.893101in}{2.311606in}}%
\pgfpathlineto{\pgfqpoint{0.934147in}{2.311606in}}%
\pgfpathlineto{\pgfqpoint{0.975192in}{2.311606in}}%
\pgfpathlineto{\pgfqpoint{1.016237in}{2.311606in}}%
\pgfpathlineto{\pgfqpoint{1.057283in}{2.311606in}}%
\pgfpathlineto{\pgfqpoint{1.098328in}{2.311606in}}%
\pgfpathlineto{\pgfqpoint{1.139373in}{2.311606in}}%
\pgfpathlineto{\pgfqpoint{1.180419in}{2.311606in}}%
\pgfpathlineto{\pgfqpoint{1.221464in}{2.311606in}}%
\pgfpathlineto{\pgfqpoint{1.262509in}{2.311606in}}%
\pgfpathlineto{\pgfqpoint{1.303555in}{2.311606in}}%
\pgfpathlineto{\pgfqpoint{1.344600in}{2.311606in}}%
\pgfpathlineto{\pgfqpoint{1.385645in}{2.311606in}}%
\pgfpathlineto{\pgfqpoint{1.426691in}{2.311606in}}%
\pgfpathlineto{\pgfqpoint{1.467736in}{2.311606in}}%
\pgfpathlineto{\pgfqpoint{1.508781in}{2.311606in}}%
\pgfpathlineto{\pgfqpoint{1.549827in}{2.311606in}}%
\pgfpathlineto{\pgfqpoint{1.590872in}{2.311606in}}%
\pgfpathlineto{\pgfqpoint{1.631917in}{2.311606in}}%
\pgfpathlineto{\pgfqpoint{1.672963in}{2.311606in}}%
\pgfpathlineto{\pgfqpoint{1.714008in}{2.311606in}}%
\pgfpathlineto{\pgfqpoint{1.755053in}{2.311606in}}%
\pgfpathlineto{\pgfqpoint{1.796099in}{2.311606in}}%
\pgfpathlineto{\pgfqpoint{1.837144in}{2.311606in}}%
\pgfpathlineto{\pgfqpoint{1.878189in}{2.311606in}}%
\pgfpathlineto{\pgfqpoint{1.919235in}{2.311606in}}%
\pgfpathlineto{\pgfqpoint{1.960280in}{2.311606in}}%
\pgfpathlineto{\pgfqpoint{2.001325in}{2.311606in}}%
\pgfpathlineto{\pgfqpoint{2.042371in}{2.311606in}}%
\pgfpathlineto{\pgfqpoint{2.083416in}{2.311606in}}%
\pgfpathlineto{\pgfqpoint{2.124461in}{2.311606in}}%
\pgfpathlineto{\pgfqpoint{2.165507in}{2.311606in}}%
\pgfpathlineto{\pgfqpoint{2.206552in}{2.311606in}}%
\pgfpathlineto{\pgfqpoint{2.247597in}{2.311606in}}%
\pgfpathlineto{\pgfqpoint{2.288643in}{2.311606in}}%
\pgfpathlineto{\pgfqpoint{2.329688in}{2.311606in}}%
\pgfpathlineto{\pgfqpoint{2.370733in}{2.311606in}}%
\pgfpathlineto{\pgfqpoint{2.411779in}{2.311606in}}%
\pgfpathlineto{\pgfqpoint{2.452824in}{2.311606in}}%
\pgfpathlineto{\pgfqpoint{2.493869in}{2.311606in}}%
\pgfpathlineto{\pgfqpoint{2.534915in}{2.311606in}}%
\pgfpathlineto{\pgfqpoint{2.575960in}{2.311606in}}%
\pgfpathlineto{\pgfqpoint{2.617005in}{2.311606in}}%
\pgfpathlineto{\pgfqpoint{2.658051in}{2.311606in}}%
\pgfpathlineto{\pgfqpoint{2.699096in}{2.311606in}}%
\pgfpathlineto{\pgfqpoint{2.740141in}{2.311606in}}%
\pgfpathlineto{\pgfqpoint{2.781187in}{2.311606in}}%
\pgfpathlineto{\pgfqpoint{2.822232in}{2.311606in}}%
\pgfpathlineto{\pgfqpoint{2.863277in}{2.311606in}}%
\pgfpathlineto{\pgfqpoint{2.904323in}{2.311606in}}%
\pgfpathlineto{\pgfqpoint{2.945368in}{2.311606in}}%
\pgfpathlineto{\pgfqpoint{2.986413in}{2.311606in}}%
\pgfpathlineto{\pgfqpoint{3.027459in}{2.311606in}}%
\pgfpathlineto{\pgfqpoint{3.068504in}{2.311606in}}%
\pgfpathlineto{\pgfqpoint{3.109549in}{2.311606in}}%
\pgfpathlineto{\pgfqpoint{3.150595in}{2.311606in}}%
\pgfpathlineto{\pgfqpoint{3.191640in}{2.311606in}}%
\pgfpathlineto{\pgfqpoint{3.232685in}{2.311606in}}%
\pgfpathlineto{\pgfqpoint{3.273731in}{2.311606in}}%
\pgfpathlineto{\pgfqpoint{3.314776in}{2.311606in}}%
\pgfpathlineto{\pgfqpoint{3.355821in}{2.311606in}}%
\pgfpathlineto{\pgfqpoint{3.396867in}{2.311606in}}%
\pgfpathlineto{\pgfqpoint{3.437912in}{2.311606in}}%
\pgfpathlineto{\pgfqpoint{3.478957in}{2.311606in}}%
\pgfpathlineto{\pgfqpoint{3.520003in}{2.311606in}}%
\pgfpathlineto{\pgfqpoint{3.561048in}{2.311606in}}%
\pgfpathlineto{\pgfqpoint{3.602093in}{2.311606in}}%
\pgfpathlineto{\pgfqpoint{3.643139in}{2.311606in}}%
\pgfpathlineto{\pgfqpoint{3.684184in}{2.311606in}}%
\pgfpathlineto{\pgfqpoint{3.725229in}{2.311606in}}%
\pgfpathlineto{\pgfqpoint{3.766275in}{2.311606in}}%
\pgfpathlineto{\pgfqpoint{3.807320in}{2.311606in}}%
\pgfpathlineto{\pgfqpoint{3.848365in}{2.311606in}}%
\pgfpathlineto{\pgfqpoint{3.889411in}{2.311606in}}%
\pgfpathlineto{\pgfqpoint{3.930456in}{2.311606in}}%
\pgfpathlineto{\pgfqpoint{3.971501in}{2.311606in}}%
\pgfpathlineto{\pgfqpoint{4.012547in}{2.311606in}}%
\pgfpathlineto{\pgfqpoint{4.053592in}{2.311606in}}%
\pgfpathlineto{\pgfqpoint{4.094637in}{2.311606in}}%
\pgfpathlineto{\pgfqpoint{4.135683in}{2.311606in}}%
\pgfpathlineto{\pgfqpoint{4.176728in}{2.311606in}}%
\pgfpathlineto{\pgfqpoint{4.217773in}{2.311606in}}%
\pgfpathlineto{\pgfqpoint{4.258819in}{2.311606in}}%
\pgfpathlineto{\pgfqpoint{4.299864in}{2.311606in}}%
\pgfpathlineto{\pgfqpoint{4.340909in}{2.311606in}}%
\pgfpathlineto{\pgfqpoint{4.381955in}{2.311606in}}%
\pgfpathlineto{\pgfqpoint{4.423000in}{2.311606in}}%
\pgfpathlineto{\pgfqpoint{4.464045in}{2.311606in}}%
\pgfpathlineto{\pgfqpoint{4.505091in}{2.311606in}}%
\pgfpathlineto{\pgfqpoint{4.546136in}{2.311606in}}%
\pgfpathlineto{\pgfqpoint{4.587181in}{2.311606in}}%
\pgfpathlineto{\pgfqpoint{4.628227in}{2.311606in}}%
\pgfpathlineto{\pgfqpoint{4.669272in}{2.311606in}}%
\pgfusepath{stroke}%
\end{pgfscope}%
\begin{pgfscope}%
\pgfpathrectangle{\pgfqpoint{0.605784in}{0.382904in}}{\pgfqpoint{4.063488in}{2.042155in}}%
\pgfusepath{clip}%
\pgfsetrectcap%
\pgfsetroundjoin%
\pgfsetlinewidth{1.003750pt}%
\definecolor{currentstroke}{rgb}{1.000000,1.000000,1.000000}%
\pgfsetstrokecolor{currentstroke}%
\pgfsetdash{}{0pt}%
\pgfpathmoveto{\pgfqpoint{0.605784in}{1.290528in}}%
\pgfpathlineto{\pgfqpoint{0.646829in}{1.270424in}}%
\pgfpathlineto{\pgfqpoint{0.687875in}{1.253405in}}%
\pgfpathlineto{\pgfqpoint{0.728920in}{1.242276in}}%
\pgfpathlineto{\pgfqpoint{0.769965in}{1.239303in}}%
\pgfpathlineto{\pgfqpoint{0.811011in}{1.246010in}}%
\pgfpathlineto{\pgfqpoint{0.852056in}{1.263041in}}%
\pgfpathlineto{\pgfqpoint{0.893101in}{1.290095in}}%
\pgfpathlineto{\pgfqpoint{0.934147in}{1.325960in}}%
\pgfpathlineto{\pgfqpoint{0.975192in}{1.368623in}}%
\pgfpathlineto{\pgfqpoint{1.016237in}{1.415447in}}%
\pgfpathlineto{\pgfqpoint{1.057283in}{1.463420in}}%
\pgfpathlineto{\pgfqpoint{1.098328in}{1.509422in}}%
\pgfpathlineto{\pgfqpoint{1.139373in}{1.550515in}}%
\pgfpathlineto{\pgfqpoint{1.180419in}{1.584208in}}%
\pgfpathlineto{\pgfqpoint{1.221464in}{1.608682in}}%
\pgfpathlineto{\pgfqpoint{1.262509in}{1.622962in}}%
\pgfpathlineto{\pgfqpoint{1.303555in}{1.626999in}}%
\pgfpathlineto{\pgfqpoint{1.344600in}{1.621679in}}%
\pgfpathlineto{\pgfqpoint{1.385645in}{1.608738in}}%
\pgfpathlineto{\pgfqpoint{1.426691in}{1.590610in}}%
\pgfpathlineto{\pgfqpoint{1.467736in}{1.570200in}}%
\pgfpathlineto{\pgfqpoint{1.508781in}{1.550621in}}%
\pgfpathlineto{\pgfqpoint{1.549827in}{1.534910in}}%
\pgfpathlineto{\pgfqpoint{1.590872in}{1.525752in}}%
\pgfpathlineto{\pgfqpoint{1.631917in}{1.525235in}}%
\pgfpathlineto{\pgfqpoint{1.672963in}{1.534659in}}%
\pgfpathlineto{\pgfqpoint{1.714008in}{1.554418in}}%
\pgfpathlineto{\pgfqpoint{1.755053in}{1.583966in}}%
\pgfpathlineto{\pgfqpoint{1.796099in}{1.621862in}}%
\pgfpathlineto{\pgfqpoint{1.837144in}{1.665906in}}%
\pgfpathlineto{\pgfqpoint{1.878189in}{1.713338in}}%
\pgfpathlineto{\pgfqpoint{1.919235in}{1.761090in}}%
\pgfpathlineto{\pgfqpoint{1.960280in}{1.806061in}}%
\pgfpathlineto{\pgfqpoint{2.001325in}{1.845409in}}%
\pgfpathlineto{\pgfqpoint{2.042371in}{1.876799in}}%
\pgfpathlineto{\pgfqpoint{2.083416in}{1.898625in}}%
\pgfpathlineto{\pgfqpoint{2.124461in}{1.910151in}}%
\pgfpathlineto{\pgfqpoint{2.165507in}{1.911580in}}%
\pgfpathlineto{\pgfqpoint{2.206552in}{1.904035in}}%
\pgfpathlineto{\pgfqpoint{2.247597in}{1.889457in}}%
\pgfpathlineto{\pgfqpoint{2.288643in}{1.870426in}}%
\pgfpathlineto{\pgfqpoint{2.329688in}{1.849931in}}%
\pgfpathlineto{\pgfqpoint{2.370733in}{1.831093in}}%
\pgfpathlineto{\pgfqpoint{2.411779in}{1.816880in}}%
\pgfpathlineto{\pgfqpoint{2.452824in}{1.809842in}}%
\pgfpathlineto{\pgfqpoint{2.493869in}{1.811873in}}%
\pgfpathlineto{\pgfqpoint{2.534915in}{1.399321in}}%
\pgfpathlineto{\pgfqpoint{2.575960in}{1.399321in}}%
\pgfpathlineto{\pgfqpoint{2.617005in}{1.399321in}}%
\pgfpathlineto{\pgfqpoint{2.658051in}{1.399321in}}%
\pgfpathlineto{\pgfqpoint{2.699096in}{1.399321in}}%
\pgfpathlineto{\pgfqpoint{2.740141in}{1.399321in}}%
\pgfpathlineto{\pgfqpoint{2.781187in}{1.399321in}}%
\pgfpathlineto{\pgfqpoint{2.822232in}{1.399321in}}%
\pgfpathlineto{\pgfqpoint{2.863277in}{1.399321in}}%
\pgfpathlineto{\pgfqpoint{2.904323in}{1.399321in}}%
\pgfpathlineto{\pgfqpoint{2.945368in}{1.399321in}}%
\pgfpathlineto{\pgfqpoint{2.986413in}{1.399321in}}%
\pgfpathlineto{\pgfqpoint{3.027459in}{1.399321in}}%
\pgfpathlineto{\pgfqpoint{3.068504in}{1.399321in}}%
\pgfpathlineto{\pgfqpoint{3.109549in}{1.399321in}}%
\pgfpathlineto{\pgfqpoint{3.150595in}{1.399321in}}%
\pgfpathlineto{\pgfqpoint{3.191640in}{1.399321in}}%
\pgfpathlineto{\pgfqpoint{3.232685in}{1.399321in}}%
\pgfpathlineto{\pgfqpoint{3.273731in}{1.399321in}}%
\pgfpathlineto{\pgfqpoint{3.314776in}{1.399321in}}%
\pgfpathlineto{\pgfqpoint{3.355821in}{1.399321in}}%
\pgfpathlineto{\pgfqpoint{3.396867in}{1.399321in}}%
\pgfpathlineto{\pgfqpoint{3.437912in}{1.399321in}}%
\pgfpathlineto{\pgfqpoint{3.478957in}{1.399321in}}%
\pgfpathlineto{\pgfqpoint{3.520003in}{1.399321in}}%
\pgfpathlineto{\pgfqpoint{3.561048in}{1.399321in}}%
\pgfpathlineto{\pgfqpoint{3.602093in}{1.399321in}}%
\pgfpathlineto{\pgfqpoint{3.643139in}{1.399321in}}%
\pgfpathlineto{\pgfqpoint{3.684184in}{1.181735in}}%
\pgfpathlineto{\pgfqpoint{3.725229in}{1.181735in}}%
\pgfpathlineto{\pgfqpoint{3.766275in}{1.181735in}}%
\pgfpathlineto{\pgfqpoint{3.807320in}{1.181735in}}%
\pgfpathlineto{\pgfqpoint{3.848365in}{1.181735in}}%
\pgfpathlineto{\pgfqpoint{3.889411in}{1.181735in}}%
\pgfpathlineto{\pgfqpoint{3.930456in}{1.181735in}}%
\pgfpathlineto{\pgfqpoint{3.971501in}{1.181735in}}%
\pgfpathlineto{\pgfqpoint{4.012547in}{1.181735in}}%
\pgfpathlineto{\pgfqpoint{4.053592in}{1.181735in}}%
\pgfpathlineto{\pgfqpoint{4.094637in}{1.181735in}}%
\pgfpathlineto{\pgfqpoint{4.135683in}{1.181735in}}%
\pgfpathlineto{\pgfqpoint{4.176728in}{1.181735in}}%
\pgfpathlineto{\pgfqpoint{4.217773in}{1.181735in}}%
\pgfpathlineto{\pgfqpoint{4.258819in}{1.181735in}}%
\pgfpathlineto{\pgfqpoint{4.299864in}{1.181735in}}%
\pgfpathlineto{\pgfqpoint{4.340909in}{1.181735in}}%
\pgfpathlineto{\pgfqpoint{4.381955in}{1.181735in}}%
\pgfpathlineto{\pgfqpoint{4.423000in}{1.181735in}}%
\pgfpathlineto{\pgfqpoint{4.464045in}{1.181735in}}%
\pgfpathlineto{\pgfqpoint{4.505091in}{1.181735in}}%
\pgfpathlineto{\pgfqpoint{4.546136in}{1.181735in}}%
\pgfpathlineto{\pgfqpoint{4.587181in}{1.181735in}}%
\pgfpathlineto{\pgfqpoint{4.628227in}{1.181735in}}%
\pgfpathlineto{\pgfqpoint{4.669272in}{1.181735in}}%
\pgfusepath{stroke}%
\end{pgfscope}%
\begin{pgfscope}%
\pgfpathrectangle{\pgfqpoint{0.605784in}{0.382904in}}{\pgfqpoint{4.063488in}{2.042155in}}%
\pgfusepath{clip}%
\pgfsetbuttcap%
\pgfsetbeveljoin%
\definecolor{currentfill}{rgb}{1.000000,1.000000,1.000000}%
\pgfsetfillcolor{currentfill}%
\pgfsetlinewidth{1.003750pt}%
\definecolor{currentstroke}{rgb}{1.000000,1.000000,1.000000}%
\pgfsetstrokecolor{currentstroke}%
\pgfsetdash{}{0pt}%
\pgfsys@defobject{currentmarker}{\pgfqpoint{-0.019814in}{-0.016855in}}{\pgfqpoint{0.019814in}{0.020833in}}{%
\pgfpathmoveto{\pgfqpoint{0.000000in}{0.020833in}}%
\pgfpathlineto{\pgfqpoint{-0.004677in}{0.006438in}}%
\pgfpathlineto{\pgfqpoint{-0.019814in}{0.006438in}}%
\pgfpathlineto{\pgfqpoint{-0.007568in}{-0.002459in}}%
\pgfpathlineto{\pgfqpoint{-0.012246in}{-0.016855in}}%
\pgfpathlineto{\pgfqpoint{-0.000000in}{-0.007958in}}%
\pgfpathlineto{\pgfqpoint{0.012246in}{-0.016855in}}%
\pgfpathlineto{\pgfqpoint{0.007568in}{-0.002459in}}%
\pgfpathlineto{\pgfqpoint{0.019814in}{0.006438in}}%
\pgfpathlineto{\pgfqpoint{0.004677in}{0.006438in}}%
\pgfpathclose%
\pgfusepath{stroke,fill}%
}%
\begin{pgfscope}%
\pgfsys@transformshift{0.808958in}{1.315941in}%
\pgfsys@useobject{currentmarker}{}%
\end{pgfscope}%
\begin{pgfscope}%
\pgfsys@transformshift{0.822503in}{1.188936in}%
\pgfsys@useobject{currentmarker}{}%
\end{pgfscope}%
\begin{pgfscope}%
\pgfsys@transformshift{0.836048in}{1.238192in}%
\pgfsys@useobject{currentmarker}{}%
\end{pgfscope}%
\begin{pgfscope}%
\pgfsys@transformshift{0.849593in}{1.212944in}%
\pgfsys@useobject{currentmarker}{}%
\end{pgfscope}%
\begin{pgfscope}%
\pgfsys@transformshift{0.863138in}{1.239792in}%
\pgfsys@useobject{currentmarker}{}%
\end{pgfscope}%
\begin{pgfscope}%
\pgfsys@transformshift{0.876683in}{1.212378in}%
\pgfsys@useobject{currentmarker}{}%
\end{pgfscope}%
\begin{pgfscope}%
\pgfsys@transformshift{0.890228in}{1.260050in}%
\pgfsys@useobject{currentmarker}{}%
\end{pgfscope}%
\begin{pgfscope}%
\pgfsys@transformshift{0.903773in}{1.170408in}%
\pgfsys@useobject{currentmarker}{}%
\end{pgfscope}%
\begin{pgfscope}%
\pgfsys@transformshift{0.917318in}{1.321117in}%
\pgfsys@useobject{currentmarker}{}%
\end{pgfscope}%
\begin{pgfscope}%
\pgfsys@transformshift{0.930863in}{1.230933in}%
\pgfsys@useobject{currentmarker}{}%
\end{pgfscope}%
\begin{pgfscope}%
\pgfsys@transformshift{0.944408in}{1.314086in}%
\pgfsys@useobject{currentmarker}{}%
\end{pgfscope}%
\begin{pgfscope}%
\pgfsys@transformshift{0.957953in}{1.247541in}%
\pgfsys@useobject{currentmarker}{}%
\end{pgfscope}%
\begin{pgfscope}%
\pgfsys@transformshift{0.971498in}{1.339468in}%
\pgfsys@useobject{currentmarker}{}%
\end{pgfscope}%
\begin{pgfscope}%
\pgfsys@transformshift{0.985043in}{1.259284in}%
\pgfsys@useobject{currentmarker}{}%
\end{pgfscope}%
\begin{pgfscope}%
\pgfsys@transformshift{0.998588in}{1.373791in}%
\pgfsys@useobject{currentmarker}{}%
\end{pgfscope}%
\begin{pgfscope}%
\pgfsys@transformshift{1.012133in}{1.293613in}%
\pgfsys@useobject{currentmarker}{}%
\end{pgfscope}%
\begin{pgfscope}%
\pgfsys@transformshift{1.025678in}{1.390059in}%
\pgfsys@useobject{currentmarker}{}%
\end{pgfscope}%
\begin{pgfscope}%
\pgfsys@transformshift{1.039223in}{1.305615in}%
\pgfsys@useobject{currentmarker}{}%
\end{pgfscope}%
\begin{pgfscope}%
\pgfsys@transformshift{1.052768in}{1.419151in}%
\pgfsys@useobject{currentmarker}{}%
\end{pgfscope}%
\begin{pgfscope}%
\pgfsys@transformshift{1.066313in}{1.347592in}%
\pgfsys@useobject{currentmarker}{}%
\end{pgfscope}%
\begin{pgfscope}%
\pgfsys@transformshift{1.079857in}{1.458850in}%
\pgfsys@useobject{currentmarker}{}%
\end{pgfscope}%
\begin{pgfscope}%
\pgfsys@transformshift{1.093402in}{1.380890in}%
\pgfsys@useobject{currentmarker}{}%
\end{pgfscope}%
\begin{pgfscope}%
\pgfsys@transformshift{1.106947in}{1.474693in}%
\pgfsys@useobject{currentmarker}{}%
\end{pgfscope}%
\begin{pgfscope}%
\pgfsys@transformshift{1.120492in}{1.390072in}%
\pgfsys@useobject{currentmarker}{}%
\end{pgfscope}%
\begin{pgfscope}%
\pgfsys@transformshift{1.134037in}{1.521032in}%
\pgfsys@useobject{currentmarker}{}%
\end{pgfscope}%
\begin{pgfscope}%
\pgfsys@transformshift{1.147582in}{1.462352in}%
\pgfsys@useobject{currentmarker}{}%
\end{pgfscope}%
\begin{pgfscope}%
\pgfsys@transformshift{1.161127in}{1.541074in}%
\pgfsys@useobject{currentmarker}{}%
\end{pgfscope}%
\begin{pgfscope}%
\pgfsys@transformshift{1.174672in}{1.484404in}%
\pgfsys@useobject{currentmarker}{}%
\end{pgfscope}%
\begin{pgfscope}%
\pgfsys@transformshift{1.188217in}{1.566405in}%
\pgfsys@useobject{currentmarker}{}%
\end{pgfscope}%
\begin{pgfscope}%
\pgfsys@transformshift{1.201762in}{1.473697in}%
\pgfsys@useobject{currentmarker}{}%
\end{pgfscope}%
\begin{pgfscope}%
\pgfsys@transformshift{1.215307in}{1.581115in}%
\pgfsys@useobject{currentmarker}{}%
\end{pgfscope}%
\begin{pgfscope}%
\pgfsys@transformshift{1.228852in}{1.515541in}%
\pgfsys@useobject{currentmarker}{}%
\end{pgfscope}%
\begin{pgfscope}%
\pgfsys@transformshift{1.242397in}{1.579454in}%
\pgfsys@useobject{currentmarker}{}%
\end{pgfscope}%
\begin{pgfscope}%
\pgfsys@transformshift{1.255942in}{1.527149in}%
\pgfsys@useobject{currentmarker}{}%
\end{pgfscope}%
\begin{pgfscope}%
\pgfsys@transformshift{1.269487in}{1.586569in}%
\pgfsys@useobject{currentmarker}{}%
\end{pgfscope}%
\begin{pgfscope}%
\pgfsys@transformshift{1.283032in}{1.522895in}%
\pgfsys@useobject{currentmarker}{}%
\end{pgfscope}%
\begin{pgfscope}%
\pgfsys@transformshift{1.296577in}{1.604869in}%
\pgfsys@useobject{currentmarker}{}%
\end{pgfscope}%
\begin{pgfscope}%
\pgfsys@transformshift{1.310122in}{1.539445in}%
\pgfsys@useobject{currentmarker}{}%
\end{pgfscope}%
\begin{pgfscope}%
\pgfsys@transformshift{1.323667in}{1.602529in}%
\pgfsys@useobject{currentmarker}{}%
\end{pgfscope}%
\begin{pgfscope}%
\pgfsys@transformshift{1.337212in}{1.549652in}%
\pgfsys@useobject{currentmarker}{}%
\end{pgfscope}%
\begin{pgfscope}%
\pgfsys@transformshift{1.350757in}{1.609509in}%
\pgfsys@useobject{currentmarker}{}%
\end{pgfscope}%
\begin{pgfscope}%
\pgfsys@transformshift{1.364302in}{1.545992in}%
\pgfsys@useobject{currentmarker}{}%
\end{pgfscope}%
\begin{pgfscope}%
\pgfsys@transformshift{1.377847in}{1.623768in}%
\pgfsys@useobject{currentmarker}{}%
\end{pgfscope}%
\begin{pgfscope}%
\pgfsys@transformshift{1.391392in}{1.552556in}%
\pgfsys@useobject{currentmarker}{}%
\end{pgfscope}%
\begin{pgfscope}%
\pgfsys@transformshift{1.404937in}{1.613523in}%
\pgfsys@useobject{currentmarker}{}%
\end{pgfscope}%
\begin{pgfscope}%
\pgfsys@transformshift{1.418481in}{1.556647in}%
\pgfsys@useobject{currentmarker}{}%
\end{pgfscope}%
\begin{pgfscope}%
\pgfsys@transformshift{1.432026in}{1.619421in}%
\pgfsys@useobject{currentmarker}{}%
\end{pgfscope}%
\begin{pgfscope}%
\pgfsys@transformshift{1.445571in}{1.542105in}%
\pgfsys@useobject{currentmarker}{}%
\end{pgfscope}%
\begin{pgfscope}%
\pgfsys@transformshift{1.459116in}{1.614773in}%
\pgfsys@useobject{currentmarker}{}%
\end{pgfscope}%
\begin{pgfscope}%
\pgfsys@transformshift{1.472661in}{1.541288in}%
\pgfsys@useobject{currentmarker}{}%
\end{pgfscope}%
\begin{pgfscope}%
\pgfsys@transformshift{1.486206in}{1.621080in}%
\pgfsys@useobject{currentmarker}{}%
\end{pgfscope}%
\begin{pgfscope}%
\pgfsys@transformshift{1.499751in}{1.535607in}%
\pgfsys@useobject{currentmarker}{}%
\end{pgfscope}%
\begin{pgfscope}%
\pgfsys@transformshift{1.513296in}{1.612624in}%
\pgfsys@useobject{currentmarker}{}%
\end{pgfscope}%
\begin{pgfscope}%
\pgfsys@transformshift{1.526841in}{1.531726in}%
\pgfsys@useobject{currentmarker}{}%
\end{pgfscope}%
\begin{pgfscope}%
\pgfsys@transformshift{1.540386in}{1.611698in}%
\pgfsys@useobject{currentmarker}{}%
\end{pgfscope}%
\begin{pgfscope}%
\pgfsys@transformshift{1.553931in}{0.965113in}%
\pgfsys@useobject{currentmarker}{}%
\end{pgfscope}%
\begin{pgfscope}%
\pgfsys@transformshift{1.567476in}{1.574214in}%
\pgfsys@useobject{currentmarker}{}%
\end{pgfscope}%
\begin{pgfscope}%
\pgfsys@transformshift{1.581021in}{1.587870in}%
\pgfsys@useobject{currentmarker}{}%
\end{pgfscope}%
\begin{pgfscope}%
\pgfsys@transformshift{1.594566in}{1.571774in}%
\pgfsys@useobject{currentmarker}{}%
\end{pgfscope}%
\begin{pgfscope}%
\pgfsys@transformshift{1.608111in}{1.597230in}%
\pgfsys@useobject{currentmarker}{}%
\end{pgfscope}%
\begin{pgfscope}%
\pgfsys@transformshift{1.621656in}{1.558193in}%
\pgfsys@useobject{currentmarker}{}%
\end{pgfscope}%
\begin{pgfscope}%
\pgfsys@transformshift{1.635201in}{1.631253in}%
\pgfsys@useobject{currentmarker}{}%
\end{pgfscope}%
\begin{pgfscope}%
\pgfsys@transformshift{1.648746in}{1.538144in}%
\pgfsys@useobject{currentmarker}{}%
\end{pgfscope}%
\begin{pgfscope}%
\pgfsys@transformshift{1.662291in}{1.613878in}%
\pgfsys@useobject{currentmarker}{}%
\end{pgfscope}%
\begin{pgfscope}%
\pgfsys@transformshift{1.675836in}{1.528512in}%
\pgfsys@useobject{currentmarker}{}%
\end{pgfscope}%
\begin{pgfscope}%
\pgfsys@transformshift{1.689381in}{1.582191in}%
\pgfsys@useobject{currentmarker}{}%
\end{pgfscope}%
\begin{pgfscope}%
\pgfsys@transformshift{1.702926in}{1.536276in}%
\pgfsys@useobject{currentmarker}{}%
\end{pgfscope}%
\begin{pgfscope}%
\pgfsys@transformshift{1.716471in}{1.573720in}%
\pgfsys@useobject{currentmarker}{}%
\end{pgfscope}%
\begin{pgfscope}%
\pgfsys@transformshift{1.730016in}{1.528610in}%
\pgfsys@useobject{currentmarker}{}%
\end{pgfscope}%
\begin{pgfscope}%
\pgfsys@transformshift{1.743561in}{1.582754in}%
\pgfsys@useobject{currentmarker}{}%
\end{pgfscope}%
\begin{pgfscope}%
\pgfsys@transformshift{1.757105in}{1.522386in}%
\pgfsys@useobject{currentmarker}{}%
\end{pgfscope}%
\begin{pgfscope}%
\pgfsys@transformshift{1.770650in}{1.572759in}%
\pgfsys@useobject{currentmarker}{}%
\end{pgfscope}%
\begin{pgfscope}%
\pgfsys@transformshift{1.784195in}{1.516749in}%
\pgfsys@useobject{currentmarker}{}%
\end{pgfscope}%
\begin{pgfscope}%
\pgfsys@transformshift{1.797740in}{1.581786in}%
\pgfsys@useobject{currentmarker}{}%
\end{pgfscope}%
\begin{pgfscope}%
\pgfsys@transformshift{1.811285in}{1.530015in}%
\pgfsys@useobject{currentmarker}{}%
\end{pgfscope}%
\begin{pgfscope}%
\pgfsys@transformshift{1.824830in}{1.581973in}%
\pgfsys@useobject{currentmarker}{}%
\end{pgfscope}%
\begin{pgfscope}%
\pgfsys@transformshift{1.838375in}{1.544887in}%
\pgfsys@useobject{currentmarker}{}%
\end{pgfscope}%
\begin{pgfscope}%
\pgfsys@transformshift{1.851920in}{1.589494in}%
\pgfsys@useobject{currentmarker}{}%
\end{pgfscope}%
\begin{pgfscope}%
\pgfsys@transformshift{1.865465in}{1.549455in}%
\pgfsys@useobject{currentmarker}{}%
\end{pgfscope}%
\begin{pgfscope}%
\pgfsys@transformshift{1.879010in}{1.595795in}%
\pgfsys@useobject{currentmarker}{}%
\end{pgfscope}%
\begin{pgfscope}%
\pgfsys@transformshift{1.892555in}{1.549985in}%
\pgfsys@useobject{currentmarker}{}%
\end{pgfscope}%
\begin{pgfscope}%
\pgfsys@transformshift{1.906100in}{1.613149in}%
\pgfsys@useobject{currentmarker}{}%
\end{pgfscope}%
\begin{pgfscope}%
\pgfsys@transformshift{1.919645in}{1.581846in}%
\pgfsys@useobject{currentmarker}{}%
\end{pgfscope}%
\begin{pgfscope}%
\pgfsys@transformshift{1.933190in}{1.622239in}%
\pgfsys@useobject{currentmarker}{}%
\end{pgfscope}%
\begin{pgfscope}%
\pgfsys@transformshift{1.946735in}{1.584310in}%
\pgfsys@useobject{currentmarker}{}%
\end{pgfscope}%
\begin{pgfscope}%
\pgfsys@transformshift{1.960280in}{1.639947in}%
\pgfsys@useobject{currentmarker}{}%
\end{pgfscope}%
\begin{pgfscope}%
\pgfsys@transformshift{1.973825in}{1.603537in}%
\pgfsys@useobject{currentmarker}{}%
\end{pgfscope}%
\begin{pgfscope}%
\pgfsys@transformshift{1.987370in}{1.668504in}%
\pgfsys@useobject{currentmarker}{}%
\end{pgfscope}%
\begin{pgfscope}%
\pgfsys@transformshift{2.000915in}{1.630449in}%
\pgfsys@useobject{currentmarker}{}%
\end{pgfscope}%
\begin{pgfscope}%
\pgfsys@transformshift{2.014460in}{1.684916in}%
\pgfsys@useobject{currentmarker}{}%
\end{pgfscope}%
\begin{pgfscope}%
\pgfsys@transformshift{2.028005in}{1.636679in}%
\pgfsys@useobject{currentmarker}{}%
\end{pgfscope}%
\begin{pgfscope}%
\pgfsys@transformshift{2.041550in}{1.707248in}%
\pgfsys@useobject{currentmarker}{}%
\end{pgfscope}%
\begin{pgfscope}%
\pgfsys@transformshift{2.055095in}{1.664151in}%
\pgfsys@useobject{currentmarker}{}%
\end{pgfscope}%
\begin{pgfscope}%
\pgfsys@transformshift{2.068640in}{1.717975in}%
\pgfsys@useobject{currentmarker}{}%
\end{pgfscope}%
\begin{pgfscope}%
\pgfsys@transformshift{2.082185in}{1.682851in}%
\pgfsys@useobject{currentmarker}{}%
\end{pgfscope}%
\begin{pgfscope}%
\pgfsys@transformshift{2.095729in}{1.743500in}%
\pgfsys@useobject{currentmarker}{}%
\end{pgfscope}%
\begin{pgfscope}%
\pgfsys@transformshift{2.109274in}{1.693799in}%
\pgfsys@useobject{currentmarker}{}%
\end{pgfscope}%
\begin{pgfscope}%
\pgfsys@transformshift{2.122819in}{1.763667in}%
\pgfsys@useobject{currentmarker}{}%
\end{pgfscope}%
\begin{pgfscope}%
\pgfsys@transformshift{2.136364in}{1.737576in}%
\pgfsys@useobject{currentmarker}{}%
\end{pgfscope}%
\begin{pgfscope}%
\pgfsys@transformshift{2.149909in}{1.789499in}%
\pgfsys@useobject{currentmarker}{}%
\end{pgfscope}%
\begin{pgfscope}%
\pgfsys@transformshift{2.163454in}{1.744457in}%
\pgfsys@useobject{currentmarker}{}%
\end{pgfscope}%
\begin{pgfscope}%
\pgfsys@transformshift{2.176999in}{1.816744in}%
\pgfsys@useobject{currentmarker}{}%
\end{pgfscope}%
\begin{pgfscope}%
\pgfsys@transformshift{2.190544in}{1.751946in}%
\pgfsys@useobject{currentmarker}{}%
\end{pgfscope}%
\begin{pgfscope}%
\pgfsys@transformshift{2.204089in}{1.827108in}%
\pgfsys@useobject{currentmarker}{}%
\end{pgfscope}%
\begin{pgfscope}%
\pgfsys@transformshift{2.217634in}{1.778781in}%
\pgfsys@useobject{currentmarker}{}%
\end{pgfscope}%
\begin{pgfscope}%
\pgfsys@transformshift{2.231179in}{1.831463in}%
\pgfsys@useobject{currentmarker}{}%
\end{pgfscope}%
\begin{pgfscope}%
\pgfsys@transformshift{2.244724in}{1.791255in}%
\pgfsys@useobject{currentmarker}{}%
\end{pgfscope}%
\begin{pgfscope}%
\pgfsys@transformshift{2.258269in}{1.836510in}%
\pgfsys@useobject{currentmarker}{}%
\end{pgfscope}%
\begin{pgfscope}%
\pgfsys@transformshift{2.271814in}{1.785867in}%
\pgfsys@useobject{currentmarker}{}%
\end{pgfscope}%
\begin{pgfscope}%
\pgfsys@transformshift{2.285359in}{1.837525in}%
\pgfsys@useobject{currentmarker}{}%
\end{pgfscope}%
\begin{pgfscope}%
\pgfsys@transformshift{2.298904in}{1.784652in}%
\pgfsys@useobject{currentmarker}{}%
\end{pgfscope}%
\begin{pgfscope}%
\pgfsys@transformshift{2.312449in}{1.834751in}%
\pgfsys@useobject{currentmarker}{}%
\end{pgfscope}%
\begin{pgfscope}%
\pgfsys@transformshift{2.325994in}{1.769804in}%
\pgfsys@useobject{currentmarker}{}%
\end{pgfscope}%
\begin{pgfscope}%
\pgfsys@transformshift{2.339539in}{1.829227in}%
\pgfsys@useobject{currentmarker}{}%
\end{pgfscope}%
\begin{pgfscope}%
\pgfsys@transformshift{2.353084in}{1.780115in}%
\pgfsys@useobject{currentmarker}{}%
\end{pgfscope}%
\begin{pgfscope}%
\pgfsys@transformshift{2.366629in}{1.838407in}%
\pgfsys@useobject{currentmarker}{}%
\end{pgfscope}%
\begin{pgfscope}%
\pgfsys@transformshift{2.380174in}{1.778636in}%
\pgfsys@useobject{currentmarker}{}%
\end{pgfscope}%
\begin{pgfscope}%
\pgfsys@transformshift{2.393719in}{1.842827in}%
\pgfsys@useobject{currentmarker}{}%
\end{pgfscope}%
\begin{pgfscope}%
\pgfsys@transformshift{2.407264in}{1.775552in}%
\pgfsys@useobject{currentmarker}{}%
\end{pgfscope}%
\begin{pgfscope}%
\pgfsys@transformshift{2.420809in}{1.833059in}%
\pgfsys@useobject{currentmarker}{}%
\end{pgfscope}%
\begin{pgfscope}%
\pgfsys@transformshift{2.434353in}{1.775070in}%
\pgfsys@useobject{currentmarker}{}%
\end{pgfscope}%
\begin{pgfscope}%
\pgfsys@transformshift{2.447898in}{1.839118in}%
\pgfsys@useobject{currentmarker}{}%
\end{pgfscope}%
\begin{pgfscope}%
\pgfsys@transformshift{2.461443in}{1.774234in}%
\pgfsys@useobject{currentmarker}{}%
\end{pgfscope}%
\begin{pgfscope}%
\pgfsys@transformshift{2.474988in}{1.832349in}%
\pgfsys@useobject{currentmarker}{}%
\end{pgfscope}%
\begin{pgfscope}%
\pgfsys@transformshift{2.488533in}{1.783101in}%
\pgfsys@useobject{currentmarker}{}%
\end{pgfscope}%
\begin{pgfscope}%
\pgfsys@transformshift{2.502078in}{1.833188in}%
\pgfsys@useobject{currentmarker}{}%
\end{pgfscope}%
\begin{pgfscope}%
\pgfsys@transformshift{2.515623in}{1.100050in}%
\pgfsys@useobject{currentmarker}{}%
\end{pgfscope}%
\begin{pgfscope}%
\pgfsys@transformshift{2.529168in}{0.723263in}%
\pgfsys@useobject{currentmarker}{}%
\end{pgfscope}%
\begin{pgfscope}%
\pgfsys@transformshift{2.542713in}{1.225598in}%
\pgfsys@useobject{currentmarker}{}%
\end{pgfscope}%
\begin{pgfscope}%
\pgfsys@transformshift{2.556258in}{1.208862in}%
\pgfsys@useobject{currentmarker}{}%
\end{pgfscope}%
\begin{pgfscope}%
\pgfsys@transformshift{2.569803in}{1.226252in}%
\pgfsys@useobject{currentmarker}{}%
\end{pgfscope}%
\begin{pgfscope}%
\pgfsys@transformshift{2.583348in}{1.201620in}%
\pgfsys@useobject{currentmarker}{}%
\end{pgfscope}%
\begin{pgfscope}%
\pgfsys@transformshift{2.596893in}{1.239098in}%
\pgfsys@useobject{currentmarker}{}%
\end{pgfscope}%
\begin{pgfscope}%
\pgfsys@transformshift{2.610438in}{1.206330in}%
\pgfsys@useobject{currentmarker}{}%
\end{pgfscope}%
\begin{pgfscope}%
\pgfsys@transformshift{2.623983in}{1.258230in}%
\pgfsys@useobject{currentmarker}{}%
\end{pgfscope}%
\begin{pgfscope}%
\pgfsys@transformshift{2.637528in}{1.201835in}%
\pgfsys@useobject{currentmarker}{}%
\end{pgfscope}%
\begin{pgfscope}%
\pgfsys@transformshift{2.651073in}{1.266318in}%
\pgfsys@useobject{currentmarker}{}%
\end{pgfscope}%
\begin{pgfscope}%
\pgfsys@transformshift{2.664618in}{1.232674in}%
\pgfsys@useobject{currentmarker}{}%
\end{pgfscope}%
\begin{pgfscope}%
\pgfsys@transformshift{2.678163in}{1.273647in}%
\pgfsys@useobject{currentmarker}{}%
\end{pgfscope}%
\begin{pgfscope}%
\pgfsys@transformshift{2.691708in}{1.250316in}%
\pgfsys@useobject{currentmarker}{}%
\end{pgfscope}%
\begin{pgfscope}%
\pgfsys@transformshift{2.705253in}{1.286709in}%
\pgfsys@useobject{currentmarker}{}%
\end{pgfscope}%
\begin{pgfscope}%
\pgfsys@transformshift{2.718798in}{1.255020in}%
\pgfsys@useobject{currentmarker}{}%
\end{pgfscope}%
\begin{pgfscope}%
\pgfsys@transformshift{2.732343in}{1.302026in}%
\pgfsys@useobject{currentmarker}{}%
\end{pgfscope}%
\begin{pgfscope}%
\pgfsys@transformshift{2.745888in}{1.262602in}%
\pgfsys@useobject{currentmarker}{}%
\end{pgfscope}%
\begin{pgfscope}%
\pgfsys@transformshift{2.759433in}{1.320779in}%
\pgfsys@useobject{currentmarker}{}%
\end{pgfscope}%
\begin{pgfscope}%
\pgfsys@transformshift{2.772977in}{1.270234in}%
\pgfsys@useobject{currentmarker}{}%
\end{pgfscope}%
\begin{pgfscope}%
\pgfsys@transformshift{2.786522in}{1.335611in}%
\pgfsys@useobject{currentmarker}{}%
\end{pgfscope}%
\begin{pgfscope}%
\pgfsys@transformshift{2.800067in}{1.286688in}%
\pgfsys@useobject{currentmarker}{}%
\end{pgfscope}%
\begin{pgfscope}%
\pgfsys@transformshift{2.813612in}{1.331353in}%
\pgfsys@useobject{currentmarker}{}%
\end{pgfscope}%
\begin{pgfscope}%
\pgfsys@transformshift{2.827157in}{1.293383in}%
\pgfsys@useobject{currentmarker}{}%
\end{pgfscope}%
\begin{pgfscope}%
\pgfsys@transformshift{2.840702in}{1.344668in}%
\pgfsys@useobject{currentmarker}{}%
\end{pgfscope}%
\begin{pgfscope}%
\pgfsys@transformshift{2.854247in}{1.294737in}%
\pgfsys@useobject{currentmarker}{}%
\end{pgfscope}%
\begin{pgfscope}%
\pgfsys@transformshift{2.867792in}{1.353765in}%
\pgfsys@useobject{currentmarker}{}%
\end{pgfscope}%
\begin{pgfscope}%
\pgfsys@transformshift{2.881337in}{1.306975in}%
\pgfsys@useobject{currentmarker}{}%
\end{pgfscope}%
\begin{pgfscope}%
\pgfsys@transformshift{2.894882in}{1.362763in}%
\pgfsys@useobject{currentmarker}{}%
\end{pgfscope}%
\begin{pgfscope}%
\pgfsys@transformshift{2.908427in}{1.315944in}%
\pgfsys@useobject{currentmarker}{}%
\end{pgfscope}%
\begin{pgfscope}%
\pgfsys@transformshift{2.921972in}{1.365909in}%
\pgfsys@useobject{currentmarker}{}%
\end{pgfscope}%
\begin{pgfscope}%
\pgfsys@transformshift{2.935517in}{1.313827in}%
\pgfsys@useobject{currentmarker}{}%
\end{pgfscope}%
\begin{pgfscope}%
\pgfsys@transformshift{2.949062in}{1.379372in}%
\pgfsys@useobject{currentmarker}{}%
\end{pgfscope}%
\begin{pgfscope}%
\pgfsys@transformshift{2.962607in}{1.328416in}%
\pgfsys@useobject{currentmarker}{}%
\end{pgfscope}%
\begin{pgfscope}%
\pgfsys@transformshift{2.976152in}{1.385272in}%
\pgfsys@useobject{currentmarker}{}%
\end{pgfscope}%
\begin{pgfscope}%
\pgfsys@transformshift{2.989697in}{1.332639in}%
\pgfsys@useobject{currentmarker}{}%
\end{pgfscope}%
\begin{pgfscope}%
\pgfsys@transformshift{3.003242in}{1.408663in}%
\pgfsys@useobject{currentmarker}{}%
\end{pgfscope}%
\begin{pgfscope}%
\pgfsys@transformshift{3.016787in}{1.337398in}%
\pgfsys@useobject{currentmarker}{}%
\end{pgfscope}%
\begin{pgfscope}%
\pgfsys@transformshift{3.030332in}{1.402876in}%
\pgfsys@useobject{currentmarker}{}%
\end{pgfscope}%
\begin{pgfscope}%
\pgfsys@transformshift{3.043877in}{1.356199in}%
\pgfsys@useobject{currentmarker}{}%
\end{pgfscope}%
\begin{pgfscope}%
\pgfsys@transformshift{3.057422in}{1.407390in}%
\pgfsys@useobject{currentmarker}{}%
\end{pgfscope}%
\begin{pgfscope}%
\pgfsys@transformshift{3.070967in}{1.358436in}%
\pgfsys@useobject{currentmarker}{}%
\end{pgfscope}%
\begin{pgfscope}%
\pgfsys@transformshift{3.084512in}{1.434308in}%
\pgfsys@useobject{currentmarker}{}%
\end{pgfscope}%
\begin{pgfscope}%
\pgfsys@transformshift{3.098057in}{1.355545in}%
\pgfsys@useobject{currentmarker}{}%
\end{pgfscope}%
\begin{pgfscope}%
\pgfsys@transformshift{3.111601in}{1.435291in}%
\pgfsys@useobject{currentmarker}{}%
\end{pgfscope}%
\begin{pgfscope}%
\pgfsys@transformshift{3.125146in}{1.367167in}%
\pgfsys@useobject{currentmarker}{}%
\end{pgfscope}%
\begin{pgfscope}%
\pgfsys@transformshift{3.138691in}{1.428887in}%
\pgfsys@useobject{currentmarker}{}%
\end{pgfscope}%
\begin{pgfscope}%
\pgfsys@transformshift{3.152236in}{1.380451in}%
\pgfsys@useobject{currentmarker}{}%
\end{pgfscope}%
\begin{pgfscope}%
\pgfsys@transformshift{3.165781in}{1.431027in}%
\pgfsys@useobject{currentmarker}{}%
\end{pgfscope}%
\begin{pgfscope}%
\pgfsys@transformshift{3.179326in}{1.370666in}%
\pgfsys@useobject{currentmarker}{}%
\end{pgfscope}%
\begin{pgfscope}%
\pgfsys@transformshift{3.192871in}{1.429352in}%
\pgfsys@useobject{currentmarker}{}%
\end{pgfscope}%
\begin{pgfscope}%
\pgfsys@transformshift{3.206416in}{1.358168in}%
\pgfsys@useobject{currentmarker}{}%
\end{pgfscope}%
\begin{pgfscope}%
\pgfsys@transformshift{3.219961in}{1.421749in}%
\pgfsys@useobject{currentmarker}{}%
\end{pgfscope}%
\begin{pgfscope}%
\pgfsys@transformshift{3.233506in}{1.363656in}%
\pgfsys@useobject{currentmarker}{}%
\end{pgfscope}%
\begin{pgfscope}%
\pgfsys@transformshift{3.247051in}{1.415376in}%
\pgfsys@useobject{currentmarker}{}%
\end{pgfscope}%
\begin{pgfscope}%
\pgfsys@transformshift{3.260596in}{1.361289in}%
\pgfsys@useobject{currentmarker}{}%
\end{pgfscope}%
\begin{pgfscope}%
\pgfsys@transformshift{3.274141in}{1.419382in}%
\pgfsys@useobject{currentmarker}{}%
\end{pgfscope}%
\begin{pgfscope}%
\pgfsys@transformshift{3.287686in}{1.359883in}%
\pgfsys@useobject{currentmarker}{}%
\end{pgfscope}%
\begin{pgfscope}%
\pgfsys@transformshift{3.301231in}{1.429781in}%
\pgfsys@useobject{currentmarker}{}%
\end{pgfscope}%
\begin{pgfscope}%
\pgfsys@transformshift{3.314776in}{1.349317in}%
\pgfsys@useobject{currentmarker}{}%
\end{pgfscope}%
\begin{pgfscope}%
\pgfsys@transformshift{3.328321in}{1.404294in}%
\pgfsys@useobject{currentmarker}{}%
\end{pgfscope}%
\begin{pgfscope}%
\pgfsys@transformshift{3.341866in}{1.357917in}%
\pgfsys@useobject{currentmarker}{}%
\end{pgfscope}%
\begin{pgfscope}%
\pgfsys@transformshift{3.355411in}{1.407971in}%
\pgfsys@useobject{currentmarker}{}%
\end{pgfscope}%
\begin{pgfscope}%
\pgfsys@transformshift{3.368956in}{1.361857in}%
\pgfsys@useobject{currentmarker}{}%
\end{pgfscope}%
\begin{pgfscope}%
\pgfsys@transformshift{3.382501in}{1.425220in}%
\pgfsys@useobject{currentmarker}{}%
\end{pgfscope}%
\begin{pgfscope}%
\pgfsys@transformshift{3.396046in}{1.357252in}%
\pgfsys@useobject{currentmarker}{}%
\end{pgfscope}%
\begin{pgfscope}%
\pgfsys@transformshift{3.409591in}{1.418203in}%
\pgfsys@useobject{currentmarker}{}%
\end{pgfscope}%
\begin{pgfscope}%
\pgfsys@transformshift{3.423136in}{1.353738in}%
\pgfsys@useobject{currentmarker}{}%
\end{pgfscope}%
\begin{pgfscope}%
\pgfsys@transformshift{3.436681in}{1.412952in}%
\pgfsys@useobject{currentmarker}{}%
\end{pgfscope}%
\begin{pgfscope}%
\pgfsys@transformshift{3.450225in}{1.350527in}%
\pgfsys@useobject{currentmarker}{}%
\end{pgfscope}%
\begin{pgfscope}%
\pgfsys@transformshift{3.463770in}{1.398034in}%
\pgfsys@useobject{currentmarker}{}%
\end{pgfscope}%
\begin{pgfscope}%
\pgfsys@transformshift{3.477315in}{1.349975in}%
\pgfsys@useobject{currentmarker}{}%
\end{pgfscope}%
\begin{pgfscope}%
\pgfsys@transformshift{3.490860in}{1.406528in}%
\pgfsys@useobject{currentmarker}{}%
\end{pgfscope}%
\begin{pgfscope}%
\pgfsys@transformshift{3.504405in}{1.346335in}%
\pgfsys@useobject{currentmarker}{}%
\end{pgfscope}%
\begin{pgfscope}%
\pgfsys@transformshift{3.517950in}{1.416749in}%
\pgfsys@useobject{currentmarker}{}%
\end{pgfscope}%
\begin{pgfscope}%
\pgfsys@transformshift{3.531495in}{1.348942in}%
\pgfsys@useobject{currentmarker}{}%
\end{pgfscope}%
\begin{pgfscope}%
\pgfsys@transformshift{3.545040in}{1.406337in}%
\pgfsys@useobject{currentmarker}{}%
\end{pgfscope}%
\begin{pgfscope}%
\pgfsys@transformshift{3.558585in}{1.339630in}%
\pgfsys@useobject{currentmarker}{}%
\end{pgfscope}%
\begin{pgfscope}%
\pgfsys@transformshift{3.572130in}{1.390264in}%
\pgfsys@useobject{currentmarker}{}%
\end{pgfscope}%
\begin{pgfscope}%
\pgfsys@transformshift{3.585675in}{1.345201in}%
\pgfsys@useobject{currentmarker}{}%
\end{pgfscope}%
\begin{pgfscope}%
\pgfsys@transformshift{3.599220in}{1.393328in}%
\pgfsys@useobject{currentmarker}{}%
\end{pgfscope}%
\begin{pgfscope}%
\pgfsys@transformshift{3.612765in}{1.347233in}%
\pgfsys@useobject{currentmarker}{}%
\end{pgfscope}%
\begin{pgfscope}%
\pgfsys@transformshift{3.626310in}{1.423684in}%
\pgfsys@useobject{currentmarker}{}%
\end{pgfscope}%
\begin{pgfscope}%
\pgfsys@transformshift{3.639855in}{1.351515in}%
\pgfsys@useobject{currentmarker}{}%
\end{pgfscope}%
\begin{pgfscope}%
\pgfsys@transformshift{3.653400in}{1.426969in}%
\pgfsys@useobject{currentmarker}{}%
\end{pgfscope}%
\begin{pgfscope}%
\pgfsys@transformshift{3.666945in}{1.298843in}%
\pgfsys@useobject{currentmarker}{}%
\end{pgfscope}%
\begin{pgfscope}%
\pgfsys@transformshift{3.680490in}{1.318828in}%
\pgfsys@useobject{currentmarker}{}%
\end{pgfscope}%
\begin{pgfscope}%
\pgfsys@transformshift{3.694035in}{1.294532in}%
\pgfsys@useobject{currentmarker}{}%
\end{pgfscope}%
\begin{pgfscope}%
\pgfsys@transformshift{3.707580in}{1.326009in}%
\pgfsys@useobject{currentmarker}{}%
\end{pgfscope}%
\begin{pgfscope}%
\pgfsys@transformshift{3.721125in}{1.256877in}%
\pgfsys@useobject{currentmarker}{}%
\end{pgfscope}%
\begin{pgfscope}%
\pgfsys@transformshift{3.734670in}{1.320585in}%
\pgfsys@useobject{currentmarker}{}%
\end{pgfscope}%
\begin{pgfscope}%
\pgfsys@transformshift{3.748215in}{1.146163in}%
\pgfsys@useobject{currentmarker}{}%
\end{pgfscope}%
\begin{pgfscope}%
\pgfsys@transformshift{3.761760in}{1.252068in}%
\pgfsys@useobject{currentmarker}{}%
\end{pgfscope}%
\begin{pgfscope}%
\pgfsys@transformshift{3.775305in}{1.170492in}%
\pgfsys@useobject{currentmarker}{}%
\end{pgfscope}%
\begin{pgfscope}%
\pgfsys@transformshift{3.788849in}{1.221775in}%
\pgfsys@useobject{currentmarker}{}%
\end{pgfscope}%
\begin{pgfscope}%
\pgfsys@transformshift{3.802394in}{1.178580in}%
\pgfsys@useobject{currentmarker}{}%
\end{pgfscope}%
\begin{pgfscope}%
\pgfsys@transformshift{3.815939in}{1.221690in}%
\pgfsys@useobject{currentmarker}{}%
\end{pgfscope}%
\begin{pgfscope}%
\pgfsys@transformshift{3.829484in}{1.166056in}%
\pgfsys@useobject{currentmarker}{}%
\end{pgfscope}%
\begin{pgfscope}%
\pgfsys@transformshift{3.843029in}{1.225848in}%
\pgfsys@useobject{currentmarker}{}%
\end{pgfscope}%
\begin{pgfscope}%
\pgfsys@transformshift{3.856574in}{1.134600in}%
\pgfsys@useobject{currentmarker}{}%
\end{pgfscope}%
\begin{pgfscope}%
\pgfsys@transformshift{3.870119in}{1.215181in}%
\pgfsys@useobject{currentmarker}{}%
\end{pgfscope}%
\begin{pgfscope}%
\pgfsys@transformshift{3.883664in}{1.144770in}%
\pgfsys@useobject{currentmarker}{}%
\end{pgfscope}%
\begin{pgfscope}%
\pgfsys@transformshift{3.897209in}{1.211841in}%
\pgfsys@useobject{currentmarker}{}%
\end{pgfscope}%
\begin{pgfscope}%
\pgfsys@transformshift{3.910754in}{1.157102in}%
\pgfsys@useobject{currentmarker}{}%
\end{pgfscope}%
\begin{pgfscope}%
\pgfsys@transformshift{3.924299in}{1.213392in}%
\pgfsys@useobject{currentmarker}{}%
\end{pgfscope}%
\begin{pgfscope}%
\pgfsys@transformshift{3.937844in}{1.145158in}%
\pgfsys@useobject{currentmarker}{}%
\end{pgfscope}%
\begin{pgfscope}%
\pgfsys@transformshift{3.951389in}{1.199210in}%
\pgfsys@useobject{currentmarker}{}%
\end{pgfscope}%
\begin{pgfscope}%
\pgfsys@transformshift{3.964934in}{1.136406in}%
\pgfsys@useobject{currentmarker}{}%
\end{pgfscope}%
\begin{pgfscope}%
\pgfsys@transformshift{3.978479in}{1.206631in}%
\pgfsys@useobject{currentmarker}{}%
\end{pgfscope}%
\begin{pgfscope}%
\pgfsys@transformshift{3.992024in}{1.137347in}%
\pgfsys@useobject{currentmarker}{}%
\end{pgfscope}%
\begin{pgfscope}%
\pgfsys@transformshift{4.005569in}{1.200533in}%
\pgfsys@useobject{currentmarker}{}%
\end{pgfscope}%
\begin{pgfscope}%
\pgfsys@transformshift{4.019114in}{1.143976in}%
\pgfsys@useobject{currentmarker}{}%
\end{pgfscope}%
\begin{pgfscope}%
\pgfsys@transformshift{4.032659in}{1.827816in}%
\pgfsys@useobject{currentmarker}{}%
\end{pgfscope}%
\begin{pgfscope}%
\pgfsys@transformshift{4.046204in}{1.172575in}%
\pgfsys@useobject{currentmarker}{}%
\end{pgfscope}%
\begin{pgfscope}%
\pgfsys@transformshift{4.059749in}{1.154612in}%
\pgfsys@useobject{currentmarker}{}%
\end{pgfscope}%
\begin{pgfscope}%
\pgfsys@transformshift{4.073294in}{1.179111in}%
\pgfsys@useobject{currentmarker}{}%
\end{pgfscope}%
\begin{pgfscope}%
\pgfsys@transformshift{4.086839in}{1.134228in}%
\pgfsys@useobject{currentmarker}{}%
\end{pgfscope}%
\begin{pgfscope}%
\pgfsys@transformshift{4.100384in}{1.201062in}%
\pgfsys@useobject{currentmarker}{}%
\end{pgfscope}%
\begin{pgfscope}%
\pgfsys@transformshift{4.113929in}{1.145105in}%
\pgfsys@useobject{currentmarker}{}%
\end{pgfscope}%
\begin{pgfscope}%
\pgfsys@transformshift{4.127473in}{1.204499in}%
\pgfsys@useobject{currentmarker}{}%
\end{pgfscope}%
\begin{pgfscope}%
\pgfsys@transformshift{4.141018in}{1.144728in}%
\pgfsys@useobject{currentmarker}{}%
\end{pgfscope}%
\begin{pgfscope}%
\pgfsys@transformshift{4.154563in}{1.224312in}%
\pgfsys@useobject{currentmarker}{}%
\end{pgfscope}%
\begin{pgfscope}%
\pgfsys@transformshift{4.168108in}{1.151198in}%
\pgfsys@useobject{currentmarker}{}%
\end{pgfscope}%
\begin{pgfscope}%
\pgfsys@transformshift{4.181653in}{1.197852in}%
\pgfsys@useobject{currentmarker}{}%
\end{pgfscope}%
\begin{pgfscope}%
\pgfsys@transformshift{4.195198in}{1.155987in}%
\pgfsys@useobject{currentmarker}{}%
\end{pgfscope}%
\begin{pgfscope}%
\pgfsys@transformshift{4.208743in}{1.198814in}%
\pgfsys@useobject{currentmarker}{}%
\end{pgfscope}%
\begin{pgfscope}%
\pgfsys@transformshift{4.222288in}{1.161109in}%
\pgfsys@useobject{currentmarker}{}%
\end{pgfscope}%
\begin{pgfscope}%
\pgfsys@transformshift{4.235833in}{1.203911in}%
\pgfsys@useobject{currentmarker}{}%
\end{pgfscope}%
\begin{pgfscope}%
\pgfsys@transformshift{4.249378in}{1.154261in}%
\pgfsys@useobject{currentmarker}{}%
\end{pgfscope}%
\begin{pgfscope}%
\pgfsys@transformshift{4.262923in}{1.206142in}%
\pgfsys@useobject{currentmarker}{}%
\end{pgfscope}%
\begin{pgfscope}%
\pgfsys@transformshift{4.276468in}{1.148709in}%
\pgfsys@useobject{currentmarker}{}%
\end{pgfscope}%
\begin{pgfscope}%
\pgfsys@transformshift{4.290013in}{1.209998in}%
\pgfsys@useobject{currentmarker}{}%
\end{pgfscope}%
\begin{pgfscope}%
\pgfsys@transformshift{4.303558in}{1.141339in}%
\pgfsys@useobject{currentmarker}{}%
\end{pgfscope}%
\begin{pgfscope}%
\pgfsys@transformshift{4.317103in}{1.198399in}%
\pgfsys@useobject{currentmarker}{}%
\end{pgfscope}%
\begin{pgfscope}%
\pgfsys@transformshift{4.330648in}{1.135688in}%
\pgfsys@useobject{currentmarker}{}%
\end{pgfscope}%
\begin{pgfscope}%
\pgfsys@transformshift{4.344193in}{1.171023in}%
\pgfsys@useobject{currentmarker}{}%
\end{pgfscope}%
\begin{pgfscope}%
\pgfsys@transformshift{4.357738in}{1.126976in}%
\pgfsys@useobject{currentmarker}{}%
\end{pgfscope}%
\begin{pgfscope}%
\pgfsys@transformshift{4.371283in}{1.179110in}%
\pgfsys@useobject{currentmarker}{}%
\end{pgfscope}%
\begin{pgfscope}%
\pgfsys@transformshift{4.384828in}{1.126119in}%
\pgfsys@useobject{currentmarker}{}%
\end{pgfscope}%
\begin{pgfscope}%
\pgfsys@transformshift{4.398373in}{1.183578in}%
\pgfsys@useobject{currentmarker}{}%
\end{pgfscope}%
\begin{pgfscope}%
\pgfsys@transformshift{4.411918in}{1.121344in}%
\pgfsys@useobject{currentmarker}{}%
\end{pgfscope}%
\begin{pgfscope}%
\pgfsys@transformshift{4.425463in}{1.181002in}%
\pgfsys@useobject{currentmarker}{}%
\end{pgfscope}%
\begin{pgfscope}%
\pgfsys@transformshift{4.439008in}{1.131768in}%
\pgfsys@useobject{currentmarker}{}%
\end{pgfscope}%
\begin{pgfscope}%
\pgfsys@transformshift{4.452553in}{1.169614in}%
\pgfsys@useobject{currentmarker}{}%
\end{pgfscope}%
\begin{pgfscope}%
\pgfsys@transformshift{4.466097in}{1.122207in}%
\pgfsys@useobject{currentmarker}{}%
\end{pgfscope}%
\begin{pgfscope}%
\pgfsys@transformshift{4.479642in}{1.183552in}%
\pgfsys@useobject{currentmarker}{}%
\end{pgfscope}%
\begin{pgfscope}%
\pgfsys@transformshift{4.493187in}{1.133697in}%
\pgfsys@useobject{currentmarker}{}%
\end{pgfscope}%
\begin{pgfscope}%
\pgfsys@transformshift{4.506732in}{1.182260in}%
\pgfsys@useobject{currentmarker}{}%
\end{pgfscope}%
\begin{pgfscope}%
\pgfsys@transformshift{4.520277in}{1.124108in}%
\pgfsys@useobject{currentmarker}{}%
\end{pgfscope}%
\begin{pgfscope}%
\pgfsys@transformshift{4.533822in}{1.191108in}%
\pgfsys@useobject{currentmarker}{}%
\end{pgfscope}%
\begin{pgfscope}%
\pgfsys@transformshift{4.547367in}{1.133448in}%
\pgfsys@useobject{currentmarker}{}%
\end{pgfscope}%
\begin{pgfscope}%
\pgfsys@transformshift{4.560912in}{1.183079in}%
\pgfsys@useobject{currentmarker}{}%
\end{pgfscope}%
\begin{pgfscope}%
\pgfsys@transformshift{4.574457in}{1.150012in}%
\pgfsys@useobject{currentmarker}{}%
\end{pgfscope}%
\begin{pgfscope}%
\pgfsys@transformshift{4.588002in}{1.197524in}%
\pgfsys@useobject{currentmarker}{}%
\end{pgfscope}%
\begin{pgfscope}%
\pgfsys@transformshift{4.601547in}{1.146726in}%
\pgfsys@useobject{currentmarker}{}%
\end{pgfscope}%
\begin{pgfscope}%
\pgfsys@transformshift{4.615092in}{1.196916in}%
\pgfsys@useobject{currentmarker}{}%
\end{pgfscope}%
\begin{pgfscope}%
\pgfsys@transformshift{4.628637in}{1.149804in}%
\pgfsys@useobject{currentmarker}{}%
\end{pgfscope}%
\begin{pgfscope}%
\pgfsys@transformshift{4.642182in}{1.222468in}%
\pgfsys@useobject{currentmarker}{}%
\end{pgfscope}%
\begin{pgfscope}%
\pgfsys@transformshift{4.655727in}{1.151109in}%
\pgfsys@useobject{currentmarker}{}%
\end{pgfscope}%
\end{pgfscope}%
\begin{pgfscope}%
\pgfpathrectangle{\pgfqpoint{0.605784in}{0.382904in}}{\pgfqpoint{4.063488in}{2.042155in}}%
\pgfusepath{clip}%
\pgfsetbuttcap%
\pgfsetroundjoin%
\definecolor{currentfill}{rgb}{1.000000,1.000000,1.000000}%
\pgfsetfillcolor{currentfill}%
\pgfsetlinewidth{1.003750pt}%
\definecolor{currentstroke}{rgb}{1.000000,1.000000,1.000000}%
\pgfsetstrokecolor{currentstroke}%
\pgfsetdash{}{0pt}%
\pgfsys@defobject{currentmarker}{\pgfqpoint{-0.020833in}{-0.020833in}}{\pgfqpoint{0.020833in}{0.020833in}}{%
\pgfpathmoveto{\pgfqpoint{0.000000in}{-0.020833in}}%
\pgfpathcurveto{\pgfqpoint{0.005525in}{-0.020833in}}{\pgfqpoint{0.010825in}{-0.018638in}}{\pgfqpoint{0.014731in}{-0.014731in}}%
\pgfpathcurveto{\pgfqpoint{0.018638in}{-0.010825in}}{\pgfqpoint{0.020833in}{-0.005525in}}{\pgfqpoint{0.020833in}{0.000000in}}%
\pgfpathcurveto{\pgfqpoint{0.020833in}{0.005525in}}{\pgfqpoint{0.018638in}{0.010825in}}{\pgfqpoint{0.014731in}{0.014731in}}%
\pgfpathcurveto{\pgfqpoint{0.010825in}{0.018638in}}{\pgfqpoint{0.005525in}{0.020833in}}{\pgfqpoint{0.000000in}{0.020833in}}%
\pgfpathcurveto{\pgfqpoint{-0.005525in}{0.020833in}}{\pgfqpoint{-0.010825in}{0.018638in}}{\pgfqpoint{-0.014731in}{0.014731in}}%
\pgfpathcurveto{\pgfqpoint{-0.018638in}{0.010825in}}{\pgfqpoint{-0.020833in}{0.005525in}}{\pgfqpoint{-0.020833in}{0.000000in}}%
\pgfpathcurveto{\pgfqpoint{-0.020833in}{-0.005525in}}{\pgfqpoint{-0.018638in}{-0.010825in}}{\pgfqpoint{-0.014731in}{-0.014731in}}%
\pgfpathcurveto{\pgfqpoint{-0.010825in}{-0.018638in}}{\pgfqpoint{-0.005525in}{-0.020833in}}{\pgfqpoint{0.000000in}{-0.020833in}}%
\pgfpathclose%
\pgfusepath{stroke,fill}%
}%
\begin{pgfscope}%
\pgfsys@transformshift{0.605784in}{1.865755in}%
\pgfsys@useobject{currentmarker}{}%
\end{pgfscope}%
\begin{pgfscope}%
\pgfsys@transformshift{0.619329in}{0.792927in}%
\pgfsys@useobject{currentmarker}{}%
\end{pgfscope}%
\begin{pgfscope}%
\pgfsys@transformshift{0.632874in}{0.806860in}%
\pgfsys@useobject{currentmarker}{}%
\end{pgfscope}%
\begin{pgfscope}%
\pgfsys@transformshift{0.646419in}{1.308442in}%
\pgfsys@useobject{currentmarker}{}%
\end{pgfscope}%
\begin{pgfscope}%
\pgfsys@transformshift{0.659964in}{2.102613in}%
\pgfsys@useobject{currentmarker}{}%
\end{pgfscope}%
\begin{pgfscope}%
\pgfsys@transformshift{0.673509in}{2.130479in}%
\pgfsys@useobject{currentmarker}{}%
\end{pgfscope}%
\begin{pgfscope}%
\pgfsys@transformshift{0.687054in}{1.280576in}%
\pgfsys@useobject{currentmarker}{}%
\end{pgfscope}%
\begin{pgfscope}%
\pgfsys@transformshift{0.700599in}{1.071584in}%
\pgfsys@useobject{currentmarker}{}%
\end{pgfscope}%
\begin{pgfscope}%
\pgfsys@transformshift{0.714144in}{2.186210in}%
\pgfsys@useobject{currentmarker}{}%
\end{pgfscope}%
\begin{pgfscope}%
\pgfsys@transformshift{0.727689in}{2.144412in}%
\pgfsys@useobject{currentmarker}{}%
\end{pgfscope}%
\begin{pgfscope}%
\pgfsys@transformshift{0.741233in}{1.893621in}%
\pgfsys@useobject{currentmarker}{}%
\end{pgfscope}%
\begin{pgfscope}%
\pgfsys@transformshift{0.754778in}{2.074748in}%
\pgfsys@useobject{currentmarker}{}%
\end{pgfscope}%
\begin{pgfscope}%
\pgfsys@transformshift{0.768323in}{2.283740in}%
\pgfsys@useobject{currentmarker}{}%
\end{pgfscope}%
\begin{pgfscope}%
\pgfsys@transformshift{0.781868in}{2.005083in}%
\pgfsys@useobject{currentmarker}{}%
\end{pgfscope}%
\begin{pgfscope}%
\pgfsys@transformshift{0.795413in}{1.043718in}%
\pgfsys@useobject{currentmarker}{}%
\end{pgfscope}%
\end{pgfscope}%
\begin{pgfscope}%
\pgfpathrectangle{\pgfqpoint{0.605784in}{0.382904in}}{\pgfqpoint{4.063488in}{2.042155in}}%
\pgfusepath{clip}%
\pgfsetbuttcap%
\pgfsetroundjoin%
\definecolor{currentfill}{rgb}{1.000000,0.000000,0.000000}%
\pgfsetfillcolor{currentfill}%
\pgfsetlinewidth{1.003750pt}%
\definecolor{currentstroke}{rgb}{1.000000,0.000000,0.000000}%
\pgfsetstrokecolor{currentstroke}%
\pgfsetdash{}{0pt}%
\pgfsys@defobject{currentmarker}{\pgfqpoint{-0.020833in}{-0.020833in}}{\pgfqpoint{0.020833in}{0.020833in}}{%
\pgfpathmoveto{\pgfqpoint{-0.020833in}{-0.020833in}}%
\pgfpathlineto{\pgfqpoint{0.020833in}{0.020833in}}%
\pgfpathmoveto{\pgfqpoint{-0.020833in}{0.020833in}}%
\pgfpathlineto{\pgfqpoint{0.020833in}{-0.020833in}}%
\pgfusepath{stroke,fill}%
}%
\begin{pgfscope}%
\pgfsys@transformshift{0.808958in}{1.232403in}%
\pgfsys@useobject{currentmarker}{}%
\end{pgfscope}%
\begin{pgfscope}%
\pgfsys@transformshift{0.822503in}{1.220346in}%
\pgfsys@useobject{currentmarker}{}%
\end{pgfscope}%
\begin{pgfscope}%
\pgfsys@transformshift{0.836048in}{1.221594in}%
\pgfsys@useobject{currentmarker}{}%
\end{pgfscope}%
\begin{pgfscope}%
\pgfsys@transformshift{0.849593in}{1.223446in}%
\pgfsys@useobject{currentmarker}{}%
\end{pgfscope}%
\begin{pgfscope}%
\pgfsys@transformshift{0.863138in}{1.225306in}%
\pgfsys@useobject{currentmarker}{}%
\end{pgfscope}%
\begin{pgfscope}%
\pgfsys@transformshift{0.876683in}{1.228725in}%
\pgfsys@useobject{currentmarker}{}%
\end{pgfscope}%
\begin{pgfscope}%
\pgfsys@transformshift{0.890228in}{1.226995in}%
\pgfsys@useobject{currentmarker}{}%
\end{pgfscope}%
\begin{pgfscope}%
\pgfsys@transformshift{0.903773in}{1.225257in}%
\pgfsys@useobject{currentmarker}{}%
\end{pgfscope}%
\begin{pgfscope}%
\pgfsys@transformshift{0.917318in}{1.237850in}%
\pgfsys@useobject{currentmarker}{}%
\end{pgfscope}%
\begin{pgfscope}%
\pgfsys@transformshift{0.930863in}{1.270307in}%
\pgfsys@useobject{currentmarker}{}%
\end{pgfscope}%
\begin{pgfscope}%
\pgfsys@transformshift{0.944408in}{1.271278in}%
\pgfsys@useobject{currentmarker}{}%
\end{pgfscope}%
\begin{pgfscope}%
\pgfsys@transformshift{0.957953in}{1.281025in}%
\pgfsys@useobject{currentmarker}{}%
\end{pgfscope}%
\begin{pgfscope}%
\pgfsys@transformshift{0.971498in}{1.285751in}%
\pgfsys@useobject{currentmarker}{}%
\end{pgfscope}%
\begin{pgfscope}%
\pgfsys@transformshift{0.985043in}{1.301458in}%
\pgfsys@useobject{currentmarker}{}%
\end{pgfscope}%
\begin{pgfscope}%
\pgfsys@transformshift{0.998588in}{1.308659in}%
\pgfsys@useobject{currentmarker}{}%
\end{pgfscope}%
\begin{pgfscope}%
\pgfsys@transformshift{1.012133in}{1.334090in}%
\pgfsys@useobject{currentmarker}{}%
\end{pgfscope}%
\begin{pgfscope}%
\pgfsys@transformshift{1.025678in}{1.336761in}%
\pgfsys@useobject{currentmarker}{}%
\end{pgfscope}%
\begin{pgfscope}%
\pgfsys@transformshift{1.039223in}{1.348264in}%
\pgfsys@useobject{currentmarker}{}%
\end{pgfscope}%
\begin{pgfscope}%
\pgfsys@transformshift{1.052768in}{1.358503in}%
\pgfsys@useobject{currentmarker}{}%
\end{pgfscope}%
\begin{pgfscope}%
\pgfsys@transformshift{1.066313in}{1.383422in}%
\pgfsys@useobject{currentmarker}{}%
\end{pgfscope}%
\begin{pgfscope}%
\pgfsys@transformshift{1.079857in}{1.398105in}%
\pgfsys@useobject{currentmarker}{}%
\end{pgfscope}%
\begin{pgfscope}%
\pgfsys@transformshift{1.093402in}{1.419899in}%
\pgfsys@useobject{currentmarker}{}%
\end{pgfscope}%
\begin{pgfscope}%
\pgfsys@transformshift{1.106947in}{1.425017in}%
\pgfsys@useobject{currentmarker}{}%
\end{pgfscope}%
\begin{pgfscope}%
\pgfsys@transformshift{1.120492in}{1.432545in}%
\pgfsys@useobject{currentmarker}{}%
\end{pgfscope}%
\begin{pgfscope}%
\pgfsys@transformshift{1.134037in}{1.457607in}%
\pgfsys@useobject{currentmarker}{}%
\end{pgfscope}%
\begin{pgfscope}%
\pgfsys@transformshift{1.147582in}{1.489946in}%
\pgfsys@useobject{currentmarker}{}%
\end{pgfscope}%
\begin{pgfscope}%
\pgfsys@transformshift{1.161127in}{1.499185in}%
\pgfsys@useobject{currentmarker}{}%
\end{pgfscope}%
\begin{pgfscope}%
\pgfsys@transformshift{1.174672in}{1.514906in}%
\pgfsys@useobject{currentmarker}{}%
\end{pgfscope}%
\begin{pgfscope}%
\pgfsys@transformshift{1.188217in}{1.520758in}%
\pgfsys@useobject{currentmarker}{}%
\end{pgfscope}%
\begin{pgfscope}%
\pgfsys@transformshift{1.201762in}{1.521561in}%
\pgfsys@useobject{currentmarker}{}%
\end{pgfscope}%
\begin{pgfscope}%
\pgfsys@transformshift{1.215307in}{1.532735in}%
\pgfsys@useobject{currentmarker}{}%
\end{pgfscope}%
\begin{pgfscope}%
\pgfsys@transformshift{1.228852in}{1.545561in}%
\pgfsys@useobject{currentmarker}{}%
\end{pgfscope}%
\begin{pgfscope}%
\pgfsys@transformshift{1.242397in}{1.549086in}%
\pgfsys@useobject{currentmarker}{}%
\end{pgfscope}%
\begin{pgfscope}%
\pgfsys@transformshift{1.255942in}{1.553978in}%
\pgfsys@useobject{currentmarker}{}%
\end{pgfscope}%
\begin{pgfscope}%
\pgfsys@transformshift{1.269487in}{1.555400in}%
\pgfsys@useobject{currentmarker}{}%
\end{pgfscope}%
\begin{pgfscope}%
\pgfsys@transformshift{1.283032in}{1.558092in}%
\pgfsys@useobject{currentmarker}{}%
\end{pgfscope}%
\begin{pgfscope}%
\pgfsys@transformshift{1.296577in}{1.563263in}%
\pgfsys@useobject{currentmarker}{}%
\end{pgfscope}%
\begin{pgfscope}%
\pgfsys@transformshift{1.310122in}{1.572708in}%
\pgfsys@useobject{currentmarker}{}%
\end{pgfscope}%
\begin{pgfscope}%
\pgfsys@transformshift{1.323667in}{1.571600in}%
\pgfsys@useobject{currentmarker}{}%
\end{pgfscope}%
\begin{pgfscope}%
\pgfsys@transformshift{1.337212in}{1.577240in}%
\pgfsys@useobject{currentmarker}{}%
\end{pgfscope}%
\begin{pgfscope}%
\pgfsys@transformshift{1.350757in}{1.578685in}%
\pgfsys@useobject{currentmarker}{}%
\end{pgfscope}%
\begin{pgfscope}%
\pgfsys@transformshift{1.364302in}{1.580407in}%
\pgfsys@useobject{currentmarker}{}%
\end{pgfscope}%
\begin{pgfscope}%
\pgfsys@transformshift{1.377847in}{1.586185in}%
\pgfsys@useobject{currentmarker}{}%
\end{pgfscope}%
\begin{pgfscope}%
\pgfsys@transformshift{1.391392in}{1.588553in}%
\pgfsys@useobject{currentmarker}{}%
\end{pgfscope}%
\begin{pgfscope}%
\pgfsys@transformshift{1.404937in}{1.585471in}%
\pgfsys@useobject{currentmarker}{}%
\end{pgfscope}%
\begin{pgfscope}%
\pgfsys@transformshift{1.418481in}{1.585357in}%
\pgfsys@useobject{currentmarker}{}%
\end{pgfscope}%
\begin{pgfscope}%
\pgfsys@transformshift{1.432026in}{1.589128in}%
\pgfsys@useobject{currentmarker}{}%
\end{pgfscope}%
\begin{pgfscope}%
\pgfsys@transformshift{1.445571in}{1.583225in}%
\pgfsys@useobject{currentmarker}{}%
\end{pgfscope}%
\begin{pgfscope}%
\pgfsys@transformshift{1.459116in}{1.581787in}%
\pgfsys@useobject{currentmarker}{}%
\end{pgfscope}%
\begin{pgfscope}%
\pgfsys@transformshift{1.472661in}{1.578808in}%
\pgfsys@useobject{currentmarker}{}%
\end{pgfscope}%
\begin{pgfscope}%
\pgfsys@transformshift{1.486206in}{1.586770in}%
\pgfsys@useobject{currentmarker}{}%
\end{pgfscope}%
\begin{pgfscope}%
\pgfsys@transformshift{1.499751in}{1.577806in}%
\pgfsys@useobject{currentmarker}{}%
\end{pgfscope}%
\begin{pgfscope}%
\pgfsys@transformshift{1.513296in}{1.581815in}%
\pgfsys@useobject{currentmarker}{}%
\end{pgfscope}%
\begin{pgfscope}%
\pgfsys@transformshift{1.526841in}{1.572742in}%
\pgfsys@useobject{currentmarker}{}%
\end{pgfscope}%
\begin{pgfscope}%
\pgfsys@transformshift{1.540386in}{1.578705in}%
\pgfsys@useobject{currentmarker}{}%
\end{pgfscope}%
\begin{pgfscope}%
\pgfsys@transformshift{1.553931in}{1.562310in}%
\pgfsys@useobject{currentmarker}{}%
\end{pgfscope}%
\begin{pgfscope}%
\pgfsys@transformshift{1.567476in}{1.581222in}%
\pgfsys@useobject{currentmarker}{}%
\end{pgfscope}%
\begin{pgfscope}%
\pgfsys@transformshift{1.581021in}{1.581680in}%
\pgfsys@useobject{currentmarker}{}%
\end{pgfscope}%
\begin{pgfscope}%
\pgfsys@transformshift{1.594566in}{1.580022in}%
\pgfsys@useobject{currentmarker}{}%
\end{pgfscope}%
\begin{pgfscope}%
\pgfsys@transformshift{1.608111in}{1.582656in}%
\pgfsys@useobject{currentmarker}{}%
\end{pgfscope}%
\begin{pgfscope}%
\pgfsys@transformshift{1.621656in}{1.581914in}%
\pgfsys@useobject{currentmarker}{}%
\end{pgfscope}%
\begin{pgfscope}%
\pgfsys@transformshift{1.635201in}{1.588231in}%
\pgfsys@useobject{currentmarker}{}%
\end{pgfscope}%
\begin{pgfscope}%
\pgfsys@transformshift{1.648746in}{1.585942in}%
\pgfsys@useobject{currentmarker}{}%
\end{pgfscope}%
\begin{pgfscope}%
\pgfsys@transformshift{1.662291in}{1.580795in}%
\pgfsys@useobject{currentmarker}{}%
\end{pgfscope}%
\begin{pgfscope}%
\pgfsys@transformshift{1.675836in}{1.568180in}%
\pgfsys@useobject{currentmarker}{}%
\end{pgfscope}%
\begin{pgfscope}%
\pgfsys@transformshift{1.689381in}{1.559291in}%
\pgfsys@useobject{currentmarker}{}%
\end{pgfscope}%
\begin{pgfscope}%
\pgfsys@transformshift{1.702926in}{1.558770in}%
\pgfsys@useobject{currentmarker}{}%
\end{pgfscope}%
\begin{pgfscope}%
\pgfsys@transformshift{1.716471in}{1.555285in}%
\pgfsys@useobject{currentmarker}{}%
\end{pgfscope}%
\begin{pgfscope}%
\pgfsys@transformshift{1.730016in}{1.554188in}%
\pgfsys@useobject{currentmarker}{}%
\end{pgfscope}%
\begin{pgfscope}%
\pgfsys@transformshift{1.743561in}{1.553583in}%
\pgfsys@useobject{currentmarker}{}%
\end{pgfscope}%
\begin{pgfscope}%
\pgfsys@transformshift{1.757105in}{1.557114in}%
\pgfsys@useobject{currentmarker}{}%
\end{pgfscope}%
\begin{pgfscope}%
\pgfsys@transformshift{1.770650in}{1.545970in}%
\pgfsys@useobject{currentmarker}{}%
\end{pgfscope}%
\begin{pgfscope}%
\pgfsys@transformshift{1.784195in}{1.549261in}%
\pgfsys@useobject{currentmarker}{}%
\end{pgfscope}%
\begin{pgfscope}%
\pgfsys@transformshift{1.797740in}{1.547908in}%
\pgfsys@useobject{currentmarker}{}%
\end{pgfscope}%
\begin{pgfscope}%
\pgfsys@transformshift{1.811285in}{1.557490in}%
\pgfsys@useobject{currentmarker}{}%
\end{pgfscope}%
\begin{pgfscope}%
\pgfsys@transformshift{1.824830in}{1.555102in}%
\pgfsys@useobject{currentmarker}{}%
\end{pgfscope}%
\begin{pgfscope}%
\pgfsys@transformshift{1.838375in}{1.564654in}%
\pgfsys@useobject{currentmarker}{}%
\end{pgfscope}%
\begin{pgfscope}%
\pgfsys@transformshift{1.851920in}{1.565503in}%
\pgfsys@useobject{currentmarker}{}%
\end{pgfscope}%
\begin{pgfscope}%
\pgfsys@transformshift{1.865465in}{1.572429in}%
\pgfsys@useobject{currentmarker}{}%
\end{pgfscope}%
\begin{pgfscope}%
\pgfsys@transformshift{1.879010in}{1.570525in}%
\pgfsys@useobject{currentmarker}{}%
\end{pgfscope}%
\begin{pgfscope}%
\pgfsys@transformshift{1.892555in}{1.575783in}%
\pgfsys@useobject{currentmarker}{}%
\end{pgfscope}%
\begin{pgfscope}%
\pgfsys@transformshift{1.906100in}{1.579381in}%
\pgfsys@useobject{currentmarker}{}%
\end{pgfscope}%
\begin{pgfscope}%
\pgfsys@transformshift{1.919645in}{1.598246in}%
\pgfsys@useobject{currentmarker}{}%
\end{pgfscope}%
\begin{pgfscope}%
\pgfsys@transformshift{1.933190in}{1.600060in}%
\pgfsys@useobject{currentmarker}{}%
\end{pgfscope}%
\begin{pgfscope}%
\pgfsys@transformshift{1.946735in}{1.605490in}%
\pgfsys@useobject{currentmarker}{}%
\end{pgfscope}%
\begin{pgfscope}%
\pgfsys@transformshift{1.960280in}{1.608840in}%
\pgfsys@useobject{currentmarker}{}%
\end{pgfscope}%
\begin{pgfscope}%
\pgfsys@transformshift{1.973825in}{1.623224in}%
\pgfsys@useobject{currentmarker}{}%
\end{pgfscope}%
\begin{pgfscope}%
\pgfsys@transformshift{1.987370in}{1.631782in}%
\pgfsys@useobject{currentmarker}{}%
\end{pgfscope}%
\begin{pgfscope}%
\pgfsys@transformshift{2.000915in}{1.649705in}%
\pgfsys@useobject{currentmarker}{}%
\end{pgfscope}%
\begin{pgfscope}%
\pgfsys@transformshift{2.014460in}{1.654469in}%
\pgfsys@useobject{currentmarker}{}%
\end{pgfscope}%
\begin{pgfscope}%
\pgfsys@transformshift{2.028005in}{1.661938in}%
\pgfsys@useobject{currentmarker}{}%
\end{pgfscope}%
\begin{pgfscope}%
\pgfsys@transformshift{2.041550in}{1.669462in}%
\pgfsys@useobject{currentmarker}{}%
\end{pgfscope}%
\begin{pgfscope}%
\pgfsys@transformshift{2.055095in}{1.685431in}%
\pgfsys@useobject{currentmarker}{}%
\end{pgfscope}%
\begin{pgfscope}%
\pgfsys@transformshift{2.068640in}{1.689177in}%
\pgfsys@useobject{currentmarker}{}%
\end{pgfscope}%
\begin{pgfscope}%
\pgfsys@transformshift{2.082185in}{1.700914in}%
\pgfsys@useobject{currentmarker}{}%
\end{pgfscope}%
\begin{pgfscope}%
\pgfsys@transformshift{2.095729in}{1.708595in}%
\pgfsys@useobject{currentmarker}{}%
\end{pgfscope}%
\begin{pgfscope}%
\pgfsys@transformshift{2.109274in}{1.719106in}%
\pgfsys@useobject{currentmarker}{}%
\end{pgfscope}%
\begin{pgfscope}%
\pgfsys@transformshift{2.122819in}{1.726346in}%
\pgfsys@useobject{currentmarker}{}%
\end{pgfscope}%
\begin{pgfscope}%
\pgfsys@transformshift{2.136364in}{1.751138in}%
\pgfsys@useobject{currentmarker}{}%
\end{pgfscope}%
\begin{pgfscope}%
\pgfsys@transformshift{2.149909in}{1.757574in}%
\pgfsys@useobject{currentmarker}{}%
\end{pgfscope}%
\begin{pgfscope}%
\pgfsys@transformshift{2.163454in}{1.769396in}%
\pgfsys@useobject{currentmarker}{}%
\end{pgfscope}%
\begin{pgfscope}%
\pgfsys@transformshift{2.176999in}{1.774575in}%
\pgfsys@useobject{currentmarker}{}%
\end{pgfscope}%
\begin{pgfscope}%
\pgfsys@transformshift{2.190544in}{1.782461in}%
\pgfsys@useobject{currentmarker}{}%
\end{pgfscope}%
\begin{pgfscope}%
\pgfsys@transformshift{2.204089in}{1.791419in}%
\pgfsys@useobject{currentmarker}{}%
\end{pgfscope}%
\begin{pgfscope}%
\pgfsys@transformshift{2.217634in}{1.799650in}%
\pgfsys@useobject{currentmarker}{}%
\end{pgfscope}%
\begin{pgfscope}%
\pgfsys@transformshift{2.231179in}{1.806399in}%
\pgfsys@useobject{currentmarker}{}%
\end{pgfscope}%
\begin{pgfscope}%
\pgfsys@transformshift{2.244724in}{1.809741in}%
\pgfsys@useobject{currentmarker}{}%
\end{pgfscope}%
\begin{pgfscope}%
\pgfsys@transformshift{2.258269in}{1.813773in}%
\pgfsys@useobject{currentmarker}{}%
\end{pgfscope}%
\begin{pgfscope}%
\pgfsys@transformshift{2.271814in}{1.809823in}%
\pgfsys@useobject{currentmarker}{}%
\end{pgfscope}%
\begin{pgfscope}%
\pgfsys@transformshift{2.285359in}{1.812250in}%
\pgfsys@useobject{currentmarker}{}%
\end{pgfscope}%
\begin{pgfscope}%
\pgfsys@transformshift{2.298904in}{1.808623in}%
\pgfsys@useobject{currentmarker}{}%
\end{pgfscope}%
\begin{pgfscope}%
\pgfsys@transformshift{2.312449in}{1.811560in}%
\pgfsys@useobject{currentmarker}{}%
\end{pgfscope}%
\begin{pgfscope}%
\pgfsys@transformshift{2.325994in}{1.799665in}%
\pgfsys@useobject{currentmarker}{}%
\end{pgfscope}%
\begin{pgfscope}%
\pgfsys@transformshift{2.339539in}{1.803827in}%
\pgfsys@useobject{currentmarker}{}%
\end{pgfscope}%
\begin{pgfscope}%
\pgfsys@transformshift{2.353084in}{1.801767in}%
\pgfsys@useobject{currentmarker}{}%
\end{pgfscope}%
\begin{pgfscope}%
\pgfsys@transformshift{2.366629in}{1.812722in}%
\pgfsys@useobject{currentmarker}{}%
\end{pgfscope}%
\begin{pgfscope}%
\pgfsys@transformshift{2.380174in}{1.805436in}%
\pgfsys@useobject{currentmarker}{}%
\end{pgfscope}%
\begin{pgfscope}%
\pgfsys@transformshift{2.393719in}{1.815697in}%
\pgfsys@useobject{currentmarker}{}%
\end{pgfscope}%
\begin{pgfscope}%
\pgfsys@transformshift{2.407264in}{1.804754in}%
\pgfsys@useobject{currentmarker}{}%
\end{pgfscope}%
\begin{pgfscope}%
\pgfsys@transformshift{2.420809in}{1.809544in}%
\pgfsys@useobject{currentmarker}{}%
\end{pgfscope}%
\begin{pgfscope}%
\pgfsys@transformshift{2.434353in}{1.800988in}%
\pgfsys@useobject{currentmarker}{}%
\end{pgfscope}%
\begin{pgfscope}%
\pgfsys@transformshift{2.447898in}{1.812258in}%
\pgfsys@useobject{currentmarker}{}%
\end{pgfscope}%
\begin{pgfscope}%
\pgfsys@transformshift{2.461443in}{1.803624in}%
\pgfsys@useobject{currentmarker}{}%
\end{pgfscope}%
\begin{pgfscope}%
\pgfsys@transformshift{2.474988in}{1.808146in}%
\pgfsys@useobject{currentmarker}{}%
\end{pgfscope}%
\begin{pgfscope}%
\pgfsys@transformshift{2.488533in}{1.805849in}%
\pgfsys@useobject{currentmarker}{}%
\end{pgfscope}%
\begin{pgfscope}%
\pgfsys@transformshift{2.502078in}{1.810514in}%
\pgfsys@useobject{currentmarker}{}%
\end{pgfscope}%
\begin{pgfscope}%
\pgfsys@transformshift{2.515623in}{1.447721in}%
\pgfsys@useobject{currentmarker}{}%
\end{pgfscope}%
\begin{pgfscope}%
\pgfsys@transformshift{2.529168in}{1.132355in}%
\pgfsys@useobject{currentmarker}{}%
\end{pgfscope}%
\begin{pgfscope}%
\pgfsys@transformshift{2.542713in}{1.174393in}%
\pgfsys@useobject{currentmarker}{}%
\end{pgfscope}%
\begin{pgfscope}%
\pgfsys@transformshift{2.556258in}{1.216619in}%
\pgfsys@useobject{currentmarker}{}%
\end{pgfscope}%
\begin{pgfscope}%
\pgfsys@transformshift{2.569803in}{1.216913in}%
\pgfsys@useobject{currentmarker}{}%
\end{pgfscope}%
\begin{pgfscope}%
\pgfsys@transformshift{2.583348in}{1.215572in}%
\pgfsys@useobject{currentmarker}{}%
\end{pgfscope}%
\begin{pgfscope}%
\pgfsys@transformshift{2.596893in}{1.217387in}%
\pgfsys@useobject{currentmarker}{}%
\end{pgfscope}%
\begin{pgfscope}%
\pgfsys@transformshift{2.610438in}{1.225417in}%
\pgfsys@useobject{currentmarker}{}%
\end{pgfscope}%
\begin{pgfscope}%
\pgfsys@transformshift{2.623983in}{1.228964in}%
\pgfsys@useobject{currentmarker}{}%
\end{pgfscope}%
\begin{pgfscope}%
\pgfsys@transformshift{2.637528in}{1.231243in}%
\pgfsys@useobject{currentmarker}{}%
\end{pgfscope}%
\begin{pgfscope}%
\pgfsys@transformshift{2.651073in}{1.235303in}%
\pgfsys@useobject{currentmarker}{}%
\end{pgfscope}%
\begin{pgfscope}%
\pgfsys@transformshift{2.664618in}{1.247906in}%
\pgfsys@useobject{currentmarker}{}%
\end{pgfscope}%
\begin{pgfscope}%
\pgfsys@transformshift{2.678163in}{1.253494in}%
\pgfsys@useobject{currentmarker}{}%
\end{pgfscope}%
\begin{pgfscope}%
\pgfsys@transformshift{2.691708in}{1.261502in}%
\pgfsys@useobject{currentmarker}{}%
\end{pgfscope}%
\begin{pgfscope}%
\pgfsys@transformshift{2.705253in}{1.266443in}%
\pgfsys@useobject{currentmarker}{}%
\end{pgfscope}%
\begin{pgfscope}%
\pgfsys@transformshift{2.718798in}{1.272251in}%
\pgfsys@useobject{currentmarker}{}%
\end{pgfscope}%
\begin{pgfscope}%
\pgfsys@transformshift{2.732343in}{1.275651in}%
\pgfsys@useobject{currentmarker}{}%
\end{pgfscope}%
\begin{pgfscope}%
\pgfsys@transformshift{2.745888in}{1.284230in}%
\pgfsys@useobject{currentmarker}{}%
\end{pgfscope}%
\begin{pgfscope}%
\pgfsys@transformshift{2.759433in}{1.288387in}%
\pgfsys@useobject{currentmarker}{}%
\end{pgfscope}%
\begin{pgfscope}%
\pgfsys@transformshift{2.772977in}{1.295147in}%
\pgfsys@useobject{currentmarker}{}%
\end{pgfscope}%
\begin{pgfscope}%
\pgfsys@transformshift{2.786522in}{1.301744in}%
\pgfsys@useobject{currentmarker}{}%
\end{pgfscope}%
\begin{pgfscope}%
\pgfsys@transformshift{2.800067in}{1.309275in}%
\pgfsys@useobject{currentmarker}{}%
\end{pgfscope}%
\begin{pgfscope}%
\pgfsys@transformshift{2.813612in}{1.309614in}%
\pgfsys@useobject{currentmarker}{}%
\end{pgfscope}%
\begin{pgfscope}%
\pgfsys@transformshift{2.827157in}{1.311687in}%
\pgfsys@useobject{currentmarker}{}%
\end{pgfscope}%
\begin{pgfscope}%
\pgfsys@transformshift{2.840702in}{1.316842in}%
\pgfsys@useobject{currentmarker}{}%
\end{pgfscope}%
\begin{pgfscope}%
\pgfsys@transformshift{2.854247in}{1.319331in}%
\pgfsys@useobject{currentmarker}{}%
\end{pgfscope}%
\begin{pgfscope}%
\pgfsys@transformshift{2.867792in}{1.322329in}%
\pgfsys@useobject{currentmarker}{}%
\end{pgfscope}%
\begin{pgfscope}%
\pgfsys@transformshift{2.881337in}{1.329390in}%
\pgfsys@useobject{currentmarker}{}%
\end{pgfscope}%
\begin{pgfscope}%
\pgfsys@transformshift{2.894882in}{1.332771in}%
\pgfsys@useobject{currentmarker}{}%
\end{pgfscope}%
\begin{pgfscope}%
\pgfsys@transformshift{2.908427in}{1.338568in}%
\pgfsys@useobject{currentmarker}{}%
\end{pgfscope}%
\begin{pgfscope}%
\pgfsys@transformshift{2.921972in}{1.338734in}%
\pgfsys@useobject{currentmarker}{}%
\end{pgfscope}%
\begin{pgfscope}%
\pgfsys@transformshift{2.935517in}{1.339172in}%
\pgfsys@useobject{currentmarker}{}%
\end{pgfscope}%
\begin{pgfscope}%
\pgfsys@transformshift{2.949062in}{1.343784in}%
\pgfsys@useobject{currentmarker}{}%
\end{pgfscope}%
\begin{pgfscope}%
\pgfsys@transformshift{2.962607in}{1.352510in}%
\pgfsys@useobject{currentmarker}{}%
\end{pgfscope}%
\begin{pgfscope}%
\pgfsys@transformshift{2.976152in}{1.354595in}%
\pgfsys@useobject{currentmarker}{}%
\end{pgfscope}%
\begin{pgfscope}%
\pgfsys@transformshift{2.989697in}{1.357936in}%
\pgfsys@useobject{currentmarker}{}%
\end{pgfscope}%
\begin{pgfscope}%
\pgfsys@transformshift{3.003242in}{1.365164in}%
\pgfsys@useobject{currentmarker}{}%
\end{pgfscope}%
\begin{pgfscope}%
\pgfsys@transformshift{3.016787in}{1.368148in}%
\pgfsys@useobject{currentmarker}{}%
\end{pgfscope}%
\begin{pgfscope}%
\pgfsys@transformshift{3.030332in}{1.370669in}%
\pgfsys@useobject{currentmarker}{}%
\end{pgfscope}%
\begin{pgfscope}%
\pgfsys@transformshift{3.043877in}{1.377282in}%
\pgfsys@useobject{currentmarker}{}%
\end{pgfscope}%
\begin{pgfscope}%
\pgfsys@transformshift{3.057422in}{1.379381in}%
\pgfsys@useobject{currentmarker}{}%
\end{pgfscope}%
\begin{pgfscope}%
\pgfsys@transformshift{3.070967in}{1.382622in}%
\pgfsys@useobject{currentmarker}{}%
\end{pgfscope}%
\begin{pgfscope}%
\pgfsys@transformshift{3.084512in}{1.388011in}%
\pgfsys@useobject{currentmarker}{}%
\end{pgfscope}%
\begin{pgfscope}%
\pgfsys@transformshift{3.098057in}{1.391851in}%
\pgfsys@useobject{currentmarker}{}%
\end{pgfscope}%
\begin{pgfscope}%
\pgfsys@transformshift{3.111601in}{1.394377in}%
\pgfsys@useobject{currentmarker}{}%
\end{pgfscope}%
\begin{pgfscope}%
\pgfsys@transformshift{3.125146in}{1.395501in}%
\pgfsys@useobject{currentmarker}{}%
\end{pgfscope}%
\begin{pgfscope}%
\pgfsys@transformshift{3.138691in}{1.397961in}%
\pgfsys@useobject{currentmarker}{}%
\end{pgfscope}%
\begin{pgfscope}%
\pgfsys@transformshift{3.152236in}{1.402137in}%
\pgfsys@useobject{currentmarker}{}%
\end{pgfscope}%
\begin{pgfscope}%
\pgfsys@transformshift{3.165781in}{1.402841in}%
\pgfsys@useobject{currentmarker}{}%
\end{pgfscope}%
\begin{pgfscope}%
\pgfsys@transformshift{3.179326in}{1.400218in}%
\pgfsys@useobject{currentmarker}{}%
\end{pgfscope}%
\begin{pgfscope}%
\pgfsys@transformshift{3.192871in}{1.396942in}%
\pgfsys@useobject{currentmarker}{}%
\end{pgfscope}%
\begin{pgfscope}%
\pgfsys@transformshift{3.206416in}{1.391729in}%
\pgfsys@useobject{currentmarker}{}%
\end{pgfscope}%
\begin{pgfscope}%
\pgfsys@transformshift{3.219961in}{1.390170in}%
\pgfsys@useobject{currentmarker}{}%
\end{pgfscope}%
\begin{pgfscope}%
\pgfsys@transformshift{3.233506in}{1.390698in}%
\pgfsys@useobject{currentmarker}{}%
\end{pgfscope}%
\begin{pgfscope}%
\pgfsys@transformshift{3.247051in}{1.388895in}%
\pgfsys@useobject{currentmarker}{}%
\end{pgfscope}%
\begin{pgfscope}%
\pgfsys@transformshift{3.260596in}{1.388051in}%
\pgfsys@useobject{currentmarker}{}%
\end{pgfscope}%
\begin{pgfscope}%
\pgfsys@transformshift{3.274141in}{1.388035in}%
\pgfsys@useobject{currentmarker}{}%
\end{pgfscope}%
\begin{pgfscope}%
\pgfsys@transformshift{3.287686in}{1.389287in}%
\pgfsys@useobject{currentmarker}{}%
\end{pgfscope}%
\begin{pgfscope}%
\pgfsys@transformshift{3.301231in}{1.391260in}%
\pgfsys@useobject{currentmarker}{}%
\end{pgfscope}%
\begin{pgfscope}%
\pgfsys@transformshift{3.314776in}{1.385460in}%
\pgfsys@useobject{currentmarker}{}%
\end{pgfscope}%
\begin{pgfscope}%
\pgfsys@transformshift{3.328321in}{1.378806in}%
\pgfsys@useobject{currentmarker}{}%
\end{pgfscope}%
\begin{pgfscope}%
\pgfsys@transformshift{3.341866in}{1.379679in}%
\pgfsys@useobject{currentmarker}{}%
\end{pgfscope}%
\begin{pgfscope}%
\pgfsys@transformshift{3.355411in}{1.381940in}%
\pgfsys@useobject{currentmarker}{}%
\end{pgfscope}%
\begin{pgfscope}%
\pgfsys@transformshift{3.368956in}{1.385605in}%
\pgfsys@useobject{currentmarker}{}%
\end{pgfscope}%
\begin{pgfscope}%
\pgfsys@transformshift{3.382501in}{1.388856in}%
\pgfsys@useobject{currentmarker}{}%
\end{pgfscope}%
\begin{pgfscope}%
\pgfsys@transformshift{3.396046in}{1.390682in}%
\pgfsys@useobject{currentmarker}{}%
\end{pgfscope}%
\begin{pgfscope}%
\pgfsys@transformshift{3.409591in}{1.386749in}%
\pgfsys@useobject{currentmarker}{}%
\end{pgfscope}%
\begin{pgfscope}%
\pgfsys@transformshift{3.423136in}{1.384979in}%
\pgfsys@useobject{currentmarker}{}%
\end{pgfscope}%
\begin{pgfscope}%
\pgfsys@transformshift{3.436681in}{1.383309in}%
\pgfsys@useobject{currentmarker}{}%
\end{pgfscope}%
\begin{pgfscope}%
\pgfsys@transformshift{3.450225in}{1.380745in}%
\pgfsys@useobject{currentmarker}{}%
\end{pgfscope}%
\begin{pgfscope}%
\pgfsys@transformshift{3.463770in}{1.374556in}%
\pgfsys@useobject{currentmarker}{}%
\end{pgfscope}%
\begin{pgfscope}%
\pgfsys@transformshift{3.477315in}{1.374272in}%
\pgfsys@useobject{currentmarker}{}%
\end{pgfscope}%
\begin{pgfscope}%
\pgfsys@transformshift{3.490860in}{1.376326in}%
\pgfsys@useobject{currentmarker}{}%
\end{pgfscope}%
\begin{pgfscope}%
\pgfsys@transformshift{3.504405in}{1.376831in}%
\pgfsys@useobject{currentmarker}{}%
\end{pgfscope}%
\begin{pgfscope}%
\pgfsys@transformshift{3.517950in}{1.379526in}%
\pgfsys@useobject{currentmarker}{}%
\end{pgfscope}%
\begin{pgfscope}%
\pgfsys@transformshift{3.531495in}{1.380538in}%
\pgfsys@useobject{currentmarker}{}%
\end{pgfscope}%
\begin{pgfscope}%
\pgfsys@transformshift{3.545040in}{1.378255in}%
\pgfsys@useobject{currentmarker}{}%
\end{pgfscope}%
\begin{pgfscope}%
\pgfsys@transformshift{3.558585in}{1.370217in}%
\pgfsys@useobject{currentmarker}{}%
\end{pgfscope}%
\begin{pgfscope}%
\pgfsys@transformshift{3.572130in}{1.366157in}%
\pgfsys@useobject{currentmarker}{}%
\end{pgfscope}%
\begin{pgfscope}%
\pgfsys@transformshift{3.585675in}{1.367539in}%
\pgfsys@useobject{currentmarker}{}%
\end{pgfscope}%
\begin{pgfscope}%
\pgfsys@transformshift{3.599220in}{1.367676in}%
\pgfsys@useobject{currentmarker}{}%
\end{pgfscope}%
\begin{pgfscope}%
\pgfsys@transformshift{3.612765in}{1.371405in}%
\pgfsys@useobject{currentmarker}{}%
\end{pgfscope}%
\begin{pgfscope}%
\pgfsys@transformshift{3.626310in}{1.378619in}%
\pgfsys@useobject{currentmarker}{}%
\end{pgfscope}%
\begin{pgfscope}%
\pgfsys@transformshift{3.639855in}{1.384965in}%
\pgfsys@useobject{currentmarker}{}%
\end{pgfscope}%
\begin{pgfscope}%
\pgfsys@transformshift{3.653400in}{1.386266in}%
\pgfsys@useobject{currentmarker}{}%
\end{pgfscope}%
\begin{pgfscope}%
\pgfsys@transformshift{3.666945in}{1.349309in}%
\pgfsys@useobject{currentmarker}{}%
\end{pgfscope}%
\begin{pgfscope}%
\pgfsys@transformshift{3.680490in}{1.309808in}%
\pgfsys@useobject{currentmarker}{}%
\end{pgfscope}%
\begin{pgfscope}%
\pgfsys@transformshift{3.694035in}{1.309130in}%
\pgfsys@useobject{currentmarker}{}%
\end{pgfscope}%
\begin{pgfscope}%
\pgfsys@transformshift{3.707580in}{1.308672in}%
\pgfsys@useobject{currentmarker}{}%
\end{pgfscope}%
\begin{pgfscope}%
\pgfsys@transformshift{3.721125in}{1.307244in}%
\pgfsys@useobject{currentmarker}{}%
\end{pgfscope}%
\begin{pgfscope}%
\pgfsys@transformshift{3.734670in}{1.285985in}%
\pgfsys@useobject{currentmarker}{}%
\end{pgfscope}%
\begin{pgfscope}%
\pgfsys@transformshift{3.748215in}{1.265798in}%
\pgfsys@useobject{currentmarker}{}%
\end{pgfscope}%
\begin{pgfscope}%
\pgfsys@transformshift{3.761760in}{1.215696in}%
\pgfsys@useobject{currentmarker}{}%
\end{pgfscope}%
\begin{pgfscope}%
\pgfsys@transformshift{3.775305in}{1.205070in}%
\pgfsys@useobject{currentmarker}{}%
\end{pgfscope}%
\begin{pgfscope}%
\pgfsys@transformshift{3.788849in}{1.200412in}%
\pgfsys@useobject{currentmarker}{}%
\end{pgfscope}%
\begin{pgfscope}%
\pgfsys@transformshift{3.802394in}{1.199983in}%
\pgfsys@useobject{currentmarker}{}%
\end{pgfscope}%
\begin{pgfscope}%
\pgfsys@transformshift{3.815939in}{1.201464in}%
\pgfsys@useobject{currentmarker}{}%
\end{pgfscope}%
\begin{pgfscope}%
\pgfsys@transformshift{3.829484in}{1.197268in}%
\pgfsys@useobject{currentmarker}{}%
\end{pgfscope}%
\begin{pgfscope}%
\pgfsys@transformshift{3.843029in}{1.194194in}%
\pgfsys@useobject{currentmarker}{}%
\end{pgfscope}%
\begin{pgfscope}%
\pgfsys@transformshift{3.856574in}{1.187573in}%
\pgfsys@useobject{currentmarker}{}%
\end{pgfscope}%
\begin{pgfscope}%
\pgfsys@transformshift{3.870119in}{1.178081in}%
\pgfsys@useobject{currentmarker}{}%
\end{pgfscope}%
\begin{pgfscope}%
\pgfsys@transformshift{3.883664in}{1.178563in}%
\pgfsys@useobject{currentmarker}{}%
\end{pgfscope}%
\begin{pgfscope}%
\pgfsys@transformshift{3.897209in}{1.181078in}%
\pgfsys@useobject{currentmarker}{}%
\end{pgfscope}%
\begin{pgfscope}%
\pgfsys@transformshift{3.910754in}{1.183233in}%
\pgfsys@useobject{currentmarker}{}%
\end{pgfscope}%
\begin{pgfscope}%
\pgfsys@transformshift{3.924299in}{1.186005in}%
\pgfsys@useobject{currentmarker}{}%
\end{pgfscope}%
\begin{pgfscope}%
\pgfsys@transformshift{3.937844in}{1.180294in}%
\pgfsys@useobject{currentmarker}{}%
\end{pgfscope}%
\begin{pgfscope}%
\pgfsys@transformshift{3.951389in}{1.171989in}%
\pgfsys@useobject{currentmarker}{}%
\end{pgfscope}%
\begin{pgfscope}%
\pgfsys@transformshift{3.964934in}{1.170112in}%
\pgfsys@useobject{currentmarker}{}%
\end{pgfscope}%
\begin{pgfscope}%
\pgfsys@transformshift{3.978479in}{1.170915in}%
\pgfsys@useobject{currentmarker}{}%
\end{pgfscope}%
\begin{pgfscope}%
\pgfsys@transformshift{3.992024in}{1.172099in}%
\pgfsys@useobject{currentmarker}{}%
\end{pgfscope}%
\begin{pgfscope}%
\pgfsys@transformshift{4.005569in}{1.169283in}%
\pgfsys@useobject{currentmarker}{}%
\end{pgfscope}%
\begin{pgfscope}%
\pgfsys@transformshift{4.019114in}{1.172003in}%
\pgfsys@useobject{currentmarker}{}%
\end{pgfscope}%
\begin{pgfscope}%
\pgfsys@transformshift{4.032659in}{1.174616in}%
\pgfsys@useobject{currentmarker}{}%
\end{pgfscope}%
\begin{pgfscope}%
\pgfsys@transformshift{4.046204in}{1.164909in}%
\pgfsys@useobject{currentmarker}{}%
\end{pgfscope}%
\begin{pgfscope}%
\pgfsys@transformshift{4.059749in}{1.164606in}%
\pgfsys@useobject{currentmarker}{}%
\end{pgfscope}%
\begin{pgfscope}%
\pgfsys@transformshift{4.073294in}{1.164671in}%
\pgfsys@useobject{currentmarker}{}%
\end{pgfscope}%
\begin{pgfscope}%
\pgfsys@transformshift{4.086839in}{1.163124in}%
\pgfsys@useobject{currentmarker}{}%
\end{pgfscope}%
\begin{pgfscope}%
\pgfsys@transformshift{4.100384in}{1.162434in}%
\pgfsys@useobject{currentmarker}{}%
\end{pgfscope}%
\begin{pgfscope}%
\pgfsys@transformshift{4.113929in}{1.175032in}%
\pgfsys@useobject{currentmarker}{}%
\end{pgfscope}%
\begin{pgfscope}%
\pgfsys@transformshift{4.127473in}{1.173675in}%
\pgfsys@useobject{currentmarker}{}%
\end{pgfscope}%
\begin{pgfscope}%
\pgfsys@transformshift{4.141018in}{1.174860in}%
\pgfsys@useobject{currentmarker}{}%
\end{pgfscope}%
\begin{pgfscope}%
\pgfsys@transformshift{4.154563in}{1.185276in}%
\pgfsys@useobject{currentmarker}{}%
\end{pgfscope}%
\begin{pgfscope}%
\pgfsys@transformshift{4.168108in}{1.182540in}%
\pgfsys@useobject{currentmarker}{}%
\end{pgfscope}%
\begin{pgfscope}%
\pgfsys@transformshift{4.181653in}{1.176707in}%
\pgfsys@useobject{currentmarker}{}%
\end{pgfscope}%
\begin{pgfscope}%
\pgfsys@transformshift{4.195198in}{1.175537in}%
\pgfsys@useobject{currentmarker}{}%
\end{pgfscope}%
\begin{pgfscope}%
\pgfsys@transformshift{4.208743in}{1.177188in}%
\pgfsys@useobject{currentmarker}{}%
\end{pgfscope}%
\begin{pgfscope}%
\pgfsys@transformshift{4.222288in}{1.180295in}%
\pgfsys@useobject{currentmarker}{}%
\end{pgfscope}%
\begin{pgfscope}%
\pgfsys@transformshift{4.235833in}{1.180342in}%
\pgfsys@useobject{currentmarker}{}%
\end{pgfscope}%
\begin{pgfscope}%
\pgfsys@transformshift{4.249378in}{1.181079in}%
\pgfsys@useobject{currentmarker}{}%
\end{pgfscope}%
\begin{pgfscope}%
\pgfsys@transformshift{4.262923in}{1.177915in}%
\pgfsys@useobject{currentmarker}{}%
\end{pgfscope}%
\begin{pgfscope}%
\pgfsys@transformshift{4.276468in}{1.178790in}%
\pgfsys@useobject{currentmarker}{}%
\end{pgfscope}%
\begin{pgfscope}%
\pgfsys@transformshift{4.290013in}{1.179357in}%
\pgfsys@useobject{currentmarker}{}%
\end{pgfscope}%
\begin{pgfscope}%
\pgfsys@transformshift{4.303558in}{1.174784in}%
\pgfsys@useobject{currentmarker}{}%
\end{pgfscope}%
\begin{pgfscope}%
\pgfsys@transformshift{4.317103in}{1.171308in}%
\pgfsys@useobject{currentmarker}{}%
\end{pgfscope}%
\begin{pgfscope}%
\pgfsys@transformshift{4.330648in}{1.166091in}%
\pgfsys@useobject{currentmarker}{}%
\end{pgfscope}%
\begin{pgfscope}%
\pgfsys@transformshift{4.344193in}{1.154620in}%
\pgfsys@useobject{currentmarker}{}%
\end{pgfscope}%
\begin{pgfscope}%
\pgfsys@transformshift{4.357738in}{1.150621in}%
\pgfsys@useobject{currentmarker}{}%
\end{pgfscope}%
\begin{pgfscope}%
\pgfsys@transformshift{4.371283in}{1.152345in}%
\pgfsys@useobject{currentmarker}{}%
\end{pgfscope}%
\begin{pgfscope}%
\pgfsys@transformshift{4.384828in}{1.154342in}%
\pgfsys@useobject{currentmarker}{}%
\end{pgfscope}%
\begin{pgfscope}%
\pgfsys@transformshift{4.398373in}{1.154615in}%
\pgfsys@useobject{currentmarker}{}%
\end{pgfscope}%
\begin{pgfscope}%
\pgfsys@transformshift{4.411918in}{1.152906in}%
\pgfsys@useobject{currentmarker}{}%
\end{pgfscope}%
\begin{pgfscope}%
\pgfsys@transformshift{4.425463in}{1.152754in}%
\pgfsys@useobject{currentmarker}{}%
\end{pgfscope}%
\begin{pgfscope}%
\pgfsys@transformshift{4.439008in}{1.154940in}%
\pgfsys@useobject{currentmarker}{}%
\end{pgfscope}%
\begin{pgfscope}%
\pgfsys@transformshift{4.452553in}{1.151143in}%
\pgfsys@useobject{currentmarker}{}%
\end{pgfscope}%
\begin{pgfscope}%
\pgfsys@transformshift{4.466097in}{1.147867in}%
\pgfsys@useobject{currentmarker}{}%
\end{pgfscope}%
\begin{pgfscope}%
\pgfsys@transformshift{4.479642in}{1.150979in}%
\pgfsys@useobject{currentmarker}{}%
\end{pgfscope}%
\begin{pgfscope}%
\pgfsys@transformshift{4.493187in}{1.158248in}%
\pgfsys@useobject{currentmarker}{}%
\end{pgfscope}%
\begin{pgfscope}%
\pgfsys@transformshift{4.506732in}{1.157533in}%
\pgfsys@useobject{currentmarker}{}%
\end{pgfscope}%
\begin{pgfscope}%
\pgfsys@transformshift{4.520277in}{1.153530in}%
\pgfsys@useobject{currentmarker}{}%
\end{pgfscope}%
\begin{pgfscope}%
\pgfsys@transformshift{4.533822in}{1.159234in}%
\pgfsys@useobject{currentmarker}{}%
\end{pgfscope}%
\begin{pgfscope}%
\pgfsys@transformshift{4.547367in}{1.159858in}%
\pgfsys@useobject{currentmarker}{}%
\end{pgfscope}%
\begin{pgfscope}%
\pgfsys@transformshift{4.560912in}{1.158985in}%
\pgfsys@useobject{currentmarker}{}%
\end{pgfscope}%
\begin{pgfscope}%
\pgfsys@transformshift{4.574457in}{1.165468in}%
\pgfsys@useobject{currentmarker}{}%
\end{pgfscope}%
\begin{pgfscope}%
\pgfsys@transformshift{4.588002in}{1.171519in}%
\pgfsys@useobject{currentmarker}{}%
\end{pgfscope}%
\begin{pgfscope}%
\pgfsys@transformshift{4.601547in}{1.172230in}%
\pgfsys@useobject{currentmarker}{}%
\end{pgfscope}%
\begin{pgfscope}%
\pgfsys@transformshift{4.615092in}{1.169859in}%
\pgfsys@useobject{currentmarker}{}%
\end{pgfscope}%
\begin{pgfscope}%
\pgfsys@transformshift{4.628637in}{1.173865in}%
\pgfsys@useobject{currentmarker}{}%
\end{pgfscope}%
\begin{pgfscope}%
\pgfsys@transformshift{4.642182in}{1.181432in}%
\pgfsys@useobject{currentmarker}{}%
\end{pgfscope}%
\begin{pgfscope}%
\pgfsys@transformshift{4.655727in}{1.183270in}%
\pgfsys@useobject{currentmarker}{}%
\end{pgfscope}%
\end{pgfscope}%
\begin{pgfscope}%
\pgfsetrectcap%
\pgfsetmiterjoin%
\pgfsetlinewidth{0.803000pt}%
\definecolor{currentstroke}{rgb}{0.000000,0.000000,0.000000}%
\pgfsetstrokecolor{currentstroke}%
\pgfsetdash{}{0pt}%
\pgfpathmoveto{\pgfqpoint{0.605784in}{0.382904in}}%
\pgfpathlineto{\pgfqpoint{0.605784in}{2.425059in}}%
\pgfusepath{stroke}%
\end{pgfscope}%
\begin{pgfscope}%
\pgfsetrectcap%
\pgfsetmiterjoin%
\pgfsetlinewidth{0.803000pt}%
\definecolor{currentstroke}{rgb}{0.000000,0.000000,0.000000}%
\pgfsetstrokecolor{currentstroke}%
\pgfsetdash{}{0pt}%
\pgfpathmoveto{\pgfqpoint{4.669272in}{0.382904in}}%
\pgfpathlineto{\pgfqpoint{4.669272in}{2.425059in}}%
\pgfusepath{stroke}%
\end{pgfscope}%
\begin{pgfscope}%
\pgfsetrectcap%
\pgfsetmiterjoin%
\pgfsetlinewidth{0.803000pt}%
\definecolor{currentstroke}{rgb}{0.000000,0.000000,0.000000}%
\pgfsetstrokecolor{currentstroke}%
\pgfsetdash{}{0pt}%
\pgfpathmoveto{\pgfqpoint{0.605784in}{0.382904in}}%
\pgfpathlineto{\pgfqpoint{4.669272in}{0.382904in}}%
\pgfusepath{stroke}%
\end{pgfscope}%
\begin{pgfscope}%
\pgfsetrectcap%
\pgfsetmiterjoin%
\pgfsetlinewidth{0.803000pt}%
\definecolor{currentstroke}{rgb}{0.000000,0.000000,0.000000}%
\pgfsetstrokecolor{currentstroke}%
\pgfsetdash{}{0pt}%
\pgfpathmoveto{\pgfqpoint{0.605784in}{2.425059in}}%
\pgfpathlineto{\pgfqpoint{4.669272in}{2.425059in}}%
\pgfusepath{stroke}%
\end{pgfscope}%
\begin{pgfscope}%
\pgfsetbuttcap%
\pgfsetmiterjoin%
\definecolor{currentfill}{rgb}{0.556863,0.729412,0.898039}%
\pgfsetfillcolor{currentfill}%
\pgfsetlinewidth{0.752812pt}%
\definecolor{currentstroke}{rgb}{0.000000,0.000000,0.000000}%
\pgfsetstrokecolor{currentstroke}%
\pgfsetdash{}{0pt}%
\pgfpathmoveto{\pgfqpoint{1.653264in}{0.382904in}}%
\pgfpathlineto{\pgfqpoint{4.669272in}{0.382904in}}%
\pgfpathlineto{\pgfqpoint{4.669272in}{0.630932in}}%
\pgfpathlineto{\pgfqpoint{1.653264in}{0.630932in}}%
\pgfpathclose%
\pgfusepath{stroke,fill}%
\end{pgfscope}%
\begin{pgfscope}%
\pgfsetrectcap%
\pgfsetroundjoin%
\pgfsetlinewidth{1.505625pt}%
\definecolor{currentstroke}{rgb}{0.000000,0.000000,0.000000}%
\pgfsetstrokecolor{currentstroke}%
\pgfsetdash{}{0pt}%
\pgfpathmoveto{\pgfqpoint{1.708820in}{0.521406in}}%
\pgfpathlineto{\pgfqpoint{1.986598in}{0.521406in}}%
\pgfusepath{stroke}%
\end{pgfscope}%
\begin{pgfscope}%
\definecolor{textcolor}{rgb}{0.000000,0.000000,0.000000}%
\pgfsetstrokecolor{textcolor}%
\pgfsetfillcolor{textcolor}%
\pgftext[x=2.097709in,y=0.472795in,left,base]{\color{textcolor}\rmfamily\fontsize{10.000000}{12.000000}\selectfont \(\displaystyle \mathcal{X}\)}%
\end{pgfscope}%
\begin{pgfscope}%
\pgfsetrectcap%
\pgfsetroundjoin%
\pgfsetlinewidth{1.003750pt}%
\definecolor{currentstroke}{rgb}{1.000000,1.000000,1.000000}%
\pgfsetstrokecolor{currentstroke}%
\pgfsetdash{}{0pt}%
\pgfpathmoveto{\pgfqpoint{2.494893in}{0.521406in}}%
\pgfpathlineto{\pgfqpoint{2.772671in}{0.521406in}}%
\pgfusepath{stroke}%
\end{pgfscope}%
\begin{pgfscope}%
\definecolor{textcolor}{rgb}{0.000000,0.000000,0.000000}%
\pgfsetstrokecolor{textcolor}%
\pgfsetfillcolor{textcolor}%
\pgftext[x=2.883782in,y=0.472795in,left,base]{\color{textcolor}\rmfamily\fontsize{10.000000}{12.000000}\selectfont \(\displaystyle \mathbf{x}_t^*\)}%
\end{pgfscope}%
\begin{pgfscope}%
\pgfsetbuttcap%
\pgfsetbeveljoin%
\definecolor{currentfill}{rgb}{1.000000,1.000000,1.000000}%
\pgfsetfillcolor{currentfill}%
\pgfsetlinewidth{1.003750pt}%
\definecolor{currentstroke}{rgb}{1.000000,1.000000,1.000000}%
\pgfsetstrokecolor{currentstroke}%
\pgfsetdash{}{0pt}%
\pgfsys@defobject{currentmarker}{\pgfqpoint{-0.019814in}{-0.016855in}}{\pgfqpoint{0.019814in}{0.020833in}}{%
\pgfpathmoveto{\pgfqpoint{0.000000in}{0.020833in}}%
\pgfpathlineto{\pgfqpoint{-0.004677in}{0.006438in}}%
\pgfpathlineto{\pgfqpoint{-0.019814in}{0.006438in}}%
\pgfpathlineto{\pgfqpoint{-0.007568in}{-0.002459in}}%
\pgfpathlineto{\pgfqpoint{-0.012246in}{-0.016855in}}%
\pgfpathlineto{\pgfqpoint{-0.000000in}{-0.007958in}}%
\pgfpathlineto{\pgfqpoint{0.012246in}{-0.016855in}}%
\pgfpathlineto{\pgfqpoint{0.007568in}{-0.002459in}}%
\pgfpathlineto{\pgfqpoint{0.019814in}{0.006438in}}%
\pgfpathlineto{\pgfqpoint{0.004677in}{0.006438in}}%
\pgfpathclose%
\pgfusepath{stroke,fill}%
}%
\begin{pgfscope}%
\pgfsys@transformshift{3.437632in}{0.521406in}%
\pgfsys@useobject{currentmarker}{}%
\end{pgfscope}%
\end{pgfscope}%
\begin{pgfscope}%
\definecolor{textcolor}{rgb}{0.000000,0.000000,0.000000}%
\pgfsetstrokecolor{textcolor}%
\pgfsetfillcolor{textcolor}%
\pgftext[x=3.687632in,y=0.472795in,left,base]{\color{textcolor}\rmfamily\fontsize{10.000000}{12.000000}\selectfont \(\displaystyle \mathbf{x}_t\)}%
\end{pgfscope}%
\begin{pgfscope}%
\pgfsetbuttcap%
\pgfsetroundjoin%
\definecolor{currentfill}{rgb}{1.000000,0.000000,0.000000}%
\pgfsetfillcolor{currentfill}%
\pgfsetlinewidth{1.003750pt}%
\definecolor{currentstroke}{rgb}{1.000000,0.000000,0.000000}%
\pgfsetstrokecolor{currentstroke}%
\pgfsetdash{}{0pt}%
\pgfsys@defobject{currentmarker}{\pgfqpoint{-0.020833in}{-0.020833in}}{\pgfqpoint{0.020833in}{0.020833in}}{%
\pgfpathmoveto{\pgfqpoint{-0.020833in}{-0.020833in}}%
\pgfpathlineto{\pgfqpoint{0.020833in}{0.020833in}}%
\pgfpathmoveto{\pgfqpoint{-0.020833in}{0.020833in}}%
\pgfpathlineto{\pgfqpoint{0.020833in}{-0.020833in}}%
\pgfusepath{stroke,fill}%
}%
\begin{pgfscope}%
\pgfsys@transformshift{4.226533in}{0.521406in}%
\pgfsys@useobject{currentmarker}{}%
\end{pgfscope}%
\end{pgfscope}%
\begin{pgfscope}%
\definecolor{textcolor}{rgb}{0.000000,0.000000,0.000000}%
\pgfsetstrokecolor{textcolor}%
\pgfsetfillcolor{textcolor}%
\pgftext[x=4.476533in,y=0.472795in,left,base]{\color{textcolor}\rmfamily\fontsize{10.000000}{12.000000}\selectfont \(\displaystyle \hat{\mathbf{x}}^*_t\)}%
\end{pgfscope}%
\begin{pgfscope}%
\pgfsetbuttcap%
\pgfsetmiterjoin%
\definecolor{currentfill}{rgb}{1.000000,1.000000,1.000000}%
\pgfsetfillcolor{currentfill}%
\pgfsetlinewidth{0.000000pt}%
\definecolor{currentstroke}{rgb}{0.000000,0.000000,0.000000}%
\pgfsetstrokecolor{currentstroke}%
\pgfsetstrokeopacity{0.000000}%
\pgfsetdash{}{0pt}%
\pgfpathmoveto{\pgfqpoint{4.756284in}{0.382904in}}%
\pgfpathlineto{\pgfqpoint{4.849110in}{0.382904in}}%
\pgfpathlineto{\pgfqpoint{4.849110in}{2.425059in}}%
\pgfpathlineto{\pgfqpoint{4.756284in}{2.425059in}}%
\pgfpathclose%
\pgfusepath{fill}%
\end{pgfscope}%
\begin{pgfscope}%
\pgfpathrectangle{\pgfqpoint{4.756284in}{0.382904in}}{\pgfqpoint{0.092825in}{2.042155in}}%
\pgfusepath{clip}%
\pgfsetbuttcap%
\pgfsetmiterjoin%
\definecolor{currentfill}{rgb}{1.000000,1.000000,1.000000}%
\pgfsetfillcolor{currentfill}%
\pgfsetlinewidth{0.010037pt}%
\definecolor{currentstroke}{rgb}{1.000000,1.000000,1.000000}%
\pgfsetstrokecolor{currentstroke}%
\pgfsetdash{}{0pt}%
\pgfpathmoveto{\pgfqpoint{4.802697in}{0.382904in}}%
\pgfpathlineto{\pgfqpoint{4.756284in}{0.475729in}}%
\pgfpathlineto{\pgfqpoint{4.756284in}{2.332233in}}%
\pgfpathlineto{\pgfqpoint{4.802697in}{2.425059in}}%
\pgfpathlineto{\pgfqpoint{4.802697in}{2.425059in}}%
\pgfpathlineto{\pgfqpoint{4.849110in}{2.332233in}}%
\pgfpathlineto{\pgfqpoint{4.849110in}{0.475729in}}%
\pgfpathlineto{\pgfqpoint{4.802697in}{0.382904in}}%
\pgfpathclose%
\pgfusepath{stroke,fill}%
\end{pgfscope}%
\begin{pgfscope}%
\pgfpathrectangle{\pgfqpoint{4.756284in}{0.382904in}}{\pgfqpoint{0.092825in}{2.042155in}}%
\pgfusepath{clip}%
\pgfsetbuttcap%
\pgfsetroundjoin%
\definecolor{currentfill}{rgb}{0.267004,0.004874,0.329415}%
\pgfsetfillcolor{currentfill}%
\pgfsetlinewidth{0.000000pt}%
\definecolor{currentstroke}{rgb}{0.000000,0.000000,0.000000}%
\pgfsetstrokecolor{currentstroke}%
\pgfsetdash{}{0pt}%
\pgfpathmoveto{\pgfqpoint{4.802697in}{0.382904in}}%
\pgfpathlineto{\pgfqpoint{4.802697in}{0.382904in}}%
\pgfpathlineto{\pgfqpoint{4.849110in}{0.475729in}}%
\pgfpathlineto{\pgfqpoint{4.756284in}{0.475729in}}%
\pgfpathlineto{\pgfqpoint{4.802697in}{0.382904in}}%
\pgfusepath{fill}%
\end{pgfscope}%
\begin{pgfscope}%
\pgfpathrectangle{\pgfqpoint{4.756284in}{0.382904in}}{\pgfqpoint{0.092825in}{2.042155in}}%
\pgfusepath{clip}%
\pgfsetbuttcap%
\pgfsetroundjoin%
\definecolor{currentfill}{rgb}{0.277941,0.056324,0.381191}%
\pgfsetfillcolor{currentfill}%
\pgfsetlinewidth{0.000000pt}%
\definecolor{currentstroke}{rgb}{0.000000,0.000000,0.000000}%
\pgfsetstrokecolor{currentstroke}%
\pgfsetdash{}{0pt}%
\pgfpathmoveto{\pgfqpoint{4.756284in}{0.475729in}}%
\pgfpathlineto{\pgfqpoint{4.849110in}{0.475729in}}%
\pgfpathlineto{\pgfqpoint{4.849110in}{0.608337in}}%
\pgfpathlineto{\pgfqpoint{4.756284in}{0.608337in}}%
\pgfpathlineto{\pgfqpoint{4.756284in}{0.475729in}}%
\pgfusepath{fill}%
\end{pgfscope}%
\begin{pgfscope}%
\pgfpathrectangle{\pgfqpoint{4.756284in}{0.382904in}}{\pgfqpoint{0.092825in}{2.042155in}}%
\pgfusepath{clip}%
\pgfsetbuttcap%
\pgfsetroundjoin%
\definecolor{currentfill}{rgb}{0.281887,0.150881,0.465405}%
\pgfsetfillcolor{currentfill}%
\pgfsetlinewidth{0.000000pt}%
\definecolor{currentstroke}{rgb}{0.000000,0.000000,0.000000}%
\pgfsetstrokecolor{currentstroke}%
\pgfsetdash{}{0pt}%
\pgfpathmoveto{\pgfqpoint{4.756284in}{0.608337in}}%
\pgfpathlineto{\pgfqpoint{4.849110in}{0.608337in}}%
\pgfpathlineto{\pgfqpoint{4.849110in}{0.740944in}}%
\pgfpathlineto{\pgfqpoint{4.756284in}{0.740944in}}%
\pgfpathlineto{\pgfqpoint{4.756284in}{0.608337in}}%
\pgfusepath{fill}%
\end{pgfscope}%
\begin{pgfscope}%
\pgfpathrectangle{\pgfqpoint{4.756284in}{0.382904in}}{\pgfqpoint{0.092825in}{2.042155in}}%
\pgfusepath{clip}%
\pgfsetbuttcap%
\pgfsetroundjoin%
\definecolor{currentfill}{rgb}{0.263663,0.237631,0.518762}%
\pgfsetfillcolor{currentfill}%
\pgfsetlinewidth{0.000000pt}%
\definecolor{currentstroke}{rgb}{0.000000,0.000000,0.000000}%
\pgfsetstrokecolor{currentstroke}%
\pgfsetdash{}{0pt}%
\pgfpathmoveto{\pgfqpoint{4.756284in}{0.740944in}}%
\pgfpathlineto{\pgfqpoint{4.849110in}{0.740944in}}%
\pgfpathlineto{\pgfqpoint{4.849110in}{0.873552in}}%
\pgfpathlineto{\pgfqpoint{4.756284in}{0.873552in}}%
\pgfpathlineto{\pgfqpoint{4.756284in}{0.740944in}}%
\pgfusepath{fill}%
\end{pgfscope}%
\begin{pgfscope}%
\pgfpathrectangle{\pgfqpoint{4.756284in}{0.382904in}}{\pgfqpoint{0.092825in}{2.042155in}}%
\pgfusepath{clip}%
\pgfsetbuttcap%
\pgfsetroundjoin%
\definecolor{currentfill}{rgb}{0.229739,0.322361,0.545706}%
\pgfsetfillcolor{currentfill}%
\pgfsetlinewidth{0.000000pt}%
\definecolor{currentstroke}{rgb}{0.000000,0.000000,0.000000}%
\pgfsetstrokecolor{currentstroke}%
\pgfsetdash{}{0pt}%
\pgfpathmoveto{\pgfqpoint{4.756284in}{0.873552in}}%
\pgfpathlineto{\pgfqpoint{4.849110in}{0.873552in}}%
\pgfpathlineto{\pgfqpoint{4.849110in}{1.006159in}}%
\pgfpathlineto{\pgfqpoint{4.756284in}{1.006159in}}%
\pgfpathlineto{\pgfqpoint{4.756284in}{0.873552in}}%
\pgfusepath{fill}%
\end{pgfscope}%
\begin{pgfscope}%
\pgfpathrectangle{\pgfqpoint{4.756284in}{0.382904in}}{\pgfqpoint{0.092825in}{2.042155in}}%
\pgfusepath{clip}%
\pgfsetbuttcap%
\pgfsetroundjoin%
\definecolor{currentfill}{rgb}{0.195860,0.395433,0.555276}%
\pgfsetfillcolor{currentfill}%
\pgfsetlinewidth{0.000000pt}%
\definecolor{currentstroke}{rgb}{0.000000,0.000000,0.000000}%
\pgfsetstrokecolor{currentstroke}%
\pgfsetdash{}{0pt}%
\pgfpathmoveto{\pgfqpoint{4.756284in}{1.006159in}}%
\pgfpathlineto{\pgfqpoint{4.849110in}{1.006159in}}%
\pgfpathlineto{\pgfqpoint{4.849110in}{1.138766in}}%
\pgfpathlineto{\pgfqpoint{4.756284in}{1.138766in}}%
\pgfpathlineto{\pgfqpoint{4.756284in}{1.006159in}}%
\pgfusepath{fill}%
\end{pgfscope}%
\begin{pgfscope}%
\pgfpathrectangle{\pgfqpoint{4.756284in}{0.382904in}}{\pgfqpoint{0.092825in}{2.042155in}}%
\pgfusepath{clip}%
\pgfsetbuttcap%
\pgfsetroundjoin%
\definecolor{currentfill}{rgb}{0.166617,0.463708,0.558119}%
\pgfsetfillcolor{currentfill}%
\pgfsetlinewidth{0.000000pt}%
\definecolor{currentstroke}{rgb}{0.000000,0.000000,0.000000}%
\pgfsetstrokecolor{currentstroke}%
\pgfsetdash{}{0pt}%
\pgfpathmoveto{\pgfqpoint{4.756284in}{1.138766in}}%
\pgfpathlineto{\pgfqpoint{4.849110in}{1.138766in}}%
\pgfpathlineto{\pgfqpoint{4.849110in}{1.271374in}}%
\pgfpathlineto{\pgfqpoint{4.756284in}{1.271374in}}%
\pgfpathlineto{\pgfqpoint{4.756284in}{1.138766in}}%
\pgfusepath{fill}%
\end{pgfscope}%
\begin{pgfscope}%
\pgfpathrectangle{\pgfqpoint{4.756284in}{0.382904in}}{\pgfqpoint{0.092825in}{2.042155in}}%
\pgfusepath{clip}%
\pgfsetbuttcap%
\pgfsetroundjoin%
\definecolor{currentfill}{rgb}{0.140536,0.530132,0.555659}%
\pgfsetfillcolor{currentfill}%
\pgfsetlinewidth{0.000000pt}%
\definecolor{currentstroke}{rgb}{0.000000,0.000000,0.000000}%
\pgfsetstrokecolor{currentstroke}%
\pgfsetdash{}{0pt}%
\pgfpathmoveto{\pgfqpoint{4.756284in}{1.271374in}}%
\pgfpathlineto{\pgfqpoint{4.849110in}{1.271374in}}%
\pgfpathlineto{\pgfqpoint{4.849110in}{1.403981in}}%
\pgfpathlineto{\pgfqpoint{4.756284in}{1.403981in}}%
\pgfpathlineto{\pgfqpoint{4.756284in}{1.271374in}}%
\pgfusepath{fill}%
\end{pgfscope}%
\begin{pgfscope}%
\pgfpathrectangle{\pgfqpoint{4.756284in}{0.382904in}}{\pgfqpoint{0.092825in}{2.042155in}}%
\pgfusepath{clip}%
\pgfsetbuttcap%
\pgfsetroundjoin%
\definecolor{currentfill}{rgb}{0.120092,0.600104,0.542530}%
\pgfsetfillcolor{currentfill}%
\pgfsetlinewidth{0.000000pt}%
\definecolor{currentstroke}{rgb}{0.000000,0.000000,0.000000}%
\pgfsetstrokecolor{currentstroke}%
\pgfsetdash{}{0pt}%
\pgfpathmoveto{\pgfqpoint{4.756284in}{1.403981in}}%
\pgfpathlineto{\pgfqpoint{4.849110in}{1.403981in}}%
\pgfpathlineto{\pgfqpoint{4.849110in}{1.536589in}}%
\pgfpathlineto{\pgfqpoint{4.756284in}{1.536589in}}%
\pgfpathlineto{\pgfqpoint{4.756284in}{1.403981in}}%
\pgfusepath{fill}%
\end{pgfscope}%
\begin{pgfscope}%
\pgfpathrectangle{\pgfqpoint{4.756284in}{0.382904in}}{\pgfqpoint{0.092825in}{2.042155in}}%
\pgfusepath{clip}%
\pgfsetbuttcap%
\pgfsetroundjoin%
\definecolor{currentfill}{rgb}{0.140210,0.665859,0.513427}%
\pgfsetfillcolor{currentfill}%
\pgfsetlinewidth{0.000000pt}%
\definecolor{currentstroke}{rgb}{0.000000,0.000000,0.000000}%
\pgfsetstrokecolor{currentstroke}%
\pgfsetdash{}{0pt}%
\pgfpathmoveto{\pgfqpoint{4.756284in}{1.536589in}}%
\pgfpathlineto{\pgfqpoint{4.849110in}{1.536589in}}%
\pgfpathlineto{\pgfqpoint{4.849110in}{1.669196in}}%
\pgfpathlineto{\pgfqpoint{4.756284in}{1.669196in}}%
\pgfpathlineto{\pgfqpoint{4.756284in}{1.536589in}}%
\pgfusepath{fill}%
\end{pgfscope}%
\begin{pgfscope}%
\pgfpathrectangle{\pgfqpoint{4.756284in}{0.382904in}}{\pgfqpoint{0.092825in}{2.042155in}}%
\pgfusepath{clip}%
\pgfsetbuttcap%
\pgfsetroundjoin%
\definecolor{currentfill}{rgb}{0.226397,0.728888,0.462789}%
\pgfsetfillcolor{currentfill}%
\pgfsetlinewidth{0.000000pt}%
\definecolor{currentstroke}{rgb}{0.000000,0.000000,0.000000}%
\pgfsetstrokecolor{currentstroke}%
\pgfsetdash{}{0pt}%
\pgfpathmoveto{\pgfqpoint{4.756284in}{1.669196in}}%
\pgfpathlineto{\pgfqpoint{4.849110in}{1.669196in}}%
\pgfpathlineto{\pgfqpoint{4.849110in}{1.801804in}}%
\pgfpathlineto{\pgfqpoint{4.756284in}{1.801804in}}%
\pgfpathlineto{\pgfqpoint{4.756284in}{1.669196in}}%
\pgfusepath{fill}%
\end{pgfscope}%
\begin{pgfscope}%
\pgfpathrectangle{\pgfqpoint{4.756284in}{0.382904in}}{\pgfqpoint{0.092825in}{2.042155in}}%
\pgfusepath{clip}%
\pgfsetbuttcap%
\pgfsetroundjoin%
\definecolor{currentfill}{rgb}{0.369214,0.788888,0.382914}%
\pgfsetfillcolor{currentfill}%
\pgfsetlinewidth{0.000000pt}%
\definecolor{currentstroke}{rgb}{0.000000,0.000000,0.000000}%
\pgfsetstrokecolor{currentstroke}%
\pgfsetdash{}{0pt}%
\pgfpathmoveto{\pgfqpoint{4.756284in}{1.801804in}}%
\pgfpathlineto{\pgfqpoint{4.849110in}{1.801804in}}%
\pgfpathlineto{\pgfqpoint{4.849110in}{1.934411in}}%
\pgfpathlineto{\pgfqpoint{4.756284in}{1.934411in}}%
\pgfpathlineto{\pgfqpoint{4.756284in}{1.801804in}}%
\pgfusepath{fill}%
\end{pgfscope}%
\begin{pgfscope}%
\pgfpathrectangle{\pgfqpoint{4.756284in}{0.382904in}}{\pgfqpoint{0.092825in}{2.042155in}}%
\pgfusepath{clip}%
\pgfsetbuttcap%
\pgfsetroundjoin%
\definecolor{currentfill}{rgb}{0.535621,0.835785,0.281908}%
\pgfsetfillcolor{currentfill}%
\pgfsetlinewidth{0.000000pt}%
\definecolor{currentstroke}{rgb}{0.000000,0.000000,0.000000}%
\pgfsetstrokecolor{currentstroke}%
\pgfsetdash{}{0pt}%
\pgfpathmoveto{\pgfqpoint{4.756284in}{1.934411in}}%
\pgfpathlineto{\pgfqpoint{4.849110in}{1.934411in}}%
\pgfpathlineto{\pgfqpoint{4.849110in}{2.067019in}}%
\pgfpathlineto{\pgfqpoint{4.756284in}{2.067019in}}%
\pgfpathlineto{\pgfqpoint{4.756284in}{1.934411in}}%
\pgfusepath{fill}%
\end{pgfscope}%
\begin{pgfscope}%
\pgfpathrectangle{\pgfqpoint{4.756284in}{0.382904in}}{\pgfqpoint{0.092825in}{2.042155in}}%
\pgfusepath{clip}%
\pgfsetbuttcap%
\pgfsetroundjoin%
\definecolor{currentfill}{rgb}{0.720391,0.870350,0.162603}%
\pgfsetfillcolor{currentfill}%
\pgfsetlinewidth{0.000000pt}%
\definecolor{currentstroke}{rgb}{0.000000,0.000000,0.000000}%
\pgfsetstrokecolor{currentstroke}%
\pgfsetdash{}{0pt}%
\pgfpathmoveto{\pgfqpoint{4.756284in}{2.067019in}}%
\pgfpathlineto{\pgfqpoint{4.849110in}{2.067019in}}%
\pgfpathlineto{\pgfqpoint{4.849110in}{2.199626in}}%
\pgfpathlineto{\pgfqpoint{4.756284in}{2.199626in}}%
\pgfpathlineto{\pgfqpoint{4.756284in}{2.067019in}}%
\pgfusepath{fill}%
\end{pgfscope}%
\begin{pgfscope}%
\pgfpathrectangle{\pgfqpoint{4.756284in}{0.382904in}}{\pgfqpoint{0.092825in}{2.042155in}}%
\pgfusepath{clip}%
\pgfsetbuttcap%
\pgfsetroundjoin%
\definecolor{currentfill}{rgb}{0.906311,0.894855,0.098125}%
\pgfsetfillcolor{currentfill}%
\pgfsetlinewidth{0.000000pt}%
\definecolor{currentstroke}{rgb}{0.000000,0.000000,0.000000}%
\pgfsetstrokecolor{currentstroke}%
\pgfsetdash{}{0pt}%
\pgfpathmoveto{\pgfqpoint{4.756284in}{2.199626in}}%
\pgfpathlineto{\pgfqpoint{4.849110in}{2.199626in}}%
\pgfpathlineto{\pgfqpoint{4.849110in}{2.332233in}}%
\pgfpathlineto{\pgfqpoint{4.756284in}{2.332233in}}%
\pgfpathlineto{\pgfqpoint{4.756284in}{2.199626in}}%
\pgfusepath{fill}%
\end{pgfscope}%
\begin{pgfscope}%
\pgfpathrectangle{\pgfqpoint{4.756284in}{0.382904in}}{\pgfqpoint{0.092825in}{2.042155in}}%
\pgfusepath{clip}%
\pgfsetbuttcap%
\pgfsetroundjoin%
\definecolor{currentfill}{rgb}{0.993248,0.906157,0.143936}%
\pgfsetfillcolor{currentfill}%
\pgfsetlinewidth{0.000000pt}%
\definecolor{currentstroke}{rgb}{0.000000,0.000000,0.000000}%
\pgfsetstrokecolor{currentstroke}%
\pgfsetdash{}{0pt}%
\pgfpathmoveto{\pgfqpoint{4.756284in}{2.332233in}}%
\pgfpathlineto{\pgfqpoint{4.849110in}{2.332233in}}%
\pgfpathlineto{\pgfqpoint{4.802697in}{2.425059in}}%
\pgfpathlineto{\pgfqpoint{4.802697in}{2.425059in}}%
\pgfpathlineto{\pgfqpoint{4.756284in}{2.332233in}}%
\pgfusepath{fill}%
\end{pgfscope}%
\begin{pgfscope}%
\pgfsetbuttcap%
\pgfsetroundjoin%
\definecolor{currentfill}{rgb}{0.000000,0.000000,0.000000}%
\pgfsetfillcolor{currentfill}%
\pgfsetlinewidth{0.803000pt}%
\definecolor{currentstroke}{rgb}{0.000000,0.000000,0.000000}%
\pgfsetstrokecolor{currentstroke}%
\pgfsetdash{}{0pt}%
\pgfsys@defobject{currentmarker}{\pgfqpoint{0.000000in}{0.000000in}}{\pgfqpoint{0.048611in}{0.000000in}}{%
\pgfpathmoveto{\pgfqpoint{0.000000in}{0.000000in}}%
\pgfpathlineto{\pgfqpoint{0.048611in}{0.000000in}}%
\pgfusepath{stroke,fill}%
}%
\begin{pgfscope}%
\pgfsys@transformshift{4.849110in}{0.475729in}%
\pgfsys@useobject{currentmarker}{}%
\end{pgfscope}%
\end{pgfscope}%
\begin{pgfscope}%
\definecolor{textcolor}{rgb}{0.000000,0.000000,0.000000}%
\pgfsetstrokecolor{textcolor}%
\pgfsetfillcolor{textcolor}%
\pgftext[x=4.946332in, y=0.427535in, left, base]{\color{textcolor}\rmfamily\fontsize{10.000000}{12.000000}\selectfont \(\displaystyle {3}\)}%
\end{pgfscope}%
\begin{pgfscope}%
\pgfsetbuttcap%
\pgfsetroundjoin%
\definecolor{currentfill}{rgb}{0.000000,0.000000,0.000000}%
\pgfsetfillcolor{currentfill}%
\pgfsetlinewidth{0.803000pt}%
\definecolor{currentstroke}{rgb}{0.000000,0.000000,0.000000}%
\pgfsetstrokecolor{currentstroke}%
\pgfsetdash{}{0pt}%
\pgfsys@defobject{currentmarker}{\pgfqpoint{0.000000in}{0.000000in}}{\pgfqpoint{0.048611in}{0.000000in}}{%
\pgfpathmoveto{\pgfqpoint{0.000000in}{0.000000in}}%
\pgfpathlineto{\pgfqpoint{0.048611in}{0.000000in}}%
\pgfusepath{stroke,fill}%
}%
\begin{pgfscope}%
\pgfsys@transformshift{4.849110in}{0.847030in}%
\pgfsys@useobject{currentmarker}{}%
\end{pgfscope}%
\end{pgfscope}%
\begin{pgfscope}%
\definecolor{textcolor}{rgb}{0.000000,0.000000,0.000000}%
\pgfsetstrokecolor{textcolor}%
\pgfsetfillcolor{textcolor}%
\pgftext[x=4.946332in, y=0.798836in, left, base]{\color{textcolor}\rmfamily\fontsize{10.000000}{12.000000}\selectfont \(\displaystyle {9}\)}%
\end{pgfscope}%
\begin{pgfscope}%
\pgfsetbuttcap%
\pgfsetroundjoin%
\definecolor{currentfill}{rgb}{0.000000,0.000000,0.000000}%
\pgfsetfillcolor{currentfill}%
\pgfsetlinewidth{0.803000pt}%
\definecolor{currentstroke}{rgb}{0.000000,0.000000,0.000000}%
\pgfsetstrokecolor{currentstroke}%
\pgfsetdash{}{0pt}%
\pgfsys@defobject{currentmarker}{\pgfqpoint{0.000000in}{0.000000in}}{\pgfqpoint{0.048611in}{0.000000in}}{%
\pgfpathmoveto{\pgfqpoint{0.000000in}{0.000000in}}%
\pgfpathlineto{\pgfqpoint{0.048611in}{0.000000in}}%
\pgfusepath{stroke,fill}%
}%
\begin{pgfscope}%
\pgfsys@transformshift{4.849110in}{1.218331in}%
\pgfsys@useobject{currentmarker}{}%
\end{pgfscope}%
\end{pgfscope}%
\begin{pgfscope}%
\definecolor{textcolor}{rgb}{0.000000,0.000000,0.000000}%
\pgfsetstrokecolor{textcolor}%
\pgfsetfillcolor{textcolor}%
\pgftext[x=4.946332in, y=1.170136in, left, base]{\color{textcolor}\rmfamily\fontsize{10.000000}{12.000000}\selectfont \(\displaystyle {15}\)}%
\end{pgfscope}%
\begin{pgfscope}%
\pgfsetbuttcap%
\pgfsetroundjoin%
\definecolor{currentfill}{rgb}{0.000000,0.000000,0.000000}%
\pgfsetfillcolor{currentfill}%
\pgfsetlinewidth{0.803000pt}%
\definecolor{currentstroke}{rgb}{0.000000,0.000000,0.000000}%
\pgfsetstrokecolor{currentstroke}%
\pgfsetdash{}{0pt}%
\pgfsys@defobject{currentmarker}{\pgfqpoint{0.000000in}{0.000000in}}{\pgfqpoint{0.048611in}{0.000000in}}{%
\pgfpathmoveto{\pgfqpoint{0.000000in}{0.000000in}}%
\pgfpathlineto{\pgfqpoint{0.048611in}{0.000000in}}%
\pgfusepath{stroke,fill}%
}%
\begin{pgfscope}%
\pgfsys@transformshift{4.849110in}{1.589632in}%
\pgfsys@useobject{currentmarker}{}%
\end{pgfscope}%
\end{pgfscope}%
\begin{pgfscope}%
\definecolor{textcolor}{rgb}{0.000000,0.000000,0.000000}%
\pgfsetstrokecolor{textcolor}%
\pgfsetfillcolor{textcolor}%
\pgftext[x=4.946332in, y=1.541437in, left, base]{\color{textcolor}\rmfamily\fontsize{10.000000}{12.000000}\selectfont \(\displaystyle {21}\)}%
\end{pgfscope}%
\begin{pgfscope}%
\pgfsetbuttcap%
\pgfsetroundjoin%
\definecolor{currentfill}{rgb}{0.000000,0.000000,0.000000}%
\pgfsetfillcolor{currentfill}%
\pgfsetlinewidth{0.803000pt}%
\definecolor{currentstroke}{rgb}{0.000000,0.000000,0.000000}%
\pgfsetstrokecolor{currentstroke}%
\pgfsetdash{}{0pt}%
\pgfsys@defobject{currentmarker}{\pgfqpoint{0.000000in}{0.000000in}}{\pgfqpoint{0.048611in}{0.000000in}}{%
\pgfpathmoveto{\pgfqpoint{0.000000in}{0.000000in}}%
\pgfpathlineto{\pgfqpoint{0.048611in}{0.000000in}}%
\pgfusepath{stroke,fill}%
}%
\begin{pgfscope}%
\pgfsys@transformshift{4.849110in}{1.960933in}%
\pgfsys@useobject{currentmarker}{}%
\end{pgfscope}%
\end{pgfscope}%
\begin{pgfscope}%
\definecolor{textcolor}{rgb}{0.000000,0.000000,0.000000}%
\pgfsetstrokecolor{textcolor}%
\pgfsetfillcolor{textcolor}%
\pgftext[x=4.946332in, y=1.912738in, left, base]{\color{textcolor}\rmfamily\fontsize{10.000000}{12.000000}\selectfont \(\displaystyle {27}\)}%
\end{pgfscope}%
\begin{pgfscope}%
\pgfsetbuttcap%
\pgfsetroundjoin%
\definecolor{currentfill}{rgb}{0.000000,0.000000,0.000000}%
\pgfsetfillcolor{currentfill}%
\pgfsetlinewidth{0.803000pt}%
\definecolor{currentstroke}{rgb}{0.000000,0.000000,0.000000}%
\pgfsetstrokecolor{currentstroke}%
\pgfsetdash{}{0pt}%
\pgfsys@defobject{currentmarker}{\pgfqpoint{0.000000in}{0.000000in}}{\pgfqpoint{0.048611in}{0.000000in}}{%
\pgfpathmoveto{\pgfqpoint{0.000000in}{0.000000in}}%
\pgfpathlineto{\pgfqpoint{0.048611in}{0.000000in}}%
\pgfusepath{stroke,fill}%
}%
\begin{pgfscope}%
\pgfsys@transformshift{4.849110in}{2.332233in}%
\pgfsys@useobject{currentmarker}{}%
\end{pgfscope}%
\end{pgfscope}%
\begin{pgfscope}%
\definecolor{textcolor}{rgb}{0.000000,0.000000,0.000000}%
\pgfsetstrokecolor{textcolor}%
\pgfsetfillcolor{textcolor}%
\pgftext[x=4.946332in, y=2.284039in, left, base]{\color{textcolor}\rmfamily\fontsize{10.000000}{12.000000}\selectfont \(\displaystyle {33}\)}%
\end{pgfscope}%
\begin{pgfscope}%
\definecolor{textcolor}{rgb}{0.000000,0.000000,0.000000}%
\pgfsetstrokecolor{textcolor}%
\pgfsetfillcolor{textcolor}%
\pgftext[x=5.140777in,y=1.403981in,,top,rotate=90.000000]{\color{textcolor}\rmfamily\fontsize{10.000000}{12.000000}\selectfont \(\displaystyle f_t(x)\)}%
\end{pgfscope}%
\begin{pgfscope}%
\pgfsetrectcap%
\pgfsetmiterjoin%
\pgfsetlinewidth{0.803000pt}%
\definecolor{currentstroke}{rgb}{0.000000,0.000000,0.000000}%
\pgfsetstrokecolor{currentstroke}%
\pgfsetdash{}{0pt}%
\pgfpathmoveto{\pgfqpoint{4.802697in}{0.382904in}}%
\pgfpathlineto{\pgfqpoint{4.756284in}{0.475729in}}%
\pgfpathlineto{\pgfqpoint{4.756284in}{2.332233in}}%
\pgfpathlineto{\pgfqpoint{4.802697in}{2.425059in}}%
\pgfpathlineto{\pgfqpoint{4.802697in}{2.425059in}}%
\pgfpathlineto{\pgfqpoint{4.849110in}{2.332233in}}%
\pgfpathlineto{\pgfqpoint{4.849110in}{0.475729in}}%
\pgfpathlineto{\pgfqpoint{4.802697in}{0.382904in}}%
\pgfpathclose%
\pgfusepath{stroke}%
\end{pgfscope}%
\end{pgfpicture}%
\makeatother%
\endgroup%

    \caption[Trajectory of unconstrained \gls{ui} forgetting for the one-dimensional moving parabola.]{Trajectory of unconstrained \gls{ui} forgetting ($\hat{\sigma}_w^2=0.01$) for the one-dimensional moving parabola. The white circles denote the initial training data.}
    \label{fig:Parabola1D_UI_unconstrained}
\end{figure}
\begin{figure}[h]
    \centering
    \input{thesis/figures/pgf_figures/Parabola1D_UI_constrained.pgf}
    \caption[Trajectory of constrained \gls{ui} forgetting for the one-dimensional moving parabola.]{Trajectory of constrained \gls{ui} forgetting ($\hat{\sigma}_w^2=0.009$) for the one-dimensional moving parabola. The white circles denote the initial training data.}
    \label{fig:Parabola1D_UI_constrained}
\end{figure}